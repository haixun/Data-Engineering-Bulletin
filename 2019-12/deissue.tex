\documentclass[11pt]{article}
\usepackage{debulletin,times,epsfig,subfigure,wrapfig,algorithmic,color,boxedminipage,graphicx,url}

% this is the template for an issue of the Data Engineering Bulletin

% all packages used by any paper must be listed here
\usepackage{subfig}
\usepackage{listings}
\usepackage{caption}
\usepackage{microtype}
\usepackage{listings}
\usepackage{booktabs}
\usepackage{listings}
\usepackage{xargs}
%\PassOptionsToPackage{hyphens}{url}
%\usepackage{hyperref}
\usepackage{multirow}
\usepackage{tabularx}
\usepackage{makecell}
\usepackage{arydshln}
\usepackage{xspace}
\usepackage{tcolorbox}
\usepackage{xpatch}

\newcommand{\Hypercallback}{Hyperupcall\xspace{}}
\newcommand{\hypercallback}{hyperupcall\xspace{}}
\newcommand{\hide}[1]{}

\begin{document}


% please enter real date, vol no, issue no
\bulletindate{December 2019}
\bulletinvolume{42}
\bulletinnumber{4}
\bulletinyear{2019}

% these are files that I have- but your part of the issue can be done without
% them
\IEEElogo{cs.pdf}
\insidefrontcover{incvA19.pdf}
%\insidebackcover[ICDE Conference]{./calls/icde-new-a.ps}

\begin{bulletin}

% the above samples assume the issue is generated from a directory structure of the following sort
% major directory name is month and year of issue
% there are sub-directorys for
% letters: directory name is "letters"
% technical articles: a directory per paper, named for an "author"
% news articles: directory name is "news"
% calls: directory name is "calls

%
%  Editor letters section.  Use the lettersection environment.
%  Each letter is contained in a letter environment, where the two required
%  options to \begin{letter} are the author and the address of the author.
%

\begin{lettersection}

% there will be other letters- and a blank page will appear in your document
% but the special issue part will be fine

\begin{letter}{Letter from the Editor-in-Chief}
{Haixun Wang}{WeWork Corporation}
\documentclass[11pt]{article} 

\usepackage{deauthor,times,graphicx}
%\usepackage{url}
\usepackage{hyperref}

\begin{document}
How to efficiently and effectively manage large-scale data is a
critical challenge in data management, scientific computing, machine
learning, and many other fields. In this issue, we look into this
problem from two angles.

Gerhard Weikum's opinion piece titled ``Entities with Quantities''
highlights development along the direction of querying the Web as a
database. We have come a long way in keyword based Web search: Today,
all major search engines support entity based question/answering to
certain extent (e.g., returning ``Eiffel Tower'' for query ``the
highest building in Paris''). Weikum is taking one important step
towards the goal of querying the Web as a database. In the article, he
discusses what it takes to find all entities that satisfy a
quantity-based search condition, for example, ``buildings taller than
500m'' or ``runners completing a marathon under 2:10h.''  It is clear
that this requires much advanced data preprocessing (e.g., information
extraction, entity linking, etc.), but more importantly, it requires
that at least part of the data on the entire Web needs to be organized
as a database.

Philippe Bonnet put together the current issue consisting of 5 papers
from leading researchers in the high performance computing and data
management communities on the topic of data management at
Exascale. Advances in exascale computing on petascale supercomputers
are pushing the frontier of scientific computing that requires complex
simulation, benefiting applications ranging from astrophysical
discovery to drug design. But with increasing amounts of data, the gap
between computation and I/O has grown significantly wider, which makes
data management a big challenge. This timely issue answers many
questions in this domain.

\end{document}


\end{letter}
%
\newpage
%
%% your introductory letter goes here
%
\begin{letter}{Letter from the Special Issue Editor}
%{Sihem Amer-Yahia}{CNRS, Univ. Grenoble Alpes}
{Sihem Amer-Yahia, Lei Chen, Atsuyuki Morishima, Senjuti asu Roy}
{}
\documentclass[11pt]{article} 

\usepackage{deauthor,times,graphicx}
%\usepackage{url}
\usepackage{hyperref}


\begin{document}
The fast development of AI technology has changed our dail life significantly,  from personal assistant such as Siri to self-driving cars, online recommendations
sites such as Netflix and Amazon and many other real-world AI applications have emerged and brought great benefits to humanity. All these AI-powered applications have shown great intelligence if their models are well trained.  For example, Alpha go, a well-trained go player defeated top human players in the world.  However, the current AI technology relies strongly on the quality of the training datasets and can be restricted by the cognitive nature of a task. Tasks such as recognizing objects from blurred images and understanding sentiment
from complicated and poorly-structured sentences still need human assistance.  Therefore, in this issue, we study an interesting topic, human-powered AI.  Compared to current AI technology which is more advanced in solving closed domain problems, human intelligence is more advanced in addressing open domain problems, such as arts and design. How to seamlessly incorporate human intelligence into the whole process of AI is the theme of human-powered AI.

In this issue, we present works on human-powered AI from different aspects, 

The first paper discusses a challenge and essential problem in Human-powered AI, how to divide computation between human and AI effectively to achieve a specific target. Based on the discussion, the authors have introduced a set of dimensions and terms to classify existing solutions on this topic. 

The second paper addresses the challenges on how to untilize human intelligence (crowdsourcing) on federate learning,  which is one of the popular solutions to overcome the problem of data isolation and data privacy in a distributed learning environment.


The third  paper addresses the challenges on providing an end to end solution in human-in-the-loop of machine learning, from data extract, data integration, data cleaning, data labelling to machine learning and inference.


The fourth  paper  presents human-powered AI from the angle of AI technology, specifically, how to use AI technology to model, discover and explore human behaviour for human intelligence data management and mining. 


The last paper presents a real application of human-powered AI  techniques for online misinformation detection, challenges on various aspects in implementing such a system are outlined.


We  would like to thank all the authors for their insightful contributions. 
\end{document}




\end{letter}

\end{lettersection}

% put the name of your special issue below

\begin{opinionsection}
\begin{opinion}{Resurrecting Middle-Tier Distributed Transactions}
{Philip A. Bernstein}{Microsoft Research, Redmond, WA 98052}
\documentclass[11pt]{article} 
\usepackage{deauthor,times,graphicx,hyperref} 

\begin{document}
\title{Resurrecting Middle-Tier Distributed Transactions}
\author{Philip A. Bernstein\\Microsoft Research\\philbe@microsoft.com}


%\maketitle

\section{Introduction}
Over the years, platforms and application requirements change. As they do, technologies come, go, and return again as the preferred solution to certain system problems. In each of its incarnations, the technology's details change but the principles remain the same. One such technology is distributed transactions on middle-tier servers. Here, we argue that after a 15-year decline, they need to return to the mainstream. 

In the 1980's, Transaction Processing (TP) monitors were a popular category of middleware product that enabled customers to build scalable distributed systems to run transactions. Example products were CICS (IBM), Tuxedo (AT\&T for Unix), ACMS (DEC for VAX/VMS), and Pathway (Tandem for Guardian) \cite{4}. Their main features were multithreaded processes (not supported natively by most operating systems), inter-process communication (usually a crude form of remote procedure call), and a forms manager (for end users to submit transaction requests). The TP monitor ran on middle-tier servers that received transaction requests from front-end processors that communicated with end-user devices, such as terminals and PC's, and with back end database servers. The top-level application code executed on the middle-tier and invoked stored procedures on the database server.  
 
In those days, database management systems (DBMS's) supported ACID transactions, but hardly any of them supported distributed transactions. The TP monitor vendors saw this as a business opportunity and worked on adding a transaction manager feature that implemented the two-phase commit protocol (2PC). Such a feature required DBMS's to expose Start, Prepare, Commit, and Abort as operations that could be invoked by the TP monitor. Unfortunately, most of them didn't support Prepare, and even if they did, they didn't expose it to applications. They were willing to do so, but they didn't want to implement a different protocol for each TP monitor product. Thus, the XA standard was born, which defined TP monitor and DBMS interfaces (including Prepare) and protocols that allowed a TP monitor to run a distributed transaction across DBMS servers \cite{18}. 
 
This middle-tier architecture for distributed transactions was popular for about 20 years, into the late 1990s. Then TP monitors were replaced by Application Servers, which integrated a TP monitor with web servers, so it could receive transaction requests over HTTP, rather than receiving them from devices connected by a local area or terminal network. Examples include Microsoft Transaction Server, later renamed COM+, and Java Enterprise Edition (JEE), implemented by IBM's WebSphere Application Server, Oracle's WebLogic Application Server, and Red Hat's JBoss Application Server \cite{12}. The back end architecture was the same as before. Each transaction started executing on a middle-tier server and invoked stored procedures to read and write the database. 
 
Although this execution model is still widely used, starting in the early 2000's it fell out of favor for new application development, especially for applications targeted for cloud computing. More database vendors offered built-in support for distributed transactions, so there was less need to control the distributed transaction from the middle tier. A larger part of database applications executed on data that was cached in the middle tier. And the NoSQL movement argued that distributed transactions were too slow, that they limited scalability, and that customers rarely needed them anyway \cite{11}. Eventual consistency became all the rage \cite{19}. 
 
The critics of distributed transactions had some good points. But in the end, developers found that mainstream programmers really did need ACID transactions to make their applications reliable in the face of concurrent access to shared data and despite server failures. Thus, some NoSQL (key-value) stores added transaction support (e.g., CosmosDB \cite{2}, DynamoDB \cite{8}). Google, which had initially avoided support for multi-row distributed transactions in Bigtable \cite{5}, later introduced them in Spanner \cite{6}. There are now many cloud storage services and database products that support distributed transactions. 
 
Like product developers, database researchers have also focused on distributed transactions for back-end database systems. Almost universally, they assume that transactions execute as stored procedures and that middle-tier applications invoke those stored procedures but do not execute the transaction logic themselves. 
 
\section{Stateful Middle-Tier Applications}
This focus on stored procedures is well justified by the needs of traditional TP applications. However, stored procedures are not a good way of encapsulating application logic for a growing fraction of stateful applications that run on the middle tier. These include multi-player games, datacenter telemetry, Internet of Things, and social and mobile applications. Objects are a natural model for the entities in these applications, such as games, players, datacenter hardware, sensors, cameras, customers, and smart phones. Such applications have a large number of long-lived stateful objects that are spread over many servers and communicate via message passing. Like most new applications, these applications are usually developed to run on cloud computing platforms.  
 
These applications typically execute on middle-tier servers, rather than as stored procedures in database servers. They do this for many reasons. They need large main memory for the state of long-lived objects. They often have heavy computation needs, such as rendering images or computing over large graphs. They use a lot of computation for message passing between objects so they can scale out. And they need computation to be elastic, independent of storage requirements. These needs are satisfied by compute servers that are cheaper than database servers because they have less storage. Hence, these apps run on compute servers in the middle tier.  

\subsection{Requirements for Mid-Tier Cloud Transactions}
Some middle-tier applications need transactions because they have functions that read and write the state of two or more stateful objects. For example, a game may allow users to buy and sell virtual game objects, such as weapons, shields, and vehicles. A telemetry application may need to process an event exactly once by removing it from a queue and updating telemetry objects based on that event. A social application may need to add a user to a group and modify the user's state to indicate membership in that group. Each of these cases needs an ACID transaction over two or more objects, which may be distributed on different servers. Since these applications are usually developed to run on cloud computing platforms, distributed transaction support must be built into the cloud platform, a capability that is rarely supported today for cloud computing. 
 
Distributed transactions for middle-tier applications on a cloud computing platform have four requirements that differ from those supported by the late-1990's products that run transactions on the middle-tier. First, like all previous transaction mechanisms, they need to offer excellent performance. But unlike previous mechanisms, it's essential that they be able to scale out to a large number of servers, leading to the first requirement: The system must have high throughput and low transaction latency, at least when transactions have low contention, and in addition must scale out to many servers.  

To scale computation independently of storage, these applications typically save their state in cloud storage. The developers' choice of cloud storage service depends on their application's requirements (e.g., records, documents, blobs, SQL), their platform provider's offerings (e.g., AWS, Azure, Google), their employer's storage standards, and their developers' expertise. Thus, we have this second requirement: The transaction mechanism must support applications that use any cloud storage service.  

The transaction mechanism needs persistent storage to track transaction state: started, prepared, committed, or aborted. Like the apps themselves, it needs to use cloud storage for this purpose, which is the third requirement: The transaction mechanism must be able to use any of the cloud storage services used by applications. 

The traditional data structure for storing transaction state is a log. The transaction manager relies on the order of records in the log to understand the order in which transactions executed. Although cloud vendors implement logs to support their database services, they do not expose database-style logging as a service for customers, leading to a fourth requirement: The transaction mechanism cannot rely on a shared log, unless it implements the log itself, in which case the log must run on a wide variety of storage services. 
 
Due to the latency of cloud storage, requirements (2)-(4) create challenges in satisfying requirement (1). 

The above requirements are a first cut, based on today's applications and platforms. It is also worth targeting variations. For example, requirement (1) could include cost/performance, which might require a tradeoff against scalability. And (4) might go away entirely if cloud platforms offer high-performance logging as a service. 

\section{An Implementation in the Orleans Framework}
The rest of this paper sketches a distributed transaction mechanism that satisfies the above requirements \cite{9}. Our group built it for Microsoft's actor-oriented programming framework, called Orleans, which is open source and runs on both Windows and Linux \cite{16}. The distributed transaction project is part of a longer-term effort to enrich Orleans with other database features to evolve it into an actor-oriented database system that supports geo-distribution, stream processing, indexing, and other database features \cite{3}. 
\subsection{Two-Phase Commit and Locking}
For ACID semantics, Orleans transactions use two-phase commit (2PC) and two-phase locking (2PL). Our first challenge was to obtain high throughput and scalability despite the requirement to use cloud storage. In our runs, a write to cloud storage within a datacenter takes ~20 ms and has high variance. With 2PC, a transaction does two synchronous writes to storage. Therefore, if 2PL is used, a transaction holds locks for ~40ms, which limits throughput to 25 transactions/second (TPS). Low-latency SSD-based cloud storage is faster, but still incurs over 10 ms latency, plus higher cost. 
To avoid this problem, we extended early lock release to 2PC \cite{1,7,10,13,14,15}. After a transaction T1 terminates, it releases locks before writing to storage in phase one of 2PC. This allows a later transaction T2 to read/update T1's updated objects. Thus, while T1 is writing to storage, a sequence of later transactions can update an object, terminate, and then unlock the object. To avoid inconsistency, the system delays committing transactions that directly or indirectly read or overwrite T1's writeset until after T1 commits. And if T1 aborts, then those later dependent transactions abort too. Using this mechanism, we have seen transaction throughput up to 20x that of strict 2PL/2PC. 
\subsection{Logging}
Our initial implementation used a centralized transaction manager (TM) per server cluster \cite{9}. It ran on an independent server and was multithreaded. Since message-passing is a potential bottleneck, it batched its messages to transaction servers. It worked well with throughput up to 100K TPS. However, it had three disadvantages: it was an obvious bottleneck for higher transaction rates; a minimum configuration required two servers (i.e., primary and backup TM) in addition to servers that execute the application; and it added configuration complexity since TM servers did not run Orleans and thus had to be deployed separately from application servers.  

These disadvantages led us to redesign the system to avoid a centralized TM. Instead, we embed a TM in each application object. Each TM's log is piggybacked on its object's storage. This TM-per-object design avoids the above disadvantages and improves transaction latency by avoiding roundtrips to a centralized TM. However, it doesn't work for objects that have no updatable storage. For example, an object that performs a money transfer calls two stateful objects, the source and target of the transfer, but it has no state itself. We allow such an object to participate in a transaction by delegating its TM function to a stateful participant in the transaction, that is, one that has updatable storage.  

Orleans transactions write object state to a log to enable undo when a transaction aborts. This is impractical for large objects and is a poor fit for concurrency control that exploits operation commutativity. We therefore developed a prototype that logs operations.  
\section{Summary}
Many new cloud applications run their logic on the middle tier, not as stored procedures. They need distributed transactions. Thus, cloud computing platforms can and should offer scalable distributed transactions.  
\section{Acknowledgments}
I'm grateful to the team that built the initial and final implementations of Orleans transactions: Jason Bragg, Sebastian Burckhardt, Tamer Eldeeb, Reuben Bond, Sergey Bykov, Christopher Meiklejohn, Alejandro Tomsic, and Xiao Zeng. I also thank Bailu Ding and Dave Lomet for suggesting many improvements to this paper. 
 
 
\vspace{-.1cm}
\bibliographystyle{ACM-Reference-Format}
\begin{thebibliography}{10}
\begin{small}
\itemsep=-.5pt
\bibitem{1}
Athanassoulis, Manos ; Johnson, Ryan ; Ailamaki, Anastasia ; Stoica, Radu, Improving OLTP Concurrency through Early Lock Release, EPFL-REPORT-152158, https://infoscience.epfl.ch/record/152158?ln=en, 2009. 

\bibitem{2}
Azure CosmosDB, https://azure.microsoft.com/en-us/services/cosmos-db/ 

\bibitem{3}
Bernstein, P.A., M., T. Kiefer, D. Maier: Indexing in an Actor-Oriented Database. CIDR 2017  

\bibitem{4}
Bernstein, P. A., E. Newcomer: Chapter 10: Transactional Middleware Products and Standards, in Principles of Transaction Processing, Morgan Kaufmann, 2nd ed., 2009. 

\bibitem{5}
Chang, F., J. Dean, S. Ghemawat, W.C. Hsieh, D.A. Wallach, M. Burrows, T. Chandra, A. Fikes, R.E. Gruber: Bigtable: A Distributed Storage System for Structured Data. ACM Trans. Comput. Syst. 26(2): 4:1-4:26 (2008) 


\bibitem{6}
%Corbett, J.C., J. Dean, M. Epstein, A. Fikes, C. Frost, J.J. Furman, S. Ghemawat, A. Gubarev, C. Heiser, P. Hochschild, W.C. Hsieh, S. Kanthak, E. Kogan, H. Li, A. Lloyd, S. Melnik, D. Mwaura, D. Nagle, S. Quinlan, R. Rao, L. Rolig, Y. Saito, M. Szymaniak, C. Taylor, R. Wang, D. Woodford: Spanner: Google's Globally Distributed Database. ACM Trans. Comput. Syst. 31(3): 8:1-8:22 (2013) 
Corbett, J.C. et al: Spanner: Google's Globally Distributed Database. ACM Trans. Comput. Syst. 31(3): 8:1-8:22 (2013) 
\bibitem{7}
DeWitt, D.J., R.H. Katz, F. Olken, L.D. Shapiro, M. Stonebraker, D.A. Wood: Implementation Techniques for Main Memory Database Systems. SIGMOD 1984: 1-8 

\bibitem{8}
DynamoDB, https://aws.amazon.com/dynamodb/ 

\bibitem{9}
Eldeeb, T. and P. Bernstein: Transactions for Distributed Actors in the Cloud. Microsoft Research Tech Report MSR-TR-2016-1001. 

\bibitem{10}
Graefe, G., M. Lillibridge, H. A. Kuno, J. Tucek, A.C. Veitch: Controlled lock violation. SIGMOD 2013: 85-96 

\bibitem{11}
Helland, P., Life beyond Distributed Transactions: an Apostate's Opinion. CIDR 2007: 132-141 

\bibitem{12}
Java EE documentation, http://www.oracle.com/technetwork/?java/javaee/documentation/index.html  

\bibitem{13}
Larson, P-A, et al.: High-Perf. Concurrency Control Mechanisms for Main-Memory Databases. PVLDB 5(4): 298-309 (2011) 
\bibitem{14}

Levandoski, L.J., D.B. Lomet, S. Sengupta, R. Stutsman, R. Wang: High Performance Transactions in Deuteronomy. CIDR 2015 

\bibitem{15}
David B. Lomet: Using Timestamping to Optimize Two Phase Commit. PDIS 1993: 48-55 

\bibitem{16}
Orleans, http://dotnet.github.io/orleans 

\bibitem{18}
The Open Group, Distributed Transaction Processing: The XA Specification, http://pubs.opengroup.org/onlinepubs/009680699/toc.pdf. 

\bibitem{19}
Vogels W., Eventually Consistent. ACM Queue 6(6): 14-19 (2008) \end{small}
\end{thebibliography}

\end{document}

\end{opinion}
\end{opinionsection}

\begin{articlesection}{Imagine All the People and AI in the Future of Work}
%
%  Contributed articles section.  Use the articlesection environment.
%  Each article is contained in an article environment, where the two required
%  options to \begin{article} are the title and author of the article
%
%\begin{article}
%{Title of article}
%{list of authors}
%\input{author-name/article.tex}
%\end{article}
\begin{article}
{Capturing Human Factors to Optimize Crowdsourced Label
Acquisition through Active Learning}
{Senjuti Basu Roy}
\graphicspath{{submissions/senjuti/}}
% link to instruction: https://tc.computer.org/tcde/tcde-bulletin-author-instructions/
% \documentclass[11pt,dvipdfm]{article}
\documentclass[11pt]{article}
\usepackage{tabularx}
\usepackage{ragged2e}  % for '\RaggedRight' macro (allows hyphenation)
\usepackage{booktabs}  % for \toprule, \midrule, and \bottomrule macros
\usepackage{textcomp}
\usepackage{amsfonts,amsmath}
\usepackage{deauthor,times}
\usepackage{graphicx} % 
\usepackage{hyperref}
\usepackage{comment}
\graphicspath{{asudeh/}}
\usepackage{soul}
\usepackage{subcaption}
\usepackage{ulem}
\usepackage{wrapfig}
\usepackage{color}
\usepackage{xspace}
\newtheorem{problem}{Problem}

%\DeclareMathOperator*{\argmax}{arg\,max}

%remove the following commands/package b4 submission
\newcommand{\hide}[1]{}
\newcommand{\eat}[1]{}
\newcommand{\resolved}[1]{\hide{#1}}
\newcommand{\abol}[1]{\textcolor{red}{Abol: #1}}
\newcommand{\mahdi}[1]{\textcolor{red}{Mahdi: #1}}
\newcommand{\nima}[1]{\textcolor{red}{Nima: #1}}

\newcommand{\dee}{\mathcal{D}}
\newcommand{\Gee}{\mathcal{G}}
\newcommand{\gee}{\mathbf{g}}
\newcommand{\ee}{\mathbf{e}}
\newcommand{\es}{\mathcal{S}}
\newcommand{\el}{\mathcal{L}}
\newcommand{\xx}{\mathcal{x}}
\newcommand{\dist}{\xi}
\newcommand{\alg}{\mathsf{A}}
\newcommand{\qu}{\mathbf{q}}
\newcommand{\ex}{\mathbf{x}}
\newcommand{\ti}{\mathbf{t}}
\newcommand{\sdt}{\mathsf{SDT}}
\newcommand{\wdt}{\mathsf{WDT}}
\newcommand{\Qu}{\mathbf{Q}}
\newcommand{\pe}{\mathbb{P}}
\newcommand{\megam}{\mathcal{M}}
\newcommand{\eps}{\varepsilon}
\newcommand{\enet}{{$\varepsilon$-{\bf net}}\xspace}
\newcommand{\net}{{\tt net}\xspace}
\newcommand{\vcd}{VC-dimension\xspace}
\newcommand{\at}[1]{{\tt \small #1}\xspace}
\newcommand{\pr}{Pr}

\newcommand{\sharpP}{\mbox{\#P}}
\newcommand{\NP}{\mathsf{NP}}
\newcommand{\LP}{\mathsf{LP}}
\newcommand{\IP}{\mathsf{IP}}
\newcommand{\ru}{{\sc {RU}}\xspace}
\newcommand{\sru}{{\sc {strongRU}}\xspace}
\newcommand{\wru}{{\sc {weakRU}}\xspace}

\newcommand{\fmsystem}{{\sc Chameleon}\xspace}
\newcommand{\fm}{$\mathcal{F}$\xspace}

\newtheorem{experiment}{Experiment}

\begin{document}

\title{Coverage-based Data-centric Approaches for \\Responsible and Trustworthy AI\thanks{This research was supported by the National Science Foundation under grant No. 2107290.}}

\author{
\begin{tabular}[t]{c@{\extracolsep{2.4em}}c@{\extracolsep{2.4em}}c@{\extracolsep{2.3em}}c} 
Nima Shahbazi & Mahdi Erfanian & Abolfazl Asudeh \\ 
University of Illinois Chicago & University of Illinois Chicago & University of Illinois Chicago\\
 nshahb3@uic.edu & merfan2@uic.edu & asudeh@uic.edu
\end{tabular}
}

\maketitle


\begin{abstract}
The grand goal of data-driven decision systems is to help make decisions easier, more accurate, at a higher scale, and also just. However, data-driven algorithms are only as good as the data they work with. Yet, data sets, especially those with social data, often do not represent minorities. The paucity of training data is a perpetual problem for AI, and the outcome of ML models for cases not represented in their training data is often not reliable. 
Hence, without properly addressing the lack of representation issues in data, we cannot expect AI-based societal solutions to have responsible and trustworthy outcomes. 

This paper focuses on data coverage as a data-centric approach for identifying and resolving misrepresentation of minorities in data.
To achieve this goal, we propose novel algorithms that (a) {\it identify} and {\it resolve} insufficient data coverage across data with different modalities and (b) use lack of representation information to generate data-centric {\it reliability warnings}.
 \end{abstract}
 
 %%%%%%%%%%%%%%%%%%%%%%%%%%%%%%%% INTRO  %%%%%%%%%%%%%%%%%%%%%%%%%%%%%%%%
\section{Introduction}\label{sec:intro} % Abstract+Intro: up to 2.5 pages 
Data-driven decision-making has shaped every corner of human life, spanning from autonomous vehicles to healthcare and even predictive policing and criminal justice. A pivotal concern, especially in applications that affect individuals, revolves around the reliability of the decisions rendered by the system.
It is easy to see that the accuracy of a data-driven decision depends, first and foremost, on the data used to make it. Essentially, the system learns the phenomena that data represent. While we may desire that the data should represent the underlying data distribution from which the production data is drawn, this alone may be insufficient, as it merely enables the model to perform well for the average case.
As a result, a model with a high accuracy could fail for specific regions in the data with insufficient representation. These regions may matter because they frequently represent some minority population in society. They could also represent cases that may not happen very often but have a relevant impact on the correctness of a critical decision.
In short, if the data fails to sufficiently represent a specific population, the outcome of the decision system for that population may not be trustworthy.

The phenomenon known as \textit{Representation Bias} can arise from how the data was originally collected, or it could be the result of biases introduced post-collection—whether historically, cognitively, or statistically.

Representation bias is essentially inevitable without a systematic approach to data collection. 
For example, in the context of survey data collection, vital steps involve identifying all populations within the underlying distribution based on desired demographic information and ensuring comprehensive coverage with sufficient samples from each group. 
Even then, only an (uncontrolled) subset of the invitees will opt-in to respond to the survey.
Another challenge lies in the fact that data scientists often lack control over the data collection process, leading to the reliance on ``found data'' in the majority of data-driven systems. Therefore, with no guarantee on the aforementioned steps in the data collection process, the found data is most likely a biased sample.
Acknowledging the potential harms of representation bias, the notion of \textit{Data Coverage}~\cite{asudeh2019assessing,shahbazi2023representation} has been proposed to ensure the adequate representation of minority groups in data sets employed for decision-making and developing sophisticated data science tools. 

Addressing representation issues in data poses various challenges depending on the modality of the data. In this paper, we focus on identifying and resolving lack of coverage issues in data with different modalities.
We start by proposing a variety of techniques (spanning from geometric and combinatorial optimization to crowd-souring) aimed at efficiently detecting insufficient coverage on structured data sets with non-ordinal categorical and continuous attributes, as well as image data sets. Next, we propose a range of approaches grounded in data integration and generative data augmentation to address the lack of coverage by enriching the data sets with more data. However, with limited control over the data collection processes, it could be difficult and expensive to resolve all misrepresentations. 
Since adding more data is not always possible, we proceed to introduce data-centric preventive solutions that warn the user about the reliability of their predictions regarding representation bias issues. These warnings assist users in determining whether they trust the outcomes of the models or exercise caution. 

 %%%%%%%%%%%%%%%%%%%%%%%%%%%%%%%% IDENTIFICATION  %%%%%%%%%%%%%%%%%%%%%%%%%%%%%%%%
\section{Detecting Insufficient Representation of Minorities}\label{sec:identification} %up to 3.5 pages
Representation bias happens when the development (training data) population under-represents 
and subsequently fails to generalize well 
for some parts of the target population, due to historical bias, sampling bias, etc.
The notion of {\it data coverage} has been studied across different settings in \cite{shahbazi2023representation} as a metric to measure representation bias. At a high level, coverage is referred to as having enough similar entries for each object in a data set. 
For a better understanding, let us go over the definition of the generalized notion of coverage:

\begin{definition}[Data Coverage]\label{def:coverage}
Consider a data set $\dee$ with $n$ tuples, each consisting of $d$ attributes of interest $\mathbf{x}=\{x_1, x_2, \cdots,x_d\}$, such as {\tt gender}, {\tt race}, {\tt salary}, {\tt age}, etc, that are used for coverage identification.
The data set also contains target attributes $\mathbf{y} = \{ y_1,\cdots,y_{d'}\}$ that may or may not be considered for the coverage problem.
A query point $q$ is not covered by the data set $\dee$, if there are not ``enough'' data points in $\dee$ that are representative of $q$.
To generalize the notion of coverage, let us define $\gee(q)$ as the universe of tuples that would represent $q$ and let $\gee_\dee(q) = \gee(q)\cap \dee$. In other words, $\gee_\dee(q)$ are the set of tuples in $\dee$ that represent $q$.
Using this notation, we define the coverage of $q$ as the size of $\gee_\dee(q)$. That is,
$cov(q,\dee) = | \gee_\dee(q)|$.
Given a value $\tau$, $q$ is covered if $cov(q,\dee)>\tau$.
Similarly, a group $\gee$ is not covered if $\gee\cap \dee<\tau$.
The {\it uncovered region} in a data set is the collection of groups that are not covered by it.
\end{definition}

\subsection{Structured Data}
In this section, we focus on identifying representation bias in structured data.
Depending on the type of the attributes of interest, we categorize the techniques into two classes based on whether they target the problem for non-ordinal {\it categorical} (e.g. {\tt race}, {\tt gender}) or ordinal {\it continuous} (e.g. {\tt age}) attributes. The attributes of interest considered for representation bias often include sensitive attributes such as {\tt race} and {\tt gender} but are not necessarily limited to them.

\subsubsection{Categorical Attributes}

For cases where attributes of interest are non-ordinal categorical,
the cartesian product of values on a subset of attributes $\mathbf{x}'\subseteq \mathbf{x}$, form a set of (sub-)groups.
For example, $\{$ {\tt white male}, {\tt white female}, {\tt black male} $,\cdots\}$ are the subgroups defined on the attributes {\tt (race,gender)}.
We refer to the number of attributes used to specify a subgroup as the {\it level} of that subgroup.
For example, the level of the subgroup {\tt white male} is 2, while the level of the subgroup {\tt male} is 1.
We use $\ell(\gee)$, to refer to the level of a subgroup $\gee$.
Similarly, we say a subgroup $\gee'$ is a subset of $\gee$, if the groups specifying $\gee'$ are a superset of the ones for $\gee$. For example {\tt (married white male)} a subset of the more general group {\tt (white male)}. That is, the set of individuals in group {\tt (married white male)} are a subset of {\tt (white male)}.
Moreover, we say a subgroup $\gee$ is a {\it parent} of the subgroup $\gee'$, if $\gee'\subset \gee$ and $\ell(\gee)=\ell(\gee')+1$. For example, the subgroup {\tt (white male)} is a parent of the subgroup {\tt (married white male)}.
We use \textit{patterns} to refer to uncovered subgroups.
A pattern $P$ is a string of $d$ values, where $P[i]$ is either a value from the domain of $x_i$, or it is ``unspecified'', specified with $X$. 
For example, consider a data set with three binary attributes of interest $\mathbf{x}=\{x_1, x_2, x_3\}$. The pattern $P=X01$ specifies all the tuples for which $x_2=0$ and $x_3=1$ ($x_1$ can have any value).
The set of patterns that identify most general uncovered subgroups are called {\it Maximal Uncovered Patterns} (MUPs).

No polynomial time algorithm can guarantee the enumeration of the entire MUPs, however, several algorithms inspired by set enumeration and the Apriori algorithm for association rule mining are proposed to efficiently address this problem~\cite{asudeh2019assessing}.
In this regard, we introduce \textit{Pattern Graph} data structure that exploits the relationship between patterns to do less work than computing all uncovered patterns by removing the non-maximal ones. 
The parent-child relationship between the patterns is represented in a graph that can be used to find better algorithms. 
\textit{Pattern-Breaker} starts from the top of the graph where the general patterns are and moves down by breaking each pattern into more specific ones. If a pattern is uncovered, then all of its descendants are also uncovered and they can not be an MUP, even if they have a parent that is covered. Therefore, this subgraph of the pattern graph can be pruned. 
The issue with \textit{Pattern-Breaker} is that it explores the covered regions of the pattern graph and for the cases where there are a few uncovered patterns, it has to explore a large portion of the exponential-size graph. 
To tackle this, \textit{Pattern-Combiner} algorithm is proposed that performs a bottom-up traversal of the pattern graph. It uses an observation that the coverage of a node at the level of the pattern graph can be computed as the sum of the coverage values of its children. 
The problem with \textit{Pattern-Combiner} is that it traverses over the uncovered nodes first and therefore, it will not perform well for the cases in which most of the nodes in the graph are uncovered. 
In fact, for the cases where most of the MUPs are placed in the middle of the graph, both \textit{Pattern-Breaker} and \textit{Pattern-Combiner} will not be as efficient as they should traverse half of the graph. Therefore, we propose \textit{Deep-Diver}, a search algorithm based on Depth-First-Search that quickly finds the MUPs, and uses them to limit the search space by pruning the nodes both dominating and dominated by the discovered MUPs.

\begin{figure*}[!tb]
    \begin{minipage}[t]{0.31\linewidth}
        \centering
        \includegraphics[width=\textwidth]{submissions/submission1/shahbazi/covcube1.jpg}
        \caption{\small Categorical attributes: the uncovered region of a toy example, as the collection of three MUPs.}
        \label{fig:covcube1}
    \end{minipage}
    \hfill
    \begin{minipage}[t]{0.31\linewidth}
        \centering
        \includegraphics[width=\textwidth]{submissions/submission1/shahbazi/cvrg_2_1.jpg}
        \caption{\small Continuous attributes, 2D: identifying the covered region in the gray Voronoi cell.}
        \label{fig:cvrg_2_1}
    \end{minipage}
    \hfill
    \begin{minipage}[t]{0.31\linewidth}
        \centering
        \includegraphics[width=\textwidth]{submissions/submission1/shahbazi/cvrg_2_2.jpg}
        \caption{ \small Continuous attributes, 2D: Uncovered region marked in red.}
        \label{fig:cvrg_2_2}
    \end{minipage}
\vspace{-5mm}
\end{figure*}

\subsubsection{Continuous Attributes}
Data in the real world often consists of a combination of continuous and discrete values. While simple solutions like binning {\tt age} into {\tt young} and {\tt old} can transform the continuous space into discrete. However, they may lead to coarse groupings that are sensitive to the thresholds chosen. It may be inappropriate to treat a 35-yo as {\tt young} but a 36-yo as {\tt old}. 
Therefore, we extend the notion of coverage to continuous space. Particularly, given data set $\dee$ with $n$ tuples over $d$ attributes, and vicinity radius $\rho$ and coverage threshold $k$, we want to identify the uncovered region -- the universe of uncovered query points.
A query point in continuous data space is covered if there are enough (at least $k$) data points in its $\rho$-vicinity neighborhood. $\rho$-vicinity neighborhood is the circle centered at the query point with radius $\rho$.

Depending on the number of attributes in a data set, we propose two algorithms for identifying uncovered regions in data~\cite{asudeh2021coverage}. 
The first algorithm known as \textit{Uncovered-2D} studies coverage over two-dimensional data sets where $\mathbf{x}=\{x_1,x_2\}$. To find the number of circles that a query point falls into and consequently discover the uncovered region, \textit{Uncovered-2D} makes a connection to $k$-th order Voronoi diagrams.
Consider a data set $\mathcal{D}$ and its corresponding $k$-th order Voronoi diagram. For every tuple $t\in \mathcal{D}$, let $\circ_t$ be the $d$-dimensional sphere ($d$-sphere) with radius $\rho$ centered at $t$.
Consider a $k$-voronoi cell $\mathcal{V}(S)$ in the $k$-th order Voronoi diagram $V_k(\mathcal{D})$.
Any point $q$ inside the intersections of the $d$-spheres of tuples in $S$, i.e. $q\in \underset{\forall t\in S}{\cap ~\circ_t}$, is covered, while all other points in the region are uncovered.
 The algorithm starts by constructing the $k$-th order Voronoi diagram of the data set and then for each Voronoi cell $\mathcal{V}(S)$ in the diagram, it computes the intersection of the circles of the tuples in $S$ and marks the portion of $\mathcal{V}(S)$ that falls outside it as uncovered.
After identifying the uncovered region, a 2D map of $\{x_1,x_2\}$ value combinations is used to report the region to the user.
The algorithm for the 2D case can be extended to the general case by relaxing the assumption on the number of attributes to discover the exact uncovered region, however, due to the curse of dimensionality, the search size space explodes as the number of dimensions increases and as a result, the algorithm will not be practical. Therefore, we propose a randomized approximation algorithm based on the geometric notion of \enet. 
Let $\mathcal{X}$ be a set and $\mathcal{R}$ be a set of subsets of $\mathcal{X}$. A set $\mathcal{N}\subset \mathcal{X}$ is an \enet for $\mathcal{X}$ if for any range $r\in\mathcal{R}$, if  $|r\cap \chi|>\eps|\chi|$, then $r$ contains at least one point of $N$.
The idea, at a high level, is to draw enough random samples from the space of potential query points to form an \enet. 
We then label the sampled query points as $\{-1,+1\}$ depending on whether those are covered or not, and learn the uncovered regions using the samples.

\subsection{Image Data}
Many known incidents of machine failures due to the lack of representation were on image data.
We consider an image data set with a fixed number of low-cardinality sensitive attributes such as {\tt\small race} and {\tt\small gender}. 
It is common that image data sets {\it lack explicit values} for sensitive attributes, which are crucial for coverage identification. An image data set is often a collection of images from different domains with little to no information about their domain and which groups they belong to. As a result, even studying coverage over low-cardinality and categorical attributes of interests is challenging in these cases.

\begin{wrapfigure}{R}{0.42\textwidth}
\centering
\vspace{-3mm}
\scriptsize
\begin{tabular}{|@{}c|@{}c@{}|@{}c@{}|@{}c@{}|} 
 \hline
{\bf data set} & {\bf classifier} & {\bf accuracy} & {\bf precision} \\ 
 &  &  & {\bf on female} \\ \hline
UTKFace:~& DeepFace (opencv) & 93.56 & {52.02}\\\cline{2-4}
({\tt females}=200,& DeepFace (retinaface) & 94.16 & {56.15}\\\cline{2-4}
{\tt males}=2800) & BaseCNN & 97.6 & 74.8\\
\hline
UTKFace:~& DeepFace (opencv) & 96.53 & {\bf 8.0}\\\cline{2-4}
({\tt females}=20,& DeepFace (retinaface) & 96.43 & {\bf 10.09}\\\cline{2-4}
{\tt males}=2980)& BaseCNN & 97.6 & {\bf 21.59}\\
\hline
\end{tabular}
\vspace{-3mm}
\caption{\small ML models' low performance for females in the presence of representation bias.~\cite{mousavi2024data}}\label{fig:mlfails}
\vspace{-3mm}
\end{wrapfigure}

In Figure~\ref{fig:mlfails}, we show that due to the issues such {\it machine bias} and {\it lack of distribution generalizability},
solely relying on state-of-the-art machine learning (ML) techniques fail to effectively identify lack of coverage in image data sets. Therefore, we propose an approach based on combining crowdsouring with ML~\cite{mousavi2024data}. 
Crowdsourcing is particularly promising for image data, for tasks such as image labeling, which, while challenging for the machine, are "easy" for human beings to conduct with minimal error. 

A key observation that enables a cost-effective crowdsourcing approach is that, while studying coverage, we would only like to find out if there are {\it enough tuples from each subgroup}.
Suppose a subgroup is covered if there are $\tau=100$ instances of it in the data set. Assume the (majority) group $\gee_1$ contains $n_1 \gg 100$ objects in the data set. 
To verify that $\gee_1$ is covered, it is enough for the crowd to discover 100 of those objects, not the entire $n_1$. 
Following this, $O(\tau)$ provides a lower bound on the number of crowd tasks required to verify a given group is covered. 
Still, this lower bound only holds for the groups that are covered, i.e., there is at least $\tau$ of those in the data set.
Surprisingly, verifying that a minority group is indeed uncovered is cumbersome, unlike the majority group.
This is because even though discovering $\tau$ objects from a group is enough for verifying that it is covered, one cannot {\it verify} a group is uncovered until there is a chance that the data set might still have enough objects from that group. Thus, assuming a non-zero probability for each unlabeled object to belong to each group, {one might need to ask the crowd to label the entire data set before they can confirm that a specific group is uncovered}.

Our idea for addressing this challenge is to
design {\it a divide and conquer algorithm} that, instead of {point queries}, uses {\it set queries} to iteratively eliminate subsets of data that {does not include any object from the given group}.
At a high level, our idea is to ask a set query from the crowd, inquiring whether the selected set contains at least one object from the given group $\gee$.
The user may provide two responses (yes/no). 
Interestingly, {in either case}, the user response provides valuable information that helps efficiently identify the coverage.
If the answer is ``No'', the set does not include any object from the given group $\gee$. As a result, the algorithm can safely prune the set, asking no further questions about it. In particular, for a group that is not covered, one can expect to see no answers on large set queries helping to prune a significant portion of the data set quickly.
On the other hand, if the answer is ``yes'', the set contains {at least} one object from the group $\gee$. As a result, the algorithm cannot prune the subset since it can have any number (larger than one) of the objects in $\gee$.
At first glance, the queries with yes answers do not provide helpful information as the algorithm cannot prune the subset (hence it needs to divide it into smaller subsets).
However, a key observation is that {the algorithm will only observe a limited number of yes answers} before it stops.
The reason is that the number of set queries with yes answers provides a {lower-bound} on the number of objects from $\gee$ in the data set. As a result, the algorithm can stop as soon as the lower bound reaches $\tau$, knowing that $\gee$ is covered.
The D\&C approach verifies the data coverage for a given group, while our goal is to identify the uncovered regions for a given set of sensitive attributes. The next question is how to utilize this algorithm for efficient coverage identification on different scenarios of sensitive attributes, forming intersectional or non-intersectional groups.
In particular, how can we find maximal uncovered patterns?
Our idea is to apply sampling and aggregate estimation techniques to find the groups that even if merged are likely to still be uncovered. This will help reduce the coverage identification cost by running the D\&C approach for the merged groups once.
 %%%%%%%%%%%%%%%%%%%%%%%%%%%%%%%% RESOLUTION  %%%%%%%%%%%%%%%%%%%%%%%%%%%%%%%%
\section{Resolving Insufficient Representation}\label{sec:resolution}

Data integration~\cite{nargesian2021tailoring,nargesian2022responsible} and data augmentation~\cite{sharma2020data,DBLP:journals/jair/ChawlaBHK02,iosifidis2018dealing,celis2020data} are considered as the primary solutions for reducing data coverage issues in a data set. 
Data integration is promising when external sources of data are available. On the other hand, recent advancements in generative AI and foundation models have enabled efficient and effective augmentation of data sets with synthetic data. 
Therefore, in the following, we review two approaches, one from each category, in the context of lack of coverage resolution.

\subsection{Data Integration}\label{sec:resolution:integration}

Data integration is to consolidate data from different sources into a single, unified view. 
Although it is an effective solution to acquire additional data from different distributions,
there are sampling policy and cost-efficiency concerns that need to be examined.  
Therefore, {\it Data Distribution Tailoring ({\sc DT})} introduces data integration techniques for resolving insufficient representation of subgroups in a data set in the most cost-effective manner~\cite{nargesian2021tailoring}.
A query to {\sc DT} 
consists of a target schema, and a set of group distribution requirements in the form of the minimum counts (e.g., ``{\tt\small 1,000 breast cancer monitoring data in Chicago with at least 30\% label=positive, and at least 20\% black patients}''). 
Collecting a fresh sample from a data view is costly (monetary, human resources, and/or computation cost)~\cite{asudeh2022towards}.
Therefore, {\sc DT} focuses on satisfying the count requirements with minimum cost. 
Given an input query and a lake of available data sources, the first step is to discover a collection of candidate data views that satisfy the target schema.
Each data view $v_i$ is a projection-join $v_i = \Pi\big(D_{i1}\bowtie\cdots\bowtie D_{ik_i} \big)$, where $D_{ij}$ is a data set in a given data lake.
Let us suppose the data views are already discovered.
At a high level, {\sc DT} follows an iterative approach that at each iteration a data view is selected to be queried.
Each query to a data view has a fixed cost and returns a sample that may or may not satisfy the query constraints.
The samples that are either not fresh, or do not satisfy the query are discarded.
Hence, the essential question towards a cost-effective data integration is {\it what data view to query next}.
Depending on the available information about the data sources, various techniques may be employed. 

For the cases when the group distributions are known, the process of collecting the target data set is a sequence of iterative steps, where at every step, the algorithm chooses a data view, queries it, and if the obtained tuple contributes to one of the groups for which the count requirement is not yet fulfilled, it is kept, otherwise discarded. To do so, a {Dynamic Programming (DP)} algorithm is proposed. An optimal source at each iteration minimizes the sum of its sampling cost plus the expected cost of collecting the remaining required groups, based on its sampling outcome.
The DP algorithm, however, has a pseudo-polynomial time complexity. Hence, it quickly becomes intractable for cases where the minimum count requirements for the groups are not small. 
For cases where the (sensitive) attribute of interest is binary, such as (biological) {\tt sex}={\tt \{male, female\}}, and the cost to query data is similar from all sources, it turns out that the optimal strategy is to query the data source with {maximum probability of obtaining a sample from the minority group}.
Expanding the binary-attributes algorithm for non-binary cases, the problem can be modeled as an extension of the ``{\it coupon collector's}'' problem~\cite{motwani1995randomized}, where the goal is to collect $m_i$ instances from each coupon (group) $\gee_i$.
At each iteration, the coupon collector's algorithm identifies a data view as most promising and queries it. In simple terms, a data view with a smaller query cost and a higher chance of obtaining minority groups is more promising.


For the cases where the group distributions are unknown, we model DT as a {\it multi-armed bandit} problem, where every data view is modeled as an arm. 
Every arm has an unknown distribution of different groups while pulling an arm (i.e., querying the corresponding data view) has a cost.
During various iterations, the algorithms pull the arms in an order that its expected total {\it reward} is maximized.
Arguing that the reward of obtaining a tuple from a group is proportional to how rare this group is across different data views, 
we design the reward function based on the expected cost one needs to pay in order to collect a tuple from a specific group.  
As the bandit strategy, we adopt {\it Upper Confidence Bound (UCB)} to balance exploration and exploitation. At every iteration, for every arm, UCB computes confidence intervals for the expected reward and selects the arm with the maximum upper bound of reward to be explored next.

\subsection{Data Augmentation using Foundation Models}

While data integration provides a promising approach for resolving coverage issues in a data set, its effectiveness is limited to the availability of external data sources that are rich enough to find sufficient fresh samples from minority groups. This, however, is not always possible, especially since the minority samples are rare and not easy to obtain.
Fortunately, recent advancements in Generative AI and Foundation Models have enabled synthesizing samples that are otherwise challenging to obtain from the real world.

Therefore, as an alternative approach to data integration, we turn our attention to the Foundation Models and Generative AI for resolving the lack of coverage. 
Particularly, models such as {\sc DALL.E}\footnote{\url{https://openai.com/dall-e-2}} have emerged as powerful tools for generating multi-modal data such as image, audio, and video.
 
We formalize the foundation model \fm as a black-box function with the following inputs, that once queried synthesize an output tuple.
\begin{itemize}
    \item {\bf Prompt}: A natural language description providing instructions on the details of the tuple to be generated. For instance, a prompt for image generation might be ``A realistic photo of a white cat running in a backyard.''
    \item {\bf Guide}: In cases where only a prompt is provided, the foundation model uses its imagination to generate the requested tuple. For the previous example, the prompt of a cat image, the breed, size, background, and other details are generated based on the model's imagination. Alternatively, a guide can be provided to influence the generation process. The guide is formalized as a pair $(t,m)$ where $t$ is a tuple and $m$ is a mask specifying which parts of the guide tuple should be changed. Using the cat example, $t$ can be a cat image and $m$ can specify the foreground to be regenerated.
\end{itemize}

There are multiple challenges towards effective data set augmentations using foundation models. 
First, we have to determine the minimal set of synthetic tuples that once added to the original data set, under-representation issues are resolved.
Second, the generated images should follow the underlying distribution represented in the input data set. Third, the generated tuples should have high quality and look realistic to a human evaluator. Last but not least, given the (often monetary) cost associated with the queries to the foundation model, we should ensure the cost-effectiveness of the data set repair process.

\begin{wrapfigure}{L}{0.45\textwidth}
\centering
\vspace{-3mm}
\scriptsize
    \includegraphics[width=.45\textwidth]{submissions/submission1/shahbazi/enhanced_pipeline.png}
\vspace{-3mm}
\caption{\small Architecture of \fmsystem for image data augmentation for coverage enhancement.}\label{fig:chameleon}
% \vspace{-3mm}
\end{wrapfigure}

\noindent Figure~\ref{fig:chameleon} shows the architecture of our system \fmsystem \cite{chameleon} for coverage enhancement using DALL-E image generator.
To address the first challenge, we define the combinations-selection problem, which minimizes the total number of synthetic tuples for resolving lack of coverage of minorities at the most general level. We show the problem is {\sc NP}-hard, and propose a greedy approximation algorithm for it.
To address the second and third challenges, \fmsystem follows a {\it rejection sampling} strategy.
It views each tuple in the data set $\dee$ as an iid sample from the underlying distribution $\xi$ it represents. It uses the vector representations (embeddings) space to describe the distribution. Then, given a newly generated tuple, it employs the one-class support vector machine (OCSVM) approach proposed by Scholkopf et al.~\cite{scholkopf1999support} to reject the tuple if it does not follow $\xi$.
Moreover, it models the quality evaluation as hypothesis testing and rejects the samples that have a higher chance of being labeled as ``unrealistic'' by a random human evaluator.
Finally, to minimize the number of queries to the foundation model, we provide a guide tuple (and a mask), in addition to the prompt, to the foundation model. We model the guide-selection problem as {\it contextual multi-armed bandit} and propose a solution based on the contextual UCB for it.

Before concluding this section, let us provide some experiment results to demonstrate the effectiveness of data augmentation with \fmsystem. We use FERET DB \cite{phillips1998feret} for this experiment, which comprises 1199 individual images and serves as a standardized facial image database for researchers to develop algorithms and report results. All images in FERET DB share the same dimensions, pose, and facial expression.
First, we identified the (level-1) uncovered ethnicity groups, using the threshold 80. We then used \fmsystem and resolved the lack of coverage issues.
To evaluate the effectiveness of the system, we trained a CNN model to predict the race of each image within this dataset. We then retrained the identical CNN on the repaired training data. Importantly, our test dataset for both experiments remains consistent and is derived from real images.
Table~\ref{tab:lackofcoverage} presents the improvements in precision, recall, and F1 score metrics for under-represented groups after repairing the dataset. The results indicate an enhancement in performance metrics for all under-represented groups following the repair process.

\begin{table}[t]
    \centering
    \caption{Illustrating the effect of lack of coverage repair using \fmsystem on \texttt{FERTDB}}
    \label{tab:lackofcoverage}
    \vspace{-3mm}
    \begin{tabular}{lcccccccc}
        \toprule
         & \multicolumn{4}{c}{\textbf{Classifier Performance on \texttt{FERTDB}}} & \multicolumn{4}{c}{\textbf{Classifier Performance on Repaired}} \\
        \cmidrule(lr){2-5} \cmidrule(lr){6-9}
        \textbf{Ethnicity Groups}& \#Images & Precision & Recall & F1-Score & \#Images & Precision & Recall & F1-Score \\
        \midrule
        Overall          & 756 & 0.81 & 0.75 & 0.78 & 987 & 0.70 & 0.75 & 0.72 \\ \hline
        Black            & 40  & 0.19 & 0.22 & 0.16 & 100 & 0.48 & 0.56 & 0.52 \\
        Hispanic         & 19  & 0.50 & 0.17 & 0.25 & 100 & 0.62 & 0.36 & 0.45 \\
        Middle Eastern   & 10  & 0.00 & 0.00 & 0.00 & 100 & 0.20 & 0.41 & 0.27 \\
        \bottomrule
    \end{tabular}
\end{table}

 %%%%%%%%%%%%%%%%%%%%%%%%%%%%%%%% RELIABILITY  %%%%%%%%%%%%%%%%%%%%%%%%%%%%%%%%
\section{Generating Reliability Warnings}\label{sec:reliability}
% up to 2.5 pages
Interpretability is a necessity for data scientists who develop predictive models for critical decision-making.
In such settings, it is important to provide additional means to support the following question:
{\it is an individual prediction of the model reliable for decision-making?} Our goal is to use the lack of representation to help decision-makers find insights about this critical question.
To further motivate this, let us use the following example:

\vspace{1mm}
\begin{example}\label{ex-0}
{\bf(Part1):} Consider a judge who needs to decide whether to accept or deny a bail request. Using data-driven predictive models is prevalent in such cases for predicting recidivism~\cite{dressel2018accuracy}.
Indeed, such models can be beneficial to help the judge make wise decisions.
Suppose the model predicts the queried individual as high risk (or low risk).
The judge is aware and concerned about the critics surrounding such models.
A major question the judge faces is whether or not they should rely on the prediction outcome to take action for this case.
Furthermore, if, for instance, they decide to ignore the outcome and hence they need to provide a statement supporting their action, what evidence can they provide? 
\end{example}

In line with the recent trend on data-centric AI~\cite{ng2021mlops}, we design {novel approaches}, {complimentary} to the existing work on trustworthy AI~\cite{wing2021trustworthy,kentour2021analysis,liu2021trustworthy,singh2021trustworthy}, to address the aforementioned trust question through the lens of {\it data}.
In particular, unlike existing works that generate trust information from a {\it given \underline{model}}, we associate {\it \underline{data sets} with proper measurements} that specify their {\it the scope of use for predicting future cases}.
We note that a predictive model provides only probabilistic guarantees on the \underline{average} loss over the distribution represented by the data set used for training it.
As a result, these predictions may not be distribution generalizable~\cite{kulynych2022you}.
Consequently, if the query point is {\it not represented} by the data, the guarantees may not hold, hence one cannot rely on the prediction outcome.
Besides, an essential requirement for a learning algorithm is that its training data $\dee$ should represent the underlying distribution $\dist$.
Even if so, the trained model $h$ only provides a probabilistic guarantee on the {expected} loss on random samples from $\dist$.  
A model that performs well on {\it majority} of samples drawn from $\dist$ will have a high performance on average. Still, as we observed in Figure~\ref{fig:mlfails},
its performance for {\it minorities} and points that are not represented is questionable. Let us consider the following toy example:

\begin{figure*}[!b] 
    \begin{minipage}[t]{0.32\linewidth}
        	\centering
        	\includegraphics[width=\textwidth]{submissions/submission1/shahbazi/example_1.png} 
        	\vspace{-9mm}\caption{\small Data set $\dee$ generated using a Gaussian distribution; $x_1$ and $x_2$ are positively correlated}
            \label{fig:ex1:1}
    \end{minipage}
    \hfill
    \begin{minipage}[t]{0.32\linewidth}
        \centering
        	\includegraphics[width =\textwidth]{submissions/submission1/shahbazi/example_2.png} 
        	\vspace{-9mm}\caption{\small The decision boundary of learned model $h$ and query points $\qu^1$ to $\qu^4$}
            \label{fig:ex1:2}
    \end{minipage}
    \hfill
    \begin{minipage}[t]{0.32\linewidth}
        	\centering
        	\includegraphics[width =\textwidth]{submissions/submission1/shahbazi/example_3.png}
        	\vspace{-9mm}\caption{\small Ground-truth boundary, overlaid on the model decision boundary and query points}
            \label{fig:ex1:3}
    \end{minipage}
    \vspace{-5mm}
\end{figure*} 

\vspace{1mm}
\begin{example}\label{ex-1}
Consider a binary classification task where the input space is $\ex=\langle x_1, x_2\rangle$ and the output space is the binary label $y$ with values $\{-1$ (red) $,+1$ (blue)$\}$.
Suppose the underlying data distribution $\dist$ follows a 2D Gaussian, where $x_1$ and $x_2$ 
are positively correlated as shown in Figure~\ref{fig:ex1:1}.
The figure shows the data set $\dee$ drawn independently from the distribution $\dist$, along with their labels as their colors.
Using $\dee$, the prediction model $h$ is constructed as shown in Figure~\ref{fig:ex1:2}. 
The decision boundary is specified in the picture; while any point above the line is predicted as +1, a query point below it is labeled as -1.
The classifier has been evaluated using a test set that is an iid sample set drawn from the underlying data set $\dist$. The accuracy on the test set is high (above 90\%), and hence, the model gets deployed.
We cherry-picked four query points, $\qu^1$ to $\qu^4$, that are also included in Figure~\ref{fig:ex1:2}. Using $h$ for prediction, $h(\qu^1)=-1$, $h(\qu^2)=+1$,  $h(\qu^3)=+1$, and $h(\qu^4)=-1$.
Figure~\ref{fig:ex1:3} adds the ground-truth boundary to the search space, revealing the true label of the query points: every point inside the red circle has the true label $-1$ while any point outside of it is $+1$.
Looking at the figure, $y^1=+1$ while the model predicted it as $h(\qu^1)=-1$.  \hfill$\square$
\end{example}
\vspace{2mm}

Let us take a closer look at the four query points in this example and their placement with regard to the tuples in $\dee$ used for training $h$. 
$\qu^2$ belongs to a {\it dense region} with many training tuples in $\dee$ surrounding it. Besides, all of the tuples in its vicinity have the same label $y=+1$. As a result, one can expect that the model's outcome $h(\qu^2)=+1$ should be a reliable prediction.
Similar to $\qu^2$, $\qu^4$ also belongs to a dense region in $\dee$; however, $\qu^4$ belongs to an {\it uncertain region}, where some of the tuples in its vicinity have a label $y=+1$, and some others have the label $y=-1$. Considering the uncertainty in the vicinity of $\qu^4$, one cannot confidently rely on the outcome of the model $h$. 
On the other hand, the neighbors of $\qu^1$ (resp. $\qu^3$) are not uncertain, all having the label $y=-1$ (resp. $y=+1$).
However, the query points $\qu^1$ and $\qu^3$ are not well represented by $\dee$. In other words, $\qu^1$ and $\qu^3$ are unlikely to be generated according to the underlying distribution $\dist$, represented by $\dee$. As a result, following the no-free-lunch theorem~\cite{kakade2003sample}, one cannot expect the outcome of model $h$ to be reliable for these points.
Looking at the ground-truth boundary in Figure~\ref{fig:ex1:3}, $h$ luckily predicted the outcome for $\qu^3$ correctly, but it was not fortunate to predict the $y^1$ correctly.
Nevertheless, 
since the model is not reliably trained for these points, 
its outcome for these query points is not trustworthy.

From Example~\ref{ex-1}, we observe that the outcome of a model $h$, trained using a data set $\dee$ is not reliable for a query point $\qu$, if:
\begin{itemize}
    \item {\bf Lack of representation:} $\qu$ is not well-represented by $\dee$.
    In such cases, the model has not seen ``enough'' samples similar to $\qu$ to reliably learn and predict the outcome of $\qu$.
    \item {\bf Lack of certainty:} $\qu$ belongs to an uncertain region, where different tuples of $\dee$ in the vicinity of $\qu$ have different target values. $\qu$ belongs to a high-fluctuating area, where tuples in the vicinity of $\qu$ have a wide range of values.
\end{itemize} \vspace{2mm}

\noindent
Based on these two observations, we propose Representation-and-Uncertainty ({\bf RU}) measures.
To identify if a query suffers from uncertainty or lack of representation, one could use a deterministic approach using a fixed threshold. Then if the number of similar samples to (resp. label fluctuation in vicinity of) $\qu$ is larger than the threshold it is considered as unrepresented (resp. uncertain).
This approach, however, would be misleading since two numbers close to the threshold could be treated very differently. Also, all points on each side of the threshold would be considered equally represented (resp., certain). Instead, we consider {\it a randomized approach}, widely popular in the literature, including~\cite{dwork2012fairness}.
That is, instead of using fixed thresholds, a Bernoulli variable (a biased coin) is used that 
assigns $\qu$ as unrepresented (resp., uncertain) based on the number of samples similar to it (resp., its neighborhood uncertainty).
Given a query point $\qu$, let $\pe_o$ be the probability indicating if $\qu$ is not represented and let $\pe_u$ be the probability indicating if $\qu$ belongs to an uncertain region. 
We represent the probability of the Bernoulli variables for lack of representation or uncertainty components as $\pe_o$ and $\pe_u$, respectively. Note that the two Bernoulli variables $\pe_o$ and $\pe_u$ are independent from each other. That simply follows the argument that after specifying the number of similar samples to $\qu$ whether or not it should be considered as unrepresented does not depend on the uncertainty in the neighborhood of $\qu$.

\begin{definition}[\sru]\label{def:sdt}
The \sru is a probabilistic measure that considers the outcome of a model for a query point $\qu$ untrustworthy if $\qu$ is not represented by $\dee$ {\it and} it belongs to an uncertain region.
Formally, the \sru measure is:
\begin{align} 
    \nonumber
    SRU(\qu) &= \pe\big((\qu \mbox{ is outlier}) \wedge (\qu \mbox{ belongs to uncertain region})\big) 
\end{align}
Since $\pe_o$ and $\pe_u$ are independent:

\vspace{-13mm}
\begin{align} \label{eq:strong}
    SRU(\qu) &= \pe_o(\qu) \times \pe_u(\qu)
\end{align}
\end{definition}

\sru raises the warning signal only when the query point fails on {\it both} conditions of being represented by $\dee$ and not belonging to an uncertain region. 
For instance, in Example~\ref{ex-1} none of the query points fail both on representation and on uncertainty; hence neither has a high \sru score.
On the other hand, 
a high \sru score for a query point $\qu$ {\it provides a strong warning signal} that one should perhaps reject the model outcome and not consider it for decision-making.

\sru is a strong signal that raises warnings only for the fearfully concerning cases that fail both on representation and uncertainty.
However, as observed in Example~\ref{ex-1} a query points failing {\it at least} one of these conditions may also not be reliable, at least for critical decision making.
We define the \wru measure to raise a warning for such cases.

\begin{definition}[\wru]\label{def:wdt}
The \wru measure is a probabilistic measure that considers the outcome of a model for a query point $\qu$ untrustworthy if $\qu$ is not represented by $\dee$ {\bf or} it belongs to an uncertain region.
Formally, the \wru is computed as:
\begin{align} \label{eq:weak}
    WRU(\qu) = \pe\big((\qu \mbox{ is outlier}) \vee (\qu \mbox{ belongs to uncertain region})\big) 
    = \pe_o(\qu) + \pe_u(\qu) - \pe_o(\qu) \times \pe_u(\qu)
\end{align}
\end{definition}

Proposing quantitative probabilistic outcomes, \ru measures are interpretable for the users, since beyond the scores, the uncertainty and lack of representation components provide an explanation to justify them. 
Please refer to \cite{techrep} for more details on how to efficiently and effectively compute the representation ($\pe_o$) and uncertainty ($\pe_u$) probabilities, using only $\dee$.
In Example~\ref{ex-0}, let us see how the \ru measures can be helpful.

\noindent{\bf Example 1. (part 2):}
{\it RU measures \underline{raise warning} when
the fitness of the data set used for drawing a prediction is questionable, helping the judge to be cautious when taking action.
Besides, these measures provide \underline{quantitative evidence} to support the judge's action when they decide to ignore a prediction outcome that is not trustworthy.
The judge, for example, can argue to ignore a model outcome for a specific case, based on the insight that 
the model has been built using a
data set that fails to represent the given case.}
\hfill$\square$

Finally, let us demonstrate the efficacy of \ru measures through a series of experiments. Since the \ru measures are {\it data-centric},
those are applicable for both classification and regression tasks, irrespective of the model used.
We use {\it Adult} dataset~\cite{adult} for classification and {\it House Sales in King County} dataset for the validation of regression tasks. From each dataset, we uniformly sample two sets from the underlying distribution. The first set serves as the training set to compute the \ru values, and the second one is used as the test set from which the queries are drawn. We validate our proposal by providing the correlation between the \ru values and the performance of an ML model's prediction on the same data. 

We start by computing the \ru values for all the query points in the test set. Next, we bucketize the query points based on their \ru values in equi-width buckets of width 0.1. We repeat this for both \sru and \wru measures. Next, we train a model on the training data set and predict the target variable for the points in each range of \ru measure. The validation results for the classification task on the {\it Adult} dataset are presented in Figures \ref{fig:exp-adult-sdt} and \ref{fig:exp-adult-wdt}. Each figure corresponds to the accuracy/error measures of the classifier over each bucket of \ru values for \sru and \wru. As the \ru values increase, the accuracy of the model drops while the FPR rises, and therefore, the model fails to capture the ground truth for the points that fall into untrustworthy regions in the data set. By repeating the aforementioned steps for the regression task on the {\it House Sales in King County} dataset, we observe similar results presented in Figures \ref{fig:exp-hs-sdt} and \ref{fig:exp-hs-wdt}. 
As the \ru value increases, the RSS of the regression model follows the same trend denoting that the model fails to perform for tuples with a high \ru value.

\begin{figure}[!tb]
    \begin{minipage}[t]{0.24\linewidth}
        \centering
        \includegraphics[width=\textwidth]{submissions/submission1/shahbazi/sdt_adult.pdf}
        \vspace{-6mm}\caption{\small{\it Adult}, efficacy of \sru  on classification}
        \label{fig:exp-adult-sdt}
    \end{minipage}\hfill
    \begin{minipage}[t]{0.24\linewidth}
        \centering
        \includegraphics[width=\textwidth]{submissions/submission1/shahbazi/wdt_adult.pdf}
        \vspace{-6mm}\caption{\small{\it Adult}, efficacy of \wru  on classification}
        \label{fig:exp-adult-wdt}
    \end{minipage}\hfill
    \begin{minipage}[t]{0.24\linewidth}
        \centering
        \includegraphics[width=\textwidth]{submissions/submission1/shahbazi/sdt_regression_house.pdf}
        \vspace{-6mm}\caption{\small{\it House Sales in King County}, efficacy of \sru on regression}
        \label{fig:exp-hs-sdt}
    \end{minipage}\hfill
    \begin{minipage}[t]{0.24\linewidth}
        \centering
        \includegraphics[width=\textwidth]{submissions/submission1/shahbazi/wdt_regression_house.pdf}
        \vspace{-6mm}\caption{\small{\it House Sales in King County}, efficacy \wru on regression}
        \label{fig:exp-hs-wdt}
    \end{minipage}
\vspace{-5mm}
\end{figure}
 %%%%%%%%%%%%%%%%%%%%%%%%%%%%%%%% RELATED WORK  %%%%%%%%%%%%%%%%%%%%%%%%%%%%%%%%
\section{Related Work}\label{related} 

Bias in data has been looked at for a long time in statistical community~\cite{neyman1936contributions} but social data presents different challenges~\cite{olteanu2019social,fairmlbook,barocas2016big,jk2019bias,drosou2017diversity}.
The diversity and representativeness of data have been widely studied~\cite{drosou2017diversity}, in fields such as social science~\cite{berrey2015enigma, dobbin2016diversity,simpson1949measurement}, political science~\cite{surowiecki2005wisdom}, and information retrieval~\cite{agrawal2009diversifying}. 
Tracing back machine bias to its source, there have been major efforts to identify different types~\cite{mehrabi2021survey, olteanu2019social,friedman1996bias} and sources~\cite{torralba2011unbiased,crawford2013hidden,diakopoulos2015algorithmic} of biases in data. Efforts to satisfy {\it responsible data} requirements~\cite{nargesian2022responsible} extend to various stages of the data analysis pipeline, including data annotation~\cite{li2020towards,lazier2023fairness}, data cleaning and repair~\cite{SalimiRHS19,tae2019data,salimi2020database}, data imputation~\cite{martinez2019fairness}, entity resolution~\cite{shahbazi2023through,fanourakis2023fairer}, data integration~\cite{nargesian2022responsible,nargesian2021tailoring}, etc. 

\paragraph{Data Coverage:}The notion of data coverage has received extensive attention from different angles. Detecting lack of coverage has been studied for datasets with discrete~\cite{asudeh2019assessing} and continuous~\cite{asudeh2021coverage} attributes populated in single or multiple \cite{lin2020identifying} relations.
To resolve insufficient coverage, \cite{accinelli2020coverage, accinelli2021impact,shetiya2022fairness}
consider resolving representation bias in preprocessing pipelines by rewriting queries into the closest operation so that certain subgroups are sufficiently represented in the downstream tasks. Alternatively, ~\cite{asudeh2019assessing,tae2021slice} propose a data collection strategy to acquire as little additional data as possible (to minimize the associated costs) to meet the representation constraints. ~\cite{sharma2020data,iosifidis2018dealing,celis2020data} opt for a data augmentation approach by adding partially altered duplicates of already existing tuples or generating new synthetic entries from existing data. Consequently, the new data set has an equal number of elements for different groups, resulting in potentially resolving the under-representation issues. Finally,  \cite{nargesian2021tailoring} utilizes data integration techniques to consolidate data from different sources into a single dataset to resolve representation bias.
Related works also include ~\cite{chung2019slice,sagadeeva2021sliceline,tae2021slice} that seek to understand if the overall performance of the model fails to reflect and performs poorly on certain slices in the data.
As alternative approaches to measure representation bias, the notion of representation rate~\cite{celis2020data} (a.k.a. equal base rate~\cite{kleinberg2016inherent}) is introduced which compared with coverage, it is more restrictive as it requires almost equal ratios from different groups.
Please refer to \cite{shahbazi2023representation} for a comprehensive survey about representation bias in data. 

\paragraph{ML Reliability:} Model-centric works for uncertainty quantification such as 
probabilistic classifiers~\cite{zadrozny2001obtaining,zadrozny2002transforming,platt1999probabilistic,niculescu2005predicting},
prediction intervals (PIs) \cite{chatfield93predictionintervals,pearce2018high,khosravi2010lower} and conformal predictions (CP)~\cite{angelopoulos2021gentle,shafer2008tutorial} that are used for measuring prediction uncertainty, are built
by maximizing the {\it expected performance} on {\it random} sample from the underlying distribution.
As a result, while providing accurate estimations for the dense regions of data (e.g. majority groups), their estimation accuracy is questionable for the poorly represented regions.
In particular, \cite{angelopoulos2021gentle} recognizes the lack of guarantees in the performance of CP for such regions.
Besides, the bulk of work on trustworthy AI provides information that {\it supports} the outcome of an ML model. For example, existing work on explainable AI, including~\cite{harradon2018causal,ribeiro2016should,gunning2019darpa}, aims to find simple explanations and rules that justify the outcome of a model.
Conversely, we aim to {\it raise warning signals} when the outcome of a model is {\it not} trustworthy. That is, to provide reasons that {\it cast doubt} on the reliability of the model outcome {for a given query point}.

 %%%%%%%%%%%%%%%%%%%%%%%%%%%%%%%% FUTURE  %%%%%%%%%%%%%%%%%%%%%%%%%%%%%%%%
% \vspace{-3mm}
\section{Final Remarks}\label{sec:conclusion}
As Data-centric AI and Responsible AI emerge as focal points in data science research, the development of Data-centric methodologies for ensuring Responsible and Trustworthy AI attracts increasing attention.
While there is some excellent work on responsible data management to achieve this goal, there remain many challenges yet to be addressed.

In this paper, we focused on a crucial aspect of responsible data -- detecting and addressing the under-representation of minorities within a data set.
We formally defined the notion of data coverage and discussed various techniques for (a) identifying lack of representation issues across different data modalities, (b) ensuring proper representation of minorities in data, and (c) limiting the scope-of-use of data sets based on their representation issues by generating proper ({\sc RU}) warning signals.
Even though the research on detecting lack of coverage issues is relatively mature, resolution techniques are still understudied.
Considering the recent advancements in Generative AI, utilizing Foundation Models and Large Language Models, and studying their limitations, for data augmentation to improve the representation of minorities at the data level seems interesting to further explore.

 %%%%%%%%%%%%%%%%%%%%%%%%%%%%%%%% BIB  %%%%%%%%%%%%%%%%%%%%%%%%%%%%%%%%
\bibliographystyle{unsrt}
\small
% \bibliography{ref}
\begin{thebibliography}{10}

\bibitem{asudeh2019assessing}
A.~Asudeh, Z.~Jin, and H.~Jagadish.
\newblock Assessing and remedying coverage for a given dataset.
\newblock In {\em ICDE}, pages 554--565. IEEE, 2019.

\bibitem{shahbazi2023representation}
N.~Shahbazi, Y.~Lin, A.~Asudeh, and H.~Jagadish.
\newblock Representation bias in data: A survey on identification and resolution techniques.
\newblock {\em ACM Computing Surveys}, 2023.

\bibitem{asudeh2021coverage}
A.~Asudeh, N.~Shahbazi, Z.~Jin, and H.~V. Jagadish.
\newblock Identifying insufficient data coverage for ordinal continuous-valued attributes.
\newblock In {\em SIGMOD}. ACM, 2021.

\bibitem{mousavi2024data}
M.~Mousavi, N.~Shahbazi, and A.~Asudeh.
\newblock Data coverage for detecting representation bias in image datasets: {A} crowdsourcing approach.
\newblock In {\em {EDBT}}, pages 47--60, 2024.

\bibitem{nargesian2021tailoring}
F.~Nargesian, A.~Asudeh, and H.~Jagadish.
\newblock Tailoring data source distributions for fairness-aware data integration.
\newblock {\em Proceedings of the VLDB Endowment}, 14(11):2519--2532, 2021.

\bibitem{nargesian2022responsible}
F.~Nargesian, A.~Asudeh, and H.~V. Jagadish.
\newblock Responsible data integration: Next-generation challenges.
\newblock {\em SIGMOD}, 2022.

\bibitem{sharma2020data}
S.~Sharma, Y.~Zhang, J.~M. R{\'\i}os~Aliaga, D.~Bouneffouf, V.~Muthusamy, and K.~R. Varshney.
\newblock Data augmentation for discrimination prevention and bias disambiguation.
\newblock In {\em AIES}, pages 358--364, 2020.

\bibitem{DBLP:journals/jair/ChawlaBHK02}
N.~V. Chawla, K.~W. Bowyer, L.~O. Hall, and W.~P. Kegelmeyer.
\newblock {SMOTE:} synthetic minority over-sampling technique.
\newblock {\em J. Artif. Intell. Res.}, 16:321--357, 2002.

\bibitem{iosifidis2018dealing}
V.~Iosifidis and E.~Ntoutsi.
\newblock Dealing with bias via data augmentation in supervised learning scenarios.
\newblock {\em Jo Bates Paul D. Clough Robert J{\"a}schke}, 24, 2018.

\bibitem{celis2020data}
L.~E. Celis, V.~Keswani, and N.~Vishnoi.
\newblock Data preprocessing to mitigate bias: A maximum entropy based approach.
\newblock In {\em ICML}, pages 1349--1359. PMLR, 2020.

\bibitem{asudeh2022towards}
A.~Asudeh and F.~Nargesian.
\newblock Towards distribution-aware query answering in data markets.
\newblock {\em Proceedings of the VLDB Endowment}, 15(11):3137--3144, 2022.

\bibitem{motwani1995randomized}
R.~Motwani and P.~Raghavan.
\newblock {\em Randomized algorithms}.
\newblock Cambridge university press, 1995.

\bibitem{chameleon}
M.~Erfanian, H.~V. Jagadish, and A.~Asudeh.
\newblock Chameleon: Foundation models for fairness-aware multi-modal data augmentation to enhance coverage of minorities.
\newblock {\em arXiv preprint arXiv:2402.01071}, 2024.

\bibitem{scholkopf1999support}
B.~Sch{\"o}lkopf, R.~C. Williamson, A.~Smola, J.~Shawe-Taylor, and J.~Platt.
\newblock Support vector method for novelty detection.
\newblock {\em NeurIPS}, 12, 1999.

\bibitem{phillips1998feret}
P.~J. Phillips, H.~Wechsler, J.~Huang, and P.~J. Rauss.
\newblock The feret database and evaluation procedure for face-recognition algorithms.
\newblock {\em Image and vision computing}, 16(5):295--306, 1998.

\bibitem{dressel2018accuracy}
J.~Dressel and H.~Farid.
\newblock The accuracy, fairness, and limits of predicting recidivism.
\newblock {\em Science advances}, 4(1):eaao5580, 2018.

\bibitem{ng2021mlops}
A.~Ng.
\newblock Mlops: From model-centric to data-centric {AI}.
\newblock 2021.

\bibitem{wing2021trustworthy}
J.~M. Wing.
\newblock Trustworthy {AI}.
\newblock {\em CACM}, 64(10):64--71, 2021.

\bibitem{kentour2021analysis}
M.~Kentour and J.~Lu.
\newblock Analysis of trustworthiness in machine learning and deep learning.
\newblock {\em InfoComp}, 2021.

\bibitem{liu2021trustworthy}
H.~Liu, Y.~Wang, W.~Fan, X.~Liu, Y.~Li, S.~Jain, A.~K. Jain, and J.~Tang.
\newblock Trustworthy {AI}: A computational perspective.
\newblock {\em arXiv preprint arXiv:2107.06641}, 2021.

\bibitem{singh2021trustworthy}
R.~Singh, M.~Vatsa, and N.~Ratha.
\newblock Trustworthy {AI}.
\newblock In {\em 8th ACM IKDD CODS and 26th COMAD}, pages 449--453. 2021.

\bibitem{kulynych2022you}
B.~Kulynych, Y.-Y. Yang, Y.~Yu, J.~B{\l}asiok, and P.~Nakkiran.
\newblock What you see is what you get: Distributional generalization for algorithm design in deep learning.
\newblock {\em arXiv preprint arXiv:2204.03230}, 2022.

\bibitem{kakade2003sample}
S.~M. Kakade.
\newblock {\em On the sample complexity of reinforcement learning}.
\newblock University of London, University College London (United Kingdom), 2003.

\bibitem{dwork2012fairness}
C.~Dwork, M.~Hardt, T.~Pitassi, O.~Reingold, and R.~Zemel.
\newblock Fairness through awareness.
\newblock In {\em ITCS}, pages 214--226, 2012.

\bibitem{techrep}
N.~Shahbazi and A.~Asudeh.
\newblock Data-centric reliability evaluation of individual predictions.
\newblock {\em CoRR, abs/2204.07682}, 2022.

\bibitem{adult}
M.~Lichman.
\newblock Adult income dataset, {UCI} machine learning repository.
\newblock \url{https://archive.ics.uci.edu/ml/datasets/adult}, 2013.

\bibitem{neyman1936contributions}
J.~Neyman and E.~S. Pearson.
\newblock Contributions to the theory of testing statistical hypotheses.
\newblock {\em Statistical Research Memoirs}, 1936.

\bibitem{olteanu2019social}
A.~Olteanu, C.~Castillo, F.~Diaz, and E.~Kiciman.
\newblock Social data: Biases, methodological pitfalls, and ethical boundaries.
\newblock {\em Frontiers in Big Data}, 2:13, 2019.

\bibitem{fairmlbook}
S.~Barocas, M.~Hardt, and A.~Narayanan.
\newblock Fairness and machine learning: Limitations and opportunities.
\newblock \url{fairmlbook.org}, 2019.

\bibitem{barocas2016big}
S.~Barocas and A.~D. Selbst.
\newblock Big data's disparate impact.
\newblock {\em Calif. L. Rev.}, 104:671, 2016.

\bibitem{jk2019bias}
J.~Kleinberg.
\newblock Fairness, rankings, and behavioral biases.
\newblock FAT*, 2019.

\bibitem{drosou2017diversity}
M.~Drosou, H.~Jagadish, E.~Pitoura, and J.~Stoyanovich.
\newblock Diversity in big data: A review.
\newblock {\em Big data}, 5(2):73--84, 2017.

\bibitem{berrey2015enigma}
E.~Berrey.
\newblock {\em The enigma of diversity: The language of race and the limits of racial justice}.
\newblock University of Chicago Press, 2015.

\bibitem{dobbin2016diversity}
F.~Dobbin and A.~Kalev.
\newblock Why diversity programs fail and what works better.
\newblock {\em Harvard Business Review}, 94(7-8):52--60, 2016.

\bibitem{simpson1949measurement}
E.~H. Simpson.
\newblock Measurement of diversity.
\newblock {\em Nature}, 163(4148), 1949.

\bibitem{surowiecki2005wisdom}
J.~Surowiecki.
\newblock {\em The wisdom of crowds}.
\newblock Anchor, 2005.

\bibitem{agrawal2009diversifying}
R.~Agrawal, S.~Gollapudi, A.~Halverson, and S.~Ieong.
\newblock Diversifying search results.
\newblock In {\em WSDM}, pages 5--14. ACM, 2009.

\bibitem{mehrabi2021survey}
N.~Mehrabi, F.~Morstatter, N.~Saxena, K.~Lerman, and A.~Galstyan.
\newblock A survey on bias and fairness in machine learning.
\newblock {\em ACM Computing Surveys (CSUR)}, 54(6):1--35, 2021.

\bibitem{friedman1996bias}
B.~Friedman and H.~Nissenbaum.
\newblock Bias in computer systems.
\newblock {\em TOIS}, 14(3):330--347, 1996.

\bibitem{torralba2011unbiased}
A.~Torralba and A.~A. Efros.
\newblock Unbiased look at dataset bias.
\newblock In {\em CVPR 2011}, pages 1521--1528. IEEE, 2011.

\bibitem{crawford2013hidden}
K.~Crawford.
\newblock The hidden biases in big data.
\newblock {\em Harvard business review}, 1(4), 2013.

\bibitem{diakopoulos2015algorithmic}
N.~Diakopoulos.
\newblock Algorithmic accountability: Journalistic investigation of computational power structures.
\newblock {\em Digital journalism}, 3(3):398--415, 2015.

\bibitem{li2020towards}
Y.~Li, H.~Sun, and W.~H. Wang.
\newblock Towards fair truth discovery from biased crowdsourced answers.
\newblock In {\em SIGKDD}, pages 599--607, 2020.

\bibitem{lazier2023fairness}
S.~Lazier, S.~Thirumuruganathan, and H.~Anahideh.
\newblock Fairness and bias in truth discovery algorithms: An experimental analysis.
\newblock {\em arXiv preprint arXiv:2304.12573}, 2023.

\bibitem{SalimiRHS19}
B.~Salimi, L.~Rodriguez, B.~Howe, and D.~Suciu.
\newblock Interventional fairness: Causal database repair for algorithmic fairness.
\newblock In {\em {SIGMOD}}, pages 793--810. {ACM}, 2019.

\bibitem{tae2019data}
K.~H. Tae, Y.~Roh, Y.~H. Oh, H.~Kim, and S.~E. Whang.
\newblock Data cleaning for accurate, fair, and robust models: A big data-{AI} integration approach.
\newblock In {\em DEEM workshop}, pages 1--4, 2019.

\bibitem{salimi2020database}
B.~Salimi, B.~Howe, and D.~Suciu.
\newblock Database repair meets algorithmic fairness.
\newblock {\em ACM SIGMOD Record}, 49(1):34--41, 2020.

\bibitem{martinez2019fairness}
F.~Mart{\'\i}nez-Plumed, C.~Ferri, D.~Nieves, and J.~Hern{\'a}ndez-Orallo.
\newblock Fairness and missing values.
\newblock {\em arXiv preprint arXiv:1905.12728}, 2019.

\bibitem{shahbazi2023through}
N.~Shahbazi, N.~Danevski, F.~Nargesian, A.~Asudeh, and D.~Srivastava.
\newblock Through the fairness lens: Experimental analysis and evaluation of entity matching.
\newblock {\em Proceedings of the VLDB Endowment}, 16(11):3279--3292, 2023.

\bibitem{fanourakis2023fairer}
N.~Fanourakis, C.~Kontousias, V.~Efthymiou, V.~Christophides, and D.~Plexousakis.
\newblock Fairer demo: Fairness-aware and explainable entity resolution.
\newblock 2023.

\bibitem{lin2020identifying}
Y.~Lin, Y.~Guan, A.~Asudeh, and H.~Jagadish.
\newblock Identifying insufficient data coverage in databases with multiple relations.
\newblock {\em Proceedings of the VLDB Endowment}, 13(12):2229--2242, 2020.

\bibitem{accinelli2020coverage}
C.~Accinelli, S.~Minisi, and B.~Catania.
\newblock Coverage-based rewriting for data preparation.
\newblock In {\em EDBT Workshops}, 2020.

\bibitem{accinelli2021impact}
C.~Accinelli, B.~Catania, G.~Guerrini, and S.~Minisi.
\newblock The impact of rewriting on coverage constraint satisfaction.
\newblock In {\em EDBT Workshops}, 2021.

\bibitem{shetiya2022fairness}
S.~Shetiya, I.~P. Swift, A.~Asudeh, and G.~Das.
\newblock Fairness-aware range queries for selecting unbiased data.
\newblock In {\em ICDE}. IEEE, 2022.

\bibitem{tae2021slice}
K.~H. Tae and S.~E. Whang.
\newblock Slice tuner: A selective data acquisition framework for accurate and fair machine learning models.
\newblock In {\em SIGMOD}, pages 1771--1783, 2021.

\bibitem{chung2019slice}
Y.~Chung, T.~Kraska, N.~Polyzotis, K.~H. Tae, and S.~E. Whang.
\newblock Slice finder: Automated data slicing for model validation.
\newblock In {\em ICDE}, pages 1550--1553. IEEE, 2019.

\bibitem{sagadeeva2021sliceline}
S.~Sagadeeva and M.~Boehm.
\newblock Sliceline: Fast, linear-algebra-based slice finding for ml model debugging.
\newblock In {\em SIGMOD}, pages 2290--2299, 2021.

\bibitem{kleinberg2016inherent}
J.~Kleinberg, S.~Mullainathan, and M.~Raghavan.
\newblock Inherent trade-offs in the fair determination of risk scores.
\newblock {\em arXiv preprint arXiv:1609.05807}, 2016.

\bibitem{zadrozny2001obtaining}
B.~Zadrozny and C.~Elkan.
\newblock Obtaining calibrated probability estimates from decision trees and naive bayesian classifiers.
\newblock In {\em ICML}, volume~1, pages 609--616. Citeseer, 2001.

\bibitem{zadrozny2002transforming}
B.~Zadrozny and C.~Elkan.
\newblock Transforming classifier scores into accurate multiclass probability estimates.
\newblock In {\em SIGKDD}, pages 694--699, 2002.

\bibitem{platt1999probabilistic}
J.~Platt et~al.
\newblock Probabilistic outputs for support vector machines and comparisons to regularized likelihood methods.
\newblock {\em Advances in large margin classifiers}, 10(3):61--74, 1999.

\bibitem{niculescu2005predicting}
A.~Niculescu-Mizil and R.~Caruana.
\newblock Predicting good probabilities with supervised learning.
\newblock In {\em Proceedings of the 22nd international conference on Machine learning}, pages 625--632, 2005.

\bibitem{chatfield93predictionintervals}
C.~Chatfield.
\newblock Prediction intervals.
\newblock {\em Journal of Business and Economic Statistics}, 11:121--135, 1993.

\bibitem{pearce2018high}
T.~Pearce, A.~Brintrup, M.~Zaki, and A.~Neely.
\newblock High-quality prediction intervals for deep learning: A distribution-free, ensembled approach.
\newblock In {\em International conference on machine learning}, pages 4075--4084. PMLR, 2018.

\bibitem{khosravi2010lower}
A.~Khosravi, S.~Nahavandi, D.~Creighton, and A.~F. Atiya.
\newblock Lower upper bound estimation method for construction of neural network-based prediction intervals.
\newblock {\em IEEE transactions on neural networks}, 22(3):337--346, 2010.

\bibitem{angelopoulos2021gentle}
A.~N. Angelopoulos and S.~Bates.
\newblock A gentle introduction to conformal prediction and distribution-free uncertainty quantification.
\newblock {\em arXiv preprint arXiv:2107.07511}, 2021.

\bibitem{shafer2008tutorial}
G.~Shafer and V.~Vovk.
\newblock A tutorial on conformal prediction.
\newblock {\em Journal of Machine Learning Research}, 9(3), 2008.

\bibitem{harradon2018causal}
M.~Harradon, J.~Druce, and B.~Ruttenberg.
\newblock Causal learning and explanation of deep neural networks via autoencoded activations.
\newblock {\em arXiv preprint arXiv:1802.00541}, 2018.

\bibitem{ribeiro2016should}
M.~T. Ribeiro, S.~Singh, and C.~Guestrin.
\newblock " why should i trust you?" explaining the predictions of any classifier.
\newblock In {\em SIGKDD}, pages 1135--1144, 2016.

\bibitem{gunning2019darpa}
D.~Gunning and D.~Aha.
\newblock Darpa’s explainable artificial intelligence ({XAI}) program.
\newblock {\em AI Magazine}, 40(2):44--58, 2019.

\end{thebibliography}

\end{document}

\end{article}


\begin{article}
{Interaction Management in Crowdsourcing}
{Yansheng Wang, Tianshu Song, Qian Tao, Yuxiang Zeng, Zimu Zhou,
Yi Xu, Yongxin Tong, Lei Chen}
\graphicspath{{submissions/yongxin}}
% link to instruction: https://tc.computer.org/tcde/tcde-bulletin-author-instructions/
% \documentclass[11pt,dvipdfm]{article}
\documentclass[11pt]{article}
\usepackage{tabularx}
\usepackage{ragged2e}  % for '\RaggedRight' macro (allows hyphenation)
\usepackage{booktabs}  % for \toprule, \midrule, and \bottomrule macros
\usepackage{textcomp}
\usepackage{amsfonts,amsmath}
\usepackage{deauthor,times}
\usepackage{graphicx} % 
\usepackage{hyperref}
\usepackage{comment}
\graphicspath{{asudeh/}}
\usepackage{soul}
\usepackage{subcaption}
\usepackage{ulem}
\usepackage{wrapfig}
\usepackage{color}
\usepackage{xspace}
\newtheorem{problem}{Problem}

%\DeclareMathOperator*{\argmax}{arg\,max}

%remove the following commands/package b4 submission
\newcommand{\hide}[1]{}
\newcommand{\eat}[1]{}
\newcommand{\resolved}[1]{\hide{#1}}
\newcommand{\abol}[1]{\textcolor{red}{Abol: #1}}
\newcommand{\mahdi}[1]{\textcolor{red}{Mahdi: #1}}
\newcommand{\nima}[1]{\textcolor{red}{Nima: #1}}

\newcommand{\dee}{\mathcal{D}}
\newcommand{\Gee}{\mathcal{G}}
\newcommand{\gee}{\mathbf{g}}
\newcommand{\ee}{\mathbf{e}}
\newcommand{\es}{\mathcal{S}}
\newcommand{\el}{\mathcal{L}}
\newcommand{\xx}{\mathcal{x}}
\newcommand{\dist}{\xi}
\newcommand{\alg}{\mathsf{A}}
\newcommand{\qu}{\mathbf{q}}
\newcommand{\ex}{\mathbf{x}}
\newcommand{\ti}{\mathbf{t}}
\newcommand{\sdt}{\mathsf{SDT}}
\newcommand{\wdt}{\mathsf{WDT}}
\newcommand{\Qu}{\mathbf{Q}}
\newcommand{\pe}{\mathbb{P}}
\newcommand{\megam}{\mathcal{M}}
\newcommand{\eps}{\varepsilon}
\newcommand{\enet}{{$\varepsilon$-{\bf net}}\xspace}
\newcommand{\net}{{\tt net}\xspace}
\newcommand{\vcd}{VC-dimension\xspace}
\newcommand{\at}[1]{{\tt \small #1}\xspace}
\newcommand{\pr}{Pr}

\newcommand{\sharpP}{\mbox{\#P}}
\newcommand{\NP}{\mathsf{NP}}
\newcommand{\LP}{\mathsf{LP}}
\newcommand{\IP}{\mathsf{IP}}
\newcommand{\ru}{{\sc {RU}}\xspace}
\newcommand{\sru}{{\sc {strongRU}}\xspace}
\newcommand{\wru}{{\sc {weakRU}}\xspace}

\newcommand{\fmsystem}{{\sc Chameleon}\xspace}
\newcommand{\fm}{$\mathcal{F}$\xspace}

\newtheorem{experiment}{Experiment}

\begin{document}

\title{Coverage-based Data-centric Approaches for \\Responsible and Trustworthy AI\thanks{This research was supported by the National Science Foundation under grant No. 2107290.}}

\author{
\begin{tabular}[t]{c@{\extracolsep{2.4em}}c@{\extracolsep{2.4em}}c@{\extracolsep{2.3em}}c} 
Nima Shahbazi & Mahdi Erfanian & Abolfazl Asudeh \\ 
University of Illinois Chicago & University of Illinois Chicago & University of Illinois Chicago\\
 nshahb3@uic.edu & merfan2@uic.edu & asudeh@uic.edu
\end{tabular}
}

\maketitle


\begin{abstract}
The grand goal of data-driven decision systems is to help make decisions easier, more accurate, at a higher scale, and also just. However, data-driven algorithms are only as good as the data they work with. Yet, data sets, especially those with social data, often do not represent minorities. The paucity of training data is a perpetual problem for AI, and the outcome of ML models for cases not represented in their training data is often not reliable. 
Hence, without properly addressing the lack of representation issues in data, we cannot expect AI-based societal solutions to have responsible and trustworthy outcomes. 

This paper focuses on data coverage as a data-centric approach for identifying and resolving misrepresentation of minorities in data.
To achieve this goal, we propose novel algorithms that (a) {\it identify} and {\it resolve} insufficient data coverage across data with different modalities and (b) use lack of representation information to generate data-centric {\it reliability warnings}.
 \end{abstract}
 
 %%%%%%%%%%%%%%%%%%%%%%%%%%%%%%%% INTRO  %%%%%%%%%%%%%%%%%%%%%%%%%%%%%%%%
\section{Introduction}\label{sec:intro} % Abstract+Intro: up to 2.5 pages 
Data-driven decision-making has shaped every corner of human life, spanning from autonomous vehicles to healthcare and even predictive policing and criminal justice. A pivotal concern, especially in applications that affect individuals, revolves around the reliability of the decisions rendered by the system.
It is easy to see that the accuracy of a data-driven decision depends, first and foremost, on the data used to make it. Essentially, the system learns the phenomena that data represent. While we may desire that the data should represent the underlying data distribution from which the production data is drawn, this alone may be insufficient, as it merely enables the model to perform well for the average case.
As a result, a model with a high accuracy could fail for specific regions in the data with insufficient representation. These regions may matter because they frequently represent some minority population in society. They could also represent cases that may not happen very often but have a relevant impact on the correctness of a critical decision.
In short, if the data fails to sufficiently represent a specific population, the outcome of the decision system for that population may not be trustworthy.

The phenomenon known as \textit{Representation Bias} can arise from how the data was originally collected, or it could be the result of biases introduced post-collection—whether historically, cognitively, or statistically.

Representation bias is essentially inevitable without a systematic approach to data collection. 
For example, in the context of survey data collection, vital steps involve identifying all populations within the underlying distribution based on desired demographic information and ensuring comprehensive coverage with sufficient samples from each group. 
Even then, only an (uncontrolled) subset of the invitees will opt-in to respond to the survey.
Another challenge lies in the fact that data scientists often lack control over the data collection process, leading to the reliance on ``found data'' in the majority of data-driven systems. Therefore, with no guarantee on the aforementioned steps in the data collection process, the found data is most likely a biased sample.
Acknowledging the potential harms of representation bias, the notion of \textit{Data Coverage}~\cite{asudeh2019assessing,shahbazi2023representation} has been proposed to ensure the adequate representation of minority groups in data sets employed for decision-making and developing sophisticated data science tools. 

Addressing representation issues in data poses various challenges depending on the modality of the data. In this paper, we focus on identifying and resolving lack of coverage issues in data with different modalities.
We start by proposing a variety of techniques (spanning from geometric and combinatorial optimization to crowd-souring) aimed at efficiently detecting insufficient coverage on structured data sets with non-ordinal categorical and continuous attributes, as well as image data sets. Next, we propose a range of approaches grounded in data integration and generative data augmentation to address the lack of coverage by enriching the data sets with more data. However, with limited control over the data collection processes, it could be difficult and expensive to resolve all misrepresentations. 
Since adding more data is not always possible, we proceed to introduce data-centric preventive solutions that warn the user about the reliability of their predictions regarding representation bias issues. These warnings assist users in determining whether they trust the outcomes of the models or exercise caution. 

 %%%%%%%%%%%%%%%%%%%%%%%%%%%%%%%% IDENTIFICATION  %%%%%%%%%%%%%%%%%%%%%%%%%%%%%%%%
\section{Detecting Insufficient Representation of Minorities}\label{sec:identification} %up to 3.5 pages
Representation bias happens when the development (training data) population under-represents 
and subsequently fails to generalize well 
for some parts of the target population, due to historical bias, sampling bias, etc.
The notion of {\it data coverage} has been studied across different settings in \cite{shahbazi2023representation} as a metric to measure representation bias. At a high level, coverage is referred to as having enough similar entries for each object in a data set. 
For a better understanding, let us go over the definition of the generalized notion of coverage:

\begin{definition}[Data Coverage]\label{def:coverage}
Consider a data set $\dee$ with $n$ tuples, each consisting of $d$ attributes of interest $\mathbf{x}=\{x_1, x_2, \cdots,x_d\}$, such as {\tt gender}, {\tt race}, {\tt salary}, {\tt age}, etc, that are used for coverage identification.
The data set also contains target attributes $\mathbf{y} = \{ y_1,\cdots,y_{d'}\}$ that may or may not be considered for the coverage problem.
A query point $q$ is not covered by the data set $\dee$, if there are not ``enough'' data points in $\dee$ that are representative of $q$.
To generalize the notion of coverage, let us define $\gee(q)$ as the universe of tuples that would represent $q$ and let $\gee_\dee(q) = \gee(q)\cap \dee$. In other words, $\gee_\dee(q)$ are the set of tuples in $\dee$ that represent $q$.
Using this notation, we define the coverage of $q$ as the size of $\gee_\dee(q)$. That is,
$cov(q,\dee) = | \gee_\dee(q)|$.
Given a value $\tau$, $q$ is covered if $cov(q,\dee)>\tau$.
Similarly, a group $\gee$ is not covered if $\gee\cap \dee<\tau$.
The {\it uncovered region} in a data set is the collection of groups that are not covered by it.
\end{definition}

\subsection{Structured Data}
In this section, we focus on identifying representation bias in structured data.
Depending on the type of the attributes of interest, we categorize the techniques into two classes based on whether they target the problem for non-ordinal {\it categorical} (e.g. {\tt race}, {\tt gender}) or ordinal {\it continuous} (e.g. {\tt age}) attributes. The attributes of interest considered for representation bias often include sensitive attributes such as {\tt race} and {\tt gender} but are not necessarily limited to them.

\subsubsection{Categorical Attributes}

For cases where attributes of interest are non-ordinal categorical,
the cartesian product of values on a subset of attributes $\mathbf{x}'\subseteq \mathbf{x}$, form a set of (sub-)groups.
For example, $\{$ {\tt white male}, {\tt white female}, {\tt black male} $,\cdots\}$ are the subgroups defined on the attributes {\tt (race,gender)}.
We refer to the number of attributes used to specify a subgroup as the {\it level} of that subgroup.
For example, the level of the subgroup {\tt white male} is 2, while the level of the subgroup {\tt male} is 1.
We use $\ell(\gee)$, to refer to the level of a subgroup $\gee$.
Similarly, we say a subgroup $\gee'$ is a subset of $\gee$, if the groups specifying $\gee'$ are a superset of the ones for $\gee$. For example {\tt (married white male)} a subset of the more general group {\tt (white male)}. That is, the set of individuals in group {\tt (married white male)} are a subset of {\tt (white male)}.
Moreover, we say a subgroup $\gee$ is a {\it parent} of the subgroup $\gee'$, if $\gee'\subset \gee$ and $\ell(\gee)=\ell(\gee')+1$. For example, the subgroup {\tt (white male)} is a parent of the subgroup {\tt (married white male)}.
We use \textit{patterns} to refer to uncovered subgroups.
A pattern $P$ is a string of $d$ values, where $P[i]$ is either a value from the domain of $x_i$, or it is ``unspecified'', specified with $X$. 
For example, consider a data set with three binary attributes of interest $\mathbf{x}=\{x_1, x_2, x_3\}$. The pattern $P=X01$ specifies all the tuples for which $x_2=0$ and $x_3=1$ ($x_1$ can have any value).
The set of patterns that identify most general uncovered subgroups are called {\it Maximal Uncovered Patterns} (MUPs).

No polynomial time algorithm can guarantee the enumeration of the entire MUPs, however, several algorithms inspired by set enumeration and the Apriori algorithm for association rule mining are proposed to efficiently address this problem~\cite{asudeh2019assessing}.
In this regard, we introduce \textit{Pattern Graph} data structure that exploits the relationship between patterns to do less work than computing all uncovered patterns by removing the non-maximal ones. 
The parent-child relationship between the patterns is represented in a graph that can be used to find better algorithms. 
\textit{Pattern-Breaker} starts from the top of the graph where the general patterns are and moves down by breaking each pattern into more specific ones. If a pattern is uncovered, then all of its descendants are also uncovered and they can not be an MUP, even if they have a parent that is covered. Therefore, this subgraph of the pattern graph can be pruned. 
The issue with \textit{Pattern-Breaker} is that it explores the covered regions of the pattern graph and for the cases where there are a few uncovered patterns, it has to explore a large portion of the exponential-size graph. 
To tackle this, \textit{Pattern-Combiner} algorithm is proposed that performs a bottom-up traversal of the pattern graph. It uses an observation that the coverage of a node at the level of the pattern graph can be computed as the sum of the coverage values of its children. 
The problem with \textit{Pattern-Combiner} is that it traverses over the uncovered nodes first and therefore, it will not perform well for the cases in which most of the nodes in the graph are uncovered. 
In fact, for the cases where most of the MUPs are placed in the middle of the graph, both \textit{Pattern-Breaker} and \textit{Pattern-Combiner} will not be as efficient as they should traverse half of the graph. Therefore, we propose \textit{Deep-Diver}, a search algorithm based on Depth-First-Search that quickly finds the MUPs, and uses them to limit the search space by pruning the nodes both dominating and dominated by the discovered MUPs.

\begin{figure*}[!tb]
    \begin{minipage}[t]{0.31\linewidth}
        \centering
        \includegraphics[width=\textwidth]{submissions/submission1/shahbazi/covcube1.jpg}
        \caption{\small Categorical attributes: the uncovered region of a toy example, as the collection of three MUPs.}
        \label{fig:covcube1}
    \end{minipage}
    \hfill
    \begin{minipage}[t]{0.31\linewidth}
        \centering
        \includegraphics[width=\textwidth]{submissions/submission1/shahbazi/cvrg_2_1.jpg}
        \caption{\small Continuous attributes, 2D: identifying the covered region in the gray Voronoi cell.}
        \label{fig:cvrg_2_1}
    \end{minipage}
    \hfill
    \begin{minipage}[t]{0.31\linewidth}
        \centering
        \includegraphics[width=\textwidth]{submissions/submission1/shahbazi/cvrg_2_2.jpg}
        \caption{ \small Continuous attributes, 2D: Uncovered region marked in red.}
        \label{fig:cvrg_2_2}
    \end{minipage}
\vspace{-5mm}
\end{figure*}

\subsubsection{Continuous Attributes}
Data in the real world often consists of a combination of continuous and discrete values. While simple solutions like binning {\tt age} into {\tt young} and {\tt old} can transform the continuous space into discrete. However, they may lead to coarse groupings that are sensitive to the thresholds chosen. It may be inappropriate to treat a 35-yo as {\tt young} but a 36-yo as {\tt old}. 
Therefore, we extend the notion of coverage to continuous space. Particularly, given data set $\dee$ with $n$ tuples over $d$ attributes, and vicinity radius $\rho$ and coverage threshold $k$, we want to identify the uncovered region -- the universe of uncovered query points.
A query point in continuous data space is covered if there are enough (at least $k$) data points in its $\rho$-vicinity neighborhood. $\rho$-vicinity neighborhood is the circle centered at the query point with radius $\rho$.

Depending on the number of attributes in a data set, we propose two algorithms for identifying uncovered regions in data~\cite{asudeh2021coverage}. 
The first algorithm known as \textit{Uncovered-2D} studies coverage over two-dimensional data sets where $\mathbf{x}=\{x_1,x_2\}$. To find the number of circles that a query point falls into and consequently discover the uncovered region, \textit{Uncovered-2D} makes a connection to $k$-th order Voronoi diagrams.
Consider a data set $\mathcal{D}$ and its corresponding $k$-th order Voronoi diagram. For every tuple $t\in \mathcal{D}$, let $\circ_t$ be the $d$-dimensional sphere ($d$-sphere) with radius $\rho$ centered at $t$.
Consider a $k$-voronoi cell $\mathcal{V}(S)$ in the $k$-th order Voronoi diagram $V_k(\mathcal{D})$.
Any point $q$ inside the intersections of the $d$-spheres of tuples in $S$, i.e. $q\in \underset{\forall t\in S}{\cap ~\circ_t}$, is covered, while all other points in the region are uncovered.
 The algorithm starts by constructing the $k$-th order Voronoi diagram of the data set and then for each Voronoi cell $\mathcal{V}(S)$ in the diagram, it computes the intersection of the circles of the tuples in $S$ and marks the portion of $\mathcal{V}(S)$ that falls outside it as uncovered.
After identifying the uncovered region, a 2D map of $\{x_1,x_2\}$ value combinations is used to report the region to the user.
The algorithm for the 2D case can be extended to the general case by relaxing the assumption on the number of attributes to discover the exact uncovered region, however, due to the curse of dimensionality, the search size space explodes as the number of dimensions increases and as a result, the algorithm will not be practical. Therefore, we propose a randomized approximation algorithm based on the geometric notion of \enet. 
Let $\mathcal{X}$ be a set and $\mathcal{R}$ be a set of subsets of $\mathcal{X}$. A set $\mathcal{N}\subset \mathcal{X}$ is an \enet for $\mathcal{X}$ if for any range $r\in\mathcal{R}$, if  $|r\cap \chi|>\eps|\chi|$, then $r$ contains at least one point of $N$.
The idea, at a high level, is to draw enough random samples from the space of potential query points to form an \enet. 
We then label the sampled query points as $\{-1,+1\}$ depending on whether those are covered or not, and learn the uncovered regions using the samples.

\subsection{Image Data}
Many known incidents of machine failures due to the lack of representation were on image data.
We consider an image data set with a fixed number of low-cardinality sensitive attributes such as {\tt\small race} and {\tt\small gender}. 
It is common that image data sets {\it lack explicit values} for sensitive attributes, which are crucial for coverage identification. An image data set is often a collection of images from different domains with little to no information about their domain and which groups they belong to. As a result, even studying coverage over low-cardinality and categorical attributes of interests is challenging in these cases.

\begin{wrapfigure}{R}{0.42\textwidth}
\centering
\vspace{-3mm}
\scriptsize
\begin{tabular}{|@{}c|@{}c@{}|@{}c@{}|@{}c@{}|} 
 \hline
{\bf data set} & {\bf classifier} & {\bf accuracy} & {\bf precision} \\ 
 &  &  & {\bf on female} \\ \hline
UTKFace:~& DeepFace (opencv) & 93.56 & {52.02}\\\cline{2-4}
({\tt females}=200,& DeepFace (retinaface) & 94.16 & {56.15}\\\cline{2-4}
{\tt males}=2800) & BaseCNN & 97.6 & 74.8\\
\hline
UTKFace:~& DeepFace (opencv) & 96.53 & {\bf 8.0}\\\cline{2-4}
({\tt females}=20,& DeepFace (retinaface) & 96.43 & {\bf 10.09}\\\cline{2-4}
{\tt males}=2980)& BaseCNN & 97.6 & {\bf 21.59}\\
\hline
\end{tabular}
\vspace{-3mm}
\caption{\small ML models' low performance for females in the presence of representation bias.~\cite{mousavi2024data}}\label{fig:mlfails}
\vspace{-3mm}
\end{wrapfigure}

In Figure~\ref{fig:mlfails}, we show that due to the issues such {\it machine bias} and {\it lack of distribution generalizability},
solely relying on state-of-the-art machine learning (ML) techniques fail to effectively identify lack of coverage in image data sets. Therefore, we propose an approach based on combining crowdsouring with ML~\cite{mousavi2024data}. 
Crowdsourcing is particularly promising for image data, for tasks such as image labeling, which, while challenging for the machine, are "easy" for human beings to conduct with minimal error. 

A key observation that enables a cost-effective crowdsourcing approach is that, while studying coverage, we would only like to find out if there are {\it enough tuples from each subgroup}.
Suppose a subgroup is covered if there are $\tau=100$ instances of it in the data set. Assume the (majority) group $\gee_1$ contains $n_1 \gg 100$ objects in the data set. 
To verify that $\gee_1$ is covered, it is enough for the crowd to discover 100 of those objects, not the entire $n_1$. 
Following this, $O(\tau)$ provides a lower bound on the number of crowd tasks required to verify a given group is covered. 
Still, this lower bound only holds for the groups that are covered, i.e., there is at least $\tau$ of those in the data set.
Surprisingly, verifying that a minority group is indeed uncovered is cumbersome, unlike the majority group.
This is because even though discovering $\tau$ objects from a group is enough for verifying that it is covered, one cannot {\it verify} a group is uncovered until there is a chance that the data set might still have enough objects from that group. Thus, assuming a non-zero probability for each unlabeled object to belong to each group, {one might need to ask the crowd to label the entire data set before they can confirm that a specific group is uncovered}.

Our idea for addressing this challenge is to
design {\it a divide and conquer algorithm} that, instead of {point queries}, uses {\it set queries} to iteratively eliminate subsets of data that {does not include any object from the given group}.
At a high level, our idea is to ask a set query from the crowd, inquiring whether the selected set contains at least one object from the given group $\gee$.
The user may provide two responses (yes/no). 
Interestingly, {in either case}, the user response provides valuable information that helps efficiently identify the coverage.
If the answer is ``No'', the set does not include any object from the given group $\gee$. As a result, the algorithm can safely prune the set, asking no further questions about it. In particular, for a group that is not covered, one can expect to see no answers on large set queries helping to prune a significant portion of the data set quickly.
On the other hand, if the answer is ``yes'', the set contains {at least} one object from the group $\gee$. As a result, the algorithm cannot prune the subset since it can have any number (larger than one) of the objects in $\gee$.
At first glance, the queries with yes answers do not provide helpful information as the algorithm cannot prune the subset (hence it needs to divide it into smaller subsets).
However, a key observation is that {the algorithm will only observe a limited number of yes answers} before it stops.
The reason is that the number of set queries with yes answers provides a {lower-bound} on the number of objects from $\gee$ in the data set. As a result, the algorithm can stop as soon as the lower bound reaches $\tau$, knowing that $\gee$ is covered.
The D\&C approach verifies the data coverage for a given group, while our goal is to identify the uncovered regions for a given set of sensitive attributes. The next question is how to utilize this algorithm for efficient coverage identification on different scenarios of sensitive attributes, forming intersectional or non-intersectional groups.
In particular, how can we find maximal uncovered patterns?
Our idea is to apply sampling and aggregate estimation techniques to find the groups that even if merged are likely to still be uncovered. This will help reduce the coverage identification cost by running the D\&C approach for the merged groups once.
 %%%%%%%%%%%%%%%%%%%%%%%%%%%%%%%% RESOLUTION  %%%%%%%%%%%%%%%%%%%%%%%%%%%%%%%%
\section{Resolving Insufficient Representation}\label{sec:resolution}

Data integration~\cite{nargesian2021tailoring,nargesian2022responsible} and data augmentation~\cite{sharma2020data,DBLP:journals/jair/ChawlaBHK02,iosifidis2018dealing,celis2020data} are considered as the primary solutions for reducing data coverage issues in a data set. 
Data integration is promising when external sources of data are available. On the other hand, recent advancements in generative AI and foundation models have enabled efficient and effective augmentation of data sets with synthetic data. 
Therefore, in the following, we review two approaches, one from each category, in the context of lack of coverage resolution.

\subsection{Data Integration}\label{sec:resolution:integration}

Data integration is to consolidate data from different sources into a single, unified view. 
Although it is an effective solution to acquire additional data from different distributions,
there are sampling policy and cost-efficiency concerns that need to be examined.  
Therefore, {\it Data Distribution Tailoring ({\sc DT})} introduces data integration techniques for resolving insufficient representation of subgroups in a data set in the most cost-effective manner~\cite{nargesian2021tailoring}.
A query to {\sc DT} 
consists of a target schema, and a set of group distribution requirements in the form of the minimum counts (e.g., ``{\tt\small 1,000 breast cancer monitoring data in Chicago with at least 30\% label=positive, and at least 20\% black patients}''). 
Collecting a fresh sample from a data view is costly (monetary, human resources, and/or computation cost)~\cite{asudeh2022towards}.
Therefore, {\sc DT} focuses on satisfying the count requirements with minimum cost. 
Given an input query and a lake of available data sources, the first step is to discover a collection of candidate data views that satisfy the target schema.
Each data view $v_i$ is a projection-join $v_i = \Pi\big(D_{i1}\bowtie\cdots\bowtie D_{ik_i} \big)$, where $D_{ij}$ is a data set in a given data lake.
Let us suppose the data views are already discovered.
At a high level, {\sc DT} follows an iterative approach that at each iteration a data view is selected to be queried.
Each query to a data view has a fixed cost and returns a sample that may or may not satisfy the query constraints.
The samples that are either not fresh, or do not satisfy the query are discarded.
Hence, the essential question towards a cost-effective data integration is {\it what data view to query next}.
Depending on the available information about the data sources, various techniques may be employed. 

For the cases when the group distributions are known, the process of collecting the target data set is a sequence of iterative steps, where at every step, the algorithm chooses a data view, queries it, and if the obtained tuple contributes to one of the groups for which the count requirement is not yet fulfilled, it is kept, otherwise discarded. To do so, a {Dynamic Programming (DP)} algorithm is proposed. An optimal source at each iteration minimizes the sum of its sampling cost plus the expected cost of collecting the remaining required groups, based on its sampling outcome.
The DP algorithm, however, has a pseudo-polynomial time complexity. Hence, it quickly becomes intractable for cases where the minimum count requirements for the groups are not small. 
For cases where the (sensitive) attribute of interest is binary, such as (biological) {\tt sex}={\tt \{male, female\}}, and the cost to query data is similar from all sources, it turns out that the optimal strategy is to query the data source with {maximum probability of obtaining a sample from the minority group}.
Expanding the binary-attributes algorithm for non-binary cases, the problem can be modeled as an extension of the ``{\it coupon collector's}'' problem~\cite{motwani1995randomized}, where the goal is to collect $m_i$ instances from each coupon (group) $\gee_i$.
At each iteration, the coupon collector's algorithm identifies a data view as most promising and queries it. In simple terms, a data view with a smaller query cost and a higher chance of obtaining minority groups is more promising.


For the cases where the group distributions are unknown, we model DT as a {\it multi-armed bandit} problem, where every data view is modeled as an arm. 
Every arm has an unknown distribution of different groups while pulling an arm (i.e., querying the corresponding data view) has a cost.
During various iterations, the algorithms pull the arms in an order that its expected total {\it reward} is maximized.
Arguing that the reward of obtaining a tuple from a group is proportional to how rare this group is across different data views, 
we design the reward function based on the expected cost one needs to pay in order to collect a tuple from a specific group.  
As the bandit strategy, we adopt {\it Upper Confidence Bound (UCB)} to balance exploration and exploitation. At every iteration, for every arm, UCB computes confidence intervals for the expected reward and selects the arm with the maximum upper bound of reward to be explored next.

\subsection{Data Augmentation using Foundation Models}

While data integration provides a promising approach for resolving coverage issues in a data set, its effectiveness is limited to the availability of external data sources that are rich enough to find sufficient fresh samples from minority groups. This, however, is not always possible, especially since the minority samples are rare and not easy to obtain.
Fortunately, recent advancements in Generative AI and Foundation Models have enabled synthesizing samples that are otherwise challenging to obtain from the real world.

Therefore, as an alternative approach to data integration, we turn our attention to the Foundation Models and Generative AI for resolving the lack of coverage. 
Particularly, models such as {\sc DALL.E}\footnote{\url{https://openai.com/dall-e-2}} have emerged as powerful tools for generating multi-modal data such as image, audio, and video.
 
We formalize the foundation model \fm as a black-box function with the following inputs, that once queried synthesize an output tuple.
\begin{itemize}
    \item {\bf Prompt}: A natural language description providing instructions on the details of the tuple to be generated. For instance, a prompt for image generation might be ``A realistic photo of a white cat running in a backyard.''
    \item {\bf Guide}: In cases where only a prompt is provided, the foundation model uses its imagination to generate the requested tuple. For the previous example, the prompt of a cat image, the breed, size, background, and other details are generated based on the model's imagination. Alternatively, a guide can be provided to influence the generation process. The guide is formalized as a pair $(t,m)$ where $t$ is a tuple and $m$ is a mask specifying which parts of the guide tuple should be changed. Using the cat example, $t$ can be a cat image and $m$ can specify the foreground to be regenerated.
\end{itemize}

There are multiple challenges towards effective data set augmentations using foundation models. 
First, we have to determine the minimal set of synthetic tuples that once added to the original data set, under-representation issues are resolved.
Second, the generated images should follow the underlying distribution represented in the input data set. Third, the generated tuples should have high quality and look realistic to a human evaluator. Last but not least, given the (often monetary) cost associated with the queries to the foundation model, we should ensure the cost-effectiveness of the data set repair process.

\begin{wrapfigure}{L}{0.45\textwidth}
\centering
\vspace{-3mm}
\scriptsize
    \includegraphics[width=.45\textwidth]{submissions/submission1/shahbazi/enhanced_pipeline.png}
\vspace{-3mm}
\caption{\small Architecture of \fmsystem for image data augmentation for coverage enhancement.}\label{fig:chameleon}
% \vspace{-3mm}
\end{wrapfigure}

\noindent Figure~\ref{fig:chameleon} shows the architecture of our system \fmsystem \cite{chameleon} for coverage enhancement using DALL-E image generator.
To address the first challenge, we define the combinations-selection problem, which minimizes the total number of synthetic tuples for resolving lack of coverage of minorities at the most general level. We show the problem is {\sc NP}-hard, and propose a greedy approximation algorithm for it.
To address the second and third challenges, \fmsystem follows a {\it rejection sampling} strategy.
It views each tuple in the data set $\dee$ as an iid sample from the underlying distribution $\xi$ it represents. It uses the vector representations (embeddings) space to describe the distribution. Then, given a newly generated tuple, it employs the one-class support vector machine (OCSVM) approach proposed by Scholkopf et al.~\cite{scholkopf1999support} to reject the tuple if it does not follow $\xi$.
Moreover, it models the quality evaluation as hypothesis testing and rejects the samples that have a higher chance of being labeled as ``unrealistic'' by a random human evaluator.
Finally, to minimize the number of queries to the foundation model, we provide a guide tuple (and a mask), in addition to the prompt, to the foundation model. We model the guide-selection problem as {\it contextual multi-armed bandit} and propose a solution based on the contextual UCB for it.

Before concluding this section, let us provide some experiment results to demonstrate the effectiveness of data augmentation with \fmsystem. We use FERET DB \cite{phillips1998feret} for this experiment, which comprises 1199 individual images and serves as a standardized facial image database for researchers to develop algorithms and report results. All images in FERET DB share the same dimensions, pose, and facial expression.
First, we identified the (level-1) uncovered ethnicity groups, using the threshold 80. We then used \fmsystem and resolved the lack of coverage issues.
To evaluate the effectiveness of the system, we trained a CNN model to predict the race of each image within this dataset. We then retrained the identical CNN on the repaired training data. Importantly, our test dataset for both experiments remains consistent and is derived from real images.
Table~\ref{tab:lackofcoverage} presents the improvements in precision, recall, and F1 score metrics for under-represented groups after repairing the dataset. The results indicate an enhancement in performance metrics for all under-represented groups following the repair process.

\begin{table}[t]
    \centering
    \caption{Illustrating the effect of lack of coverage repair using \fmsystem on \texttt{FERTDB}}
    \label{tab:lackofcoverage}
    \vspace{-3mm}
    \begin{tabular}{lcccccccc}
        \toprule
         & \multicolumn{4}{c}{\textbf{Classifier Performance on \texttt{FERTDB}}} & \multicolumn{4}{c}{\textbf{Classifier Performance on Repaired}} \\
        \cmidrule(lr){2-5} \cmidrule(lr){6-9}
        \textbf{Ethnicity Groups}& \#Images & Precision & Recall & F1-Score & \#Images & Precision & Recall & F1-Score \\
        \midrule
        Overall          & 756 & 0.81 & 0.75 & 0.78 & 987 & 0.70 & 0.75 & 0.72 \\ \hline
        Black            & 40  & 0.19 & 0.22 & 0.16 & 100 & 0.48 & 0.56 & 0.52 \\
        Hispanic         & 19  & 0.50 & 0.17 & 0.25 & 100 & 0.62 & 0.36 & 0.45 \\
        Middle Eastern   & 10  & 0.00 & 0.00 & 0.00 & 100 & 0.20 & 0.41 & 0.27 \\
        \bottomrule
    \end{tabular}
\end{table}

 %%%%%%%%%%%%%%%%%%%%%%%%%%%%%%%% RELIABILITY  %%%%%%%%%%%%%%%%%%%%%%%%%%%%%%%%
\section{Generating Reliability Warnings}\label{sec:reliability}
% up to 2.5 pages
Interpretability is a necessity for data scientists who develop predictive models for critical decision-making.
In such settings, it is important to provide additional means to support the following question:
{\it is an individual prediction of the model reliable for decision-making?} Our goal is to use the lack of representation to help decision-makers find insights about this critical question.
To further motivate this, let us use the following example:

\vspace{1mm}
\begin{example}\label{ex-0}
{\bf(Part1):} Consider a judge who needs to decide whether to accept or deny a bail request. Using data-driven predictive models is prevalent in such cases for predicting recidivism~\cite{dressel2018accuracy}.
Indeed, such models can be beneficial to help the judge make wise decisions.
Suppose the model predicts the queried individual as high risk (or low risk).
The judge is aware and concerned about the critics surrounding such models.
A major question the judge faces is whether or not they should rely on the prediction outcome to take action for this case.
Furthermore, if, for instance, they decide to ignore the outcome and hence they need to provide a statement supporting their action, what evidence can they provide? 
\end{example}

In line with the recent trend on data-centric AI~\cite{ng2021mlops}, we design {novel approaches}, {complimentary} to the existing work on trustworthy AI~\cite{wing2021trustworthy,kentour2021analysis,liu2021trustworthy,singh2021trustworthy}, to address the aforementioned trust question through the lens of {\it data}.
In particular, unlike existing works that generate trust information from a {\it given \underline{model}}, we associate {\it \underline{data sets} with proper measurements} that specify their {\it the scope of use for predicting future cases}.
We note that a predictive model provides only probabilistic guarantees on the \underline{average} loss over the distribution represented by the data set used for training it.
As a result, these predictions may not be distribution generalizable~\cite{kulynych2022you}.
Consequently, if the query point is {\it not represented} by the data, the guarantees may not hold, hence one cannot rely on the prediction outcome.
Besides, an essential requirement for a learning algorithm is that its training data $\dee$ should represent the underlying distribution $\dist$.
Even if so, the trained model $h$ only provides a probabilistic guarantee on the {expected} loss on random samples from $\dist$.  
A model that performs well on {\it majority} of samples drawn from $\dist$ will have a high performance on average. Still, as we observed in Figure~\ref{fig:mlfails},
its performance for {\it minorities} and points that are not represented is questionable. Let us consider the following toy example:

\begin{figure*}[!b] 
    \begin{minipage}[t]{0.32\linewidth}
        	\centering
        	\includegraphics[width=\textwidth]{submissions/submission1/shahbazi/example_1.png} 
        	\vspace{-9mm}\caption{\small Data set $\dee$ generated using a Gaussian distribution; $x_1$ and $x_2$ are positively correlated}
            \label{fig:ex1:1}
    \end{minipage}
    \hfill
    \begin{minipage}[t]{0.32\linewidth}
        \centering
        	\includegraphics[width =\textwidth]{submissions/submission1/shahbazi/example_2.png} 
        	\vspace{-9mm}\caption{\small The decision boundary of learned model $h$ and query points $\qu^1$ to $\qu^4$}
            \label{fig:ex1:2}
    \end{minipage}
    \hfill
    \begin{minipage}[t]{0.32\linewidth}
        	\centering
        	\includegraphics[width =\textwidth]{submissions/submission1/shahbazi/example_3.png}
        	\vspace{-9mm}\caption{\small Ground-truth boundary, overlaid on the model decision boundary and query points}
            \label{fig:ex1:3}
    \end{minipage}
    \vspace{-5mm}
\end{figure*} 

\vspace{1mm}
\begin{example}\label{ex-1}
Consider a binary classification task where the input space is $\ex=\langle x_1, x_2\rangle$ and the output space is the binary label $y$ with values $\{-1$ (red) $,+1$ (blue)$\}$.
Suppose the underlying data distribution $\dist$ follows a 2D Gaussian, where $x_1$ and $x_2$ 
are positively correlated as shown in Figure~\ref{fig:ex1:1}.
The figure shows the data set $\dee$ drawn independently from the distribution $\dist$, along with their labels as their colors.
Using $\dee$, the prediction model $h$ is constructed as shown in Figure~\ref{fig:ex1:2}. 
The decision boundary is specified in the picture; while any point above the line is predicted as +1, a query point below it is labeled as -1.
The classifier has been evaluated using a test set that is an iid sample set drawn from the underlying data set $\dist$. The accuracy on the test set is high (above 90\%), and hence, the model gets deployed.
We cherry-picked four query points, $\qu^1$ to $\qu^4$, that are also included in Figure~\ref{fig:ex1:2}. Using $h$ for prediction, $h(\qu^1)=-1$, $h(\qu^2)=+1$,  $h(\qu^3)=+1$, and $h(\qu^4)=-1$.
Figure~\ref{fig:ex1:3} adds the ground-truth boundary to the search space, revealing the true label of the query points: every point inside the red circle has the true label $-1$ while any point outside of it is $+1$.
Looking at the figure, $y^1=+1$ while the model predicted it as $h(\qu^1)=-1$.  \hfill$\square$
\end{example}
\vspace{2mm}

Let us take a closer look at the four query points in this example and their placement with regard to the tuples in $\dee$ used for training $h$. 
$\qu^2$ belongs to a {\it dense region} with many training tuples in $\dee$ surrounding it. Besides, all of the tuples in its vicinity have the same label $y=+1$. As a result, one can expect that the model's outcome $h(\qu^2)=+1$ should be a reliable prediction.
Similar to $\qu^2$, $\qu^4$ also belongs to a dense region in $\dee$; however, $\qu^4$ belongs to an {\it uncertain region}, where some of the tuples in its vicinity have a label $y=+1$, and some others have the label $y=-1$. Considering the uncertainty in the vicinity of $\qu^4$, one cannot confidently rely on the outcome of the model $h$. 
On the other hand, the neighbors of $\qu^1$ (resp. $\qu^3$) are not uncertain, all having the label $y=-1$ (resp. $y=+1$).
However, the query points $\qu^1$ and $\qu^3$ are not well represented by $\dee$. In other words, $\qu^1$ and $\qu^3$ are unlikely to be generated according to the underlying distribution $\dist$, represented by $\dee$. As a result, following the no-free-lunch theorem~\cite{kakade2003sample}, one cannot expect the outcome of model $h$ to be reliable for these points.
Looking at the ground-truth boundary in Figure~\ref{fig:ex1:3}, $h$ luckily predicted the outcome for $\qu^3$ correctly, but it was not fortunate to predict the $y^1$ correctly.
Nevertheless, 
since the model is not reliably trained for these points, 
its outcome for these query points is not trustworthy.

From Example~\ref{ex-1}, we observe that the outcome of a model $h$, trained using a data set $\dee$ is not reliable for a query point $\qu$, if:
\begin{itemize}
    \item {\bf Lack of representation:} $\qu$ is not well-represented by $\dee$.
    In such cases, the model has not seen ``enough'' samples similar to $\qu$ to reliably learn and predict the outcome of $\qu$.
    \item {\bf Lack of certainty:} $\qu$ belongs to an uncertain region, where different tuples of $\dee$ in the vicinity of $\qu$ have different target values. $\qu$ belongs to a high-fluctuating area, where tuples in the vicinity of $\qu$ have a wide range of values.
\end{itemize} \vspace{2mm}

\noindent
Based on these two observations, we propose Representation-and-Uncertainty ({\bf RU}) measures.
To identify if a query suffers from uncertainty or lack of representation, one could use a deterministic approach using a fixed threshold. Then if the number of similar samples to (resp. label fluctuation in vicinity of) $\qu$ is larger than the threshold it is considered as unrepresented (resp. uncertain).
This approach, however, would be misleading since two numbers close to the threshold could be treated very differently. Also, all points on each side of the threshold would be considered equally represented (resp., certain). Instead, we consider {\it a randomized approach}, widely popular in the literature, including~\cite{dwork2012fairness}.
That is, instead of using fixed thresholds, a Bernoulli variable (a biased coin) is used that 
assigns $\qu$ as unrepresented (resp., uncertain) based on the number of samples similar to it (resp., its neighborhood uncertainty).
Given a query point $\qu$, let $\pe_o$ be the probability indicating if $\qu$ is not represented and let $\pe_u$ be the probability indicating if $\qu$ belongs to an uncertain region. 
We represent the probability of the Bernoulli variables for lack of representation or uncertainty components as $\pe_o$ and $\pe_u$, respectively. Note that the two Bernoulli variables $\pe_o$ and $\pe_u$ are independent from each other. That simply follows the argument that after specifying the number of similar samples to $\qu$ whether or not it should be considered as unrepresented does not depend on the uncertainty in the neighborhood of $\qu$.

\begin{definition}[\sru]\label{def:sdt}
The \sru is a probabilistic measure that considers the outcome of a model for a query point $\qu$ untrustworthy if $\qu$ is not represented by $\dee$ {\it and} it belongs to an uncertain region.
Formally, the \sru measure is:
\begin{align} 
    \nonumber
    SRU(\qu) &= \pe\big((\qu \mbox{ is outlier}) \wedge (\qu \mbox{ belongs to uncertain region})\big) 
\end{align}
Since $\pe_o$ and $\pe_u$ are independent:

\vspace{-13mm}
\begin{align} \label{eq:strong}
    SRU(\qu) &= \pe_o(\qu) \times \pe_u(\qu)
\end{align}
\end{definition}

\sru raises the warning signal only when the query point fails on {\it both} conditions of being represented by $\dee$ and not belonging to an uncertain region. 
For instance, in Example~\ref{ex-1} none of the query points fail both on representation and on uncertainty; hence neither has a high \sru score.
On the other hand, 
a high \sru score for a query point $\qu$ {\it provides a strong warning signal} that one should perhaps reject the model outcome and not consider it for decision-making.

\sru is a strong signal that raises warnings only for the fearfully concerning cases that fail both on representation and uncertainty.
However, as observed in Example~\ref{ex-1} a query points failing {\it at least} one of these conditions may also not be reliable, at least for critical decision making.
We define the \wru measure to raise a warning for such cases.

\begin{definition}[\wru]\label{def:wdt}
The \wru measure is a probabilistic measure that considers the outcome of a model for a query point $\qu$ untrustworthy if $\qu$ is not represented by $\dee$ {\bf or} it belongs to an uncertain region.
Formally, the \wru is computed as:
\begin{align} \label{eq:weak}
    WRU(\qu) = \pe\big((\qu \mbox{ is outlier}) \vee (\qu \mbox{ belongs to uncertain region})\big) 
    = \pe_o(\qu) + \pe_u(\qu) - \pe_o(\qu) \times \pe_u(\qu)
\end{align}
\end{definition}

Proposing quantitative probabilistic outcomes, \ru measures are interpretable for the users, since beyond the scores, the uncertainty and lack of representation components provide an explanation to justify them. 
Please refer to \cite{techrep} for more details on how to efficiently and effectively compute the representation ($\pe_o$) and uncertainty ($\pe_u$) probabilities, using only $\dee$.
In Example~\ref{ex-0}, let us see how the \ru measures can be helpful.

\noindent{\bf Example 1. (part 2):}
{\it RU measures \underline{raise warning} when
the fitness of the data set used for drawing a prediction is questionable, helping the judge to be cautious when taking action.
Besides, these measures provide \underline{quantitative evidence} to support the judge's action when they decide to ignore a prediction outcome that is not trustworthy.
The judge, for example, can argue to ignore a model outcome for a specific case, based on the insight that 
the model has been built using a
data set that fails to represent the given case.}
\hfill$\square$

Finally, let us demonstrate the efficacy of \ru measures through a series of experiments. Since the \ru measures are {\it data-centric},
those are applicable for both classification and regression tasks, irrespective of the model used.
We use {\it Adult} dataset~\cite{adult} for classification and {\it House Sales in King County} dataset for the validation of regression tasks. From each dataset, we uniformly sample two sets from the underlying distribution. The first set serves as the training set to compute the \ru values, and the second one is used as the test set from which the queries are drawn. We validate our proposal by providing the correlation between the \ru values and the performance of an ML model's prediction on the same data. 

We start by computing the \ru values for all the query points in the test set. Next, we bucketize the query points based on their \ru values in equi-width buckets of width 0.1. We repeat this for both \sru and \wru measures. Next, we train a model on the training data set and predict the target variable for the points in each range of \ru measure. The validation results for the classification task on the {\it Adult} dataset are presented in Figures \ref{fig:exp-adult-sdt} and \ref{fig:exp-adult-wdt}. Each figure corresponds to the accuracy/error measures of the classifier over each bucket of \ru values for \sru and \wru. As the \ru values increase, the accuracy of the model drops while the FPR rises, and therefore, the model fails to capture the ground truth for the points that fall into untrustworthy regions in the data set. By repeating the aforementioned steps for the regression task on the {\it House Sales in King County} dataset, we observe similar results presented in Figures \ref{fig:exp-hs-sdt} and \ref{fig:exp-hs-wdt}. 
As the \ru value increases, the RSS of the regression model follows the same trend denoting that the model fails to perform for tuples with a high \ru value.

\begin{figure}[!tb]
    \begin{minipage}[t]{0.24\linewidth}
        \centering
        \includegraphics[width=\textwidth]{submissions/submission1/shahbazi/sdt_adult.pdf}
        \vspace{-6mm}\caption{\small{\it Adult}, efficacy of \sru  on classification}
        \label{fig:exp-adult-sdt}
    \end{minipage}\hfill
    \begin{minipage}[t]{0.24\linewidth}
        \centering
        \includegraphics[width=\textwidth]{submissions/submission1/shahbazi/wdt_adult.pdf}
        \vspace{-6mm}\caption{\small{\it Adult}, efficacy of \wru  on classification}
        \label{fig:exp-adult-wdt}
    \end{minipage}\hfill
    \begin{minipage}[t]{0.24\linewidth}
        \centering
        \includegraphics[width=\textwidth]{submissions/submission1/shahbazi/sdt_regression_house.pdf}
        \vspace{-6mm}\caption{\small{\it House Sales in King County}, efficacy of \sru on regression}
        \label{fig:exp-hs-sdt}
    \end{minipage}\hfill
    \begin{minipage}[t]{0.24\linewidth}
        \centering
        \includegraphics[width=\textwidth]{submissions/submission1/shahbazi/wdt_regression_house.pdf}
        \vspace{-6mm}\caption{\small{\it House Sales in King County}, efficacy \wru on regression}
        \label{fig:exp-hs-wdt}
    \end{minipage}
\vspace{-5mm}
\end{figure}
 %%%%%%%%%%%%%%%%%%%%%%%%%%%%%%%% RELATED WORK  %%%%%%%%%%%%%%%%%%%%%%%%%%%%%%%%
\section{Related Work}\label{related} 

Bias in data has been looked at for a long time in statistical community~\cite{neyman1936contributions} but social data presents different challenges~\cite{olteanu2019social,fairmlbook,barocas2016big,jk2019bias,drosou2017diversity}.
The diversity and representativeness of data have been widely studied~\cite{drosou2017diversity}, in fields such as social science~\cite{berrey2015enigma, dobbin2016diversity,simpson1949measurement}, political science~\cite{surowiecki2005wisdom}, and information retrieval~\cite{agrawal2009diversifying}. 
Tracing back machine bias to its source, there have been major efforts to identify different types~\cite{mehrabi2021survey, olteanu2019social,friedman1996bias} and sources~\cite{torralba2011unbiased,crawford2013hidden,diakopoulos2015algorithmic} of biases in data. Efforts to satisfy {\it responsible data} requirements~\cite{nargesian2022responsible} extend to various stages of the data analysis pipeline, including data annotation~\cite{li2020towards,lazier2023fairness}, data cleaning and repair~\cite{SalimiRHS19,tae2019data,salimi2020database}, data imputation~\cite{martinez2019fairness}, entity resolution~\cite{shahbazi2023through,fanourakis2023fairer}, data integration~\cite{nargesian2022responsible,nargesian2021tailoring}, etc. 

\paragraph{Data Coverage:}The notion of data coverage has received extensive attention from different angles. Detecting lack of coverage has been studied for datasets with discrete~\cite{asudeh2019assessing} and continuous~\cite{asudeh2021coverage} attributes populated in single or multiple \cite{lin2020identifying} relations.
To resolve insufficient coverage, \cite{accinelli2020coverage, accinelli2021impact,shetiya2022fairness}
consider resolving representation bias in preprocessing pipelines by rewriting queries into the closest operation so that certain subgroups are sufficiently represented in the downstream tasks. Alternatively, ~\cite{asudeh2019assessing,tae2021slice} propose a data collection strategy to acquire as little additional data as possible (to minimize the associated costs) to meet the representation constraints. ~\cite{sharma2020data,iosifidis2018dealing,celis2020data} opt for a data augmentation approach by adding partially altered duplicates of already existing tuples or generating new synthetic entries from existing data. Consequently, the new data set has an equal number of elements for different groups, resulting in potentially resolving the under-representation issues. Finally,  \cite{nargesian2021tailoring} utilizes data integration techniques to consolidate data from different sources into a single dataset to resolve representation bias.
Related works also include ~\cite{chung2019slice,sagadeeva2021sliceline,tae2021slice} that seek to understand if the overall performance of the model fails to reflect and performs poorly on certain slices in the data.
As alternative approaches to measure representation bias, the notion of representation rate~\cite{celis2020data} (a.k.a. equal base rate~\cite{kleinberg2016inherent}) is introduced which compared with coverage, it is more restrictive as it requires almost equal ratios from different groups.
Please refer to \cite{shahbazi2023representation} for a comprehensive survey about representation bias in data. 

\paragraph{ML Reliability:} Model-centric works for uncertainty quantification such as 
probabilistic classifiers~\cite{zadrozny2001obtaining,zadrozny2002transforming,platt1999probabilistic,niculescu2005predicting},
prediction intervals (PIs) \cite{chatfield93predictionintervals,pearce2018high,khosravi2010lower} and conformal predictions (CP)~\cite{angelopoulos2021gentle,shafer2008tutorial} that are used for measuring prediction uncertainty, are built
by maximizing the {\it expected performance} on {\it random} sample from the underlying distribution.
As a result, while providing accurate estimations for the dense regions of data (e.g. majority groups), their estimation accuracy is questionable for the poorly represented regions.
In particular, \cite{angelopoulos2021gentle} recognizes the lack of guarantees in the performance of CP for such regions.
Besides, the bulk of work on trustworthy AI provides information that {\it supports} the outcome of an ML model. For example, existing work on explainable AI, including~\cite{harradon2018causal,ribeiro2016should,gunning2019darpa}, aims to find simple explanations and rules that justify the outcome of a model.
Conversely, we aim to {\it raise warning signals} when the outcome of a model is {\it not} trustworthy. That is, to provide reasons that {\it cast doubt} on the reliability of the model outcome {for a given query point}.

 %%%%%%%%%%%%%%%%%%%%%%%%%%%%%%%% FUTURE  %%%%%%%%%%%%%%%%%%%%%%%%%%%%%%%%
% \vspace{-3mm}
\section{Final Remarks}\label{sec:conclusion}
As Data-centric AI and Responsible AI emerge as focal points in data science research, the development of Data-centric methodologies for ensuring Responsible and Trustworthy AI attracts increasing attention.
While there is some excellent work on responsible data management to achieve this goal, there remain many challenges yet to be addressed.

In this paper, we focused on a crucial aspect of responsible data -- detecting and addressing the under-representation of minorities within a data set.
We formally defined the notion of data coverage and discussed various techniques for (a) identifying lack of representation issues across different data modalities, (b) ensuring proper representation of minorities in data, and (c) limiting the scope-of-use of data sets based on their representation issues by generating proper ({\sc RU}) warning signals.
Even though the research on detecting lack of coverage issues is relatively mature, resolution techniques are still understudied.
Considering the recent advancements in Generative AI, utilizing Foundation Models and Large Language Models, and studying their limitations, for data augmentation to improve the representation of minorities at the data level seems interesting to further explore.

 %%%%%%%%%%%%%%%%%%%%%%%%%%%%%%%% BIB  %%%%%%%%%%%%%%%%%%%%%%%%%%%%%%%%
\bibliographystyle{unsrt}
\small
% \bibliography{ref}
\begin{thebibliography}{10}

\bibitem{asudeh2019assessing}
A.~Asudeh, Z.~Jin, and H.~Jagadish.
\newblock Assessing and remedying coverage for a given dataset.
\newblock In {\em ICDE}, pages 554--565. IEEE, 2019.

\bibitem{shahbazi2023representation}
N.~Shahbazi, Y.~Lin, A.~Asudeh, and H.~Jagadish.
\newblock Representation bias in data: A survey on identification and resolution techniques.
\newblock {\em ACM Computing Surveys}, 2023.

\bibitem{asudeh2021coverage}
A.~Asudeh, N.~Shahbazi, Z.~Jin, and H.~V. Jagadish.
\newblock Identifying insufficient data coverage for ordinal continuous-valued attributes.
\newblock In {\em SIGMOD}. ACM, 2021.

\bibitem{mousavi2024data}
M.~Mousavi, N.~Shahbazi, and A.~Asudeh.
\newblock Data coverage for detecting representation bias in image datasets: {A} crowdsourcing approach.
\newblock In {\em {EDBT}}, pages 47--60, 2024.

\bibitem{nargesian2021tailoring}
F.~Nargesian, A.~Asudeh, and H.~Jagadish.
\newblock Tailoring data source distributions for fairness-aware data integration.
\newblock {\em Proceedings of the VLDB Endowment}, 14(11):2519--2532, 2021.

\bibitem{nargesian2022responsible}
F.~Nargesian, A.~Asudeh, and H.~V. Jagadish.
\newblock Responsible data integration: Next-generation challenges.
\newblock {\em SIGMOD}, 2022.

\bibitem{sharma2020data}
S.~Sharma, Y.~Zhang, J.~M. R{\'\i}os~Aliaga, D.~Bouneffouf, V.~Muthusamy, and K.~R. Varshney.
\newblock Data augmentation for discrimination prevention and bias disambiguation.
\newblock In {\em AIES}, pages 358--364, 2020.

\bibitem{DBLP:journals/jair/ChawlaBHK02}
N.~V. Chawla, K.~W. Bowyer, L.~O. Hall, and W.~P. Kegelmeyer.
\newblock {SMOTE:} synthetic minority over-sampling technique.
\newblock {\em J. Artif. Intell. Res.}, 16:321--357, 2002.

\bibitem{iosifidis2018dealing}
V.~Iosifidis and E.~Ntoutsi.
\newblock Dealing with bias via data augmentation in supervised learning scenarios.
\newblock {\em Jo Bates Paul D. Clough Robert J{\"a}schke}, 24, 2018.

\bibitem{celis2020data}
L.~E. Celis, V.~Keswani, and N.~Vishnoi.
\newblock Data preprocessing to mitigate bias: A maximum entropy based approach.
\newblock In {\em ICML}, pages 1349--1359. PMLR, 2020.

\bibitem{asudeh2022towards}
A.~Asudeh and F.~Nargesian.
\newblock Towards distribution-aware query answering in data markets.
\newblock {\em Proceedings of the VLDB Endowment}, 15(11):3137--3144, 2022.

\bibitem{motwani1995randomized}
R.~Motwani and P.~Raghavan.
\newblock {\em Randomized algorithms}.
\newblock Cambridge university press, 1995.

\bibitem{chameleon}
M.~Erfanian, H.~V. Jagadish, and A.~Asudeh.
\newblock Chameleon: Foundation models for fairness-aware multi-modal data augmentation to enhance coverage of minorities.
\newblock {\em arXiv preprint arXiv:2402.01071}, 2024.

\bibitem{scholkopf1999support}
B.~Sch{\"o}lkopf, R.~C. Williamson, A.~Smola, J.~Shawe-Taylor, and J.~Platt.
\newblock Support vector method for novelty detection.
\newblock {\em NeurIPS}, 12, 1999.

\bibitem{phillips1998feret}
P.~J. Phillips, H.~Wechsler, J.~Huang, and P.~J. Rauss.
\newblock The feret database and evaluation procedure for face-recognition algorithms.
\newblock {\em Image and vision computing}, 16(5):295--306, 1998.

\bibitem{dressel2018accuracy}
J.~Dressel and H.~Farid.
\newblock The accuracy, fairness, and limits of predicting recidivism.
\newblock {\em Science advances}, 4(1):eaao5580, 2018.

\bibitem{ng2021mlops}
A.~Ng.
\newblock Mlops: From model-centric to data-centric {AI}.
\newblock 2021.

\bibitem{wing2021trustworthy}
J.~M. Wing.
\newblock Trustworthy {AI}.
\newblock {\em CACM}, 64(10):64--71, 2021.

\bibitem{kentour2021analysis}
M.~Kentour and J.~Lu.
\newblock Analysis of trustworthiness in machine learning and deep learning.
\newblock {\em InfoComp}, 2021.

\bibitem{liu2021trustworthy}
H.~Liu, Y.~Wang, W.~Fan, X.~Liu, Y.~Li, S.~Jain, A.~K. Jain, and J.~Tang.
\newblock Trustworthy {AI}: A computational perspective.
\newblock {\em arXiv preprint arXiv:2107.06641}, 2021.

\bibitem{singh2021trustworthy}
R.~Singh, M.~Vatsa, and N.~Ratha.
\newblock Trustworthy {AI}.
\newblock In {\em 8th ACM IKDD CODS and 26th COMAD}, pages 449--453. 2021.

\bibitem{kulynych2022you}
B.~Kulynych, Y.-Y. Yang, Y.~Yu, J.~B{\l}asiok, and P.~Nakkiran.
\newblock What you see is what you get: Distributional generalization for algorithm design in deep learning.
\newblock {\em arXiv preprint arXiv:2204.03230}, 2022.

\bibitem{kakade2003sample}
S.~M. Kakade.
\newblock {\em On the sample complexity of reinforcement learning}.
\newblock University of London, University College London (United Kingdom), 2003.

\bibitem{dwork2012fairness}
C.~Dwork, M.~Hardt, T.~Pitassi, O.~Reingold, and R.~Zemel.
\newblock Fairness through awareness.
\newblock In {\em ITCS}, pages 214--226, 2012.

\bibitem{techrep}
N.~Shahbazi and A.~Asudeh.
\newblock Data-centric reliability evaluation of individual predictions.
\newblock {\em CoRR, abs/2204.07682}, 2022.

\bibitem{adult}
M.~Lichman.
\newblock Adult income dataset, {UCI} machine learning repository.
\newblock \url{https://archive.ics.uci.edu/ml/datasets/adult}, 2013.

\bibitem{neyman1936contributions}
J.~Neyman and E.~S. Pearson.
\newblock Contributions to the theory of testing statistical hypotheses.
\newblock {\em Statistical Research Memoirs}, 1936.

\bibitem{olteanu2019social}
A.~Olteanu, C.~Castillo, F.~Diaz, and E.~Kiciman.
\newblock Social data: Biases, methodological pitfalls, and ethical boundaries.
\newblock {\em Frontiers in Big Data}, 2:13, 2019.

\bibitem{fairmlbook}
S.~Barocas, M.~Hardt, and A.~Narayanan.
\newblock Fairness and machine learning: Limitations and opportunities.
\newblock \url{fairmlbook.org}, 2019.

\bibitem{barocas2016big}
S.~Barocas and A.~D. Selbst.
\newblock Big data's disparate impact.
\newblock {\em Calif. L. Rev.}, 104:671, 2016.

\bibitem{jk2019bias}
J.~Kleinberg.
\newblock Fairness, rankings, and behavioral biases.
\newblock FAT*, 2019.

\bibitem{drosou2017diversity}
M.~Drosou, H.~Jagadish, E.~Pitoura, and J.~Stoyanovich.
\newblock Diversity in big data: A review.
\newblock {\em Big data}, 5(2):73--84, 2017.

\bibitem{berrey2015enigma}
E.~Berrey.
\newblock {\em The enigma of diversity: The language of race and the limits of racial justice}.
\newblock University of Chicago Press, 2015.

\bibitem{dobbin2016diversity}
F.~Dobbin and A.~Kalev.
\newblock Why diversity programs fail and what works better.
\newblock {\em Harvard Business Review}, 94(7-8):52--60, 2016.

\bibitem{simpson1949measurement}
E.~H. Simpson.
\newblock Measurement of diversity.
\newblock {\em Nature}, 163(4148), 1949.

\bibitem{surowiecki2005wisdom}
J.~Surowiecki.
\newblock {\em The wisdom of crowds}.
\newblock Anchor, 2005.

\bibitem{agrawal2009diversifying}
R.~Agrawal, S.~Gollapudi, A.~Halverson, and S.~Ieong.
\newblock Diversifying search results.
\newblock In {\em WSDM}, pages 5--14. ACM, 2009.

\bibitem{mehrabi2021survey}
N.~Mehrabi, F.~Morstatter, N.~Saxena, K.~Lerman, and A.~Galstyan.
\newblock A survey on bias and fairness in machine learning.
\newblock {\em ACM Computing Surveys (CSUR)}, 54(6):1--35, 2021.

\bibitem{friedman1996bias}
B.~Friedman and H.~Nissenbaum.
\newblock Bias in computer systems.
\newblock {\em TOIS}, 14(3):330--347, 1996.

\bibitem{torralba2011unbiased}
A.~Torralba and A.~A. Efros.
\newblock Unbiased look at dataset bias.
\newblock In {\em CVPR 2011}, pages 1521--1528. IEEE, 2011.

\bibitem{crawford2013hidden}
K.~Crawford.
\newblock The hidden biases in big data.
\newblock {\em Harvard business review}, 1(4), 2013.

\bibitem{diakopoulos2015algorithmic}
N.~Diakopoulos.
\newblock Algorithmic accountability: Journalistic investigation of computational power structures.
\newblock {\em Digital journalism}, 3(3):398--415, 2015.

\bibitem{li2020towards}
Y.~Li, H.~Sun, and W.~H. Wang.
\newblock Towards fair truth discovery from biased crowdsourced answers.
\newblock In {\em SIGKDD}, pages 599--607, 2020.

\bibitem{lazier2023fairness}
S.~Lazier, S.~Thirumuruganathan, and H.~Anahideh.
\newblock Fairness and bias in truth discovery algorithms: An experimental analysis.
\newblock {\em arXiv preprint arXiv:2304.12573}, 2023.

\bibitem{SalimiRHS19}
B.~Salimi, L.~Rodriguez, B.~Howe, and D.~Suciu.
\newblock Interventional fairness: Causal database repair for algorithmic fairness.
\newblock In {\em {SIGMOD}}, pages 793--810. {ACM}, 2019.

\bibitem{tae2019data}
K.~H. Tae, Y.~Roh, Y.~H. Oh, H.~Kim, and S.~E. Whang.
\newblock Data cleaning for accurate, fair, and robust models: A big data-{AI} integration approach.
\newblock In {\em DEEM workshop}, pages 1--4, 2019.

\bibitem{salimi2020database}
B.~Salimi, B.~Howe, and D.~Suciu.
\newblock Database repair meets algorithmic fairness.
\newblock {\em ACM SIGMOD Record}, 49(1):34--41, 2020.

\bibitem{martinez2019fairness}
F.~Mart{\'\i}nez-Plumed, C.~Ferri, D.~Nieves, and J.~Hern{\'a}ndez-Orallo.
\newblock Fairness and missing values.
\newblock {\em arXiv preprint arXiv:1905.12728}, 2019.

\bibitem{shahbazi2023through}
N.~Shahbazi, N.~Danevski, F.~Nargesian, A.~Asudeh, and D.~Srivastava.
\newblock Through the fairness lens: Experimental analysis and evaluation of entity matching.
\newblock {\em Proceedings of the VLDB Endowment}, 16(11):3279--3292, 2023.

\bibitem{fanourakis2023fairer}
N.~Fanourakis, C.~Kontousias, V.~Efthymiou, V.~Christophides, and D.~Plexousakis.
\newblock Fairer demo: Fairness-aware and explainable entity resolution.
\newblock 2023.

\bibitem{lin2020identifying}
Y.~Lin, Y.~Guan, A.~Asudeh, and H.~Jagadish.
\newblock Identifying insufficient data coverage in databases with multiple relations.
\newblock {\em Proceedings of the VLDB Endowment}, 13(12):2229--2242, 2020.

\bibitem{accinelli2020coverage}
C.~Accinelli, S.~Minisi, and B.~Catania.
\newblock Coverage-based rewriting for data preparation.
\newblock In {\em EDBT Workshops}, 2020.

\bibitem{accinelli2021impact}
C.~Accinelli, B.~Catania, G.~Guerrini, and S.~Minisi.
\newblock The impact of rewriting on coverage constraint satisfaction.
\newblock In {\em EDBT Workshops}, 2021.

\bibitem{shetiya2022fairness}
S.~Shetiya, I.~P. Swift, A.~Asudeh, and G.~Das.
\newblock Fairness-aware range queries for selecting unbiased data.
\newblock In {\em ICDE}. IEEE, 2022.

\bibitem{tae2021slice}
K.~H. Tae and S.~E. Whang.
\newblock Slice tuner: A selective data acquisition framework for accurate and fair machine learning models.
\newblock In {\em SIGMOD}, pages 1771--1783, 2021.

\bibitem{chung2019slice}
Y.~Chung, T.~Kraska, N.~Polyzotis, K.~H. Tae, and S.~E. Whang.
\newblock Slice finder: Automated data slicing for model validation.
\newblock In {\em ICDE}, pages 1550--1553. IEEE, 2019.

\bibitem{sagadeeva2021sliceline}
S.~Sagadeeva and M.~Boehm.
\newblock Sliceline: Fast, linear-algebra-based slice finding for ml model debugging.
\newblock In {\em SIGMOD}, pages 2290--2299, 2021.

\bibitem{kleinberg2016inherent}
J.~Kleinberg, S.~Mullainathan, and M.~Raghavan.
\newblock Inherent trade-offs in the fair determination of risk scores.
\newblock {\em arXiv preprint arXiv:1609.05807}, 2016.

\bibitem{zadrozny2001obtaining}
B.~Zadrozny and C.~Elkan.
\newblock Obtaining calibrated probability estimates from decision trees and naive bayesian classifiers.
\newblock In {\em ICML}, volume~1, pages 609--616. Citeseer, 2001.

\bibitem{zadrozny2002transforming}
B.~Zadrozny and C.~Elkan.
\newblock Transforming classifier scores into accurate multiclass probability estimates.
\newblock In {\em SIGKDD}, pages 694--699, 2002.

\bibitem{platt1999probabilistic}
J.~Platt et~al.
\newblock Probabilistic outputs for support vector machines and comparisons to regularized likelihood methods.
\newblock {\em Advances in large margin classifiers}, 10(3):61--74, 1999.

\bibitem{niculescu2005predicting}
A.~Niculescu-Mizil and R.~Caruana.
\newblock Predicting good probabilities with supervised learning.
\newblock In {\em Proceedings of the 22nd international conference on Machine learning}, pages 625--632, 2005.

\bibitem{chatfield93predictionintervals}
C.~Chatfield.
\newblock Prediction intervals.
\newblock {\em Journal of Business and Economic Statistics}, 11:121--135, 1993.

\bibitem{pearce2018high}
T.~Pearce, A.~Brintrup, M.~Zaki, and A.~Neely.
\newblock High-quality prediction intervals for deep learning: A distribution-free, ensembled approach.
\newblock In {\em International conference on machine learning}, pages 4075--4084. PMLR, 2018.

\bibitem{khosravi2010lower}
A.~Khosravi, S.~Nahavandi, D.~Creighton, and A.~F. Atiya.
\newblock Lower upper bound estimation method for construction of neural network-based prediction intervals.
\newblock {\em IEEE transactions on neural networks}, 22(3):337--346, 2010.

\bibitem{angelopoulos2021gentle}
A.~N. Angelopoulos and S.~Bates.
\newblock A gentle introduction to conformal prediction and distribution-free uncertainty quantification.
\newblock {\em arXiv preprint arXiv:2107.07511}, 2021.

\bibitem{shafer2008tutorial}
G.~Shafer and V.~Vovk.
\newblock A tutorial on conformal prediction.
\newblock {\em Journal of Machine Learning Research}, 9(3), 2008.

\bibitem{harradon2018causal}
M.~Harradon, J.~Druce, and B.~Ruttenberg.
\newblock Causal learning and explanation of deep neural networks via autoencoded activations.
\newblock {\em arXiv preprint arXiv:1802.00541}, 2018.

\bibitem{ribeiro2016should}
M.~T. Ribeiro, S.~Singh, and C.~Guestrin.
\newblock " why should i trust you?" explaining the predictions of any classifier.
\newblock In {\em SIGKDD}, pages 1135--1144, 2016.

\bibitem{gunning2019darpa}
D.~Gunning and D.~Aha.
\newblock Darpa’s explainable artificial intelligence ({XAI}) program.
\newblock {\em AI Magazine}, 40(2):44--58, 2019.

\end{thebibliography}

\end{document}

\end{article}


\begin{article}
{Platform Design for Crowdsourcing and Future of Work}
{David Gross-Amblard, Atsuyuki Morishima, Saravanan Thirumuruganathan,Marion Tommasi, Ko Yoshida}
\graphicspath{{submissions/atsuyuki/}}
% link to instruction: https://tc.computer.org/tcde/tcde-bulletin-author-instructions/
% \documentclass[11pt,dvipdfm]{article}
\documentclass[11pt]{article}
\usepackage{tabularx}
\usepackage{ragged2e}  % for '\RaggedRight' macro (allows hyphenation)
\usepackage{booktabs}  % for \toprule, \midrule, and \bottomrule macros
\usepackage{textcomp}
\usepackage{amsfonts,amsmath}
\usepackage{deauthor,times}
\usepackage{graphicx} % 
\usepackage{hyperref}
\usepackage{comment}
\graphicspath{{asudeh/}}
\usepackage{soul}
\usepackage{subcaption}
\usepackage{ulem}
\usepackage{wrapfig}
\usepackage{color}
\usepackage{xspace}
\newtheorem{problem}{Problem}

%\DeclareMathOperator*{\argmax}{arg\,max}

%remove the following commands/package b4 submission
\newcommand{\hide}[1]{}
\newcommand{\eat}[1]{}
\newcommand{\resolved}[1]{\hide{#1}}
\newcommand{\abol}[1]{\textcolor{red}{Abol: #1}}
\newcommand{\mahdi}[1]{\textcolor{red}{Mahdi: #1}}
\newcommand{\nima}[1]{\textcolor{red}{Nima: #1}}

\newcommand{\dee}{\mathcal{D}}
\newcommand{\Gee}{\mathcal{G}}
\newcommand{\gee}{\mathbf{g}}
\newcommand{\ee}{\mathbf{e}}
\newcommand{\es}{\mathcal{S}}
\newcommand{\el}{\mathcal{L}}
\newcommand{\xx}{\mathcal{x}}
\newcommand{\dist}{\xi}
\newcommand{\alg}{\mathsf{A}}
\newcommand{\qu}{\mathbf{q}}
\newcommand{\ex}{\mathbf{x}}
\newcommand{\ti}{\mathbf{t}}
\newcommand{\sdt}{\mathsf{SDT}}
\newcommand{\wdt}{\mathsf{WDT}}
\newcommand{\Qu}{\mathbf{Q}}
\newcommand{\pe}{\mathbb{P}}
\newcommand{\megam}{\mathcal{M}}
\newcommand{\eps}{\varepsilon}
\newcommand{\enet}{{$\varepsilon$-{\bf net}}\xspace}
\newcommand{\net}{{\tt net}\xspace}
\newcommand{\vcd}{VC-dimension\xspace}
\newcommand{\at}[1]{{\tt \small #1}\xspace}
\newcommand{\pr}{Pr}

\newcommand{\sharpP}{\mbox{\#P}}
\newcommand{\NP}{\mathsf{NP}}
\newcommand{\LP}{\mathsf{LP}}
\newcommand{\IP}{\mathsf{IP}}
\newcommand{\ru}{{\sc {RU}}\xspace}
\newcommand{\sru}{{\sc {strongRU}}\xspace}
\newcommand{\wru}{{\sc {weakRU}}\xspace}

\newcommand{\fmsystem}{{\sc Chameleon}\xspace}
\newcommand{\fm}{$\mathcal{F}$\xspace}

\newtheorem{experiment}{Experiment}

\begin{document}

\title{Coverage-based Data-centric Approaches for \\Responsible and Trustworthy AI\thanks{This research was supported by the National Science Foundation under grant No. 2107290.}}

\author{
\begin{tabular}[t]{c@{\extracolsep{2.4em}}c@{\extracolsep{2.4em}}c@{\extracolsep{2.3em}}c} 
Nima Shahbazi & Mahdi Erfanian & Abolfazl Asudeh \\ 
University of Illinois Chicago & University of Illinois Chicago & University of Illinois Chicago\\
 nshahb3@uic.edu & merfan2@uic.edu & asudeh@uic.edu
\end{tabular}
}

\maketitle


\begin{abstract}
The grand goal of data-driven decision systems is to help make decisions easier, more accurate, at a higher scale, and also just. However, data-driven algorithms are only as good as the data they work with. Yet, data sets, especially those with social data, often do not represent minorities. The paucity of training data is a perpetual problem for AI, and the outcome of ML models for cases not represented in their training data is often not reliable. 
Hence, without properly addressing the lack of representation issues in data, we cannot expect AI-based societal solutions to have responsible and trustworthy outcomes. 

This paper focuses on data coverage as a data-centric approach for identifying and resolving misrepresentation of minorities in data.
To achieve this goal, we propose novel algorithms that (a) {\it identify} and {\it resolve} insufficient data coverage across data with different modalities and (b) use lack of representation information to generate data-centric {\it reliability warnings}.
 \end{abstract}
 
 %%%%%%%%%%%%%%%%%%%%%%%%%%%%%%%% INTRO  %%%%%%%%%%%%%%%%%%%%%%%%%%%%%%%%
\section{Introduction}\label{sec:intro} % Abstract+Intro: up to 2.5 pages 
Data-driven decision-making has shaped every corner of human life, spanning from autonomous vehicles to healthcare and even predictive policing and criminal justice. A pivotal concern, especially in applications that affect individuals, revolves around the reliability of the decisions rendered by the system.
It is easy to see that the accuracy of a data-driven decision depends, first and foremost, on the data used to make it. Essentially, the system learns the phenomena that data represent. While we may desire that the data should represent the underlying data distribution from which the production data is drawn, this alone may be insufficient, as it merely enables the model to perform well for the average case.
As a result, a model with a high accuracy could fail for specific regions in the data with insufficient representation. These regions may matter because they frequently represent some minority population in society. They could also represent cases that may not happen very often but have a relevant impact on the correctness of a critical decision.
In short, if the data fails to sufficiently represent a specific population, the outcome of the decision system for that population may not be trustworthy.

The phenomenon known as \textit{Representation Bias} can arise from how the data was originally collected, or it could be the result of biases introduced post-collection—whether historically, cognitively, or statistically.

Representation bias is essentially inevitable without a systematic approach to data collection. 
For example, in the context of survey data collection, vital steps involve identifying all populations within the underlying distribution based on desired demographic information and ensuring comprehensive coverage with sufficient samples from each group. 
Even then, only an (uncontrolled) subset of the invitees will opt-in to respond to the survey.
Another challenge lies in the fact that data scientists often lack control over the data collection process, leading to the reliance on ``found data'' in the majority of data-driven systems. Therefore, with no guarantee on the aforementioned steps in the data collection process, the found data is most likely a biased sample.
Acknowledging the potential harms of representation bias, the notion of \textit{Data Coverage}~\cite{asudeh2019assessing,shahbazi2023representation} has been proposed to ensure the adequate representation of minority groups in data sets employed for decision-making and developing sophisticated data science tools. 

Addressing representation issues in data poses various challenges depending on the modality of the data. In this paper, we focus on identifying and resolving lack of coverage issues in data with different modalities.
We start by proposing a variety of techniques (spanning from geometric and combinatorial optimization to crowd-souring) aimed at efficiently detecting insufficient coverage on structured data sets with non-ordinal categorical and continuous attributes, as well as image data sets. Next, we propose a range of approaches grounded in data integration and generative data augmentation to address the lack of coverage by enriching the data sets with more data. However, with limited control over the data collection processes, it could be difficult and expensive to resolve all misrepresentations. 
Since adding more data is not always possible, we proceed to introduce data-centric preventive solutions that warn the user about the reliability of their predictions regarding representation bias issues. These warnings assist users in determining whether they trust the outcomes of the models or exercise caution. 

 %%%%%%%%%%%%%%%%%%%%%%%%%%%%%%%% IDENTIFICATION  %%%%%%%%%%%%%%%%%%%%%%%%%%%%%%%%
\section{Detecting Insufficient Representation of Minorities}\label{sec:identification} %up to 3.5 pages
Representation bias happens when the development (training data) population under-represents 
and subsequently fails to generalize well 
for some parts of the target population, due to historical bias, sampling bias, etc.
The notion of {\it data coverage} has been studied across different settings in \cite{shahbazi2023representation} as a metric to measure representation bias. At a high level, coverage is referred to as having enough similar entries for each object in a data set. 
For a better understanding, let us go over the definition of the generalized notion of coverage:

\begin{definition}[Data Coverage]\label{def:coverage}
Consider a data set $\dee$ with $n$ tuples, each consisting of $d$ attributes of interest $\mathbf{x}=\{x_1, x_2, \cdots,x_d\}$, such as {\tt gender}, {\tt race}, {\tt salary}, {\tt age}, etc, that are used for coverage identification.
The data set also contains target attributes $\mathbf{y} = \{ y_1,\cdots,y_{d'}\}$ that may or may not be considered for the coverage problem.
A query point $q$ is not covered by the data set $\dee$, if there are not ``enough'' data points in $\dee$ that are representative of $q$.
To generalize the notion of coverage, let us define $\gee(q)$ as the universe of tuples that would represent $q$ and let $\gee_\dee(q) = \gee(q)\cap \dee$. In other words, $\gee_\dee(q)$ are the set of tuples in $\dee$ that represent $q$.
Using this notation, we define the coverage of $q$ as the size of $\gee_\dee(q)$. That is,
$cov(q,\dee) = | \gee_\dee(q)|$.
Given a value $\tau$, $q$ is covered if $cov(q,\dee)>\tau$.
Similarly, a group $\gee$ is not covered if $\gee\cap \dee<\tau$.
The {\it uncovered region} in a data set is the collection of groups that are not covered by it.
\end{definition}

\subsection{Structured Data}
In this section, we focus on identifying representation bias in structured data.
Depending on the type of the attributes of interest, we categorize the techniques into two classes based on whether they target the problem for non-ordinal {\it categorical} (e.g. {\tt race}, {\tt gender}) or ordinal {\it continuous} (e.g. {\tt age}) attributes. The attributes of interest considered for representation bias often include sensitive attributes such as {\tt race} and {\tt gender} but are not necessarily limited to them.

\subsubsection{Categorical Attributes}

For cases where attributes of interest are non-ordinal categorical,
the cartesian product of values on a subset of attributes $\mathbf{x}'\subseteq \mathbf{x}$, form a set of (sub-)groups.
For example, $\{$ {\tt white male}, {\tt white female}, {\tt black male} $,\cdots\}$ are the subgroups defined on the attributes {\tt (race,gender)}.
We refer to the number of attributes used to specify a subgroup as the {\it level} of that subgroup.
For example, the level of the subgroup {\tt white male} is 2, while the level of the subgroup {\tt male} is 1.
We use $\ell(\gee)$, to refer to the level of a subgroup $\gee$.
Similarly, we say a subgroup $\gee'$ is a subset of $\gee$, if the groups specifying $\gee'$ are a superset of the ones for $\gee$. For example {\tt (married white male)} a subset of the more general group {\tt (white male)}. That is, the set of individuals in group {\tt (married white male)} are a subset of {\tt (white male)}.
Moreover, we say a subgroup $\gee$ is a {\it parent} of the subgroup $\gee'$, if $\gee'\subset \gee$ and $\ell(\gee)=\ell(\gee')+1$. For example, the subgroup {\tt (white male)} is a parent of the subgroup {\tt (married white male)}.
We use \textit{patterns} to refer to uncovered subgroups.
A pattern $P$ is a string of $d$ values, where $P[i]$ is either a value from the domain of $x_i$, or it is ``unspecified'', specified with $X$. 
For example, consider a data set with three binary attributes of interest $\mathbf{x}=\{x_1, x_2, x_3\}$. The pattern $P=X01$ specifies all the tuples for which $x_2=0$ and $x_3=1$ ($x_1$ can have any value).
The set of patterns that identify most general uncovered subgroups are called {\it Maximal Uncovered Patterns} (MUPs).

No polynomial time algorithm can guarantee the enumeration of the entire MUPs, however, several algorithms inspired by set enumeration and the Apriori algorithm for association rule mining are proposed to efficiently address this problem~\cite{asudeh2019assessing}.
In this regard, we introduce \textit{Pattern Graph} data structure that exploits the relationship between patterns to do less work than computing all uncovered patterns by removing the non-maximal ones. 
The parent-child relationship between the patterns is represented in a graph that can be used to find better algorithms. 
\textit{Pattern-Breaker} starts from the top of the graph where the general patterns are and moves down by breaking each pattern into more specific ones. If a pattern is uncovered, then all of its descendants are also uncovered and they can not be an MUP, even if they have a parent that is covered. Therefore, this subgraph of the pattern graph can be pruned. 
The issue with \textit{Pattern-Breaker} is that it explores the covered regions of the pattern graph and for the cases where there are a few uncovered patterns, it has to explore a large portion of the exponential-size graph. 
To tackle this, \textit{Pattern-Combiner} algorithm is proposed that performs a bottom-up traversal of the pattern graph. It uses an observation that the coverage of a node at the level of the pattern graph can be computed as the sum of the coverage values of its children. 
The problem with \textit{Pattern-Combiner} is that it traverses over the uncovered nodes first and therefore, it will not perform well for the cases in which most of the nodes in the graph are uncovered. 
In fact, for the cases where most of the MUPs are placed in the middle of the graph, both \textit{Pattern-Breaker} and \textit{Pattern-Combiner} will not be as efficient as they should traverse half of the graph. Therefore, we propose \textit{Deep-Diver}, a search algorithm based on Depth-First-Search that quickly finds the MUPs, and uses them to limit the search space by pruning the nodes both dominating and dominated by the discovered MUPs.

\begin{figure*}[!tb]
    \begin{minipage}[t]{0.31\linewidth}
        \centering
        \includegraphics[width=\textwidth]{submissions/submission1/shahbazi/covcube1.jpg}
        \caption{\small Categorical attributes: the uncovered region of a toy example, as the collection of three MUPs.}
        \label{fig:covcube1}
    \end{minipage}
    \hfill
    \begin{minipage}[t]{0.31\linewidth}
        \centering
        \includegraphics[width=\textwidth]{submissions/submission1/shahbazi/cvrg_2_1.jpg}
        \caption{\small Continuous attributes, 2D: identifying the covered region in the gray Voronoi cell.}
        \label{fig:cvrg_2_1}
    \end{minipage}
    \hfill
    \begin{minipage}[t]{0.31\linewidth}
        \centering
        \includegraphics[width=\textwidth]{submissions/submission1/shahbazi/cvrg_2_2.jpg}
        \caption{ \small Continuous attributes, 2D: Uncovered region marked in red.}
        \label{fig:cvrg_2_2}
    \end{minipage}
\vspace{-5mm}
\end{figure*}

\subsubsection{Continuous Attributes}
Data in the real world often consists of a combination of continuous and discrete values. While simple solutions like binning {\tt age} into {\tt young} and {\tt old} can transform the continuous space into discrete. However, they may lead to coarse groupings that are sensitive to the thresholds chosen. It may be inappropriate to treat a 35-yo as {\tt young} but a 36-yo as {\tt old}. 
Therefore, we extend the notion of coverage to continuous space. Particularly, given data set $\dee$ with $n$ tuples over $d$ attributes, and vicinity radius $\rho$ and coverage threshold $k$, we want to identify the uncovered region -- the universe of uncovered query points.
A query point in continuous data space is covered if there are enough (at least $k$) data points in its $\rho$-vicinity neighborhood. $\rho$-vicinity neighborhood is the circle centered at the query point with radius $\rho$.

Depending on the number of attributes in a data set, we propose two algorithms for identifying uncovered regions in data~\cite{asudeh2021coverage}. 
The first algorithm known as \textit{Uncovered-2D} studies coverage over two-dimensional data sets where $\mathbf{x}=\{x_1,x_2\}$. To find the number of circles that a query point falls into and consequently discover the uncovered region, \textit{Uncovered-2D} makes a connection to $k$-th order Voronoi diagrams.
Consider a data set $\mathcal{D}$ and its corresponding $k$-th order Voronoi diagram. For every tuple $t\in \mathcal{D}$, let $\circ_t$ be the $d$-dimensional sphere ($d$-sphere) with radius $\rho$ centered at $t$.
Consider a $k$-voronoi cell $\mathcal{V}(S)$ in the $k$-th order Voronoi diagram $V_k(\mathcal{D})$.
Any point $q$ inside the intersections of the $d$-spheres of tuples in $S$, i.e. $q\in \underset{\forall t\in S}{\cap ~\circ_t}$, is covered, while all other points in the region are uncovered.
 The algorithm starts by constructing the $k$-th order Voronoi diagram of the data set and then for each Voronoi cell $\mathcal{V}(S)$ in the diagram, it computes the intersection of the circles of the tuples in $S$ and marks the portion of $\mathcal{V}(S)$ that falls outside it as uncovered.
After identifying the uncovered region, a 2D map of $\{x_1,x_2\}$ value combinations is used to report the region to the user.
The algorithm for the 2D case can be extended to the general case by relaxing the assumption on the number of attributes to discover the exact uncovered region, however, due to the curse of dimensionality, the search size space explodes as the number of dimensions increases and as a result, the algorithm will not be practical. Therefore, we propose a randomized approximation algorithm based on the geometric notion of \enet. 
Let $\mathcal{X}$ be a set and $\mathcal{R}$ be a set of subsets of $\mathcal{X}$. A set $\mathcal{N}\subset \mathcal{X}$ is an \enet for $\mathcal{X}$ if for any range $r\in\mathcal{R}$, if  $|r\cap \chi|>\eps|\chi|$, then $r$ contains at least one point of $N$.
The idea, at a high level, is to draw enough random samples from the space of potential query points to form an \enet. 
We then label the sampled query points as $\{-1,+1\}$ depending on whether those are covered or not, and learn the uncovered regions using the samples.

\subsection{Image Data}
Many known incidents of machine failures due to the lack of representation were on image data.
We consider an image data set with a fixed number of low-cardinality sensitive attributes such as {\tt\small race} and {\tt\small gender}. 
It is common that image data sets {\it lack explicit values} for sensitive attributes, which are crucial for coverage identification. An image data set is often a collection of images from different domains with little to no information about their domain and which groups they belong to. As a result, even studying coverage over low-cardinality and categorical attributes of interests is challenging in these cases.

\begin{wrapfigure}{R}{0.42\textwidth}
\centering
\vspace{-3mm}
\scriptsize
\begin{tabular}{|@{}c|@{}c@{}|@{}c@{}|@{}c@{}|} 
 \hline
{\bf data set} & {\bf classifier} & {\bf accuracy} & {\bf precision} \\ 
 &  &  & {\bf on female} \\ \hline
UTKFace:~& DeepFace (opencv) & 93.56 & {52.02}\\\cline{2-4}
({\tt females}=200,& DeepFace (retinaface) & 94.16 & {56.15}\\\cline{2-4}
{\tt males}=2800) & BaseCNN & 97.6 & 74.8\\
\hline
UTKFace:~& DeepFace (opencv) & 96.53 & {\bf 8.0}\\\cline{2-4}
({\tt females}=20,& DeepFace (retinaface) & 96.43 & {\bf 10.09}\\\cline{2-4}
{\tt males}=2980)& BaseCNN & 97.6 & {\bf 21.59}\\
\hline
\end{tabular}
\vspace{-3mm}
\caption{\small ML models' low performance for females in the presence of representation bias.~\cite{mousavi2024data}}\label{fig:mlfails}
\vspace{-3mm}
\end{wrapfigure}

In Figure~\ref{fig:mlfails}, we show that due to the issues such {\it machine bias} and {\it lack of distribution generalizability},
solely relying on state-of-the-art machine learning (ML) techniques fail to effectively identify lack of coverage in image data sets. Therefore, we propose an approach based on combining crowdsouring with ML~\cite{mousavi2024data}. 
Crowdsourcing is particularly promising for image data, for tasks such as image labeling, which, while challenging for the machine, are "easy" for human beings to conduct with minimal error. 

A key observation that enables a cost-effective crowdsourcing approach is that, while studying coverage, we would only like to find out if there are {\it enough tuples from each subgroup}.
Suppose a subgroup is covered if there are $\tau=100$ instances of it in the data set. Assume the (majority) group $\gee_1$ contains $n_1 \gg 100$ objects in the data set. 
To verify that $\gee_1$ is covered, it is enough for the crowd to discover 100 of those objects, not the entire $n_1$. 
Following this, $O(\tau)$ provides a lower bound on the number of crowd tasks required to verify a given group is covered. 
Still, this lower bound only holds for the groups that are covered, i.e., there is at least $\tau$ of those in the data set.
Surprisingly, verifying that a minority group is indeed uncovered is cumbersome, unlike the majority group.
This is because even though discovering $\tau$ objects from a group is enough for verifying that it is covered, one cannot {\it verify} a group is uncovered until there is a chance that the data set might still have enough objects from that group. Thus, assuming a non-zero probability for each unlabeled object to belong to each group, {one might need to ask the crowd to label the entire data set before they can confirm that a specific group is uncovered}.

Our idea for addressing this challenge is to
design {\it a divide and conquer algorithm} that, instead of {point queries}, uses {\it set queries} to iteratively eliminate subsets of data that {does not include any object from the given group}.
At a high level, our idea is to ask a set query from the crowd, inquiring whether the selected set contains at least one object from the given group $\gee$.
The user may provide two responses (yes/no). 
Interestingly, {in either case}, the user response provides valuable information that helps efficiently identify the coverage.
If the answer is ``No'', the set does not include any object from the given group $\gee$. As a result, the algorithm can safely prune the set, asking no further questions about it. In particular, for a group that is not covered, one can expect to see no answers on large set queries helping to prune a significant portion of the data set quickly.
On the other hand, if the answer is ``yes'', the set contains {at least} one object from the group $\gee$. As a result, the algorithm cannot prune the subset since it can have any number (larger than one) of the objects in $\gee$.
At first glance, the queries with yes answers do not provide helpful information as the algorithm cannot prune the subset (hence it needs to divide it into smaller subsets).
However, a key observation is that {the algorithm will only observe a limited number of yes answers} before it stops.
The reason is that the number of set queries with yes answers provides a {lower-bound} on the number of objects from $\gee$ in the data set. As a result, the algorithm can stop as soon as the lower bound reaches $\tau$, knowing that $\gee$ is covered.
The D\&C approach verifies the data coverage for a given group, while our goal is to identify the uncovered regions for a given set of sensitive attributes. The next question is how to utilize this algorithm for efficient coverage identification on different scenarios of sensitive attributes, forming intersectional or non-intersectional groups.
In particular, how can we find maximal uncovered patterns?
Our idea is to apply sampling and aggregate estimation techniques to find the groups that even if merged are likely to still be uncovered. This will help reduce the coverage identification cost by running the D\&C approach for the merged groups once.
 %%%%%%%%%%%%%%%%%%%%%%%%%%%%%%%% RESOLUTION  %%%%%%%%%%%%%%%%%%%%%%%%%%%%%%%%
\section{Resolving Insufficient Representation}\label{sec:resolution}

Data integration~\cite{nargesian2021tailoring,nargesian2022responsible} and data augmentation~\cite{sharma2020data,DBLP:journals/jair/ChawlaBHK02,iosifidis2018dealing,celis2020data} are considered as the primary solutions for reducing data coverage issues in a data set. 
Data integration is promising when external sources of data are available. On the other hand, recent advancements in generative AI and foundation models have enabled efficient and effective augmentation of data sets with synthetic data. 
Therefore, in the following, we review two approaches, one from each category, in the context of lack of coverage resolution.

\subsection{Data Integration}\label{sec:resolution:integration}

Data integration is to consolidate data from different sources into a single, unified view. 
Although it is an effective solution to acquire additional data from different distributions,
there are sampling policy and cost-efficiency concerns that need to be examined.  
Therefore, {\it Data Distribution Tailoring ({\sc DT})} introduces data integration techniques for resolving insufficient representation of subgroups in a data set in the most cost-effective manner~\cite{nargesian2021tailoring}.
A query to {\sc DT} 
consists of a target schema, and a set of group distribution requirements in the form of the minimum counts (e.g., ``{\tt\small 1,000 breast cancer monitoring data in Chicago with at least 30\% label=positive, and at least 20\% black patients}''). 
Collecting a fresh sample from a data view is costly (monetary, human resources, and/or computation cost)~\cite{asudeh2022towards}.
Therefore, {\sc DT} focuses on satisfying the count requirements with minimum cost. 
Given an input query and a lake of available data sources, the first step is to discover a collection of candidate data views that satisfy the target schema.
Each data view $v_i$ is a projection-join $v_i = \Pi\big(D_{i1}\bowtie\cdots\bowtie D_{ik_i} \big)$, where $D_{ij}$ is a data set in a given data lake.
Let us suppose the data views are already discovered.
At a high level, {\sc DT} follows an iterative approach that at each iteration a data view is selected to be queried.
Each query to a data view has a fixed cost and returns a sample that may or may not satisfy the query constraints.
The samples that are either not fresh, or do not satisfy the query are discarded.
Hence, the essential question towards a cost-effective data integration is {\it what data view to query next}.
Depending on the available information about the data sources, various techniques may be employed. 

For the cases when the group distributions are known, the process of collecting the target data set is a sequence of iterative steps, where at every step, the algorithm chooses a data view, queries it, and if the obtained tuple contributes to one of the groups for which the count requirement is not yet fulfilled, it is kept, otherwise discarded. To do so, a {Dynamic Programming (DP)} algorithm is proposed. An optimal source at each iteration minimizes the sum of its sampling cost plus the expected cost of collecting the remaining required groups, based on its sampling outcome.
The DP algorithm, however, has a pseudo-polynomial time complexity. Hence, it quickly becomes intractable for cases where the minimum count requirements for the groups are not small. 
For cases where the (sensitive) attribute of interest is binary, such as (biological) {\tt sex}={\tt \{male, female\}}, and the cost to query data is similar from all sources, it turns out that the optimal strategy is to query the data source with {maximum probability of obtaining a sample from the minority group}.
Expanding the binary-attributes algorithm for non-binary cases, the problem can be modeled as an extension of the ``{\it coupon collector's}'' problem~\cite{motwani1995randomized}, where the goal is to collect $m_i$ instances from each coupon (group) $\gee_i$.
At each iteration, the coupon collector's algorithm identifies a data view as most promising and queries it. In simple terms, a data view with a smaller query cost and a higher chance of obtaining minority groups is more promising.


For the cases where the group distributions are unknown, we model DT as a {\it multi-armed bandit} problem, where every data view is modeled as an arm. 
Every arm has an unknown distribution of different groups while pulling an arm (i.e., querying the corresponding data view) has a cost.
During various iterations, the algorithms pull the arms in an order that its expected total {\it reward} is maximized.
Arguing that the reward of obtaining a tuple from a group is proportional to how rare this group is across different data views, 
we design the reward function based on the expected cost one needs to pay in order to collect a tuple from a specific group.  
As the bandit strategy, we adopt {\it Upper Confidence Bound (UCB)} to balance exploration and exploitation. At every iteration, for every arm, UCB computes confidence intervals for the expected reward and selects the arm with the maximum upper bound of reward to be explored next.

\subsection{Data Augmentation using Foundation Models}

While data integration provides a promising approach for resolving coverage issues in a data set, its effectiveness is limited to the availability of external data sources that are rich enough to find sufficient fresh samples from minority groups. This, however, is not always possible, especially since the minority samples are rare and not easy to obtain.
Fortunately, recent advancements in Generative AI and Foundation Models have enabled synthesizing samples that are otherwise challenging to obtain from the real world.

Therefore, as an alternative approach to data integration, we turn our attention to the Foundation Models and Generative AI for resolving the lack of coverage. 
Particularly, models such as {\sc DALL.E}\footnote{\url{https://openai.com/dall-e-2}} have emerged as powerful tools for generating multi-modal data such as image, audio, and video.
 
We formalize the foundation model \fm as a black-box function with the following inputs, that once queried synthesize an output tuple.
\begin{itemize}
    \item {\bf Prompt}: A natural language description providing instructions on the details of the tuple to be generated. For instance, a prompt for image generation might be ``A realistic photo of a white cat running in a backyard.''
    \item {\bf Guide}: In cases where only a prompt is provided, the foundation model uses its imagination to generate the requested tuple. For the previous example, the prompt of a cat image, the breed, size, background, and other details are generated based on the model's imagination. Alternatively, a guide can be provided to influence the generation process. The guide is formalized as a pair $(t,m)$ where $t$ is a tuple and $m$ is a mask specifying which parts of the guide tuple should be changed. Using the cat example, $t$ can be a cat image and $m$ can specify the foreground to be regenerated.
\end{itemize}

There are multiple challenges towards effective data set augmentations using foundation models. 
First, we have to determine the minimal set of synthetic tuples that once added to the original data set, under-representation issues are resolved.
Second, the generated images should follow the underlying distribution represented in the input data set. Third, the generated tuples should have high quality and look realistic to a human evaluator. Last but not least, given the (often monetary) cost associated with the queries to the foundation model, we should ensure the cost-effectiveness of the data set repair process.

\begin{wrapfigure}{L}{0.45\textwidth}
\centering
\vspace{-3mm}
\scriptsize
    \includegraphics[width=.45\textwidth]{submissions/submission1/shahbazi/enhanced_pipeline.png}
\vspace{-3mm}
\caption{\small Architecture of \fmsystem for image data augmentation for coverage enhancement.}\label{fig:chameleon}
% \vspace{-3mm}
\end{wrapfigure}

\noindent Figure~\ref{fig:chameleon} shows the architecture of our system \fmsystem \cite{chameleon} for coverage enhancement using DALL-E image generator.
To address the first challenge, we define the combinations-selection problem, which minimizes the total number of synthetic tuples for resolving lack of coverage of minorities at the most general level. We show the problem is {\sc NP}-hard, and propose a greedy approximation algorithm for it.
To address the second and third challenges, \fmsystem follows a {\it rejection sampling} strategy.
It views each tuple in the data set $\dee$ as an iid sample from the underlying distribution $\xi$ it represents. It uses the vector representations (embeddings) space to describe the distribution. Then, given a newly generated tuple, it employs the one-class support vector machine (OCSVM) approach proposed by Scholkopf et al.~\cite{scholkopf1999support} to reject the tuple if it does not follow $\xi$.
Moreover, it models the quality evaluation as hypothesis testing and rejects the samples that have a higher chance of being labeled as ``unrealistic'' by a random human evaluator.
Finally, to minimize the number of queries to the foundation model, we provide a guide tuple (and a mask), in addition to the prompt, to the foundation model. We model the guide-selection problem as {\it contextual multi-armed bandit} and propose a solution based on the contextual UCB for it.

Before concluding this section, let us provide some experiment results to demonstrate the effectiveness of data augmentation with \fmsystem. We use FERET DB \cite{phillips1998feret} for this experiment, which comprises 1199 individual images and serves as a standardized facial image database for researchers to develop algorithms and report results. All images in FERET DB share the same dimensions, pose, and facial expression.
First, we identified the (level-1) uncovered ethnicity groups, using the threshold 80. We then used \fmsystem and resolved the lack of coverage issues.
To evaluate the effectiveness of the system, we trained a CNN model to predict the race of each image within this dataset. We then retrained the identical CNN on the repaired training data. Importantly, our test dataset for both experiments remains consistent and is derived from real images.
Table~\ref{tab:lackofcoverage} presents the improvements in precision, recall, and F1 score metrics for under-represented groups after repairing the dataset. The results indicate an enhancement in performance metrics for all under-represented groups following the repair process.

\begin{table}[t]
    \centering
    \caption{Illustrating the effect of lack of coverage repair using \fmsystem on \texttt{FERTDB}}
    \label{tab:lackofcoverage}
    \vspace{-3mm}
    \begin{tabular}{lcccccccc}
        \toprule
         & \multicolumn{4}{c}{\textbf{Classifier Performance on \texttt{FERTDB}}} & \multicolumn{4}{c}{\textbf{Classifier Performance on Repaired}} \\
        \cmidrule(lr){2-5} \cmidrule(lr){6-9}
        \textbf{Ethnicity Groups}& \#Images & Precision & Recall & F1-Score & \#Images & Precision & Recall & F1-Score \\
        \midrule
        Overall          & 756 & 0.81 & 0.75 & 0.78 & 987 & 0.70 & 0.75 & 0.72 \\ \hline
        Black            & 40  & 0.19 & 0.22 & 0.16 & 100 & 0.48 & 0.56 & 0.52 \\
        Hispanic         & 19  & 0.50 & 0.17 & 0.25 & 100 & 0.62 & 0.36 & 0.45 \\
        Middle Eastern   & 10  & 0.00 & 0.00 & 0.00 & 100 & 0.20 & 0.41 & 0.27 \\
        \bottomrule
    \end{tabular}
\end{table}

 %%%%%%%%%%%%%%%%%%%%%%%%%%%%%%%% RELIABILITY  %%%%%%%%%%%%%%%%%%%%%%%%%%%%%%%%
\section{Generating Reliability Warnings}\label{sec:reliability}
% up to 2.5 pages
Interpretability is a necessity for data scientists who develop predictive models for critical decision-making.
In such settings, it is important to provide additional means to support the following question:
{\it is an individual prediction of the model reliable for decision-making?} Our goal is to use the lack of representation to help decision-makers find insights about this critical question.
To further motivate this, let us use the following example:

\vspace{1mm}
\begin{example}\label{ex-0}
{\bf(Part1):} Consider a judge who needs to decide whether to accept or deny a bail request. Using data-driven predictive models is prevalent in such cases for predicting recidivism~\cite{dressel2018accuracy}.
Indeed, such models can be beneficial to help the judge make wise decisions.
Suppose the model predicts the queried individual as high risk (or low risk).
The judge is aware and concerned about the critics surrounding such models.
A major question the judge faces is whether or not they should rely on the prediction outcome to take action for this case.
Furthermore, if, for instance, they decide to ignore the outcome and hence they need to provide a statement supporting their action, what evidence can they provide? 
\end{example}

In line with the recent trend on data-centric AI~\cite{ng2021mlops}, we design {novel approaches}, {complimentary} to the existing work on trustworthy AI~\cite{wing2021trustworthy,kentour2021analysis,liu2021trustworthy,singh2021trustworthy}, to address the aforementioned trust question through the lens of {\it data}.
In particular, unlike existing works that generate trust information from a {\it given \underline{model}}, we associate {\it \underline{data sets} with proper measurements} that specify their {\it the scope of use for predicting future cases}.
We note that a predictive model provides only probabilistic guarantees on the \underline{average} loss over the distribution represented by the data set used for training it.
As a result, these predictions may not be distribution generalizable~\cite{kulynych2022you}.
Consequently, if the query point is {\it not represented} by the data, the guarantees may not hold, hence one cannot rely on the prediction outcome.
Besides, an essential requirement for a learning algorithm is that its training data $\dee$ should represent the underlying distribution $\dist$.
Even if so, the trained model $h$ only provides a probabilistic guarantee on the {expected} loss on random samples from $\dist$.  
A model that performs well on {\it majority} of samples drawn from $\dist$ will have a high performance on average. Still, as we observed in Figure~\ref{fig:mlfails},
its performance for {\it minorities} and points that are not represented is questionable. Let us consider the following toy example:

\begin{figure*}[!b] 
    \begin{minipage}[t]{0.32\linewidth}
        	\centering
        	\includegraphics[width=\textwidth]{submissions/submission1/shahbazi/example_1.png} 
        	\vspace{-9mm}\caption{\small Data set $\dee$ generated using a Gaussian distribution; $x_1$ and $x_2$ are positively correlated}
            \label{fig:ex1:1}
    \end{minipage}
    \hfill
    \begin{minipage}[t]{0.32\linewidth}
        \centering
        	\includegraphics[width =\textwidth]{submissions/submission1/shahbazi/example_2.png} 
        	\vspace{-9mm}\caption{\small The decision boundary of learned model $h$ and query points $\qu^1$ to $\qu^4$}
            \label{fig:ex1:2}
    \end{minipage}
    \hfill
    \begin{minipage}[t]{0.32\linewidth}
        	\centering
        	\includegraphics[width =\textwidth]{submissions/submission1/shahbazi/example_3.png}
        	\vspace{-9mm}\caption{\small Ground-truth boundary, overlaid on the model decision boundary and query points}
            \label{fig:ex1:3}
    \end{minipage}
    \vspace{-5mm}
\end{figure*} 

\vspace{1mm}
\begin{example}\label{ex-1}
Consider a binary classification task where the input space is $\ex=\langle x_1, x_2\rangle$ and the output space is the binary label $y$ with values $\{-1$ (red) $,+1$ (blue)$\}$.
Suppose the underlying data distribution $\dist$ follows a 2D Gaussian, where $x_1$ and $x_2$ 
are positively correlated as shown in Figure~\ref{fig:ex1:1}.
The figure shows the data set $\dee$ drawn independently from the distribution $\dist$, along with their labels as their colors.
Using $\dee$, the prediction model $h$ is constructed as shown in Figure~\ref{fig:ex1:2}. 
The decision boundary is specified in the picture; while any point above the line is predicted as +1, a query point below it is labeled as -1.
The classifier has been evaluated using a test set that is an iid sample set drawn from the underlying data set $\dist$. The accuracy on the test set is high (above 90\%), and hence, the model gets deployed.
We cherry-picked four query points, $\qu^1$ to $\qu^4$, that are also included in Figure~\ref{fig:ex1:2}. Using $h$ for prediction, $h(\qu^1)=-1$, $h(\qu^2)=+1$,  $h(\qu^3)=+1$, and $h(\qu^4)=-1$.
Figure~\ref{fig:ex1:3} adds the ground-truth boundary to the search space, revealing the true label of the query points: every point inside the red circle has the true label $-1$ while any point outside of it is $+1$.
Looking at the figure, $y^1=+1$ while the model predicted it as $h(\qu^1)=-1$.  \hfill$\square$
\end{example}
\vspace{2mm}

Let us take a closer look at the four query points in this example and their placement with regard to the tuples in $\dee$ used for training $h$. 
$\qu^2$ belongs to a {\it dense region} with many training tuples in $\dee$ surrounding it. Besides, all of the tuples in its vicinity have the same label $y=+1$. As a result, one can expect that the model's outcome $h(\qu^2)=+1$ should be a reliable prediction.
Similar to $\qu^2$, $\qu^4$ also belongs to a dense region in $\dee$; however, $\qu^4$ belongs to an {\it uncertain region}, where some of the tuples in its vicinity have a label $y=+1$, and some others have the label $y=-1$. Considering the uncertainty in the vicinity of $\qu^4$, one cannot confidently rely on the outcome of the model $h$. 
On the other hand, the neighbors of $\qu^1$ (resp. $\qu^3$) are not uncertain, all having the label $y=-1$ (resp. $y=+1$).
However, the query points $\qu^1$ and $\qu^3$ are not well represented by $\dee$. In other words, $\qu^1$ and $\qu^3$ are unlikely to be generated according to the underlying distribution $\dist$, represented by $\dee$. As a result, following the no-free-lunch theorem~\cite{kakade2003sample}, one cannot expect the outcome of model $h$ to be reliable for these points.
Looking at the ground-truth boundary in Figure~\ref{fig:ex1:3}, $h$ luckily predicted the outcome for $\qu^3$ correctly, but it was not fortunate to predict the $y^1$ correctly.
Nevertheless, 
since the model is not reliably trained for these points, 
its outcome for these query points is not trustworthy.

From Example~\ref{ex-1}, we observe that the outcome of a model $h$, trained using a data set $\dee$ is not reliable for a query point $\qu$, if:
\begin{itemize}
    \item {\bf Lack of representation:} $\qu$ is not well-represented by $\dee$.
    In such cases, the model has not seen ``enough'' samples similar to $\qu$ to reliably learn and predict the outcome of $\qu$.
    \item {\bf Lack of certainty:} $\qu$ belongs to an uncertain region, where different tuples of $\dee$ in the vicinity of $\qu$ have different target values. $\qu$ belongs to a high-fluctuating area, where tuples in the vicinity of $\qu$ have a wide range of values.
\end{itemize} \vspace{2mm}

\noindent
Based on these two observations, we propose Representation-and-Uncertainty ({\bf RU}) measures.
To identify if a query suffers from uncertainty or lack of representation, one could use a deterministic approach using a fixed threshold. Then if the number of similar samples to (resp. label fluctuation in vicinity of) $\qu$ is larger than the threshold it is considered as unrepresented (resp. uncertain).
This approach, however, would be misleading since two numbers close to the threshold could be treated very differently. Also, all points on each side of the threshold would be considered equally represented (resp., certain). Instead, we consider {\it a randomized approach}, widely popular in the literature, including~\cite{dwork2012fairness}.
That is, instead of using fixed thresholds, a Bernoulli variable (a biased coin) is used that 
assigns $\qu$ as unrepresented (resp., uncertain) based on the number of samples similar to it (resp., its neighborhood uncertainty).
Given a query point $\qu$, let $\pe_o$ be the probability indicating if $\qu$ is not represented and let $\pe_u$ be the probability indicating if $\qu$ belongs to an uncertain region. 
We represent the probability of the Bernoulli variables for lack of representation or uncertainty components as $\pe_o$ and $\pe_u$, respectively. Note that the two Bernoulli variables $\pe_o$ and $\pe_u$ are independent from each other. That simply follows the argument that after specifying the number of similar samples to $\qu$ whether or not it should be considered as unrepresented does not depend on the uncertainty in the neighborhood of $\qu$.

\begin{definition}[\sru]\label{def:sdt}
The \sru is a probabilistic measure that considers the outcome of a model for a query point $\qu$ untrustworthy if $\qu$ is not represented by $\dee$ {\it and} it belongs to an uncertain region.
Formally, the \sru measure is:
\begin{align} 
    \nonumber
    SRU(\qu) &= \pe\big((\qu \mbox{ is outlier}) \wedge (\qu \mbox{ belongs to uncertain region})\big) 
\end{align}
Since $\pe_o$ and $\pe_u$ are independent:

\vspace{-13mm}
\begin{align} \label{eq:strong}
    SRU(\qu) &= \pe_o(\qu) \times \pe_u(\qu)
\end{align}
\end{definition}

\sru raises the warning signal only when the query point fails on {\it both} conditions of being represented by $\dee$ and not belonging to an uncertain region. 
For instance, in Example~\ref{ex-1} none of the query points fail both on representation and on uncertainty; hence neither has a high \sru score.
On the other hand, 
a high \sru score for a query point $\qu$ {\it provides a strong warning signal} that one should perhaps reject the model outcome and not consider it for decision-making.

\sru is a strong signal that raises warnings only for the fearfully concerning cases that fail both on representation and uncertainty.
However, as observed in Example~\ref{ex-1} a query points failing {\it at least} one of these conditions may also not be reliable, at least for critical decision making.
We define the \wru measure to raise a warning for such cases.

\begin{definition}[\wru]\label{def:wdt}
The \wru measure is a probabilistic measure that considers the outcome of a model for a query point $\qu$ untrustworthy if $\qu$ is not represented by $\dee$ {\bf or} it belongs to an uncertain region.
Formally, the \wru is computed as:
\begin{align} \label{eq:weak}
    WRU(\qu) = \pe\big((\qu \mbox{ is outlier}) \vee (\qu \mbox{ belongs to uncertain region})\big) 
    = \pe_o(\qu) + \pe_u(\qu) - \pe_o(\qu) \times \pe_u(\qu)
\end{align}
\end{definition}

Proposing quantitative probabilistic outcomes, \ru measures are interpretable for the users, since beyond the scores, the uncertainty and lack of representation components provide an explanation to justify them. 
Please refer to \cite{techrep} for more details on how to efficiently and effectively compute the representation ($\pe_o$) and uncertainty ($\pe_u$) probabilities, using only $\dee$.
In Example~\ref{ex-0}, let us see how the \ru measures can be helpful.

\noindent{\bf Example 1. (part 2):}
{\it RU measures \underline{raise warning} when
the fitness of the data set used for drawing a prediction is questionable, helping the judge to be cautious when taking action.
Besides, these measures provide \underline{quantitative evidence} to support the judge's action when they decide to ignore a prediction outcome that is not trustworthy.
The judge, for example, can argue to ignore a model outcome for a specific case, based on the insight that 
the model has been built using a
data set that fails to represent the given case.}
\hfill$\square$

Finally, let us demonstrate the efficacy of \ru measures through a series of experiments. Since the \ru measures are {\it data-centric},
those are applicable for both classification and regression tasks, irrespective of the model used.
We use {\it Adult} dataset~\cite{adult} for classification and {\it House Sales in King County} dataset for the validation of regression tasks. From each dataset, we uniformly sample two sets from the underlying distribution. The first set serves as the training set to compute the \ru values, and the second one is used as the test set from which the queries are drawn. We validate our proposal by providing the correlation between the \ru values and the performance of an ML model's prediction on the same data. 

We start by computing the \ru values for all the query points in the test set. Next, we bucketize the query points based on their \ru values in equi-width buckets of width 0.1. We repeat this for both \sru and \wru measures. Next, we train a model on the training data set and predict the target variable for the points in each range of \ru measure. The validation results for the classification task on the {\it Adult} dataset are presented in Figures \ref{fig:exp-adult-sdt} and \ref{fig:exp-adult-wdt}. Each figure corresponds to the accuracy/error measures of the classifier over each bucket of \ru values for \sru and \wru. As the \ru values increase, the accuracy of the model drops while the FPR rises, and therefore, the model fails to capture the ground truth for the points that fall into untrustworthy regions in the data set. By repeating the aforementioned steps for the regression task on the {\it House Sales in King County} dataset, we observe similar results presented in Figures \ref{fig:exp-hs-sdt} and \ref{fig:exp-hs-wdt}. 
As the \ru value increases, the RSS of the regression model follows the same trend denoting that the model fails to perform for tuples with a high \ru value.

\begin{figure}[!tb]
    \begin{minipage}[t]{0.24\linewidth}
        \centering
        \includegraphics[width=\textwidth]{submissions/submission1/shahbazi/sdt_adult.pdf}
        \vspace{-6mm}\caption{\small{\it Adult}, efficacy of \sru  on classification}
        \label{fig:exp-adult-sdt}
    \end{minipage}\hfill
    \begin{minipage}[t]{0.24\linewidth}
        \centering
        \includegraphics[width=\textwidth]{submissions/submission1/shahbazi/wdt_adult.pdf}
        \vspace{-6mm}\caption{\small{\it Adult}, efficacy of \wru  on classification}
        \label{fig:exp-adult-wdt}
    \end{minipage}\hfill
    \begin{minipage}[t]{0.24\linewidth}
        \centering
        \includegraphics[width=\textwidth]{submissions/submission1/shahbazi/sdt_regression_house.pdf}
        \vspace{-6mm}\caption{\small{\it House Sales in King County}, efficacy of \sru on regression}
        \label{fig:exp-hs-sdt}
    \end{minipage}\hfill
    \begin{minipage}[t]{0.24\linewidth}
        \centering
        \includegraphics[width=\textwidth]{submissions/submission1/shahbazi/wdt_regression_house.pdf}
        \vspace{-6mm}\caption{\small{\it House Sales in King County}, efficacy \wru on regression}
        \label{fig:exp-hs-wdt}
    \end{minipage}
\vspace{-5mm}
\end{figure}
 %%%%%%%%%%%%%%%%%%%%%%%%%%%%%%%% RELATED WORK  %%%%%%%%%%%%%%%%%%%%%%%%%%%%%%%%
\section{Related Work}\label{related} 

Bias in data has been looked at for a long time in statistical community~\cite{neyman1936contributions} but social data presents different challenges~\cite{olteanu2019social,fairmlbook,barocas2016big,jk2019bias,drosou2017diversity}.
The diversity and representativeness of data have been widely studied~\cite{drosou2017diversity}, in fields such as social science~\cite{berrey2015enigma, dobbin2016diversity,simpson1949measurement}, political science~\cite{surowiecki2005wisdom}, and information retrieval~\cite{agrawal2009diversifying}. 
Tracing back machine bias to its source, there have been major efforts to identify different types~\cite{mehrabi2021survey, olteanu2019social,friedman1996bias} and sources~\cite{torralba2011unbiased,crawford2013hidden,diakopoulos2015algorithmic} of biases in data. Efforts to satisfy {\it responsible data} requirements~\cite{nargesian2022responsible} extend to various stages of the data analysis pipeline, including data annotation~\cite{li2020towards,lazier2023fairness}, data cleaning and repair~\cite{SalimiRHS19,tae2019data,salimi2020database}, data imputation~\cite{martinez2019fairness}, entity resolution~\cite{shahbazi2023through,fanourakis2023fairer}, data integration~\cite{nargesian2022responsible,nargesian2021tailoring}, etc. 

\paragraph{Data Coverage:}The notion of data coverage has received extensive attention from different angles. Detecting lack of coverage has been studied for datasets with discrete~\cite{asudeh2019assessing} and continuous~\cite{asudeh2021coverage} attributes populated in single or multiple \cite{lin2020identifying} relations.
To resolve insufficient coverage, \cite{accinelli2020coverage, accinelli2021impact,shetiya2022fairness}
consider resolving representation bias in preprocessing pipelines by rewriting queries into the closest operation so that certain subgroups are sufficiently represented in the downstream tasks. Alternatively, ~\cite{asudeh2019assessing,tae2021slice} propose a data collection strategy to acquire as little additional data as possible (to minimize the associated costs) to meet the representation constraints. ~\cite{sharma2020data,iosifidis2018dealing,celis2020data} opt for a data augmentation approach by adding partially altered duplicates of already existing tuples or generating new synthetic entries from existing data. Consequently, the new data set has an equal number of elements for different groups, resulting in potentially resolving the under-representation issues. Finally,  \cite{nargesian2021tailoring} utilizes data integration techniques to consolidate data from different sources into a single dataset to resolve representation bias.
Related works also include ~\cite{chung2019slice,sagadeeva2021sliceline,tae2021slice} that seek to understand if the overall performance of the model fails to reflect and performs poorly on certain slices in the data.
As alternative approaches to measure representation bias, the notion of representation rate~\cite{celis2020data} (a.k.a. equal base rate~\cite{kleinberg2016inherent}) is introduced which compared with coverage, it is more restrictive as it requires almost equal ratios from different groups.
Please refer to \cite{shahbazi2023representation} for a comprehensive survey about representation bias in data. 

\paragraph{ML Reliability:} Model-centric works for uncertainty quantification such as 
probabilistic classifiers~\cite{zadrozny2001obtaining,zadrozny2002transforming,platt1999probabilistic,niculescu2005predicting},
prediction intervals (PIs) \cite{chatfield93predictionintervals,pearce2018high,khosravi2010lower} and conformal predictions (CP)~\cite{angelopoulos2021gentle,shafer2008tutorial} that are used for measuring prediction uncertainty, are built
by maximizing the {\it expected performance} on {\it random} sample from the underlying distribution.
As a result, while providing accurate estimations for the dense regions of data (e.g. majority groups), their estimation accuracy is questionable for the poorly represented regions.
In particular, \cite{angelopoulos2021gentle} recognizes the lack of guarantees in the performance of CP for such regions.
Besides, the bulk of work on trustworthy AI provides information that {\it supports} the outcome of an ML model. For example, existing work on explainable AI, including~\cite{harradon2018causal,ribeiro2016should,gunning2019darpa}, aims to find simple explanations and rules that justify the outcome of a model.
Conversely, we aim to {\it raise warning signals} when the outcome of a model is {\it not} trustworthy. That is, to provide reasons that {\it cast doubt} on the reliability of the model outcome {for a given query point}.

 %%%%%%%%%%%%%%%%%%%%%%%%%%%%%%%% FUTURE  %%%%%%%%%%%%%%%%%%%%%%%%%%%%%%%%
% \vspace{-3mm}
\section{Final Remarks}\label{sec:conclusion}
As Data-centric AI and Responsible AI emerge as focal points in data science research, the development of Data-centric methodologies for ensuring Responsible and Trustworthy AI attracts increasing attention.
While there is some excellent work on responsible data management to achieve this goal, there remain many challenges yet to be addressed.

In this paper, we focused on a crucial aspect of responsible data -- detecting and addressing the under-representation of minorities within a data set.
We formally defined the notion of data coverage and discussed various techniques for (a) identifying lack of representation issues across different data modalities, (b) ensuring proper representation of minorities in data, and (c) limiting the scope-of-use of data sets based on their representation issues by generating proper ({\sc RU}) warning signals.
Even though the research on detecting lack of coverage issues is relatively mature, resolution techniques are still understudied.
Considering the recent advancements in Generative AI, utilizing Foundation Models and Large Language Models, and studying their limitations, for data augmentation to improve the representation of minorities at the data level seems interesting to further explore.

 %%%%%%%%%%%%%%%%%%%%%%%%%%%%%%%% BIB  %%%%%%%%%%%%%%%%%%%%%%%%%%%%%%%%
\bibliographystyle{unsrt}
\small
% \bibliography{ref}
\begin{thebibliography}{10}

\bibitem{asudeh2019assessing}
A.~Asudeh, Z.~Jin, and H.~Jagadish.
\newblock Assessing and remedying coverage for a given dataset.
\newblock In {\em ICDE}, pages 554--565. IEEE, 2019.

\bibitem{shahbazi2023representation}
N.~Shahbazi, Y.~Lin, A.~Asudeh, and H.~Jagadish.
\newblock Representation bias in data: A survey on identification and resolution techniques.
\newblock {\em ACM Computing Surveys}, 2023.

\bibitem{asudeh2021coverage}
A.~Asudeh, N.~Shahbazi, Z.~Jin, and H.~V. Jagadish.
\newblock Identifying insufficient data coverage for ordinal continuous-valued attributes.
\newblock In {\em SIGMOD}. ACM, 2021.

\bibitem{mousavi2024data}
M.~Mousavi, N.~Shahbazi, and A.~Asudeh.
\newblock Data coverage for detecting representation bias in image datasets: {A} crowdsourcing approach.
\newblock In {\em {EDBT}}, pages 47--60, 2024.

\bibitem{nargesian2021tailoring}
F.~Nargesian, A.~Asudeh, and H.~Jagadish.
\newblock Tailoring data source distributions for fairness-aware data integration.
\newblock {\em Proceedings of the VLDB Endowment}, 14(11):2519--2532, 2021.

\bibitem{nargesian2022responsible}
F.~Nargesian, A.~Asudeh, and H.~V. Jagadish.
\newblock Responsible data integration: Next-generation challenges.
\newblock {\em SIGMOD}, 2022.

\bibitem{sharma2020data}
S.~Sharma, Y.~Zhang, J.~M. R{\'\i}os~Aliaga, D.~Bouneffouf, V.~Muthusamy, and K.~R. Varshney.
\newblock Data augmentation for discrimination prevention and bias disambiguation.
\newblock In {\em AIES}, pages 358--364, 2020.

\bibitem{DBLP:journals/jair/ChawlaBHK02}
N.~V. Chawla, K.~W. Bowyer, L.~O. Hall, and W.~P. Kegelmeyer.
\newblock {SMOTE:} synthetic minority over-sampling technique.
\newblock {\em J. Artif. Intell. Res.}, 16:321--357, 2002.

\bibitem{iosifidis2018dealing}
V.~Iosifidis and E.~Ntoutsi.
\newblock Dealing with bias via data augmentation in supervised learning scenarios.
\newblock {\em Jo Bates Paul D. Clough Robert J{\"a}schke}, 24, 2018.

\bibitem{celis2020data}
L.~E. Celis, V.~Keswani, and N.~Vishnoi.
\newblock Data preprocessing to mitigate bias: A maximum entropy based approach.
\newblock In {\em ICML}, pages 1349--1359. PMLR, 2020.

\bibitem{asudeh2022towards}
A.~Asudeh and F.~Nargesian.
\newblock Towards distribution-aware query answering in data markets.
\newblock {\em Proceedings of the VLDB Endowment}, 15(11):3137--3144, 2022.

\bibitem{motwani1995randomized}
R.~Motwani and P.~Raghavan.
\newblock {\em Randomized algorithms}.
\newblock Cambridge university press, 1995.

\bibitem{chameleon}
M.~Erfanian, H.~V. Jagadish, and A.~Asudeh.
\newblock Chameleon: Foundation models for fairness-aware multi-modal data augmentation to enhance coverage of minorities.
\newblock {\em arXiv preprint arXiv:2402.01071}, 2024.

\bibitem{scholkopf1999support}
B.~Sch{\"o}lkopf, R.~C. Williamson, A.~Smola, J.~Shawe-Taylor, and J.~Platt.
\newblock Support vector method for novelty detection.
\newblock {\em NeurIPS}, 12, 1999.

\bibitem{phillips1998feret}
P.~J. Phillips, H.~Wechsler, J.~Huang, and P.~J. Rauss.
\newblock The feret database and evaluation procedure for face-recognition algorithms.
\newblock {\em Image and vision computing}, 16(5):295--306, 1998.

\bibitem{dressel2018accuracy}
J.~Dressel and H.~Farid.
\newblock The accuracy, fairness, and limits of predicting recidivism.
\newblock {\em Science advances}, 4(1):eaao5580, 2018.

\bibitem{ng2021mlops}
A.~Ng.
\newblock Mlops: From model-centric to data-centric {AI}.
\newblock 2021.

\bibitem{wing2021trustworthy}
J.~M. Wing.
\newblock Trustworthy {AI}.
\newblock {\em CACM}, 64(10):64--71, 2021.

\bibitem{kentour2021analysis}
M.~Kentour and J.~Lu.
\newblock Analysis of trustworthiness in machine learning and deep learning.
\newblock {\em InfoComp}, 2021.

\bibitem{liu2021trustworthy}
H.~Liu, Y.~Wang, W.~Fan, X.~Liu, Y.~Li, S.~Jain, A.~K. Jain, and J.~Tang.
\newblock Trustworthy {AI}: A computational perspective.
\newblock {\em arXiv preprint arXiv:2107.06641}, 2021.

\bibitem{singh2021trustworthy}
R.~Singh, M.~Vatsa, and N.~Ratha.
\newblock Trustworthy {AI}.
\newblock In {\em 8th ACM IKDD CODS and 26th COMAD}, pages 449--453. 2021.

\bibitem{kulynych2022you}
B.~Kulynych, Y.-Y. Yang, Y.~Yu, J.~B{\l}asiok, and P.~Nakkiran.
\newblock What you see is what you get: Distributional generalization for algorithm design in deep learning.
\newblock {\em arXiv preprint arXiv:2204.03230}, 2022.

\bibitem{kakade2003sample}
S.~M. Kakade.
\newblock {\em On the sample complexity of reinforcement learning}.
\newblock University of London, University College London (United Kingdom), 2003.

\bibitem{dwork2012fairness}
C.~Dwork, M.~Hardt, T.~Pitassi, O.~Reingold, and R.~Zemel.
\newblock Fairness through awareness.
\newblock In {\em ITCS}, pages 214--226, 2012.

\bibitem{techrep}
N.~Shahbazi and A.~Asudeh.
\newblock Data-centric reliability evaluation of individual predictions.
\newblock {\em CoRR, abs/2204.07682}, 2022.

\bibitem{adult}
M.~Lichman.
\newblock Adult income dataset, {UCI} machine learning repository.
\newblock \url{https://archive.ics.uci.edu/ml/datasets/adult}, 2013.

\bibitem{neyman1936contributions}
J.~Neyman and E.~S. Pearson.
\newblock Contributions to the theory of testing statistical hypotheses.
\newblock {\em Statistical Research Memoirs}, 1936.

\bibitem{olteanu2019social}
A.~Olteanu, C.~Castillo, F.~Diaz, and E.~Kiciman.
\newblock Social data: Biases, methodological pitfalls, and ethical boundaries.
\newblock {\em Frontiers in Big Data}, 2:13, 2019.

\bibitem{fairmlbook}
S.~Barocas, M.~Hardt, and A.~Narayanan.
\newblock Fairness and machine learning: Limitations and opportunities.
\newblock \url{fairmlbook.org}, 2019.

\bibitem{barocas2016big}
S.~Barocas and A.~D. Selbst.
\newblock Big data's disparate impact.
\newblock {\em Calif. L. Rev.}, 104:671, 2016.

\bibitem{jk2019bias}
J.~Kleinberg.
\newblock Fairness, rankings, and behavioral biases.
\newblock FAT*, 2019.

\bibitem{drosou2017diversity}
M.~Drosou, H.~Jagadish, E.~Pitoura, and J.~Stoyanovich.
\newblock Diversity in big data: A review.
\newblock {\em Big data}, 5(2):73--84, 2017.

\bibitem{berrey2015enigma}
E.~Berrey.
\newblock {\em The enigma of diversity: The language of race and the limits of racial justice}.
\newblock University of Chicago Press, 2015.

\bibitem{dobbin2016diversity}
F.~Dobbin and A.~Kalev.
\newblock Why diversity programs fail and what works better.
\newblock {\em Harvard Business Review}, 94(7-8):52--60, 2016.

\bibitem{simpson1949measurement}
E.~H. Simpson.
\newblock Measurement of diversity.
\newblock {\em Nature}, 163(4148), 1949.

\bibitem{surowiecki2005wisdom}
J.~Surowiecki.
\newblock {\em The wisdom of crowds}.
\newblock Anchor, 2005.

\bibitem{agrawal2009diversifying}
R.~Agrawal, S.~Gollapudi, A.~Halverson, and S.~Ieong.
\newblock Diversifying search results.
\newblock In {\em WSDM}, pages 5--14. ACM, 2009.

\bibitem{mehrabi2021survey}
N.~Mehrabi, F.~Morstatter, N.~Saxena, K.~Lerman, and A.~Galstyan.
\newblock A survey on bias and fairness in machine learning.
\newblock {\em ACM Computing Surveys (CSUR)}, 54(6):1--35, 2021.

\bibitem{friedman1996bias}
B.~Friedman and H.~Nissenbaum.
\newblock Bias in computer systems.
\newblock {\em TOIS}, 14(3):330--347, 1996.

\bibitem{torralba2011unbiased}
A.~Torralba and A.~A. Efros.
\newblock Unbiased look at dataset bias.
\newblock In {\em CVPR 2011}, pages 1521--1528. IEEE, 2011.

\bibitem{crawford2013hidden}
K.~Crawford.
\newblock The hidden biases in big data.
\newblock {\em Harvard business review}, 1(4), 2013.

\bibitem{diakopoulos2015algorithmic}
N.~Diakopoulos.
\newblock Algorithmic accountability: Journalistic investigation of computational power structures.
\newblock {\em Digital journalism}, 3(3):398--415, 2015.

\bibitem{li2020towards}
Y.~Li, H.~Sun, and W.~H. Wang.
\newblock Towards fair truth discovery from biased crowdsourced answers.
\newblock In {\em SIGKDD}, pages 599--607, 2020.

\bibitem{lazier2023fairness}
S.~Lazier, S.~Thirumuruganathan, and H.~Anahideh.
\newblock Fairness and bias in truth discovery algorithms: An experimental analysis.
\newblock {\em arXiv preprint arXiv:2304.12573}, 2023.

\bibitem{SalimiRHS19}
B.~Salimi, L.~Rodriguez, B.~Howe, and D.~Suciu.
\newblock Interventional fairness: Causal database repair for algorithmic fairness.
\newblock In {\em {SIGMOD}}, pages 793--810. {ACM}, 2019.

\bibitem{tae2019data}
K.~H. Tae, Y.~Roh, Y.~H. Oh, H.~Kim, and S.~E. Whang.
\newblock Data cleaning for accurate, fair, and robust models: A big data-{AI} integration approach.
\newblock In {\em DEEM workshop}, pages 1--4, 2019.

\bibitem{salimi2020database}
B.~Salimi, B.~Howe, and D.~Suciu.
\newblock Database repair meets algorithmic fairness.
\newblock {\em ACM SIGMOD Record}, 49(1):34--41, 2020.

\bibitem{martinez2019fairness}
F.~Mart{\'\i}nez-Plumed, C.~Ferri, D.~Nieves, and J.~Hern{\'a}ndez-Orallo.
\newblock Fairness and missing values.
\newblock {\em arXiv preprint arXiv:1905.12728}, 2019.

\bibitem{shahbazi2023through}
N.~Shahbazi, N.~Danevski, F.~Nargesian, A.~Asudeh, and D.~Srivastava.
\newblock Through the fairness lens: Experimental analysis and evaluation of entity matching.
\newblock {\em Proceedings of the VLDB Endowment}, 16(11):3279--3292, 2023.

\bibitem{fanourakis2023fairer}
N.~Fanourakis, C.~Kontousias, V.~Efthymiou, V.~Christophides, and D.~Plexousakis.
\newblock Fairer demo: Fairness-aware and explainable entity resolution.
\newblock 2023.

\bibitem{lin2020identifying}
Y.~Lin, Y.~Guan, A.~Asudeh, and H.~Jagadish.
\newblock Identifying insufficient data coverage in databases with multiple relations.
\newblock {\em Proceedings of the VLDB Endowment}, 13(12):2229--2242, 2020.

\bibitem{accinelli2020coverage}
C.~Accinelli, S.~Minisi, and B.~Catania.
\newblock Coverage-based rewriting for data preparation.
\newblock In {\em EDBT Workshops}, 2020.

\bibitem{accinelli2021impact}
C.~Accinelli, B.~Catania, G.~Guerrini, and S.~Minisi.
\newblock The impact of rewriting on coverage constraint satisfaction.
\newblock In {\em EDBT Workshops}, 2021.

\bibitem{shetiya2022fairness}
S.~Shetiya, I.~P. Swift, A.~Asudeh, and G.~Das.
\newblock Fairness-aware range queries for selecting unbiased data.
\newblock In {\em ICDE}. IEEE, 2022.

\bibitem{tae2021slice}
K.~H. Tae and S.~E. Whang.
\newblock Slice tuner: A selective data acquisition framework for accurate and fair machine learning models.
\newblock In {\em SIGMOD}, pages 1771--1783, 2021.

\bibitem{chung2019slice}
Y.~Chung, T.~Kraska, N.~Polyzotis, K.~H. Tae, and S.~E. Whang.
\newblock Slice finder: Automated data slicing for model validation.
\newblock In {\em ICDE}, pages 1550--1553. IEEE, 2019.

\bibitem{sagadeeva2021sliceline}
S.~Sagadeeva and M.~Boehm.
\newblock Sliceline: Fast, linear-algebra-based slice finding for ml model debugging.
\newblock In {\em SIGMOD}, pages 2290--2299, 2021.

\bibitem{kleinberg2016inherent}
J.~Kleinberg, S.~Mullainathan, and M.~Raghavan.
\newblock Inherent trade-offs in the fair determination of risk scores.
\newblock {\em arXiv preprint arXiv:1609.05807}, 2016.

\bibitem{zadrozny2001obtaining}
B.~Zadrozny and C.~Elkan.
\newblock Obtaining calibrated probability estimates from decision trees and naive bayesian classifiers.
\newblock In {\em ICML}, volume~1, pages 609--616. Citeseer, 2001.

\bibitem{zadrozny2002transforming}
B.~Zadrozny and C.~Elkan.
\newblock Transforming classifier scores into accurate multiclass probability estimates.
\newblock In {\em SIGKDD}, pages 694--699, 2002.

\bibitem{platt1999probabilistic}
J.~Platt et~al.
\newblock Probabilistic outputs for support vector machines and comparisons to regularized likelihood methods.
\newblock {\em Advances in large margin classifiers}, 10(3):61--74, 1999.

\bibitem{niculescu2005predicting}
A.~Niculescu-Mizil and R.~Caruana.
\newblock Predicting good probabilities with supervised learning.
\newblock In {\em Proceedings of the 22nd international conference on Machine learning}, pages 625--632, 2005.

\bibitem{chatfield93predictionintervals}
C.~Chatfield.
\newblock Prediction intervals.
\newblock {\em Journal of Business and Economic Statistics}, 11:121--135, 1993.

\bibitem{pearce2018high}
T.~Pearce, A.~Brintrup, M.~Zaki, and A.~Neely.
\newblock High-quality prediction intervals for deep learning: A distribution-free, ensembled approach.
\newblock In {\em International conference on machine learning}, pages 4075--4084. PMLR, 2018.

\bibitem{khosravi2010lower}
A.~Khosravi, S.~Nahavandi, D.~Creighton, and A.~F. Atiya.
\newblock Lower upper bound estimation method for construction of neural network-based prediction intervals.
\newblock {\em IEEE transactions on neural networks}, 22(3):337--346, 2010.

\bibitem{angelopoulos2021gentle}
A.~N. Angelopoulos and S.~Bates.
\newblock A gentle introduction to conformal prediction and distribution-free uncertainty quantification.
\newblock {\em arXiv preprint arXiv:2107.07511}, 2021.

\bibitem{shafer2008tutorial}
G.~Shafer and V.~Vovk.
\newblock A tutorial on conformal prediction.
\newblock {\em Journal of Machine Learning Research}, 9(3), 2008.

\bibitem{harradon2018causal}
M.~Harradon, J.~Druce, and B.~Ruttenberg.
\newblock Causal learning and explanation of deep neural networks via autoencoded activations.
\newblock {\em arXiv preprint arXiv:1802.00541}, 2018.

\bibitem{ribeiro2016should}
M.~T. Ribeiro, S.~Singh, and C.~Guestrin.
\newblock " why should i trust you?" explaining the predictions of any classifier.
\newblock In {\em SIGKDD}, pages 1135--1144, 2016.

\bibitem{gunning2019darpa}
D.~Gunning and D.~Aha.
\newblock Darpa’s explainable artificial intelligence ({XAI}) program.
\newblock {\em AI Magazine}, 40(2):44--58, 2019.

\end{thebibliography}

\end{document}

\end{article}



\begin{article}
{On Benchmarking for Crowdsourcing and Future of Work Platforms}
{Ria Mae Borromeo, Lei Chen, Abhishek Dubey, Sudeepa Roy, Saravanan Thirumuruganathan}
\graphicspath{{submissions/lei/}}
% link to instruction: https://tc.computer.org/tcde/tcde-bulletin-author-instructions/
% \documentclass[11pt,dvipdfm]{article}
\documentclass[11pt]{article}
\usepackage{tabularx}
\usepackage{ragged2e}  % for '\RaggedRight' macro (allows hyphenation)
\usepackage{booktabs}  % for \toprule, \midrule, and \bottomrule macros
\usepackage{textcomp}
\usepackage{amsfonts,amsmath}
\usepackage{deauthor,times}
\usepackage{graphicx} % 
\usepackage{hyperref}
\usepackage{comment}
\graphicspath{{asudeh/}}
\usepackage{soul}
\usepackage{subcaption}
\usepackage{ulem}
\usepackage{wrapfig}
\usepackage{color}
\usepackage{xspace}
\newtheorem{problem}{Problem}

%\DeclareMathOperator*{\argmax}{arg\,max}

%remove the following commands/package b4 submission
\newcommand{\hide}[1]{}
\newcommand{\eat}[1]{}
\newcommand{\resolved}[1]{\hide{#1}}
\newcommand{\abol}[1]{\textcolor{red}{Abol: #1}}
\newcommand{\mahdi}[1]{\textcolor{red}{Mahdi: #1}}
\newcommand{\nima}[1]{\textcolor{red}{Nima: #1}}

\newcommand{\dee}{\mathcal{D}}
\newcommand{\Gee}{\mathcal{G}}
\newcommand{\gee}{\mathbf{g}}
\newcommand{\ee}{\mathbf{e}}
\newcommand{\es}{\mathcal{S}}
\newcommand{\el}{\mathcal{L}}
\newcommand{\xx}{\mathcal{x}}
\newcommand{\dist}{\xi}
\newcommand{\alg}{\mathsf{A}}
\newcommand{\qu}{\mathbf{q}}
\newcommand{\ex}{\mathbf{x}}
\newcommand{\ti}{\mathbf{t}}
\newcommand{\sdt}{\mathsf{SDT}}
\newcommand{\wdt}{\mathsf{WDT}}
\newcommand{\Qu}{\mathbf{Q}}
\newcommand{\pe}{\mathbb{P}}
\newcommand{\megam}{\mathcal{M}}
\newcommand{\eps}{\varepsilon}
\newcommand{\enet}{{$\varepsilon$-{\bf net}}\xspace}
\newcommand{\net}{{\tt net}\xspace}
\newcommand{\vcd}{VC-dimension\xspace}
\newcommand{\at}[1]{{\tt \small #1}\xspace}
\newcommand{\pr}{Pr}

\newcommand{\sharpP}{\mbox{\#P}}
\newcommand{\NP}{\mathsf{NP}}
\newcommand{\LP}{\mathsf{LP}}
\newcommand{\IP}{\mathsf{IP}}
\newcommand{\ru}{{\sc {RU}}\xspace}
\newcommand{\sru}{{\sc {strongRU}}\xspace}
\newcommand{\wru}{{\sc {weakRU}}\xspace}

\newcommand{\fmsystem}{{\sc Chameleon}\xspace}
\newcommand{\fm}{$\mathcal{F}$\xspace}

\newtheorem{experiment}{Experiment}

\begin{document}

\title{Coverage-based Data-centric Approaches for \\Responsible and Trustworthy AI\thanks{This research was supported by the National Science Foundation under grant No. 2107290.}}

\author{
\begin{tabular}[t]{c@{\extracolsep{2.4em}}c@{\extracolsep{2.4em}}c@{\extracolsep{2.3em}}c} 
Nima Shahbazi & Mahdi Erfanian & Abolfazl Asudeh \\ 
University of Illinois Chicago & University of Illinois Chicago & University of Illinois Chicago\\
 nshahb3@uic.edu & merfan2@uic.edu & asudeh@uic.edu
\end{tabular}
}

\maketitle


\begin{abstract}
The grand goal of data-driven decision systems is to help make decisions easier, more accurate, at a higher scale, and also just. However, data-driven algorithms are only as good as the data they work with. Yet, data sets, especially those with social data, often do not represent minorities. The paucity of training data is a perpetual problem for AI, and the outcome of ML models for cases not represented in their training data is often not reliable. 
Hence, without properly addressing the lack of representation issues in data, we cannot expect AI-based societal solutions to have responsible and trustworthy outcomes. 

This paper focuses on data coverage as a data-centric approach for identifying and resolving misrepresentation of minorities in data.
To achieve this goal, we propose novel algorithms that (a) {\it identify} and {\it resolve} insufficient data coverage across data with different modalities and (b) use lack of representation information to generate data-centric {\it reliability warnings}.
 \end{abstract}
 
 %%%%%%%%%%%%%%%%%%%%%%%%%%%%%%%% INTRO  %%%%%%%%%%%%%%%%%%%%%%%%%%%%%%%%
\section{Introduction}\label{sec:intro} % Abstract+Intro: up to 2.5 pages 
Data-driven decision-making has shaped every corner of human life, spanning from autonomous vehicles to healthcare and even predictive policing and criminal justice. A pivotal concern, especially in applications that affect individuals, revolves around the reliability of the decisions rendered by the system.
It is easy to see that the accuracy of a data-driven decision depends, first and foremost, on the data used to make it. Essentially, the system learns the phenomena that data represent. While we may desire that the data should represent the underlying data distribution from which the production data is drawn, this alone may be insufficient, as it merely enables the model to perform well for the average case.
As a result, a model with a high accuracy could fail for specific regions in the data with insufficient representation. These regions may matter because they frequently represent some minority population in society. They could also represent cases that may not happen very often but have a relevant impact on the correctness of a critical decision.
In short, if the data fails to sufficiently represent a specific population, the outcome of the decision system for that population may not be trustworthy.

The phenomenon known as \textit{Representation Bias} can arise from how the data was originally collected, or it could be the result of biases introduced post-collection—whether historically, cognitively, or statistically.

Representation bias is essentially inevitable without a systematic approach to data collection. 
For example, in the context of survey data collection, vital steps involve identifying all populations within the underlying distribution based on desired demographic information and ensuring comprehensive coverage with sufficient samples from each group. 
Even then, only an (uncontrolled) subset of the invitees will opt-in to respond to the survey.
Another challenge lies in the fact that data scientists often lack control over the data collection process, leading to the reliance on ``found data'' in the majority of data-driven systems. Therefore, with no guarantee on the aforementioned steps in the data collection process, the found data is most likely a biased sample.
Acknowledging the potential harms of representation bias, the notion of \textit{Data Coverage}~\cite{asudeh2019assessing,shahbazi2023representation} has been proposed to ensure the adequate representation of minority groups in data sets employed for decision-making and developing sophisticated data science tools. 

Addressing representation issues in data poses various challenges depending on the modality of the data. In this paper, we focus on identifying and resolving lack of coverage issues in data with different modalities.
We start by proposing a variety of techniques (spanning from geometric and combinatorial optimization to crowd-souring) aimed at efficiently detecting insufficient coverage on structured data sets with non-ordinal categorical and continuous attributes, as well as image data sets. Next, we propose a range of approaches grounded in data integration and generative data augmentation to address the lack of coverage by enriching the data sets with more data. However, with limited control over the data collection processes, it could be difficult and expensive to resolve all misrepresentations. 
Since adding more data is not always possible, we proceed to introduce data-centric preventive solutions that warn the user about the reliability of their predictions regarding representation bias issues. These warnings assist users in determining whether they trust the outcomes of the models or exercise caution. 

 %%%%%%%%%%%%%%%%%%%%%%%%%%%%%%%% IDENTIFICATION  %%%%%%%%%%%%%%%%%%%%%%%%%%%%%%%%
\section{Detecting Insufficient Representation of Minorities}\label{sec:identification} %up to 3.5 pages
Representation bias happens when the development (training data) population under-represents 
and subsequently fails to generalize well 
for some parts of the target population, due to historical bias, sampling bias, etc.
The notion of {\it data coverage} has been studied across different settings in \cite{shahbazi2023representation} as a metric to measure representation bias. At a high level, coverage is referred to as having enough similar entries for each object in a data set. 
For a better understanding, let us go over the definition of the generalized notion of coverage:

\begin{definition}[Data Coverage]\label{def:coverage}
Consider a data set $\dee$ with $n$ tuples, each consisting of $d$ attributes of interest $\mathbf{x}=\{x_1, x_2, \cdots,x_d\}$, such as {\tt gender}, {\tt race}, {\tt salary}, {\tt age}, etc, that are used for coverage identification.
The data set also contains target attributes $\mathbf{y} = \{ y_1,\cdots,y_{d'}\}$ that may or may not be considered for the coverage problem.
A query point $q$ is not covered by the data set $\dee$, if there are not ``enough'' data points in $\dee$ that are representative of $q$.
To generalize the notion of coverage, let us define $\gee(q)$ as the universe of tuples that would represent $q$ and let $\gee_\dee(q) = \gee(q)\cap \dee$. In other words, $\gee_\dee(q)$ are the set of tuples in $\dee$ that represent $q$.
Using this notation, we define the coverage of $q$ as the size of $\gee_\dee(q)$. That is,
$cov(q,\dee) = | \gee_\dee(q)|$.
Given a value $\tau$, $q$ is covered if $cov(q,\dee)>\tau$.
Similarly, a group $\gee$ is not covered if $\gee\cap \dee<\tau$.
The {\it uncovered region} in a data set is the collection of groups that are not covered by it.
\end{definition}

\subsection{Structured Data}
In this section, we focus on identifying representation bias in structured data.
Depending on the type of the attributes of interest, we categorize the techniques into two classes based on whether they target the problem for non-ordinal {\it categorical} (e.g. {\tt race}, {\tt gender}) or ordinal {\it continuous} (e.g. {\tt age}) attributes. The attributes of interest considered for representation bias often include sensitive attributes such as {\tt race} and {\tt gender} but are not necessarily limited to them.

\subsubsection{Categorical Attributes}

For cases where attributes of interest are non-ordinal categorical,
the cartesian product of values on a subset of attributes $\mathbf{x}'\subseteq \mathbf{x}$, form a set of (sub-)groups.
For example, $\{$ {\tt white male}, {\tt white female}, {\tt black male} $,\cdots\}$ are the subgroups defined on the attributes {\tt (race,gender)}.
We refer to the number of attributes used to specify a subgroup as the {\it level} of that subgroup.
For example, the level of the subgroup {\tt white male} is 2, while the level of the subgroup {\tt male} is 1.
We use $\ell(\gee)$, to refer to the level of a subgroup $\gee$.
Similarly, we say a subgroup $\gee'$ is a subset of $\gee$, if the groups specifying $\gee'$ are a superset of the ones for $\gee$. For example {\tt (married white male)} a subset of the more general group {\tt (white male)}. That is, the set of individuals in group {\tt (married white male)} are a subset of {\tt (white male)}.
Moreover, we say a subgroup $\gee$ is a {\it parent} of the subgroup $\gee'$, if $\gee'\subset \gee$ and $\ell(\gee)=\ell(\gee')+1$. For example, the subgroup {\tt (white male)} is a parent of the subgroup {\tt (married white male)}.
We use \textit{patterns} to refer to uncovered subgroups.
A pattern $P$ is a string of $d$ values, where $P[i]$ is either a value from the domain of $x_i$, or it is ``unspecified'', specified with $X$. 
For example, consider a data set with three binary attributes of interest $\mathbf{x}=\{x_1, x_2, x_3\}$. The pattern $P=X01$ specifies all the tuples for which $x_2=0$ and $x_3=1$ ($x_1$ can have any value).
The set of patterns that identify most general uncovered subgroups are called {\it Maximal Uncovered Patterns} (MUPs).

No polynomial time algorithm can guarantee the enumeration of the entire MUPs, however, several algorithms inspired by set enumeration and the Apriori algorithm for association rule mining are proposed to efficiently address this problem~\cite{asudeh2019assessing}.
In this regard, we introduce \textit{Pattern Graph} data structure that exploits the relationship between patterns to do less work than computing all uncovered patterns by removing the non-maximal ones. 
The parent-child relationship between the patterns is represented in a graph that can be used to find better algorithms. 
\textit{Pattern-Breaker} starts from the top of the graph where the general patterns are and moves down by breaking each pattern into more specific ones. If a pattern is uncovered, then all of its descendants are also uncovered and they can not be an MUP, even if they have a parent that is covered. Therefore, this subgraph of the pattern graph can be pruned. 
The issue with \textit{Pattern-Breaker} is that it explores the covered regions of the pattern graph and for the cases where there are a few uncovered patterns, it has to explore a large portion of the exponential-size graph. 
To tackle this, \textit{Pattern-Combiner} algorithm is proposed that performs a bottom-up traversal of the pattern graph. It uses an observation that the coverage of a node at the level of the pattern graph can be computed as the sum of the coverage values of its children. 
The problem with \textit{Pattern-Combiner} is that it traverses over the uncovered nodes first and therefore, it will not perform well for the cases in which most of the nodes in the graph are uncovered. 
In fact, for the cases where most of the MUPs are placed in the middle of the graph, both \textit{Pattern-Breaker} and \textit{Pattern-Combiner} will not be as efficient as they should traverse half of the graph. Therefore, we propose \textit{Deep-Diver}, a search algorithm based on Depth-First-Search that quickly finds the MUPs, and uses them to limit the search space by pruning the nodes both dominating and dominated by the discovered MUPs.

\begin{figure*}[!tb]
    \begin{minipage}[t]{0.31\linewidth}
        \centering
        \includegraphics[width=\textwidth]{submissions/submission1/shahbazi/covcube1.jpg}
        \caption{\small Categorical attributes: the uncovered region of a toy example, as the collection of three MUPs.}
        \label{fig:covcube1}
    \end{minipage}
    \hfill
    \begin{minipage}[t]{0.31\linewidth}
        \centering
        \includegraphics[width=\textwidth]{submissions/submission1/shahbazi/cvrg_2_1.jpg}
        \caption{\small Continuous attributes, 2D: identifying the covered region in the gray Voronoi cell.}
        \label{fig:cvrg_2_1}
    \end{minipage}
    \hfill
    \begin{minipage}[t]{0.31\linewidth}
        \centering
        \includegraphics[width=\textwidth]{submissions/submission1/shahbazi/cvrg_2_2.jpg}
        \caption{ \small Continuous attributes, 2D: Uncovered region marked in red.}
        \label{fig:cvrg_2_2}
    \end{minipage}
\vspace{-5mm}
\end{figure*}

\subsubsection{Continuous Attributes}
Data in the real world often consists of a combination of continuous and discrete values. While simple solutions like binning {\tt age} into {\tt young} and {\tt old} can transform the continuous space into discrete. However, they may lead to coarse groupings that are sensitive to the thresholds chosen. It may be inappropriate to treat a 35-yo as {\tt young} but a 36-yo as {\tt old}. 
Therefore, we extend the notion of coverage to continuous space. Particularly, given data set $\dee$ with $n$ tuples over $d$ attributes, and vicinity radius $\rho$ and coverage threshold $k$, we want to identify the uncovered region -- the universe of uncovered query points.
A query point in continuous data space is covered if there are enough (at least $k$) data points in its $\rho$-vicinity neighborhood. $\rho$-vicinity neighborhood is the circle centered at the query point with radius $\rho$.

Depending on the number of attributes in a data set, we propose two algorithms for identifying uncovered regions in data~\cite{asudeh2021coverage}. 
The first algorithm known as \textit{Uncovered-2D} studies coverage over two-dimensional data sets where $\mathbf{x}=\{x_1,x_2\}$. To find the number of circles that a query point falls into and consequently discover the uncovered region, \textit{Uncovered-2D} makes a connection to $k$-th order Voronoi diagrams.
Consider a data set $\mathcal{D}$ and its corresponding $k$-th order Voronoi diagram. For every tuple $t\in \mathcal{D}$, let $\circ_t$ be the $d$-dimensional sphere ($d$-sphere) with radius $\rho$ centered at $t$.
Consider a $k$-voronoi cell $\mathcal{V}(S)$ in the $k$-th order Voronoi diagram $V_k(\mathcal{D})$.
Any point $q$ inside the intersections of the $d$-spheres of tuples in $S$, i.e. $q\in \underset{\forall t\in S}{\cap ~\circ_t}$, is covered, while all other points in the region are uncovered.
 The algorithm starts by constructing the $k$-th order Voronoi diagram of the data set and then for each Voronoi cell $\mathcal{V}(S)$ in the diagram, it computes the intersection of the circles of the tuples in $S$ and marks the portion of $\mathcal{V}(S)$ that falls outside it as uncovered.
After identifying the uncovered region, a 2D map of $\{x_1,x_2\}$ value combinations is used to report the region to the user.
The algorithm for the 2D case can be extended to the general case by relaxing the assumption on the number of attributes to discover the exact uncovered region, however, due to the curse of dimensionality, the search size space explodes as the number of dimensions increases and as a result, the algorithm will not be practical. Therefore, we propose a randomized approximation algorithm based on the geometric notion of \enet. 
Let $\mathcal{X}$ be a set and $\mathcal{R}$ be a set of subsets of $\mathcal{X}$. A set $\mathcal{N}\subset \mathcal{X}$ is an \enet for $\mathcal{X}$ if for any range $r\in\mathcal{R}$, if  $|r\cap \chi|>\eps|\chi|$, then $r$ contains at least one point of $N$.
The idea, at a high level, is to draw enough random samples from the space of potential query points to form an \enet. 
We then label the sampled query points as $\{-1,+1\}$ depending on whether those are covered or not, and learn the uncovered regions using the samples.

\subsection{Image Data}
Many known incidents of machine failures due to the lack of representation were on image data.
We consider an image data set with a fixed number of low-cardinality sensitive attributes such as {\tt\small race} and {\tt\small gender}. 
It is common that image data sets {\it lack explicit values} for sensitive attributes, which are crucial for coverage identification. An image data set is often a collection of images from different domains with little to no information about their domain and which groups they belong to. As a result, even studying coverage over low-cardinality and categorical attributes of interests is challenging in these cases.

\begin{wrapfigure}{R}{0.42\textwidth}
\centering
\vspace{-3mm}
\scriptsize
\begin{tabular}{|@{}c|@{}c@{}|@{}c@{}|@{}c@{}|} 
 \hline
{\bf data set} & {\bf classifier} & {\bf accuracy} & {\bf precision} \\ 
 &  &  & {\bf on female} \\ \hline
UTKFace:~& DeepFace (opencv) & 93.56 & {52.02}\\\cline{2-4}
({\tt females}=200,& DeepFace (retinaface) & 94.16 & {56.15}\\\cline{2-4}
{\tt males}=2800) & BaseCNN & 97.6 & 74.8\\
\hline
UTKFace:~& DeepFace (opencv) & 96.53 & {\bf 8.0}\\\cline{2-4}
({\tt females}=20,& DeepFace (retinaface) & 96.43 & {\bf 10.09}\\\cline{2-4}
{\tt males}=2980)& BaseCNN & 97.6 & {\bf 21.59}\\
\hline
\end{tabular}
\vspace{-3mm}
\caption{\small ML models' low performance for females in the presence of representation bias.~\cite{mousavi2024data}}\label{fig:mlfails}
\vspace{-3mm}
\end{wrapfigure}

In Figure~\ref{fig:mlfails}, we show that due to the issues such {\it machine bias} and {\it lack of distribution generalizability},
solely relying on state-of-the-art machine learning (ML) techniques fail to effectively identify lack of coverage in image data sets. Therefore, we propose an approach based on combining crowdsouring with ML~\cite{mousavi2024data}. 
Crowdsourcing is particularly promising for image data, for tasks such as image labeling, which, while challenging for the machine, are "easy" for human beings to conduct with minimal error. 

A key observation that enables a cost-effective crowdsourcing approach is that, while studying coverage, we would only like to find out if there are {\it enough tuples from each subgroup}.
Suppose a subgroup is covered if there are $\tau=100$ instances of it in the data set. Assume the (majority) group $\gee_1$ contains $n_1 \gg 100$ objects in the data set. 
To verify that $\gee_1$ is covered, it is enough for the crowd to discover 100 of those objects, not the entire $n_1$. 
Following this, $O(\tau)$ provides a lower bound on the number of crowd tasks required to verify a given group is covered. 
Still, this lower bound only holds for the groups that are covered, i.e., there is at least $\tau$ of those in the data set.
Surprisingly, verifying that a minority group is indeed uncovered is cumbersome, unlike the majority group.
This is because even though discovering $\tau$ objects from a group is enough for verifying that it is covered, one cannot {\it verify} a group is uncovered until there is a chance that the data set might still have enough objects from that group. Thus, assuming a non-zero probability for each unlabeled object to belong to each group, {one might need to ask the crowd to label the entire data set before they can confirm that a specific group is uncovered}.

Our idea for addressing this challenge is to
design {\it a divide and conquer algorithm} that, instead of {point queries}, uses {\it set queries} to iteratively eliminate subsets of data that {does not include any object from the given group}.
At a high level, our idea is to ask a set query from the crowd, inquiring whether the selected set contains at least one object from the given group $\gee$.
The user may provide two responses (yes/no). 
Interestingly, {in either case}, the user response provides valuable information that helps efficiently identify the coverage.
If the answer is ``No'', the set does not include any object from the given group $\gee$. As a result, the algorithm can safely prune the set, asking no further questions about it. In particular, for a group that is not covered, one can expect to see no answers on large set queries helping to prune a significant portion of the data set quickly.
On the other hand, if the answer is ``yes'', the set contains {at least} one object from the group $\gee$. As a result, the algorithm cannot prune the subset since it can have any number (larger than one) of the objects in $\gee$.
At first glance, the queries with yes answers do not provide helpful information as the algorithm cannot prune the subset (hence it needs to divide it into smaller subsets).
However, a key observation is that {the algorithm will only observe a limited number of yes answers} before it stops.
The reason is that the number of set queries with yes answers provides a {lower-bound} on the number of objects from $\gee$ in the data set. As a result, the algorithm can stop as soon as the lower bound reaches $\tau$, knowing that $\gee$ is covered.
The D\&C approach verifies the data coverage for a given group, while our goal is to identify the uncovered regions for a given set of sensitive attributes. The next question is how to utilize this algorithm for efficient coverage identification on different scenarios of sensitive attributes, forming intersectional or non-intersectional groups.
In particular, how can we find maximal uncovered patterns?
Our idea is to apply sampling and aggregate estimation techniques to find the groups that even if merged are likely to still be uncovered. This will help reduce the coverage identification cost by running the D\&C approach for the merged groups once.
 %%%%%%%%%%%%%%%%%%%%%%%%%%%%%%%% RESOLUTION  %%%%%%%%%%%%%%%%%%%%%%%%%%%%%%%%
\section{Resolving Insufficient Representation}\label{sec:resolution}

Data integration~\cite{nargesian2021tailoring,nargesian2022responsible} and data augmentation~\cite{sharma2020data,DBLP:journals/jair/ChawlaBHK02,iosifidis2018dealing,celis2020data} are considered as the primary solutions for reducing data coverage issues in a data set. 
Data integration is promising when external sources of data are available. On the other hand, recent advancements in generative AI and foundation models have enabled efficient and effective augmentation of data sets with synthetic data. 
Therefore, in the following, we review two approaches, one from each category, in the context of lack of coverage resolution.

\subsection{Data Integration}\label{sec:resolution:integration}

Data integration is to consolidate data from different sources into a single, unified view. 
Although it is an effective solution to acquire additional data from different distributions,
there are sampling policy and cost-efficiency concerns that need to be examined.  
Therefore, {\it Data Distribution Tailoring ({\sc DT})} introduces data integration techniques for resolving insufficient representation of subgroups in a data set in the most cost-effective manner~\cite{nargesian2021tailoring}.
A query to {\sc DT} 
consists of a target schema, and a set of group distribution requirements in the form of the minimum counts (e.g., ``{\tt\small 1,000 breast cancer monitoring data in Chicago with at least 30\% label=positive, and at least 20\% black patients}''). 
Collecting a fresh sample from a data view is costly (monetary, human resources, and/or computation cost)~\cite{asudeh2022towards}.
Therefore, {\sc DT} focuses on satisfying the count requirements with minimum cost. 
Given an input query and a lake of available data sources, the first step is to discover a collection of candidate data views that satisfy the target schema.
Each data view $v_i$ is a projection-join $v_i = \Pi\big(D_{i1}\bowtie\cdots\bowtie D_{ik_i} \big)$, where $D_{ij}$ is a data set in a given data lake.
Let us suppose the data views are already discovered.
At a high level, {\sc DT} follows an iterative approach that at each iteration a data view is selected to be queried.
Each query to a data view has a fixed cost and returns a sample that may or may not satisfy the query constraints.
The samples that are either not fresh, or do not satisfy the query are discarded.
Hence, the essential question towards a cost-effective data integration is {\it what data view to query next}.
Depending on the available information about the data sources, various techniques may be employed. 

For the cases when the group distributions are known, the process of collecting the target data set is a sequence of iterative steps, where at every step, the algorithm chooses a data view, queries it, and if the obtained tuple contributes to one of the groups for which the count requirement is not yet fulfilled, it is kept, otherwise discarded. To do so, a {Dynamic Programming (DP)} algorithm is proposed. An optimal source at each iteration minimizes the sum of its sampling cost plus the expected cost of collecting the remaining required groups, based on its sampling outcome.
The DP algorithm, however, has a pseudo-polynomial time complexity. Hence, it quickly becomes intractable for cases where the minimum count requirements for the groups are not small. 
For cases where the (sensitive) attribute of interest is binary, such as (biological) {\tt sex}={\tt \{male, female\}}, and the cost to query data is similar from all sources, it turns out that the optimal strategy is to query the data source with {maximum probability of obtaining a sample from the minority group}.
Expanding the binary-attributes algorithm for non-binary cases, the problem can be modeled as an extension of the ``{\it coupon collector's}'' problem~\cite{motwani1995randomized}, where the goal is to collect $m_i$ instances from each coupon (group) $\gee_i$.
At each iteration, the coupon collector's algorithm identifies a data view as most promising and queries it. In simple terms, a data view with a smaller query cost and a higher chance of obtaining minority groups is more promising.


For the cases where the group distributions are unknown, we model DT as a {\it multi-armed bandit} problem, where every data view is modeled as an arm. 
Every arm has an unknown distribution of different groups while pulling an arm (i.e., querying the corresponding data view) has a cost.
During various iterations, the algorithms pull the arms in an order that its expected total {\it reward} is maximized.
Arguing that the reward of obtaining a tuple from a group is proportional to how rare this group is across different data views, 
we design the reward function based on the expected cost one needs to pay in order to collect a tuple from a specific group.  
As the bandit strategy, we adopt {\it Upper Confidence Bound (UCB)} to balance exploration and exploitation. At every iteration, for every arm, UCB computes confidence intervals for the expected reward and selects the arm with the maximum upper bound of reward to be explored next.

\subsection{Data Augmentation using Foundation Models}

While data integration provides a promising approach for resolving coverage issues in a data set, its effectiveness is limited to the availability of external data sources that are rich enough to find sufficient fresh samples from minority groups. This, however, is not always possible, especially since the minority samples are rare and not easy to obtain.
Fortunately, recent advancements in Generative AI and Foundation Models have enabled synthesizing samples that are otherwise challenging to obtain from the real world.

Therefore, as an alternative approach to data integration, we turn our attention to the Foundation Models and Generative AI for resolving the lack of coverage. 
Particularly, models such as {\sc DALL.E}\footnote{\url{https://openai.com/dall-e-2}} have emerged as powerful tools for generating multi-modal data such as image, audio, and video.
 
We formalize the foundation model \fm as a black-box function with the following inputs, that once queried synthesize an output tuple.
\begin{itemize}
    \item {\bf Prompt}: A natural language description providing instructions on the details of the tuple to be generated. For instance, a prompt for image generation might be ``A realistic photo of a white cat running in a backyard.''
    \item {\bf Guide}: In cases where only a prompt is provided, the foundation model uses its imagination to generate the requested tuple. For the previous example, the prompt of a cat image, the breed, size, background, and other details are generated based on the model's imagination. Alternatively, a guide can be provided to influence the generation process. The guide is formalized as a pair $(t,m)$ where $t$ is a tuple and $m$ is a mask specifying which parts of the guide tuple should be changed. Using the cat example, $t$ can be a cat image and $m$ can specify the foreground to be regenerated.
\end{itemize}

There are multiple challenges towards effective data set augmentations using foundation models. 
First, we have to determine the minimal set of synthetic tuples that once added to the original data set, under-representation issues are resolved.
Second, the generated images should follow the underlying distribution represented in the input data set. Third, the generated tuples should have high quality and look realistic to a human evaluator. Last but not least, given the (often monetary) cost associated with the queries to the foundation model, we should ensure the cost-effectiveness of the data set repair process.

\begin{wrapfigure}{L}{0.45\textwidth}
\centering
\vspace{-3mm}
\scriptsize
    \includegraphics[width=.45\textwidth]{submissions/submission1/shahbazi/enhanced_pipeline.png}
\vspace{-3mm}
\caption{\small Architecture of \fmsystem for image data augmentation for coverage enhancement.}\label{fig:chameleon}
% \vspace{-3mm}
\end{wrapfigure}

\noindent Figure~\ref{fig:chameleon} shows the architecture of our system \fmsystem \cite{chameleon} for coverage enhancement using DALL-E image generator.
To address the first challenge, we define the combinations-selection problem, which minimizes the total number of synthetic tuples for resolving lack of coverage of minorities at the most general level. We show the problem is {\sc NP}-hard, and propose a greedy approximation algorithm for it.
To address the second and third challenges, \fmsystem follows a {\it rejection sampling} strategy.
It views each tuple in the data set $\dee$ as an iid sample from the underlying distribution $\xi$ it represents. It uses the vector representations (embeddings) space to describe the distribution. Then, given a newly generated tuple, it employs the one-class support vector machine (OCSVM) approach proposed by Scholkopf et al.~\cite{scholkopf1999support} to reject the tuple if it does not follow $\xi$.
Moreover, it models the quality evaluation as hypothesis testing and rejects the samples that have a higher chance of being labeled as ``unrealistic'' by a random human evaluator.
Finally, to minimize the number of queries to the foundation model, we provide a guide tuple (and a mask), in addition to the prompt, to the foundation model. We model the guide-selection problem as {\it contextual multi-armed bandit} and propose a solution based on the contextual UCB for it.

Before concluding this section, let us provide some experiment results to demonstrate the effectiveness of data augmentation with \fmsystem. We use FERET DB \cite{phillips1998feret} for this experiment, which comprises 1199 individual images and serves as a standardized facial image database for researchers to develop algorithms and report results. All images in FERET DB share the same dimensions, pose, and facial expression.
First, we identified the (level-1) uncovered ethnicity groups, using the threshold 80. We then used \fmsystem and resolved the lack of coverage issues.
To evaluate the effectiveness of the system, we trained a CNN model to predict the race of each image within this dataset. We then retrained the identical CNN on the repaired training data. Importantly, our test dataset for both experiments remains consistent and is derived from real images.
Table~\ref{tab:lackofcoverage} presents the improvements in precision, recall, and F1 score metrics for under-represented groups after repairing the dataset. The results indicate an enhancement in performance metrics for all under-represented groups following the repair process.

\begin{table}[t]
    \centering
    \caption{Illustrating the effect of lack of coverage repair using \fmsystem on \texttt{FERTDB}}
    \label{tab:lackofcoverage}
    \vspace{-3mm}
    \begin{tabular}{lcccccccc}
        \toprule
         & \multicolumn{4}{c}{\textbf{Classifier Performance on \texttt{FERTDB}}} & \multicolumn{4}{c}{\textbf{Classifier Performance on Repaired}} \\
        \cmidrule(lr){2-5} \cmidrule(lr){6-9}
        \textbf{Ethnicity Groups}& \#Images & Precision & Recall & F1-Score & \#Images & Precision & Recall & F1-Score \\
        \midrule
        Overall          & 756 & 0.81 & 0.75 & 0.78 & 987 & 0.70 & 0.75 & 0.72 \\ \hline
        Black            & 40  & 0.19 & 0.22 & 0.16 & 100 & 0.48 & 0.56 & 0.52 \\
        Hispanic         & 19  & 0.50 & 0.17 & 0.25 & 100 & 0.62 & 0.36 & 0.45 \\
        Middle Eastern   & 10  & 0.00 & 0.00 & 0.00 & 100 & 0.20 & 0.41 & 0.27 \\
        \bottomrule
    \end{tabular}
\end{table}

 %%%%%%%%%%%%%%%%%%%%%%%%%%%%%%%% RELIABILITY  %%%%%%%%%%%%%%%%%%%%%%%%%%%%%%%%
\section{Generating Reliability Warnings}\label{sec:reliability}
% up to 2.5 pages
Interpretability is a necessity for data scientists who develop predictive models for critical decision-making.
In such settings, it is important to provide additional means to support the following question:
{\it is an individual prediction of the model reliable for decision-making?} Our goal is to use the lack of representation to help decision-makers find insights about this critical question.
To further motivate this, let us use the following example:

\vspace{1mm}
\begin{example}\label{ex-0}
{\bf(Part1):} Consider a judge who needs to decide whether to accept or deny a bail request. Using data-driven predictive models is prevalent in such cases for predicting recidivism~\cite{dressel2018accuracy}.
Indeed, such models can be beneficial to help the judge make wise decisions.
Suppose the model predicts the queried individual as high risk (or low risk).
The judge is aware and concerned about the critics surrounding such models.
A major question the judge faces is whether or not they should rely on the prediction outcome to take action for this case.
Furthermore, if, for instance, they decide to ignore the outcome and hence they need to provide a statement supporting their action, what evidence can they provide? 
\end{example}

In line with the recent trend on data-centric AI~\cite{ng2021mlops}, we design {novel approaches}, {complimentary} to the existing work on trustworthy AI~\cite{wing2021trustworthy,kentour2021analysis,liu2021trustworthy,singh2021trustworthy}, to address the aforementioned trust question through the lens of {\it data}.
In particular, unlike existing works that generate trust information from a {\it given \underline{model}}, we associate {\it \underline{data sets} with proper measurements} that specify their {\it the scope of use for predicting future cases}.
We note that a predictive model provides only probabilistic guarantees on the \underline{average} loss over the distribution represented by the data set used for training it.
As a result, these predictions may not be distribution generalizable~\cite{kulynych2022you}.
Consequently, if the query point is {\it not represented} by the data, the guarantees may not hold, hence one cannot rely on the prediction outcome.
Besides, an essential requirement for a learning algorithm is that its training data $\dee$ should represent the underlying distribution $\dist$.
Even if so, the trained model $h$ only provides a probabilistic guarantee on the {expected} loss on random samples from $\dist$.  
A model that performs well on {\it majority} of samples drawn from $\dist$ will have a high performance on average. Still, as we observed in Figure~\ref{fig:mlfails},
its performance for {\it minorities} and points that are not represented is questionable. Let us consider the following toy example:

\begin{figure*}[!b] 
    \begin{minipage}[t]{0.32\linewidth}
        	\centering
        	\includegraphics[width=\textwidth]{submissions/submission1/shahbazi/example_1.png} 
        	\vspace{-9mm}\caption{\small Data set $\dee$ generated using a Gaussian distribution; $x_1$ and $x_2$ are positively correlated}
            \label{fig:ex1:1}
    \end{minipage}
    \hfill
    \begin{minipage}[t]{0.32\linewidth}
        \centering
        	\includegraphics[width =\textwidth]{submissions/submission1/shahbazi/example_2.png} 
        	\vspace{-9mm}\caption{\small The decision boundary of learned model $h$ and query points $\qu^1$ to $\qu^4$}
            \label{fig:ex1:2}
    \end{minipage}
    \hfill
    \begin{minipage}[t]{0.32\linewidth}
        	\centering
        	\includegraphics[width =\textwidth]{submissions/submission1/shahbazi/example_3.png}
        	\vspace{-9mm}\caption{\small Ground-truth boundary, overlaid on the model decision boundary and query points}
            \label{fig:ex1:3}
    \end{minipage}
    \vspace{-5mm}
\end{figure*} 

\vspace{1mm}
\begin{example}\label{ex-1}
Consider a binary classification task where the input space is $\ex=\langle x_1, x_2\rangle$ and the output space is the binary label $y$ with values $\{-1$ (red) $,+1$ (blue)$\}$.
Suppose the underlying data distribution $\dist$ follows a 2D Gaussian, where $x_1$ and $x_2$ 
are positively correlated as shown in Figure~\ref{fig:ex1:1}.
The figure shows the data set $\dee$ drawn independently from the distribution $\dist$, along with their labels as their colors.
Using $\dee$, the prediction model $h$ is constructed as shown in Figure~\ref{fig:ex1:2}. 
The decision boundary is specified in the picture; while any point above the line is predicted as +1, a query point below it is labeled as -1.
The classifier has been evaluated using a test set that is an iid sample set drawn from the underlying data set $\dist$. The accuracy on the test set is high (above 90\%), and hence, the model gets deployed.
We cherry-picked four query points, $\qu^1$ to $\qu^4$, that are also included in Figure~\ref{fig:ex1:2}. Using $h$ for prediction, $h(\qu^1)=-1$, $h(\qu^2)=+1$,  $h(\qu^3)=+1$, and $h(\qu^4)=-1$.
Figure~\ref{fig:ex1:3} adds the ground-truth boundary to the search space, revealing the true label of the query points: every point inside the red circle has the true label $-1$ while any point outside of it is $+1$.
Looking at the figure, $y^1=+1$ while the model predicted it as $h(\qu^1)=-1$.  \hfill$\square$
\end{example}
\vspace{2mm}

Let us take a closer look at the four query points in this example and their placement with regard to the tuples in $\dee$ used for training $h$. 
$\qu^2$ belongs to a {\it dense region} with many training tuples in $\dee$ surrounding it. Besides, all of the tuples in its vicinity have the same label $y=+1$. As a result, one can expect that the model's outcome $h(\qu^2)=+1$ should be a reliable prediction.
Similar to $\qu^2$, $\qu^4$ also belongs to a dense region in $\dee$; however, $\qu^4$ belongs to an {\it uncertain region}, where some of the tuples in its vicinity have a label $y=+1$, and some others have the label $y=-1$. Considering the uncertainty in the vicinity of $\qu^4$, one cannot confidently rely on the outcome of the model $h$. 
On the other hand, the neighbors of $\qu^1$ (resp. $\qu^3$) are not uncertain, all having the label $y=-1$ (resp. $y=+1$).
However, the query points $\qu^1$ and $\qu^3$ are not well represented by $\dee$. In other words, $\qu^1$ and $\qu^3$ are unlikely to be generated according to the underlying distribution $\dist$, represented by $\dee$. As a result, following the no-free-lunch theorem~\cite{kakade2003sample}, one cannot expect the outcome of model $h$ to be reliable for these points.
Looking at the ground-truth boundary in Figure~\ref{fig:ex1:3}, $h$ luckily predicted the outcome for $\qu^3$ correctly, but it was not fortunate to predict the $y^1$ correctly.
Nevertheless, 
since the model is not reliably trained for these points, 
its outcome for these query points is not trustworthy.

From Example~\ref{ex-1}, we observe that the outcome of a model $h$, trained using a data set $\dee$ is not reliable for a query point $\qu$, if:
\begin{itemize}
    \item {\bf Lack of representation:} $\qu$ is not well-represented by $\dee$.
    In such cases, the model has not seen ``enough'' samples similar to $\qu$ to reliably learn and predict the outcome of $\qu$.
    \item {\bf Lack of certainty:} $\qu$ belongs to an uncertain region, where different tuples of $\dee$ in the vicinity of $\qu$ have different target values. $\qu$ belongs to a high-fluctuating area, where tuples in the vicinity of $\qu$ have a wide range of values.
\end{itemize} \vspace{2mm}

\noindent
Based on these two observations, we propose Representation-and-Uncertainty ({\bf RU}) measures.
To identify if a query suffers from uncertainty or lack of representation, one could use a deterministic approach using a fixed threshold. Then if the number of similar samples to (resp. label fluctuation in vicinity of) $\qu$ is larger than the threshold it is considered as unrepresented (resp. uncertain).
This approach, however, would be misleading since two numbers close to the threshold could be treated very differently. Also, all points on each side of the threshold would be considered equally represented (resp., certain). Instead, we consider {\it a randomized approach}, widely popular in the literature, including~\cite{dwork2012fairness}.
That is, instead of using fixed thresholds, a Bernoulli variable (a biased coin) is used that 
assigns $\qu$ as unrepresented (resp., uncertain) based on the number of samples similar to it (resp., its neighborhood uncertainty).
Given a query point $\qu$, let $\pe_o$ be the probability indicating if $\qu$ is not represented and let $\pe_u$ be the probability indicating if $\qu$ belongs to an uncertain region. 
We represent the probability of the Bernoulli variables for lack of representation or uncertainty components as $\pe_o$ and $\pe_u$, respectively. Note that the two Bernoulli variables $\pe_o$ and $\pe_u$ are independent from each other. That simply follows the argument that after specifying the number of similar samples to $\qu$ whether or not it should be considered as unrepresented does not depend on the uncertainty in the neighborhood of $\qu$.

\begin{definition}[\sru]\label{def:sdt}
The \sru is a probabilistic measure that considers the outcome of a model for a query point $\qu$ untrustworthy if $\qu$ is not represented by $\dee$ {\it and} it belongs to an uncertain region.
Formally, the \sru measure is:
\begin{align} 
    \nonumber
    SRU(\qu) &= \pe\big((\qu \mbox{ is outlier}) \wedge (\qu \mbox{ belongs to uncertain region})\big) 
\end{align}
Since $\pe_o$ and $\pe_u$ are independent:

\vspace{-13mm}
\begin{align} \label{eq:strong}
    SRU(\qu) &= \pe_o(\qu) \times \pe_u(\qu)
\end{align}
\end{definition}

\sru raises the warning signal only when the query point fails on {\it both} conditions of being represented by $\dee$ and not belonging to an uncertain region. 
For instance, in Example~\ref{ex-1} none of the query points fail both on representation and on uncertainty; hence neither has a high \sru score.
On the other hand, 
a high \sru score for a query point $\qu$ {\it provides a strong warning signal} that one should perhaps reject the model outcome and not consider it for decision-making.

\sru is a strong signal that raises warnings only for the fearfully concerning cases that fail both on representation and uncertainty.
However, as observed in Example~\ref{ex-1} a query points failing {\it at least} one of these conditions may also not be reliable, at least for critical decision making.
We define the \wru measure to raise a warning for such cases.

\begin{definition}[\wru]\label{def:wdt}
The \wru measure is a probabilistic measure that considers the outcome of a model for a query point $\qu$ untrustworthy if $\qu$ is not represented by $\dee$ {\bf or} it belongs to an uncertain region.
Formally, the \wru is computed as:
\begin{align} \label{eq:weak}
    WRU(\qu) = \pe\big((\qu \mbox{ is outlier}) \vee (\qu \mbox{ belongs to uncertain region})\big) 
    = \pe_o(\qu) + \pe_u(\qu) - \pe_o(\qu) \times \pe_u(\qu)
\end{align}
\end{definition}

Proposing quantitative probabilistic outcomes, \ru measures are interpretable for the users, since beyond the scores, the uncertainty and lack of representation components provide an explanation to justify them. 
Please refer to \cite{techrep} for more details on how to efficiently and effectively compute the representation ($\pe_o$) and uncertainty ($\pe_u$) probabilities, using only $\dee$.
In Example~\ref{ex-0}, let us see how the \ru measures can be helpful.

\noindent{\bf Example 1. (part 2):}
{\it RU measures \underline{raise warning} when
the fitness of the data set used for drawing a prediction is questionable, helping the judge to be cautious when taking action.
Besides, these measures provide \underline{quantitative evidence} to support the judge's action when they decide to ignore a prediction outcome that is not trustworthy.
The judge, for example, can argue to ignore a model outcome for a specific case, based on the insight that 
the model has been built using a
data set that fails to represent the given case.}
\hfill$\square$

Finally, let us demonstrate the efficacy of \ru measures through a series of experiments. Since the \ru measures are {\it data-centric},
those are applicable for both classification and regression tasks, irrespective of the model used.
We use {\it Adult} dataset~\cite{adult} for classification and {\it House Sales in King County} dataset for the validation of regression tasks. From each dataset, we uniformly sample two sets from the underlying distribution. The first set serves as the training set to compute the \ru values, and the second one is used as the test set from which the queries are drawn. We validate our proposal by providing the correlation between the \ru values and the performance of an ML model's prediction on the same data. 

We start by computing the \ru values for all the query points in the test set. Next, we bucketize the query points based on their \ru values in equi-width buckets of width 0.1. We repeat this for both \sru and \wru measures. Next, we train a model on the training data set and predict the target variable for the points in each range of \ru measure. The validation results for the classification task on the {\it Adult} dataset are presented in Figures \ref{fig:exp-adult-sdt} and \ref{fig:exp-adult-wdt}. Each figure corresponds to the accuracy/error measures of the classifier over each bucket of \ru values for \sru and \wru. As the \ru values increase, the accuracy of the model drops while the FPR rises, and therefore, the model fails to capture the ground truth for the points that fall into untrustworthy regions in the data set. By repeating the aforementioned steps for the regression task on the {\it House Sales in King County} dataset, we observe similar results presented in Figures \ref{fig:exp-hs-sdt} and \ref{fig:exp-hs-wdt}. 
As the \ru value increases, the RSS of the regression model follows the same trend denoting that the model fails to perform for tuples with a high \ru value.

\begin{figure}[!tb]
    \begin{minipage}[t]{0.24\linewidth}
        \centering
        \includegraphics[width=\textwidth]{submissions/submission1/shahbazi/sdt_adult.pdf}
        \vspace{-6mm}\caption{\small{\it Adult}, efficacy of \sru  on classification}
        \label{fig:exp-adult-sdt}
    \end{minipage}\hfill
    \begin{minipage}[t]{0.24\linewidth}
        \centering
        \includegraphics[width=\textwidth]{submissions/submission1/shahbazi/wdt_adult.pdf}
        \vspace{-6mm}\caption{\small{\it Adult}, efficacy of \wru  on classification}
        \label{fig:exp-adult-wdt}
    \end{minipage}\hfill
    \begin{minipage}[t]{0.24\linewidth}
        \centering
        \includegraphics[width=\textwidth]{submissions/submission1/shahbazi/sdt_regression_house.pdf}
        \vspace{-6mm}\caption{\small{\it House Sales in King County}, efficacy of \sru on regression}
        \label{fig:exp-hs-sdt}
    \end{minipage}\hfill
    \begin{minipage}[t]{0.24\linewidth}
        \centering
        \includegraphics[width=\textwidth]{submissions/submission1/shahbazi/wdt_regression_house.pdf}
        \vspace{-6mm}\caption{\small{\it House Sales in King County}, efficacy \wru on regression}
        \label{fig:exp-hs-wdt}
    \end{minipage}
\vspace{-5mm}
\end{figure}
 %%%%%%%%%%%%%%%%%%%%%%%%%%%%%%%% RELATED WORK  %%%%%%%%%%%%%%%%%%%%%%%%%%%%%%%%
\section{Related Work}\label{related} 

Bias in data has been looked at for a long time in statistical community~\cite{neyman1936contributions} but social data presents different challenges~\cite{olteanu2019social,fairmlbook,barocas2016big,jk2019bias,drosou2017diversity}.
The diversity and representativeness of data have been widely studied~\cite{drosou2017diversity}, in fields such as social science~\cite{berrey2015enigma, dobbin2016diversity,simpson1949measurement}, political science~\cite{surowiecki2005wisdom}, and information retrieval~\cite{agrawal2009diversifying}. 
Tracing back machine bias to its source, there have been major efforts to identify different types~\cite{mehrabi2021survey, olteanu2019social,friedman1996bias} and sources~\cite{torralba2011unbiased,crawford2013hidden,diakopoulos2015algorithmic} of biases in data. Efforts to satisfy {\it responsible data} requirements~\cite{nargesian2022responsible} extend to various stages of the data analysis pipeline, including data annotation~\cite{li2020towards,lazier2023fairness}, data cleaning and repair~\cite{SalimiRHS19,tae2019data,salimi2020database}, data imputation~\cite{martinez2019fairness}, entity resolution~\cite{shahbazi2023through,fanourakis2023fairer}, data integration~\cite{nargesian2022responsible,nargesian2021tailoring}, etc. 

\paragraph{Data Coverage:}The notion of data coverage has received extensive attention from different angles. Detecting lack of coverage has been studied for datasets with discrete~\cite{asudeh2019assessing} and continuous~\cite{asudeh2021coverage} attributes populated in single or multiple \cite{lin2020identifying} relations.
To resolve insufficient coverage, \cite{accinelli2020coverage, accinelli2021impact,shetiya2022fairness}
consider resolving representation bias in preprocessing pipelines by rewriting queries into the closest operation so that certain subgroups are sufficiently represented in the downstream tasks. Alternatively, ~\cite{asudeh2019assessing,tae2021slice} propose a data collection strategy to acquire as little additional data as possible (to minimize the associated costs) to meet the representation constraints. ~\cite{sharma2020data,iosifidis2018dealing,celis2020data} opt for a data augmentation approach by adding partially altered duplicates of already existing tuples or generating new synthetic entries from existing data. Consequently, the new data set has an equal number of elements for different groups, resulting in potentially resolving the under-representation issues. Finally,  \cite{nargesian2021tailoring} utilizes data integration techniques to consolidate data from different sources into a single dataset to resolve representation bias.
Related works also include ~\cite{chung2019slice,sagadeeva2021sliceline,tae2021slice} that seek to understand if the overall performance of the model fails to reflect and performs poorly on certain slices in the data.
As alternative approaches to measure representation bias, the notion of representation rate~\cite{celis2020data} (a.k.a. equal base rate~\cite{kleinberg2016inherent}) is introduced which compared with coverage, it is more restrictive as it requires almost equal ratios from different groups.
Please refer to \cite{shahbazi2023representation} for a comprehensive survey about representation bias in data. 

\paragraph{ML Reliability:} Model-centric works for uncertainty quantification such as 
probabilistic classifiers~\cite{zadrozny2001obtaining,zadrozny2002transforming,platt1999probabilistic,niculescu2005predicting},
prediction intervals (PIs) \cite{chatfield93predictionintervals,pearce2018high,khosravi2010lower} and conformal predictions (CP)~\cite{angelopoulos2021gentle,shafer2008tutorial} that are used for measuring prediction uncertainty, are built
by maximizing the {\it expected performance} on {\it random} sample from the underlying distribution.
As a result, while providing accurate estimations for the dense regions of data (e.g. majority groups), their estimation accuracy is questionable for the poorly represented regions.
In particular, \cite{angelopoulos2021gentle} recognizes the lack of guarantees in the performance of CP for such regions.
Besides, the bulk of work on trustworthy AI provides information that {\it supports} the outcome of an ML model. For example, existing work on explainable AI, including~\cite{harradon2018causal,ribeiro2016should,gunning2019darpa}, aims to find simple explanations and rules that justify the outcome of a model.
Conversely, we aim to {\it raise warning signals} when the outcome of a model is {\it not} trustworthy. That is, to provide reasons that {\it cast doubt} on the reliability of the model outcome {for a given query point}.

 %%%%%%%%%%%%%%%%%%%%%%%%%%%%%%%% FUTURE  %%%%%%%%%%%%%%%%%%%%%%%%%%%%%%%%
% \vspace{-3mm}
\section{Final Remarks}\label{sec:conclusion}
As Data-centric AI and Responsible AI emerge as focal points in data science research, the development of Data-centric methodologies for ensuring Responsible and Trustworthy AI attracts increasing attention.
While there is some excellent work on responsible data management to achieve this goal, there remain many challenges yet to be addressed.

In this paper, we focused on a crucial aspect of responsible data -- detecting and addressing the under-representation of minorities within a data set.
We formally defined the notion of data coverage and discussed various techniques for (a) identifying lack of representation issues across different data modalities, (b) ensuring proper representation of minorities in data, and (c) limiting the scope-of-use of data sets based on their representation issues by generating proper ({\sc RU}) warning signals.
Even though the research on detecting lack of coverage issues is relatively mature, resolution techniques are still understudied.
Considering the recent advancements in Generative AI, utilizing Foundation Models and Large Language Models, and studying their limitations, for data augmentation to improve the representation of minorities at the data level seems interesting to further explore.

 %%%%%%%%%%%%%%%%%%%%%%%%%%%%%%%% BIB  %%%%%%%%%%%%%%%%%%%%%%%%%%%%%%%%
\bibliographystyle{unsrt}
\small
% \bibliography{ref}
\begin{thebibliography}{10}

\bibitem{asudeh2019assessing}
A.~Asudeh, Z.~Jin, and H.~Jagadish.
\newblock Assessing and remedying coverage for a given dataset.
\newblock In {\em ICDE}, pages 554--565. IEEE, 2019.

\bibitem{shahbazi2023representation}
N.~Shahbazi, Y.~Lin, A.~Asudeh, and H.~Jagadish.
\newblock Representation bias in data: A survey on identification and resolution techniques.
\newblock {\em ACM Computing Surveys}, 2023.

\bibitem{asudeh2021coverage}
A.~Asudeh, N.~Shahbazi, Z.~Jin, and H.~V. Jagadish.
\newblock Identifying insufficient data coverage for ordinal continuous-valued attributes.
\newblock In {\em SIGMOD}. ACM, 2021.

\bibitem{mousavi2024data}
M.~Mousavi, N.~Shahbazi, and A.~Asudeh.
\newblock Data coverage for detecting representation bias in image datasets: {A} crowdsourcing approach.
\newblock In {\em {EDBT}}, pages 47--60, 2024.

\bibitem{nargesian2021tailoring}
F.~Nargesian, A.~Asudeh, and H.~Jagadish.
\newblock Tailoring data source distributions for fairness-aware data integration.
\newblock {\em Proceedings of the VLDB Endowment}, 14(11):2519--2532, 2021.

\bibitem{nargesian2022responsible}
F.~Nargesian, A.~Asudeh, and H.~V. Jagadish.
\newblock Responsible data integration: Next-generation challenges.
\newblock {\em SIGMOD}, 2022.

\bibitem{sharma2020data}
S.~Sharma, Y.~Zhang, J.~M. R{\'\i}os~Aliaga, D.~Bouneffouf, V.~Muthusamy, and K.~R. Varshney.
\newblock Data augmentation for discrimination prevention and bias disambiguation.
\newblock In {\em AIES}, pages 358--364, 2020.

\bibitem{DBLP:journals/jair/ChawlaBHK02}
N.~V. Chawla, K.~W. Bowyer, L.~O. Hall, and W.~P. Kegelmeyer.
\newblock {SMOTE:} synthetic minority over-sampling technique.
\newblock {\em J. Artif. Intell. Res.}, 16:321--357, 2002.

\bibitem{iosifidis2018dealing}
V.~Iosifidis and E.~Ntoutsi.
\newblock Dealing with bias via data augmentation in supervised learning scenarios.
\newblock {\em Jo Bates Paul D. Clough Robert J{\"a}schke}, 24, 2018.

\bibitem{celis2020data}
L.~E. Celis, V.~Keswani, and N.~Vishnoi.
\newblock Data preprocessing to mitigate bias: A maximum entropy based approach.
\newblock In {\em ICML}, pages 1349--1359. PMLR, 2020.

\bibitem{asudeh2022towards}
A.~Asudeh and F.~Nargesian.
\newblock Towards distribution-aware query answering in data markets.
\newblock {\em Proceedings of the VLDB Endowment}, 15(11):3137--3144, 2022.

\bibitem{motwani1995randomized}
R.~Motwani and P.~Raghavan.
\newblock {\em Randomized algorithms}.
\newblock Cambridge university press, 1995.

\bibitem{chameleon}
M.~Erfanian, H.~V. Jagadish, and A.~Asudeh.
\newblock Chameleon: Foundation models for fairness-aware multi-modal data augmentation to enhance coverage of minorities.
\newblock {\em arXiv preprint arXiv:2402.01071}, 2024.

\bibitem{scholkopf1999support}
B.~Sch{\"o}lkopf, R.~C. Williamson, A.~Smola, J.~Shawe-Taylor, and J.~Platt.
\newblock Support vector method for novelty detection.
\newblock {\em NeurIPS}, 12, 1999.

\bibitem{phillips1998feret}
P.~J. Phillips, H.~Wechsler, J.~Huang, and P.~J. Rauss.
\newblock The feret database and evaluation procedure for face-recognition algorithms.
\newblock {\em Image and vision computing}, 16(5):295--306, 1998.

\bibitem{dressel2018accuracy}
J.~Dressel and H.~Farid.
\newblock The accuracy, fairness, and limits of predicting recidivism.
\newblock {\em Science advances}, 4(1):eaao5580, 2018.

\bibitem{ng2021mlops}
A.~Ng.
\newblock Mlops: From model-centric to data-centric {AI}.
\newblock 2021.

\bibitem{wing2021trustworthy}
J.~M. Wing.
\newblock Trustworthy {AI}.
\newblock {\em CACM}, 64(10):64--71, 2021.

\bibitem{kentour2021analysis}
M.~Kentour and J.~Lu.
\newblock Analysis of trustworthiness in machine learning and deep learning.
\newblock {\em InfoComp}, 2021.

\bibitem{liu2021trustworthy}
H.~Liu, Y.~Wang, W.~Fan, X.~Liu, Y.~Li, S.~Jain, A.~K. Jain, and J.~Tang.
\newblock Trustworthy {AI}: A computational perspective.
\newblock {\em arXiv preprint arXiv:2107.06641}, 2021.

\bibitem{singh2021trustworthy}
R.~Singh, M.~Vatsa, and N.~Ratha.
\newblock Trustworthy {AI}.
\newblock In {\em 8th ACM IKDD CODS and 26th COMAD}, pages 449--453. 2021.

\bibitem{kulynych2022you}
B.~Kulynych, Y.-Y. Yang, Y.~Yu, J.~B{\l}asiok, and P.~Nakkiran.
\newblock What you see is what you get: Distributional generalization for algorithm design in deep learning.
\newblock {\em arXiv preprint arXiv:2204.03230}, 2022.

\bibitem{kakade2003sample}
S.~M. Kakade.
\newblock {\em On the sample complexity of reinforcement learning}.
\newblock University of London, University College London (United Kingdom), 2003.

\bibitem{dwork2012fairness}
C.~Dwork, M.~Hardt, T.~Pitassi, O.~Reingold, and R.~Zemel.
\newblock Fairness through awareness.
\newblock In {\em ITCS}, pages 214--226, 2012.

\bibitem{techrep}
N.~Shahbazi and A.~Asudeh.
\newblock Data-centric reliability evaluation of individual predictions.
\newblock {\em CoRR, abs/2204.07682}, 2022.

\bibitem{adult}
M.~Lichman.
\newblock Adult income dataset, {UCI} machine learning repository.
\newblock \url{https://archive.ics.uci.edu/ml/datasets/adult}, 2013.

\bibitem{neyman1936contributions}
J.~Neyman and E.~S. Pearson.
\newblock Contributions to the theory of testing statistical hypotheses.
\newblock {\em Statistical Research Memoirs}, 1936.

\bibitem{olteanu2019social}
A.~Olteanu, C.~Castillo, F.~Diaz, and E.~Kiciman.
\newblock Social data: Biases, methodological pitfalls, and ethical boundaries.
\newblock {\em Frontiers in Big Data}, 2:13, 2019.

\bibitem{fairmlbook}
S.~Barocas, M.~Hardt, and A.~Narayanan.
\newblock Fairness and machine learning: Limitations and opportunities.
\newblock \url{fairmlbook.org}, 2019.

\bibitem{barocas2016big}
S.~Barocas and A.~D. Selbst.
\newblock Big data's disparate impact.
\newblock {\em Calif. L. Rev.}, 104:671, 2016.

\bibitem{jk2019bias}
J.~Kleinberg.
\newblock Fairness, rankings, and behavioral biases.
\newblock FAT*, 2019.

\bibitem{drosou2017diversity}
M.~Drosou, H.~Jagadish, E.~Pitoura, and J.~Stoyanovich.
\newblock Diversity in big data: A review.
\newblock {\em Big data}, 5(2):73--84, 2017.

\bibitem{berrey2015enigma}
E.~Berrey.
\newblock {\em The enigma of diversity: The language of race and the limits of racial justice}.
\newblock University of Chicago Press, 2015.

\bibitem{dobbin2016diversity}
F.~Dobbin and A.~Kalev.
\newblock Why diversity programs fail and what works better.
\newblock {\em Harvard Business Review}, 94(7-8):52--60, 2016.

\bibitem{simpson1949measurement}
E.~H. Simpson.
\newblock Measurement of diversity.
\newblock {\em Nature}, 163(4148), 1949.

\bibitem{surowiecki2005wisdom}
J.~Surowiecki.
\newblock {\em The wisdom of crowds}.
\newblock Anchor, 2005.

\bibitem{agrawal2009diversifying}
R.~Agrawal, S.~Gollapudi, A.~Halverson, and S.~Ieong.
\newblock Diversifying search results.
\newblock In {\em WSDM}, pages 5--14. ACM, 2009.

\bibitem{mehrabi2021survey}
N.~Mehrabi, F.~Morstatter, N.~Saxena, K.~Lerman, and A.~Galstyan.
\newblock A survey on bias and fairness in machine learning.
\newblock {\em ACM Computing Surveys (CSUR)}, 54(6):1--35, 2021.

\bibitem{friedman1996bias}
B.~Friedman and H.~Nissenbaum.
\newblock Bias in computer systems.
\newblock {\em TOIS}, 14(3):330--347, 1996.

\bibitem{torralba2011unbiased}
A.~Torralba and A.~A. Efros.
\newblock Unbiased look at dataset bias.
\newblock In {\em CVPR 2011}, pages 1521--1528. IEEE, 2011.

\bibitem{crawford2013hidden}
K.~Crawford.
\newblock The hidden biases in big data.
\newblock {\em Harvard business review}, 1(4), 2013.

\bibitem{diakopoulos2015algorithmic}
N.~Diakopoulos.
\newblock Algorithmic accountability: Journalistic investigation of computational power structures.
\newblock {\em Digital journalism}, 3(3):398--415, 2015.

\bibitem{li2020towards}
Y.~Li, H.~Sun, and W.~H. Wang.
\newblock Towards fair truth discovery from biased crowdsourced answers.
\newblock In {\em SIGKDD}, pages 599--607, 2020.

\bibitem{lazier2023fairness}
S.~Lazier, S.~Thirumuruganathan, and H.~Anahideh.
\newblock Fairness and bias in truth discovery algorithms: An experimental analysis.
\newblock {\em arXiv preprint arXiv:2304.12573}, 2023.

\bibitem{SalimiRHS19}
B.~Salimi, L.~Rodriguez, B.~Howe, and D.~Suciu.
\newblock Interventional fairness: Causal database repair for algorithmic fairness.
\newblock In {\em {SIGMOD}}, pages 793--810. {ACM}, 2019.

\bibitem{tae2019data}
K.~H. Tae, Y.~Roh, Y.~H. Oh, H.~Kim, and S.~E. Whang.
\newblock Data cleaning for accurate, fair, and robust models: A big data-{AI} integration approach.
\newblock In {\em DEEM workshop}, pages 1--4, 2019.

\bibitem{salimi2020database}
B.~Salimi, B.~Howe, and D.~Suciu.
\newblock Database repair meets algorithmic fairness.
\newblock {\em ACM SIGMOD Record}, 49(1):34--41, 2020.

\bibitem{martinez2019fairness}
F.~Mart{\'\i}nez-Plumed, C.~Ferri, D.~Nieves, and J.~Hern{\'a}ndez-Orallo.
\newblock Fairness and missing values.
\newblock {\em arXiv preprint arXiv:1905.12728}, 2019.

\bibitem{shahbazi2023through}
N.~Shahbazi, N.~Danevski, F.~Nargesian, A.~Asudeh, and D.~Srivastava.
\newblock Through the fairness lens: Experimental analysis and evaluation of entity matching.
\newblock {\em Proceedings of the VLDB Endowment}, 16(11):3279--3292, 2023.

\bibitem{fanourakis2023fairer}
N.~Fanourakis, C.~Kontousias, V.~Efthymiou, V.~Christophides, and D.~Plexousakis.
\newblock Fairer demo: Fairness-aware and explainable entity resolution.
\newblock 2023.

\bibitem{lin2020identifying}
Y.~Lin, Y.~Guan, A.~Asudeh, and H.~Jagadish.
\newblock Identifying insufficient data coverage in databases with multiple relations.
\newblock {\em Proceedings of the VLDB Endowment}, 13(12):2229--2242, 2020.

\bibitem{accinelli2020coverage}
C.~Accinelli, S.~Minisi, and B.~Catania.
\newblock Coverage-based rewriting for data preparation.
\newblock In {\em EDBT Workshops}, 2020.

\bibitem{accinelli2021impact}
C.~Accinelli, B.~Catania, G.~Guerrini, and S.~Minisi.
\newblock The impact of rewriting on coverage constraint satisfaction.
\newblock In {\em EDBT Workshops}, 2021.

\bibitem{shetiya2022fairness}
S.~Shetiya, I.~P. Swift, A.~Asudeh, and G.~Das.
\newblock Fairness-aware range queries for selecting unbiased data.
\newblock In {\em ICDE}. IEEE, 2022.

\bibitem{tae2021slice}
K.~H. Tae and S.~E. Whang.
\newblock Slice tuner: A selective data acquisition framework for accurate and fair machine learning models.
\newblock In {\em SIGMOD}, pages 1771--1783, 2021.

\bibitem{chung2019slice}
Y.~Chung, T.~Kraska, N.~Polyzotis, K.~H. Tae, and S.~E. Whang.
\newblock Slice finder: Automated data slicing for model validation.
\newblock In {\em ICDE}, pages 1550--1553. IEEE, 2019.

\bibitem{sagadeeva2021sliceline}
S.~Sagadeeva and M.~Boehm.
\newblock Sliceline: Fast, linear-algebra-based slice finding for ml model debugging.
\newblock In {\em SIGMOD}, pages 2290--2299, 2021.

\bibitem{kleinberg2016inherent}
J.~Kleinberg, S.~Mullainathan, and M.~Raghavan.
\newblock Inherent trade-offs in the fair determination of risk scores.
\newblock {\em arXiv preprint arXiv:1609.05807}, 2016.

\bibitem{zadrozny2001obtaining}
B.~Zadrozny and C.~Elkan.
\newblock Obtaining calibrated probability estimates from decision trees and naive bayesian classifiers.
\newblock In {\em ICML}, volume~1, pages 609--616. Citeseer, 2001.

\bibitem{zadrozny2002transforming}
B.~Zadrozny and C.~Elkan.
\newblock Transforming classifier scores into accurate multiclass probability estimates.
\newblock In {\em SIGKDD}, pages 694--699, 2002.

\bibitem{platt1999probabilistic}
J.~Platt et~al.
\newblock Probabilistic outputs for support vector machines and comparisons to regularized likelihood methods.
\newblock {\em Advances in large margin classifiers}, 10(3):61--74, 1999.

\bibitem{niculescu2005predicting}
A.~Niculescu-Mizil and R.~Caruana.
\newblock Predicting good probabilities with supervised learning.
\newblock In {\em Proceedings of the 22nd international conference on Machine learning}, pages 625--632, 2005.

\bibitem{chatfield93predictionintervals}
C.~Chatfield.
\newblock Prediction intervals.
\newblock {\em Journal of Business and Economic Statistics}, 11:121--135, 1993.

\bibitem{pearce2018high}
T.~Pearce, A.~Brintrup, M.~Zaki, and A.~Neely.
\newblock High-quality prediction intervals for deep learning: A distribution-free, ensembled approach.
\newblock In {\em International conference on machine learning}, pages 4075--4084. PMLR, 2018.

\bibitem{khosravi2010lower}
A.~Khosravi, S.~Nahavandi, D.~Creighton, and A.~F. Atiya.
\newblock Lower upper bound estimation method for construction of neural network-based prediction intervals.
\newblock {\em IEEE transactions on neural networks}, 22(3):337--346, 2010.

\bibitem{angelopoulos2021gentle}
A.~N. Angelopoulos and S.~Bates.
\newblock A gentle introduction to conformal prediction and distribution-free uncertainty quantification.
\newblock {\em arXiv preprint arXiv:2107.07511}, 2021.

\bibitem{shafer2008tutorial}
G.~Shafer and V.~Vovk.
\newblock A tutorial on conformal prediction.
\newblock {\em Journal of Machine Learning Research}, 9(3), 2008.

\bibitem{harradon2018causal}
M.~Harradon, J.~Druce, and B.~Ruttenberg.
\newblock Causal learning and explanation of deep neural networks via autoencoded activations.
\newblock {\em arXiv preprint arXiv:1802.00541}, 2018.

\bibitem{ribeiro2016should}
M.~T. Ribeiro, S.~Singh, and C.~Guestrin.
\newblock " why should i trust you?" explaining the predictions of any classifier.
\newblock In {\em SIGKDD}, pages 1135--1144, 2016.

\bibitem{gunning2019darpa}
D.~Gunning and D.~Aha.
\newblock Darpa’s explainable artificial intelligence ({XAI}) program.
\newblock {\em AI Magazine}, 40(2):44--58, 2019.

\end{thebibliography}

\end{document}

\end{article}



\begin{article}
{Ethical Challenges in the Future of Work}
{Pierre Bourhis, Gianluca Demartini, Shady Elbassuoni, Emile Hoareau, H. Raghav Rao}
\graphicspath{{submissions/gianluca/}}
% link to instruction: https://tc.computer.org/tcde/tcde-bulletin-author-instructions/
% \documentclass[11pt,dvipdfm]{article}
\documentclass[11pt]{article}
\usepackage{tabularx}
\usepackage{ragged2e}  % for '\RaggedRight' macro (allows hyphenation)
\usepackage{booktabs}  % for \toprule, \midrule, and \bottomrule macros
\usepackage{textcomp}
\usepackage{amsfonts,amsmath}
\usepackage{deauthor,times}
\usepackage{graphicx} % 
\usepackage{hyperref}
\usepackage{comment}
\graphicspath{{asudeh/}}
\usepackage{soul}
\usepackage{subcaption}
\usepackage{ulem}
\usepackage{wrapfig}
\usepackage{color}
\usepackage{xspace}
\newtheorem{problem}{Problem}

%\DeclareMathOperator*{\argmax}{arg\,max}

%remove the following commands/package b4 submission
\newcommand{\hide}[1]{}
\newcommand{\eat}[1]{}
\newcommand{\resolved}[1]{\hide{#1}}
\newcommand{\abol}[1]{\textcolor{red}{Abol: #1}}
\newcommand{\mahdi}[1]{\textcolor{red}{Mahdi: #1}}
\newcommand{\nima}[1]{\textcolor{red}{Nima: #1}}

\newcommand{\dee}{\mathcal{D}}
\newcommand{\Gee}{\mathcal{G}}
\newcommand{\gee}{\mathbf{g}}
\newcommand{\ee}{\mathbf{e}}
\newcommand{\es}{\mathcal{S}}
\newcommand{\el}{\mathcal{L}}
\newcommand{\xx}{\mathcal{x}}
\newcommand{\dist}{\xi}
\newcommand{\alg}{\mathsf{A}}
\newcommand{\qu}{\mathbf{q}}
\newcommand{\ex}{\mathbf{x}}
\newcommand{\ti}{\mathbf{t}}
\newcommand{\sdt}{\mathsf{SDT}}
\newcommand{\wdt}{\mathsf{WDT}}
\newcommand{\Qu}{\mathbf{Q}}
\newcommand{\pe}{\mathbb{P}}
\newcommand{\megam}{\mathcal{M}}
\newcommand{\eps}{\varepsilon}
\newcommand{\enet}{{$\varepsilon$-{\bf net}}\xspace}
\newcommand{\net}{{\tt net}\xspace}
\newcommand{\vcd}{VC-dimension\xspace}
\newcommand{\at}[1]{{\tt \small #1}\xspace}
\newcommand{\pr}{Pr}

\newcommand{\sharpP}{\mbox{\#P}}
\newcommand{\NP}{\mathsf{NP}}
\newcommand{\LP}{\mathsf{LP}}
\newcommand{\IP}{\mathsf{IP}}
\newcommand{\ru}{{\sc {RU}}\xspace}
\newcommand{\sru}{{\sc {strongRU}}\xspace}
\newcommand{\wru}{{\sc {weakRU}}\xspace}

\newcommand{\fmsystem}{{\sc Chameleon}\xspace}
\newcommand{\fm}{$\mathcal{F}$\xspace}

\newtheorem{experiment}{Experiment}

\begin{document}

\title{Coverage-based Data-centric Approaches for \\Responsible and Trustworthy AI\thanks{This research was supported by the National Science Foundation under grant No. 2107290.}}

\author{
\begin{tabular}[t]{c@{\extracolsep{2.4em}}c@{\extracolsep{2.4em}}c@{\extracolsep{2.3em}}c} 
Nima Shahbazi & Mahdi Erfanian & Abolfazl Asudeh \\ 
University of Illinois Chicago & University of Illinois Chicago & University of Illinois Chicago\\
 nshahb3@uic.edu & merfan2@uic.edu & asudeh@uic.edu
\end{tabular}
}

\maketitle


\begin{abstract}
The grand goal of data-driven decision systems is to help make decisions easier, more accurate, at a higher scale, and also just. However, data-driven algorithms are only as good as the data they work with. Yet, data sets, especially those with social data, often do not represent minorities. The paucity of training data is a perpetual problem for AI, and the outcome of ML models for cases not represented in their training data is often not reliable. 
Hence, without properly addressing the lack of representation issues in data, we cannot expect AI-based societal solutions to have responsible and trustworthy outcomes. 

This paper focuses on data coverage as a data-centric approach for identifying and resolving misrepresentation of minorities in data.
To achieve this goal, we propose novel algorithms that (a) {\it identify} and {\it resolve} insufficient data coverage across data with different modalities and (b) use lack of representation information to generate data-centric {\it reliability warnings}.
 \end{abstract}
 
 %%%%%%%%%%%%%%%%%%%%%%%%%%%%%%%% INTRO  %%%%%%%%%%%%%%%%%%%%%%%%%%%%%%%%
\section{Introduction}\label{sec:intro} % Abstract+Intro: up to 2.5 pages 
Data-driven decision-making has shaped every corner of human life, spanning from autonomous vehicles to healthcare and even predictive policing and criminal justice. A pivotal concern, especially in applications that affect individuals, revolves around the reliability of the decisions rendered by the system.
It is easy to see that the accuracy of a data-driven decision depends, first and foremost, on the data used to make it. Essentially, the system learns the phenomena that data represent. While we may desire that the data should represent the underlying data distribution from which the production data is drawn, this alone may be insufficient, as it merely enables the model to perform well for the average case.
As a result, a model with a high accuracy could fail for specific regions in the data with insufficient representation. These regions may matter because they frequently represent some minority population in society. They could also represent cases that may not happen very often but have a relevant impact on the correctness of a critical decision.
In short, if the data fails to sufficiently represent a specific population, the outcome of the decision system for that population may not be trustworthy.

The phenomenon known as \textit{Representation Bias} can arise from how the data was originally collected, or it could be the result of biases introduced post-collection—whether historically, cognitively, or statistically.

Representation bias is essentially inevitable without a systematic approach to data collection. 
For example, in the context of survey data collection, vital steps involve identifying all populations within the underlying distribution based on desired demographic information and ensuring comprehensive coverage with sufficient samples from each group. 
Even then, only an (uncontrolled) subset of the invitees will opt-in to respond to the survey.
Another challenge lies in the fact that data scientists often lack control over the data collection process, leading to the reliance on ``found data'' in the majority of data-driven systems. Therefore, with no guarantee on the aforementioned steps in the data collection process, the found data is most likely a biased sample.
Acknowledging the potential harms of representation bias, the notion of \textit{Data Coverage}~\cite{asudeh2019assessing,shahbazi2023representation} has been proposed to ensure the adequate representation of minority groups in data sets employed for decision-making and developing sophisticated data science tools. 

Addressing representation issues in data poses various challenges depending on the modality of the data. In this paper, we focus on identifying and resolving lack of coverage issues in data with different modalities.
We start by proposing a variety of techniques (spanning from geometric and combinatorial optimization to crowd-souring) aimed at efficiently detecting insufficient coverage on structured data sets with non-ordinal categorical and continuous attributes, as well as image data sets. Next, we propose a range of approaches grounded in data integration and generative data augmentation to address the lack of coverage by enriching the data sets with more data. However, with limited control over the data collection processes, it could be difficult and expensive to resolve all misrepresentations. 
Since adding more data is not always possible, we proceed to introduce data-centric preventive solutions that warn the user about the reliability of their predictions regarding representation bias issues. These warnings assist users in determining whether they trust the outcomes of the models or exercise caution. 

 %%%%%%%%%%%%%%%%%%%%%%%%%%%%%%%% IDENTIFICATION  %%%%%%%%%%%%%%%%%%%%%%%%%%%%%%%%
\section{Detecting Insufficient Representation of Minorities}\label{sec:identification} %up to 3.5 pages
Representation bias happens when the development (training data) population under-represents 
and subsequently fails to generalize well 
for some parts of the target population, due to historical bias, sampling bias, etc.
The notion of {\it data coverage} has been studied across different settings in \cite{shahbazi2023representation} as a metric to measure representation bias. At a high level, coverage is referred to as having enough similar entries for each object in a data set. 
For a better understanding, let us go over the definition of the generalized notion of coverage:

\begin{definition}[Data Coverage]\label{def:coverage}
Consider a data set $\dee$ with $n$ tuples, each consisting of $d$ attributes of interest $\mathbf{x}=\{x_1, x_2, \cdots,x_d\}$, such as {\tt gender}, {\tt race}, {\tt salary}, {\tt age}, etc, that are used for coverage identification.
The data set also contains target attributes $\mathbf{y} = \{ y_1,\cdots,y_{d'}\}$ that may or may not be considered for the coverage problem.
A query point $q$ is not covered by the data set $\dee$, if there are not ``enough'' data points in $\dee$ that are representative of $q$.
To generalize the notion of coverage, let us define $\gee(q)$ as the universe of tuples that would represent $q$ and let $\gee_\dee(q) = \gee(q)\cap \dee$. In other words, $\gee_\dee(q)$ are the set of tuples in $\dee$ that represent $q$.
Using this notation, we define the coverage of $q$ as the size of $\gee_\dee(q)$. That is,
$cov(q,\dee) = | \gee_\dee(q)|$.
Given a value $\tau$, $q$ is covered if $cov(q,\dee)>\tau$.
Similarly, a group $\gee$ is not covered if $\gee\cap \dee<\tau$.
The {\it uncovered region} in a data set is the collection of groups that are not covered by it.
\end{definition}

\subsection{Structured Data}
In this section, we focus on identifying representation bias in structured data.
Depending on the type of the attributes of interest, we categorize the techniques into two classes based on whether they target the problem for non-ordinal {\it categorical} (e.g. {\tt race}, {\tt gender}) or ordinal {\it continuous} (e.g. {\tt age}) attributes. The attributes of interest considered for representation bias often include sensitive attributes such as {\tt race} and {\tt gender} but are not necessarily limited to them.

\subsubsection{Categorical Attributes}

For cases where attributes of interest are non-ordinal categorical,
the cartesian product of values on a subset of attributes $\mathbf{x}'\subseteq \mathbf{x}$, form a set of (sub-)groups.
For example, $\{$ {\tt white male}, {\tt white female}, {\tt black male} $,\cdots\}$ are the subgroups defined on the attributes {\tt (race,gender)}.
We refer to the number of attributes used to specify a subgroup as the {\it level} of that subgroup.
For example, the level of the subgroup {\tt white male} is 2, while the level of the subgroup {\tt male} is 1.
We use $\ell(\gee)$, to refer to the level of a subgroup $\gee$.
Similarly, we say a subgroup $\gee'$ is a subset of $\gee$, if the groups specifying $\gee'$ are a superset of the ones for $\gee$. For example {\tt (married white male)} a subset of the more general group {\tt (white male)}. That is, the set of individuals in group {\tt (married white male)} are a subset of {\tt (white male)}.
Moreover, we say a subgroup $\gee$ is a {\it parent} of the subgroup $\gee'$, if $\gee'\subset \gee$ and $\ell(\gee)=\ell(\gee')+1$. For example, the subgroup {\tt (white male)} is a parent of the subgroup {\tt (married white male)}.
We use \textit{patterns} to refer to uncovered subgroups.
A pattern $P$ is a string of $d$ values, where $P[i]$ is either a value from the domain of $x_i$, or it is ``unspecified'', specified with $X$. 
For example, consider a data set with three binary attributes of interest $\mathbf{x}=\{x_1, x_2, x_3\}$. The pattern $P=X01$ specifies all the tuples for which $x_2=0$ and $x_3=1$ ($x_1$ can have any value).
The set of patterns that identify most general uncovered subgroups are called {\it Maximal Uncovered Patterns} (MUPs).

No polynomial time algorithm can guarantee the enumeration of the entire MUPs, however, several algorithms inspired by set enumeration and the Apriori algorithm for association rule mining are proposed to efficiently address this problem~\cite{asudeh2019assessing}.
In this regard, we introduce \textit{Pattern Graph} data structure that exploits the relationship between patterns to do less work than computing all uncovered patterns by removing the non-maximal ones. 
The parent-child relationship between the patterns is represented in a graph that can be used to find better algorithms. 
\textit{Pattern-Breaker} starts from the top of the graph where the general patterns are and moves down by breaking each pattern into more specific ones. If a pattern is uncovered, then all of its descendants are also uncovered and they can not be an MUP, even if they have a parent that is covered. Therefore, this subgraph of the pattern graph can be pruned. 
The issue with \textit{Pattern-Breaker} is that it explores the covered regions of the pattern graph and for the cases where there are a few uncovered patterns, it has to explore a large portion of the exponential-size graph. 
To tackle this, \textit{Pattern-Combiner} algorithm is proposed that performs a bottom-up traversal of the pattern graph. It uses an observation that the coverage of a node at the level of the pattern graph can be computed as the sum of the coverage values of its children. 
The problem with \textit{Pattern-Combiner} is that it traverses over the uncovered nodes first and therefore, it will not perform well for the cases in which most of the nodes in the graph are uncovered. 
In fact, for the cases where most of the MUPs are placed in the middle of the graph, both \textit{Pattern-Breaker} and \textit{Pattern-Combiner} will not be as efficient as they should traverse half of the graph. Therefore, we propose \textit{Deep-Diver}, a search algorithm based on Depth-First-Search that quickly finds the MUPs, and uses them to limit the search space by pruning the nodes both dominating and dominated by the discovered MUPs.

\begin{figure*}[!tb]
    \begin{minipage}[t]{0.31\linewidth}
        \centering
        \includegraphics[width=\textwidth]{submissions/submission1/shahbazi/covcube1.jpg}
        \caption{\small Categorical attributes: the uncovered region of a toy example, as the collection of three MUPs.}
        \label{fig:covcube1}
    \end{minipage}
    \hfill
    \begin{minipage}[t]{0.31\linewidth}
        \centering
        \includegraphics[width=\textwidth]{submissions/submission1/shahbazi/cvrg_2_1.jpg}
        \caption{\small Continuous attributes, 2D: identifying the covered region in the gray Voronoi cell.}
        \label{fig:cvrg_2_1}
    \end{minipage}
    \hfill
    \begin{minipage}[t]{0.31\linewidth}
        \centering
        \includegraphics[width=\textwidth]{submissions/submission1/shahbazi/cvrg_2_2.jpg}
        \caption{ \small Continuous attributes, 2D: Uncovered region marked in red.}
        \label{fig:cvrg_2_2}
    \end{minipage}
\vspace{-5mm}
\end{figure*}

\subsubsection{Continuous Attributes}
Data in the real world often consists of a combination of continuous and discrete values. While simple solutions like binning {\tt age} into {\tt young} and {\tt old} can transform the continuous space into discrete. However, they may lead to coarse groupings that are sensitive to the thresholds chosen. It may be inappropriate to treat a 35-yo as {\tt young} but a 36-yo as {\tt old}. 
Therefore, we extend the notion of coverage to continuous space. Particularly, given data set $\dee$ with $n$ tuples over $d$ attributes, and vicinity radius $\rho$ and coverage threshold $k$, we want to identify the uncovered region -- the universe of uncovered query points.
A query point in continuous data space is covered if there are enough (at least $k$) data points in its $\rho$-vicinity neighborhood. $\rho$-vicinity neighborhood is the circle centered at the query point with radius $\rho$.

Depending on the number of attributes in a data set, we propose two algorithms for identifying uncovered regions in data~\cite{asudeh2021coverage}. 
The first algorithm known as \textit{Uncovered-2D} studies coverage over two-dimensional data sets where $\mathbf{x}=\{x_1,x_2\}$. To find the number of circles that a query point falls into and consequently discover the uncovered region, \textit{Uncovered-2D} makes a connection to $k$-th order Voronoi diagrams.
Consider a data set $\mathcal{D}$ and its corresponding $k$-th order Voronoi diagram. For every tuple $t\in \mathcal{D}$, let $\circ_t$ be the $d$-dimensional sphere ($d$-sphere) with radius $\rho$ centered at $t$.
Consider a $k$-voronoi cell $\mathcal{V}(S)$ in the $k$-th order Voronoi diagram $V_k(\mathcal{D})$.
Any point $q$ inside the intersections of the $d$-spheres of tuples in $S$, i.e. $q\in \underset{\forall t\in S}{\cap ~\circ_t}$, is covered, while all other points in the region are uncovered.
 The algorithm starts by constructing the $k$-th order Voronoi diagram of the data set and then for each Voronoi cell $\mathcal{V}(S)$ in the diagram, it computes the intersection of the circles of the tuples in $S$ and marks the portion of $\mathcal{V}(S)$ that falls outside it as uncovered.
After identifying the uncovered region, a 2D map of $\{x_1,x_2\}$ value combinations is used to report the region to the user.
The algorithm for the 2D case can be extended to the general case by relaxing the assumption on the number of attributes to discover the exact uncovered region, however, due to the curse of dimensionality, the search size space explodes as the number of dimensions increases and as a result, the algorithm will not be practical. Therefore, we propose a randomized approximation algorithm based on the geometric notion of \enet. 
Let $\mathcal{X}$ be a set and $\mathcal{R}$ be a set of subsets of $\mathcal{X}$. A set $\mathcal{N}\subset \mathcal{X}$ is an \enet for $\mathcal{X}$ if for any range $r\in\mathcal{R}$, if  $|r\cap \chi|>\eps|\chi|$, then $r$ contains at least one point of $N$.
The idea, at a high level, is to draw enough random samples from the space of potential query points to form an \enet. 
We then label the sampled query points as $\{-1,+1\}$ depending on whether those are covered or not, and learn the uncovered regions using the samples.

\subsection{Image Data}
Many known incidents of machine failures due to the lack of representation were on image data.
We consider an image data set with a fixed number of low-cardinality sensitive attributes such as {\tt\small race} and {\tt\small gender}. 
It is common that image data sets {\it lack explicit values} for sensitive attributes, which are crucial for coverage identification. An image data set is often a collection of images from different domains with little to no information about their domain and which groups they belong to. As a result, even studying coverage over low-cardinality and categorical attributes of interests is challenging in these cases.

\begin{wrapfigure}{R}{0.42\textwidth}
\centering
\vspace{-3mm}
\scriptsize
\begin{tabular}{|@{}c|@{}c@{}|@{}c@{}|@{}c@{}|} 
 \hline
{\bf data set} & {\bf classifier} & {\bf accuracy} & {\bf precision} \\ 
 &  &  & {\bf on female} \\ \hline
UTKFace:~& DeepFace (opencv) & 93.56 & {52.02}\\\cline{2-4}
({\tt females}=200,& DeepFace (retinaface) & 94.16 & {56.15}\\\cline{2-4}
{\tt males}=2800) & BaseCNN & 97.6 & 74.8\\
\hline
UTKFace:~& DeepFace (opencv) & 96.53 & {\bf 8.0}\\\cline{2-4}
({\tt females}=20,& DeepFace (retinaface) & 96.43 & {\bf 10.09}\\\cline{2-4}
{\tt males}=2980)& BaseCNN & 97.6 & {\bf 21.59}\\
\hline
\end{tabular}
\vspace{-3mm}
\caption{\small ML models' low performance for females in the presence of representation bias.~\cite{mousavi2024data}}\label{fig:mlfails}
\vspace{-3mm}
\end{wrapfigure}

In Figure~\ref{fig:mlfails}, we show that due to the issues such {\it machine bias} and {\it lack of distribution generalizability},
solely relying on state-of-the-art machine learning (ML) techniques fail to effectively identify lack of coverage in image data sets. Therefore, we propose an approach based on combining crowdsouring with ML~\cite{mousavi2024data}. 
Crowdsourcing is particularly promising for image data, for tasks such as image labeling, which, while challenging for the machine, are "easy" for human beings to conduct with minimal error. 

A key observation that enables a cost-effective crowdsourcing approach is that, while studying coverage, we would only like to find out if there are {\it enough tuples from each subgroup}.
Suppose a subgroup is covered if there are $\tau=100$ instances of it in the data set. Assume the (majority) group $\gee_1$ contains $n_1 \gg 100$ objects in the data set. 
To verify that $\gee_1$ is covered, it is enough for the crowd to discover 100 of those objects, not the entire $n_1$. 
Following this, $O(\tau)$ provides a lower bound on the number of crowd tasks required to verify a given group is covered. 
Still, this lower bound only holds for the groups that are covered, i.e., there is at least $\tau$ of those in the data set.
Surprisingly, verifying that a minority group is indeed uncovered is cumbersome, unlike the majority group.
This is because even though discovering $\tau$ objects from a group is enough for verifying that it is covered, one cannot {\it verify} a group is uncovered until there is a chance that the data set might still have enough objects from that group. Thus, assuming a non-zero probability for each unlabeled object to belong to each group, {one might need to ask the crowd to label the entire data set before they can confirm that a specific group is uncovered}.

Our idea for addressing this challenge is to
design {\it a divide and conquer algorithm} that, instead of {point queries}, uses {\it set queries} to iteratively eliminate subsets of data that {does not include any object from the given group}.
At a high level, our idea is to ask a set query from the crowd, inquiring whether the selected set contains at least one object from the given group $\gee$.
The user may provide two responses (yes/no). 
Interestingly, {in either case}, the user response provides valuable information that helps efficiently identify the coverage.
If the answer is ``No'', the set does not include any object from the given group $\gee$. As a result, the algorithm can safely prune the set, asking no further questions about it. In particular, for a group that is not covered, one can expect to see no answers on large set queries helping to prune a significant portion of the data set quickly.
On the other hand, if the answer is ``yes'', the set contains {at least} one object from the group $\gee$. As a result, the algorithm cannot prune the subset since it can have any number (larger than one) of the objects in $\gee$.
At first glance, the queries with yes answers do not provide helpful information as the algorithm cannot prune the subset (hence it needs to divide it into smaller subsets).
However, a key observation is that {the algorithm will only observe a limited number of yes answers} before it stops.
The reason is that the number of set queries with yes answers provides a {lower-bound} on the number of objects from $\gee$ in the data set. As a result, the algorithm can stop as soon as the lower bound reaches $\tau$, knowing that $\gee$ is covered.
The D\&C approach verifies the data coverage for a given group, while our goal is to identify the uncovered regions for a given set of sensitive attributes. The next question is how to utilize this algorithm for efficient coverage identification on different scenarios of sensitive attributes, forming intersectional or non-intersectional groups.
In particular, how can we find maximal uncovered patterns?
Our idea is to apply sampling and aggregate estimation techniques to find the groups that even if merged are likely to still be uncovered. This will help reduce the coverage identification cost by running the D\&C approach for the merged groups once.
 %%%%%%%%%%%%%%%%%%%%%%%%%%%%%%%% RESOLUTION  %%%%%%%%%%%%%%%%%%%%%%%%%%%%%%%%
\section{Resolving Insufficient Representation}\label{sec:resolution}

Data integration~\cite{nargesian2021tailoring,nargesian2022responsible} and data augmentation~\cite{sharma2020data,DBLP:journals/jair/ChawlaBHK02,iosifidis2018dealing,celis2020data} are considered as the primary solutions for reducing data coverage issues in a data set. 
Data integration is promising when external sources of data are available. On the other hand, recent advancements in generative AI and foundation models have enabled efficient and effective augmentation of data sets with synthetic data. 
Therefore, in the following, we review two approaches, one from each category, in the context of lack of coverage resolution.

\subsection{Data Integration}\label{sec:resolution:integration}

Data integration is to consolidate data from different sources into a single, unified view. 
Although it is an effective solution to acquire additional data from different distributions,
there are sampling policy and cost-efficiency concerns that need to be examined.  
Therefore, {\it Data Distribution Tailoring ({\sc DT})} introduces data integration techniques for resolving insufficient representation of subgroups in a data set in the most cost-effective manner~\cite{nargesian2021tailoring}.
A query to {\sc DT} 
consists of a target schema, and a set of group distribution requirements in the form of the minimum counts (e.g., ``{\tt\small 1,000 breast cancer monitoring data in Chicago with at least 30\% label=positive, and at least 20\% black patients}''). 
Collecting a fresh sample from a data view is costly (monetary, human resources, and/or computation cost)~\cite{asudeh2022towards}.
Therefore, {\sc DT} focuses on satisfying the count requirements with minimum cost. 
Given an input query and a lake of available data sources, the first step is to discover a collection of candidate data views that satisfy the target schema.
Each data view $v_i$ is a projection-join $v_i = \Pi\big(D_{i1}\bowtie\cdots\bowtie D_{ik_i} \big)$, where $D_{ij}$ is a data set in a given data lake.
Let us suppose the data views are already discovered.
At a high level, {\sc DT} follows an iterative approach that at each iteration a data view is selected to be queried.
Each query to a data view has a fixed cost and returns a sample that may or may not satisfy the query constraints.
The samples that are either not fresh, or do not satisfy the query are discarded.
Hence, the essential question towards a cost-effective data integration is {\it what data view to query next}.
Depending on the available information about the data sources, various techniques may be employed. 

For the cases when the group distributions are known, the process of collecting the target data set is a sequence of iterative steps, where at every step, the algorithm chooses a data view, queries it, and if the obtained tuple contributes to one of the groups for which the count requirement is not yet fulfilled, it is kept, otherwise discarded. To do so, a {Dynamic Programming (DP)} algorithm is proposed. An optimal source at each iteration minimizes the sum of its sampling cost plus the expected cost of collecting the remaining required groups, based on its sampling outcome.
The DP algorithm, however, has a pseudo-polynomial time complexity. Hence, it quickly becomes intractable for cases where the minimum count requirements for the groups are not small. 
For cases where the (sensitive) attribute of interest is binary, such as (biological) {\tt sex}={\tt \{male, female\}}, and the cost to query data is similar from all sources, it turns out that the optimal strategy is to query the data source with {maximum probability of obtaining a sample from the minority group}.
Expanding the binary-attributes algorithm for non-binary cases, the problem can be modeled as an extension of the ``{\it coupon collector's}'' problem~\cite{motwani1995randomized}, where the goal is to collect $m_i$ instances from each coupon (group) $\gee_i$.
At each iteration, the coupon collector's algorithm identifies a data view as most promising and queries it. In simple terms, a data view with a smaller query cost and a higher chance of obtaining minority groups is more promising.


For the cases where the group distributions are unknown, we model DT as a {\it multi-armed bandit} problem, where every data view is modeled as an arm. 
Every arm has an unknown distribution of different groups while pulling an arm (i.e., querying the corresponding data view) has a cost.
During various iterations, the algorithms pull the arms in an order that its expected total {\it reward} is maximized.
Arguing that the reward of obtaining a tuple from a group is proportional to how rare this group is across different data views, 
we design the reward function based on the expected cost one needs to pay in order to collect a tuple from a specific group.  
As the bandit strategy, we adopt {\it Upper Confidence Bound (UCB)} to balance exploration and exploitation. At every iteration, for every arm, UCB computes confidence intervals for the expected reward and selects the arm with the maximum upper bound of reward to be explored next.

\subsection{Data Augmentation using Foundation Models}

While data integration provides a promising approach for resolving coverage issues in a data set, its effectiveness is limited to the availability of external data sources that are rich enough to find sufficient fresh samples from minority groups. This, however, is not always possible, especially since the minority samples are rare and not easy to obtain.
Fortunately, recent advancements in Generative AI and Foundation Models have enabled synthesizing samples that are otherwise challenging to obtain from the real world.

Therefore, as an alternative approach to data integration, we turn our attention to the Foundation Models and Generative AI for resolving the lack of coverage. 
Particularly, models such as {\sc DALL.E}\footnote{\url{https://openai.com/dall-e-2}} have emerged as powerful tools for generating multi-modal data such as image, audio, and video.
 
We formalize the foundation model \fm as a black-box function with the following inputs, that once queried synthesize an output tuple.
\begin{itemize}
    \item {\bf Prompt}: A natural language description providing instructions on the details of the tuple to be generated. For instance, a prompt for image generation might be ``A realistic photo of a white cat running in a backyard.''
    \item {\bf Guide}: In cases where only a prompt is provided, the foundation model uses its imagination to generate the requested tuple. For the previous example, the prompt of a cat image, the breed, size, background, and other details are generated based on the model's imagination. Alternatively, a guide can be provided to influence the generation process. The guide is formalized as a pair $(t,m)$ where $t$ is a tuple and $m$ is a mask specifying which parts of the guide tuple should be changed. Using the cat example, $t$ can be a cat image and $m$ can specify the foreground to be regenerated.
\end{itemize}

There are multiple challenges towards effective data set augmentations using foundation models. 
First, we have to determine the minimal set of synthetic tuples that once added to the original data set, under-representation issues are resolved.
Second, the generated images should follow the underlying distribution represented in the input data set. Third, the generated tuples should have high quality and look realistic to a human evaluator. Last but not least, given the (often monetary) cost associated with the queries to the foundation model, we should ensure the cost-effectiveness of the data set repair process.

\begin{wrapfigure}{L}{0.45\textwidth}
\centering
\vspace{-3mm}
\scriptsize
    \includegraphics[width=.45\textwidth]{submissions/submission1/shahbazi/enhanced_pipeline.png}
\vspace{-3mm}
\caption{\small Architecture of \fmsystem for image data augmentation for coverage enhancement.}\label{fig:chameleon}
% \vspace{-3mm}
\end{wrapfigure}

\noindent Figure~\ref{fig:chameleon} shows the architecture of our system \fmsystem \cite{chameleon} for coverage enhancement using DALL-E image generator.
To address the first challenge, we define the combinations-selection problem, which minimizes the total number of synthetic tuples for resolving lack of coverage of minorities at the most general level. We show the problem is {\sc NP}-hard, and propose a greedy approximation algorithm for it.
To address the second and third challenges, \fmsystem follows a {\it rejection sampling} strategy.
It views each tuple in the data set $\dee$ as an iid sample from the underlying distribution $\xi$ it represents. It uses the vector representations (embeddings) space to describe the distribution. Then, given a newly generated tuple, it employs the one-class support vector machine (OCSVM) approach proposed by Scholkopf et al.~\cite{scholkopf1999support} to reject the tuple if it does not follow $\xi$.
Moreover, it models the quality evaluation as hypothesis testing and rejects the samples that have a higher chance of being labeled as ``unrealistic'' by a random human evaluator.
Finally, to minimize the number of queries to the foundation model, we provide a guide tuple (and a mask), in addition to the prompt, to the foundation model. We model the guide-selection problem as {\it contextual multi-armed bandit} and propose a solution based on the contextual UCB for it.

Before concluding this section, let us provide some experiment results to demonstrate the effectiveness of data augmentation with \fmsystem. We use FERET DB \cite{phillips1998feret} for this experiment, which comprises 1199 individual images and serves as a standardized facial image database for researchers to develop algorithms and report results. All images in FERET DB share the same dimensions, pose, and facial expression.
First, we identified the (level-1) uncovered ethnicity groups, using the threshold 80. We then used \fmsystem and resolved the lack of coverage issues.
To evaluate the effectiveness of the system, we trained a CNN model to predict the race of each image within this dataset. We then retrained the identical CNN on the repaired training data. Importantly, our test dataset for both experiments remains consistent and is derived from real images.
Table~\ref{tab:lackofcoverage} presents the improvements in precision, recall, and F1 score metrics for under-represented groups after repairing the dataset. The results indicate an enhancement in performance metrics for all under-represented groups following the repair process.

\begin{table}[t]
    \centering
    \caption{Illustrating the effect of lack of coverage repair using \fmsystem on \texttt{FERTDB}}
    \label{tab:lackofcoverage}
    \vspace{-3mm}
    \begin{tabular}{lcccccccc}
        \toprule
         & \multicolumn{4}{c}{\textbf{Classifier Performance on \texttt{FERTDB}}} & \multicolumn{4}{c}{\textbf{Classifier Performance on Repaired}} \\
        \cmidrule(lr){2-5} \cmidrule(lr){6-9}
        \textbf{Ethnicity Groups}& \#Images & Precision & Recall & F1-Score & \#Images & Precision & Recall & F1-Score \\
        \midrule
        Overall          & 756 & 0.81 & 0.75 & 0.78 & 987 & 0.70 & 0.75 & 0.72 \\ \hline
        Black            & 40  & 0.19 & 0.22 & 0.16 & 100 & 0.48 & 0.56 & 0.52 \\
        Hispanic         & 19  & 0.50 & 0.17 & 0.25 & 100 & 0.62 & 0.36 & 0.45 \\
        Middle Eastern   & 10  & 0.00 & 0.00 & 0.00 & 100 & 0.20 & 0.41 & 0.27 \\
        \bottomrule
    \end{tabular}
\end{table}

 %%%%%%%%%%%%%%%%%%%%%%%%%%%%%%%% RELIABILITY  %%%%%%%%%%%%%%%%%%%%%%%%%%%%%%%%
\section{Generating Reliability Warnings}\label{sec:reliability}
% up to 2.5 pages
Interpretability is a necessity for data scientists who develop predictive models for critical decision-making.
In such settings, it is important to provide additional means to support the following question:
{\it is an individual prediction of the model reliable for decision-making?} Our goal is to use the lack of representation to help decision-makers find insights about this critical question.
To further motivate this, let us use the following example:

\vspace{1mm}
\begin{example}\label{ex-0}
{\bf(Part1):} Consider a judge who needs to decide whether to accept or deny a bail request. Using data-driven predictive models is prevalent in such cases for predicting recidivism~\cite{dressel2018accuracy}.
Indeed, such models can be beneficial to help the judge make wise decisions.
Suppose the model predicts the queried individual as high risk (or low risk).
The judge is aware and concerned about the critics surrounding such models.
A major question the judge faces is whether or not they should rely on the prediction outcome to take action for this case.
Furthermore, if, for instance, they decide to ignore the outcome and hence they need to provide a statement supporting their action, what evidence can they provide? 
\end{example}

In line with the recent trend on data-centric AI~\cite{ng2021mlops}, we design {novel approaches}, {complimentary} to the existing work on trustworthy AI~\cite{wing2021trustworthy,kentour2021analysis,liu2021trustworthy,singh2021trustworthy}, to address the aforementioned trust question through the lens of {\it data}.
In particular, unlike existing works that generate trust information from a {\it given \underline{model}}, we associate {\it \underline{data sets} with proper measurements} that specify their {\it the scope of use for predicting future cases}.
We note that a predictive model provides only probabilistic guarantees on the \underline{average} loss over the distribution represented by the data set used for training it.
As a result, these predictions may not be distribution generalizable~\cite{kulynych2022you}.
Consequently, if the query point is {\it not represented} by the data, the guarantees may not hold, hence one cannot rely on the prediction outcome.
Besides, an essential requirement for a learning algorithm is that its training data $\dee$ should represent the underlying distribution $\dist$.
Even if so, the trained model $h$ only provides a probabilistic guarantee on the {expected} loss on random samples from $\dist$.  
A model that performs well on {\it majority} of samples drawn from $\dist$ will have a high performance on average. Still, as we observed in Figure~\ref{fig:mlfails},
its performance for {\it minorities} and points that are not represented is questionable. Let us consider the following toy example:

\begin{figure*}[!b] 
    \begin{minipage}[t]{0.32\linewidth}
        	\centering
        	\includegraphics[width=\textwidth]{submissions/submission1/shahbazi/example_1.png} 
        	\vspace{-9mm}\caption{\small Data set $\dee$ generated using a Gaussian distribution; $x_1$ and $x_2$ are positively correlated}
            \label{fig:ex1:1}
    \end{minipage}
    \hfill
    \begin{minipage}[t]{0.32\linewidth}
        \centering
        	\includegraphics[width =\textwidth]{submissions/submission1/shahbazi/example_2.png} 
        	\vspace{-9mm}\caption{\small The decision boundary of learned model $h$ and query points $\qu^1$ to $\qu^4$}
            \label{fig:ex1:2}
    \end{minipage}
    \hfill
    \begin{minipage}[t]{0.32\linewidth}
        	\centering
        	\includegraphics[width =\textwidth]{submissions/submission1/shahbazi/example_3.png}
        	\vspace{-9mm}\caption{\small Ground-truth boundary, overlaid on the model decision boundary and query points}
            \label{fig:ex1:3}
    \end{minipage}
    \vspace{-5mm}
\end{figure*} 

\vspace{1mm}
\begin{example}\label{ex-1}
Consider a binary classification task where the input space is $\ex=\langle x_1, x_2\rangle$ and the output space is the binary label $y$ with values $\{-1$ (red) $,+1$ (blue)$\}$.
Suppose the underlying data distribution $\dist$ follows a 2D Gaussian, where $x_1$ and $x_2$ 
are positively correlated as shown in Figure~\ref{fig:ex1:1}.
The figure shows the data set $\dee$ drawn independently from the distribution $\dist$, along with their labels as their colors.
Using $\dee$, the prediction model $h$ is constructed as shown in Figure~\ref{fig:ex1:2}. 
The decision boundary is specified in the picture; while any point above the line is predicted as +1, a query point below it is labeled as -1.
The classifier has been evaluated using a test set that is an iid sample set drawn from the underlying data set $\dist$. The accuracy on the test set is high (above 90\%), and hence, the model gets deployed.
We cherry-picked four query points, $\qu^1$ to $\qu^4$, that are also included in Figure~\ref{fig:ex1:2}. Using $h$ for prediction, $h(\qu^1)=-1$, $h(\qu^2)=+1$,  $h(\qu^3)=+1$, and $h(\qu^4)=-1$.
Figure~\ref{fig:ex1:3} adds the ground-truth boundary to the search space, revealing the true label of the query points: every point inside the red circle has the true label $-1$ while any point outside of it is $+1$.
Looking at the figure, $y^1=+1$ while the model predicted it as $h(\qu^1)=-1$.  \hfill$\square$
\end{example}
\vspace{2mm}

Let us take a closer look at the four query points in this example and their placement with regard to the tuples in $\dee$ used for training $h$. 
$\qu^2$ belongs to a {\it dense region} with many training tuples in $\dee$ surrounding it. Besides, all of the tuples in its vicinity have the same label $y=+1$. As a result, one can expect that the model's outcome $h(\qu^2)=+1$ should be a reliable prediction.
Similar to $\qu^2$, $\qu^4$ also belongs to a dense region in $\dee$; however, $\qu^4$ belongs to an {\it uncertain region}, where some of the tuples in its vicinity have a label $y=+1$, and some others have the label $y=-1$. Considering the uncertainty in the vicinity of $\qu^4$, one cannot confidently rely on the outcome of the model $h$. 
On the other hand, the neighbors of $\qu^1$ (resp. $\qu^3$) are not uncertain, all having the label $y=-1$ (resp. $y=+1$).
However, the query points $\qu^1$ and $\qu^3$ are not well represented by $\dee$. In other words, $\qu^1$ and $\qu^3$ are unlikely to be generated according to the underlying distribution $\dist$, represented by $\dee$. As a result, following the no-free-lunch theorem~\cite{kakade2003sample}, one cannot expect the outcome of model $h$ to be reliable for these points.
Looking at the ground-truth boundary in Figure~\ref{fig:ex1:3}, $h$ luckily predicted the outcome for $\qu^3$ correctly, but it was not fortunate to predict the $y^1$ correctly.
Nevertheless, 
since the model is not reliably trained for these points, 
its outcome for these query points is not trustworthy.

From Example~\ref{ex-1}, we observe that the outcome of a model $h$, trained using a data set $\dee$ is not reliable for a query point $\qu$, if:
\begin{itemize}
    \item {\bf Lack of representation:} $\qu$ is not well-represented by $\dee$.
    In such cases, the model has not seen ``enough'' samples similar to $\qu$ to reliably learn and predict the outcome of $\qu$.
    \item {\bf Lack of certainty:} $\qu$ belongs to an uncertain region, where different tuples of $\dee$ in the vicinity of $\qu$ have different target values. $\qu$ belongs to a high-fluctuating area, where tuples in the vicinity of $\qu$ have a wide range of values.
\end{itemize} \vspace{2mm}

\noindent
Based on these two observations, we propose Representation-and-Uncertainty ({\bf RU}) measures.
To identify if a query suffers from uncertainty or lack of representation, one could use a deterministic approach using a fixed threshold. Then if the number of similar samples to (resp. label fluctuation in vicinity of) $\qu$ is larger than the threshold it is considered as unrepresented (resp. uncertain).
This approach, however, would be misleading since two numbers close to the threshold could be treated very differently. Also, all points on each side of the threshold would be considered equally represented (resp., certain). Instead, we consider {\it a randomized approach}, widely popular in the literature, including~\cite{dwork2012fairness}.
That is, instead of using fixed thresholds, a Bernoulli variable (a biased coin) is used that 
assigns $\qu$ as unrepresented (resp., uncertain) based on the number of samples similar to it (resp., its neighborhood uncertainty).
Given a query point $\qu$, let $\pe_o$ be the probability indicating if $\qu$ is not represented and let $\pe_u$ be the probability indicating if $\qu$ belongs to an uncertain region. 
We represent the probability of the Bernoulli variables for lack of representation or uncertainty components as $\pe_o$ and $\pe_u$, respectively. Note that the two Bernoulli variables $\pe_o$ and $\pe_u$ are independent from each other. That simply follows the argument that after specifying the number of similar samples to $\qu$ whether or not it should be considered as unrepresented does not depend on the uncertainty in the neighborhood of $\qu$.

\begin{definition}[\sru]\label{def:sdt}
The \sru is a probabilistic measure that considers the outcome of a model for a query point $\qu$ untrustworthy if $\qu$ is not represented by $\dee$ {\it and} it belongs to an uncertain region.
Formally, the \sru measure is:
\begin{align} 
    \nonumber
    SRU(\qu) &= \pe\big((\qu \mbox{ is outlier}) \wedge (\qu \mbox{ belongs to uncertain region})\big) 
\end{align}
Since $\pe_o$ and $\pe_u$ are independent:

\vspace{-13mm}
\begin{align} \label{eq:strong}
    SRU(\qu) &= \pe_o(\qu) \times \pe_u(\qu)
\end{align}
\end{definition}

\sru raises the warning signal only when the query point fails on {\it both} conditions of being represented by $\dee$ and not belonging to an uncertain region. 
For instance, in Example~\ref{ex-1} none of the query points fail both on representation and on uncertainty; hence neither has a high \sru score.
On the other hand, 
a high \sru score for a query point $\qu$ {\it provides a strong warning signal} that one should perhaps reject the model outcome and not consider it for decision-making.

\sru is a strong signal that raises warnings only for the fearfully concerning cases that fail both on representation and uncertainty.
However, as observed in Example~\ref{ex-1} a query points failing {\it at least} one of these conditions may also not be reliable, at least for critical decision making.
We define the \wru measure to raise a warning for such cases.

\begin{definition}[\wru]\label{def:wdt}
The \wru measure is a probabilistic measure that considers the outcome of a model for a query point $\qu$ untrustworthy if $\qu$ is not represented by $\dee$ {\bf or} it belongs to an uncertain region.
Formally, the \wru is computed as:
\begin{align} \label{eq:weak}
    WRU(\qu) = \pe\big((\qu \mbox{ is outlier}) \vee (\qu \mbox{ belongs to uncertain region})\big) 
    = \pe_o(\qu) + \pe_u(\qu) - \pe_o(\qu) \times \pe_u(\qu)
\end{align}
\end{definition}

Proposing quantitative probabilistic outcomes, \ru measures are interpretable for the users, since beyond the scores, the uncertainty and lack of representation components provide an explanation to justify them. 
Please refer to \cite{techrep} for more details on how to efficiently and effectively compute the representation ($\pe_o$) and uncertainty ($\pe_u$) probabilities, using only $\dee$.
In Example~\ref{ex-0}, let us see how the \ru measures can be helpful.

\noindent{\bf Example 1. (part 2):}
{\it RU measures \underline{raise warning} when
the fitness of the data set used for drawing a prediction is questionable, helping the judge to be cautious when taking action.
Besides, these measures provide \underline{quantitative evidence} to support the judge's action when they decide to ignore a prediction outcome that is not trustworthy.
The judge, for example, can argue to ignore a model outcome for a specific case, based on the insight that 
the model has been built using a
data set that fails to represent the given case.}
\hfill$\square$

Finally, let us demonstrate the efficacy of \ru measures through a series of experiments. Since the \ru measures are {\it data-centric},
those are applicable for both classification and regression tasks, irrespective of the model used.
We use {\it Adult} dataset~\cite{adult} for classification and {\it House Sales in King County} dataset for the validation of regression tasks. From each dataset, we uniformly sample two sets from the underlying distribution. The first set serves as the training set to compute the \ru values, and the second one is used as the test set from which the queries are drawn. We validate our proposal by providing the correlation between the \ru values and the performance of an ML model's prediction on the same data. 

We start by computing the \ru values for all the query points in the test set. Next, we bucketize the query points based on their \ru values in equi-width buckets of width 0.1. We repeat this for both \sru and \wru measures. Next, we train a model on the training data set and predict the target variable for the points in each range of \ru measure. The validation results for the classification task on the {\it Adult} dataset are presented in Figures \ref{fig:exp-adult-sdt} and \ref{fig:exp-adult-wdt}. Each figure corresponds to the accuracy/error measures of the classifier over each bucket of \ru values for \sru and \wru. As the \ru values increase, the accuracy of the model drops while the FPR rises, and therefore, the model fails to capture the ground truth for the points that fall into untrustworthy regions in the data set. By repeating the aforementioned steps for the regression task on the {\it House Sales in King County} dataset, we observe similar results presented in Figures \ref{fig:exp-hs-sdt} and \ref{fig:exp-hs-wdt}. 
As the \ru value increases, the RSS of the regression model follows the same trend denoting that the model fails to perform for tuples with a high \ru value.

\begin{figure}[!tb]
    \begin{minipage}[t]{0.24\linewidth}
        \centering
        \includegraphics[width=\textwidth]{submissions/submission1/shahbazi/sdt_adult.pdf}
        \vspace{-6mm}\caption{\small{\it Adult}, efficacy of \sru  on classification}
        \label{fig:exp-adult-sdt}
    \end{minipage}\hfill
    \begin{minipage}[t]{0.24\linewidth}
        \centering
        \includegraphics[width=\textwidth]{submissions/submission1/shahbazi/wdt_adult.pdf}
        \vspace{-6mm}\caption{\small{\it Adult}, efficacy of \wru  on classification}
        \label{fig:exp-adult-wdt}
    \end{minipage}\hfill
    \begin{minipage}[t]{0.24\linewidth}
        \centering
        \includegraphics[width=\textwidth]{submissions/submission1/shahbazi/sdt_regression_house.pdf}
        \vspace{-6mm}\caption{\small{\it House Sales in King County}, efficacy of \sru on regression}
        \label{fig:exp-hs-sdt}
    \end{minipage}\hfill
    \begin{minipage}[t]{0.24\linewidth}
        \centering
        \includegraphics[width=\textwidth]{submissions/submission1/shahbazi/wdt_regression_house.pdf}
        \vspace{-6mm}\caption{\small{\it House Sales in King County}, efficacy \wru on regression}
        \label{fig:exp-hs-wdt}
    \end{minipage}
\vspace{-5mm}
\end{figure}
 %%%%%%%%%%%%%%%%%%%%%%%%%%%%%%%% RELATED WORK  %%%%%%%%%%%%%%%%%%%%%%%%%%%%%%%%
\section{Related Work}\label{related} 

Bias in data has been looked at for a long time in statistical community~\cite{neyman1936contributions} but social data presents different challenges~\cite{olteanu2019social,fairmlbook,barocas2016big,jk2019bias,drosou2017diversity}.
The diversity and representativeness of data have been widely studied~\cite{drosou2017diversity}, in fields such as social science~\cite{berrey2015enigma, dobbin2016diversity,simpson1949measurement}, political science~\cite{surowiecki2005wisdom}, and information retrieval~\cite{agrawal2009diversifying}. 
Tracing back machine bias to its source, there have been major efforts to identify different types~\cite{mehrabi2021survey, olteanu2019social,friedman1996bias} and sources~\cite{torralba2011unbiased,crawford2013hidden,diakopoulos2015algorithmic} of biases in data. Efforts to satisfy {\it responsible data} requirements~\cite{nargesian2022responsible} extend to various stages of the data analysis pipeline, including data annotation~\cite{li2020towards,lazier2023fairness}, data cleaning and repair~\cite{SalimiRHS19,tae2019data,salimi2020database}, data imputation~\cite{martinez2019fairness}, entity resolution~\cite{shahbazi2023through,fanourakis2023fairer}, data integration~\cite{nargesian2022responsible,nargesian2021tailoring}, etc. 

\paragraph{Data Coverage:}The notion of data coverage has received extensive attention from different angles. Detecting lack of coverage has been studied for datasets with discrete~\cite{asudeh2019assessing} and continuous~\cite{asudeh2021coverage} attributes populated in single or multiple \cite{lin2020identifying} relations.
To resolve insufficient coverage, \cite{accinelli2020coverage, accinelli2021impact,shetiya2022fairness}
consider resolving representation bias in preprocessing pipelines by rewriting queries into the closest operation so that certain subgroups are sufficiently represented in the downstream tasks. Alternatively, ~\cite{asudeh2019assessing,tae2021slice} propose a data collection strategy to acquire as little additional data as possible (to minimize the associated costs) to meet the representation constraints. ~\cite{sharma2020data,iosifidis2018dealing,celis2020data} opt for a data augmentation approach by adding partially altered duplicates of already existing tuples or generating new synthetic entries from existing data. Consequently, the new data set has an equal number of elements for different groups, resulting in potentially resolving the under-representation issues. Finally,  \cite{nargesian2021tailoring} utilizes data integration techniques to consolidate data from different sources into a single dataset to resolve representation bias.
Related works also include ~\cite{chung2019slice,sagadeeva2021sliceline,tae2021slice} that seek to understand if the overall performance of the model fails to reflect and performs poorly on certain slices in the data.
As alternative approaches to measure representation bias, the notion of representation rate~\cite{celis2020data} (a.k.a. equal base rate~\cite{kleinberg2016inherent}) is introduced which compared with coverage, it is more restrictive as it requires almost equal ratios from different groups.
Please refer to \cite{shahbazi2023representation} for a comprehensive survey about representation bias in data. 

\paragraph{ML Reliability:} Model-centric works for uncertainty quantification such as 
probabilistic classifiers~\cite{zadrozny2001obtaining,zadrozny2002transforming,platt1999probabilistic,niculescu2005predicting},
prediction intervals (PIs) \cite{chatfield93predictionintervals,pearce2018high,khosravi2010lower} and conformal predictions (CP)~\cite{angelopoulos2021gentle,shafer2008tutorial} that are used for measuring prediction uncertainty, are built
by maximizing the {\it expected performance} on {\it random} sample from the underlying distribution.
As a result, while providing accurate estimations for the dense regions of data (e.g. majority groups), their estimation accuracy is questionable for the poorly represented regions.
In particular, \cite{angelopoulos2021gentle} recognizes the lack of guarantees in the performance of CP for such regions.
Besides, the bulk of work on trustworthy AI provides information that {\it supports} the outcome of an ML model. For example, existing work on explainable AI, including~\cite{harradon2018causal,ribeiro2016should,gunning2019darpa}, aims to find simple explanations and rules that justify the outcome of a model.
Conversely, we aim to {\it raise warning signals} when the outcome of a model is {\it not} trustworthy. That is, to provide reasons that {\it cast doubt} on the reliability of the model outcome {for a given query point}.

 %%%%%%%%%%%%%%%%%%%%%%%%%%%%%%%% FUTURE  %%%%%%%%%%%%%%%%%%%%%%%%%%%%%%%%
% \vspace{-3mm}
\section{Final Remarks}\label{sec:conclusion}
As Data-centric AI and Responsible AI emerge as focal points in data science research, the development of Data-centric methodologies for ensuring Responsible and Trustworthy AI attracts increasing attention.
While there is some excellent work on responsible data management to achieve this goal, there remain many challenges yet to be addressed.

In this paper, we focused on a crucial aspect of responsible data -- detecting and addressing the under-representation of minorities within a data set.
We formally defined the notion of data coverage and discussed various techniques for (a) identifying lack of representation issues across different data modalities, (b) ensuring proper representation of minorities in data, and (c) limiting the scope-of-use of data sets based on their representation issues by generating proper ({\sc RU}) warning signals.
Even though the research on detecting lack of coverage issues is relatively mature, resolution techniques are still understudied.
Considering the recent advancements in Generative AI, utilizing Foundation Models and Large Language Models, and studying their limitations, for data augmentation to improve the representation of minorities at the data level seems interesting to further explore.

 %%%%%%%%%%%%%%%%%%%%%%%%%%%%%%%% BIB  %%%%%%%%%%%%%%%%%%%%%%%%%%%%%%%%
\bibliographystyle{unsrt}
\small
% \bibliography{ref}
\begin{thebibliography}{10}

\bibitem{asudeh2019assessing}
A.~Asudeh, Z.~Jin, and H.~Jagadish.
\newblock Assessing and remedying coverage for a given dataset.
\newblock In {\em ICDE}, pages 554--565. IEEE, 2019.

\bibitem{shahbazi2023representation}
N.~Shahbazi, Y.~Lin, A.~Asudeh, and H.~Jagadish.
\newblock Representation bias in data: A survey on identification and resolution techniques.
\newblock {\em ACM Computing Surveys}, 2023.

\bibitem{asudeh2021coverage}
A.~Asudeh, N.~Shahbazi, Z.~Jin, and H.~V. Jagadish.
\newblock Identifying insufficient data coverage for ordinal continuous-valued attributes.
\newblock In {\em SIGMOD}. ACM, 2021.

\bibitem{mousavi2024data}
M.~Mousavi, N.~Shahbazi, and A.~Asudeh.
\newblock Data coverage for detecting representation bias in image datasets: {A} crowdsourcing approach.
\newblock In {\em {EDBT}}, pages 47--60, 2024.

\bibitem{nargesian2021tailoring}
F.~Nargesian, A.~Asudeh, and H.~Jagadish.
\newblock Tailoring data source distributions for fairness-aware data integration.
\newblock {\em Proceedings of the VLDB Endowment}, 14(11):2519--2532, 2021.

\bibitem{nargesian2022responsible}
F.~Nargesian, A.~Asudeh, and H.~V. Jagadish.
\newblock Responsible data integration: Next-generation challenges.
\newblock {\em SIGMOD}, 2022.

\bibitem{sharma2020data}
S.~Sharma, Y.~Zhang, J.~M. R{\'\i}os~Aliaga, D.~Bouneffouf, V.~Muthusamy, and K.~R. Varshney.
\newblock Data augmentation for discrimination prevention and bias disambiguation.
\newblock In {\em AIES}, pages 358--364, 2020.

\bibitem{DBLP:journals/jair/ChawlaBHK02}
N.~V. Chawla, K.~W. Bowyer, L.~O. Hall, and W.~P. Kegelmeyer.
\newblock {SMOTE:} synthetic minority over-sampling technique.
\newblock {\em J. Artif. Intell. Res.}, 16:321--357, 2002.

\bibitem{iosifidis2018dealing}
V.~Iosifidis and E.~Ntoutsi.
\newblock Dealing with bias via data augmentation in supervised learning scenarios.
\newblock {\em Jo Bates Paul D. Clough Robert J{\"a}schke}, 24, 2018.

\bibitem{celis2020data}
L.~E. Celis, V.~Keswani, and N.~Vishnoi.
\newblock Data preprocessing to mitigate bias: A maximum entropy based approach.
\newblock In {\em ICML}, pages 1349--1359. PMLR, 2020.

\bibitem{asudeh2022towards}
A.~Asudeh and F.~Nargesian.
\newblock Towards distribution-aware query answering in data markets.
\newblock {\em Proceedings of the VLDB Endowment}, 15(11):3137--3144, 2022.

\bibitem{motwani1995randomized}
R.~Motwani and P.~Raghavan.
\newblock {\em Randomized algorithms}.
\newblock Cambridge university press, 1995.

\bibitem{chameleon}
M.~Erfanian, H.~V. Jagadish, and A.~Asudeh.
\newblock Chameleon: Foundation models for fairness-aware multi-modal data augmentation to enhance coverage of minorities.
\newblock {\em arXiv preprint arXiv:2402.01071}, 2024.

\bibitem{scholkopf1999support}
B.~Sch{\"o}lkopf, R.~C. Williamson, A.~Smola, J.~Shawe-Taylor, and J.~Platt.
\newblock Support vector method for novelty detection.
\newblock {\em NeurIPS}, 12, 1999.

\bibitem{phillips1998feret}
P.~J. Phillips, H.~Wechsler, J.~Huang, and P.~J. Rauss.
\newblock The feret database and evaluation procedure for face-recognition algorithms.
\newblock {\em Image and vision computing}, 16(5):295--306, 1998.

\bibitem{dressel2018accuracy}
J.~Dressel and H.~Farid.
\newblock The accuracy, fairness, and limits of predicting recidivism.
\newblock {\em Science advances}, 4(1):eaao5580, 2018.

\bibitem{ng2021mlops}
A.~Ng.
\newblock Mlops: From model-centric to data-centric {AI}.
\newblock 2021.

\bibitem{wing2021trustworthy}
J.~M. Wing.
\newblock Trustworthy {AI}.
\newblock {\em CACM}, 64(10):64--71, 2021.

\bibitem{kentour2021analysis}
M.~Kentour and J.~Lu.
\newblock Analysis of trustworthiness in machine learning and deep learning.
\newblock {\em InfoComp}, 2021.

\bibitem{liu2021trustworthy}
H.~Liu, Y.~Wang, W.~Fan, X.~Liu, Y.~Li, S.~Jain, A.~K. Jain, and J.~Tang.
\newblock Trustworthy {AI}: A computational perspective.
\newblock {\em arXiv preprint arXiv:2107.06641}, 2021.

\bibitem{singh2021trustworthy}
R.~Singh, M.~Vatsa, and N.~Ratha.
\newblock Trustworthy {AI}.
\newblock In {\em 8th ACM IKDD CODS and 26th COMAD}, pages 449--453. 2021.

\bibitem{kulynych2022you}
B.~Kulynych, Y.-Y. Yang, Y.~Yu, J.~B{\l}asiok, and P.~Nakkiran.
\newblock What you see is what you get: Distributional generalization for algorithm design in deep learning.
\newblock {\em arXiv preprint arXiv:2204.03230}, 2022.

\bibitem{kakade2003sample}
S.~M. Kakade.
\newblock {\em On the sample complexity of reinforcement learning}.
\newblock University of London, University College London (United Kingdom), 2003.

\bibitem{dwork2012fairness}
C.~Dwork, M.~Hardt, T.~Pitassi, O.~Reingold, and R.~Zemel.
\newblock Fairness through awareness.
\newblock In {\em ITCS}, pages 214--226, 2012.

\bibitem{techrep}
N.~Shahbazi and A.~Asudeh.
\newblock Data-centric reliability evaluation of individual predictions.
\newblock {\em CoRR, abs/2204.07682}, 2022.

\bibitem{adult}
M.~Lichman.
\newblock Adult income dataset, {UCI} machine learning repository.
\newblock \url{https://archive.ics.uci.edu/ml/datasets/adult}, 2013.

\bibitem{neyman1936contributions}
J.~Neyman and E.~S. Pearson.
\newblock Contributions to the theory of testing statistical hypotheses.
\newblock {\em Statistical Research Memoirs}, 1936.

\bibitem{olteanu2019social}
A.~Olteanu, C.~Castillo, F.~Diaz, and E.~Kiciman.
\newblock Social data: Biases, methodological pitfalls, and ethical boundaries.
\newblock {\em Frontiers in Big Data}, 2:13, 2019.

\bibitem{fairmlbook}
S.~Barocas, M.~Hardt, and A.~Narayanan.
\newblock Fairness and machine learning: Limitations and opportunities.
\newblock \url{fairmlbook.org}, 2019.

\bibitem{barocas2016big}
S.~Barocas and A.~D. Selbst.
\newblock Big data's disparate impact.
\newblock {\em Calif. L. Rev.}, 104:671, 2016.

\bibitem{jk2019bias}
J.~Kleinberg.
\newblock Fairness, rankings, and behavioral biases.
\newblock FAT*, 2019.

\bibitem{drosou2017diversity}
M.~Drosou, H.~Jagadish, E.~Pitoura, and J.~Stoyanovich.
\newblock Diversity in big data: A review.
\newblock {\em Big data}, 5(2):73--84, 2017.

\bibitem{berrey2015enigma}
E.~Berrey.
\newblock {\em The enigma of diversity: The language of race and the limits of racial justice}.
\newblock University of Chicago Press, 2015.

\bibitem{dobbin2016diversity}
F.~Dobbin and A.~Kalev.
\newblock Why diversity programs fail and what works better.
\newblock {\em Harvard Business Review}, 94(7-8):52--60, 2016.

\bibitem{simpson1949measurement}
E.~H. Simpson.
\newblock Measurement of diversity.
\newblock {\em Nature}, 163(4148), 1949.

\bibitem{surowiecki2005wisdom}
J.~Surowiecki.
\newblock {\em The wisdom of crowds}.
\newblock Anchor, 2005.

\bibitem{agrawal2009diversifying}
R.~Agrawal, S.~Gollapudi, A.~Halverson, and S.~Ieong.
\newblock Diversifying search results.
\newblock In {\em WSDM}, pages 5--14. ACM, 2009.

\bibitem{mehrabi2021survey}
N.~Mehrabi, F.~Morstatter, N.~Saxena, K.~Lerman, and A.~Galstyan.
\newblock A survey on bias and fairness in machine learning.
\newblock {\em ACM Computing Surveys (CSUR)}, 54(6):1--35, 2021.

\bibitem{friedman1996bias}
B.~Friedman and H.~Nissenbaum.
\newblock Bias in computer systems.
\newblock {\em TOIS}, 14(3):330--347, 1996.

\bibitem{torralba2011unbiased}
A.~Torralba and A.~A. Efros.
\newblock Unbiased look at dataset bias.
\newblock In {\em CVPR 2011}, pages 1521--1528. IEEE, 2011.

\bibitem{crawford2013hidden}
K.~Crawford.
\newblock The hidden biases in big data.
\newblock {\em Harvard business review}, 1(4), 2013.

\bibitem{diakopoulos2015algorithmic}
N.~Diakopoulos.
\newblock Algorithmic accountability: Journalistic investigation of computational power structures.
\newblock {\em Digital journalism}, 3(3):398--415, 2015.

\bibitem{li2020towards}
Y.~Li, H.~Sun, and W.~H. Wang.
\newblock Towards fair truth discovery from biased crowdsourced answers.
\newblock In {\em SIGKDD}, pages 599--607, 2020.

\bibitem{lazier2023fairness}
S.~Lazier, S.~Thirumuruganathan, and H.~Anahideh.
\newblock Fairness and bias in truth discovery algorithms: An experimental analysis.
\newblock {\em arXiv preprint arXiv:2304.12573}, 2023.

\bibitem{SalimiRHS19}
B.~Salimi, L.~Rodriguez, B.~Howe, and D.~Suciu.
\newblock Interventional fairness: Causal database repair for algorithmic fairness.
\newblock In {\em {SIGMOD}}, pages 793--810. {ACM}, 2019.

\bibitem{tae2019data}
K.~H. Tae, Y.~Roh, Y.~H. Oh, H.~Kim, and S.~E. Whang.
\newblock Data cleaning for accurate, fair, and robust models: A big data-{AI} integration approach.
\newblock In {\em DEEM workshop}, pages 1--4, 2019.

\bibitem{salimi2020database}
B.~Salimi, B.~Howe, and D.~Suciu.
\newblock Database repair meets algorithmic fairness.
\newblock {\em ACM SIGMOD Record}, 49(1):34--41, 2020.

\bibitem{martinez2019fairness}
F.~Mart{\'\i}nez-Plumed, C.~Ferri, D.~Nieves, and J.~Hern{\'a}ndez-Orallo.
\newblock Fairness and missing values.
\newblock {\em arXiv preprint arXiv:1905.12728}, 2019.

\bibitem{shahbazi2023through}
N.~Shahbazi, N.~Danevski, F.~Nargesian, A.~Asudeh, and D.~Srivastava.
\newblock Through the fairness lens: Experimental analysis and evaluation of entity matching.
\newblock {\em Proceedings of the VLDB Endowment}, 16(11):3279--3292, 2023.

\bibitem{fanourakis2023fairer}
N.~Fanourakis, C.~Kontousias, V.~Efthymiou, V.~Christophides, and D.~Plexousakis.
\newblock Fairer demo: Fairness-aware and explainable entity resolution.
\newblock 2023.

\bibitem{lin2020identifying}
Y.~Lin, Y.~Guan, A.~Asudeh, and H.~Jagadish.
\newblock Identifying insufficient data coverage in databases with multiple relations.
\newblock {\em Proceedings of the VLDB Endowment}, 13(12):2229--2242, 2020.

\bibitem{accinelli2020coverage}
C.~Accinelli, S.~Minisi, and B.~Catania.
\newblock Coverage-based rewriting for data preparation.
\newblock In {\em EDBT Workshops}, 2020.

\bibitem{accinelli2021impact}
C.~Accinelli, B.~Catania, G.~Guerrini, and S.~Minisi.
\newblock The impact of rewriting on coverage constraint satisfaction.
\newblock In {\em EDBT Workshops}, 2021.

\bibitem{shetiya2022fairness}
S.~Shetiya, I.~P. Swift, A.~Asudeh, and G.~Das.
\newblock Fairness-aware range queries for selecting unbiased data.
\newblock In {\em ICDE}. IEEE, 2022.

\bibitem{tae2021slice}
K.~H. Tae and S.~E. Whang.
\newblock Slice tuner: A selective data acquisition framework for accurate and fair machine learning models.
\newblock In {\em SIGMOD}, pages 1771--1783, 2021.

\bibitem{chung2019slice}
Y.~Chung, T.~Kraska, N.~Polyzotis, K.~H. Tae, and S.~E. Whang.
\newblock Slice finder: Automated data slicing for model validation.
\newblock In {\em ICDE}, pages 1550--1553. IEEE, 2019.

\bibitem{sagadeeva2021sliceline}
S.~Sagadeeva and M.~Boehm.
\newblock Sliceline: Fast, linear-algebra-based slice finding for ml model debugging.
\newblock In {\em SIGMOD}, pages 2290--2299, 2021.

\bibitem{kleinberg2016inherent}
J.~Kleinberg, S.~Mullainathan, and M.~Raghavan.
\newblock Inherent trade-offs in the fair determination of risk scores.
\newblock {\em arXiv preprint arXiv:1609.05807}, 2016.

\bibitem{zadrozny2001obtaining}
B.~Zadrozny and C.~Elkan.
\newblock Obtaining calibrated probability estimates from decision trees and naive bayesian classifiers.
\newblock In {\em ICML}, volume~1, pages 609--616. Citeseer, 2001.

\bibitem{zadrozny2002transforming}
B.~Zadrozny and C.~Elkan.
\newblock Transforming classifier scores into accurate multiclass probability estimates.
\newblock In {\em SIGKDD}, pages 694--699, 2002.

\bibitem{platt1999probabilistic}
J.~Platt et~al.
\newblock Probabilistic outputs for support vector machines and comparisons to regularized likelihood methods.
\newblock {\em Advances in large margin classifiers}, 10(3):61--74, 1999.

\bibitem{niculescu2005predicting}
A.~Niculescu-Mizil and R.~Caruana.
\newblock Predicting good probabilities with supervised learning.
\newblock In {\em Proceedings of the 22nd international conference on Machine learning}, pages 625--632, 2005.

\bibitem{chatfield93predictionintervals}
C.~Chatfield.
\newblock Prediction intervals.
\newblock {\em Journal of Business and Economic Statistics}, 11:121--135, 1993.

\bibitem{pearce2018high}
T.~Pearce, A.~Brintrup, M.~Zaki, and A.~Neely.
\newblock High-quality prediction intervals for deep learning: A distribution-free, ensembled approach.
\newblock In {\em International conference on machine learning}, pages 4075--4084. PMLR, 2018.

\bibitem{khosravi2010lower}
A.~Khosravi, S.~Nahavandi, D.~Creighton, and A.~F. Atiya.
\newblock Lower upper bound estimation method for construction of neural network-based prediction intervals.
\newblock {\em IEEE transactions on neural networks}, 22(3):337--346, 2010.

\bibitem{angelopoulos2021gentle}
A.~N. Angelopoulos and S.~Bates.
\newblock A gentle introduction to conformal prediction and distribution-free uncertainty quantification.
\newblock {\em arXiv preprint arXiv:2107.07511}, 2021.

\bibitem{shafer2008tutorial}
G.~Shafer and V.~Vovk.
\newblock A tutorial on conformal prediction.
\newblock {\em Journal of Machine Learning Research}, 9(3), 2008.

\bibitem{harradon2018causal}
M.~Harradon, J.~Druce, and B.~Ruttenberg.
\newblock Causal learning and explanation of deep neural networks via autoencoded activations.
\newblock {\em arXiv preprint arXiv:1802.00541}, 2018.

\bibitem{ribeiro2016should}
M.~T. Ribeiro, S.~Singh, and C.~Guestrin.
\newblock " why should i trust you?" explaining the predictions of any classifier.
\newblock In {\em SIGKDD}, pages 1135--1144, 2016.

\bibitem{gunning2019darpa}
D.~Gunning and D.~Aha.
\newblock Darpa’s explainable artificial intelligence ({XAI}) program.
\newblock {\em AI Magazine}, 40(2):44--58, 2019.

\end{thebibliography}

\end{document}

\end{article}



\end{articlesection}

% put the news items below- there can be multiple news sections
% each with its own title
% news will usually have an author as well as a title, 
% e.g. TCDE elections
% news articles are in the same format as letters
% typically, news articles will be stored in a directory called "news"

%\begin{newssection}{News headline}

% insert news items here; news will typically have authors
% see the Sept. 2018 issue for an example

%\begin{news}{news item title}
%{author name}{author affiliation}
%\input{news/news-article.tex}
%\end{news}
%
%\newpage


%\end{newssection}



\begin{callsection}

%  This section will be empty for your version
%
%  Calls for papers section.  Use the callsection environment.
%  Each call for papers is contained in an call environment, where the single 
%  required options to \begin{call} is the name of the conference.
% typically calls are stored in a "calls" directory
%
%\begin{call}{name of conference}
%\centerline{\includegraphics[width=\textwidth, bb= 0 0 590 760]{calls/conference-name.pdf}}
%\end{call}
%\begin{call}{ICDE 2019 Conference}
%\centerline{\includegraphics[width=\textwidth, bb= 0 0 610 790] {../Dec-2018/calls/icde19.pdf}} 
%\centerline{\includegraphics[width=\textwidth, bb= 0 0 590 760] {calls/icde19.pdf}}
%\end{call}
\begin{call}{TCDE Membership Form}
%\centerline{\includegraphics[width=\textwidth, bb= 0 0 610 790]
\centerline{\includegraphics[width=\textwidth, bb= 0 0 590 760] {../Dec-2018/calls/tcde.pdf}}
\end{call}

\end{callsection}

\end{bulletin}
\end{document}
