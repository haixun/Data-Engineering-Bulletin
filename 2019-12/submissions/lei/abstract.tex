\begin{abstract}
Online crowdsourcing platforms have proliferated over the last few years and cover a number of important domains, these platforms include from worker-task platforms such Amazon Mechanical Turk, worker-for-hire platforms such as TaskRabbit to specialized platforms with specific tasks such as ridesharing like Uber, Lyft, Ola etc.
An increasing proportion of human workforce will be employed by these platforms in the near future.
The crowdsourcing community has done yeoman's work in designing
effective algorithms for various key components, such as incentive design, task assignment and quality control. Given the increasing importance of these crowdsourcing platforms,
it is now time to design mechanisms so that it is easier to evaluate the effectiveness of these platforms. Specifically, we advocate developing benchmarks for crowdsourcing research.

Benchmarks often identify important issues for the community to focus and improve upon.
This has played a key role in the development of research domains as diverse as
databases and deep learning.
We believe that developing appropriate benchmarks for crowdsourcing will ignite further innovations.
However, crowdsourcing -- and future of work, in general -- is a very diverse field
that makes developing benchmarks much more challenging.
Substantial effort is needed that spans across developing benchmarks for
datasets, metrics, algorithms, platforms and so on.
In this article, we initiate some discussion into this important problem and
issue a call-to-arms for the community to work on this important initiative.
\end{abstract}
