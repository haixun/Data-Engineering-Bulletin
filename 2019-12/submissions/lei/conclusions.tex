\section{Conclusion}\label{sec:conclusions}
Alternative job arrangements such as online job platforms are becoming increasingly important
and a major source of employment in the near future.
It is important for the crowdsourcing community to take the lead on researching the Feature of Work.
In this article, we argue that a key pre-requisite is the creation of benchmarks for such research.
The diversity of crowdsourcing research by various communities is a key challenge.
We propose a taxonomy of dimensions for which benchmarks has to be developed.
We also have enumerated a list of metrics that are most relevant for effective crowdsourcing. Moreover, we list essential factors that need to be considered during the process of creating benchmarking data to test the effectiveness and robutness of crowdsouring platforms. Benchmarks have had a dramatic impact in the development of various domains.
We issue a call-to-arms to the crowdsourcing community to seize the opportunity!
