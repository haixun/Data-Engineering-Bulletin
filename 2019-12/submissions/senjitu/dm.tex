\vspace{-0.2in}
\section{Data Model}\label{dm}
\vspace{-0.1in}
We introduce different variables and characterize human factors~\cite{hf1,motiv00,motiv0,motiv1,motiv2,motiv3,motiv4}. A crowdsourcing platform typically comprises of workers and tasks that serve as the foundation of the framework we propose. We also note that not all the variables are pertinent to every application domain (for example, citizen science applications are usually voluntary contributions). Our effort is to propose a generalization nevertheless. 

{\bf Domains/types:} A set $D=\{d_1,d_2,\ldots,d_{l}\}$  of given domains is used to describe the different types of tasks in an application. Using Example~\ref{ex1}, a particular species may construe a domain. 

{\bf Workers:} A set of $m$ human workers $\mathcal{U}=\{u_1,u_2,\ldots,u_m\}$ are available in a crowdsourcing platform.
% Each worker $u$ is described by a set of  {\em human factors}~\cite{hf1,motiv00,motiv0,motiv1,motiv2,motiv3,motiv4}, i.e., variables that are important to understand their behavior in a crowdsourcing platform.

{\bf Tasks and sub-tasks:} A task $\mathcal{T}$ is a hybrid human-machine computational task (classification for example), with a quality condition $Q^\mathcal{T}$  and an overall monetary budget $B^\mathcal{T}$ that decide its termination. Using Example~\ref{ex1}, $\mathcal{T}$ is a classification task which terminates, when $Q^\mathcal{T} = 80\%$ accuracy is achieved, or $B^\mathcal{T} =\$100$ is exhausted.

Without loss of generality, $\mathcal{T}$  comprises of a set of $n$ subtasks, i.e., $\mathcal{T} =\{t_1,t_2,\ldots,t_n\}$. These sub-tasks are of interests to us,  as workers will be involved to undertake these sub-tasks. Each sub-task can either be performed by human workers or computed (inferred) by machine algorithms. We consider {\em pool based active learning}, where a finite pool of sub-tasks exists and given.

{\em Sub-tasks:}
For single label, a sub-task is an unlabeled instance of the data that requires labeling. Considering Example~\ref{ex1}, this is analogous to confirming the presence or absence of a species in a particular geographic location. 
%To generalize, we consider similar settings as that of {\em pool-based active learning}, where a pool of unlabeled instances are available for further labeling.
For multi-label scenario, a sub-task requires multiple labels to be obtained. Using Example~\ref{ex1}, this is analogous to obtaining Kingdom, Phylum, Class, Order, etc of the insect.

{\bf Worker Response:} We assume that a worker $u$'s {\em response to a particular sub-task $t$ may be erroneous, which is used by the machine algorithm in one or more rounds of interactions}. Our framework may ask multiple workers to undertake the same task to reduce the error probability, and may decide which questions to ask in the next round to whom based on the answers obtained in the previous round.

{\bf Human Factors:} These are the variables that characterize the behavior of the workers  in a crowdsourcing platform~\cite{hf1,motiv00,motiv0,motiv1,motiv2,motiv3,motiv4}.

{\em Skill (Expertise/Accuracy):}  Worker's skill in a domain is her expertise/accuracy. Skill of a worker in a domain $d$ is quantified in a continuous $[0,1]$ scale (to allow a probabilistic  interpretation). A worker $u$ may have skills in one or more domains (e.g., different species observation accuracy). 
%Given $l$ domains, $u$'s skill is described by a $l$-dimensional vector, where $s^{u_i}$ is her skill in the $i$-th domain. 

{\em Wage:} A worker $u$ may have a fixed wage $w_u$, or may have to accept the wage a particular task offers. $u$'s may have different wage for different types of tasks.

{\em Motivation:} Motivation aims at capturing the worker's willingness to perform a task. A related work~\cite{motiv1} proposes a theoretical foundation in motivation theory in crowdsourcing platform and characterizes them in two different ways: 

{\em (a) Intrinsic motivation:} Intrinsic motivation exists if an individual works for fulfillment generated by the activity (e.g. working just for fun). Furthermore, related works~\cite{motiv1,motiv2,motiv3} have identified that intrinsic motivation emerges in the following ways: (1) skill variety (refers to the extent to which a worker can utilize multiple skills), (2) task identity (the degree to which an individual produces a whole, identifiable unit of work, versus completion of a small unit which is not an identifiable final product), (3) task significance (the degree to which the task has an influence over others), (4) autonomy (the degree to which an individual holding a job is able to schedule his or her activities), (5) feedback (the extent to which precise information about the effectiveness of performance is conveyed). 

%Hackman and Oldham~\cite{hackman1976motivation} have combined these factors mathematically and defined {\em motivating potential score (MPS)} to capture intrinsic motivation:

%\begin{equation}\label{eqn}
%\begin{aligned}
%MPS= \frac{\text{skill-variety} + \text{task-identity} + \text{task-significance}}{3}  \\
%* \text{ autonomy } * \text{ feedback}
%\end{aligned}
%\end{equation}


{\em (b) Extrinsic motivation:} Extrinsic motivation is  an instrument for achieving a certain desired outcome (e.g. making money).

%{\em Acceptance ratio:} Acceptance ratio describes the {\em probability} at which a worker actually participates in a task (notice that a worker can always decline a task).Acceptance ratio may be correlated to worker motivation.
 %Given $l$ domains, $u$'s acceptance ratio is described by a $l$-dimensional vector $A^u$. 

The challenge however is, either the values of these factors have to be explicitly given or they have to be estimated. Related works, including our own, have proposed solutions to estimate skill~\cite{skill,DBLP:conf/kdd/JoglekarGP13} by analyzing historical data. %Acceptance ratio or wage of the workers are estimated by designing surveys and asking explicit questions~\cite{roy2015task}.
Nevertheless, we are not aware of any effort that models motivational factors or design optimization involving them.

{\em Worker specific constraints:} Additionally, a worker may specify certain constraints (e.g., can not work more than $6$ hours, or travel farther than $10$ miles from her current location).

{\bf Characterizing sub-tasks considering human factors:}  It is easy to notice that the motivational factors described above are actually related to tasks (i.e, sub-tasks). 

Formally, we describe that a set $A$ of attributes or meta-data is available to characterize each sub-task $t$. They are its required skill-domain\footnote{\small for simplicity, we assume that each sub-task requires one skill, whereas, in reality, multiple skills may be needed for a sub-task. The latter assumption is trivially extensible by our framework.} $s^t$ , cost/wage $w^t$, duration $time^t$, location $location^t$, significance $sig^t$, identity $iden^t$, autonomy $auto^t$, task feedback $fb^t$. Each $t$, if performed correctly, contributes by a quantity  $q^t$ to $Q^\mathcal{T}$. These contributions are purely dictated by the active learning principles, such as how much it reduces the uncertainty.



%{\em Unlike other human factors, there does not exist any mathematical model that captures human motivation}. We therefore make an effort to mathematically model human motivation. 

 %and has a cost $b^t$. The cost of a sub-task is the money that has to be paid when it is performed by a human worker.

 






