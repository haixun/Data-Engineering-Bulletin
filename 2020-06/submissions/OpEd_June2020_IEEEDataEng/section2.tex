
\section{Emerging Research: The Challenge of the Unknown}

Our landscape has shifted, and our paths forward will require unparalleled levels of creativity, persistence, and humility as we strengthen connections across sectors and disciplines, translating ideas into action. 
Although the data are rarely complete and extracting useful signals from the noise can be elusive, a number of COVID-19 data-enabled research efforts are emerging, including:

\subsection*{Analysis and Modeling}
Considerable emphasis on data analysis and modeling has spurred discussion that bridges statistical methods, machine learning, and artificial intelligence with policy development and risk communication~\cite{Lewnardm1923, best_boice_2020, institute, covidactnow}. 
For example, N. Alteri et al. have developed several predictors after curating available data, collaborating with nonprofit organizations and healthcare providers to address medical supply needs for individual hospitals~\cite{altieri2020curating}. 
U. Seljak et al. have employed robust statistical methods to identify systematic errors and correct mortality rates that are integral to further analysis~\cite{Modi2020.04.15.20067074}, while S. Yadlowsky et al. have modeled the infection prevalence of the virus~\cite{Yadlowsky2020.03.24.20043067}, with J. Steinhardt and A. Ilyas noting shortcomings of existing tracking measures~\cite{steinhardt}. 

Such efforts can help communities visualize the nature of fluid events and iteratively explore reasonable response strategies when faced with unprecedented scenarios. Notably, as communities across the globe adjust their behavior against a backdrop of changing policies, guidance, and tactics, we have an opportunity to improve our assessments about the spread of the virus and to understand how our actions and other factors relate to key outcomes. Intentional and thoughtful data collection and analysis during recovery, focused on public health risks and implications—and acknowledging the critical lag between knowledge and informed action—will be of paramount importance.

\subsection*{Exploration to Aid Medical Therapeutics}
As researchers and medical professionals from a variety of disciplines seek to accelerate actionable knowledge, collaborative frameworks and data exploration efforts are forming. National laboratories and science centers are searching databases for drug candidates that could be re-purposed for COVID-19 or used to inform clinical practice, sharing and updating progress more rapidly than traditional publication timeframes~\cite{Smith2020, Chen2020.04.17.047548}. Consortia aiming to speed dialogue and visualization of SARS-CoV-2 genomes through open online reports~\cite{nextstrain}, as well as clinical study groups and collaborators working together in new ways, have the potential to shape and improve diagnostics, vaccine development, and outcomes~\cite{schneider2019rethinking, kaggle}. 