\section{Shared Human-Centered Context: The Need To Explore Difficult Questions, Together}

As our collective response to the pandemic evolves, openly updating and sharing code, data, and insights as they develop is particularly critical for scientific reproducibility and effective coordination. Yet, a tension exists when evaluating the quality of information and choosing how to provide data, analysis, or tools to others. Noting the importance of clear, accurate, updated, and actionable information, the State of California coronavirus response team published a Digital Crisis Standard~\cite{crisisstandard} with principles and framing questions that highlight accessibility, interoperability, and a focus on user needs and accountability. 

Indeed, in the realm of disaster response and recovery, essential questions and unexplored frontiers arise in a fundamentally human-centered context. Issues stemming from ethics concerns and socioeconomic implications \emph{must} be addressed early and often. Existing frameworks, theories, and practices that cut across disciplines are being put to the test and reconsidered as we face COVID-19, including:

\subsection*{Incorporating Privacy by Design}
Adopted by the International Assembly of Privacy Commissioners and Data Protection Authorities nearly a decade ago, the foundational principles of the Privacy by Design framework~\cite{cavoukian2013privacy} build upon views and practices emphasizing proactive, “by default”, and embedded mechanisms that respect user privacy. With a backdrop of the General Data Protection Regulation and increasing prevalence of data privacy acts, laws, and norms~\cite{GDPR, mares2019iot}, emerging work towards privacy-sensitive proximity contact tracing looks to enable data exchange without compromising civil liberties, to query encrypted data, and to comply with unfolding guidelines for privacy safeguards~\cite{FPF}.

Sharing information that could be relevant for individual or community health outcomes surfaces a number of deeper questions, where personal sentiment and expectations can vary widely. \emph{What information should remain private, under what circumstances? What might I share in hopes of helping others? Will my privacy preferences change? Am I empowered to make choices about my privacy and data?} The tensions surrounding privacy and nuances around data sharing, while not new by any means~\cite{10.1007/3-540-45427-6_23, 10.1145/642611.642635}, are integral to our COVID-19 response and recovery.   

\subsection*{Addressing Disparities and Inequities}
With increased capacity to view data about disease incidence and mortality across demographic information, analysis showing disproportionate COVID-19 impacts on black communities, Latinx communities, and additional historically marginalized groups has surfaced~\cite{covidrace}. Z. Obermeyer et al. have noted how existing measures can obscure rather than demonstrate inequality~\cite{Obermeyer447}. For example, communities with less access to testing for SARS-CoV-2 will have fewer diagnosed cases, making the epidemic look less severe. This can lead to disparities in attention and funding, and distort algorithms meant to help. 

Networks of government cohorts have featured data-driven equity visualization tools and are increasingly convening public discussions focused on inclusion, a sense of belonging, and equity~\cite{GARE}. In order to support meaningful change, we will need to find ways to support empathy and shared understanding, directing energy towards measurable improvements. Partnerships to enable data compilation, analysis, and deeper research are essential, but we will need to leverage our ingenuity holistically and hold ourselves accountable to the bigger picture. \emph{What levers do we have to address systemic issues? Who is missing from the conversation? How might we support public awareness, engagement, and education, creatively?} 

Globally, with 265 million people projected to suffer from acute hunger by the end of this year, data-driven analysis is creating new options to help mitigate the most devastating impacts of COVID-19 in low- and middle-income countries.  For example, applications of machine learning to satellite imagery and mobile phone data are helping identify those individuals most in need of immediate humanitarian aid, and complement conventional methods that are limited by a lack of reliable and up-to-date data~\cite{blumenstock2020machine, blumenstock2016fighting}. 

\subsection*{Mechanisms for Supporting Risk-Based Action}
Design through the lens of risk underscores the need to better quantify and qualify threats, vulnerabilities, and consequences. Throughout the pandemic, the corpus of available information and the assessment of risk are continuously changing as more is discovered about the virus and how it interacts with our communities. In a world where the opportunity for technology-driven situational awareness during a crisis is strikingly juxtaposed with an “infodemic” of extensive misinformation and dis-information~\cite{zade, UN, Hany}, we are starving for transparent, appropriately vetted, and curated information in context.  

In the clinical setting, aggregating data to capture patient risk indices could help communities more methodically assess and navigate difficult situations with new information, but effectively focusing attention and limited resources towards optimized patient outcomes remains an ongoing, and often overwhelming, challenge ~\cite{zhou2020clinical, onder2020case}. In all too many cases, the pandemic is exacerbating existing vulnerabilities. Any “solutions” or approaches need to consider the nature of risks and how risk evolves over time.  
