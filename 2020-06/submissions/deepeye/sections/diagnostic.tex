%!TEX root = ../main.tex
\section{Diagnostic Data Analytics of COVID-19}
\label{sec:diagnostic}
Another purpose of data visualization is to perform hypothesis testing, we show how to design visualizations to test two hypotheses -- urban (population) density \textit{vs.} total confirmed cases, and temperature \textit{vs.} total confirmed cases.

\stitle{Urban (Population) Density.}
One intuitive hypothesis is that whether high population densities catalyze the spread of COVID-19? 
Since we want to know the relationship (correlation) between the population density and the spread of COVID-19, we first visualize the population density in the map named \textit{population density map}, and then we map the confirmed cases on the top of \textit{population density map}. 
As shown in Figure~\ref{fig:diagnostic}(a), it takes United States as an example. It shows that in areas with high population density and without lockdown policy, \eg New York and California, more people are infected with coronavirus.
For example,  New York with a relatively high population density is likely more vulnerable to the spread of the coronavirus. 
This conclusion is reasonable, because the intensive contact greatly increases the probability of coronavirus transmission~\cite{rocklov2020high}.

\stitle{Temperature.}
We also design visualization to show the relationship between the outbreak of COVID-19 and the temperature factor.  Similarly, we first visualize heatmap using temperature data, and then we map the confirmed cases on the top of the heatmap.
As shown in Figure~\ref{fig:diagnostic}(b), it is hard for us to make conclusions like the higher temperature, the more infected cases, or the lower temperature, the less infected cases. 
For example, the average temperature in the central United States is lower than in California, but there are also hundreds of thousands of infected people in the central United States. Comparing Figure~\ref{fig:diagnostic}(a) and Figure~\ref{fig:diagnostic}(b), we can find that under the background of no lockdown, the population density has a stronger correlation with the number of confirmed cases.
