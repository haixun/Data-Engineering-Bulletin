%!TEX root = ../main.tex
\section{Introduction}
\label{sec:intro}

\subsection{Where do we stand today on COVID-19?}

\stitle{The history of pandemics.} 
Nothing has killed more people than infectious disease throughout the human history\footnote{https://www.visualcapitalist.com/history-of-pandemics-deadliest/}. Many pandemics changed history; for example, the Antonine Plague (165-180) and the Black Death (1347-1351) have changed the history of Europe.
%
Situations are getting worse in the last two decades, because epidemics happened much more frequently than what we have seen in history: SARS (2002-2003), Swine Flue (2009-2010), MERS (2012-present), Ebola (2014-2016), and now the COVID-19 (2019-present).


\stitle{The history of COVID-19.}
Shortly after the first confirmed case in early January 2020, and a statement at January 21 from WHO's mission to China saying that there was evidence of human-to-human transmission, COVID-19 quickly spread out to almost every corner of the world. WHO officially named it a pandemic at March 11 2020. Till the date of June 1, 2020, there are more than 5.5 million confirmed cases and it has caused more than 350K deaths worldwide.

\stitle{No system to stop a pandemic.}
Bill Gates envisioned, during his TedTalk on 2015\footnote{https://www.ted.com/talks/bill\_gates\_the\_next\_outbreak\_we\_re\_not\_ready?language=en}, that if something will kill more than 10 million people in the next few decades, it will be infectious disease rather than wars.
He has also questioned: ``Are we ready for the next outbreak?'' 
Sadly, the answer was not that the current health systems do not work for pandemics. 
Instead, there is {\em no global health system at all} for such pandemics.


\stitle{Where do we stand today?}
Undoubtedly, COVID-19 has changed and will keep changing our history from many different dimensions, such as living style, studying style, the way of traveling, among many others. Indeed, we are not even at the peak of the 1st wave of COVID-19, and there might have the 2nd and the 3rd waves, such as the 1918 influenza pandemic and the 2009-2010 H1N1 pandemic, where the 2nd and 3rd waves were much more deadly than the 1st wave -- we need to buckle up because the ride is just beginning.
In order to stop or get better prepared to stop pandemics, global public health systems need to be ready, like military is ready for wars. 
%
Meanwhile, to help stop pandemics, {\em data science} has played a key role in discovering insights to guide decision makers and general people to make wise and informative decisions.

\subsection{Do we have good data science systems for monitoring and exploring COVID-19?}

\stitle{From the real world pandemics to the virtual world infodemics.}
As the WHO Director-General Tedros Adhanom Ghebreyesus said\footnote{https://www.who.int/dg/speeches/detail/munich-security-conference}: ``We’re not just fighting an epidemic; we’re fighting an {\em infodemic}.'' Here, the term ``infodemic'' generally refers to an excessive amount of information about a problem, which makes it difficult to identify a solution. 
%
Similar to the pandemics in the real world, infodemics are in the virtual world, where the goals of scientists in many domains are to deal with the information from the infodemics of the virtual world, so as to find meaningful insights that can be used to fight against the pandemics of the real world.
%


\stitle{No data science system to stop an infodemic.}
Unfortunately, similar to the case there is no global health system that is ready to stop a pandemic, there is no data science system that is ready to deal with infodemics, even if we have seen the fruitful successes from many communities for data science, such as database, data mining, machine learning, natural language processing, bioinformatics, and many others.
%
Essentially, all scientific methods are based on empirical or measurable {\em evidence} that is subject to the principles of logic and reasoning.
%
In terms of infodemics, the evidence is the data that has been collected.
%
However, the central problems are:
%
(i) {\em Data is inaccurate:} the real confirmed cases and the number of death are conjectured to be much higher than the reported numbers.
%
(ii) {\em Data is missing:} nobody knows precisely when and where did COVID-19 start, therefore many data is missing from its origin to the date that the data was first collected.
%
(iii) {\em Data is inconsistent:} there are misinformation, disinformation, and rumors that are widely spread.
%
(iv) {\em Data is not directly comparable:} different countries calculate mortality rate in different measures, \eg the death totals compiled by US CDC include `probable' cases starting from April 16, 2020\footnote{https://edition.cnn.com/2020/04/15/health/us-coronavirus-deaths-trends-wednesday/index.html}, while UK mainly considers the cases from the hospitals from the department and health of social care (DHSC) data\footnote{https://www.health.org.uk/news-and-comment/blogs/understanding-the-data-about-covid-19-related-deaths}.


\stitle{The goal of an ideal data science system for infodemics.}
The {\em dark side} of infodemics is that {\em we never have the data well prepared for deriving the truth in any context}. Note, however, that the only certainty in (data) science is uncertainty, just like every single decision we have ever made. Hence, the {\em bright side of the dark side} is that this infodemics dilemma forces us to re-evaluate and re-design existing data science systems, for fighting against infodemics.
%
Broadly speaking, the main goal of an ideal data science system for infodemics is about giving that data a purpose, which can provide hints to guide wise decisions. For example, although the reproduction number (Ro, pronounced R-nought or r-zero) of COVID-19 keeps changing, we are certain that {\em social distancing} is important. 
%
More concretely, an ideal data science system should be able to tackle all traditional data analytical tasks but under the situation of infomedics -- data is very messy (the above i--iv) and new (and conflicting) data keeps coming -- that can:
(1) effectively collect, integrate and clean data from different sources;
(2) make {\em descriptive analytics} to tell what happened in the past; 
(3) conduct {\em diagnostic analytics} to help understand why something happened in the past; 
(4) do {\em predictive analytics} to predict what will happen in the future; and 
(5) perform {\em prescriptive analytics} to recommend actions that one can take to affect those outcomes.


\subsection{\textsc{DeepEye}: A small step towards a (ideal) data science system for infodemics}

In this paper, we present \sys, an end-to-end data science system for collecting, cleaning, analyzing, and visualizing COVID-19 data. Since the system was launched in {05/02/2020}, we have accumulated more than 2 million visits in a month. Some news media in China have reported our system\footnote{https://news.tsinghua.edu.cn/info/1416/77464.htm}\footnote{https://www.aminer.cn/research\_report/5e774aa1e34bad84366f1f7e?download=false}, and we have launched online public courses to give tutorial for the system\footnote{https://www.bilibili.com/video/BV1FE411W7p4/?spm\_id\_from=333.788.videocard.0}, which attracted nearly 100,000 people to participate. 
Besides, as a research-oriented platform, some research institutions (\eg CMU, The University of Hong Kong) also conduct preliminary processing and analysis of COVID-19 epidemiological data based on our platform.

%\stitle{System architecture.} 
%We will overview the architecture of \sys (Figure~\ref{fig:framwork}), which consists of three layers, the data preparation layer, the data analytics layer, and the user interaction layer (Section~\ref{sec:system}).
Generally speaking, \sys consists of three layers: data preparation layer, data analytics layer, and user interaction layer (Section~\ref{sec:system}).


In {\em data preparation layer}, a key observation we made is that it is impossible to prepare the data in the traditional way for some tasks, simply because data preparation is expensive and error-prone, especially when the data keeps changing every day. In particular, we will discuss task-driven data preparation, with the basic intuition that it is much cheaper to prepare the data that is needed for a given task (Section~\ref{sec:dataprep}).

The {\em data analytics layer} contains several components.
%
For {\em descriptive analytics}, we use linked data visualization to provide sufficient context for a user to understand what happened in the past. We also use visualization recommendation techniques that can automatically discover interesting stories (Section~\ref{sec:descriptive}).
%
For {\em diagnostic analytics}, we show that by combining with other (domain) knowledge, we can test our hypotheses (\eg the effect of urban (population) density or temperature to COVID-19) about why something happened (Section~\ref{sec:diagnostic}). 
%
For {\em predictive analytics}, we have tried some predictive model, in particular Susceptible-Exposed-Infectious-Recovered (SEIR)~\cite{carcione2020simulation1}, that has been widely used for epidemiology. However, our empirical results show that the prediction for COVID-19 is typically inaccurate that may due to the messy data of infodemics, hence we will not discuss further about it in this paper.
%
For {\em prescriptive analytics}, we will discuss our collaboration with China mobile that help the government to recommend actions (Section~\ref{sec:prescriptive}).

The discussion of {\em user interaction layer} will be blended with the discussion of data analytics layer, while describing various use cases.
