\documentclass[11pt]{article} 

\usepackage{deauthor,times,graphicx}
%\usepackage{url}
\usepackage{hyperref}

\begin{document}
In this issue, we continue our discussions on data science. Following
Jeff Ullman's opinion piece in the June 2020 issue titled ``The battle
for data science,'' we feature an opinion piece from M. Tamer
\"{O}zsu. \"{O}zsu brings up a few critical questions about data
science ranging from ``Is data science a discipline?'', ``Who is a
data scientist?'' to ``Who owns data science?'' These questions are
important not only for academia but also in the tech industry, where
``data science'' remains a vague term to this day. Different tech
companies often define data science in different ways, resulting in an
identity crisis for data scientists as they balance their roles as
applied scientists, machine learning engineers, statisticians, data
analysts, data engineers, etc. More importantly, as the discipline is
not well-defined, tools, platforms, and eco-systems cannot be
``standardized,'' which means they are unlikely to be
perfected. Please read \"{O}zsu's opinion piece ``A Systematic View of
Data Science.''

Lei Chen organized this special issue on the topic of human-powered AI
or human-in-the-loop AI. While these are not exactly new topics as
there has been a lot of work in areas ranging from crowdsourcing to
active learning, the topic of making AI and humans working together
and benefiting each other is gaining a lot of traction. In this issue,
we look into problems such as how to divide labor between humans and
machines, how to utilize crowdsourcing on federate learning, how to
provide end-to-end solution in human-in-the-loop systems, how to use
AI to model human behavior, and how human-powered AI can help with
online misinformation detection.


\end{document}

