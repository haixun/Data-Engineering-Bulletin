%\vspace{-.5cm}
\section{Conclusion}
\label{sec:conclusion}

In the recent years, more and more information on tasks is being circulated on the internet, and there are many labor resources accessible through it.
In addition, we see the rapid growth of AI technology that allows us to have them workers for our tasks.
This motivates us to take the computational approach to division of labor, in which we use AI agents that implement algorithms  to assign appropriate tasks to human and AI workers to achieve specific goals. 
This paper investigated problems on computational division of labor around crowdsourcing and data-centric human-in-the-loop systems research. 
We explained what computational division of labor problems deal with and introduced a set of dimensions and terms to classify existing solutions for problems related to this topic. We identified the current status of this topic and showed that there are a number of interesting research challenges.

\section*{Acknowledgment}
This work was partially supported by JST CREST Grant Number JPMJCR16E3, Japan.


