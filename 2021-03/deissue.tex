\documentclass[11pt]{article}
\usepackage{debulletin,times,epsfig,subfigure,wrapfig,algorithmic,color,boxedminipage,graphicx,url}

% this is the template for an issue of the Data Engineering Bulletin

% all packages used by any paper must be listed here
%\usepackage{hyperref}
%\usepackage{authblk}
%\setlength{\affilsep}{0em}
%\usepackage{inputenc}
%\usepackage{debulletin}
%\usepackage{times}
%\usepackage{graphicx}
%\usepackage{array}
%\usepackage{wrapfig}
%\usepackage[table]{xcolor}
%\usepackage{tcolorbox}
%\usepackage{amssymb}
%%\usepackage[labelfont=bf,labelsep=space,list=true]{subcaption}
%\usepackage{url}
%\usepackage{mathtools, bm}
%\usepackage{float}
%\usepackage{multirow}
%\usepackage{multicol}
%\usepackage{algorithm}
%\usepackage{subfig}
%\usepackage{algpseudocode}
%\setlength{\intextsep}{10pt plus 2pt minus 2pt}
%\hyphenation{finally}
\usepackage[utf8]{inputenc}

\usepackage{amsmath, amssymb, amsfonts}

\usepackage{hyperref}
\usepackage{enumitem}
\usepackage{xspace} 
%\usepackage{xcolor}
\usepackage{tikz}
\usepackage[T1]{fontenc}
\usepackage{beramono}
\usepackage{listings}
\usepackage{xcolor}



\usepackage{wrapfig}

\usepackage{graphics}
\usepackage{pifont}
%\usepackage{subcaption} % for subtable
%\usepackage{threeparttable} % for tablenotes




\usepackage[numbers]{natbib}
% \documentclass{article}
% Recommended, but optional, packages for figures and better typesetting:
\usepackage{microtype}
\usepackage{graphicx}
% \usepackage{subfigure}
\usepackage{booktabs} % for professional tables
%\usepackage[table,dvipsnames]{xcolor}
\usepackage{epsfig}
\usepackage{pgfplotstable}
\usepackage{pgfplots}
\usepgfplotslibrary{groupplots}
\usepackage{bbm}
\usepackage{booktabs}
\usepackage{verbatim}
\usepackage[T1]{fontenc}
\usepackage{caption}
\usepackage{siunitx}
\usepackage{xspace}
%\usepackage[colorinlistoftodos,textsize=footnotesize]{todonotes}
%\usepackage[utf8]{inputenc}
\usepackage[autostyle, english=american]{csquotes}
\usepackage{breakurl}




\begin{document}


% please enter real date, vol no, issue no
\bulletindate{March 2021}
\bulletinvolume{43}
\bulletinnumber{4}
\bulletinyear{2021}

% these are files that I have- but your part of the issue can be done without
% them
\IEEElogo{cs.pdf}
\insidefrontcover{incvA19.pdf}
%\insidebackcover[ICDE Conference]{./calls/icde-new-a.ps}

\begin{bulletin}

% the above samples assume the issue is generated from a directory structure of the following sort
% major directory name is month and year of issue
% there are sub-directorys for
% letters: directory name is "letters"
% technical articles: a directory per paper, named for an "author"
% news articles: directory name is "news"
% calls: directory name is "calls

%
%  Editor letters section.  Use the lettersection environment.
%  Each letter is contained in a letter environment, where the two required
%  options to \begin{letter} are the author and the address of the author.
%

\begin{lettersection}

% there will be other letters- and a blank page will appear in your document
% but the special issue part will be fine

\begin{letter}{Letter from the Editor-in-Chief}
{Haixun Wang}{Instacart}
\documentclass[11pt]{article} 

\usepackage{deauthor,times,graphicx}
%\usepackage{url}
\usepackage{hyperref}

\begin{document}
How to efficiently and effectively manage large-scale data is a
critical challenge in data management, scientific computing, machine
learning, and many other fields. In this issue, we look into this
problem from two angles.

Gerhard Weikum's opinion piece titled ``Entities with Quantities''
highlights development along the direction of querying the Web as a
database. We have come a long way in keyword based Web search: Today,
all major search engines support entity based question/answering to
certain extent (e.g., returning ``Eiffel Tower'' for query ``the
highest building in Paris''). Weikum is taking one important step
towards the goal of querying the Web as a database. In the article, he
discusses what it takes to find all entities that satisfy a
quantity-based search condition, for example, ``buildings taller than
500m'' or ``runners completing a marathon under 2:10h.''  It is clear
that this requires much advanced data preprocessing (e.g., information
extraction, entity linking, etc.), but more importantly, it requires
that at least part of the data on the entire Web needs to be organized
as a database.

Philippe Bonnet put together the current issue consisting of 5 papers
from leading researchers in the high performance computing and data
management communities on the topic of data management at
Exascale. Advances in exascale computing on petascale supercomputers
are pushing the frontier of scientific computing that requires complex
simulation, benefiting applications ranging from astrophysical
discovery to drug design. But with increasing amounts of data, the gap
between computation and I/O has grown significantly wider, which makes
data management a big challenge. This timely issue answers many
questions in this domain.

\end{document}


\end{letter}
%
\newpage
%
%% your introductory letter goes here
%
%\begin{letter}{Letter from the Special Issue Editor}
\begin{letter}{Letter from the Special Issue Editor} %JF: made it editors, plural
{Sebastian Schelter}{University of Amsterdam \& Ahold Delhaize Research, Netherlands}
\documentclass[11pt]{article} 

\usepackage{deauthor,times,graphicx}
%\usepackage{url}

\begin{document}


Software systems that store and process personal data have become ubiquitous over the last years and have enabled numerous economic, technological and scientific advances. Unfortunately, the benefits of data-driven analysis and decision making have also been accompanied by several negative developments. Examples include the increased surveillance capabilities of the state~\cite{noplacetohide} and private companies~\cite{surveillancecapitalism}, negative impact on economic inequality~\cite{automatinginequality} and traumatic experiences for individuals~\cite{beyondbroken}. As a reaction, many countries have started to regulate data storage and processing to guarantee and protect the rights of individuals. The most comprehensive such regulation is the 
General Data Protection Regulation (GDPR, {\footnotesize\url{https://gdpr.eu}}) issued by the European Union. 

In this special issue on {\em Directions Towards GDPR-Compliant Data Systems and Applications}, we continue the ongoing discussion in the data management community~\cite{gdpr-benchmark} on how to redesign data systems and applications to be compliant with such regulation. 


\vspace{0.1cm}
\noindent\textbf{Data deletion as a first-class-citizen.} The first three papers of this issue address an important question originating from the ``right-to-be-forgotten'' postulated by GDPR: {\em How can we design efficient data systems that support the timely deletion of data as a first-class citizen?} 
%
The first paper on {\em Disposal by Design} presents a vision for automating data disposal which takes into account processing constraints, regulatory constraints as well as storage constraints, and gives concrete examples from the e-commerce domain, including a suggestion of how to to find summaries of relational data with machine learning. 
%
The second paper on {\em Building Deletion-Compliant Data Systems} argues that the the requirement of timely deletion of user data is becoming central in modern data management scenarios. The authors present a new framework for building deletion-compliant data systems from a holistic perspective, analyse the requirements derived from the new policies, and propose changes in the application and the system layer of data management systems.
%
The third paper called {\em Provenance-based Model Maintenance: Implications for Privacy} focuses on efficient data deletion in a machine learning context. In particular, the authors focus on the extremely challenging problem to refresh existing models after the removal of training samples, which is called ``machine unlearning''. They argue that GDPR regulations imply that the removed samples must be fully erased from the models so that they cannot be leaked to an adversary. The paper reviews two provenance-based solutions and shows how they can guard against ``model inversion attacks", which reconstruct the removed training samples from the updated models after the unlearning process.

\vspace{0.1cm}
\noindent\textbf{Efficient data processing under regulatory constraints.} The subsequent two papers of this issue adress an orthogonal systems-related question originating from GDPR: {\em How can we design efficient data systems that comply with data processing regulations?} 
%
The fourth paper of this issue on {\em Navigating Compliance with Data Transfers in Federated Data Processing} presents work on novel systems and methods for federated data processing, where the processing of geo-distributed data is subjected to data transfer regulations. The authors showcase recent work on compliant geo-distributed data processing and present research challenges and opportunities in making federated data processing systems GDPR-compliant.
%
The fifth paper called {\em Towards Privacy by Design for Data with \MakeUppercase{strm} privacy} discusses the practical challenges of engineering teams to balance privacy and innovation, with respect to effort, data utility and computation costs. The authors argue that current approaches in scalable data systems often treat privacy as an access problem, which is at odds with important legal and design principles. Instead, the authors propose that engineering teams should shift their data privacy efforts to the point of data collection, and discuss an architectural setup for privacy-compliant stream processing applications, which is in production usage.

\vspace{1cm}
\noindent\small{\em{This work was supported by Ahold Delhaize. All content represents the opinion of the author(s), which is not necessarily shared or endorsed by their respective employers and/or sponsors.}}

\begin{thebibliography}{10}
\itemsep=1pt
\begin{small}

\bibitem{gdpr-benchmark}
S.~Supreeth, V.~Banakar, M.~Wasserman, A.~Kumar, and V.~Chidambaram.
\newblock Understanding and Benchmarking the Impact of GDPR on Database Systems.
\newblock Proceedings of the VLDB Endowment 13(7), 2019.

\bibitem{automatinginequality}
V.~Eubanks.
\newblock Automating inequality: How high-tech tools profile, police, and punish the poor. 
\newblock St. Martin's Press, 2018.

\bibitem{noplacetohide}
G.~Greenwald.
\newblock No place to hide: Edward Snowden, the NSA, and the US surveillance state. 
\newblock Macmillan, 2014.

\bibitem{surveillancecapitalism}
S.~Zuboff.
\newblock The age of surveillance capitalism: The fight for a human future at the new frontier of power.
\newblock Profile books,~2019.

\bibitem{beyondbroken}
V.~Warmerdam.
\newblock Beyond Broken - Horrible Remedies for Broken Recommenders.
\newblock Published online at \footnotesize{\texttt{https://koaning.io/posts/beyond-broken/}}, 2021.



\end{small}
\end{thebibliography}



\end{document}


\end{letter}

\end{lettersection}

% put the name of your special issue below

%\begin{opinionsection}
%\begin{opinion}{Entities with Quantities}
%{Gerhard Weikum}{Max Planck Institute for Informatics}
%\documentclass[11pt]{article} 

\usepackage{deauthor,times,graphicx}
%\usepackage{url}
\usepackage{hyperref}

\begin{document}
How to efficiently and effectively manage large-scale data is a
critical challenge in data management, scientific computing, machine
learning, and many other fields. In this issue, we look into this
problem from two angles.

Gerhard Weikum's opinion piece titled ``Entities with Quantities''
highlights development along the direction of querying the Web as a
database. We have come a long way in keyword based Web search: Today,
all major search engines support entity based question/answering to
certain extent (e.g., returning ``Eiffel Tower'' for query ``the
highest building in Paris''). Weikum is taking one important step
towards the goal of querying the Web as a database. In the article, he
discusses what it takes to find all entities that satisfy a
quantity-based search condition, for example, ``buildings taller than
500m'' or ``runners completing a marathon under 2:10h.''  It is clear
that this requires much advanced data preprocessing (e.g., information
extraction, entity linking, etc.), but more importantly, it requires
that at least part of the data on the entire Web needs to be organized
as a database.

Philippe Bonnet put together the current issue consisting of 5 papers
from leading researchers in the high performance computing and data
management communities on the topic of data management at
Exascale. Advances in exascale computing on petascale supercomputers
are pushing the frontier of scientific computing that requires complex
simulation, benefiting applications ranging from astrophysical
discovery to drug design. But with increasing amounts of data, the gap
between computation and I/O has grown significantly wider, which makes
data management a big challenge. This timely issue answers many
questions in this domain.

\end{document}


%\end{opinion}
%\end{opinionsection}

\begin{articlesection}{Data Validation for Machine Learning Models and Applications}
%
%  Contributed articles section.  Use the articlesection environment.
%  Each article is contained in an article environment, where the two required
%  options to \begin{article} are the title and author of the article
%

%\makeatletter
%\renewcommand{\AB@affillist}{}
%\renewcommand{\AB@authlist}{}
%\setcounter{authors}{0}
%\makeatother

% \begin{article}
% {Transforming the Culture: Internet Research at the Crossroads}
% {Safiya Umoja Noble and Sarah T. Roberts}
% \graphicspath{{submissions/NobleRoberts_final/}}
% %\documentclass[11pt,dvipdfm]{article}
\documentclass[11pt]{article}
\usepackage{deauthor,times,graphicx,hyperref} 

\usepackage{amsmath, amssymb, amsfonts}  

%\usepackage{algorithmic}
%\usepackage{graphicx}
%\usepackage{textcomp}
%\usepackage{xcolor}
\def\BibTeX{{\rm B\kern-.05em{\sc i\kern-.025em b}\kern-.08em
    T\kern-.1667em\lower.7ex\hbox{E}\kern-.125emX}}

%\usepackage{graphicx}
%\usepackage{subfigure}
%\usepackage{hyperref}
%\usepackage{enumitem}
%\usepackage{multirow}
%\usepackage{dsfont}
%\usepackage{algorithm2e}
%\usepackage[table,xcdraw]{xcolor}
%\usepackage{booktabs}

%\usepackage{tikz}
%\usetikzlibrary{bayesnet}

\usepackage{breakcites} %Fixes citations exceeding the margin!!

% \newtheorem{example}{Example} 
% \newtheorem{theorem}{Theorem}
% \newtheorem{lemma}[theorem]{Lemma} 
% \newtheorem{proposition}[theorem]{Proposition} 
 %\newtheorem{remark}[theorem]{Remark}
% \newtheorem{corollary}[theorem]{Corollary}
% \newtheorem{definition}[theorem]{Definition}
% \newtheorem{conjecture}[theorem]{Conjecture}
% \newtheorem{axiom}[theorem]{Axiom}
%%%
%\newtheorem{dfn}[theorem]{Definition}

%\usepackage{todonotes}
%\newcommand{\jf}[1]{{\bf \color{orange}{jf: #1}}}
%\newcommand{\shimei}[1]{{\bf \color{blue}{shimei: #1}}}
\setcounter{topnumber}{2}
\setcounter{bottomnumber}{2}
\setcounter{totalnumber}{4}
\renewcommand{\topfraction}{0.85}
\renewcommand{\bottomfraction}{0.85}
\renewcommand{\textfraction}{0.15}
\renewcommand{\floatpagefraction}{0.7}

% Definitions of handy macros can go here

\newcommand{\dataset}{{\cal D}}
\newcommand{\fracpartial}[2]{\frac{\partial #1}{\partial  #2}}

\begin{document}
\title{Transforming the Culture: Internet Research at the Crossroads}
\author{Safiya Umoja Noble \\
University of California, Los Angeles \\
snoble@g.ucla.edu
\and
Sarah T. Roberts\\ 
University of California, Los Angeles \\ 
sarah.roberts@ucla.edu}


\maketitle
\begin{abstract}
The topic of justice, fairness, bias, labor and their relation the products and practices of technology and internet companies has been a subject of our concern for nearly a decade. We see these challenges--from the organizing logics of the technology sector with respect to algorithmic discrimination, to labor practices in commercial content moderation, as key pathways into better understanding the creation and maintenance of problems made by the technology sector that cannot be solved with techno-solutionism. While our work has been closely aligned to research and advocacy broadly construed in the domain of ethics and AI, we seek to expand the conversations about sociotechnical systems beyond individual, moral and ethical concerns to those of structures, practices, policies which would allow for interdisciplinary frameworks from the fields of critical information studies, sociology, and the social sciences, and the humanities. To make legible the paradigm-shifting work we think could be taken up by scholars at colleges and universities, we will outline the contours and specifics of institutionalizing these approaches through a research center, the UCLA Center for Critical Internet Inquiry (C2i2) and its activities, By making visible the need for such a space, and our experiences and values, our hope is that it will make transparent the process and possibilities for centering justice and fairness in the world, rather than the prevailing technosolutionism we see emerging within conversations and initiatives focused on ethics and technology.
\end{abstract}

\section{Introduction}
In the summer of 2018, we received approval for the establishment of a new UCLA Center for Critical Internet Inquiry (C2i2). The proposed Center would be based within the Department of Information Studies, in the Graduate School of Education \& Information Studies, but would be campus-wide in scope, The effort was designed to address the societal impact of internet platforms, the social construction and effects of data they generate and disseminate, and their various drivers with a keen focus on issues of racial justice and gender equity. At the time of our founding, UCLA did not have any organized research unit that exclusively focused on this extraordinarily important and pervasive area of twenty-first century life, culture and economy. 

Our effort to develop and centralize a robust, visible institutional infrastructure at UCLA and beyond that provides researchers and instructors with a locus from which to inform internet development and policy, has not been without challenges. This work, by its very nature, challenges received notions of the internet and other digital technologies as primarily liberatory, beneficent or, at the least, value-neutral. Efforts to address the potentials and pitfalls of the internet for people and communities who are marginalized and underrepresented with respect to the digital is happening at a moment of austerity, when universities are increasingly reliant upon corporate and private donations to stay afloat in the wake of shrinking allocations from the legislators in Sacramento to its robust California Community Colleges, its world-class California State University system, and its flagship research campuses across the University of California.

\section{The State of Internet Research}
The public is increasingly eager to develop its own understanding and ability to actively participate in the steering of the digital technologies, social media platforms, and internet usage that now characterize much of everyday life, yet there are few mechanisms that afford such intervention. Those who should act in their stead, such as legislators and policy makers, legal professionals, educators, and others in gatekeeping capacities often lack a full picture of these technologies, their processes, and their social implications--even when they are sympathetic and energized to the public’s desire to wrest back control. For much of their existence, Silicon Valley’s social media firms have enjoyed close -- even cozy -- relationships with legislators in Washington. Even as the tide of general sentiment has turned over the past few years, the grilling that Senate subcommittees have intended to give the executives of those firms has often fallen flat simply due to a lack of precision or understanding on the part of the questioners. In both cases, the public has stood to lose.

Meanwhile, there has been an effort by these same firms, often along with university partners that have been heretofore largely uncritical of them, to get out ahead of any potential lawmaker curtailing their activities by appearing to self-regulate. The most common iteration of this attempt has come in the nascent development of a variety of “ethics” initiatives, boards and research teams popping up inside of and adjacent to the major corporations. Those initiatives, too, are not without significant flaws.  We are fundamentally concerned with the industry regulating itself in lieu of responding to public policy and providing accountability. While self-reflection is important, as are increasing efforts to broaden frameworks of responsibility, we largely see that self-regulation is insufficient as the industry leaders are the subjects of antitrust lawsuits, EEOC violations, and investigations into consumer harm.

We are indebted to a number of scholars who are influencing our thinking about the politics and power embedded in the digital, and whose work we are in dialog with on a continual basis ~\cite{benjamin2019race,chun2008control,daniels2009cyber,eubanks2018automating,hoffmann2019fairness,noble2018algorithms1,pasquale2016black,roberts2019behind,vaidhyanathan2006afterword,vaidhyanathan2018antisocial}. There are of course, so many important social scientists and humanists whose work has been the opening for scholars in computing to take the systemic issues of fairness and equity. What we often find is that scholars in computer science and related engineering fields do not cite the work of the scholars who have framed the debate, thus making the need for epicenters of interdisciplinary critical scholarship even more crucial. Indeed, the ability to invoke issues of fairness has been made legible and plausible because humanists and social scientists have provided the evidence that has forced these issues into view. 


For instance, \cite{binns2018fairness}'s  review of ethics and fairness in the fields of machine learning and artificial intelligence is an important overview of how our colleagues in computing fields are increasingly limited by the origins of Western philosophy as they cultivate “ethical AI.” We will not repeat here the work done to trace the histories of liberalism and its limitations as applied to computing and digital technologies\footnote{See \cite{BuiNoble}.} , but we note that this previous careful study of the origins of liberal philosophy that bolsters the field of ethics deeply informs our own disposition toward the limits of this emerging field. In particular, we believe that the field of “ethical AI” must contend with how it affects and is affected by power structures that encode systems of sexism, racism, and class. Instead of depoliticizing these systems, we embrace a sociological orientation, in the tradition of scholars like \cite{daniels2009cyber}, who has adeptly framed and helped us better understand more powerful analytics like oppression and discrimination in lieu of words like bias and ethics, which obfuscate the power analyses and interventions so desperately needed. 


In our research, we use structural and systems-level analyses that can properly account for the impact of the socio-technical assemblages that make up digital  ecosystems and infrastructures. Through our studies of the digital, we uncover opportunities for accountability from harms that extend beyond individual moral and ethical choices to public policy, labor and employment practices, supply-chain business practices, environmental interactions, and a variety of approaches that can have tremendous impact at scale. Without these approaches, the work of ethical AI is greatly reduced to individual, technical, and organizational-level  failings against some imagined “fair” standard that, itself, is dislocated from fairness as a matter of civil, human, or sovereign rights tied to political, economic, and social struggles. Because of these analytical  framings, we are able to examine the material dimensions of internet-enabled digital infrastructures and practices that involve many factors that extend beyond algorithms and AI to include workers, legal and financial practices, and consolidations of power.   
\cite{BuiNoble} wrote about the way in which technology corporations, in an effort to minimize risk from the damages associated with their discriminatory and faulty products, are performing reputation management through claims to be more accountable, fair, transparent, and ethical. Several in-house ethical AI teams, corporate-sponsored research think tanks, and non-profits aligned with industry are producing myriad conferences, white papers, research publications and campaigns that seek to define the landscape of ethical AI. They note:
\begin{quote}
Moreover, data trusts and research partnerships between universities, policy think tanks, and technology corporations have been established and revamped as a go-to strategy for effecting a more democratic and inclusive mediated society, again calling for fairness, accountability, and transparency (FAT) as key ideals within the future of AI, yet often leaving and ignoring notions of intersectional power relations out of their ethical imaginaries and frameworks. As a point of departure, many are invested in linking conversations about ethics to the moral genesis and failures caused by structural racism, sexism, capitalism, and the fostering of inequality, with an eye toward understanding how the digital is implicated in social, political, and economic systems that buttress systemic failures. Complicating these conversations are concerns about neo-colonial technology supply chains and the total integration of the digital into global economic systems \cite{BuiNoble}.
\end{quote}

We are equally influenced in the making of space for feminist and critical interrogations of fairness models by the work of \cite{hoffmann2019fairness}, whose work we see at the forefront of design-thinking that accounts for systemic oppression rather than technosolutions that are rooted in ideologies of colorblindness, genderblindness, and disavowals of their politics. We are heartened to see in the last proceedings of the ACM Fairness, Accountability and Transparency conference the model of \cite{abebe2020roles} in thinking about the complexities and role of computing in social change, and see this as a powerful possibility for reimagining how we do interdisciplinary work that makes for new normativities around social justice in the fields of computing.


As we think about the work before us in 2021, the limits of the ethical AI-academic-industrial complex, with respect to true interventions that need to be made in the business models that promulgate unfairness and discrimination were on powerful display with the unexpected and headline-grabbing December 2020 firing of one of the most prominent AI ethicists in the world, Dr. Timnit Gebru of Google. Indeed, as 2020 drew to a close, it was with daily news stories and tweets about a range of problems Gebru had faced, from the hostile work environment she experienced as a Black woman to attempts to silence and suppress the evidence she found of algorithmic discrimination in Google’s natural language processing (NLP) models \cite{Hao2020}. Indeed, her work referenced many well-known and broadly understood negative impacts of AI, from discrimination to environmental impact \cite{crawford2019ai1}, while in this case, specifically linking these flaws to Google’s products. Gebru’s scholarship in the area of discriminatory and unfair technologies is deep and unparalleled \cite{gebru2019oxford,gebru2018datasheets,buolamwini2018gender1}. What this case demonstrates in practice is that doing the hard work of tracing discrimination and harm cannot withstand the profit imperative that technology companies prioritize at all cost -- even at the expense of their own claims to prioritizing ethics. 

We believe this necessitates, more than ever, independent spaces for the study of these problems, without exertion and pressure from the interests of shareholders, and without impinging upon academic freedom and the need for researchers to speak truth to power through their analyses and discoveries. 

Moreover, the firing of Gebru is not unlike the firing and intimidation of workers in a variety of technology companies who, when confronting their employers with evidence of the harms of their products or labor conditions, have been summarily dismissed \cite{campbell2018tech,Kan2019,Solon2018}. Therein lies a profound contradiction at the claims to fairness and ethics in product development while evidence of unfairness, discrimination, wage disparity, misrepresentation, hostile and damaging workplaces, harassment, and so forth are standard operating procedures across the major internet companies. We need spaces for research and a variety of interventions – at social, political, and technical dimensions –  that are not controlled by the interests of the very actors that benefit from these types of corporate practices.

We see the limits of possibility for intervention in industrial-academic ethics labs, and we recognize the roster of university- and industry-based centers engaging at the intersection of internet and society is long, but few are specifically and directly concerned with articulating the critical issues of asymmetrical power with respect to digital technologies. Simply put, we believe the time to do so is now and we are attempting to do so at UCLA. Even fewer centers of internet inquiry are institutionalized at public research universities: some of the most visible centers have been the University of Oxford’s Oxford Internet Institute (OII), Harvard University’s Berkman Klein Center, Yale’s Internet Society Project and Stanford University’s Center for Internet \& Society and Stanford Center for Human-Centered Artificial Intelligence, which are often industry focused and not without associated challenges. As industry and commercial projects are increasingly moving to the foreground in the public sphere, and having significant impact on shaping the activities and nature of public institutions--including public K-12 education and libraries, higher education, and public media, inquiry into these projects and their trajectories is well-suited to UCLA as the leading public research university in the United States. In our case, we are interested in research and policy interventions that center the most vulnerable. We believe that this type of research, expressly embedded in public universities, strengthens the democratic, public-interest counterweights that are so clearly needed to foster broader interdisciplinary research efforts that prioritize various publics.

\section{An Effort to Transform the Culture of Internet Studies}
The UCLA Center for Critical Internet Inquiry (C2i2) is an interdisciplinary center that promotes the technological, historical, social and humanistic study of the internet and digital life with respect to the values of fairness, justice, equity, and sustainability in the digital world. C2i2’s innovation and orientation to its study of the internet is not simply based on the objects of investigation with which we engage, but, rather, our theoretical orientation to this work. We are humanities-informed social scientists who are also technologists. As such, we are concerned with the social implications and impact of technology. Our disciplinary and theoretical orientation reflects what we describe as the broad, and still somewhat nascent, subfield of critical information studies \cite{vaidhyanathan2006afterword}. In our intellectual practice, critical information studies itself, by its nature, necessitates interdisciplinary contact and intellectual influence bridged between and among it and the fields of library \& information science, internet studies, media studies, communication, African American studies, gender studies, labor studies, sociology, science and technology studies, and other key and relevant points of scholarly contact. 

The conceptual basis for an expressly critical information studies, in particular, is a stipulation that information is fundamentally and inherently a matter to be regarded as existing along axes of social, political and cultural production, import, values and impact. It therefore follows that power analyses of information along these axes, as they are undertaken in a critical information studies theoretical practice, can be used to apprehend, describe, critique and intervene upon the medium as well as the meanings of texts, images, and ideas and the ways they are produced, displayed, systematized, circulated, consumed, stored and/or discarded within and among digital systems and along those same axes of power. This analytic process fundamentally and inherently relies upon political economic critiques to examine how information is controlled, owned, and distributed. 

Under a critical information studies framework, the political economic analysis is then engaged in a further, intersectional power analysis that recognizes that these informational phenomena occur in relation to, and at varying uneven degrees, based on historical distributions of power along multiple additional axes: those of race, ethnicity, and gender, to name but a few. Herein, the focus is a dedication to studying the ways in which race and gender function in/are deployed by the digital technology practices and products of multinational digital internet media corporations. In this way, we both broaden and sharpen the kinds of analytical tools that can be used to understand technology and/as power and its impacts on the world.

For us, the making of C2I2 is an effort to promote investigations into the politics, economics, and impacts of technological systems, with the goal of understanding the relationships between digital technologies and the internet as a site to enhance the public good. In practical terms, the Center supports both undergraduate and graduate research and education through collaborations with a variety of academic units as well as through the programs within UCLA’s Department of Information Studies and the School of Education \& Information Studies. Our research and teaching emphasize internet and information scholarship and practice as relevant to a variety of disciplines and domains. 


\section{Our Guiding Principles}
Our guiding principles have been an effort to make visible a set of priorities that we hope can be taken up, strengthened and added to by a robust network of multiple internet and society centers and initiatives. We start from statements of our fundamental principles and core values:
\begin{itemize}
\item	We believe our research should have community impact and foster racial justice and social improvement
\item	We promote outreach, inclusion, and translation of research to the public for greater impact and positive social change
\item	We invite funders to support the work of C2i2 with an understanding that support for high-quality research is best realized with total independence from funder control over the research agenda, operations, communications, etc. of C2i2
\item	We recognize that transparency of sources of funding is an important ethical dimension of the work we do, and we seek to make our funders visible while clearly articulating the boundaries and firewalls we place between donations and research outcomes
\item	We believe in and support global networked relationships with other sites of research and advocacy, worldwide, and we employ a “big umbrella” approach to supporting people and projects that are interested in critical inquiries of the internet and society
\item	We aspire to relationships and operational practices of “mutual respect, care, pluralism and the duty of repair,”\footnote{In this quote, we draw upon recent efforts in the UCLA Department of Information Studies to crystallize and clearly articulate its own commitments.}  consistent with the strategic mission and vision of the UCLA Department of Information Studies
\item	We value difficult conversations and debates
\item	We engage in cyclical review of the research and initiatives of C2i2 to ensure that we are creating a sustainable research environment where faculty, students, staff and community members can develop robust programs of research and action
\item	We foster an environment of challenge and professional development for our affiliates at all stages in their careers and professional lives
\item	We believe in a holistic approach to scholarship that puts physical and mental health and wellness of our colleagues and ourselves at the fore and underscores the importance of a healthy, supportive working environment
\item	We value learning and dissemination of the research of C2i2 for the benefit of all of our communities and for the larger public good 
\item	We use multiple modalities to transfer our findings in legible, accessible ways for a variety of audiences
\end{itemize}

\section{Critical Internet Studies on the Rise}

As of this writing, a series of public circumstances have shaken confidence in internet technologies and platforms. We see these points of failure as having the potential for a profound moment of reconfiguration and repair, as they have opened up new possibilities for reimagining the possibilities of digital networks and their effects. We recognize both the positive affordances, and possible consequences of under-developed or asymmetrical technologies, and seek to study these more robustly. The public is increasingly eager to develop its own understanding and ability to actively participate in the steering of the digital technologies, social media platforms, and internet usage that now characterize much of everyday life, yet there are few mechanisms that afford such intervention. Those who should act in their stead, such as legislators and policy makers, legal professionals, educators, and others in gatekeeping capacities often lack a full picture of these technologies, their processes, and their implications--even when they are sympathetic and energized to the public’s desire to wrest back control. Building upon existing faculty research strengths, C2i2 is attempting to serve as a vital bridge to close this gap in knowledge for academics, policy makers, engaged industry personnel and the public at large by providing both original insights derived from empirical research, as well as the expert analysis and interpretation of those data to positively impact and reimagine digital technologies’ influence in society.

The making of a campus-wide interdisciplinary center that promotes the study of the internet with respect to the values of fairness, justice, equity, and sustainability in the digital world has been difficult in the wake of COVID-19 and the austerity measures now facing higher education.  Private foundations have been the lifeblood of our ability to pursue agenda-setting and proactive research, teaching and service while maintaining our intellectual independence, as we seek the bridging of academia, industry, and policy to effect positive change within and among these domains. We engage with scholars, activists, advocates, technologists, policy makers and others who are interested in the ways in which digital technologies are shaping and transforming humanity through initiatives that reflect a broad range of social and ethical concerns that require sustained, open and multi-stakeholder debate and exploration.
Developing a center that openly values justice, equity, diversity, community building, environmental sustainability, labor and worker health and well-being, and public trust in democratic institutions with respect to the role of the internet and its constituent platforms and technologies in maximizing or eroding these possibilities has also been less popular than one might believe. We note that our many internet and society counterparts around the world who have been better resourced and supported over the past decade have often enjoyed a more remunerative and expedient direct relationship to the industries they seek to study and critique, whereas our nascent work in centering social justice in information and internet studies has been slowly waxing. It is now firmly on the rise.

We see our work furthering joint curriculum development by the Departments of Information Studies and Education to respond to calls by the State of California for increased digital and media literacy in K-12 schools (SB 830), and see our presence as faculty members within the School of Education \& Information Studies as an inherent strength.  Likewise, we also value collaboration with centers for the study of the internet and society at other leading universities in the US and elsewhere. As such, we value public intellectual work and  public programming. Our plans for public engagement also include outreach to public libraries and archives, educational institutions and community organizations,  as well as collaboration with other UCLA campus centers such as Bunche Center, the Institute for Research on Labor and Employment, the Center for Global Digital Cultures, the Center for Information as Evidence, the UCLA Law Promise Institute, the UCLA Community Archives Lab, and the UCLA Game Lab.


The possibility for our work has been launched through our inaugural Minderoo Initiative on Technology and Power, established through a \$3M gift over 5 years that began on July 1, 2020. C2i2 is one of the North American nodes of Minderoo Foundation's global tech impact network, employing an expert team to develop model frameworks for laws that protect the public from the harms of predatory big data and digital platforms. In our work, we will identify the existing compliance issues of AI use, provide an independent source of public-facing evaluation and knowledge for people seeking greater information, protection and redress, and deliver a model legislative package that upholds dignity, equality, and transparency in government's use of algorithmic and human moderated digital and data-reliant systems.

We anticipate outputs of interest not only to the greater scholarly community but also with direct and meaningful application in the areas of policy development, advocacy, industry and to an interested and engaged public. 


\section{Conclusion: Strengthening Research Agendas at the Intersection of Society \& Big Data}
Currently, there are only a handful of Internet Studies departments that endeavor to cover these topics holistically in the way we propose. We believe this is therefore a tremendous opportunity not just for UCLA, but for many public universities to make an investment in robust collaboration with extant partners across campuses to cultivate a graduates prepared to enter a variety of professions where they can have direct impact in areas of algorithmic discrimination, trust \& safety, internet policy, social media and content development, and public-advocacy and community organizing. We believe the time is now to create new paradigms for the public to understand the costs of tech platforms, predictive technologies, advertising-driven algorithmic content, and the work of digital laborers. Of course, central to the harms caused by dis- and mis-information is the work of Commercial Content Moderators \cite{roberts2019behind}. We have already been at the helm of strengthening global research networks for the study of commercial content moderation of social media platforms at scale. 

Of course, we also think there are important roles computer scientists and engineers can play in this effort given their expertise. We believe strongly in interdisciplinary approaches and through C2i2  seek opportunities to collaborate, teach, and undertake  research between data scientists, computer scientists, information professionals, social scientists and humanists. In the most forward-thinking of ways, we imagine the social and technical given equal footing and resources to solve the most pressing issues facing humanity, the planet, and to address pervasive global inequality and injustice. Our research demonstrates conclusively that internet, social media and tech companies can no longer deny, downplay or ignore their own culpability in some of these crises; those that will thrive beyond regulation and public pushback will see such critique not simply as unfounded criticism for its own sake, but as a tremendous opportunity for real restoration and repair.


There is an important and timely need for both research and public policy development -- with civil, human and sovereign rights organizations and stakeholders at the table with technologists, social scientists and humanists -- around the importance of restoration and reparations to democracy. For this reason, we see our work, a decade on as collaborators and a year into the existence of C2i2, as having only just begun, and our Center as an ideal place from which to advocate for change. It is nothing less than the agenda of our lifetimes.


\vspace{-.1cm}
%\bibliographystyle{ACM-Reference-Format}
%\bibliographystyle{apalike}
%\bibliography{references}


%%%%% Example bibliography using bibitems
\begin{thebibliography}{10}
%\begin{small}
\itemsep=-.5pt
 
\bibitem[Abebe et~al., 2020]{abebe2020roles}
Abebe, R., Barocas, S., Kleinberg, J., Levy, K., Raghavan, M., and Robinson,
  D.~G. (2020).
\newblock Roles for computing in social change.
\newblock In {\em Proceedings of the 2020 Conference on Fairness,
  Accountability, and Transparency}, pages 252--260.

\bibitem[Benjamin, 2019]{benjamin2019race}
Benjamin, R. (2019).
\newblock Race after technology: Abolitionist tools for the {New Jim Code}.
\newblock {\em Social Forces}.

\bibitem[Binns, 2018]{binns2018fairness}
Binns, R. (2018).
\newblock Fairness in {M}achine {L}earning: Lessons from political philosophy.
\newblock In {\em Conference on Fairness, Accountability and Transparency},
  pages 149--159. PMLR.

\bibitem[Bui and Noble, 2020]{BuiNoble}
Bui, M.~L. and Noble, S.~U. (2020).
\newblock We're missing a moral framework of justice in {A}rtificial
  {I}ntelligence: On the limits, failings, and ethics of fairness.
\newblock In Dubber, M., Pasquale, F., and Das, S., editors, {\em The Oxford
  Handbook of Ethics of AI}. Oxford University Press.

\bibitem[Buolamwini and Gebru, 2018]{buolamwini2018gender1}
Buolamwini, J. and Gebru, T. (2018).
\newblock Gender shades: Intersectional accuracy disparities in commercial
  gender classification.
\newblock In {\em Conference on fairness, accountability and transparency},
  pages 77--91.

\bibitem[Campbell, 2018]{campbell2018tech}
Campbell, A. (2018).
\newblock How tech employees are pushing {S}ilicon {V}alley to put ethics
  before profit.
\newblock {\em Vox}.

\bibitem[Chun, 2008]{chun2008control}
Chun, W. H.~K. (2008).
\newblock {\em Control and Freedom: Power and Paranoia in the Age of Fiber
  Optics}.
\newblock MIT Press.

\bibitem[Crawford et~al., 2019]{crawford2019ai1}
Crawford, K., Dobbe, R., Dryer, T., Fried, G., Green, B., Kaziunas, E., Kak,
  A., Mathur, V., McElroy, E., S{\'a}nchez, A.~N., et~al. (2019).
\newblock {AI} {N}ow 2019 report.
\newblock {\em New York, NY: AI Now Institute}.

\bibitem[Daniels, 2009]{daniels2009cyber}
Daniels, J. (2009).
\newblock {\em Cyber racism: White supremacy online and the new attack on civil
  rights}.
\newblock Rowman \& Littlefield Publishers.

\bibitem[Eubanks, 2018]{eubanks2018automating}
Eubanks, V. (2018).
\newblock {\em Automating inequality: How high-tech tools profile, police, and
  punish the poor}.
\newblock St. Martin's Press.

\bibitem[Gebru, 2019]{gebru2019oxford}
Gebru, T. (2019).
\newblock Oxford handbook on {AI} ethics book chapter on race and gender.
\newblock {\em arXiv preprint arXiv:1908.06165}.

\bibitem[Gebru et~al., 2018]{gebru2018datasheets}
Gebru, T., Morgenstern, J., Vecchione, B., Vaughan, J.~W., Wallach, H.,
  Daum{\'e}~III, H., and Crawford, K. (2018).
\newblock Datasheets for datasets.
\newblock {\em arXiv preprint arXiv:1803.09010}.

\bibitem[Hao, 2020]{Hao2020}
Hao, K. (2020).
\newblock We read the paper that forced {T}imnit {G}ebru out of {G}oogle.
  here?s what it says.
\newblock {\em MIT Technology Review}.

\bibitem[Hoffmann, 2019]{hoffmann2019fairness}
Hoffmann, A.~L. (2019).
\newblock Where fairness fails: data, algorithms, and the limits of
  antidiscrimination discourse.
\newblock {\em Information, Communication \& Society}, 22(7):900--915.

\bibitem[Kan, 2019]{Kan2019}
Kan, M. (2019).
\newblock {G}oogle workers protest conservative thinker on {AI} board.
\newblock {\em PCMAG}.

\bibitem[Noble, 2018]{noble2018algorithms1}
Noble, S. (2018).
\newblock {\em Algorithms of Oppression: How Search Engines Reinforce Racism}.
\newblock NYU Press.

\bibitem[Pasquale, 2016]{pasquale2016black}
Pasquale, F. (2016).
\newblock {\em The Black Box Society: The Secret Algorithms behind Money and
  Information}.
\newblock Harvard University Press.

\bibitem[Roberts, 2019]{roberts2019behind}
Roberts, S.~T. (2019).
\newblock {\em Behind the screen: Content moderation in the shadows of social
  media}.
\newblock Yale University Press.

\bibitem[Solon, 2018]{Solon2018}
Solon, O. (2018).
\newblock When should a tech company refuse to build tools for the government?
\newblock {\em The Guardian}.

\bibitem[Vaidhyanathan, 2006]{vaidhyanathan2006afterword}
Vaidhyanathan, S. (2006).
\newblock Afterword: Critical information studies: A bibliographic manifesto.
\newblock {\em Cultural Studies}, 20(2-3):292--315.

\bibitem[Vaidhyanathan, 2018]{vaidhyanathan2018antisocial}
Vaidhyanathan, S. (2018).
\newblock {\em Antisocial media: How {F}acebook disconnects us and undermines
  democracy}.
\newblock Oxford University Press.

\end{thebibliography}

\end{document}

% \end{article}


\begin{article}
{A Data Quality-Driven View of MLOps}
{Cedric Renggli, Luka Rimanic, Nezihe Merve Gürel, Bojan Karlaš, Wentao Wu and Ce Zhang}
\documentclass[11pt]{article}

\usepackage{deauthor}
\usepackage{times}
\usepackage{paralist} 
\usepackage{verbatim}
%\usepackage{subcaption}
\usepackage{subfigure}
\usepackage{color}
\usepackage{multirow}
\usepackage{listings}
\usepackage{array}
\usepackage{graphicx}
\usepackage{tabularx}
\usepackage{lscape}
\usepackage{rotating}
\usepackage{algorithm}
\usepackage{algorithmic}
\usepackage{multicol}
\usepackage{xspace}
\usepackage{amsfonts}
%\usepackage[noadjust]{cite}
\usepackage{cite}
\usepackage{booktabs}

\usepackage{wrapfig}

%\usepackage[pdftex]{hyperref}
\usepackage{hyperref}

\def \cD {\mathcal D}
\def \cH {\mathcal H}
\def \cX {\mathcal X}
\def \cY {\mathcal Y}
\newcommand{\RR}{\mathbb{R}}

\def\compactify{\itemsep=0pt \topsep=0pt \partopsep=0pt \parsep=0pt}

\begin{document}

\title{A Data Quality-Driven View of MLOps}
\author{Cedric Renggli$^\dagger$~~Luka Rimanic$^\dagger$~~Nezihe Merve Gürel$^\dagger$~~Bojan Karla\v{s}$^\dagger$~~Wentao Wu$^\ddagger$~~Ce Zhang$^\dagger$\\
$^\dagger$ETH Zurich\\
$^\ddagger$Microsoft Research\\
\{cedric.renggli, luka.rimanic, nezihe.guerel, bojan.karlas, ce.zhang\}@inf.ethz.ch\\
wentao.wu@microsoft.com
}

\maketitle

\begin{abstract}
Developing machine learning models can be seen as 
a process similar to the one established for 
traditional software development.
A key difference between the two lies in the 
strong dependency between the quality of a machine learning model
and the quality of the data used to train or perform evaluations.
In this work, we demonstrate how different aspects
of data quality propagate through various stages
of machine learning development.
By performing a joint analysis of the impact 
of well-known data quality dimensions and 
the downstream machine learning process, we show that different components
of a typical MLOps pipeline can be
efficiently designed, providing both a technical and theoretical 
perspective.
\end{abstract}

\section{Introduction}

A machine learning (ML) model 
is a software artifact ``compiled'' from data~\cite{karpathy2017software2}. 
This point of view motivates a study of both similarities
and distinctions when compared to 
traditional software.
\textit{\underline{Similar to}} traditional software artifacts,
an ML model deployed in production 
inevitably undergoes the DevOps process ---
a process whose aim is to
``\textit{shorten the system development life cycle
and provide continuous delivery with 
high software quality}''~\cite{bass2015devops}.
The term ``MLOps'' is used
when this DevOps process is specifically applied to ML~\cite{alla2021mlops}.
\textit{\underline{Different from}}
traditional software artifacts,
the quality of an ML model 
(e.g., accuracy, fairness, and robustness) is often 
a reflection of the
\textit{quality of the underlying data}, 
e.g., noises, imbalances, and additional adversarial perturbations.

Therefore, one of the most promising 
ways to improve the accuracy, fairness, and robustness of an ML model is often to 
improve the dataset, via means such as
data cleaning, integration, and label acquisition. As MLOps aims to 
\textit{understand}, \textit{measure}, and \textit{improve} the quality
of ML models, it is not surprising to see that 
\textit{data quality} is playing a prominent and central role in MLOps. In fact,
many researchers have conducted fascinating and seminal
work around MLOps by looking into different aspects of 
data quality. Substantial effort has been made in the areas of data acquisition with weak supervision (e.g., Snorkel~\cite{ratner2017snorkel}), ML engineering pipelines (e.g., TFX~\cite{katsiapis2019towards}), data cleaning (e.g., ActiveClean~\cite{krishnan2016activeclean}), data quality verification (e.g., Deequ~\cite{schelter2019differential, schelter2018automating}), interaction (e.g., Northstar~\cite{kraska2018northstar}), or fine-grained monitoring and improvement (e.g., Overton~\cite{re2019overton}), to name a few.

Meanwhile, for 
decades data quality has been an active and exciting
research area led by the data management community~\cite{batini2009methodologies,strong1997data,scannapieco2002data},
having in mind that the majority of the studies are 
agnostic to the downstream ML models
(with prominent recent exceptions such as
ActiveClean~\cite{krishnan2016activeclean}).
Independent of downstream ML models,
researchers have studied different aspects of
data quality that can naturally be split across the following four \textit{dimensions}~\cite{batini2009methodologies}: (1)~\textit{accuracy} -- the extent to which the data are correct, reliable and certified for the task at hand;
(2)~\textit{completeness} -- the degree to which the given data collection includes data that describe the corresponding set of real-world objects;
(3)~\textit{consistency} -- the extent of violation of semantic rules defined over a set of data; and 
(4)~\textit{timeliness} (also referred to as \textit{currency} or \textit{volatility}) -- the extent to which data are up-to-date for a task.

\begin{wrapfigure}{r}{0.5\textwidth}
\vspace{-1em}
\begin{center}
\includegraphics[width=0.5\textwidth]{submissions/data-quality-ml-ops/figs/techq.pdf}
\end{center}
\vspace{-2em}
\vspace{-1em}
\end{wrapfigure}

\vspace{-0.5em}
\paragraph*{Our Experiences and Opinions} 
In this paper, we 
provide a bird's-eye view 
of some of our previous 
works that are related to
enabling different functionalities with respect to MLOps. These works are 
inspired by our experience working hand-in-hand with academic and industrial users to build
ML applications~\cite{schawinski2017generative, su2018generative, sartori2018model, stark2018psfgan, ackermann2018using, schawinski2018exploring, girardi2018patient, glaser2019radiogan, beck2019sensing, sartori2019forward}, together with  
our effort of building 
\texttt{ease.ml}~\cite{aguilar2021ease}, a prototype
system that defines an 
end-to-end MLOps process.

Our key observation is 
that often \textit{
MLOps challenges are bound to
data management challenges} --- 
given the aforementioned 
strong dependency 
between the quality of 
ML models and the quality of data, the never-ending
pursuit of \textit{understanding,
measuring}, and \textit{improving the
quality of ML models},
often hinges on 
\textit{understanding,
measuring}, and \textit{improving
the underlying 
data quality issues}.
From a technical perspective, this poses 
unique challenges and opportunities.
As we will see, we find it necessary to 
revisit decades of 
data quality research
that are agnostic to 
downstream ML models and 
try to 
understand different 
data quality dimensions -- accuracy, completeness,
consistency, and timeliness -- jointly with the downstream ML process. 

In this paper, we describe 
four of such examples, originated from our 
previous research~\cite{karlavs2020nearest, renggli2020automatic, karimi2020online, renggli2019continuous}.
Table~\ref{tbl:overview}
summarizes 
these examples, each of which tackles one
specific problem in MLOps and
poses technical challenges 
of jointly analyzing data quality 
and downstream ML processes.

\begin{table}[t!]
\caption{Overview of 
our explorations with data quality propagation at different stages of an MLOps 
process.}
\label{tbl:overview}
\vspace{-0.5em}
\small
\centering
\begin{tabular}{@{}llll@{}}
\toprule
\textbf{Technical Problem for ML}                           & \textbf{MLOps Stage}         & \textbf{MLOps Question}                                   & \textbf{Data Quality Dimensions}        \\ \midrule
Data Cleaning (Sec.~\ref{sec:cpclean})~\cite{karlavs2020nearest}              & Pre Training  & Which training sample 
to clean?              &
Accuracy \& Completeness\\
Feasibility Study (Sec.~\ref{sec:snoopy})~\cite{renggli2020automatic}         & Pre Training  & Is my target accuracy realistic? & Accuracy \& Completeness\\
CI/CD (Sec.~\ref{sec:ci})~\cite{renggli2019continuous} & Post Training & 
Am I overfitting to val/test? & Timeliness\\
Model Selection (Sec.~\ref{sec:model_picker})~\cite{karimi2020online}        & Post Training & Which samples should I label?                        & Completeness \& Timeliness\\
\bottomrule
\end{tabular}
\end{table}

\vspace{-0.5em}
\paragraph*{Outline}
In Section~\ref{sec:ml} we provide a setup for studying this topic, highlighting the importance of taking the underlying probability distribution into account. In Sections~\ref{sec:cpclean}-\ref{sec:model_picker} we revisit components of different stages of the \texttt{ease.ml} system purely from a data quality perspective. Due to the nature of this paper, we avoid going into the details of the interactions between these components or
their technical details.
Finally, in Section~\ref{sec:limitations} we describe a common limitation that all the components share, and motivate interesting future work in this area.

\section{Machine Learning Preliminaries}
\label{sec:ml}

In order to highlight the strong dependency between the data samples used to train or validate an ML model and its assumed underlying probability distribution, we start by giving a short primer on ML. In this paper we restrict ourselves on supervised learning in which, given a \textit{feature space} $\cX$ and a \textit{label space} $\cY$, a user is given access to a dataset with $n$ samples  $\cD := \{(x_i, y_i)\}_{i \in [n]}$, where $x_i \in \cX$ and $y_i \in \cY$. Usually $\cX \subset \RR^d$, in which case a sample is simply a $d$-dimensional vector, whereas $\cY$ depends on the task at hand. For a regression task one usually takes $\cY = \RR$, whilst for a classification task on $C$ classes one usually assumes $\cY = \{1,2,\ldots,C\}$. We restrict ourselves to classification problems.

Supervised ML aims at \textit{learning} a map $h: \cX \rightarrow \cY$ that generalizes to unseen samples based on the provided labeled dataset $\cD$. 
A common assumption used to learn the mapping is that all data points in $\cD$ are sampled identically and independently (i.i.d.) from an unknown distribution $p(X,Y)$, where $X,Y$ are random variables taking values in $\cX$ and $\cY$, respectively. For a single realisation $(x, y)$, we abbreviate $p(x,y) = p(X\!=\!x,\!Y\!=\!y)$. 

The goal is to choose $h(\cdot) \in \cH$, 
where $\cH$ represents the hypothesis space, that minimizes the expected risk with respect to the underlying probability distribution~\cite{shalev2014understanding}. In other words, one wants to construct $h^*$ such that
\begin{equation}
\label{eq:expected_risk}
h^* = \arg \min_{h \in \cH} \mathbb{E}_{X,Y} \left( L(h(x), y) \right) = \arg \min_{h \in \cH} \int_\cX \int_\cY L(h(x), y) p(x, y)\,dy\,dx ,
\end{equation}
with $L(\hat{y}, y)$ being a loss function that penalizes wrongly predicted labels  $\hat{y}$. For example, $L(\hat{y}, y) = \mathbf{1} (\hat{y} = y)$ represents the 0-1 loss, commonly chosen for classification problems. 
Finding the optimal mapping $h^*$ is not feasible in practice: (1) the underlying probability $p(X,Y)$ is typically unknown and it can only be approximated using a finite number of samples, (2) even if the distribution were known, calculating the integral is intractable for many possible choices of $p(X,Y)$. Therefore, in practice one performs an empirical risk minimization (ERM) by solving 
$\hat{h} = \arg \min_{h \in \cH} \frac{1}{n} \sum_{i=1}^n L(h(x_i), y_i).$
Despite the fact that the model is learned using a finite number of data samples, the ultimate goal is to learn a model which generalizes to any sample originating from the underlying probability distribution, by approximating its posterior $p(Y|X)$.
Using $\hat{h}$ to approximate $h^*$ can run into what-is-known as ``overfitting'' to the training set $\cD$, which reduces the generalization property of the mapping.
However, advances in statistical learning theory managed to considerably lower the expected risk for many real-world applications whilst avoiding overfitting~\cite{friedman2001elements, vapnik2015uniform, shalev2014understanding}.  Altogether, any aspect of data quality for ML application development should not only be treated with respect to the dataset $\cD$ or individual data points therein, but also \emph{with respect to the underlying probability distribution the dataset $\cD$ is sampled from}.

\paragraph{Validation and Test}

Standard ML cookbooks suggest that the data should be represented by three disjoint sets to \emph{train}, \emph{validate}, and \emph{test}. 
The validation set accuracy is typically used to choose the best possible set of hyper-parameters used by the model trained on the training set.
The final accuracy and generalization properties are then evaluated on the test set. Following this, we use the term \textit{validation} for evaluating models in the pre-training phase, and the term \textit{testing} for evaluating models in the post-training phase.

\paragraph{Bayes Error Rate}
Given a probability distribution $p(X,Y)$, the lowest possible error rate achievable by \textit{any} classifier is known in the literature as the Bayes Error Rate (BER). It can be written as
\begin{equation}
\label{eq:bayes_error}
R_{X,Y}^* = \mathbb{E}_{X} \left[ 1- \max_{y\in\cY} p(y| x) \right],
\end{equation}
and the map $h_{opt}(x)=\arg \max_{y\in\cY} p(y \vert x)$ is called the \textit{Bayes Optimal Classifier}. It is important to note that, even though $h_{opt}$ is the best possible classifier (that is often intractable for the reason stated above), its expected risk might still be greater than zero, which results in the accuracy being at most $1 - R_{X,Y}^*$. In Section~\ref{sec:snoopy}, we will outline multiple reasons and provide examples for a non-zero BER.

\paragraph{Concept Shift}

The general idea of ML described so far assumes that the probability distribution $P(X, Y)$ remains fixed over time, which is sometimes not the case in practice~\cite{widmer1996learning, tsymbal2004problem, gama2014survey}. Any change of distribution over time is known as a \textit{concept shift}.
Furthermore, it is often assumed that both the feature space $\cX$ and label space $\cY$ remain identical over a change of distribution, which could also be false in practice. A change of $\cX$ or $p(X)$ (marginalized over $Y$) is often referred to as a \textit{data drift}, which can result in missing values for training or evaluating a model. We will cover this specific aspect in Section~\ref{sec:cpclean}. When a change in $p(X)$ modifies $p(Y\vert X)$, this is known as a \textit{real drift} or a \textit{model drift}, whereas when $p(Y\vert X)$ stays intact it is a \textit{virtual drift}. Fortunately, virtual drifts have little to no impact on the trained ML model, assuming one managed to successfully approximate the posterior probability distribution over the entire feature space $\cX$.

\section{MLOps Task 1: Effective ML Quality Optimization}
\label{sec:cpclean}

\begin{figure}[t!]
\centering
\includegraphics[width=0.9\textwidth]{submissions/data-quality-ml-ops/figs/cpclean.png}
\caption{Illustration of the relation between Certain Answers and Certain Predictions~\cite{karlavs2020nearest}. On the right, Q1 represents a \textit{checking query}, whereas Q2 is a \textit{counting query}.}
\label{fig:cpclean}
\end{figure}


\vspace{-0.5em}
One key operation in MLOps is seeking a way to 
improve the quality (e.g., accuracy) of a model.
Apart from trying new architectures and models,
improving the quality and quantity of the training
data has been known to be at least as important~\cite{li2019cleanml,fernandez2014we}.
Among many other approaches, data cleaning~\cite{ilyas2019data}, the practice of fixing 
or removing noisy and dirty samples,
has been a well-known strategy for improving 
the quality of data.

\vspace{-0.5em}
\paragraph*{MLOps Challenge}
When it comes to MLOps, a challenge is that 
not all noisy or dirty samples matter equally 
to the quality of the final ML model. In other words --
when ``propagating'' through the ML training process,
noise and uncertainty of different input samples
might have vastly different effects. As a result,
simply cleaning the input data artifacts either
randomly or agnostic to the ML 
training process might lead to a sub-optimal improvement
of the downstream ML model~\cite{li2019cleanml}.
Since the cleaning task itself is often 
performed ``semi-automatically'' 
by human annotators, with guidance from automatic
tools, the goal of a \textit{successful} cleaning strategy from an MLOps perspective should be to minimize the amount of human effort. This typically leads to a partially cleaned dataset, with the property that cleaning additional training samples would not affect the outcome of the trained model (i.e., the predictions and accuracy on a validation set are maintained).

 \vspace{-0.5em}
\paragraph*{A Data Quality View}
A principled solution to the above 
challenge requires a \textit{joint}
analysis of the impact of incomplete and noisy data in the training
set on the quality of an ML model trained over
such a set. Multiple seminal works have studied this problem, e.g., ActiveClean~\cite{krishnan2016activeclean}.
Inspired by these, we introduced 
a principled framework called CPClean that models
and analyzes such a 
noise propagation process together with principled cleaning algorithms based on 
sequential information maximization~\cite{karlavs2020nearest}.

\paragraph{Our Approach: Cleaning with CPClean}
CPClean directly models the noise propagation ---
the noises and incompleteness introduce
multiple possible datasets, called \emph{possible worlds} in relational database theory,
and the impact of these noises to 
final ML training is simply the \textit{entropy} 
of training multiple ML models, one for each of these possible worlds. Intuitively,
the smaller the entropy, the less impactful
the input noise is to the downstream 
ML training process.
Following this, we start by initiating all possible worlds (i.e., possible versions of the training data after cleaning) by applying multiple well-established cleaning rules and algorithms independently over missing feature values.
CPClean then operates in multiple iterations. At each round, the framework suggests the training data to clean that minimizes the \textit{conditional entropy} of possible worlds over the partially clean dataset. Once a training data sample is cleaned, it is replaced by its cleaned-up version in all possible worlds. At its core, it uses a \emph{sequential information-maximization} algorithm that finds an approximate solution (to this NP-Hard problem) with theoretical guarantees~\cite{karlavs2020nearest}.
Calculating such an entropy is often difficult,
whereas in CPClean we provide efficient algorithms which can calculate this term in polynomial time for a specific family of 
classifiers, namely k-nearest-neighbour
classifiers (kNN).

This notion of learning over incomplete data using \emph{certain predictions} is inspired by research on \emph{certain answers} over incomplete data~\cite{abiteboul1995foundations, suciu2011probabilistic, arenas1999consistent}. In a nutshell, the latter reasons about \emph{certainty} or \emph{consistency} of the answer to a given input, which consists of a query and an incomplete dataset, by enumerating the results over all possible worlds. Extending this view of data incompleteness to non-relational operator (e.g., an ML model) is a natural yet non-trivial endeavor, and Figure~\ref{fig:cpclean} illustrates the connection.

\paragraph{Limitations}
Taking the downstream ML model into account for prioritizing human cleaning effort is not new. ActiveClean~\cite{krishnan2016activeclean} suggests to use information about the \textit{gradient} of a fixed model to solve this task. Alternatively, our framework relies on consistent predictions and, thus, works on an unlabeled validation set
and on ML models that are not differentiable. 
In~\cite{karlavs2020nearest}
we use kNN as a proxy to an arbitrary classifier, given its efficient implementation despite exponentially many possible worlds. However, it still remains to be seen
how to extend this principled framework 
to other types of classifiers. Moreover, combining both approaches and supporting a labor-efficient cleaning approach for general ML models remains an open research problem.

\section{MLOps Task 2: Preventing Unrealistic Expectations}
\label{sec:snoopy}

In DevOps practices, new projects are typically initiated with a \textit{feasibility study}, in order to evaluate and understand the probability of success. The goal of such a study is to prevent users with unrealistic expectations from spending a lot of of money and time on developing solutions that are doomed to fail. However, when it
comes to MLOps practices, such a 
feasibility study step is largely missing --- we often
see users with high expectations, but with a very noisy dataset, starting an
expensive training process which is almost surely
doomed to fail.


\paragraph*{MLOps Challenge}
One principled way to model the feasibility
study problem for ML is to ask:
\textit{Given an ML task, defined by its training and validation sets, how to estimate the 
error that the best
possible ML model can achieve, without
running expensive ML training?}
The answer to this question 
is linked to a traditional ML problem, i.e.,
to estimate the \emph{Bayes error rate} (also called \emph{irreducible error}). It is a quantity related to the underlying data distribution and estimating it using finite amount of data is known to be a notoriously hard problem. Despite decades of study~\cite{cover1967nearest, fukunaga1987bayes, sekeh2018multi},
providing a practical BER estimator is still 
an open research problem and there are no known practical 
systems that can work on real-world large-scale datasets. One key challenge to make feasibility
study a practical MLOps step is to understand 
how to utilize decades of theoretical studies
on the BER estimation and which  
compromises and optimizations to perform.


\begin{figure}[t!]
%\begin{subfigure}[t]{.32\textwidth}
\begin{subfigure}%{\linewidth}
\centering%\captionsetup{width=.3\textwidth}
%\label{fig:imnet_missingfeature}
\includegraphics[width=0.3\textwidth]{submissions/data-quality-ml-ops/figs/ILSVRC2012_val_00006874.JPEG}  
%\caption{}
\end{subfigure}
\hfill
\begin{subfigure}%{\linewidth}
\centering%\captionsetup{width=.3\textwidth}
\includegraphics[width=0.3\textwidth]{submissions/data-quality-ml-ops/figs/ILSVRC2012_val_00004463.JPEG}  
%\caption{Image (\#4463) showing multiple classes for a fixed sample.}
%\label{fig:imnet_missinglabel}
\end{subfigure}
\hfill
\begin{subfigure}%{.32\linewidth}
\centering%\captionsetup{width=.3\textwidth}
\includegraphics[width=0.3\textwidth]{submissions/data-quality-ml-ops/figs/ILSVRC2012_val_00032040.JPEG}  
%\caption{Image (\#32040) mislabeled as a ``pizza''.}
%\label{fig:imnet_labelnoise}
\end{subfigure}
\caption{ImageNet examples from the validation set illustrating possible reasons for a non-zero Bayes Error. Image (\#6874) on the left illustrates a non-unique label probability, image (\#4463) in the middle shows multiple classes for a fixed sample, and image (\#32040) on the right is mislabeled as a ``pizza''.}
\label{fig:imagenet_examples}
\end{figure}

\subparagraph{Non-Zero Bayes Error and Data Quality Issues}
At the first glance, even understanding why the BER is not zero for every task can be quite mysterious --- 
\textit{if we have enough amount of data and 
a powerful ML model, what would stop us from 
achieving perfect accuracy?}
The answer to this is deeply connected to 
data quality.
There are two classical data quality dimensions that constitute the reasons for a non-zero BER: (1)~\textit{completeness} of the data, violated by an insufficient definition of either the feature space or label space, and (2)~\textit{accuracy} of the data, mirrored in the amount of noisy labels. 
On a purely mathematical level, the reason for a non-zero BER lies in \textit{overlapping} posterior probabilities for different classes, given a realisable input feature.
More intuitively, for a given sample the label might not be unique.
In Figure~\ref{fig:imagenet_examples} we illustrate some real-world examples from the validation set of \textit{ImageNet}~\cite{deng2009imagenet}.
For instance, the image on the left is labeled as a golfcart (n03445924) 
whereas there is a non-zero probability that the vehicle belongs to another category, for instance a tractor (n04465501) -- additional features can resolve such an issue by providing more information and thus leading to a single possible label.
Alternatively, there might in fact be multiple ``true'' labels for a given image. The center image shows such an example, where the posterior of class rooster (n01514668) is equal to the posterior of the class peacock (n01806143), despite being only labeled as a rooster in the dataset -- changing the task to a multi-label problem would resolve this issue.
Finally, having noisy labels in the validation set yields another sufficient condition for a non-zero BER.
The image on the left shows such an example, where a pie is incorrectly labeled as a pizza (n07873807).

\paragraph*{A Data Quality View}
There are two main challenges in building 
a practical BER estimator for ML models to
characterize the impact of data quality to 
downstream ML models: (1) the computational requirements and (2) the choice of hyper-parameters. Having to estimate the BER in today's high-dimensional feature spaces
requires a large amount of data in order to give a reasonable estimate in terms of accuracy, which results in a high computational cost. Furthermore, any practical estimator should be insensitive to different hyper-parameters, as no information about the data or its underlying distribution is known \emph{prior to} running the feasibility study.

\begin{wrapfigure}{r}{0.5\textwidth}
\vspace{-2em}
\centering
\includegraphics[width=0.5\textwidth]{submissions/data-quality-ml-ops/figs/snoopy.png}
\label{fig:snoopy}
\vspace{-2em}
\end{wrapfigure}

\paragraph{Our Approach: \texttt{ease.ml/snoopy}}
We design a novel BER estimation method that (1) has no hyper-parameters to tune, as it is based on nearest-neighbor estimators, which are non-parametric; and (2) uses pre-trained embeddings, from public sources such as PyTorch Hub or Tensorflow Hub\footnote{\url{https://pytorch.org/hub} and \url{https://tfhub.dev}}, to considerably decrease the dimension of the feature space. The aforementioned functionality of performing a feasibility study using \texttt{ease.ml/snoopy} is illustrated in the above figure. For more details we refer interested readers to both the full paper~\cite{renggli2020automatic} and the demo paper for this component~\cite{renggli2020ease}. The usefulness and practicality of this novel approach is evaluated on well-studied standard ML benchmarks through a new evaluation methodology that injects label noise of various amounts and follows the evolution of the BER~\cite{renggli2020automatic}. It relies on our theoretical work~\cite{rimanic2020convergence}, in which we furthermore provide an in-depth explanation for the behavior of kNN over (possibly pre-trained) feature transformations by showing a clear trade-off between the increase of the BER and the boost in convergence speed that a transformation can yield.


\paragraph{Limitations}
The standard definition of the BER assumes that both the training and validation data are drawn i.i.d. from the \textit{same} distribution, an assumption that does not always hold in practice.
Extending our work to a setup that takes into account two different distributions for training and validation data, for instance as a direct consequence of applying data programming or weak supervision techniques~\cite{ratner2017snorkel}, offers an interesting line of future research, together with developing even more practical 
BER estimators for the i.i.d. case.


\section{MLOps Task 3: Rigorous Model Testing Against Overfitting}
\label{sec:ci}

One of the major advances in running fast and robust cycles in the software development process is known as continuous integration (CI)~\cite{duvall2007continuous}. The core idea is to carefully define and run a set of conditions in the form of tests that the software needs to successfully pass every time prior to being pushed into production. This ensures the robustness of the system and prevents unexpected failures of production code even when being updated.
However, when it comes to MLOps, the traditional 
way of reusing the same test cases repeatedly can 
introduce serious risk of overfitting, thus 
compromise the test result.

\paragraph*{MLOps Challenge}
In order to generalize to the unknown underlying probability distribution, when training an ML model, one has to be careful not to overfit to the (finite) training dataset. However, much less attention has been devoted to the statistical generalization properties of the \textit{test set}.
Following best ML practices, the ultimate testing phase of a new ML model should either be executed only once per test set, or has to be completely obfuscated from the developer. Handling the test set in one way or the other ensures that no information of the test set is \emph{leaked} to the developer, hence preventing potential overfitting. Unfortunately, in ML development environments it is often impractical to implement either of these two approaches.

\paragraph*{A Data Quality View}
Adopting the idea of continuously testing and integrating ML models in productions has two major caveats: (1) test results are inherently random, due to the nature of ML tasks and models, and (2) revealing the outcome of a test to the developer could mislead them into overfitting towards the test set.
The first aspect can be tackled by using well-established concentration bounds known from the theory of statistics.
To deal with the second aspect, which we refer to as the \textit{timeliness} property of testing data, there is an approach pioneered by Ladder~\cite{blum2015ladder}, together with the general area of \emph{adaptive analytics} (cf.~\cite{dwork2015reusable}), that enable multiple reuses of the same test set with feedback to the developers. The key insight of this line of work is that the \textit{statistical power} of a fixed dataset shrinks when increasing the number of times it is reused. In other words, requiring a minimum statistically-sound confidence in the generalization properties of a finite dataset limits the number of times that it can be reused in practice.

\paragraph{Our Approach: Continuous Integration of ML Models with \texttt{ease.ml/ci}}

\begin{figure}
\centering
\includegraphics[width=1.0\textwidth]{submissions/data-quality-ml-ops/figs/ci.png}
\caption{The workflow of \texttt{ease.ml/ci}, our CI/CD engine for ML models~\cite{renggli2019continuous}.}
\label{fig:ci}
\end{figure}

As part of the \texttt{ease.ml} pipeline, we designed a CI engine to address both aforementioned challenges. The workflow of the system is summarized in Figure~\ref{fig:ci}. The key ingredients of our system lie in (a) the syntax and semantics of the test conditions and how to accurately evaluate them, and (b) an optimized \emph{sample-size estimator} that yields a budget of test set re-uses before it needs to be refreshed. For a full description of the workflow as well as advanced system optimizations deployed in our engine, we refer the reader to our initial paper~\cite{renggli2019continuous} and the followup work~\cite{karlavs2020building}, which further discusses the integration into existing software development ecosystems.

We next outline the key technical details falling under the 
general area of ensuring generalization properties of finite data used to test the accuracy of a trained ML model repetitively.

\subparagraph{Test Condition Specifications}

A key difference between classical CI test conditions and testing ML models lies in the fact that the test outcome of any CI for ML engine is inherently probabilistic. Therefore, when evaluating result of a test condition that we call a \textit{score}, one has to define the desired confidence level and tolerance as an $(\epsilon,\delta)$ requirement. Here $\epsilon$ (e.g., $1\%$) indicates the size of the confidence interval in which the estimated score has to lie with probability at least $1-\delta$ (e.g., $99\%$). For instance, the condition \verb|n - o > 0.02 +/- 0.01| requires that the new model is at least 2 points better in accuracy than the old one, with a confidence interval of 1 point. Our system additionally supports the variable \verb|d| that captures the fraction of different predictions between the new and old model.  
For testing whether a test condition passes or fails one needs to distinguish two scenarios. On one hand, if the score lies outside the confidence interval (e.g., \verb|n - o > 0.03| or \verb|n - o < 0.01|), the test immediately passes or fails.
On the other hand, the outcome is ill-defined if the score lies inside the confidence interval. Depending on the task, user can choose to allow \emph{false positive} or \emph{false negative} results (also known as ``type I'' and ``type II'' errors 
in statistical hypothesis testing), after which all the scores lying inside the confidence interval will be automatically rejected or accepted.


\subparagraph{Test Set Re-Uses}

In the case of a non-adaptive scenario in which no information is revealed to the developer after running the test, the least amount of samples needed to perform $H$ evaluations with the same dataset is the same as running a single evaluation with $\delta / H$ error probability, since the $H$ models are independent.
Therefore, revealing any kind of information to the developer would result in a dataset of size $H$ multiplied by the number of samples required for one evaluation with $\delta / H$ error. However, this trivial strategy is very costly and usually impractical. The general design of our system offers a different approach that significantly reduces the amount of test samples needed for using the same test set multiple times. More precisely, after every commit the system only reveals a binary pass/fail signal to the developer. Therefore, there are $2^H$ different possible sequences of pass/fail responses, 
which yields that the number of samples needed for $H$ iterations is the same as running a single iteration with $\delta / 2^H$ error probability -- much smaller than the previous $\delta / H$ one. 
We remark that further optimizations can be deployed by making use of the structure or low variance properties that are present in certain test conditions, for which we refer the interested readers to the full paper~\cite{renggli2019continuous}.

\paragraph{Limitations}

The main limitation consists of the worst-case analysis which happens when the developer acts as an adversarial player that aims to overfit towards the hidden test set. Pursuing other, less pragmatic approaches to model the behavior of developers could enable further optimization to reduce the number of test samples needed in this case.
A second limitation lies in the lack of ability to handle concept shifts. Monitoring a concept shift could be thought of as a similar process of CI -- instead of fixing the test set and testing multiple models, one could fix a single model and test its generalization over multiple test sets. From that perspective, we hope that some of the optimizations that we have derived in our work could potentially be applied to monitoring concept shifts as well. Nevertheless, this needs further study and forms an interesting research path for the future.

%\newpage
\section{MLOps Task 4: Efficient Continuous Quality Testing}
\label{sec:model_picker}

One of the key motivations for DevOps principles in the first place is the ability to perform fast cycles and continuously ensure the robustness of a system by quickly adapting to changes. At the same time, both are well-known requirements from traditional software development that naturally extend to the MLOps world.
One challenge faced by many MLOps practitioners is the necessity
to deal with the shift of data distributions when models are
in production. When new production 
data comes from a different (unknown) distribution,
models trained over previously seen 
data distributions might not perform well
anymore. 

\paragraph*{MLOps Challenge}
While there has been various research on 
automatic domain adaption~\cite{shimodaira2000improving, sugiyama2008direct, zhang2013domain}, we identify a 
different challenge when presented with a collection of models, each of which could be a ``staled model'' or 
an automatically adapted model given some domain adaption method. 
This scenario is quite common in many 
companies --- they often train distinct models on
different slices of data independently (for instance one model for each season) and 
automatically adapt each of these
models using different methods for new data. As a result,
they often have access to a large 
set of models that could be deployed, hoping to know which one to use given
a fresh collection of production data (e.g., the current time period such as the current day).
The challenge is, given an unlabeled 
production data stream, to pick the model that
performs best. From the MLOps perspective,
the goal is to minimize the amount of labels needed to 
acquire in order to make such a distinction. 


\paragraph*{A Data Quality View}
Concept shift is by its definition related to the \textit{timeliness} properties of the data. The available pre-trained models are intended to capture the changes of training data over time. Naturally, simple rules can be applied to choose the current model if one has access to some meta information about both the current timestamp and the pre-trained models (e.g., the current weekday and for each model the day it represents).
This problem gets particularly difficult when there are no such meta-data available. In that case, having access to a fully labeled clean dataset would result in trivially selecting the pre-trained model that achieves the highest accuracy. Collecting labels for a large enough test set is very costly in practice compared to simply gathering a set of unlabeled samples though. The reason is that accurately labeling samples requires human, if not expert level, annotators. Consequently, one wishes to robustly solve the problem of \textit{picking} the best model for the current time span with the fewest amount of labeling effort necessary and thus relying on \textit{incomplete} test data with respect to their labels.

\paragraph{Our Approach: \texttt{ease.ml/ModelPicker}}
Model Picker is an online model selection approach to selectively sample instances that are informative for ranking pre-trained models~\cite{karimi2020online}. Specifically, given a set of pre-trained models and a stream of unlabeled data samples that arrive sequentially from a data source, the Model Picker algorithm answers when to query the label of an instance, in order to pick the best model under limited labeling budget. We conduct a rigorous theoretical analysis to show that Model Picker has no regret for adversarial streams (e.g., non-i.i.d. data), and is effective in online prediction tasks for both adversarial and stochastic streams. Moreover, our theoretical bounds match (up to constants) those of existing online algorithms that have access to all the labels. 

\paragraph{Limitations}
One immediate extension of Model Picker is towards a setting in which the user at once has access to a pool of unlabeled data samples. In such a \textit{pool-based sampling} case~\cite{settles2009active}, one can rank the entire collection of data samples to select the most informative example instead of scanning through the data sequentially to decide whether to query a label or not. Despite the applicability of Model Picker to such a scenario where one can form a stream by sampling i.i.d. from the pool of samples, the availability of entire data collection can be exploited to further reduce the annotation costs with a more specialized strategy for pool-based scenarios.

%\newpage
\section{Moving Forward}
\label{sec:limitations}

We have briefly described four 
of our previous works with a unified theme ---
all of them provide, in our opinion, \textit{
functionalities that are useful to 
facilitate a better MLOps process}, which, on 
the flip side, introduce new fundamental 
technical problems that require us to 
\textit{jointly analyze the impact
of data quality issues to downstream ML processes.}
When studying these technical problems,
we often need to 
go beyond an ML-agnostic view 
of data quality and, instead, need to develop 
new methods that \emph{simultaneously} combine the two aspects of ML \emph{and} data quality.
Despite the progress that we have made so far, this endeavor is still at its early stages. 
In the following, we present two future directions that, in our opinion,
are necessary to facilitate both MLOps as an important
functionality and ML-aware data quality as a fundamental research area.


\paragraph*{ML-Aware Data Quality}
From a technical perspective,
jointly understanding data quality and 
downstream ML processes is both interesting
and challenging. All results we discussed
in this paper are arguably limited~\cite{karlavs2020nearest, renggli2020automatic, karimi2020online, renggli2019continuous} --- after starting 
from a principled formulation of a 
problem, reaching fundamental 
computational challenges within these  
principled frameworks is inevitable. We get around
those by either (1) opting for simpler 
proxy models for which we can derive 
stronger results and/or more efficient algorithms
(e.g., kNN used in \texttt{ease.ml/snoopy}~\cite{renggli2020automatic}
and CPClean~\cite{karlavs2020nearest})
or (2) optimizing for specific cases 
commonly used in practice (e.g., the patterns in \texttt{ease.ml/ci}~\cite{renggli2019continuous} that we optimized for).
To further facilitate MLOps in general,
we are in dire need for an 
ML-aware data quality that is not only principled,
but also practical for a larger 
collection of scenarios and ML models.
These are all technically challenging --- simply
extending the methods that we developed is unlikely to succeed.
We hope that our current endeavors~\cite{karlavs2020nearest, renggli2020automatic, karimi2020online, renggli2019continuous} can serve, in
some ways, as ``examples of failures''
that other researchers can draw inspirations from.

\paragraph{Beyond Accuracy}
Another common limitation of our work~\cite{karlavs2020nearest, renggli2020automatic, karimi2020online, renggli2019continuous}
is that they all focus on improving the \textit{accuracy} of an ML model artifact. Although this is one of the most important 
aspects of model quality, recently researchers have
also identified multiple interesting dimensions
of model quality such as robustness, fairness,
and explainability. 
Even though we expect these quality dimensions to
become the core of the MLOps process in the future,
how to extend functionalities that we 
developed for improving accuracy to 
these quality dimensions is still an open question.
Jointly analyzing the impact of all data-quality dimensions with respect to more than a single metric that quantifies ML models is a large and promising research area that we believe will provide further understanding and improvements of the MLOps process.


\section*{Acknowledgments}
\small
We are grateful to many collaborators that we have 
been working with over the years, especially (in the context of this paper) those
who also contributed to the development of techniques in \cite{karlavs2020nearest, renggli2020automatic, karimi2020online, renggli2019continuous} including Peng Li, Prof. Xu Chu, Mohammad Reza Karimi, and Prof. Andreas Krause. These technical results would not be possible without their contributions.

CZ and the DS3Lab gratefully acknowledge the support from the Swiss National Science Foundation (Project Number 200021\_184628), Innosuisse/SNF BRIDGE Discovery (Project Number 40B2-0\_187132), European Union Horizon 2020 Research and Innovation Programme (DAPHNE, 957407), Botnar Research Centre for Child Health, Swiss Data Science Center, Alibaba, Cisco, eBay, Google Focused Research Awards, Oracle Labs, Swisscom, Zurich Insurance, Chinese Scholarship Council, and the Department of Computer Science at ETH Zurich.

\newpage
\bibliographystyle{abbrv}
\bibliography{submissions/data-quality-ml-ops/ieeedeb}

\end{document}
\end{article}

\begin{article}
{From Cleaning before ML to Cleaning for ML}
{Felix Neutatz, Binger Chen, Ziawasch Abedjan and Eugene Wu}
% link to instruction: https://tc.computer.org/tcde/tcde-bulletin-author-instructions/
% \documentclass[11pt,dvipdfm]{article}
\documentclass[11pt]{article}
\usepackage{tabularx}
\usepackage{ragged2e}  % for '\RaggedRight' macro (allows hyphenation)
\usepackage{booktabs}  % for \toprule, \midrule, and \bottomrule macros
\usepackage{textcomp}
\usepackage{amsfonts,amsmath}
\usepackage{deauthor,times}
\usepackage{graphicx} % 
\usepackage{hyperref}
\usepackage{comment}
\graphicspath{{asudeh/}}
\usepackage{soul}
\usepackage{subcaption}
\usepackage{ulem}
\usepackage{wrapfig}
\usepackage{color}
\usepackage{xspace}
\newtheorem{problem}{Problem}

%\DeclareMathOperator*{\argmax}{arg\,max}

%remove the following commands/package b4 submission
\newcommand{\hide}[1]{}
\newcommand{\eat}[1]{}
\newcommand{\resolved}[1]{\hide{#1}}
\newcommand{\abol}[1]{\textcolor{red}{Abol: #1}}
\newcommand{\mahdi}[1]{\textcolor{red}{Mahdi: #1}}
\newcommand{\nima}[1]{\textcolor{red}{Nima: #1}}

\newcommand{\dee}{\mathcal{D}}
\newcommand{\Gee}{\mathcal{G}}
\newcommand{\gee}{\mathbf{g}}
\newcommand{\ee}{\mathbf{e}}
\newcommand{\es}{\mathcal{S}}
\newcommand{\el}{\mathcal{L}}
\newcommand{\xx}{\mathcal{x}}
\newcommand{\dist}{\xi}
\newcommand{\alg}{\mathsf{A}}
\newcommand{\qu}{\mathbf{q}}
\newcommand{\ex}{\mathbf{x}}
\newcommand{\ti}{\mathbf{t}}
\newcommand{\sdt}{\mathsf{SDT}}
\newcommand{\wdt}{\mathsf{WDT}}
\newcommand{\Qu}{\mathbf{Q}}
\newcommand{\pe}{\mathbb{P}}
\newcommand{\megam}{\mathcal{M}}
\newcommand{\eps}{\varepsilon}
\newcommand{\enet}{{$\varepsilon$-{\bf net}}\xspace}
\newcommand{\net}{{\tt net}\xspace}
\newcommand{\vcd}{VC-dimension\xspace}
\newcommand{\at}[1]{{\tt \small #1}\xspace}
\newcommand{\pr}{Pr}

\newcommand{\sharpP}{\mbox{\#P}}
\newcommand{\NP}{\mathsf{NP}}
\newcommand{\LP}{\mathsf{LP}}
\newcommand{\IP}{\mathsf{IP}}
\newcommand{\ru}{{\sc {RU}}\xspace}
\newcommand{\sru}{{\sc {strongRU}}\xspace}
\newcommand{\wru}{{\sc {weakRU}}\xspace}

\newcommand{\fmsystem}{{\sc Chameleon}\xspace}
\newcommand{\fm}{$\mathcal{F}$\xspace}

\newtheorem{experiment}{Experiment}

\begin{document}

\title{Coverage-based Data-centric Approaches for \\Responsible and Trustworthy AI\thanks{This research was supported by the National Science Foundation under grant No. 2107290.}}

\author{
\begin{tabular}[t]{c@{\extracolsep{2.4em}}c@{\extracolsep{2.4em}}c@{\extracolsep{2.3em}}c} 
Nima Shahbazi & Mahdi Erfanian & Abolfazl Asudeh \\ 
University of Illinois Chicago & University of Illinois Chicago & University of Illinois Chicago\\
 nshahb3@uic.edu & merfan2@uic.edu & asudeh@uic.edu
\end{tabular}
}

\maketitle


\begin{abstract}
The grand goal of data-driven decision systems is to help make decisions easier, more accurate, at a higher scale, and also just. However, data-driven algorithms are only as good as the data they work with. Yet, data sets, especially those with social data, often do not represent minorities. The paucity of training data is a perpetual problem for AI, and the outcome of ML models for cases not represented in their training data is often not reliable. 
Hence, without properly addressing the lack of representation issues in data, we cannot expect AI-based societal solutions to have responsible and trustworthy outcomes. 

This paper focuses on data coverage as a data-centric approach for identifying and resolving misrepresentation of minorities in data.
To achieve this goal, we propose novel algorithms that (a) {\it identify} and {\it resolve} insufficient data coverage across data with different modalities and (b) use lack of representation information to generate data-centric {\it reliability warnings}.
 \end{abstract}
 
 %%%%%%%%%%%%%%%%%%%%%%%%%%%%%%%% INTRO  %%%%%%%%%%%%%%%%%%%%%%%%%%%%%%%%
\section{Introduction}\label{sec:intro} % Abstract+Intro: up to 2.5 pages 
Data-driven decision-making has shaped every corner of human life, spanning from autonomous vehicles to healthcare and even predictive policing and criminal justice. A pivotal concern, especially in applications that affect individuals, revolves around the reliability of the decisions rendered by the system.
It is easy to see that the accuracy of a data-driven decision depends, first and foremost, on the data used to make it. Essentially, the system learns the phenomena that data represent. While we may desire that the data should represent the underlying data distribution from which the production data is drawn, this alone may be insufficient, as it merely enables the model to perform well for the average case.
As a result, a model with a high accuracy could fail for specific regions in the data with insufficient representation. These regions may matter because they frequently represent some minority population in society. They could also represent cases that may not happen very often but have a relevant impact on the correctness of a critical decision.
In short, if the data fails to sufficiently represent a specific population, the outcome of the decision system for that population may not be trustworthy.

The phenomenon known as \textit{Representation Bias} can arise from how the data was originally collected, or it could be the result of biases introduced post-collection—whether historically, cognitively, or statistically.

Representation bias is essentially inevitable without a systematic approach to data collection. 
For example, in the context of survey data collection, vital steps involve identifying all populations within the underlying distribution based on desired demographic information and ensuring comprehensive coverage with sufficient samples from each group. 
Even then, only an (uncontrolled) subset of the invitees will opt-in to respond to the survey.
Another challenge lies in the fact that data scientists often lack control over the data collection process, leading to the reliance on ``found data'' in the majority of data-driven systems. Therefore, with no guarantee on the aforementioned steps in the data collection process, the found data is most likely a biased sample.
Acknowledging the potential harms of representation bias, the notion of \textit{Data Coverage}~\cite{asudeh2019assessing,shahbazi2023representation} has been proposed to ensure the adequate representation of minority groups in data sets employed for decision-making and developing sophisticated data science tools. 

Addressing representation issues in data poses various challenges depending on the modality of the data. In this paper, we focus on identifying and resolving lack of coverage issues in data with different modalities.
We start by proposing a variety of techniques (spanning from geometric and combinatorial optimization to crowd-souring) aimed at efficiently detecting insufficient coverage on structured data sets with non-ordinal categorical and continuous attributes, as well as image data sets. Next, we propose a range of approaches grounded in data integration and generative data augmentation to address the lack of coverage by enriching the data sets with more data. However, with limited control over the data collection processes, it could be difficult and expensive to resolve all misrepresentations. 
Since adding more data is not always possible, we proceed to introduce data-centric preventive solutions that warn the user about the reliability of their predictions regarding representation bias issues. These warnings assist users in determining whether they trust the outcomes of the models or exercise caution. 

 %%%%%%%%%%%%%%%%%%%%%%%%%%%%%%%% IDENTIFICATION  %%%%%%%%%%%%%%%%%%%%%%%%%%%%%%%%
\section{Detecting Insufficient Representation of Minorities}\label{sec:identification} %up to 3.5 pages
Representation bias happens when the development (training data) population under-represents 
and subsequently fails to generalize well 
for some parts of the target population, due to historical bias, sampling bias, etc.
The notion of {\it data coverage} has been studied across different settings in \cite{shahbazi2023representation} as a metric to measure representation bias. At a high level, coverage is referred to as having enough similar entries for each object in a data set. 
For a better understanding, let us go over the definition of the generalized notion of coverage:

\begin{definition}[Data Coverage]\label{def:coverage}
Consider a data set $\dee$ with $n$ tuples, each consisting of $d$ attributes of interest $\mathbf{x}=\{x_1, x_2, \cdots,x_d\}$, such as {\tt gender}, {\tt race}, {\tt salary}, {\tt age}, etc, that are used for coverage identification.
The data set also contains target attributes $\mathbf{y} = \{ y_1,\cdots,y_{d'}\}$ that may or may not be considered for the coverage problem.
A query point $q$ is not covered by the data set $\dee$, if there are not ``enough'' data points in $\dee$ that are representative of $q$.
To generalize the notion of coverage, let us define $\gee(q)$ as the universe of tuples that would represent $q$ and let $\gee_\dee(q) = \gee(q)\cap \dee$. In other words, $\gee_\dee(q)$ are the set of tuples in $\dee$ that represent $q$.
Using this notation, we define the coverage of $q$ as the size of $\gee_\dee(q)$. That is,
$cov(q,\dee) = | \gee_\dee(q)|$.
Given a value $\tau$, $q$ is covered if $cov(q,\dee)>\tau$.
Similarly, a group $\gee$ is not covered if $\gee\cap \dee<\tau$.
The {\it uncovered region} in a data set is the collection of groups that are not covered by it.
\end{definition}

\subsection{Structured Data}
In this section, we focus on identifying representation bias in structured data.
Depending on the type of the attributes of interest, we categorize the techniques into two classes based on whether they target the problem for non-ordinal {\it categorical} (e.g. {\tt race}, {\tt gender}) or ordinal {\it continuous} (e.g. {\tt age}) attributes. The attributes of interest considered for representation bias often include sensitive attributes such as {\tt race} and {\tt gender} but are not necessarily limited to them.

\subsubsection{Categorical Attributes}

For cases where attributes of interest are non-ordinal categorical,
the cartesian product of values on a subset of attributes $\mathbf{x}'\subseteq \mathbf{x}$, form a set of (sub-)groups.
For example, $\{$ {\tt white male}, {\tt white female}, {\tt black male} $,\cdots\}$ are the subgroups defined on the attributes {\tt (race,gender)}.
We refer to the number of attributes used to specify a subgroup as the {\it level} of that subgroup.
For example, the level of the subgroup {\tt white male} is 2, while the level of the subgroup {\tt male} is 1.
We use $\ell(\gee)$, to refer to the level of a subgroup $\gee$.
Similarly, we say a subgroup $\gee'$ is a subset of $\gee$, if the groups specifying $\gee'$ are a superset of the ones for $\gee$. For example {\tt (married white male)} a subset of the more general group {\tt (white male)}. That is, the set of individuals in group {\tt (married white male)} are a subset of {\tt (white male)}.
Moreover, we say a subgroup $\gee$ is a {\it parent} of the subgroup $\gee'$, if $\gee'\subset \gee$ and $\ell(\gee)=\ell(\gee')+1$. For example, the subgroup {\tt (white male)} is a parent of the subgroup {\tt (married white male)}.
We use \textit{patterns} to refer to uncovered subgroups.
A pattern $P$ is a string of $d$ values, where $P[i]$ is either a value from the domain of $x_i$, or it is ``unspecified'', specified with $X$. 
For example, consider a data set with three binary attributes of interest $\mathbf{x}=\{x_1, x_2, x_3\}$. The pattern $P=X01$ specifies all the tuples for which $x_2=0$ and $x_3=1$ ($x_1$ can have any value).
The set of patterns that identify most general uncovered subgroups are called {\it Maximal Uncovered Patterns} (MUPs).

No polynomial time algorithm can guarantee the enumeration of the entire MUPs, however, several algorithms inspired by set enumeration and the Apriori algorithm for association rule mining are proposed to efficiently address this problem~\cite{asudeh2019assessing}.
In this regard, we introduce \textit{Pattern Graph} data structure that exploits the relationship between patterns to do less work than computing all uncovered patterns by removing the non-maximal ones. 
The parent-child relationship between the patterns is represented in a graph that can be used to find better algorithms. 
\textit{Pattern-Breaker} starts from the top of the graph where the general patterns are and moves down by breaking each pattern into more specific ones. If a pattern is uncovered, then all of its descendants are also uncovered and they can not be an MUP, even if they have a parent that is covered. Therefore, this subgraph of the pattern graph can be pruned. 
The issue with \textit{Pattern-Breaker} is that it explores the covered regions of the pattern graph and for the cases where there are a few uncovered patterns, it has to explore a large portion of the exponential-size graph. 
To tackle this, \textit{Pattern-Combiner} algorithm is proposed that performs a bottom-up traversal of the pattern graph. It uses an observation that the coverage of a node at the level of the pattern graph can be computed as the sum of the coverage values of its children. 
The problem with \textit{Pattern-Combiner} is that it traverses over the uncovered nodes first and therefore, it will not perform well for the cases in which most of the nodes in the graph are uncovered. 
In fact, for the cases where most of the MUPs are placed in the middle of the graph, both \textit{Pattern-Breaker} and \textit{Pattern-Combiner} will not be as efficient as they should traverse half of the graph. Therefore, we propose \textit{Deep-Diver}, a search algorithm based on Depth-First-Search that quickly finds the MUPs, and uses them to limit the search space by pruning the nodes both dominating and dominated by the discovered MUPs.

\begin{figure*}[!tb]
    \begin{minipage}[t]{0.31\linewidth}
        \centering
        \includegraphics[width=\textwidth]{submissions/submission1/shahbazi/covcube1.jpg}
        \caption{\small Categorical attributes: the uncovered region of a toy example, as the collection of three MUPs.}
        \label{fig:covcube1}
    \end{minipage}
    \hfill
    \begin{minipage}[t]{0.31\linewidth}
        \centering
        \includegraphics[width=\textwidth]{submissions/submission1/shahbazi/cvrg_2_1.jpg}
        \caption{\small Continuous attributes, 2D: identifying the covered region in the gray Voronoi cell.}
        \label{fig:cvrg_2_1}
    \end{minipage}
    \hfill
    \begin{minipage}[t]{0.31\linewidth}
        \centering
        \includegraphics[width=\textwidth]{submissions/submission1/shahbazi/cvrg_2_2.jpg}
        \caption{ \small Continuous attributes, 2D: Uncovered region marked in red.}
        \label{fig:cvrg_2_2}
    \end{minipage}
\vspace{-5mm}
\end{figure*}

\subsubsection{Continuous Attributes}
Data in the real world often consists of a combination of continuous and discrete values. While simple solutions like binning {\tt age} into {\tt young} and {\tt old} can transform the continuous space into discrete. However, they may lead to coarse groupings that are sensitive to the thresholds chosen. It may be inappropriate to treat a 35-yo as {\tt young} but a 36-yo as {\tt old}. 
Therefore, we extend the notion of coverage to continuous space. Particularly, given data set $\dee$ with $n$ tuples over $d$ attributes, and vicinity radius $\rho$ and coverage threshold $k$, we want to identify the uncovered region -- the universe of uncovered query points.
A query point in continuous data space is covered if there are enough (at least $k$) data points in its $\rho$-vicinity neighborhood. $\rho$-vicinity neighborhood is the circle centered at the query point with radius $\rho$.

Depending on the number of attributes in a data set, we propose two algorithms for identifying uncovered regions in data~\cite{asudeh2021coverage}. 
The first algorithm known as \textit{Uncovered-2D} studies coverage over two-dimensional data sets where $\mathbf{x}=\{x_1,x_2\}$. To find the number of circles that a query point falls into and consequently discover the uncovered region, \textit{Uncovered-2D} makes a connection to $k$-th order Voronoi diagrams.
Consider a data set $\mathcal{D}$ and its corresponding $k$-th order Voronoi diagram. For every tuple $t\in \mathcal{D}$, let $\circ_t$ be the $d$-dimensional sphere ($d$-sphere) with radius $\rho$ centered at $t$.
Consider a $k$-voronoi cell $\mathcal{V}(S)$ in the $k$-th order Voronoi diagram $V_k(\mathcal{D})$.
Any point $q$ inside the intersections of the $d$-spheres of tuples in $S$, i.e. $q\in \underset{\forall t\in S}{\cap ~\circ_t}$, is covered, while all other points in the region are uncovered.
 The algorithm starts by constructing the $k$-th order Voronoi diagram of the data set and then for each Voronoi cell $\mathcal{V}(S)$ in the diagram, it computes the intersection of the circles of the tuples in $S$ and marks the portion of $\mathcal{V}(S)$ that falls outside it as uncovered.
After identifying the uncovered region, a 2D map of $\{x_1,x_2\}$ value combinations is used to report the region to the user.
The algorithm for the 2D case can be extended to the general case by relaxing the assumption on the number of attributes to discover the exact uncovered region, however, due to the curse of dimensionality, the search size space explodes as the number of dimensions increases and as a result, the algorithm will not be practical. Therefore, we propose a randomized approximation algorithm based on the geometric notion of \enet. 
Let $\mathcal{X}$ be a set and $\mathcal{R}$ be a set of subsets of $\mathcal{X}$. A set $\mathcal{N}\subset \mathcal{X}$ is an \enet for $\mathcal{X}$ if for any range $r\in\mathcal{R}$, if  $|r\cap \chi|>\eps|\chi|$, then $r$ contains at least one point of $N$.
The idea, at a high level, is to draw enough random samples from the space of potential query points to form an \enet. 
We then label the sampled query points as $\{-1,+1\}$ depending on whether those are covered or not, and learn the uncovered regions using the samples.

\subsection{Image Data}
Many known incidents of machine failures due to the lack of representation were on image data.
We consider an image data set with a fixed number of low-cardinality sensitive attributes such as {\tt\small race} and {\tt\small gender}. 
It is common that image data sets {\it lack explicit values} for sensitive attributes, which are crucial for coverage identification. An image data set is often a collection of images from different domains with little to no information about their domain and which groups they belong to. As a result, even studying coverage over low-cardinality and categorical attributes of interests is challenging in these cases.

\begin{wrapfigure}{R}{0.42\textwidth}
\centering
\vspace{-3mm}
\scriptsize
\begin{tabular}{|@{}c|@{}c@{}|@{}c@{}|@{}c@{}|} 
 \hline
{\bf data set} & {\bf classifier} & {\bf accuracy} & {\bf precision} \\ 
 &  &  & {\bf on female} \\ \hline
UTKFace:~& DeepFace (opencv) & 93.56 & {52.02}\\\cline{2-4}
({\tt females}=200,& DeepFace (retinaface) & 94.16 & {56.15}\\\cline{2-4}
{\tt males}=2800) & BaseCNN & 97.6 & 74.8\\
\hline
UTKFace:~& DeepFace (opencv) & 96.53 & {\bf 8.0}\\\cline{2-4}
({\tt females}=20,& DeepFace (retinaface) & 96.43 & {\bf 10.09}\\\cline{2-4}
{\tt males}=2980)& BaseCNN & 97.6 & {\bf 21.59}\\
\hline
\end{tabular}
\vspace{-3mm}
\caption{\small ML models' low performance for females in the presence of representation bias.~\cite{mousavi2024data}}\label{fig:mlfails}
\vspace{-3mm}
\end{wrapfigure}

In Figure~\ref{fig:mlfails}, we show that due to the issues such {\it machine bias} and {\it lack of distribution generalizability},
solely relying on state-of-the-art machine learning (ML) techniques fail to effectively identify lack of coverage in image data sets. Therefore, we propose an approach based on combining crowdsouring with ML~\cite{mousavi2024data}. 
Crowdsourcing is particularly promising for image data, for tasks such as image labeling, which, while challenging for the machine, are "easy" for human beings to conduct with minimal error. 

A key observation that enables a cost-effective crowdsourcing approach is that, while studying coverage, we would only like to find out if there are {\it enough tuples from each subgroup}.
Suppose a subgroup is covered if there are $\tau=100$ instances of it in the data set. Assume the (majority) group $\gee_1$ contains $n_1 \gg 100$ objects in the data set. 
To verify that $\gee_1$ is covered, it is enough for the crowd to discover 100 of those objects, not the entire $n_1$. 
Following this, $O(\tau)$ provides a lower bound on the number of crowd tasks required to verify a given group is covered. 
Still, this lower bound only holds for the groups that are covered, i.e., there is at least $\tau$ of those in the data set.
Surprisingly, verifying that a minority group is indeed uncovered is cumbersome, unlike the majority group.
This is because even though discovering $\tau$ objects from a group is enough for verifying that it is covered, one cannot {\it verify} a group is uncovered until there is a chance that the data set might still have enough objects from that group. Thus, assuming a non-zero probability for each unlabeled object to belong to each group, {one might need to ask the crowd to label the entire data set before they can confirm that a specific group is uncovered}.

Our idea for addressing this challenge is to
design {\it a divide and conquer algorithm} that, instead of {point queries}, uses {\it set queries} to iteratively eliminate subsets of data that {does not include any object from the given group}.
At a high level, our idea is to ask a set query from the crowd, inquiring whether the selected set contains at least one object from the given group $\gee$.
The user may provide two responses (yes/no). 
Interestingly, {in either case}, the user response provides valuable information that helps efficiently identify the coverage.
If the answer is ``No'', the set does not include any object from the given group $\gee$. As a result, the algorithm can safely prune the set, asking no further questions about it. In particular, for a group that is not covered, one can expect to see no answers on large set queries helping to prune a significant portion of the data set quickly.
On the other hand, if the answer is ``yes'', the set contains {at least} one object from the group $\gee$. As a result, the algorithm cannot prune the subset since it can have any number (larger than one) of the objects in $\gee$.
At first glance, the queries with yes answers do not provide helpful information as the algorithm cannot prune the subset (hence it needs to divide it into smaller subsets).
However, a key observation is that {the algorithm will only observe a limited number of yes answers} before it stops.
The reason is that the number of set queries with yes answers provides a {lower-bound} on the number of objects from $\gee$ in the data set. As a result, the algorithm can stop as soon as the lower bound reaches $\tau$, knowing that $\gee$ is covered.
The D\&C approach verifies the data coverage for a given group, while our goal is to identify the uncovered regions for a given set of sensitive attributes. The next question is how to utilize this algorithm for efficient coverage identification on different scenarios of sensitive attributes, forming intersectional or non-intersectional groups.
In particular, how can we find maximal uncovered patterns?
Our idea is to apply sampling and aggregate estimation techniques to find the groups that even if merged are likely to still be uncovered. This will help reduce the coverage identification cost by running the D\&C approach for the merged groups once.
 %%%%%%%%%%%%%%%%%%%%%%%%%%%%%%%% RESOLUTION  %%%%%%%%%%%%%%%%%%%%%%%%%%%%%%%%
\section{Resolving Insufficient Representation}\label{sec:resolution}

Data integration~\cite{nargesian2021tailoring,nargesian2022responsible} and data augmentation~\cite{sharma2020data,DBLP:journals/jair/ChawlaBHK02,iosifidis2018dealing,celis2020data} are considered as the primary solutions for reducing data coverage issues in a data set. 
Data integration is promising when external sources of data are available. On the other hand, recent advancements in generative AI and foundation models have enabled efficient and effective augmentation of data sets with synthetic data. 
Therefore, in the following, we review two approaches, one from each category, in the context of lack of coverage resolution.

\subsection{Data Integration}\label{sec:resolution:integration}

Data integration is to consolidate data from different sources into a single, unified view. 
Although it is an effective solution to acquire additional data from different distributions,
there are sampling policy and cost-efficiency concerns that need to be examined.  
Therefore, {\it Data Distribution Tailoring ({\sc DT})} introduces data integration techniques for resolving insufficient representation of subgroups in a data set in the most cost-effective manner~\cite{nargesian2021tailoring}.
A query to {\sc DT} 
consists of a target schema, and a set of group distribution requirements in the form of the minimum counts (e.g., ``{\tt\small 1,000 breast cancer monitoring data in Chicago with at least 30\% label=positive, and at least 20\% black patients}''). 
Collecting a fresh sample from a data view is costly (monetary, human resources, and/or computation cost)~\cite{asudeh2022towards}.
Therefore, {\sc DT} focuses on satisfying the count requirements with minimum cost. 
Given an input query and a lake of available data sources, the first step is to discover a collection of candidate data views that satisfy the target schema.
Each data view $v_i$ is a projection-join $v_i = \Pi\big(D_{i1}\bowtie\cdots\bowtie D_{ik_i} \big)$, where $D_{ij}$ is a data set in a given data lake.
Let us suppose the data views are already discovered.
At a high level, {\sc DT} follows an iterative approach that at each iteration a data view is selected to be queried.
Each query to a data view has a fixed cost and returns a sample that may or may not satisfy the query constraints.
The samples that are either not fresh, or do not satisfy the query are discarded.
Hence, the essential question towards a cost-effective data integration is {\it what data view to query next}.
Depending on the available information about the data sources, various techniques may be employed. 

For the cases when the group distributions are known, the process of collecting the target data set is a sequence of iterative steps, where at every step, the algorithm chooses a data view, queries it, and if the obtained tuple contributes to one of the groups for which the count requirement is not yet fulfilled, it is kept, otherwise discarded. To do so, a {Dynamic Programming (DP)} algorithm is proposed. An optimal source at each iteration minimizes the sum of its sampling cost plus the expected cost of collecting the remaining required groups, based on its sampling outcome.
The DP algorithm, however, has a pseudo-polynomial time complexity. Hence, it quickly becomes intractable for cases where the minimum count requirements for the groups are not small. 
For cases where the (sensitive) attribute of interest is binary, such as (biological) {\tt sex}={\tt \{male, female\}}, and the cost to query data is similar from all sources, it turns out that the optimal strategy is to query the data source with {maximum probability of obtaining a sample from the minority group}.
Expanding the binary-attributes algorithm for non-binary cases, the problem can be modeled as an extension of the ``{\it coupon collector's}'' problem~\cite{motwani1995randomized}, where the goal is to collect $m_i$ instances from each coupon (group) $\gee_i$.
At each iteration, the coupon collector's algorithm identifies a data view as most promising and queries it. In simple terms, a data view with a smaller query cost and a higher chance of obtaining minority groups is more promising.


For the cases where the group distributions are unknown, we model DT as a {\it multi-armed bandit} problem, where every data view is modeled as an arm. 
Every arm has an unknown distribution of different groups while pulling an arm (i.e., querying the corresponding data view) has a cost.
During various iterations, the algorithms pull the arms in an order that its expected total {\it reward} is maximized.
Arguing that the reward of obtaining a tuple from a group is proportional to how rare this group is across different data views, 
we design the reward function based on the expected cost one needs to pay in order to collect a tuple from a specific group.  
As the bandit strategy, we adopt {\it Upper Confidence Bound (UCB)} to balance exploration and exploitation. At every iteration, for every arm, UCB computes confidence intervals for the expected reward and selects the arm with the maximum upper bound of reward to be explored next.

\subsection{Data Augmentation using Foundation Models}

While data integration provides a promising approach for resolving coverage issues in a data set, its effectiveness is limited to the availability of external data sources that are rich enough to find sufficient fresh samples from minority groups. This, however, is not always possible, especially since the minority samples are rare and not easy to obtain.
Fortunately, recent advancements in Generative AI and Foundation Models have enabled synthesizing samples that are otherwise challenging to obtain from the real world.

Therefore, as an alternative approach to data integration, we turn our attention to the Foundation Models and Generative AI for resolving the lack of coverage. 
Particularly, models such as {\sc DALL.E}\footnote{\url{https://openai.com/dall-e-2}} have emerged as powerful tools for generating multi-modal data such as image, audio, and video.
 
We formalize the foundation model \fm as a black-box function with the following inputs, that once queried synthesize an output tuple.
\begin{itemize}
    \item {\bf Prompt}: A natural language description providing instructions on the details of the tuple to be generated. For instance, a prompt for image generation might be ``A realistic photo of a white cat running in a backyard.''
    \item {\bf Guide}: In cases where only a prompt is provided, the foundation model uses its imagination to generate the requested tuple. For the previous example, the prompt of a cat image, the breed, size, background, and other details are generated based on the model's imagination. Alternatively, a guide can be provided to influence the generation process. The guide is formalized as a pair $(t,m)$ where $t$ is a tuple and $m$ is a mask specifying which parts of the guide tuple should be changed. Using the cat example, $t$ can be a cat image and $m$ can specify the foreground to be regenerated.
\end{itemize}

There are multiple challenges towards effective data set augmentations using foundation models. 
First, we have to determine the minimal set of synthetic tuples that once added to the original data set, under-representation issues are resolved.
Second, the generated images should follow the underlying distribution represented in the input data set. Third, the generated tuples should have high quality and look realistic to a human evaluator. Last but not least, given the (often monetary) cost associated with the queries to the foundation model, we should ensure the cost-effectiveness of the data set repair process.

\begin{wrapfigure}{L}{0.45\textwidth}
\centering
\vspace{-3mm}
\scriptsize
    \includegraphics[width=.45\textwidth]{submissions/submission1/shahbazi/enhanced_pipeline.png}
\vspace{-3mm}
\caption{\small Architecture of \fmsystem for image data augmentation for coverage enhancement.}\label{fig:chameleon}
% \vspace{-3mm}
\end{wrapfigure}

\noindent Figure~\ref{fig:chameleon} shows the architecture of our system \fmsystem \cite{chameleon} for coverage enhancement using DALL-E image generator.
To address the first challenge, we define the combinations-selection problem, which minimizes the total number of synthetic tuples for resolving lack of coverage of minorities at the most general level. We show the problem is {\sc NP}-hard, and propose a greedy approximation algorithm for it.
To address the second and third challenges, \fmsystem follows a {\it rejection sampling} strategy.
It views each tuple in the data set $\dee$ as an iid sample from the underlying distribution $\xi$ it represents. It uses the vector representations (embeddings) space to describe the distribution. Then, given a newly generated tuple, it employs the one-class support vector machine (OCSVM) approach proposed by Scholkopf et al.~\cite{scholkopf1999support} to reject the tuple if it does not follow $\xi$.
Moreover, it models the quality evaluation as hypothesis testing and rejects the samples that have a higher chance of being labeled as ``unrealistic'' by a random human evaluator.
Finally, to minimize the number of queries to the foundation model, we provide a guide tuple (and a mask), in addition to the prompt, to the foundation model. We model the guide-selection problem as {\it contextual multi-armed bandit} and propose a solution based on the contextual UCB for it.

Before concluding this section, let us provide some experiment results to demonstrate the effectiveness of data augmentation with \fmsystem. We use FERET DB \cite{phillips1998feret} for this experiment, which comprises 1199 individual images and serves as a standardized facial image database for researchers to develop algorithms and report results. All images in FERET DB share the same dimensions, pose, and facial expression.
First, we identified the (level-1) uncovered ethnicity groups, using the threshold 80. We then used \fmsystem and resolved the lack of coverage issues.
To evaluate the effectiveness of the system, we trained a CNN model to predict the race of each image within this dataset. We then retrained the identical CNN on the repaired training data. Importantly, our test dataset for both experiments remains consistent and is derived from real images.
Table~\ref{tab:lackofcoverage} presents the improvements in precision, recall, and F1 score metrics for under-represented groups after repairing the dataset. The results indicate an enhancement in performance metrics for all under-represented groups following the repair process.

\begin{table}[t]
    \centering
    \caption{Illustrating the effect of lack of coverage repair using \fmsystem on \texttt{FERTDB}}
    \label{tab:lackofcoverage}
    \vspace{-3mm}
    \begin{tabular}{lcccccccc}
        \toprule
         & \multicolumn{4}{c}{\textbf{Classifier Performance on \texttt{FERTDB}}} & \multicolumn{4}{c}{\textbf{Classifier Performance on Repaired}} \\
        \cmidrule(lr){2-5} \cmidrule(lr){6-9}
        \textbf{Ethnicity Groups}& \#Images & Precision & Recall & F1-Score & \#Images & Precision & Recall & F1-Score \\
        \midrule
        Overall          & 756 & 0.81 & 0.75 & 0.78 & 987 & 0.70 & 0.75 & 0.72 \\ \hline
        Black            & 40  & 0.19 & 0.22 & 0.16 & 100 & 0.48 & 0.56 & 0.52 \\
        Hispanic         & 19  & 0.50 & 0.17 & 0.25 & 100 & 0.62 & 0.36 & 0.45 \\
        Middle Eastern   & 10  & 0.00 & 0.00 & 0.00 & 100 & 0.20 & 0.41 & 0.27 \\
        \bottomrule
    \end{tabular}
\end{table}

 %%%%%%%%%%%%%%%%%%%%%%%%%%%%%%%% RELIABILITY  %%%%%%%%%%%%%%%%%%%%%%%%%%%%%%%%
\section{Generating Reliability Warnings}\label{sec:reliability}
% up to 2.5 pages
Interpretability is a necessity for data scientists who develop predictive models for critical decision-making.
In such settings, it is important to provide additional means to support the following question:
{\it is an individual prediction of the model reliable for decision-making?} Our goal is to use the lack of representation to help decision-makers find insights about this critical question.
To further motivate this, let us use the following example:

\vspace{1mm}
\begin{example}\label{ex-0}
{\bf(Part1):} Consider a judge who needs to decide whether to accept or deny a bail request. Using data-driven predictive models is prevalent in such cases for predicting recidivism~\cite{dressel2018accuracy}.
Indeed, such models can be beneficial to help the judge make wise decisions.
Suppose the model predicts the queried individual as high risk (or low risk).
The judge is aware and concerned about the critics surrounding such models.
A major question the judge faces is whether or not they should rely on the prediction outcome to take action for this case.
Furthermore, if, for instance, they decide to ignore the outcome and hence they need to provide a statement supporting their action, what evidence can they provide? 
\end{example}

In line with the recent trend on data-centric AI~\cite{ng2021mlops}, we design {novel approaches}, {complimentary} to the existing work on trustworthy AI~\cite{wing2021trustworthy,kentour2021analysis,liu2021trustworthy,singh2021trustworthy}, to address the aforementioned trust question through the lens of {\it data}.
In particular, unlike existing works that generate trust information from a {\it given \underline{model}}, we associate {\it \underline{data sets} with proper measurements} that specify their {\it the scope of use for predicting future cases}.
We note that a predictive model provides only probabilistic guarantees on the \underline{average} loss over the distribution represented by the data set used for training it.
As a result, these predictions may not be distribution generalizable~\cite{kulynych2022you}.
Consequently, if the query point is {\it not represented} by the data, the guarantees may not hold, hence one cannot rely on the prediction outcome.
Besides, an essential requirement for a learning algorithm is that its training data $\dee$ should represent the underlying distribution $\dist$.
Even if so, the trained model $h$ only provides a probabilistic guarantee on the {expected} loss on random samples from $\dist$.  
A model that performs well on {\it majority} of samples drawn from $\dist$ will have a high performance on average. Still, as we observed in Figure~\ref{fig:mlfails},
its performance for {\it minorities} and points that are not represented is questionable. Let us consider the following toy example:

\begin{figure*}[!b] 
    \begin{minipage}[t]{0.32\linewidth}
        	\centering
        	\includegraphics[width=\textwidth]{submissions/submission1/shahbazi/example_1.png} 
        	\vspace{-9mm}\caption{\small Data set $\dee$ generated using a Gaussian distribution; $x_1$ and $x_2$ are positively correlated}
            \label{fig:ex1:1}
    \end{minipage}
    \hfill
    \begin{minipage}[t]{0.32\linewidth}
        \centering
        	\includegraphics[width =\textwidth]{submissions/submission1/shahbazi/example_2.png} 
        	\vspace{-9mm}\caption{\small The decision boundary of learned model $h$ and query points $\qu^1$ to $\qu^4$}
            \label{fig:ex1:2}
    \end{minipage}
    \hfill
    \begin{minipage}[t]{0.32\linewidth}
        	\centering
        	\includegraphics[width =\textwidth]{submissions/submission1/shahbazi/example_3.png}
        	\vspace{-9mm}\caption{\small Ground-truth boundary, overlaid on the model decision boundary and query points}
            \label{fig:ex1:3}
    \end{minipage}
    \vspace{-5mm}
\end{figure*} 

\vspace{1mm}
\begin{example}\label{ex-1}
Consider a binary classification task where the input space is $\ex=\langle x_1, x_2\rangle$ and the output space is the binary label $y$ with values $\{-1$ (red) $,+1$ (blue)$\}$.
Suppose the underlying data distribution $\dist$ follows a 2D Gaussian, where $x_1$ and $x_2$ 
are positively correlated as shown in Figure~\ref{fig:ex1:1}.
The figure shows the data set $\dee$ drawn independently from the distribution $\dist$, along with their labels as their colors.
Using $\dee$, the prediction model $h$ is constructed as shown in Figure~\ref{fig:ex1:2}. 
The decision boundary is specified in the picture; while any point above the line is predicted as +1, a query point below it is labeled as -1.
The classifier has been evaluated using a test set that is an iid sample set drawn from the underlying data set $\dist$. The accuracy on the test set is high (above 90\%), and hence, the model gets deployed.
We cherry-picked four query points, $\qu^1$ to $\qu^4$, that are also included in Figure~\ref{fig:ex1:2}. Using $h$ for prediction, $h(\qu^1)=-1$, $h(\qu^2)=+1$,  $h(\qu^3)=+1$, and $h(\qu^4)=-1$.
Figure~\ref{fig:ex1:3} adds the ground-truth boundary to the search space, revealing the true label of the query points: every point inside the red circle has the true label $-1$ while any point outside of it is $+1$.
Looking at the figure, $y^1=+1$ while the model predicted it as $h(\qu^1)=-1$.  \hfill$\square$
\end{example}
\vspace{2mm}

Let us take a closer look at the four query points in this example and their placement with regard to the tuples in $\dee$ used for training $h$. 
$\qu^2$ belongs to a {\it dense region} with many training tuples in $\dee$ surrounding it. Besides, all of the tuples in its vicinity have the same label $y=+1$. As a result, one can expect that the model's outcome $h(\qu^2)=+1$ should be a reliable prediction.
Similar to $\qu^2$, $\qu^4$ also belongs to a dense region in $\dee$; however, $\qu^4$ belongs to an {\it uncertain region}, where some of the tuples in its vicinity have a label $y=+1$, and some others have the label $y=-1$. Considering the uncertainty in the vicinity of $\qu^4$, one cannot confidently rely on the outcome of the model $h$. 
On the other hand, the neighbors of $\qu^1$ (resp. $\qu^3$) are not uncertain, all having the label $y=-1$ (resp. $y=+1$).
However, the query points $\qu^1$ and $\qu^3$ are not well represented by $\dee$. In other words, $\qu^1$ and $\qu^3$ are unlikely to be generated according to the underlying distribution $\dist$, represented by $\dee$. As a result, following the no-free-lunch theorem~\cite{kakade2003sample}, one cannot expect the outcome of model $h$ to be reliable for these points.
Looking at the ground-truth boundary in Figure~\ref{fig:ex1:3}, $h$ luckily predicted the outcome for $\qu^3$ correctly, but it was not fortunate to predict the $y^1$ correctly.
Nevertheless, 
since the model is not reliably trained for these points, 
its outcome for these query points is not trustworthy.

From Example~\ref{ex-1}, we observe that the outcome of a model $h$, trained using a data set $\dee$ is not reliable for a query point $\qu$, if:
\begin{itemize}
    \item {\bf Lack of representation:} $\qu$ is not well-represented by $\dee$.
    In such cases, the model has not seen ``enough'' samples similar to $\qu$ to reliably learn and predict the outcome of $\qu$.
    \item {\bf Lack of certainty:} $\qu$ belongs to an uncertain region, where different tuples of $\dee$ in the vicinity of $\qu$ have different target values. $\qu$ belongs to a high-fluctuating area, where tuples in the vicinity of $\qu$ have a wide range of values.
\end{itemize} \vspace{2mm}

\noindent
Based on these two observations, we propose Representation-and-Uncertainty ({\bf RU}) measures.
To identify if a query suffers from uncertainty or lack of representation, one could use a deterministic approach using a fixed threshold. Then if the number of similar samples to (resp. label fluctuation in vicinity of) $\qu$ is larger than the threshold it is considered as unrepresented (resp. uncertain).
This approach, however, would be misleading since two numbers close to the threshold could be treated very differently. Also, all points on each side of the threshold would be considered equally represented (resp., certain). Instead, we consider {\it a randomized approach}, widely popular in the literature, including~\cite{dwork2012fairness}.
That is, instead of using fixed thresholds, a Bernoulli variable (a biased coin) is used that 
assigns $\qu$ as unrepresented (resp., uncertain) based on the number of samples similar to it (resp., its neighborhood uncertainty).
Given a query point $\qu$, let $\pe_o$ be the probability indicating if $\qu$ is not represented and let $\pe_u$ be the probability indicating if $\qu$ belongs to an uncertain region. 
We represent the probability of the Bernoulli variables for lack of representation or uncertainty components as $\pe_o$ and $\pe_u$, respectively. Note that the two Bernoulli variables $\pe_o$ and $\pe_u$ are independent from each other. That simply follows the argument that after specifying the number of similar samples to $\qu$ whether or not it should be considered as unrepresented does not depend on the uncertainty in the neighborhood of $\qu$.

\begin{definition}[\sru]\label{def:sdt}
The \sru is a probabilistic measure that considers the outcome of a model for a query point $\qu$ untrustworthy if $\qu$ is not represented by $\dee$ {\it and} it belongs to an uncertain region.
Formally, the \sru measure is:
\begin{align} 
    \nonumber
    SRU(\qu) &= \pe\big((\qu \mbox{ is outlier}) \wedge (\qu \mbox{ belongs to uncertain region})\big) 
\end{align}
Since $\pe_o$ and $\pe_u$ are independent:

\vspace{-13mm}
\begin{align} \label{eq:strong}
    SRU(\qu) &= \pe_o(\qu) \times \pe_u(\qu)
\end{align}
\end{definition}

\sru raises the warning signal only when the query point fails on {\it both} conditions of being represented by $\dee$ and not belonging to an uncertain region. 
For instance, in Example~\ref{ex-1} none of the query points fail both on representation and on uncertainty; hence neither has a high \sru score.
On the other hand, 
a high \sru score for a query point $\qu$ {\it provides a strong warning signal} that one should perhaps reject the model outcome and not consider it for decision-making.

\sru is a strong signal that raises warnings only for the fearfully concerning cases that fail both on representation and uncertainty.
However, as observed in Example~\ref{ex-1} a query points failing {\it at least} one of these conditions may also not be reliable, at least for critical decision making.
We define the \wru measure to raise a warning for such cases.

\begin{definition}[\wru]\label{def:wdt}
The \wru measure is a probabilistic measure that considers the outcome of a model for a query point $\qu$ untrustworthy if $\qu$ is not represented by $\dee$ {\bf or} it belongs to an uncertain region.
Formally, the \wru is computed as:
\begin{align} \label{eq:weak}
    WRU(\qu) = \pe\big((\qu \mbox{ is outlier}) \vee (\qu \mbox{ belongs to uncertain region})\big) 
    = \pe_o(\qu) + \pe_u(\qu) - \pe_o(\qu) \times \pe_u(\qu)
\end{align}
\end{definition}

Proposing quantitative probabilistic outcomes, \ru measures are interpretable for the users, since beyond the scores, the uncertainty and lack of representation components provide an explanation to justify them. 
Please refer to \cite{techrep} for more details on how to efficiently and effectively compute the representation ($\pe_o$) and uncertainty ($\pe_u$) probabilities, using only $\dee$.
In Example~\ref{ex-0}, let us see how the \ru measures can be helpful.

\noindent{\bf Example 1. (part 2):}
{\it RU measures \underline{raise warning} when
the fitness of the data set used for drawing a prediction is questionable, helping the judge to be cautious when taking action.
Besides, these measures provide \underline{quantitative evidence} to support the judge's action when they decide to ignore a prediction outcome that is not trustworthy.
The judge, for example, can argue to ignore a model outcome for a specific case, based on the insight that 
the model has been built using a
data set that fails to represent the given case.}
\hfill$\square$

Finally, let us demonstrate the efficacy of \ru measures through a series of experiments. Since the \ru measures are {\it data-centric},
those are applicable for both classification and regression tasks, irrespective of the model used.
We use {\it Adult} dataset~\cite{adult} for classification and {\it House Sales in King County} dataset for the validation of regression tasks. From each dataset, we uniformly sample two sets from the underlying distribution. The first set serves as the training set to compute the \ru values, and the second one is used as the test set from which the queries are drawn. We validate our proposal by providing the correlation between the \ru values and the performance of an ML model's prediction on the same data. 

We start by computing the \ru values for all the query points in the test set. Next, we bucketize the query points based on their \ru values in equi-width buckets of width 0.1. We repeat this for both \sru and \wru measures. Next, we train a model on the training data set and predict the target variable for the points in each range of \ru measure. The validation results for the classification task on the {\it Adult} dataset are presented in Figures \ref{fig:exp-adult-sdt} and \ref{fig:exp-adult-wdt}. Each figure corresponds to the accuracy/error measures of the classifier over each bucket of \ru values for \sru and \wru. As the \ru values increase, the accuracy of the model drops while the FPR rises, and therefore, the model fails to capture the ground truth for the points that fall into untrustworthy regions in the data set. By repeating the aforementioned steps for the regression task on the {\it House Sales in King County} dataset, we observe similar results presented in Figures \ref{fig:exp-hs-sdt} and \ref{fig:exp-hs-wdt}. 
As the \ru value increases, the RSS of the regression model follows the same trend denoting that the model fails to perform for tuples with a high \ru value.

\begin{figure}[!tb]
    \begin{minipage}[t]{0.24\linewidth}
        \centering
        \includegraphics[width=\textwidth]{submissions/submission1/shahbazi/sdt_adult.pdf}
        \vspace{-6mm}\caption{\small{\it Adult}, efficacy of \sru  on classification}
        \label{fig:exp-adult-sdt}
    \end{minipage}\hfill
    \begin{minipage}[t]{0.24\linewidth}
        \centering
        \includegraphics[width=\textwidth]{submissions/submission1/shahbazi/wdt_adult.pdf}
        \vspace{-6mm}\caption{\small{\it Adult}, efficacy of \wru  on classification}
        \label{fig:exp-adult-wdt}
    \end{minipage}\hfill
    \begin{minipage}[t]{0.24\linewidth}
        \centering
        \includegraphics[width=\textwidth]{submissions/submission1/shahbazi/sdt_regression_house.pdf}
        \vspace{-6mm}\caption{\small{\it House Sales in King County}, efficacy of \sru on regression}
        \label{fig:exp-hs-sdt}
    \end{minipage}\hfill
    \begin{minipage}[t]{0.24\linewidth}
        \centering
        \includegraphics[width=\textwidth]{submissions/submission1/shahbazi/wdt_regression_house.pdf}
        \vspace{-6mm}\caption{\small{\it House Sales in King County}, efficacy \wru on regression}
        \label{fig:exp-hs-wdt}
    \end{minipage}
\vspace{-5mm}
\end{figure}
 %%%%%%%%%%%%%%%%%%%%%%%%%%%%%%%% RELATED WORK  %%%%%%%%%%%%%%%%%%%%%%%%%%%%%%%%
\section{Related Work}\label{related} 

Bias in data has been looked at for a long time in statistical community~\cite{neyman1936contributions} but social data presents different challenges~\cite{olteanu2019social,fairmlbook,barocas2016big,jk2019bias,drosou2017diversity}.
The diversity and representativeness of data have been widely studied~\cite{drosou2017diversity}, in fields such as social science~\cite{berrey2015enigma, dobbin2016diversity,simpson1949measurement}, political science~\cite{surowiecki2005wisdom}, and information retrieval~\cite{agrawal2009diversifying}. 
Tracing back machine bias to its source, there have been major efforts to identify different types~\cite{mehrabi2021survey, olteanu2019social,friedman1996bias} and sources~\cite{torralba2011unbiased,crawford2013hidden,diakopoulos2015algorithmic} of biases in data. Efforts to satisfy {\it responsible data} requirements~\cite{nargesian2022responsible} extend to various stages of the data analysis pipeline, including data annotation~\cite{li2020towards,lazier2023fairness}, data cleaning and repair~\cite{SalimiRHS19,tae2019data,salimi2020database}, data imputation~\cite{martinez2019fairness}, entity resolution~\cite{shahbazi2023through,fanourakis2023fairer}, data integration~\cite{nargesian2022responsible,nargesian2021tailoring}, etc. 

\paragraph{Data Coverage:}The notion of data coverage has received extensive attention from different angles. Detecting lack of coverage has been studied for datasets with discrete~\cite{asudeh2019assessing} and continuous~\cite{asudeh2021coverage} attributes populated in single or multiple \cite{lin2020identifying} relations.
To resolve insufficient coverage, \cite{accinelli2020coverage, accinelli2021impact,shetiya2022fairness}
consider resolving representation bias in preprocessing pipelines by rewriting queries into the closest operation so that certain subgroups are sufficiently represented in the downstream tasks. Alternatively, ~\cite{asudeh2019assessing,tae2021slice} propose a data collection strategy to acquire as little additional data as possible (to minimize the associated costs) to meet the representation constraints. ~\cite{sharma2020data,iosifidis2018dealing,celis2020data} opt for a data augmentation approach by adding partially altered duplicates of already existing tuples or generating new synthetic entries from existing data. Consequently, the new data set has an equal number of elements for different groups, resulting in potentially resolving the under-representation issues. Finally,  \cite{nargesian2021tailoring} utilizes data integration techniques to consolidate data from different sources into a single dataset to resolve representation bias.
Related works also include ~\cite{chung2019slice,sagadeeva2021sliceline,tae2021slice} that seek to understand if the overall performance of the model fails to reflect and performs poorly on certain slices in the data.
As alternative approaches to measure representation bias, the notion of representation rate~\cite{celis2020data} (a.k.a. equal base rate~\cite{kleinberg2016inherent}) is introduced which compared with coverage, it is more restrictive as it requires almost equal ratios from different groups.
Please refer to \cite{shahbazi2023representation} for a comprehensive survey about representation bias in data. 

\paragraph{ML Reliability:} Model-centric works for uncertainty quantification such as 
probabilistic classifiers~\cite{zadrozny2001obtaining,zadrozny2002transforming,platt1999probabilistic,niculescu2005predicting},
prediction intervals (PIs) \cite{chatfield93predictionintervals,pearce2018high,khosravi2010lower} and conformal predictions (CP)~\cite{angelopoulos2021gentle,shafer2008tutorial} that are used for measuring prediction uncertainty, are built
by maximizing the {\it expected performance} on {\it random} sample from the underlying distribution.
As a result, while providing accurate estimations for the dense regions of data (e.g. majority groups), their estimation accuracy is questionable for the poorly represented regions.
In particular, \cite{angelopoulos2021gentle} recognizes the lack of guarantees in the performance of CP for such regions.
Besides, the bulk of work on trustworthy AI provides information that {\it supports} the outcome of an ML model. For example, existing work on explainable AI, including~\cite{harradon2018causal,ribeiro2016should,gunning2019darpa}, aims to find simple explanations and rules that justify the outcome of a model.
Conversely, we aim to {\it raise warning signals} when the outcome of a model is {\it not} trustworthy. That is, to provide reasons that {\it cast doubt} on the reliability of the model outcome {for a given query point}.

 %%%%%%%%%%%%%%%%%%%%%%%%%%%%%%%% FUTURE  %%%%%%%%%%%%%%%%%%%%%%%%%%%%%%%%
% \vspace{-3mm}
\section{Final Remarks}\label{sec:conclusion}
As Data-centric AI and Responsible AI emerge as focal points in data science research, the development of Data-centric methodologies for ensuring Responsible and Trustworthy AI attracts increasing attention.
While there is some excellent work on responsible data management to achieve this goal, there remain many challenges yet to be addressed.

In this paper, we focused on a crucial aspect of responsible data -- detecting and addressing the under-representation of minorities within a data set.
We formally defined the notion of data coverage and discussed various techniques for (a) identifying lack of representation issues across different data modalities, (b) ensuring proper representation of minorities in data, and (c) limiting the scope-of-use of data sets based on their representation issues by generating proper ({\sc RU}) warning signals.
Even though the research on detecting lack of coverage issues is relatively mature, resolution techniques are still understudied.
Considering the recent advancements in Generative AI, utilizing Foundation Models and Large Language Models, and studying their limitations, for data augmentation to improve the representation of minorities at the data level seems interesting to further explore.

 %%%%%%%%%%%%%%%%%%%%%%%%%%%%%%%% BIB  %%%%%%%%%%%%%%%%%%%%%%%%%%%%%%%%
\bibliographystyle{unsrt}
\small
% \bibliography{ref}
\begin{thebibliography}{10}

\bibitem{asudeh2019assessing}
A.~Asudeh, Z.~Jin, and H.~Jagadish.
\newblock Assessing and remedying coverage for a given dataset.
\newblock In {\em ICDE}, pages 554--565. IEEE, 2019.

\bibitem{shahbazi2023representation}
N.~Shahbazi, Y.~Lin, A.~Asudeh, and H.~Jagadish.
\newblock Representation bias in data: A survey on identification and resolution techniques.
\newblock {\em ACM Computing Surveys}, 2023.

\bibitem{asudeh2021coverage}
A.~Asudeh, N.~Shahbazi, Z.~Jin, and H.~V. Jagadish.
\newblock Identifying insufficient data coverage for ordinal continuous-valued attributes.
\newblock In {\em SIGMOD}. ACM, 2021.

\bibitem{mousavi2024data}
M.~Mousavi, N.~Shahbazi, and A.~Asudeh.
\newblock Data coverage for detecting representation bias in image datasets: {A} crowdsourcing approach.
\newblock In {\em {EDBT}}, pages 47--60, 2024.

\bibitem{nargesian2021tailoring}
F.~Nargesian, A.~Asudeh, and H.~Jagadish.
\newblock Tailoring data source distributions for fairness-aware data integration.
\newblock {\em Proceedings of the VLDB Endowment}, 14(11):2519--2532, 2021.

\bibitem{nargesian2022responsible}
F.~Nargesian, A.~Asudeh, and H.~V. Jagadish.
\newblock Responsible data integration: Next-generation challenges.
\newblock {\em SIGMOD}, 2022.

\bibitem{sharma2020data}
S.~Sharma, Y.~Zhang, J.~M. R{\'\i}os~Aliaga, D.~Bouneffouf, V.~Muthusamy, and K.~R. Varshney.
\newblock Data augmentation for discrimination prevention and bias disambiguation.
\newblock In {\em AIES}, pages 358--364, 2020.

\bibitem{DBLP:journals/jair/ChawlaBHK02}
N.~V. Chawla, K.~W. Bowyer, L.~O. Hall, and W.~P. Kegelmeyer.
\newblock {SMOTE:} synthetic minority over-sampling technique.
\newblock {\em J. Artif. Intell. Res.}, 16:321--357, 2002.

\bibitem{iosifidis2018dealing}
V.~Iosifidis and E.~Ntoutsi.
\newblock Dealing with bias via data augmentation in supervised learning scenarios.
\newblock {\em Jo Bates Paul D. Clough Robert J{\"a}schke}, 24, 2018.

\bibitem{celis2020data}
L.~E. Celis, V.~Keswani, and N.~Vishnoi.
\newblock Data preprocessing to mitigate bias: A maximum entropy based approach.
\newblock In {\em ICML}, pages 1349--1359. PMLR, 2020.

\bibitem{asudeh2022towards}
A.~Asudeh and F.~Nargesian.
\newblock Towards distribution-aware query answering in data markets.
\newblock {\em Proceedings of the VLDB Endowment}, 15(11):3137--3144, 2022.

\bibitem{motwani1995randomized}
R.~Motwani and P.~Raghavan.
\newblock {\em Randomized algorithms}.
\newblock Cambridge university press, 1995.

\bibitem{chameleon}
M.~Erfanian, H.~V. Jagadish, and A.~Asudeh.
\newblock Chameleon: Foundation models for fairness-aware multi-modal data augmentation to enhance coverage of minorities.
\newblock {\em arXiv preprint arXiv:2402.01071}, 2024.

\bibitem{scholkopf1999support}
B.~Sch{\"o}lkopf, R.~C. Williamson, A.~Smola, J.~Shawe-Taylor, and J.~Platt.
\newblock Support vector method for novelty detection.
\newblock {\em NeurIPS}, 12, 1999.

\bibitem{phillips1998feret}
P.~J. Phillips, H.~Wechsler, J.~Huang, and P.~J. Rauss.
\newblock The feret database and evaluation procedure for face-recognition algorithms.
\newblock {\em Image and vision computing}, 16(5):295--306, 1998.

\bibitem{dressel2018accuracy}
J.~Dressel and H.~Farid.
\newblock The accuracy, fairness, and limits of predicting recidivism.
\newblock {\em Science advances}, 4(1):eaao5580, 2018.

\bibitem{ng2021mlops}
A.~Ng.
\newblock Mlops: From model-centric to data-centric {AI}.
\newblock 2021.

\bibitem{wing2021trustworthy}
J.~M. Wing.
\newblock Trustworthy {AI}.
\newblock {\em CACM}, 64(10):64--71, 2021.

\bibitem{kentour2021analysis}
M.~Kentour and J.~Lu.
\newblock Analysis of trustworthiness in machine learning and deep learning.
\newblock {\em InfoComp}, 2021.

\bibitem{liu2021trustworthy}
H.~Liu, Y.~Wang, W.~Fan, X.~Liu, Y.~Li, S.~Jain, A.~K. Jain, and J.~Tang.
\newblock Trustworthy {AI}: A computational perspective.
\newblock {\em arXiv preprint arXiv:2107.06641}, 2021.

\bibitem{singh2021trustworthy}
R.~Singh, M.~Vatsa, and N.~Ratha.
\newblock Trustworthy {AI}.
\newblock In {\em 8th ACM IKDD CODS and 26th COMAD}, pages 449--453. 2021.

\bibitem{kulynych2022you}
B.~Kulynych, Y.-Y. Yang, Y.~Yu, J.~B{\l}asiok, and P.~Nakkiran.
\newblock What you see is what you get: Distributional generalization for algorithm design in deep learning.
\newblock {\em arXiv preprint arXiv:2204.03230}, 2022.

\bibitem{kakade2003sample}
S.~M. Kakade.
\newblock {\em On the sample complexity of reinforcement learning}.
\newblock University of London, University College London (United Kingdom), 2003.

\bibitem{dwork2012fairness}
C.~Dwork, M.~Hardt, T.~Pitassi, O.~Reingold, and R.~Zemel.
\newblock Fairness through awareness.
\newblock In {\em ITCS}, pages 214--226, 2012.

\bibitem{techrep}
N.~Shahbazi and A.~Asudeh.
\newblock Data-centric reliability evaluation of individual predictions.
\newblock {\em CoRR, abs/2204.07682}, 2022.

\bibitem{adult}
M.~Lichman.
\newblock Adult income dataset, {UCI} machine learning repository.
\newblock \url{https://archive.ics.uci.edu/ml/datasets/adult}, 2013.

\bibitem{neyman1936contributions}
J.~Neyman and E.~S. Pearson.
\newblock Contributions to the theory of testing statistical hypotheses.
\newblock {\em Statistical Research Memoirs}, 1936.

\bibitem{olteanu2019social}
A.~Olteanu, C.~Castillo, F.~Diaz, and E.~Kiciman.
\newblock Social data: Biases, methodological pitfalls, and ethical boundaries.
\newblock {\em Frontiers in Big Data}, 2:13, 2019.

\bibitem{fairmlbook}
S.~Barocas, M.~Hardt, and A.~Narayanan.
\newblock Fairness and machine learning: Limitations and opportunities.
\newblock \url{fairmlbook.org}, 2019.

\bibitem{barocas2016big}
S.~Barocas and A.~D. Selbst.
\newblock Big data's disparate impact.
\newblock {\em Calif. L. Rev.}, 104:671, 2016.

\bibitem{jk2019bias}
J.~Kleinberg.
\newblock Fairness, rankings, and behavioral biases.
\newblock FAT*, 2019.

\bibitem{drosou2017diversity}
M.~Drosou, H.~Jagadish, E.~Pitoura, and J.~Stoyanovich.
\newblock Diversity in big data: A review.
\newblock {\em Big data}, 5(2):73--84, 2017.

\bibitem{berrey2015enigma}
E.~Berrey.
\newblock {\em The enigma of diversity: The language of race and the limits of racial justice}.
\newblock University of Chicago Press, 2015.

\bibitem{dobbin2016diversity}
F.~Dobbin and A.~Kalev.
\newblock Why diversity programs fail and what works better.
\newblock {\em Harvard Business Review}, 94(7-8):52--60, 2016.

\bibitem{simpson1949measurement}
E.~H. Simpson.
\newblock Measurement of diversity.
\newblock {\em Nature}, 163(4148), 1949.

\bibitem{surowiecki2005wisdom}
J.~Surowiecki.
\newblock {\em The wisdom of crowds}.
\newblock Anchor, 2005.

\bibitem{agrawal2009diversifying}
R.~Agrawal, S.~Gollapudi, A.~Halverson, and S.~Ieong.
\newblock Diversifying search results.
\newblock In {\em WSDM}, pages 5--14. ACM, 2009.

\bibitem{mehrabi2021survey}
N.~Mehrabi, F.~Morstatter, N.~Saxena, K.~Lerman, and A.~Galstyan.
\newblock A survey on bias and fairness in machine learning.
\newblock {\em ACM Computing Surveys (CSUR)}, 54(6):1--35, 2021.

\bibitem{friedman1996bias}
B.~Friedman and H.~Nissenbaum.
\newblock Bias in computer systems.
\newblock {\em TOIS}, 14(3):330--347, 1996.

\bibitem{torralba2011unbiased}
A.~Torralba and A.~A. Efros.
\newblock Unbiased look at dataset bias.
\newblock In {\em CVPR 2011}, pages 1521--1528. IEEE, 2011.

\bibitem{crawford2013hidden}
K.~Crawford.
\newblock The hidden biases in big data.
\newblock {\em Harvard business review}, 1(4), 2013.

\bibitem{diakopoulos2015algorithmic}
N.~Diakopoulos.
\newblock Algorithmic accountability: Journalistic investigation of computational power structures.
\newblock {\em Digital journalism}, 3(3):398--415, 2015.

\bibitem{li2020towards}
Y.~Li, H.~Sun, and W.~H. Wang.
\newblock Towards fair truth discovery from biased crowdsourced answers.
\newblock In {\em SIGKDD}, pages 599--607, 2020.

\bibitem{lazier2023fairness}
S.~Lazier, S.~Thirumuruganathan, and H.~Anahideh.
\newblock Fairness and bias in truth discovery algorithms: An experimental analysis.
\newblock {\em arXiv preprint arXiv:2304.12573}, 2023.

\bibitem{SalimiRHS19}
B.~Salimi, L.~Rodriguez, B.~Howe, and D.~Suciu.
\newblock Interventional fairness: Causal database repair for algorithmic fairness.
\newblock In {\em {SIGMOD}}, pages 793--810. {ACM}, 2019.

\bibitem{tae2019data}
K.~H. Tae, Y.~Roh, Y.~H. Oh, H.~Kim, and S.~E. Whang.
\newblock Data cleaning for accurate, fair, and robust models: A big data-{AI} integration approach.
\newblock In {\em DEEM workshop}, pages 1--4, 2019.

\bibitem{salimi2020database}
B.~Salimi, B.~Howe, and D.~Suciu.
\newblock Database repair meets algorithmic fairness.
\newblock {\em ACM SIGMOD Record}, 49(1):34--41, 2020.

\bibitem{martinez2019fairness}
F.~Mart{\'\i}nez-Plumed, C.~Ferri, D.~Nieves, and J.~Hern{\'a}ndez-Orallo.
\newblock Fairness and missing values.
\newblock {\em arXiv preprint arXiv:1905.12728}, 2019.

\bibitem{shahbazi2023through}
N.~Shahbazi, N.~Danevski, F.~Nargesian, A.~Asudeh, and D.~Srivastava.
\newblock Through the fairness lens: Experimental analysis and evaluation of entity matching.
\newblock {\em Proceedings of the VLDB Endowment}, 16(11):3279--3292, 2023.

\bibitem{fanourakis2023fairer}
N.~Fanourakis, C.~Kontousias, V.~Efthymiou, V.~Christophides, and D.~Plexousakis.
\newblock Fairer demo: Fairness-aware and explainable entity resolution.
\newblock 2023.

\bibitem{lin2020identifying}
Y.~Lin, Y.~Guan, A.~Asudeh, and H.~Jagadish.
\newblock Identifying insufficient data coverage in databases with multiple relations.
\newblock {\em Proceedings of the VLDB Endowment}, 13(12):2229--2242, 2020.

\bibitem{accinelli2020coverage}
C.~Accinelli, S.~Minisi, and B.~Catania.
\newblock Coverage-based rewriting for data preparation.
\newblock In {\em EDBT Workshops}, 2020.

\bibitem{accinelli2021impact}
C.~Accinelli, B.~Catania, G.~Guerrini, and S.~Minisi.
\newblock The impact of rewriting on coverage constraint satisfaction.
\newblock In {\em EDBT Workshops}, 2021.

\bibitem{shetiya2022fairness}
S.~Shetiya, I.~P. Swift, A.~Asudeh, and G.~Das.
\newblock Fairness-aware range queries for selecting unbiased data.
\newblock In {\em ICDE}. IEEE, 2022.

\bibitem{tae2021slice}
K.~H. Tae and S.~E. Whang.
\newblock Slice tuner: A selective data acquisition framework for accurate and fair machine learning models.
\newblock In {\em SIGMOD}, pages 1771--1783, 2021.

\bibitem{chung2019slice}
Y.~Chung, T.~Kraska, N.~Polyzotis, K.~H. Tae, and S.~E. Whang.
\newblock Slice finder: Automated data slicing for model validation.
\newblock In {\em ICDE}, pages 1550--1553. IEEE, 2019.

\bibitem{sagadeeva2021sliceline}
S.~Sagadeeva and M.~Boehm.
\newblock Sliceline: Fast, linear-algebra-based slice finding for ml model debugging.
\newblock In {\em SIGMOD}, pages 2290--2299, 2021.

\bibitem{kleinberg2016inherent}
J.~Kleinberg, S.~Mullainathan, and M.~Raghavan.
\newblock Inherent trade-offs in the fair determination of risk scores.
\newblock {\em arXiv preprint arXiv:1609.05807}, 2016.

\bibitem{zadrozny2001obtaining}
B.~Zadrozny and C.~Elkan.
\newblock Obtaining calibrated probability estimates from decision trees and naive bayesian classifiers.
\newblock In {\em ICML}, volume~1, pages 609--616. Citeseer, 2001.

\bibitem{zadrozny2002transforming}
B.~Zadrozny and C.~Elkan.
\newblock Transforming classifier scores into accurate multiclass probability estimates.
\newblock In {\em SIGKDD}, pages 694--699, 2002.

\bibitem{platt1999probabilistic}
J.~Platt et~al.
\newblock Probabilistic outputs for support vector machines and comparisons to regularized likelihood methods.
\newblock {\em Advances in large margin classifiers}, 10(3):61--74, 1999.

\bibitem{niculescu2005predicting}
A.~Niculescu-Mizil and R.~Caruana.
\newblock Predicting good probabilities with supervised learning.
\newblock In {\em Proceedings of the 22nd international conference on Machine learning}, pages 625--632, 2005.

\bibitem{chatfield93predictionintervals}
C.~Chatfield.
\newblock Prediction intervals.
\newblock {\em Journal of Business and Economic Statistics}, 11:121--135, 1993.

\bibitem{pearce2018high}
T.~Pearce, A.~Brintrup, M.~Zaki, and A.~Neely.
\newblock High-quality prediction intervals for deep learning: A distribution-free, ensembled approach.
\newblock In {\em International conference on machine learning}, pages 4075--4084. PMLR, 2018.

\bibitem{khosravi2010lower}
A.~Khosravi, S.~Nahavandi, D.~Creighton, and A.~F. Atiya.
\newblock Lower upper bound estimation method for construction of neural network-based prediction intervals.
\newblock {\em IEEE transactions on neural networks}, 22(3):337--346, 2010.

\bibitem{angelopoulos2021gentle}
A.~N. Angelopoulos and S.~Bates.
\newblock A gentle introduction to conformal prediction and distribution-free uncertainty quantification.
\newblock {\em arXiv preprint arXiv:2107.07511}, 2021.

\bibitem{shafer2008tutorial}
G.~Shafer and V.~Vovk.
\newblock A tutorial on conformal prediction.
\newblock {\em Journal of Machine Learning Research}, 9(3), 2008.

\bibitem{harradon2018causal}
M.~Harradon, J.~Druce, and B.~Ruttenberg.
\newblock Causal learning and explanation of deep neural networks via autoencoded activations.
\newblock {\em arXiv preprint arXiv:1802.00541}, 2018.

\bibitem{ribeiro2016should}
M.~T. Ribeiro, S.~Singh, and C.~Guestrin.
\newblock " why should i trust you?" explaining the predictions of any classifier.
\newblock In {\em SIGKDD}, pages 1135--1144, 2016.

\bibitem{gunning2019darpa}
D.~Gunning and D.~Aha.
\newblock Darpa’s explainable artificial intelligence ({XAI}) program.
\newblock {\em AI Magazine}, 40(2):44--58, 2019.

\end{thebibliography}

\end{document}

\end{article}

\begin{article}
{Validating Data and Models in Continuous ML Pipelines}
{Mike Dreves, Gene Huang, Zhuo Peng, Neoklis Polyzotis, Evan Rosen and Paul Suganthan G. C.}
% link to instruction: https://tc.computer.org/tcde/tcde-bulletin-author-instructions/
% \documentclass[11pt,dvipdfm]{article}
\documentclass[11pt]{article}
\usepackage{tabularx}
\usepackage{ragged2e}  % for '\RaggedRight' macro (allows hyphenation)
\usepackage{booktabs}  % for \toprule, \midrule, and \bottomrule macros
\usepackage{textcomp}
\usepackage{amsfonts,amsmath}
\usepackage{deauthor,times}
\usepackage{graphicx} % 
\usepackage{hyperref}
\usepackage{comment}
\graphicspath{{asudeh/}}
\usepackage{soul}
\usepackage{subcaption}
\usepackage{ulem}
\usepackage{wrapfig}
\usepackage{color}
\usepackage{xspace}
\newtheorem{problem}{Problem}

%\DeclareMathOperator*{\argmax}{arg\,max}

%remove the following commands/package b4 submission
\newcommand{\hide}[1]{}
\newcommand{\eat}[1]{}
\newcommand{\resolved}[1]{\hide{#1}}
\newcommand{\abol}[1]{\textcolor{red}{Abol: #1}}
\newcommand{\mahdi}[1]{\textcolor{red}{Mahdi: #1}}
\newcommand{\nima}[1]{\textcolor{red}{Nima: #1}}

\newcommand{\dee}{\mathcal{D}}
\newcommand{\Gee}{\mathcal{G}}
\newcommand{\gee}{\mathbf{g}}
\newcommand{\ee}{\mathbf{e}}
\newcommand{\es}{\mathcal{S}}
\newcommand{\el}{\mathcal{L}}
\newcommand{\xx}{\mathcal{x}}
\newcommand{\dist}{\xi}
\newcommand{\alg}{\mathsf{A}}
\newcommand{\qu}{\mathbf{q}}
\newcommand{\ex}{\mathbf{x}}
\newcommand{\ti}{\mathbf{t}}
\newcommand{\sdt}{\mathsf{SDT}}
\newcommand{\wdt}{\mathsf{WDT}}
\newcommand{\Qu}{\mathbf{Q}}
\newcommand{\pe}{\mathbb{P}}
\newcommand{\megam}{\mathcal{M}}
\newcommand{\eps}{\varepsilon}
\newcommand{\enet}{{$\varepsilon$-{\bf net}}\xspace}
\newcommand{\net}{{\tt net}\xspace}
\newcommand{\vcd}{VC-dimension\xspace}
\newcommand{\at}[1]{{\tt \small #1}\xspace}
\newcommand{\pr}{Pr}

\newcommand{\sharpP}{\mbox{\#P}}
\newcommand{\NP}{\mathsf{NP}}
\newcommand{\LP}{\mathsf{LP}}
\newcommand{\IP}{\mathsf{IP}}
\newcommand{\ru}{{\sc {RU}}\xspace}
\newcommand{\sru}{{\sc {strongRU}}\xspace}
\newcommand{\wru}{{\sc {weakRU}}\xspace}

\newcommand{\fmsystem}{{\sc Chameleon}\xspace}
\newcommand{\fm}{$\mathcal{F}$\xspace}

\newtheorem{experiment}{Experiment}

\begin{document}

\title{Coverage-based Data-centric Approaches for \\Responsible and Trustworthy AI\thanks{This research was supported by the National Science Foundation under grant No. 2107290.}}

\author{
\begin{tabular}[t]{c@{\extracolsep{2.4em}}c@{\extracolsep{2.4em}}c@{\extracolsep{2.3em}}c} 
Nima Shahbazi & Mahdi Erfanian & Abolfazl Asudeh \\ 
University of Illinois Chicago & University of Illinois Chicago & University of Illinois Chicago\\
 nshahb3@uic.edu & merfan2@uic.edu & asudeh@uic.edu
\end{tabular}
}

\maketitle


\begin{abstract}
The grand goal of data-driven decision systems is to help make decisions easier, more accurate, at a higher scale, and also just. However, data-driven algorithms are only as good as the data they work with. Yet, data sets, especially those with social data, often do not represent minorities. The paucity of training data is a perpetual problem for AI, and the outcome of ML models for cases not represented in their training data is often not reliable. 
Hence, without properly addressing the lack of representation issues in data, we cannot expect AI-based societal solutions to have responsible and trustworthy outcomes. 

This paper focuses on data coverage as a data-centric approach for identifying and resolving misrepresentation of minorities in data.
To achieve this goal, we propose novel algorithms that (a) {\it identify} and {\it resolve} insufficient data coverage across data with different modalities and (b) use lack of representation information to generate data-centric {\it reliability warnings}.
 \end{abstract}
 
 %%%%%%%%%%%%%%%%%%%%%%%%%%%%%%%% INTRO  %%%%%%%%%%%%%%%%%%%%%%%%%%%%%%%%
\section{Introduction}\label{sec:intro} % Abstract+Intro: up to 2.5 pages 
Data-driven decision-making has shaped every corner of human life, spanning from autonomous vehicles to healthcare and even predictive policing and criminal justice. A pivotal concern, especially in applications that affect individuals, revolves around the reliability of the decisions rendered by the system.
It is easy to see that the accuracy of a data-driven decision depends, first and foremost, on the data used to make it. Essentially, the system learns the phenomena that data represent. While we may desire that the data should represent the underlying data distribution from which the production data is drawn, this alone may be insufficient, as it merely enables the model to perform well for the average case.
As a result, a model with a high accuracy could fail for specific regions in the data with insufficient representation. These regions may matter because they frequently represent some minority population in society. They could also represent cases that may not happen very often but have a relevant impact on the correctness of a critical decision.
In short, if the data fails to sufficiently represent a specific population, the outcome of the decision system for that population may not be trustworthy.

The phenomenon known as \textit{Representation Bias} can arise from how the data was originally collected, or it could be the result of biases introduced post-collection—whether historically, cognitively, or statistically.

Representation bias is essentially inevitable without a systematic approach to data collection. 
For example, in the context of survey data collection, vital steps involve identifying all populations within the underlying distribution based on desired demographic information and ensuring comprehensive coverage with sufficient samples from each group. 
Even then, only an (uncontrolled) subset of the invitees will opt-in to respond to the survey.
Another challenge lies in the fact that data scientists often lack control over the data collection process, leading to the reliance on ``found data'' in the majority of data-driven systems. Therefore, with no guarantee on the aforementioned steps in the data collection process, the found data is most likely a biased sample.
Acknowledging the potential harms of representation bias, the notion of \textit{Data Coverage}~\cite{asudeh2019assessing,shahbazi2023representation} has been proposed to ensure the adequate representation of minority groups in data sets employed for decision-making and developing sophisticated data science tools. 

Addressing representation issues in data poses various challenges depending on the modality of the data. In this paper, we focus on identifying and resolving lack of coverage issues in data with different modalities.
We start by proposing a variety of techniques (spanning from geometric and combinatorial optimization to crowd-souring) aimed at efficiently detecting insufficient coverage on structured data sets with non-ordinal categorical and continuous attributes, as well as image data sets. Next, we propose a range of approaches grounded in data integration and generative data augmentation to address the lack of coverage by enriching the data sets with more data. However, with limited control over the data collection processes, it could be difficult and expensive to resolve all misrepresentations. 
Since adding more data is not always possible, we proceed to introduce data-centric preventive solutions that warn the user about the reliability of their predictions regarding representation bias issues. These warnings assist users in determining whether they trust the outcomes of the models or exercise caution. 

 %%%%%%%%%%%%%%%%%%%%%%%%%%%%%%%% IDENTIFICATION  %%%%%%%%%%%%%%%%%%%%%%%%%%%%%%%%
\section{Detecting Insufficient Representation of Minorities}\label{sec:identification} %up to 3.5 pages
Representation bias happens when the development (training data) population under-represents 
and subsequently fails to generalize well 
for some parts of the target population, due to historical bias, sampling bias, etc.
The notion of {\it data coverage} has been studied across different settings in \cite{shahbazi2023representation} as a metric to measure representation bias. At a high level, coverage is referred to as having enough similar entries for each object in a data set. 
For a better understanding, let us go over the definition of the generalized notion of coverage:

\begin{definition}[Data Coverage]\label{def:coverage}
Consider a data set $\dee$ with $n$ tuples, each consisting of $d$ attributes of interest $\mathbf{x}=\{x_1, x_2, \cdots,x_d\}$, such as {\tt gender}, {\tt race}, {\tt salary}, {\tt age}, etc, that are used for coverage identification.
The data set also contains target attributes $\mathbf{y} = \{ y_1,\cdots,y_{d'}\}$ that may or may not be considered for the coverage problem.
A query point $q$ is not covered by the data set $\dee$, if there are not ``enough'' data points in $\dee$ that are representative of $q$.
To generalize the notion of coverage, let us define $\gee(q)$ as the universe of tuples that would represent $q$ and let $\gee_\dee(q) = \gee(q)\cap \dee$. In other words, $\gee_\dee(q)$ are the set of tuples in $\dee$ that represent $q$.
Using this notation, we define the coverage of $q$ as the size of $\gee_\dee(q)$. That is,
$cov(q,\dee) = | \gee_\dee(q)|$.
Given a value $\tau$, $q$ is covered if $cov(q,\dee)>\tau$.
Similarly, a group $\gee$ is not covered if $\gee\cap \dee<\tau$.
The {\it uncovered region} in a data set is the collection of groups that are not covered by it.
\end{definition}

\subsection{Structured Data}
In this section, we focus on identifying representation bias in structured data.
Depending on the type of the attributes of interest, we categorize the techniques into two classes based on whether they target the problem for non-ordinal {\it categorical} (e.g. {\tt race}, {\tt gender}) or ordinal {\it continuous} (e.g. {\tt age}) attributes. The attributes of interest considered for representation bias often include sensitive attributes such as {\tt race} and {\tt gender} but are not necessarily limited to them.

\subsubsection{Categorical Attributes}

For cases where attributes of interest are non-ordinal categorical,
the cartesian product of values on a subset of attributes $\mathbf{x}'\subseteq \mathbf{x}$, form a set of (sub-)groups.
For example, $\{$ {\tt white male}, {\tt white female}, {\tt black male} $,\cdots\}$ are the subgroups defined on the attributes {\tt (race,gender)}.
We refer to the number of attributes used to specify a subgroup as the {\it level} of that subgroup.
For example, the level of the subgroup {\tt white male} is 2, while the level of the subgroup {\tt male} is 1.
We use $\ell(\gee)$, to refer to the level of a subgroup $\gee$.
Similarly, we say a subgroup $\gee'$ is a subset of $\gee$, if the groups specifying $\gee'$ are a superset of the ones for $\gee$. For example {\tt (married white male)} a subset of the more general group {\tt (white male)}. That is, the set of individuals in group {\tt (married white male)} are a subset of {\tt (white male)}.
Moreover, we say a subgroup $\gee$ is a {\it parent} of the subgroup $\gee'$, if $\gee'\subset \gee$ and $\ell(\gee)=\ell(\gee')+1$. For example, the subgroup {\tt (white male)} is a parent of the subgroup {\tt (married white male)}.
We use \textit{patterns} to refer to uncovered subgroups.
A pattern $P$ is a string of $d$ values, where $P[i]$ is either a value from the domain of $x_i$, or it is ``unspecified'', specified with $X$. 
For example, consider a data set with three binary attributes of interest $\mathbf{x}=\{x_1, x_2, x_3\}$. The pattern $P=X01$ specifies all the tuples for which $x_2=0$ and $x_3=1$ ($x_1$ can have any value).
The set of patterns that identify most general uncovered subgroups are called {\it Maximal Uncovered Patterns} (MUPs).

No polynomial time algorithm can guarantee the enumeration of the entire MUPs, however, several algorithms inspired by set enumeration and the Apriori algorithm for association rule mining are proposed to efficiently address this problem~\cite{asudeh2019assessing}.
In this regard, we introduce \textit{Pattern Graph} data structure that exploits the relationship between patterns to do less work than computing all uncovered patterns by removing the non-maximal ones. 
The parent-child relationship between the patterns is represented in a graph that can be used to find better algorithms. 
\textit{Pattern-Breaker} starts from the top of the graph where the general patterns are and moves down by breaking each pattern into more specific ones. If a pattern is uncovered, then all of its descendants are also uncovered and they can not be an MUP, even if they have a parent that is covered. Therefore, this subgraph of the pattern graph can be pruned. 
The issue with \textit{Pattern-Breaker} is that it explores the covered regions of the pattern graph and for the cases where there are a few uncovered patterns, it has to explore a large portion of the exponential-size graph. 
To tackle this, \textit{Pattern-Combiner} algorithm is proposed that performs a bottom-up traversal of the pattern graph. It uses an observation that the coverage of a node at the level of the pattern graph can be computed as the sum of the coverage values of its children. 
The problem with \textit{Pattern-Combiner} is that it traverses over the uncovered nodes first and therefore, it will not perform well for the cases in which most of the nodes in the graph are uncovered. 
In fact, for the cases where most of the MUPs are placed in the middle of the graph, both \textit{Pattern-Breaker} and \textit{Pattern-Combiner} will not be as efficient as they should traverse half of the graph. Therefore, we propose \textit{Deep-Diver}, a search algorithm based on Depth-First-Search that quickly finds the MUPs, and uses them to limit the search space by pruning the nodes both dominating and dominated by the discovered MUPs.

\begin{figure*}[!tb]
    \begin{minipage}[t]{0.31\linewidth}
        \centering
        \includegraphics[width=\textwidth]{submissions/submission1/shahbazi/covcube1.jpg}
        \caption{\small Categorical attributes: the uncovered region of a toy example, as the collection of three MUPs.}
        \label{fig:covcube1}
    \end{minipage}
    \hfill
    \begin{minipage}[t]{0.31\linewidth}
        \centering
        \includegraphics[width=\textwidth]{submissions/submission1/shahbazi/cvrg_2_1.jpg}
        \caption{\small Continuous attributes, 2D: identifying the covered region in the gray Voronoi cell.}
        \label{fig:cvrg_2_1}
    \end{minipage}
    \hfill
    \begin{minipage}[t]{0.31\linewidth}
        \centering
        \includegraphics[width=\textwidth]{submissions/submission1/shahbazi/cvrg_2_2.jpg}
        \caption{ \small Continuous attributes, 2D: Uncovered region marked in red.}
        \label{fig:cvrg_2_2}
    \end{minipage}
\vspace{-5mm}
\end{figure*}

\subsubsection{Continuous Attributes}
Data in the real world often consists of a combination of continuous and discrete values. While simple solutions like binning {\tt age} into {\tt young} and {\tt old} can transform the continuous space into discrete. However, they may lead to coarse groupings that are sensitive to the thresholds chosen. It may be inappropriate to treat a 35-yo as {\tt young} but a 36-yo as {\tt old}. 
Therefore, we extend the notion of coverage to continuous space. Particularly, given data set $\dee$ with $n$ tuples over $d$ attributes, and vicinity radius $\rho$ and coverage threshold $k$, we want to identify the uncovered region -- the universe of uncovered query points.
A query point in continuous data space is covered if there are enough (at least $k$) data points in its $\rho$-vicinity neighborhood. $\rho$-vicinity neighborhood is the circle centered at the query point with radius $\rho$.

Depending on the number of attributes in a data set, we propose two algorithms for identifying uncovered regions in data~\cite{asudeh2021coverage}. 
The first algorithm known as \textit{Uncovered-2D} studies coverage over two-dimensional data sets where $\mathbf{x}=\{x_1,x_2\}$. To find the number of circles that a query point falls into and consequently discover the uncovered region, \textit{Uncovered-2D} makes a connection to $k$-th order Voronoi diagrams.
Consider a data set $\mathcal{D}$ and its corresponding $k$-th order Voronoi diagram. For every tuple $t\in \mathcal{D}$, let $\circ_t$ be the $d$-dimensional sphere ($d$-sphere) with radius $\rho$ centered at $t$.
Consider a $k$-voronoi cell $\mathcal{V}(S)$ in the $k$-th order Voronoi diagram $V_k(\mathcal{D})$.
Any point $q$ inside the intersections of the $d$-spheres of tuples in $S$, i.e. $q\in \underset{\forall t\in S}{\cap ~\circ_t}$, is covered, while all other points in the region are uncovered.
 The algorithm starts by constructing the $k$-th order Voronoi diagram of the data set and then for each Voronoi cell $\mathcal{V}(S)$ in the diagram, it computes the intersection of the circles of the tuples in $S$ and marks the portion of $\mathcal{V}(S)$ that falls outside it as uncovered.
After identifying the uncovered region, a 2D map of $\{x_1,x_2\}$ value combinations is used to report the region to the user.
The algorithm for the 2D case can be extended to the general case by relaxing the assumption on the number of attributes to discover the exact uncovered region, however, due to the curse of dimensionality, the search size space explodes as the number of dimensions increases and as a result, the algorithm will not be practical. Therefore, we propose a randomized approximation algorithm based on the geometric notion of \enet. 
Let $\mathcal{X}$ be a set and $\mathcal{R}$ be a set of subsets of $\mathcal{X}$. A set $\mathcal{N}\subset \mathcal{X}$ is an \enet for $\mathcal{X}$ if for any range $r\in\mathcal{R}$, if  $|r\cap \chi|>\eps|\chi|$, then $r$ contains at least one point of $N$.
The idea, at a high level, is to draw enough random samples from the space of potential query points to form an \enet. 
We then label the sampled query points as $\{-1,+1\}$ depending on whether those are covered or not, and learn the uncovered regions using the samples.

\subsection{Image Data}
Many known incidents of machine failures due to the lack of representation were on image data.
We consider an image data set with a fixed number of low-cardinality sensitive attributes such as {\tt\small race} and {\tt\small gender}. 
It is common that image data sets {\it lack explicit values} for sensitive attributes, which are crucial for coverage identification. An image data set is often a collection of images from different domains with little to no information about their domain and which groups they belong to. As a result, even studying coverage over low-cardinality and categorical attributes of interests is challenging in these cases.

\begin{wrapfigure}{R}{0.42\textwidth}
\centering
\vspace{-3mm}
\scriptsize
\begin{tabular}{|@{}c|@{}c@{}|@{}c@{}|@{}c@{}|} 
 \hline
{\bf data set} & {\bf classifier} & {\bf accuracy} & {\bf precision} \\ 
 &  &  & {\bf on female} \\ \hline
UTKFace:~& DeepFace (opencv) & 93.56 & {52.02}\\\cline{2-4}
({\tt females}=200,& DeepFace (retinaface) & 94.16 & {56.15}\\\cline{2-4}
{\tt males}=2800) & BaseCNN & 97.6 & 74.8\\
\hline
UTKFace:~& DeepFace (opencv) & 96.53 & {\bf 8.0}\\\cline{2-4}
({\tt females}=20,& DeepFace (retinaface) & 96.43 & {\bf 10.09}\\\cline{2-4}
{\tt males}=2980)& BaseCNN & 97.6 & {\bf 21.59}\\
\hline
\end{tabular}
\vspace{-3mm}
\caption{\small ML models' low performance for females in the presence of representation bias.~\cite{mousavi2024data}}\label{fig:mlfails}
\vspace{-3mm}
\end{wrapfigure}

In Figure~\ref{fig:mlfails}, we show that due to the issues such {\it machine bias} and {\it lack of distribution generalizability},
solely relying on state-of-the-art machine learning (ML) techniques fail to effectively identify lack of coverage in image data sets. Therefore, we propose an approach based on combining crowdsouring with ML~\cite{mousavi2024data}. 
Crowdsourcing is particularly promising for image data, for tasks such as image labeling, which, while challenging for the machine, are "easy" for human beings to conduct with minimal error. 

A key observation that enables a cost-effective crowdsourcing approach is that, while studying coverage, we would only like to find out if there are {\it enough tuples from each subgroup}.
Suppose a subgroup is covered if there are $\tau=100$ instances of it in the data set. Assume the (majority) group $\gee_1$ contains $n_1 \gg 100$ objects in the data set. 
To verify that $\gee_1$ is covered, it is enough for the crowd to discover 100 of those objects, not the entire $n_1$. 
Following this, $O(\tau)$ provides a lower bound on the number of crowd tasks required to verify a given group is covered. 
Still, this lower bound only holds for the groups that are covered, i.e., there is at least $\tau$ of those in the data set.
Surprisingly, verifying that a minority group is indeed uncovered is cumbersome, unlike the majority group.
This is because even though discovering $\tau$ objects from a group is enough for verifying that it is covered, one cannot {\it verify} a group is uncovered until there is a chance that the data set might still have enough objects from that group. Thus, assuming a non-zero probability for each unlabeled object to belong to each group, {one might need to ask the crowd to label the entire data set before they can confirm that a specific group is uncovered}.

Our idea for addressing this challenge is to
design {\it a divide and conquer algorithm} that, instead of {point queries}, uses {\it set queries} to iteratively eliminate subsets of data that {does not include any object from the given group}.
At a high level, our idea is to ask a set query from the crowd, inquiring whether the selected set contains at least one object from the given group $\gee$.
The user may provide two responses (yes/no). 
Interestingly, {in either case}, the user response provides valuable information that helps efficiently identify the coverage.
If the answer is ``No'', the set does not include any object from the given group $\gee$. As a result, the algorithm can safely prune the set, asking no further questions about it. In particular, for a group that is not covered, one can expect to see no answers on large set queries helping to prune a significant portion of the data set quickly.
On the other hand, if the answer is ``yes'', the set contains {at least} one object from the group $\gee$. As a result, the algorithm cannot prune the subset since it can have any number (larger than one) of the objects in $\gee$.
At first glance, the queries with yes answers do not provide helpful information as the algorithm cannot prune the subset (hence it needs to divide it into smaller subsets).
However, a key observation is that {the algorithm will only observe a limited number of yes answers} before it stops.
The reason is that the number of set queries with yes answers provides a {lower-bound} on the number of objects from $\gee$ in the data set. As a result, the algorithm can stop as soon as the lower bound reaches $\tau$, knowing that $\gee$ is covered.
The D\&C approach verifies the data coverage for a given group, while our goal is to identify the uncovered regions for a given set of sensitive attributes. The next question is how to utilize this algorithm for efficient coverage identification on different scenarios of sensitive attributes, forming intersectional or non-intersectional groups.
In particular, how can we find maximal uncovered patterns?
Our idea is to apply sampling and aggregate estimation techniques to find the groups that even if merged are likely to still be uncovered. This will help reduce the coverage identification cost by running the D\&C approach for the merged groups once.
 %%%%%%%%%%%%%%%%%%%%%%%%%%%%%%%% RESOLUTION  %%%%%%%%%%%%%%%%%%%%%%%%%%%%%%%%
\section{Resolving Insufficient Representation}\label{sec:resolution}

Data integration~\cite{nargesian2021tailoring,nargesian2022responsible} and data augmentation~\cite{sharma2020data,DBLP:journals/jair/ChawlaBHK02,iosifidis2018dealing,celis2020data} are considered as the primary solutions for reducing data coverage issues in a data set. 
Data integration is promising when external sources of data are available. On the other hand, recent advancements in generative AI and foundation models have enabled efficient and effective augmentation of data sets with synthetic data. 
Therefore, in the following, we review two approaches, one from each category, in the context of lack of coverage resolution.

\subsection{Data Integration}\label{sec:resolution:integration}

Data integration is to consolidate data from different sources into a single, unified view. 
Although it is an effective solution to acquire additional data from different distributions,
there are sampling policy and cost-efficiency concerns that need to be examined.  
Therefore, {\it Data Distribution Tailoring ({\sc DT})} introduces data integration techniques for resolving insufficient representation of subgroups in a data set in the most cost-effective manner~\cite{nargesian2021tailoring}.
A query to {\sc DT} 
consists of a target schema, and a set of group distribution requirements in the form of the minimum counts (e.g., ``{\tt\small 1,000 breast cancer monitoring data in Chicago with at least 30\% label=positive, and at least 20\% black patients}''). 
Collecting a fresh sample from a data view is costly (monetary, human resources, and/or computation cost)~\cite{asudeh2022towards}.
Therefore, {\sc DT} focuses on satisfying the count requirements with minimum cost. 
Given an input query and a lake of available data sources, the first step is to discover a collection of candidate data views that satisfy the target schema.
Each data view $v_i$ is a projection-join $v_i = \Pi\big(D_{i1}\bowtie\cdots\bowtie D_{ik_i} \big)$, where $D_{ij}$ is a data set in a given data lake.
Let us suppose the data views are already discovered.
At a high level, {\sc DT} follows an iterative approach that at each iteration a data view is selected to be queried.
Each query to a data view has a fixed cost and returns a sample that may or may not satisfy the query constraints.
The samples that are either not fresh, or do not satisfy the query are discarded.
Hence, the essential question towards a cost-effective data integration is {\it what data view to query next}.
Depending on the available information about the data sources, various techniques may be employed. 

For the cases when the group distributions are known, the process of collecting the target data set is a sequence of iterative steps, where at every step, the algorithm chooses a data view, queries it, and if the obtained tuple contributes to one of the groups for which the count requirement is not yet fulfilled, it is kept, otherwise discarded. To do so, a {Dynamic Programming (DP)} algorithm is proposed. An optimal source at each iteration minimizes the sum of its sampling cost plus the expected cost of collecting the remaining required groups, based on its sampling outcome.
The DP algorithm, however, has a pseudo-polynomial time complexity. Hence, it quickly becomes intractable for cases where the minimum count requirements for the groups are not small. 
For cases where the (sensitive) attribute of interest is binary, such as (biological) {\tt sex}={\tt \{male, female\}}, and the cost to query data is similar from all sources, it turns out that the optimal strategy is to query the data source with {maximum probability of obtaining a sample from the minority group}.
Expanding the binary-attributes algorithm for non-binary cases, the problem can be modeled as an extension of the ``{\it coupon collector's}'' problem~\cite{motwani1995randomized}, where the goal is to collect $m_i$ instances from each coupon (group) $\gee_i$.
At each iteration, the coupon collector's algorithm identifies a data view as most promising and queries it. In simple terms, a data view with a smaller query cost and a higher chance of obtaining minority groups is more promising.


For the cases where the group distributions are unknown, we model DT as a {\it multi-armed bandit} problem, where every data view is modeled as an arm. 
Every arm has an unknown distribution of different groups while pulling an arm (i.e., querying the corresponding data view) has a cost.
During various iterations, the algorithms pull the arms in an order that its expected total {\it reward} is maximized.
Arguing that the reward of obtaining a tuple from a group is proportional to how rare this group is across different data views, 
we design the reward function based on the expected cost one needs to pay in order to collect a tuple from a specific group.  
As the bandit strategy, we adopt {\it Upper Confidence Bound (UCB)} to balance exploration and exploitation. At every iteration, for every arm, UCB computes confidence intervals for the expected reward and selects the arm with the maximum upper bound of reward to be explored next.

\subsection{Data Augmentation using Foundation Models}

While data integration provides a promising approach for resolving coverage issues in a data set, its effectiveness is limited to the availability of external data sources that are rich enough to find sufficient fresh samples from minority groups. This, however, is not always possible, especially since the minority samples are rare and not easy to obtain.
Fortunately, recent advancements in Generative AI and Foundation Models have enabled synthesizing samples that are otherwise challenging to obtain from the real world.

Therefore, as an alternative approach to data integration, we turn our attention to the Foundation Models and Generative AI for resolving the lack of coverage. 
Particularly, models such as {\sc DALL.E}\footnote{\url{https://openai.com/dall-e-2}} have emerged as powerful tools for generating multi-modal data such as image, audio, and video.
 
We formalize the foundation model \fm as a black-box function with the following inputs, that once queried synthesize an output tuple.
\begin{itemize}
    \item {\bf Prompt}: A natural language description providing instructions on the details of the tuple to be generated. For instance, a prompt for image generation might be ``A realistic photo of a white cat running in a backyard.''
    \item {\bf Guide}: In cases where only a prompt is provided, the foundation model uses its imagination to generate the requested tuple. For the previous example, the prompt of a cat image, the breed, size, background, and other details are generated based on the model's imagination. Alternatively, a guide can be provided to influence the generation process. The guide is formalized as a pair $(t,m)$ where $t$ is a tuple and $m$ is a mask specifying which parts of the guide tuple should be changed. Using the cat example, $t$ can be a cat image and $m$ can specify the foreground to be regenerated.
\end{itemize}

There are multiple challenges towards effective data set augmentations using foundation models. 
First, we have to determine the minimal set of synthetic tuples that once added to the original data set, under-representation issues are resolved.
Second, the generated images should follow the underlying distribution represented in the input data set. Third, the generated tuples should have high quality and look realistic to a human evaluator. Last but not least, given the (often monetary) cost associated with the queries to the foundation model, we should ensure the cost-effectiveness of the data set repair process.

\begin{wrapfigure}{L}{0.45\textwidth}
\centering
\vspace{-3mm}
\scriptsize
    \includegraphics[width=.45\textwidth]{submissions/submission1/shahbazi/enhanced_pipeline.png}
\vspace{-3mm}
\caption{\small Architecture of \fmsystem for image data augmentation for coverage enhancement.}\label{fig:chameleon}
% \vspace{-3mm}
\end{wrapfigure}

\noindent Figure~\ref{fig:chameleon} shows the architecture of our system \fmsystem \cite{chameleon} for coverage enhancement using DALL-E image generator.
To address the first challenge, we define the combinations-selection problem, which minimizes the total number of synthetic tuples for resolving lack of coverage of minorities at the most general level. We show the problem is {\sc NP}-hard, and propose a greedy approximation algorithm for it.
To address the second and third challenges, \fmsystem follows a {\it rejection sampling} strategy.
It views each tuple in the data set $\dee$ as an iid sample from the underlying distribution $\xi$ it represents. It uses the vector representations (embeddings) space to describe the distribution. Then, given a newly generated tuple, it employs the one-class support vector machine (OCSVM) approach proposed by Scholkopf et al.~\cite{scholkopf1999support} to reject the tuple if it does not follow $\xi$.
Moreover, it models the quality evaluation as hypothesis testing and rejects the samples that have a higher chance of being labeled as ``unrealistic'' by a random human evaluator.
Finally, to minimize the number of queries to the foundation model, we provide a guide tuple (and a mask), in addition to the prompt, to the foundation model. We model the guide-selection problem as {\it contextual multi-armed bandit} and propose a solution based on the contextual UCB for it.

Before concluding this section, let us provide some experiment results to demonstrate the effectiveness of data augmentation with \fmsystem. We use FERET DB \cite{phillips1998feret} for this experiment, which comprises 1199 individual images and serves as a standardized facial image database for researchers to develop algorithms and report results. All images in FERET DB share the same dimensions, pose, and facial expression.
First, we identified the (level-1) uncovered ethnicity groups, using the threshold 80. We then used \fmsystem and resolved the lack of coverage issues.
To evaluate the effectiveness of the system, we trained a CNN model to predict the race of each image within this dataset. We then retrained the identical CNN on the repaired training data. Importantly, our test dataset for both experiments remains consistent and is derived from real images.
Table~\ref{tab:lackofcoverage} presents the improvements in precision, recall, and F1 score metrics for under-represented groups after repairing the dataset. The results indicate an enhancement in performance metrics for all under-represented groups following the repair process.

\begin{table}[t]
    \centering
    \caption{Illustrating the effect of lack of coverage repair using \fmsystem on \texttt{FERTDB}}
    \label{tab:lackofcoverage}
    \vspace{-3mm}
    \begin{tabular}{lcccccccc}
        \toprule
         & \multicolumn{4}{c}{\textbf{Classifier Performance on \texttt{FERTDB}}} & \multicolumn{4}{c}{\textbf{Classifier Performance on Repaired}} \\
        \cmidrule(lr){2-5} \cmidrule(lr){6-9}
        \textbf{Ethnicity Groups}& \#Images & Precision & Recall & F1-Score & \#Images & Precision & Recall & F1-Score \\
        \midrule
        Overall          & 756 & 0.81 & 0.75 & 0.78 & 987 & 0.70 & 0.75 & 0.72 \\ \hline
        Black            & 40  & 0.19 & 0.22 & 0.16 & 100 & 0.48 & 0.56 & 0.52 \\
        Hispanic         & 19  & 0.50 & 0.17 & 0.25 & 100 & 0.62 & 0.36 & 0.45 \\
        Middle Eastern   & 10  & 0.00 & 0.00 & 0.00 & 100 & 0.20 & 0.41 & 0.27 \\
        \bottomrule
    \end{tabular}
\end{table}

 %%%%%%%%%%%%%%%%%%%%%%%%%%%%%%%% RELIABILITY  %%%%%%%%%%%%%%%%%%%%%%%%%%%%%%%%
\section{Generating Reliability Warnings}\label{sec:reliability}
% up to 2.5 pages
Interpretability is a necessity for data scientists who develop predictive models for critical decision-making.
In such settings, it is important to provide additional means to support the following question:
{\it is an individual prediction of the model reliable for decision-making?} Our goal is to use the lack of representation to help decision-makers find insights about this critical question.
To further motivate this, let us use the following example:

\vspace{1mm}
\begin{example}\label{ex-0}
{\bf(Part1):} Consider a judge who needs to decide whether to accept or deny a bail request. Using data-driven predictive models is prevalent in such cases for predicting recidivism~\cite{dressel2018accuracy}.
Indeed, such models can be beneficial to help the judge make wise decisions.
Suppose the model predicts the queried individual as high risk (or low risk).
The judge is aware and concerned about the critics surrounding such models.
A major question the judge faces is whether or not they should rely on the prediction outcome to take action for this case.
Furthermore, if, for instance, they decide to ignore the outcome and hence they need to provide a statement supporting their action, what evidence can they provide? 
\end{example}

In line with the recent trend on data-centric AI~\cite{ng2021mlops}, we design {novel approaches}, {complimentary} to the existing work on trustworthy AI~\cite{wing2021trustworthy,kentour2021analysis,liu2021trustworthy,singh2021trustworthy}, to address the aforementioned trust question through the lens of {\it data}.
In particular, unlike existing works that generate trust information from a {\it given \underline{model}}, we associate {\it \underline{data sets} with proper measurements} that specify their {\it the scope of use for predicting future cases}.
We note that a predictive model provides only probabilistic guarantees on the \underline{average} loss over the distribution represented by the data set used for training it.
As a result, these predictions may not be distribution generalizable~\cite{kulynych2022you}.
Consequently, if the query point is {\it not represented} by the data, the guarantees may not hold, hence one cannot rely on the prediction outcome.
Besides, an essential requirement for a learning algorithm is that its training data $\dee$ should represent the underlying distribution $\dist$.
Even if so, the trained model $h$ only provides a probabilistic guarantee on the {expected} loss on random samples from $\dist$.  
A model that performs well on {\it majority} of samples drawn from $\dist$ will have a high performance on average. Still, as we observed in Figure~\ref{fig:mlfails},
its performance for {\it minorities} and points that are not represented is questionable. Let us consider the following toy example:

\begin{figure*}[!b] 
    \begin{minipage}[t]{0.32\linewidth}
        	\centering
        	\includegraphics[width=\textwidth]{submissions/submission1/shahbazi/example_1.png} 
        	\vspace{-9mm}\caption{\small Data set $\dee$ generated using a Gaussian distribution; $x_1$ and $x_2$ are positively correlated}
            \label{fig:ex1:1}
    \end{minipage}
    \hfill
    \begin{minipage}[t]{0.32\linewidth}
        \centering
        	\includegraphics[width =\textwidth]{submissions/submission1/shahbazi/example_2.png} 
        	\vspace{-9mm}\caption{\small The decision boundary of learned model $h$ and query points $\qu^1$ to $\qu^4$}
            \label{fig:ex1:2}
    \end{minipage}
    \hfill
    \begin{minipage}[t]{0.32\linewidth}
        	\centering
        	\includegraphics[width =\textwidth]{submissions/submission1/shahbazi/example_3.png}
        	\vspace{-9mm}\caption{\small Ground-truth boundary, overlaid on the model decision boundary and query points}
            \label{fig:ex1:3}
    \end{minipage}
    \vspace{-5mm}
\end{figure*} 

\vspace{1mm}
\begin{example}\label{ex-1}
Consider a binary classification task where the input space is $\ex=\langle x_1, x_2\rangle$ and the output space is the binary label $y$ with values $\{-1$ (red) $,+1$ (blue)$\}$.
Suppose the underlying data distribution $\dist$ follows a 2D Gaussian, where $x_1$ and $x_2$ 
are positively correlated as shown in Figure~\ref{fig:ex1:1}.
The figure shows the data set $\dee$ drawn independently from the distribution $\dist$, along with their labels as their colors.
Using $\dee$, the prediction model $h$ is constructed as shown in Figure~\ref{fig:ex1:2}. 
The decision boundary is specified in the picture; while any point above the line is predicted as +1, a query point below it is labeled as -1.
The classifier has been evaluated using a test set that is an iid sample set drawn from the underlying data set $\dist$. The accuracy on the test set is high (above 90\%), and hence, the model gets deployed.
We cherry-picked four query points, $\qu^1$ to $\qu^4$, that are also included in Figure~\ref{fig:ex1:2}. Using $h$ for prediction, $h(\qu^1)=-1$, $h(\qu^2)=+1$,  $h(\qu^3)=+1$, and $h(\qu^4)=-1$.
Figure~\ref{fig:ex1:3} adds the ground-truth boundary to the search space, revealing the true label of the query points: every point inside the red circle has the true label $-1$ while any point outside of it is $+1$.
Looking at the figure, $y^1=+1$ while the model predicted it as $h(\qu^1)=-1$.  \hfill$\square$
\end{example}
\vspace{2mm}

Let us take a closer look at the four query points in this example and their placement with regard to the tuples in $\dee$ used for training $h$. 
$\qu^2$ belongs to a {\it dense region} with many training tuples in $\dee$ surrounding it. Besides, all of the tuples in its vicinity have the same label $y=+1$. As a result, one can expect that the model's outcome $h(\qu^2)=+1$ should be a reliable prediction.
Similar to $\qu^2$, $\qu^4$ also belongs to a dense region in $\dee$; however, $\qu^4$ belongs to an {\it uncertain region}, where some of the tuples in its vicinity have a label $y=+1$, and some others have the label $y=-1$. Considering the uncertainty in the vicinity of $\qu^4$, one cannot confidently rely on the outcome of the model $h$. 
On the other hand, the neighbors of $\qu^1$ (resp. $\qu^3$) are not uncertain, all having the label $y=-1$ (resp. $y=+1$).
However, the query points $\qu^1$ and $\qu^3$ are not well represented by $\dee$. In other words, $\qu^1$ and $\qu^3$ are unlikely to be generated according to the underlying distribution $\dist$, represented by $\dee$. As a result, following the no-free-lunch theorem~\cite{kakade2003sample}, one cannot expect the outcome of model $h$ to be reliable for these points.
Looking at the ground-truth boundary in Figure~\ref{fig:ex1:3}, $h$ luckily predicted the outcome for $\qu^3$ correctly, but it was not fortunate to predict the $y^1$ correctly.
Nevertheless, 
since the model is not reliably trained for these points, 
its outcome for these query points is not trustworthy.

From Example~\ref{ex-1}, we observe that the outcome of a model $h$, trained using a data set $\dee$ is not reliable for a query point $\qu$, if:
\begin{itemize}
    \item {\bf Lack of representation:} $\qu$ is not well-represented by $\dee$.
    In such cases, the model has not seen ``enough'' samples similar to $\qu$ to reliably learn and predict the outcome of $\qu$.
    \item {\bf Lack of certainty:} $\qu$ belongs to an uncertain region, where different tuples of $\dee$ in the vicinity of $\qu$ have different target values. $\qu$ belongs to a high-fluctuating area, where tuples in the vicinity of $\qu$ have a wide range of values.
\end{itemize} \vspace{2mm}

\noindent
Based on these two observations, we propose Representation-and-Uncertainty ({\bf RU}) measures.
To identify if a query suffers from uncertainty or lack of representation, one could use a deterministic approach using a fixed threshold. Then if the number of similar samples to (resp. label fluctuation in vicinity of) $\qu$ is larger than the threshold it is considered as unrepresented (resp. uncertain).
This approach, however, would be misleading since two numbers close to the threshold could be treated very differently. Also, all points on each side of the threshold would be considered equally represented (resp., certain). Instead, we consider {\it a randomized approach}, widely popular in the literature, including~\cite{dwork2012fairness}.
That is, instead of using fixed thresholds, a Bernoulli variable (a biased coin) is used that 
assigns $\qu$ as unrepresented (resp., uncertain) based on the number of samples similar to it (resp., its neighborhood uncertainty).
Given a query point $\qu$, let $\pe_o$ be the probability indicating if $\qu$ is not represented and let $\pe_u$ be the probability indicating if $\qu$ belongs to an uncertain region. 
We represent the probability of the Bernoulli variables for lack of representation or uncertainty components as $\pe_o$ and $\pe_u$, respectively. Note that the two Bernoulli variables $\pe_o$ and $\pe_u$ are independent from each other. That simply follows the argument that after specifying the number of similar samples to $\qu$ whether or not it should be considered as unrepresented does not depend on the uncertainty in the neighborhood of $\qu$.

\begin{definition}[\sru]\label{def:sdt}
The \sru is a probabilistic measure that considers the outcome of a model for a query point $\qu$ untrustworthy if $\qu$ is not represented by $\dee$ {\it and} it belongs to an uncertain region.
Formally, the \sru measure is:
\begin{align} 
    \nonumber
    SRU(\qu) &= \pe\big((\qu \mbox{ is outlier}) \wedge (\qu \mbox{ belongs to uncertain region})\big) 
\end{align}
Since $\pe_o$ and $\pe_u$ are independent:

\vspace{-13mm}
\begin{align} \label{eq:strong}
    SRU(\qu) &= \pe_o(\qu) \times \pe_u(\qu)
\end{align}
\end{definition}

\sru raises the warning signal only when the query point fails on {\it both} conditions of being represented by $\dee$ and not belonging to an uncertain region. 
For instance, in Example~\ref{ex-1} none of the query points fail both on representation and on uncertainty; hence neither has a high \sru score.
On the other hand, 
a high \sru score for a query point $\qu$ {\it provides a strong warning signal} that one should perhaps reject the model outcome and not consider it for decision-making.

\sru is a strong signal that raises warnings only for the fearfully concerning cases that fail both on representation and uncertainty.
However, as observed in Example~\ref{ex-1} a query points failing {\it at least} one of these conditions may also not be reliable, at least for critical decision making.
We define the \wru measure to raise a warning for such cases.

\begin{definition}[\wru]\label{def:wdt}
The \wru measure is a probabilistic measure that considers the outcome of a model for a query point $\qu$ untrustworthy if $\qu$ is not represented by $\dee$ {\bf or} it belongs to an uncertain region.
Formally, the \wru is computed as:
\begin{align} \label{eq:weak}
    WRU(\qu) = \pe\big((\qu \mbox{ is outlier}) \vee (\qu \mbox{ belongs to uncertain region})\big) 
    = \pe_o(\qu) + \pe_u(\qu) - \pe_o(\qu) \times \pe_u(\qu)
\end{align}
\end{definition}

Proposing quantitative probabilistic outcomes, \ru measures are interpretable for the users, since beyond the scores, the uncertainty and lack of representation components provide an explanation to justify them. 
Please refer to \cite{techrep} for more details on how to efficiently and effectively compute the representation ($\pe_o$) and uncertainty ($\pe_u$) probabilities, using only $\dee$.
In Example~\ref{ex-0}, let us see how the \ru measures can be helpful.

\noindent{\bf Example 1. (part 2):}
{\it RU measures \underline{raise warning} when
the fitness of the data set used for drawing a prediction is questionable, helping the judge to be cautious when taking action.
Besides, these measures provide \underline{quantitative evidence} to support the judge's action when they decide to ignore a prediction outcome that is not trustworthy.
The judge, for example, can argue to ignore a model outcome for a specific case, based on the insight that 
the model has been built using a
data set that fails to represent the given case.}
\hfill$\square$

Finally, let us demonstrate the efficacy of \ru measures through a series of experiments. Since the \ru measures are {\it data-centric},
those are applicable for both classification and regression tasks, irrespective of the model used.
We use {\it Adult} dataset~\cite{adult} for classification and {\it House Sales in King County} dataset for the validation of regression tasks. From each dataset, we uniformly sample two sets from the underlying distribution. The first set serves as the training set to compute the \ru values, and the second one is used as the test set from which the queries are drawn. We validate our proposal by providing the correlation between the \ru values and the performance of an ML model's prediction on the same data. 

We start by computing the \ru values for all the query points in the test set. Next, we bucketize the query points based on their \ru values in equi-width buckets of width 0.1. We repeat this for both \sru and \wru measures. Next, we train a model on the training data set and predict the target variable for the points in each range of \ru measure. The validation results for the classification task on the {\it Adult} dataset are presented in Figures \ref{fig:exp-adult-sdt} and \ref{fig:exp-adult-wdt}. Each figure corresponds to the accuracy/error measures of the classifier over each bucket of \ru values for \sru and \wru. As the \ru values increase, the accuracy of the model drops while the FPR rises, and therefore, the model fails to capture the ground truth for the points that fall into untrustworthy regions in the data set. By repeating the aforementioned steps for the regression task on the {\it House Sales in King County} dataset, we observe similar results presented in Figures \ref{fig:exp-hs-sdt} and \ref{fig:exp-hs-wdt}. 
As the \ru value increases, the RSS of the regression model follows the same trend denoting that the model fails to perform for tuples with a high \ru value.

\begin{figure}[!tb]
    \begin{minipage}[t]{0.24\linewidth}
        \centering
        \includegraphics[width=\textwidth]{submissions/submission1/shahbazi/sdt_adult.pdf}
        \vspace{-6mm}\caption{\small{\it Adult}, efficacy of \sru  on classification}
        \label{fig:exp-adult-sdt}
    \end{minipage}\hfill
    \begin{minipage}[t]{0.24\linewidth}
        \centering
        \includegraphics[width=\textwidth]{submissions/submission1/shahbazi/wdt_adult.pdf}
        \vspace{-6mm}\caption{\small{\it Adult}, efficacy of \wru  on classification}
        \label{fig:exp-adult-wdt}
    \end{minipage}\hfill
    \begin{minipage}[t]{0.24\linewidth}
        \centering
        \includegraphics[width=\textwidth]{submissions/submission1/shahbazi/sdt_regression_house.pdf}
        \vspace{-6mm}\caption{\small{\it House Sales in King County}, efficacy of \sru on regression}
        \label{fig:exp-hs-sdt}
    \end{minipage}\hfill
    \begin{minipage}[t]{0.24\linewidth}
        \centering
        \includegraphics[width=\textwidth]{submissions/submission1/shahbazi/wdt_regression_house.pdf}
        \vspace{-6mm}\caption{\small{\it House Sales in King County}, efficacy \wru on regression}
        \label{fig:exp-hs-wdt}
    \end{minipage}
\vspace{-5mm}
\end{figure}
 %%%%%%%%%%%%%%%%%%%%%%%%%%%%%%%% RELATED WORK  %%%%%%%%%%%%%%%%%%%%%%%%%%%%%%%%
\section{Related Work}\label{related} 

Bias in data has been looked at for a long time in statistical community~\cite{neyman1936contributions} but social data presents different challenges~\cite{olteanu2019social,fairmlbook,barocas2016big,jk2019bias,drosou2017diversity}.
The diversity and representativeness of data have been widely studied~\cite{drosou2017diversity}, in fields such as social science~\cite{berrey2015enigma, dobbin2016diversity,simpson1949measurement}, political science~\cite{surowiecki2005wisdom}, and information retrieval~\cite{agrawal2009diversifying}. 
Tracing back machine bias to its source, there have been major efforts to identify different types~\cite{mehrabi2021survey, olteanu2019social,friedman1996bias} and sources~\cite{torralba2011unbiased,crawford2013hidden,diakopoulos2015algorithmic} of biases in data. Efforts to satisfy {\it responsible data} requirements~\cite{nargesian2022responsible} extend to various stages of the data analysis pipeline, including data annotation~\cite{li2020towards,lazier2023fairness}, data cleaning and repair~\cite{SalimiRHS19,tae2019data,salimi2020database}, data imputation~\cite{martinez2019fairness}, entity resolution~\cite{shahbazi2023through,fanourakis2023fairer}, data integration~\cite{nargesian2022responsible,nargesian2021tailoring}, etc. 

\paragraph{Data Coverage:}The notion of data coverage has received extensive attention from different angles. Detecting lack of coverage has been studied for datasets with discrete~\cite{asudeh2019assessing} and continuous~\cite{asudeh2021coverage} attributes populated in single or multiple \cite{lin2020identifying} relations.
To resolve insufficient coverage, \cite{accinelli2020coverage, accinelli2021impact,shetiya2022fairness}
consider resolving representation bias in preprocessing pipelines by rewriting queries into the closest operation so that certain subgroups are sufficiently represented in the downstream tasks. Alternatively, ~\cite{asudeh2019assessing,tae2021slice} propose a data collection strategy to acquire as little additional data as possible (to minimize the associated costs) to meet the representation constraints. ~\cite{sharma2020data,iosifidis2018dealing,celis2020data} opt for a data augmentation approach by adding partially altered duplicates of already existing tuples or generating new synthetic entries from existing data. Consequently, the new data set has an equal number of elements for different groups, resulting in potentially resolving the under-representation issues. Finally,  \cite{nargesian2021tailoring} utilizes data integration techniques to consolidate data from different sources into a single dataset to resolve representation bias.
Related works also include ~\cite{chung2019slice,sagadeeva2021sliceline,tae2021slice} that seek to understand if the overall performance of the model fails to reflect and performs poorly on certain slices in the data.
As alternative approaches to measure representation bias, the notion of representation rate~\cite{celis2020data} (a.k.a. equal base rate~\cite{kleinberg2016inherent}) is introduced which compared with coverage, it is more restrictive as it requires almost equal ratios from different groups.
Please refer to \cite{shahbazi2023representation} for a comprehensive survey about representation bias in data. 

\paragraph{ML Reliability:} Model-centric works for uncertainty quantification such as 
probabilistic classifiers~\cite{zadrozny2001obtaining,zadrozny2002transforming,platt1999probabilistic,niculescu2005predicting},
prediction intervals (PIs) \cite{chatfield93predictionintervals,pearce2018high,khosravi2010lower} and conformal predictions (CP)~\cite{angelopoulos2021gentle,shafer2008tutorial} that are used for measuring prediction uncertainty, are built
by maximizing the {\it expected performance} on {\it random} sample from the underlying distribution.
As a result, while providing accurate estimations for the dense regions of data (e.g. majority groups), their estimation accuracy is questionable for the poorly represented regions.
In particular, \cite{angelopoulos2021gentle} recognizes the lack of guarantees in the performance of CP for such regions.
Besides, the bulk of work on trustworthy AI provides information that {\it supports} the outcome of an ML model. For example, existing work on explainable AI, including~\cite{harradon2018causal,ribeiro2016should,gunning2019darpa}, aims to find simple explanations and rules that justify the outcome of a model.
Conversely, we aim to {\it raise warning signals} when the outcome of a model is {\it not} trustworthy. That is, to provide reasons that {\it cast doubt} on the reliability of the model outcome {for a given query point}.

 %%%%%%%%%%%%%%%%%%%%%%%%%%%%%%%% FUTURE  %%%%%%%%%%%%%%%%%%%%%%%%%%%%%%%%
% \vspace{-3mm}
\section{Final Remarks}\label{sec:conclusion}
As Data-centric AI and Responsible AI emerge as focal points in data science research, the development of Data-centric methodologies for ensuring Responsible and Trustworthy AI attracts increasing attention.
While there is some excellent work on responsible data management to achieve this goal, there remain many challenges yet to be addressed.

In this paper, we focused on a crucial aspect of responsible data -- detecting and addressing the under-representation of minorities within a data set.
We formally defined the notion of data coverage and discussed various techniques for (a) identifying lack of representation issues across different data modalities, (b) ensuring proper representation of minorities in data, and (c) limiting the scope-of-use of data sets based on their representation issues by generating proper ({\sc RU}) warning signals.
Even though the research on detecting lack of coverage issues is relatively mature, resolution techniques are still understudied.
Considering the recent advancements in Generative AI, utilizing Foundation Models and Large Language Models, and studying their limitations, for data augmentation to improve the representation of minorities at the data level seems interesting to further explore.

 %%%%%%%%%%%%%%%%%%%%%%%%%%%%%%%% BIB  %%%%%%%%%%%%%%%%%%%%%%%%%%%%%%%%
\bibliographystyle{unsrt}
\small
% \bibliography{ref}
\begin{thebibliography}{10}

\bibitem{asudeh2019assessing}
A.~Asudeh, Z.~Jin, and H.~Jagadish.
\newblock Assessing and remedying coverage for a given dataset.
\newblock In {\em ICDE}, pages 554--565. IEEE, 2019.

\bibitem{shahbazi2023representation}
N.~Shahbazi, Y.~Lin, A.~Asudeh, and H.~Jagadish.
\newblock Representation bias in data: A survey on identification and resolution techniques.
\newblock {\em ACM Computing Surveys}, 2023.

\bibitem{asudeh2021coverage}
A.~Asudeh, N.~Shahbazi, Z.~Jin, and H.~V. Jagadish.
\newblock Identifying insufficient data coverage for ordinal continuous-valued attributes.
\newblock In {\em SIGMOD}. ACM, 2021.

\bibitem{mousavi2024data}
M.~Mousavi, N.~Shahbazi, and A.~Asudeh.
\newblock Data coverage for detecting representation bias in image datasets: {A} crowdsourcing approach.
\newblock In {\em {EDBT}}, pages 47--60, 2024.

\bibitem{nargesian2021tailoring}
F.~Nargesian, A.~Asudeh, and H.~Jagadish.
\newblock Tailoring data source distributions for fairness-aware data integration.
\newblock {\em Proceedings of the VLDB Endowment}, 14(11):2519--2532, 2021.

\bibitem{nargesian2022responsible}
F.~Nargesian, A.~Asudeh, and H.~V. Jagadish.
\newblock Responsible data integration: Next-generation challenges.
\newblock {\em SIGMOD}, 2022.

\bibitem{sharma2020data}
S.~Sharma, Y.~Zhang, J.~M. R{\'\i}os~Aliaga, D.~Bouneffouf, V.~Muthusamy, and K.~R. Varshney.
\newblock Data augmentation for discrimination prevention and bias disambiguation.
\newblock In {\em AIES}, pages 358--364, 2020.

\bibitem{DBLP:journals/jair/ChawlaBHK02}
N.~V. Chawla, K.~W. Bowyer, L.~O. Hall, and W.~P. Kegelmeyer.
\newblock {SMOTE:} synthetic minority over-sampling technique.
\newblock {\em J. Artif. Intell. Res.}, 16:321--357, 2002.

\bibitem{iosifidis2018dealing}
V.~Iosifidis and E.~Ntoutsi.
\newblock Dealing with bias via data augmentation in supervised learning scenarios.
\newblock {\em Jo Bates Paul D. Clough Robert J{\"a}schke}, 24, 2018.

\bibitem{celis2020data}
L.~E. Celis, V.~Keswani, and N.~Vishnoi.
\newblock Data preprocessing to mitigate bias: A maximum entropy based approach.
\newblock In {\em ICML}, pages 1349--1359. PMLR, 2020.

\bibitem{asudeh2022towards}
A.~Asudeh and F.~Nargesian.
\newblock Towards distribution-aware query answering in data markets.
\newblock {\em Proceedings of the VLDB Endowment}, 15(11):3137--3144, 2022.

\bibitem{motwani1995randomized}
R.~Motwani and P.~Raghavan.
\newblock {\em Randomized algorithms}.
\newblock Cambridge university press, 1995.

\bibitem{chameleon}
M.~Erfanian, H.~V. Jagadish, and A.~Asudeh.
\newblock Chameleon: Foundation models for fairness-aware multi-modal data augmentation to enhance coverage of minorities.
\newblock {\em arXiv preprint arXiv:2402.01071}, 2024.

\bibitem{scholkopf1999support}
B.~Sch{\"o}lkopf, R.~C. Williamson, A.~Smola, J.~Shawe-Taylor, and J.~Platt.
\newblock Support vector method for novelty detection.
\newblock {\em NeurIPS}, 12, 1999.

\bibitem{phillips1998feret}
P.~J. Phillips, H.~Wechsler, J.~Huang, and P.~J. Rauss.
\newblock The feret database and evaluation procedure for face-recognition algorithms.
\newblock {\em Image and vision computing}, 16(5):295--306, 1998.

\bibitem{dressel2018accuracy}
J.~Dressel and H.~Farid.
\newblock The accuracy, fairness, and limits of predicting recidivism.
\newblock {\em Science advances}, 4(1):eaao5580, 2018.

\bibitem{ng2021mlops}
A.~Ng.
\newblock Mlops: From model-centric to data-centric {AI}.
\newblock 2021.

\bibitem{wing2021trustworthy}
J.~M. Wing.
\newblock Trustworthy {AI}.
\newblock {\em CACM}, 64(10):64--71, 2021.

\bibitem{kentour2021analysis}
M.~Kentour and J.~Lu.
\newblock Analysis of trustworthiness in machine learning and deep learning.
\newblock {\em InfoComp}, 2021.

\bibitem{liu2021trustworthy}
H.~Liu, Y.~Wang, W.~Fan, X.~Liu, Y.~Li, S.~Jain, A.~K. Jain, and J.~Tang.
\newblock Trustworthy {AI}: A computational perspective.
\newblock {\em arXiv preprint arXiv:2107.06641}, 2021.

\bibitem{singh2021trustworthy}
R.~Singh, M.~Vatsa, and N.~Ratha.
\newblock Trustworthy {AI}.
\newblock In {\em 8th ACM IKDD CODS and 26th COMAD}, pages 449--453. 2021.

\bibitem{kulynych2022you}
B.~Kulynych, Y.-Y. Yang, Y.~Yu, J.~B{\l}asiok, and P.~Nakkiran.
\newblock What you see is what you get: Distributional generalization for algorithm design in deep learning.
\newblock {\em arXiv preprint arXiv:2204.03230}, 2022.

\bibitem{kakade2003sample}
S.~M. Kakade.
\newblock {\em On the sample complexity of reinforcement learning}.
\newblock University of London, University College London (United Kingdom), 2003.

\bibitem{dwork2012fairness}
C.~Dwork, M.~Hardt, T.~Pitassi, O.~Reingold, and R.~Zemel.
\newblock Fairness through awareness.
\newblock In {\em ITCS}, pages 214--226, 2012.

\bibitem{techrep}
N.~Shahbazi and A.~Asudeh.
\newblock Data-centric reliability evaluation of individual predictions.
\newblock {\em CoRR, abs/2204.07682}, 2022.

\bibitem{adult}
M.~Lichman.
\newblock Adult income dataset, {UCI} machine learning repository.
\newblock \url{https://archive.ics.uci.edu/ml/datasets/adult}, 2013.

\bibitem{neyman1936contributions}
J.~Neyman and E.~S. Pearson.
\newblock Contributions to the theory of testing statistical hypotheses.
\newblock {\em Statistical Research Memoirs}, 1936.

\bibitem{olteanu2019social}
A.~Olteanu, C.~Castillo, F.~Diaz, and E.~Kiciman.
\newblock Social data: Biases, methodological pitfalls, and ethical boundaries.
\newblock {\em Frontiers in Big Data}, 2:13, 2019.

\bibitem{fairmlbook}
S.~Barocas, M.~Hardt, and A.~Narayanan.
\newblock Fairness and machine learning: Limitations and opportunities.
\newblock \url{fairmlbook.org}, 2019.

\bibitem{barocas2016big}
S.~Barocas and A.~D. Selbst.
\newblock Big data's disparate impact.
\newblock {\em Calif. L. Rev.}, 104:671, 2016.

\bibitem{jk2019bias}
J.~Kleinberg.
\newblock Fairness, rankings, and behavioral biases.
\newblock FAT*, 2019.

\bibitem{drosou2017diversity}
M.~Drosou, H.~Jagadish, E.~Pitoura, and J.~Stoyanovich.
\newblock Diversity in big data: A review.
\newblock {\em Big data}, 5(2):73--84, 2017.

\bibitem{berrey2015enigma}
E.~Berrey.
\newblock {\em The enigma of diversity: The language of race and the limits of racial justice}.
\newblock University of Chicago Press, 2015.

\bibitem{dobbin2016diversity}
F.~Dobbin and A.~Kalev.
\newblock Why diversity programs fail and what works better.
\newblock {\em Harvard Business Review}, 94(7-8):52--60, 2016.

\bibitem{simpson1949measurement}
E.~H. Simpson.
\newblock Measurement of diversity.
\newblock {\em Nature}, 163(4148), 1949.

\bibitem{surowiecki2005wisdom}
J.~Surowiecki.
\newblock {\em The wisdom of crowds}.
\newblock Anchor, 2005.

\bibitem{agrawal2009diversifying}
R.~Agrawal, S.~Gollapudi, A.~Halverson, and S.~Ieong.
\newblock Diversifying search results.
\newblock In {\em WSDM}, pages 5--14. ACM, 2009.

\bibitem{mehrabi2021survey}
N.~Mehrabi, F.~Morstatter, N.~Saxena, K.~Lerman, and A.~Galstyan.
\newblock A survey on bias and fairness in machine learning.
\newblock {\em ACM Computing Surveys (CSUR)}, 54(6):1--35, 2021.

\bibitem{friedman1996bias}
B.~Friedman and H.~Nissenbaum.
\newblock Bias in computer systems.
\newblock {\em TOIS}, 14(3):330--347, 1996.

\bibitem{torralba2011unbiased}
A.~Torralba and A.~A. Efros.
\newblock Unbiased look at dataset bias.
\newblock In {\em CVPR 2011}, pages 1521--1528. IEEE, 2011.

\bibitem{crawford2013hidden}
K.~Crawford.
\newblock The hidden biases in big data.
\newblock {\em Harvard business review}, 1(4), 2013.

\bibitem{diakopoulos2015algorithmic}
N.~Diakopoulos.
\newblock Algorithmic accountability: Journalistic investigation of computational power structures.
\newblock {\em Digital journalism}, 3(3):398--415, 2015.

\bibitem{li2020towards}
Y.~Li, H.~Sun, and W.~H. Wang.
\newblock Towards fair truth discovery from biased crowdsourced answers.
\newblock In {\em SIGKDD}, pages 599--607, 2020.

\bibitem{lazier2023fairness}
S.~Lazier, S.~Thirumuruganathan, and H.~Anahideh.
\newblock Fairness and bias in truth discovery algorithms: An experimental analysis.
\newblock {\em arXiv preprint arXiv:2304.12573}, 2023.

\bibitem{SalimiRHS19}
B.~Salimi, L.~Rodriguez, B.~Howe, and D.~Suciu.
\newblock Interventional fairness: Causal database repair for algorithmic fairness.
\newblock In {\em {SIGMOD}}, pages 793--810. {ACM}, 2019.

\bibitem{tae2019data}
K.~H. Tae, Y.~Roh, Y.~H. Oh, H.~Kim, and S.~E. Whang.
\newblock Data cleaning for accurate, fair, and robust models: A big data-{AI} integration approach.
\newblock In {\em DEEM workshop}, pages 1--4, 2019.

\bibitem{salimi2020database}
B.~Salimi, B.~Howe, and D.~Suciu.
\newblock Database repair meets algorithmic fairness.
\newblock {\em ACM SIGMOD Record}, 49(1):34--41, 2020.

\bibitem{martinez2019fairness}
F.~Mart{\'\i}nez-Plumed, C.~Ferri, D.~Nieves, and J.~Hern{\'a}ndez-Orallo.
\newblock Fairness and missing values.
\newblock {\em arXiv preprint arXiv:1905.12728}, 2019.

\bibitem{shahbazi2023through}
N.~Shahbazi, N.~Danevski, F.~Nargesian, A.~Asudeh, and D.~Srivastava.
\newblock Through the fairness lens: Experimental analysis and evaluation of entity matching.
\newblock {\em Proceedings of the VLDB Endowment}, 16(11):3279--3292, 2023.

\bibitem{fanourakis2023fairer}
N.~Fanourakis, C.~Kontousias, V.~Efthymiou, V.~Christophides, and D.~Plexousakis.
\newblock Fairer demo: Fairness-aware and explainable entity resolution.
\newblock 2023.

\bibitem{lin2020identifying}
Y.~Lin, Y.~Guan, A.~Asudeh, and H.~Jagadish.
\newblock Identifying insufficient data coverage in databases with multiple relations.
\newblock {\em Proceedings of the VLDB Endowment}, 13(12):2229--2242, 2020.

\bibitem{accinelli2020coverage}
C.~Accinelli, S.~Minisi, and B.~Catania.
\newblock Coverage-based rewriting for data preparation.
\newblock In {\em EDBT Workshops}, 2020.

\bibitem{accinelli2021impact}
C.~Accinelli, B.~Catania, G.~Guerrini, and S.~Minisi.
\newblock The impact of rewriting on coverage constraint satisfaction.
\newblock In {\em EDBT Workshops}, 2021.

\bibitem{shetiya2022fairness}
S.~Shetiya, I.~P. Swift, A.~Asudeh, and G.~Das.
\newblock Fairness-aware range queries for selecting unbiased data.
\newblock In {\em ICDE}. IEEE, 2022.

\bibitem{tae2021slice}
K.~H. Tae and S.~E. Whang.
\newblock Slice tuner: A selective data acquisition framework for accurate and fair machine learning models.
\newblock In {\em SIGMOD}, pages 1771--1783, 2021.

\bibitem{chung2019slice}
Y.~Chung, T.~Kraska, N.~Polyzotis, K.~H. Tae, and S.~E. Whang.
\newblock Slice finder: Automated data slicing for model validation.
\newblock In {\em ICDE}, pages 1550--1553. IEEE, 2019.

\bibitem{sagadeeva2021sliceline}
S.~Sagadeeva and M.~Boehm.
\newblock Sliceline: Fast, linear-algebra-based slice finding for ml model debugging.
\newblock In {\em SIGMOD}, pages 2290--2299, 2021.

\bibitem{kleinberg2016inherent}
J.~Kleinberg, S.~Mullainathan, and M.~Raghavan.
\newblock Inherent trade-offs in the fair determination of risk scores.
\newblock {\em arXiv preprint arXiv:1609.05807}, 2016.

\bibitem{zadrozny2001obtaining}
B.~Zadrozny and C.~Elkan.
\newblock Obtaining calibrated probability estimates from decision trees and naive bayesian classifiers.
\newblock In {\em ICML}, volume~1, pages 609--616. Citeseer, 2001.

\bibitem{zadrozny2002transforming}
B.~Zadrozny and C.~Elkan.
\newblock Transforming classifier scores into accurate multiclass probability estimates.
\newblock In {\em SIGKDD}, pages 694--699, 2002.

\bibitem{platt1999probabilistic}
J.~Platt et~al.
\newblock Probabilistic outputs for support vector machines and comparisons to regularized likelihood methods.
\newblock {\em Advances in large margin classifiers}, 10(3):61--74, 1999.

\bibitem{niculescu2005predicting}
A.~Niculescu-Mizil and R.~Caruana.
\newblock Predicting good probabilities with supervised learning.
\newblock In {\em Proceedings of the 22nd international conference on Machine learning}, pages 625--632, 2005.

\bibitem{chatfield93predictionintervals}
C.~Chatfield.
\newblock Prediction intervals.
\newblock {\em Journal of Business and Economic Statistics}, 11:121--135, 1993.

\bibitem{pearce2018high}
T.~Pearce, A.~Brintrup, M.~Zaki, and A.~Neely.
\newblock High-quality prediction intervals for deep learning: A distribution-free, ensembled approach.
\newblock In {\em International conference on machine learning}, pages 4075--4084. PMLR, 2018.

\bibitem{khosravi2010lower}
A.~Khosravi, S.~Nahavandi, D.~Creighton, and A.~F. Atiya.
\newblock Lower upper bound estimation method for construction of neural network-based prediction intervals.
\newblock {\em IEEE transactions on neural networks}, 22(3):337--346, 2010.

\bibitem{angelopoulos2021gentle}
A.~N. Angelopoulos and S.~Bates.
\newblock A gentle introduction to conformal prediction and distribution-free uncertainty quantification.
\newblock {\em arXiv preprint arXiv:2107.07511}, 2021.

\bibitem{shafer2008tutorial}
G.~Shafer and V.~Vovk.
\newblock A tutorial on conformal prediction.
\newblock {\em Journal of Machine Learning Research}, 9(3), 2008.

\bibitem{harradon2018causal}
M.~Harradon, J.~Druce, and B.~Ruttenberg.
\newblock Causal learning and explanation of deep neural networks via autoencoded activations.
\newblock {\em arXiv preprint arXiv:1802.00541}, 2018.

\bibitem{ribeiro2016should}
M.~T. Ribeiro, S.~Singh, and C.~Guestrin.
\newblock " why should i trust you?" explaining the predictions of any classifier.
\newblock In {\em SIGKDD}, pages 1135--1144, 2016.

\bibitem{gunning2019darpa}
D.~Gunning and D.~Aha.
\newblock Darpa’s explainable artificial intelligence ({XAI}) program.
\newblock {\em AI Magazine}, 40(2):44--58, 2019.

\end{thebibliography}

\end{document}

\end{article}


\begin{article}
{Automated Data Validation in Machine Learning Systems}
{Felix Biessmann, Jacek Golebiowski, Tammo Rukat, Dustin Lange and Philipp Schmidt}
\documentclass[11pt]{article}

\usepackage{deauthor}
\usepackage{times, enumerate}
\usepackage{graphicx}
\usepackage{authblk}
\usepackage{url}
\def\UrlBreaks{\do\/\do-}
\usepackage{breakurl}
%\usepackage[breaklinks]{hyperref}
\usepackage{wrapfig}
\usepackage{enumitem}

\graphicspath{{figs/}}

%\usepackage[usenames,dvipsnames,svgnames]{xcolor}

\newcommand{\hdr}[1]{\noindent\textbf{#1.}}
\newcommand{\todo}[1]{\textcolor{purple}{\textbf{[#1]}}}
\newcommand{\tammo}[1]{\textcolor{blue}{\textbf{#1}}}
\newcommand{\felix}[1]{\textcolor{green}{\textbf{[#1]}}}
\newcommand{\jacek}[1]{\textcolor{magenta}{\textbf{[#1]}}}


\begin{document}

\title{Automated Data Validation in Machine Learning Systems}
%
\author{Felix Biessmann, Jacek Golebiowski, Tammo Rukat, Dustin Lange, Philipp Schmidt\\
{\small \texttt{\{biessman,jacekgo,tammruka,langed,phschmid\}@amazon.com}}\\
Amazon}

\maketitle

\section*{Abstract}
This article presents the \method system for question answering over unstructured text, structured tables, and knowledge graphs, with unified treatment of all sources.
The system adopts a RAG-based architecture, with a pipeline of evidence retrieval followed by answer generation, with the latter powered by a 
moderate-sized
language model.
Additionally and uniquely, \method
has components for question understanding, to derive crisper input for evidence retrieval, and for re-ranking and filtering the retrieved evidence before feeding the most informative pieces into the answer generation.
Experiments with three different benchmarks demonstrate the high answering quality of our approach, being on par with or better than large GPT models, while keeping the computational cost and energy consumption orders of magnitude lower.

% !TEX root = ../main.tex
\section{The Need for a Data-Centric Perspective on Responsible~AI}
\label{sec:intro}

Software systems that learn from data with AI and machine learning (ML) are becoming ubiquitous and are increasingly used to automate impactful decisions. The risks arising from this widespread use are garnering attention from policymakers, scientists, and the media, and lead to the question of what data management research can contribute to reduce the dangers and malfunctions of data-driven AI/ML applications. 

\header{AI/ML malfunctions threaten vulnerable populations} In recent years, we have been regularly alarmed by media reports about the harm potential of faulty AI/ML systems in devastating real-world incidents. Examples include failures of automated decision-making systems, e.g., an eight-month pregnant woman in Detroit was mistakenly arrested based on a faulty prediction from a facial recognition system, held in jail for several hours and needed medical care upon her release~\cite{aiface2023}. Another example is that one of the largest health insurers in the US allegedly applies a faulty AI model with a 90\% error rate to deny critical health care services to elderly patients~\cite{aihealth2023}. The recent rise of generative AI produces new types of harm as well. A recent study of AI detection tools, for example, found that these systems are biased against non-native English speakers~\cite{aicheating2023} and often falsely accuse international students of cheating. Furthermore, an AI supermarket meal planner recently went rogue and suggested a recipe that would create chlorine gas~\cite{airecipe2023}.


\header{Technical bias in ML applications} The reasons that data-driven systems are susceptible to producing unfair, harmful outcomes are multi-faceted~\cite{stoyanovich2022responsible,whang2021responsible,groth2013transparency}, as we are ultimately dealing with socio-technological systems~\cite{birhane2021large}, which suffer from various types of bias~\cite{friedman1996bias}. In this work, we focus on \textit{technical bias}, which arises from the design decisions and operations in a technical system itself. Such bias is not well understood, especially in the context of large end-to-end systems, which include data preparation and data cleaning stages, deployed models and feedback loops. Recent research on technical bias identifies issues such as the lack of sufficient, representative training data for certain demographic groups~\cite{lin2020identifying,asudeh2019assessing,chen2018my}, biased training data with undesirable stereotypes~\cite{birhane2021multimodal} or unintended side effects from automated data cleaning operations~\cite{guha2024automated,tae2019data,shahbazi2023through}.


\header{Existing and upcoming regulation} The dangers arising from data-driven AI/ML applications have been recognised by regulators and lawmakers several years ago already, and led to the introduction of regulation all over the world. The ``General Data Protection Regulation'' (GPDR) in Europe, for example, grants citizens the right to find out what information an organisation has about them and to issue deletion requests for their data as part of the ``right-to-be-forgotten''~\cite{GDPRart17,GDPRrec74}. The upcoming European AI Act~\cite{euaiact} will be the first comprehensive regulation for the application of AI/ML in Europe. This act is expected to outlaw the usage of ML in selected application areas and to strongly regulate its application in certain other areas. It defines different levels of risk in AI usage scenarios and imposes a set of comprehensive technical requirements, such as ``logging of activity to ensure traceability of results'', ``detailed documentation providing all information necessary on the system and its purpose for authorities to assess its compliance'', and ``appropriate human oversight measures to minimize risk''. We note that outside Europe, similar regulations are being adopted~\cite{cppa,digichina}.

\header{The need for a data-centric perspective} Unfortunately, as evidenced by the media reports cited previously, we currently lack the ability to efficiently implement technical measures to detect and mitigate the harms present in AI/ML applications. This is confirmed by a recent survey study with industry practitioners~\cite{holstein2019improving}, which outlines several alarming shortcomings in addressing fairness and bias issues. The interviewed practitioners report that academic research on de-biasing models falls short of addressing their concerns and often falsely ``view[s] the training data as fixed'', while they ``consider data collection, rather than model development, as the most important place to intervene''. At the same time, only ``65\% of survey respondents [...] reported that their teams have some control over data collection and curation'', and the study also finds a high demand for future research to ``support [...] practitioners in [...] curating high-quality datasets''. Another example of the dire situation in the industry is a recent court case against Facebook~\cite{facebookdata}, where two veteran engineers of the company told the court that the company does not keep track of the exact locations where personal data is stored and processed.

In the research community, several widely used training datasets for computer vision, such as LAION-5B~\cite{schuhmann2022laion} or TinyImages~\cite{torralba200880}, have been taken offline after the discovery of highly problematic content in them~\cite{birhane2021large,birhane2023hate}. Moreover, it is unlikely, though, that all models that had been trained on these problematic datasets have been retracted as well. For the current wave of closed, proprietary pretrained models available behind commercial APIs, the situation is even worse, as we do not even have a way to determine what data they have been trained on.

\header{Paper inspiration} In order to find inspiration for the outlined questions and challenges, we take a look into safety measures outside of the computer science domain, as our societies have had to deal with the dangers of complex and distributed technical processes for a long time already. In particular, we discuss how the U.S. Food and Drug Administration (FDA) combats the outbreaks of foodborne illnesses (\Cref{sec:inspiration}). We ask ourselves what we can learn from the millennial pursuit of food safety. What type of technical and regulatory frameworks exist such that we trust what we put on the table for our family every day? We use the FDA's established processes as an inspiration for a data-centric vision towards responsible~AI in \Cref{sec:vision}, with the goal to obtain the same level of trust for our data products that we have for our food.







%!TEX root = data-validation-ml-systems.tex
\section{Data Validation Dimensions}
\label{sec:dimensions}

As ML systems learn from data, validation of the data ingested during training and prediction is a fundamental prerequisite for well functioning and robust ML systems \cite{Sculley2015, Bose2017a}. Relevant data validation dimensions for ML systems can be broadly categorized into those notions of data quality commonly considered in classical data base management systems (DBMS) and ML model dependent dimensions. In the following, we highlight some of the dimensions and provide examples for each category.

\subsection{Classical DBMS dimensions}

Independent of ML applications there are many data validation dimensions that have been investigated in the DBMS community, more specifically in the data profiling literature\cite{Abedjan2018}:

\paragraph{Data Correctness:} This dimension refers to schema violations, such as string values in numeric columns, or generally syntactically wrong data. Defects in this dimension can often break ML processing pipelines. Some of these violations can be detected easily for instance by type checks, and corrupted data instances can be filtered out.


\paragraph{Data Consistency} refers to defects that are not captured by syntactic correctness, including duplicate entries or invalid values. Detecting these cases can be difficult and computationally challenging, but there exist efficient approaches to de-duplication and detection of semantic inconsistencies. Violations of semantic inconsistencies can for instance be detected by validation of functional or approximate functional dependencies \cite{Papenbrock2015, Kruse2018}.

\paragraph{Completeness} of a data source reflects the ratio of missing values in a data set. In principle this dimension is simple to probe, however only under the assumption that the missing value symbol is known. Unfortunately this assumption is often violated. As missing values often cause problems in data pipelines feeding into ML systems, a common strategy of avoid such problems is to replace missing values with syntactically valid values. These transformations often make the missing values pass validations for correctness and consistency.

\newpage
\paragraph{Statistical Properties:} This dimension includes basic descriptive statistics of an attribute, such as mean of a numeric or mode of a categorical variable, and more complex parametric or non-parametric statistical aggregates of a data set. A popular validation strategy in this data dimension is anomaly detection.

\subsection{ML model dependent dimensions}

Validating the above data quality dimensions can be challenging, but these tests of the input data are independent of the downstream ML model. This implies that testing only for these dimensions can lead to both false negatives, when a data corruption is not detected that has a negative impact on the downstream ML model, as well as false positives, when a data corruption raises an alarm but does not have any effect on the downstream ML model. Examples for false negatives of ML model independent validations include adversarial attacks, such as for instance avoiding filters for adult language by {\em leet speak}\footnote{\url{https://en.wikipedia.org/wiki/Leet}} strings obfuscated by misspellings and replacing letters with numeric characters. Examples for false positive alarms are ML models employing strong $L_1$ or sparsity-enforcing regularization constraints. In these models, a large number of input features is often not affecting the ML model output, hence distributional shifts in these input features would not be reflected in shifts of the ML model outputs and validation of the input data independent of the model outputs will raise false alarms.

Data validation in the context of ML production systems thus has to take into account the current state of the downstream ML model to ensure robust and reliable performance. The state of the ML model and and the impact of data corruptions on ML performance can be monitored by a number of different validation dimensions of the ML model {\em output} data. These have to be validated along with the usual criteria common to all production software systems, such as requirements on the latency and availability of the system. SageMaker Model Monitor\footnote{\url{https://docs.aws.amazon.com/sagemaker/latest/dg/model-monitor.html}} is a solution that enables monitoring of models deployed in production. For example, alarms are triggered when data distributions shifts are detected.

\paragraph{Predictive performance metrics} are usually the most important metrics optimized in production ML systems. These metrics include accuracy, precision or F1-score of a classifier, the mean-squared error, mean absolute error or $r^2$-scores for regression models or various ranking losses, evaluated via cross-validation on a test set that was not used for training the respective ML model. When the data processed by  trained and deployed ML model is drawn from the same distribution as the training and test data, then these cross-validated metrics reflect the predictive performance reliably. But in production ML systems, this assumption is often violated. Examples include shifts in the input data (covariate shifts) or shifts in the target data distribution (label shift).

\paragraph{Robustness}
The shifts induced by corruptions in the data that are a) not caught by upstream classical data validation techniques and that have b) a negative impact on the predictive performance of a production ML system can be difficult to detect, as there is no ground truth label information available for the data. In the absence of ground truth data needed to compute predictive performance metrics, one can resort to classical data validation techniques applied to the outputs of ML models. But this approach can fail to detect certain covariate shifts induced by outliers, adversarial attacks or other corruptions.

\paragraph{Privacy metrics} have become increasingly important since ML models are used to process personal data such as health care data, financial transactions, movement patterns or really almost every aspect of our lives. Popular concepts include {\em differential privacy} \cite{Dworak_2006_diff} and {\em k-anonymity} \cite{Sweeney2002}. Note that k-anonymization~\cite{Sweeney2002} is not model dependent, it is a property of a data base where user data is considered private if information about each user cannot be distinguished from at least k-1 other users whose information is in the dataset.
In the field of ML, differential privacy is more useful than k-anonymity. Two main reasons for that are a) k-anonymity is defined for a fixed data set, if new rows are added, one might need to convert the data again in order to achieve k-anonymity, and b) if a k-anonymous data set is combined with other information, it is often possible to de-anonymize the data.
%
A large body of literature in the ML community builds upon the definition of $\epsilon$-differential privacy~\cite{Dworak2014differantial} which deals with the problem of learning little to nothing about individual data-points while learning a lot about the population by ensuring the same conclusions can be drawn whether any single data-point is included in the dataset or not. Most implementations of differentially private ML models add noise to the data, gradients or the model coefficients~\cite{Abadi2016, Shokri2017, papernot2016semisupervised, Mironov2017}. This can be an effective countermeasure to model inversion attacks~\cite{Fredrikson2015} or membership inference attacks \cite{Shokri2017}. The amount of noise added determines the amount of privacy preserved -- but it also bounds the predictive performance or {\em utility} of the ML model \cite{Jayaraman2019}.

\paragraph{Robustness to adversarial data} Recent developments have shown that ML models, placed at the centre of security-critical systems such as self-driving cars, can be targeted by dedicated adversarial examples - inputs that are very difficult to distinguish from regular data but are misclassified by the model~\cite{goodfellow2014explaining}. Those inputs, created by applying a small but very significant perturbation to regular examples can be used to manipulate the model to predict outputs desired by an adversary~\cite{szegedy2013intriguing, Biggio_2013, Nguyen_2015}. The robustness to adversarial attacks of the model can be measured by (1) generating a set of adversarial examples based on the test set~\cite{szegedy2013intriguing, Kurakin2017AdversarialEI, Hosseini2019AreOR, MoosaviDezfooli2016DeepFoolAS, goodfellow2014explaining} and (2) computing what is the change in models accuracy given a fixed average distortion, or what is the minimum average distortion needed to reach 0\% accuracy on the test set.


\paragraph{Fairness}
Another dimension that is highly relevant when putting ML systems in production is that of fairness or more generally ethical aspects of ML predictions \cite{Mehrabi2019}. There are a number of examples when ML technology used as assistive technology for instance in border control\footnote{\url{https://www.bbc.com/news/technology-53650758}} or policing\footnote{\url{https://www.nytimes.com/2019/07/10/opinion/facial-recognition-race.html}} led to racist or otherwise biased decisions. It has been widely recognized in the ML research community that when building ML applications validating the ML predictions with respect to fairness is a fundamental prerequisite. Evaluating this dimension is however amongst the most difficult of all validation  challenges: there are different fairness definitions, and even if one could agree on a small set of definitions, validating these requires insight into the respective grouping variables, which are, for ethical and legal reasons, typically not available. The relevant grouping information is thus often reflected in a data set through other features, which are considered not ethically problematic, such as zip code, but which are often correlated with ethically sensitive features. Other implicit biases of data sets can emerge from data preprocessing, for instance how missing values are dealt with when they are distributed not equally across ethnical groups \cite{Yang2020}. Detecting these biases requires not only domain expertise, but also often information that is not contained in the data set itself.



%
%\begin{enumerate\
%
%\item Classical DBMS dimensions
%\begin{enumerate}
%\item Correctness (e.g. Unknown schemas / data types, Wrong entries)
%\item Consistency (e.g. Duplicates, Invalid values)
%\item Statistical properties (Anomalies, Are value distributions changing over time? Is a column predictable from another column?)
%\item Completeness (How many missing values?)
%\end{enumerate}

%\item ML model dependent dimensions
%\begin{enumerate}
%\item Generalization performance
%\item Performance under (covariate, label) shift
%\item Performance for single (rare) classes
%\item Performance for single subgroups ({\bf fairness})
%\item Robustness to adversarial data
%\item Differential privacy
%\item Constraints on compute time (some data points could lead to long prediction times, e.g. due to lack of sparsity)
%\end{enumerate}
%\end{enumerate}

%!TEX root = data-validation-ml-systems.tex
\section{Current Solutions}
\label{sec:solutions}

Validating data at each of the stages in a ML workflow as sketched in \autoref{fig:ml-system} requires a wide range of competencies from data engineering over ML algorithms to user interface design. These competencies are often distributed across different roles in a team, and especially in small (5 to 10 members) teams, it is not unlikely that only one person has the competency to dive deep into a root cause analysis of the various data validation dimensions sketched in \autoref{sec:dimensions}. Solving problems in an on-call situation or when scaling up an ML system requires reliable automation for the data validation. As many of these validations have to be data dependent, such automation often amounts to {\em using ML to validate data in ML workflows} which has, for instance, been done for outlier removal \cite{Zhao2019}, data cleaning \cite{Krishnan2019} or missing value imputation \cite{Biessmann2018a}. 

\newpage
There are, however, simpler approaches that do not leverage ML models to validate data in ML systems, such as Tensorflow Extended (TFX) \cite{Baylor2017} or DEEQU \cite{Schelter2018}. In the following we will highlight some of the state of the art solutions to automated data validation in the context of ML systems.


\subsection{Schema Validation}

Validating syntactic correctness, also referred to as schema validation, is -- at first glance -- relatively simple and often integral part of data ingestion. Usually data violating data type constraints will immediately raise errors and break a data pipeline. Yet, these validations are often counteracted by humans. In practice it is not uncommon to use generic data types such as strings for data that is actually numeric, ordinal or categorical. While this will ensure that schema violations are minimized, as any data type can be cast to string, it imposes additional work on any downstream validation for data correctness and consistency. This problem is worsened by the fact that some software libraries popular in the ML community often cast string data that contains numeric data automatically to numeric data types. This can have effects on entire scientific communities. A recent study reports that casting errors induced by excel in biomedical data occur in approximately one fifth of the papers in leading genomics journals \cite{Ziemann2016}.

While simple validation such the data type can be trivially automated, it is unlikely that this will prevent engineers from having to deal with data type errors. There will always be on-call engineers who need to fix problems as fast as possible and the cost of having stale or broken data pipelines in a production system is often far bigger then the difficulty to quantify cost of accumulating hidden technical debt by specifying generic data types like strings and thereby circumventing data type checks. This is why this simple task of data type detection is actually an interesting research challenge. Often simple heuristics will do the job and the popular libraries use these heuristics to determine the data type of columns. The simplest heuristic is probably to try and cast a value to a certain data type and try other data types in case an error is raised\footnote{\url{https://github.com/pandas-dev/pandas/blob/v1.1.4/pandas/core/dtypes/common.py\#L138}}. Such heuristics often work well in practice but can be insufficient, especially when dealing with heterogeneous data, such as mixed data types in one column. Also, for the seemingly simple task of automating data type inference, there is interesting recent work on using ML models to infer a probabilistic notion of the most likely data type given the symbols used in a data base~\cite{Ceritli2020}. Approaches combining scalable heuristics with more sophisticated ML based data type inference techniques are a promising alternative to the current situation in which restrictive data type checks and broken data pipelines often lead data engineers to opt for too generic data types.

\subsection{Data Correctness and Data Consistency}

Validation of data consistency and correctness are one of the areas where classical DBMS research has made significant contributions, for instance in the data profiling literature \cite{Abedjan2018}. Popular approaches to validate correctness and consistency are rule based systems such as {\em NADEEF} \cite{Dallachiesat2013}. There are commercial versions of simple rule based systems, a popular example is trifacta\footnote{\url{https://www.trifacta.com/}}. And there are open source systems for semi-automated validation of consistency, such as Open-Refine\footnote{\url{https://openrefine.org/}}. Other approaches primarily investigated in an academic context include attempts to learn {\em (approximate/relaxed) functional dependencies} between columns of a table \cite{Papenbrock2015}. These functional dependencies between columns include for example a dependency between a column containing zip codes and another column containing city names. These dependencies can be used to automatically learn validation checks related to consistency and correctness \cite{Rekatsinas2017}. Such approaches are well established and there exist efficient methods for large scale functional dependency discovery \cite{Kruse2018}. Yet, in the context of data pipelines for ML systems none of these approaches have reached broad adoption. We discuss some potential reasons in \autoref{sec:conclusion}.

In addition to the solutions of the DBMS community there are a number of approaches developed at the intersection of DBMS and ML  communities, tailored to ML specific workflows \cite{Breck2019, Schelter2018}. The research is accompanied by open source libraries\footnote{\url{https://www.tensorflow.org/tfx/guide/tfdv}}\footnote{\url{https://github.com/awslabs/deequ}} that implement schema validation by checking for correct data types, but they also offer validation options for or other basic data properties, for instance whether the values are contained in a predefined set. Prominent examples for a data validation solution for ML systems are the ones integrated in SageMaker Data Wrangler\footnote{\url{https://aws.amazon.com/sagemaker/data-wrangler/}} and TFX \cite{Breck2019}. Their schema validation conveniently allows users to define integrity checks on the serving data in an ML system, e.g. for data type or value ranges. Schema validation parameters can also be derived from the training data, which allows to automate many of these checks.

\subsection{Validation of Statistical Properties}

All data sets are subject to noise, be it because humans enter wrong values into a form or because human software developers write code that leads to faulty data ingestion. Dealing with this noise is difficult for exact validation rules such as functional dependencies. Statistical learning techniques can help to deal with noisy data.

\paragraph{Error-Detection Solutions}
Taking a probabilistic stance on DBMSs is not a new idea \cite{Cavallo1987}. Recently, the idea of using statistical techniques for data validation tasks has become a major focus of research. One example are statistical learning approaches to error detection in data bases \cite{Rekatsinas2017,Heidari2019,Huang2017,MahdaviBerlinmahdavilahijani}. These approaches have the potential to automatically validate data correctness and to automatically clean data sets. Some of these systems can be automated to a large extent \cite{Krishnan2017, Krishnan2019}, others rely on a  semi-supervised approach \cite{Krishnan:2016b}.

\paragraph{ML Frameworks for Statistical Data Validation}
Also data validation solutions like TFX data validation \cite{Breck2019} or DEEQU \cite{Schelter2018} address data validation including statistical rules, such as deviation of values around the mean of a numeric column. These validation solutions can be very helpful, if one knows what to test for. But data sources can easily have billions of rows and hundreds of columns. For these cases it can infeasible to manually create, validate and maintain data quality checks. To alleviate the burden of manually creating constraints, the authors of DEEQU \cite{Schelter2018} propose to utilize historic data to generate column profiles and generate data quality constraints from these. These quality constraints can also make use of the temporal structure of data quality metrics collected over time using time series models or other anomaly detection methods.

\paragraph{Anomaly Detection Methods}
Instead of applying them to metrics computed on a data set, anomaly detection methods are also often used to detect anomalies in the data tuples directly. There are many methods available in easy to use software libraries, see for instance \cite{Zhao2019}, and there are commercial products that allow to automate anomaly detection in large scale industrial settings\footnote{\url{https://aws.amazon.com/de/blogs/big-data/real-time-clickstream-anomaly-detection-with-amazon-kinesis-analytics/}}. While the goal of these anomaly detection approaches is similar to the above error detection approaches originating in the DBMS community, many anomaly detection solutions emerged from the ML community. One of the most important differences is that anomaly detection approaches, as most ML methods, usually expect the data to come in matrix form, with all columns being numeric. Most methods from the DBMS community expect the data to be high cardinality categorical data, some also are applicable to numeric values, but that is not very common in research on functional dependencies for instance \cite{Papenbrock2015}. So applying anomaly detection methods to heterogeneous data sets with non-numeric (categorical, ordinal or even free text data) requires to apply {\em feature extraction} techniques to bring the data into a numeric format. This is a standard preprocessing step in ML pipelines, but popular software libraries for anomaly detection, such as  \cite{Zhao2019}, do not include this important preprocessing step. This makes these libraries difficult to apply for automated data validation. It is possible to integrate standard anomaly detection methods as statistical data validation steps in ML systems, but this imposes two additional challenges onto the engineering team. For one, this integration requires to write 'glue code' for the feature extraction -- which is often one of the major sources for accumulating {\em technical debt} in a software system. And secondly this requires to have a good evaluation metric for the anomaly detection. Which is, in contrast to standard supervised learning scenarios, often difficult to define and get ground truth data for.

\paragraph{Model Robustness and Generalization Performance}
Another central problem with all of the above approaches to statistical data validation in a ML context is that most of these methods are blind to the impact of data errors on downstream ML components. This is however one of the most important aspects for ML systems. Often it does not matter for the ML model, whether a feature is affected by a data shift, for instance when regularization of an ML model forces the ML model to ignore a feature. And in other cases tiny changes in the input, which human observers would not be able to detect, can have devastating consequences on the ML model output \cite{Athalye18}. Recent work in the ML community has shown that especially novel deep learning methods can suffer from severe stability problems \cite{DAmour2020}. Some aspects of this can be mitigated by employing standard ML concepts such as k-fold cross-validation (CV). This approach has unfortunately lost popularity due to the sheer compute demand of modern deep learning methods. Most deep learning papers usually use just a single train/validation/test split. Standard nested k-fold CV can have decisive advantages when it comes to measuring robustness of a ML model. However, these techniques only work when there is ground truth data available. In a production setting, this is often not the case. There exist however also other methods to measure the robustness of ML models when there is no ground truth data available. For instance in \cite{Schelter2020} the authors leverage a set of declaratively defined data errors applied to data for which ground truth is available and measure the predictive performance of a ML model under these perturbations. This allows to train a meta model that can be used to predict the predictive performance on new unseen data with high precision. Such an approach can be useful in production ML systems to automatically validate data errors with respect to their impact on downstream ML model performance.

\subsection{Fairness Validation}

Fairness is one of the most prominent examples of how ML systems can fail and severely influence the public opinion about a company or an entire scientific community. It is thus of utmost importance to ensure that this dimension of data validation in the context of ML systems is not neglected. Yet validating this dimension is especially difficult, for a number of reasons. First and foremost it is difficult to define fairness. An excellent overview over the last decades of fairness research with a focus on ML systems can be found in \cite{Mehrabi2019}. The main insight here is that fairness validation it is not only a technical challenge. Instead, it is imperative to include multiple scientific disciplines in this research, in particular also researchers from sociology, psychology and law. Setting the stage for such transdisciplinary research is a challenge in itself, for instance finding a common terminology is not trivial. But we have seen that the research community has made progress by fostering transdisciplinary discussions at newly emerging conferences\footnote{See for instance \url{https://dl.acm.org/conference/fat}}. The joint efforts of different communities have helped to identify many ways in which biases leading to unfair ML systems can enter a workflow. Many of these biases arise in the data generating process. Enabling scientists and engineers to identify such biases should be part of the research agenda of the data management and ML community \cite{Stoyanovich2020}. One example in this direction is {\em FairBench}~\cite{Yang2020}, an open source library that helps to trace changes in data distributions and visualize distortions with respect to protected group memberships throughout the pipeline. Another example is SageMaker Clarify\footnote{https://aws.amazon.com/sagemaker/clarify/}, an explainability feature for Amazon SageMaker that provides insights into data and ML models by identifying biases and explaining predictions. It is deeply integrated into Amazon SageMaker, a fully managed service that enables data scientists and developers to build, train, and deploy ML models at any scale. Clarify supports bias detection and feature importance computation across the ML lifecycle, during data preparation, model evaluation, and post-deployment monitoring. Libraries like these  are a prerequisite to better automation of fairness validation.
%
However, another more fundamental problem of fairness validation remains even if technical solutions for visualizing and detecting biases are available: Group membership variables are required for most technical solutions to fairness. Storing these variables can
be more challenging, from an operational stance, than storing other non-confidential data.


\subsection{Privacy Validation}
Similar to fairness, also privacy is a difficult to define validation dimension. However in contrast to fairness, the research community has converged to a few concepts that are relatively straightforward in terms of their definitions. Popular concepts include {\em differential privacy} \cite{Dworak_2006_diff} and {\em k-anonymity} \cite{Sweeney2002}, see also \autoref{sec:dimensions}.
Most ML related work on privacy focuses on differential privacy, where noise is added the data, gradients or the model coefficients. Validating and trading off privacy against predictive performance or {\em utility} of the ML model can be challenging \cite{Jayaraman2019}.
Empirically, evaluating privacy is often done using {\em membership inference attacks} \cite{Shokri2017}, which has also been adopted for unsupervised ML models \cite{Hayes2019}. One limitation of these approaches is that privacy validation is always dependent on a specific model and data set. General statements about privacy and utility independent of models and data is hence difficult.

\subsection{Validation of Robustness against adversarial attacks}

Privacy validation is aiming at defending the ML system against a certain type of adversarial attack, where for instance the identity of data points used for training the ML system is revealed. There are however other types of adversarial attacks, for instance when artificial examples are generated to result in ML predictions with a certain outcome. Validation of robustness against such types of attacks can be achieved by perturbations around the data manifold~\cite{pang2017robust, aleks2017deep}. This can be achieved by extracting latent representations of the input data~\cite{hendrycks2016early} or of the predictions~\cite{Bhagoji_2018, feinman2017detecting}. Alternative methods rely on training an additional classifier used to decide whether an example is adversarial or not~\cite{gong2017adversarial, grosse2017statistical, metzen2017detecting}. Complementary to the work on validating adversarial robustness, a lot of work has been devoted to making ML models more robust to adversarial attacks by augmenting the training datasets with adversarial examples~\cite{goodfellow2014explaining,aleks2017deep, zhang2019, Dingyuan2019}.


\subsection{Human in the loop evaluation}

Most of the above data quality dimensions are easy for humans to assess. This is why human audits are still one of the most direct and most robust options for data validation in ML systems. Expert auditors, such as researchers developing a new service, often can quickly identify errors and their root causes by simply inspecting input and outputs of a ML system. Among the most important disadvantages with this approach is that these validations are expensive and do not scale well. Sometimes human-in-the-loop validations can be scaled up using crowd-sourcing platforms such as Yandex' Toloka or Amazon Mechanical Turk. Increasing the quality of crowd-sourced validations is an active topic of ML research \cite{WortmanVaughan2018}. For instance there are attempts to render audits more efficient by including transparency of ML model predictions \cite{schmidt2019quantifying} or by providing more inciting incentives~\cite{Ho_2015, wang_2018}. Still, this approach can be difficult to automate and is generally used as an andon cord in a modelling pipeline rather than an integrated quality test. This is not only due to the fact that it is difficult to integrate human audits in build processes. Focussing on human judgements only can lead to biased validations, especially when using transparent ML methods~\cite{schmidt2020risk}.

%!TEX root = data-validation-ml-systems.tex
\section{Conclusion}
\label{sec:conclusion}

Validation of input and output data in ML production systems has many facets that require competencies often distributed across a heterogeneous team of engineers and scientists, as illustrated in \autoref{fig:ml-system}. While some of the data validation challenges, such as schema validation or data consistency, can be tackled with traditional data profiling methods established the DBMS community, other validation dimensions are specific to ML systems. These ML model dependent validation challenges include aspects like accuracy and robustness under data shifts, fairness of ML model predictions and privacy concerns. 

In \autoref{sec:solutions} we highlight a number of solutions to validate single aspects. Many of these approaches are typically tailored to specific use cases and often require considerable integration efforts in production ML systems. 
A good example are the various solutions from both the ML community as well as the DMBS community for checking simple data properties, such as the data types, and also more complex dimensions like data consistency. Many of these approached allow for automating the generation of validation checks. Yet in practice it is not trivial to automate the generation of validations for a new ML system that ingests and produces millions of rows and columns. For instance, there are many cases when simple validation checks on input data will lead to false alarms when shifts or errors in a single feature do not impact the output of a downstream ML model -- maybe because that feature was neglected by the ML model, when strong regularization during the ML model training phase taught the model to ignore that feature. 

Despite the rapid progress in recent years to automate and monitor ML systems: to the best of our knowledge there exists no data validation system that has reached broad adoption and which takes into account all of the data validation dimensions sketched in \autoref{sec:dimensions}. One reason for this is the difficulty of combining the multitude of validation strategies listed in \autoref{sec:solutions} into one unified framework. Considering the rapid pace of research at the intersection of ML and DBMS, see for instance \cite{Dong2018}, it is fair to assume that it is merely a matter of a few years until one framework or some open standard for data validation in the context of ML systems will have reached broad adoption. 

There are many data validation challenges in ML systems that go beyond technical aspects. Many of them are due to the complex nature of the data transformations induced by ML models. For instance identifying unfair biases often requires domain knowledge or access to grouping variables, which are often not available. And even if those are available, it is not always clear how fairness in the context of ML systems can be defined  \cite{Zhang2020}. A conceptual challenge related to privacy is for instance the trade-off between utility and differential privacy of a ML system \cite{Jayaraman2019}: how much predictive accuracy should be sacrificed to ensure privacy? Sacrificing accuracy against privacy in domains like health care or jurisdiction is a difficult question for which ethical and legal dimensions are more important than technical aspects. Next to these ethical and legal aspects, there is one key factor hindering adoption of comprehensive data validation in ML systems and that more related to cultural aspects. Many scientists like to build new models and tools, but writing tests, integrating monitoring and validation stages in an ML system are not exactly the most popular tasks amongst researchers. But often the competencies of the scientists who built a model is required to build well functioning monitoring and validation solutions in ML systems. 

Based on these observations we derive some suggestions for how to drive innovation and adoption of data validation in the context of ML systems. First, we hope that the current trend for research at the intersection of ML and DBMS communities will continue to grow and identify more synergies leveraging and complementing each others expertise. We have seen some great examples of constructive but vivid discussion between the two communities, for instance that sparked by Kraska and colleagues around their work on learning index structures \cite{Kraska2018}. This work is unrelated to data validation and mentioned merely as an example of transdisciplinary research debates. 
Second, when building ML systems there is a broad spectrum of operational challenges and seamless integration with cloud infrastructure is key to reaching broad adoption. 

\newpage
We conclude that establishing data validation in ML systems will require a stronger focus on usability and simple APIs. Third we believe that data validation in ML systems will reach broad adoption once the research community will have found better ways of automating the validation workflow, most importantly the generation of checks for each of the data validation dimensions listed in \autoref{sec:dimensions}. 

In the past years we have seen great examples of automated tooling for tracking ML experiments \cite{Schelter2017}, experimentation in ML production systems \cite{Bose2017a}, input data validation \cite{Schelter2018,Breck2019} and validation strategies for predictions of ML systems \cite{Rabanser2018,Schelter2020,DAmour2020}. One example of how some of these data validation techniques could be integrated into an automated workflow would be that presented in \cite{Rukat2020}, where the authors propose to iterate through a sequence of data validation~\cite{Schelter2018}, data cleaning\cite{Biessmann2018a} and quantification of downstream impact on ML predictive performance~\cite{Schelter2020} to achieve an automated ML workflow. We believe that increasing the level of usability through automation in data validation will enable researchers to focus on more important questions like the conceptual, ethical and legal questions and ultimately lead to more responsible usage of ML systems in production systems.


%\begin{small}
%	\bibliographystyle{plain}
%	\bibliography{data-validation-ml-systems}
%\end{small}


\begin{thebibliography}{10}

\bibitem{Polyzotis2018}
{\em {Data Lifecycle Challenges in Production Machine Learning: A Survey}},
  number (Vol. 47, No. 2). SIGMOD Record, 2018.

\bibitem{Abadi2016}
Martin Abadi, Andy Chu, Ian Goodfellow, H.~Brendan McMahan, Ilya Mironov, Kunal
  Talwar, and Li~Zhang.
\newblock Deep learning with differential privacy.
\newblock In {\em Proceedings of the 2016 ACM SIGSAC Conference on Computer and
  Communications Security}, CCS '16, pages 308--318, New York, NY, USA, 2016.
  Association for Computing Machinery.

\bibitem{Abedjan2018}
Ziawasch Abedjan, Lukasz Golab, Felix Naumann, and Thorsten Papenbrock.
\newblock {Data Profiling}.
\newblock {\em Synthesis Lectures on Data Management}, 10(4):1--154, nov 2018.

\bibitem{Athalye18}
Anish Athalye, Logan Engstrom, Andrew Ilyas, and Kevin Kwok.
\newblock Synthesizing robust adversarial examples.
\newblock volume~80 of {\em Proceedings of Machine Learning Research}, pages
  284--293, Stockholmsm{\"a}ssan, Stockholm Sweden, 10--15 Jul 2018. PMLR.

\bibitem{Bagdasaryan2019}
Eugene Bagdasaryan, Omid Poursaeed, and Vitaly Shmatikov.
\newblock Differential privacy has disparate impact on model accuracy.
\newblock In H.~Wallach, H.~Larochelle, A.~Beygelzimer, F.~Alch\'{e}-Buc,
  E.~Fox, and R.~Garnett, editors, {\em Advances in Neural Information
  Processing Systems}, volume~32, pages 15479--15488. Curran Associates, Inc.,
  2019.

\bibitem{Baylor2017}
Denis Baylor, Eric Breck, Heng~Tze Cheng, Noah Fiedel, Chuan~Yu Foo, Zakaria
  Haque, Salem Haykal, Mustafa Ispir, Vihan Jain, Levent Koc, Chiu~Yuen Koo,
  Lukasz Lew, Clemens Mewald, Akshay~Naresh Modi, Neoklis Polyzotis, Sukriti
  Ramesh, Sudip Roy, Steven~Euijong Whang, Martin Wicke, Jarek Wilkiewicz, Xin
  Zhang, and Martin Zinkevich.
\newblock {TFX: A TensorFlow-based production-scale machine learning platform}.
\newblock In {\em Proceedings of the ACM SIGKDD International Conference on
  Knowledge Discovery and Data Mining}, volume Part F129685, pages 1387--1395,
  New York, NY, USA, aug 2017. Association for Computing Machinery.

\bibitem{Bhagoji_2018}
Arjun~Nitin Bhagoji, Daniel Cullina, Chawin Sitawarin, and Prateek Mittal.
\newblock Enhancing robustness of machine learning systems via data
  transformations.
\newblock {\em 2018 52nd Annual Conference on Information Sciences and Systems
  (CISS)}, Mar 2018.

\bibitem{Biessmann2018a}
Felix Biessmann, David Salinas, Sebastian Schelter, Philipp Schmidt, and Dustin
  Lange.
\newblock {"Deep" Learning for Missing Value Imputationin Tables with
  Non-Numerical Data}.
\newblock In {\em Proceedings of the 27th ACM International Conference on
  Information and Knowledge Management - CIKM '18}, pages 2017--2025, New York,
  New York, USA, 2018. ACM Press.

\bibitem{Biggio_2013}
Battista Biggio, Igino Corona, Davide Maiorca, Blaine Nelson, Nedim {\v
  S}rndi{\'c}, Pavel Laskov, Giorgio Giacinto, and Fabio Roli.
\newblock Evasion attacks against machine learning at test time.
\newblock {\em Lecture Notes in Computer Science}, pages 387--402, 2013.

\bibitem{Bose2017a}
Joos~Hendrik B{\"{o}}se, Valentin Flunkert, Jan Gasthaus, Tim Januschowski,
  Dustin Lange, David Salinas, Sebastian Schelter, Matthias Seeger, and Yuyang
  Wang.
\newblock {Probabilistic demand forecasting at scale}.
\newblock {\em Proceedings of the VLDB Endowment}, 10(12):1694--1705, 2017.

\bibitem{Breck2019}
Eric Breck, Neoklis Polyzotis, Sudip Roy, Steven~Euijong Whang, and Martin
  Zinkevich.
\newblock {DATA VALIDATION FOR MACHINE LEARNING}.
\newblock Technical report, 2019.

\bibitem{Ceritli2020}
Taha Ceritli, Christopher~K.I. Williams, and James Geddes.
\newblock ptype: probabilistic type inference.
\newblock {\em Data Mining and Knowledge Discovery}, 34(3):870--904, may 2020.

\bibitem{Dallachiesat2013}
Michele Dallachiesat, Amr Ebaid, Ahmed Eldawy, Ahmed Elmagarmid, Ihab~F. Ilyas,
  Mourad Ouzzani, and Nan Tang.
\newblock {NADEEF: A commodity data cleaning system}.
\newblock In {\em Proceedings of the ACM SIGMOD International Conference on
  Management of Data}, 2013.

\bibitem{DAmour2020}
Alexander D'Amour, Katherine Heller, Dan Moldovan, Ben Adlam, Babak Alipanahi,
  Alex Beutel, Christina Chen, Jonathan Deaton, Jacob Eisenstein, Matthew~D.
  Hoffman, Farhad Hormozdiari, Neil Houlsby, Shaobo Hou, Ghassen Jerfel, Alan
  Karthikesalingam, Mario Lucic, Yian Ma, Cory McLean, Diana Mincu, Akinori
  Mitani, Andrea Montanari, Zachary Nado, Vivek Natarajan, Christopher Nielson,
  Thomas~F. Osborne, Rajiv Raman, Kim Ramasamy, Rory Sayres, Jessica Schrouff,
  Martin Seneviratne, Shannon Sequeira, Harini Suresh, Victor Veitch, Max
  Vladymyrov, Xuezhi Wang, Kellie Webster, Steve Yadlowsky, Taedong Yun,
  Xiaohua Zhai, and D.~Sculley.
\newblock {Underspecification Presents Challenges for Credibility in Modern
  Machine Learning}.
\newblock 2020.

\bibitem{Dong2018}
Xin~Luna Dong and Theodoros Rekatsinas.
\newblock {Data Integration and Machine Learning: A Natural Synergy}.
\newblock {\em Proceedings ofthe VLDB Endowment, Vol.}, 11(12):2094--2097,
  2018.

\bibitem{Dwork08differentialprivacy}
Cynthia Dwork.
\newblock Differential privacy: A survey of results.
\newblock In {\em In Theory and Applications of Models of Computation}, pages
  1--19. Springer, 2008.

\bibitem{Dworak_2006_diff}
Cynthia Dwork, Frank McSherry, Kobbi Nissim, and Adam Smith.
\newblock Calibrating noise to sensitivity in private data analysis.
\newblock In Shai Halevi and Tal Rabin, editors, {\em Theory of Cryptography},
  pages 265--284, Berlin, Heidelberg, 2006. Springer Berlin Heidelberg.

\bibitem{Dworak2014differantial}
Cynthia Dwork and Aaron Roth.
\newblock The algorithmic foundations of differential privacy.
\newblock {\em Found. Trends Theor. Comput. Sci.}, 9(3--4):211--407, August
  2014.

\bibitem{feinman2017detecting}
Reuben Feinman, Ryan~R. Curtin, Saurabh Shintre, and Andrew~B. Gardner.
\newblock Detecting adversarial samples from artifacts, 2017.

\bibitem{Fredrikson2015}
Matt Fredrikson, Somesh Jha, and Thomas Ristenpart.
\newblock {Model inversion attacks that exploit confidence information and
  basic countermeasures}.
\newblock In {\em Proceedings of the ACM Conference on Computer and
  Communications Security}, 2015.

\bibitem{gong2017adversarial}
Zhitao Gong, Wenlu Wang, and Wei-Shinn Ku.
\newblock Adversarial and clean data are not twins, 2017.

\bibitem{goodfellow2014explaining}
Ian~J. Goodfellow, Jonathon Shlens, and Christian Szegedy.
\newblock Explaining and harnessing adversarial examples, 2014.

\bibitem{grosse2017statistical}
Kathrin Grosse, Praveen Manoharan, Nicolas Papernot, Michael Backes, and
  Patrick McDaniel.
\newblock On the (statistical) detection of adversarial examples, 2017.

\bibitem{Hayes2019}
Jamie Hayes, Luca Melis, George Danezis, and Emiliano~De Cristofaro.
\newblock {LOGAN: Membership Inference Attacks Against Generative Models}.
\newblock {\em Proceedings on Privacy Enhancing Technologies},
  2019(1):133--152, 2019.

\bibitem{Heidari2019}
Alireza Heidari, Joshua McGrath, Ihab~F Ilyas, and Theodoros Rekatsinas.
\newblock {HoloDetect: Few-Shot Learning for Error Detection}.
\newblock {\em ACM Reference Format}, 2019.

\bibitem{hendrycks2016early}
Dan Hendrycks and Kevin Gimpel.
\newblock Early methods for detecting adversarial images, 2016.

\bibitem{Ho_2015}
Chien-Ju Ho, Aleksandrs Slivkins, Siddharth Suri, and Jennifer~Wortman Vaughan.
\newblock Incentivizing high quality crowdwork.
\newblock {\em Proceedings of the 24th International Conference on World Wide
  Web - WWW '15}, 2015.

\bibitem{Hosseini2019AreOR}
H.~Hosseini, S.~Kannan, and R.~Poovendran.
\newblock Are odds really odd? bypassing statistical detection of adversarial
  examples.
\newblock {\em ArXiv}, abs/1907.12138, 2019.

\bibitem{Huang2017}
Zhipeng Huang and Yeye He.
\newblock {Auto-Detect: Data-Driven Error Detection in Tables}.
\newblock {\em SIGMOD}, 2017.

\bibitem{Jayaraman2019}
Bargav Jayaraman and David Evans.
\newblock {\em {Evaluating Differentially Private Machine Learning in
  Practice}}.
\newblock 28th USENIX Security Symposium, 2019.

\bibitem{Kraska2018}
Tim Kraska, Alex Beutel, Ed~H Chi, Jeffrey Dean, and Neoklis Polyzotis.
\newblock {The case for learned index structures}.
\newblock In {\em Proceedings of the ACM SIGMOD International Conference on
  Management of Data}, pages 489--504, 2018.

\bibitem{Krishnan2017}
Sanjay Krishnan, Michael~J Franklin, Ken Goldberg, and Eugene Wu.
\newblock {BoostClean: Automated Error Detection and Repair for Machine
  Learning}.

\bibitem{Krishnan:2016b}
Sanjay Krishnan, Jiannan Wang, Eugene Wu, Michael~J. Franklin, and K.~Goldberg.
\newblock {ActiveClean: Interactive data cleaning for statistical modeling}.
\newblock {\em Proceedings of the VLDB Endowment}, 9(12):948--959, 2016.

\bibitem{Krishnan2019}
Sanjay Krishnan and Eugene Wu.
\newblock {AlphaClean: Automatic Generation of Data Cleaning Pipelines}.
\newblock apr 2019.

\bibitem{Kruse2018}
Sebastian Kruse and Felix Naumann.
\newblock {Efficient Discovery of Approximate Dependencies}.
\newblock {\em PVLDB}, 11(7):759--772, 2018.

\bibitem{Kumar2017}
Arun Kumar, Matthias Boehm, and Jun Yang.
\newblock {Data Management in Machine Learning: Challenges, Techniques, and
  Systems}.
\newblock In {\em SIGMOD}, 2017.

\bibitem{Kurakin2017AdversarialEI}
A.~Kurakin, Ian~J. Goodfellow, and S.~Bengio.
\newblock Adversarial examples in the physical world.
\newblock {\em ArXiv}, abs/1607.02533, 2017.

\bibitem{aleks2017deep}
Aleksander Madry, Aleksandar Makelov, Ludwig Schmidt, Dimitris Tsipras, and
  Adrian Vladu.
\newblock Towards deep learning models resistant to adversarial attacks, 2017.

\bibitem{MahdaviBerlinmahdavilahijani}
Mohammad~TU {Mahdavi Berlin mahdavilahijani}, tu-berlinde {Ziawasch Abedjan},
  Raul {Castro Fernandez MIT}, Samuel {Madden MIT}, Mourad Ouzzani, Michael
  {Stonebraker MIT}, Nan Tang, Mohammad Mahdavi, Ziawasch Abedjan, Raul {Castro
  Fernandez}, Samuel Madden, and Michael Stonebraker.
\newblock {Raha: A Configuration-Free Error Detection System}.
\newblock 18(19).

\bibitem{Mehrabi2019}
Ninareh Mehrabi, Fred Morstatter, Nripsuta Saxena, Kristina Lerman, and Aram
  Galstyan.
\newblock {A Survey on Bias and Fairness in Machine Learning}.
\newblock aug 2019.

\bibitem{metzen2017detecting}
Jan~Hendrik Metzen, Tim Genewein, Volker Fischer, and Bastian Bischoff.
\newblock On detecting adversarial perturbations, 2017.

\bibitem{Mironov2017}
I.~{Mironov}.
\newblock R{\'e}nyi differential privacy.
\newblock In {\em 2017 IEEE 30th Computer Security Foundations Symposium
  (CSF)}, pages 263--275, 2017.

\bibitem{MoosaviDezfooli2016DeepFoolAS}
Seyed-Mohsen Moosavi-Dezfooli, Alhussein Fawzi, and P.~Frossard.
\newblock Deepfool: A simple and accurate method to fool deep neural networks.
\newblock {\em 2016 IEEE Conference on Computer Vision and Pattern Recognition
  (CVPR)}, pages 2574--2582, 2016.

\bibitem{Nguyen_2015}
Anh Nguyen, Jason Yosinski, and Jeff Clune.
\newblock Deep neural networks are easily fooled: High confidence predictions
  for unrecognizable images.
\newblock {\em 2015 IEEE Conference on Computer Vision and Pattern Recognition
  (CVPR)}, Jun 2015.

\bibitem{pang2017robust}
Tianyu Pang, Chao Du, Yinpeng Dong, and Jun Zhu.
\newblock Towards robust detection of adversarial examples, 2017.

\bibitem{Papenbrock2015}
Thorsten Papenbrock, Jens Ehrlich, Jannik Marten, Tommy Neubert, Jan-Peer
  Rudolph, Martin Sch{\"{o}}nberg, Jakob Zwiener, and Felix Naumann.
\newblock {Functional dependency discovery}.
\newblock {\em Proceedings of the VLDB Endowment}, 8(10):1082--1093, jun 2015.

\bibitem{papernot2016semisupervised}
Nicolas Papernot, Mart{\'\i}n Abadi, {\'U}lfar Erlingsson, Ian Goodfellow, and
  Kunal Talwar.
\newblock Semi-supervised knowledge transfer for deep learning from private
  training data, 2016.

\bibitem{Cavallo1987}
Michael Pittarelli.
\newblock {Roger Cavallo}.
\newblock {\em Proceedings of the Thirteenth International Conference on Very
  Large Data Bases}, (January):71--81, 1987.

\bibitem{Rabanser2018}
Stephan Rabanser, Stephan G{\"{u}}nnemann, and Zachary~C Lipton.
\newblock {Failing loudly: An empirical study of methods for detecting dataset
  shift}, 2018.

\bibitem{Rekatsinas2017}
Theodoros Rekatsinas, Xu~Chu, Ihab~F Ilyas, and Christopher R{\'{e}}.
\newblock {HoloClean: Holistic Data Repairs with Probabilistic Inference}.
\newblock {\em Proceedings ofthe VLDB Endowment}, 10(11), 2017.

\bibitem{Rukat2020}
Tammo Rukat, Dustin Lange, Sebastian Schelter, and Felix Biessmann.
\newblock {Towards Automated ML Model Monitoring: Measure, Improve and Quantify
  Data Quality}.
\newblock Technical report.

\bibitem{Samek2019}
Wojciech Samek and Klaus~Robert M{\"{u}}ller.
\newblock {Towards Explainable Artificial Intelligence}.
\newblock In {\em Lecture Notes in Computer Science (including subseries
  Lecture Notes in Artificial Intelligence and Lecture Notes in
  Bioinformatics)}, volume 11700 LNCS, pages 5--22. Springer Verlag, 2019.

\bibitem{Schelter2017}
Sebastian Schelter, Joos-Hendrik Boese, Johannes Kirschnick, Thoralf Klein, and
  Stephan Seufert.
\newblock {Automatically tracking metadata and provenance of machine learning
  experiments}.
\newblock {\em Machine Learning Systems workshop at NeurIPS}, 2017.

\bibitem{Schelter2018}
Sebastian Schelter, Dustin Lange, Philipp Schmidt, Meltem Celikel, Felix
  Biessmann, Andreas Grafberger, and Meltem Ce-Likel.
\newblock {Automating Large-Scale Data Quality Verification}.
\newblock {\em PVLDB}, 11(12):1781--1794, 2018.

\bibitem{Schelter2020}
Sebastian Schelter, Rukat Tammo, and Felix Biessmann.
\newblock {Learning to Validate the Predictions of Black Box Classifiers on
  Unseen Data}.
\newblock {\em SIGMOD}, 2020.

\bibitem{schmidt2019quantifying}
Philipp Schmidt and Felix Biessmann.
\newblock Quantifying interpretability and trust in machine learning systems,
  2019.

\bibitem{schmidt2020risk}
Philipp Schmidt and Felix Biessmann.
\newblock Calibrating human-ai collaboration: Impact of risk, ambiguity and
  transparency on algorithmic bias.
\newblock In Andreas Holzinger, Peter Kieseberg, A~Min Tjoa, and Edgar Weippl,
  editors, {\em Machine Learning and Knowledge Extraction}, pages 431--449,
  Cham, 2020. Springer International Publishing.

\bibitem{Sculley2015}
D~Sculley, Gary Holt, Daniel Golovin, Eugene Davydov, Todd Phillips, Dietmar
  Ebner, Vinay Chaudhary, Michael Young, and Dan Dennison.
\newblock {Hidden Technical Debt in Machine Learning Systems}.
\newblock {\em Nips}, pages 2494--2502, 2015.

\bibitem{Shokri2017}
Reza Shokri, Marco Stronati, Congzheng Song, and Vitaly Shmatikov.
\newblock {Membership Inference Attacks Against Machine Learning Models}.
\newblock {\em Proceedings - IEEE Symposium on Security and Privacy}, pages
  3--18, 2017.

\bibitem{Stoyanovich2020}
Julia Stoyanovich, Bill Howe, and H.~V. Jagadish.
\newblock {Responsible data management}.
\newblock {\em Proceedings of the VLDB Endowment}, 13(12):3474--3488, 2020.

\bibitem{Sweeney2002}
Latanya Sweeney.
\newblock K-anonymity: A model for protecting privacy.
\newblock {\em Int. J. Uncertain. Fuzziness Knowl.-Based Syst.},
  10(5):557--570, October 2002.

\bibitem{szegedy2013intriguing}
Christian Szegedy, Wojciech Zaremba, Ilya Sutskever, Joan Bruna, Dumitru Erhan,
  Ian Goodfellow, and Rob Fergus.
\newblock Intriguing properties of neural networks, 2013.

\bibitem{wang_2018}
H.~{Wang}, S.~{Guo}, J.~{Cao}, and M.~{Guo}.
\newblock Melody: A long-term dynamic quality-aware incentive mechanism for
  crowdsourcing.
\newblock {\em IEEE Transactions on Parallel and Distributed Systems},
  29(4):901--914, 2018.

\bibitem{WortmanVaughan2018}
Jennifer {Wortman Vaughan}.
\newblock {Making Better Use of the Crowd: How Crowdsourcing Can Advance
  Machine Learning Research}.
\newblock Technical report, 2018.

\bibitem{Yang2020}
Ke~Yang, Biao Huang, and Sebastian Schelter.
\newblock {Fairness-Aware Instrumentation of Preprocessing Pipelines for
  Machine Learning}.
\newblock 2020.

\bibitem{zhang2019}
Haichao Zhang and Jianyu Wang.
\newblock Defense against adversarial attacks using feature scattering-based
  adversarial training, 2019.

\bibitem{Zhang2020}
Jie~M. Zhang, Mark Harman, Lei Ma, and Yang Liu.
\newblock {Machine Learning Testing: Survey, Landscapes and Horizons}.
\newblock {\em IEEE Transactions on Software Engineering}, pages 1--1, feb
  2020.

\bibitem{Zhao2019}
Yue Zhao, Zain Nasrullah, and Zheng Li.
\newblock {PyOD: A python toolbox for scalable outlier detection}.
\newblock {\em Journal of Machine Learning Research}, 2019.

\bibitem{Dingyuan2019}
Dingyuan Zhu, Ziwei Zhang, Peng Cui, and Wenwu Zhu.
\newblock Robust graph convolutional networks against adversarial attacks.
\newblock In {\em Proceedings of the 25th ACM SIGKDD International Conference
  on Knowledge Discovery and Data Mining}, KDD '19, pages 1399--1407, New York,
  NY, USA, 2019. Association for Computing Machinery.

\bibitem{Ziemann2016}
Mark Ziemann, Yotam Eren, and Assam El-Osta.
\newblock {Gene name errors are widespread in the scientific literature}.
\newblock {\em Genome Biology}, 17(1):177, aug 2016.

\end{thebibliography}


\end{document}

\end{article}

\begin{article}
{Enhancing the Interactivity of Dataframe Queries by Leveraging Think Time}
{Doris Xin, Devin Petersohn, Dixin Tang, Yifan Wu, Joseph E. Gonzalez, Joseph M. Hellerstein, Anthony D. Joseph and Aditya G. Parameswaran}
%\graphicspath{{submissions/NobleRoberts_final/}}
% link to instruction: https://tc.computer.org/tcde/tcde-bulletin-author-instructions/
% \documentclass[11pt,dvipdfm]{article}
\documentclass[11pt]{article}
\usepackage{tabularx}
\usepackage{ragged2e}  % for '\RaggedRight' macro (allows hyphenation)
\usepackage{booktabs}  % for \toprule, \midrule, and \bottomrule macros
\usepackage{textcomp}
\usepackage{amsfonts,amsmath}
\usepackage{deauthor,times}
\usepackage{graphicx} % 
\usepackage{hyperref}
\usepackage{comment}
\graphicspath{{asudeh/}}
\usepackage{soul}
\usepackage{subcaption}
\usepackage{ulem}
\usepackage{wrapfig}
\usepackage{color}
\usepackage{xspace}
\newtheorem{problem}{Problem}

%\DeclareMathOperator*{\argmax}{arg\,max}

%remove the following commands/package b4 submission
\newcommand{\hide}[1]{}
\newcommand{\eat}[1]{}
\newcommand{\resolved}[1]{\hide{#1}}
\newcommand{\abol}[1]{\textcolor{red}{Abol: #1}}
\newcommand{\mahdi}[1]{\textcolor{red}{Mahdi: #1}}
\newcommand{\nima}[1]{\textcolor{red}{Nima: #1}}

\newcommand{\dee}{\mathcal{D}}
\newcommand{\Gee}{\mathcal{G}}
\newcommand{\gee}{\mathbf{g}}
\newcommand{\ee}{\mathbf{e}}
\newcommand{\es}{\mathcal{S}}
\newcommand{\el}{\mathcal{L}}
\newcommand{\xx}{\mathcal{x}}
\newcommand{\dist}{\xi}
\newcommand{\alg}{\mathsf{A}}
\newcommand{\qu}{\mathbf{q}}
\newcommand{\ex}{\mathbf{x}}
\newcommand{\ti}{\mathbf{t}}
\newcommand{\sdt}{\mathsf{SDT}}
\newcommand{\wdt}{\mathsf{WDT}}
\newcommand{\Qu}{\mathbf{Q}}
\newcommand{\pe}{\mathbb{P}}
\newcommand{\megam}{\mathcal{M}}
\newcommand{\eps}{\varepsilon}
\newcommand{\enet}{{$\varepsilon$-{\bf net}}\xspace}
\newcommand{\net}{{\tt net}\xspace}
\newcommand{\vcd}{VC-dimension\xspace}
\newcommand{\at}[1]{{\tt \small #1}\xspace}
\newcommand{\pr}{Pr}

\newcommand{\sharpP}{\mbox{\#P}}
\newcommand{\NP}{\mathsf{NP}}
\newcommand{\LP}{\mathsf{LP}}
\newcommand{\IP}{\mathsf{IP}}
\newcommand{\ru}{{\sc {RU}}\xspace}
\newcommand{\sru}{{\sc {strongRU}}\xspace}
\newcommand{\wru}{{\sc {weakRU}}\xspace}

\newcommand{\fmsystem}{{\sc Chameleon}\xspace}
\newcommand{\fm}{$\mathcal{F}$\xspace}

\newtheorem{experiment}{Experiment}

\begin{document}

\title{Coverage-based Data-centric Approaches for \\Responsible and Trustworthy AI\thanks{This research was supported by the National Science Foundation under grant No. 2107290.}}

\author{
\begin{tabular}[t]{c@{\extracolsep{2.4em}}c@{\extracolsep{2.4em}}c@{\extracolsep{2.3em}}c} 
Nima Shahbazi & Mahdi Erfanian & Abolfazl Asudeh \\ 
University of Illinois Chicago & University of Illinois Chicago & University of Illinois Chicago\\
 nshahb3@uic.edu & merfan2@uic.edu & asudeh@uic.edu
\end{tabular}
}

\maketitle


\begin{abstract}
The grand goal of data-driven decision systems is to help make decisions easier, more accurate, at a higher scale, and also just. However, data-driven algorithms are only as good as the data they work with. Yet, data sets, especially those with social data, often do not represent minorities. The paucity of training data is a perpetual problem for AI, and the outcome of ML models for cases not represented in their training data is often not reliable. 
Hence, without properly addressing the lack of representation issues in data, we cannot expect AI-based societal solutions to have responsible and trustworthy outcomes. 

This paper focuses on data coverage as a data-centric approach for identifying and resolving misrepresentation of minorities in data.
To achieve this goal, we propose novel algorithms that (a) {\it identify} and {\it resolve} insufficient data coverage across data with different modalities and (b) use lack of representation information to generate data-centric {\it reliability warnings}.
 \end{abstract}
 
 %%%%%%%%%%%%%%%%%%%%%%%%%%%%%%%% INTRO  %%%%%%%%%%%%%%%%%%%%%%%%%%%%%%%%
\section{Introduction}\label{sec:intro} % Abstract+Intro: up to 2.5 pages 
Data-driven decision-making has shaped every corner of human life, spanning from autonomous vehicles to healthcare and even predictive policing and criminal justice. A pivotal concern, especially in applications that affect individuals, revolves around the reliability of the decisions rendered by the system.
It is easy to see that the accuracy of a data-driven decision depends, first and foremost, on the data used to make it. Essentially, the system learns the phenomena that data represent. While we may desire that the data should represent the underlying data distribution from which the production data is drawn, this alone may be insufficient, as it merely enables the model to perform well for the average case.
As a result, a model with a high accuracy could fail for specific regions in the data with insufficient representation. These regions may matter because they frequently represent some minority population in society. They could also represent cases that may not happen very often but have a relevant impact on the correctness of a critical decision.
In short, if the data fails to sufficiently represent a specific population, the outcome of the decision system for that population may not be trustworthy.

The phenomenon known as \textit{Representation Bias} can arise from how the data was originally collected, or it could be the result of biases introduced post-collection—whether historically, cognitively, or statistically.

Representation bias is essentially inevitable without a systematic approach to data collection. 
For example, in the context of survey data collection, vital steps involve identifying all populations within the underlying distribution based on desired demographic information and ensuring comprehensive coverage with sufficient samples from each group. 
Even then, only an (uncontrolled) subset of the invitees will opt-in to respond to the survey.
Another challenge lies in the fact that data scientists often lack control over the data collection process, leading to the reliance on ``found data'' in the majority of data-driven systems. Therefore, with no guarantee on the aforementioned steps in the data collection process, the found data is most likely a biased sample.
Acknowledging the potential harms of representation bias, the notion of \textit{Data Coverage}~\cite{asudeh2019assessing,shahbazi2023representation} has been proposed to ensure the adequate representation of minority groups in data sets employed for decision-making and developing sophisticated data science tools. 

Addressing representation issues in data poses various challenges depending on the modality of the data. In this paper, we focus on identifying and resolving lack of coverage issues in data with different modalities.
We start by proposing a variety of techniques (spanning from geometric and combinatorial optimization to crowd-souring) aimed at efficiently detecting insufficient coverage on structured data sets with non-ordinal categorical and continuous attributes, as well as image data sets. Next, we propose a range of approaches grounded in data integration and generative data augmentation to address the lack of coverage by enriching the data sets with more data. However, with limited control over the data collection processes, it could be difficult and expensive to resolve all misrepresentations. 
Since adding more data is not always possible, we proceed to introduce data-centric preventive solutions that warn the user about the reliability of their predictions regarding representation bias issues. These warnings assist users in determining whether they trust the outcomes of the models or exercise caution. 

 %%%%%%%%%%%%%%%%%%%%%%%%%%%%%%%% IDENTIFICATION  %%%%%%%%%%%%%%%%%%%%%%%%%%%%%%%%
\section{Detecting Insufficient Representation of Minorities}\label{sec:identification} %up to 3.5 pages
Representation bias happens when the development (training data) population under-represents 
and subsequently fails to generalize well 
for some parts of the target population, due to historical bias, sampling bias, etc.
The notion of {\it data coverage} has been studied across different settings in \cite{shahbazi2023representation} as a metric to measure representation bias. At a high level, coverage is referred to as having enough similar entries for each object in a data set. 
For a better understanding, let us go over the definition of the generalized notion of coverage:

\begin{definition}[Data Coverage]\label{def:coverage}
Consider a data set $\dee$ with $n$ tuples, each consisting of $d$ attributes of interest $\mathbf{x}=\{x_1, x_2, \cdots,x_d\}$, such as {\tt gender}, {\tt race}, {\tt salary}, {\tt age}, etc, that are used for coverage identification.
The data set also contains target attributes $\mathbf{y} = \{ y_1,\cdots,y_{d'}\}$ that may or may not be considered for the coverage problem.
A query point $q$ is not covered by the data set $\dee$, if there are not ``enough'' data points in $\dee$ that are representative of $q$.
To generalize the notion of coverage, let us define $\gee(q)$ as the universe of tuples that would represent $q$ and let $\gee_\dee(q) = \gee(q)\cap \dee$. In other words, $\gee_\dee(q)$ are the set of tuples in $\dee$ that represent $q$.
Using this notation, we define the coverage of $q$ as the size of $\gee_\dee(q)$. That is,
$cov(q,\dee) = | \gee_\dee(q)|$.
Given a value $\tau$, $q$ is covered if $cov(q,\dee)>\tau$.
Similarly, a group $\gee$ is not covered if $\gee\cap \dee<\tau$.
The {\it uncovered region} in a data set is the collection of groups that are not covered by it.
\end{definition}

\subsection{Structured Data}
In this section, we focus on identifying representation bias in structured data.
Depending on the type of the attributes of interest, we categorize the techniques into two classes based on whether they target the problem for non-ordinal {\it categorical} (e.g. {\tt race}, {\tt gender}) or ordinal {\it continuous} (e.g. {\tt age}) attributes. The attributes of interest considered for representation bias often include sensitive attributes such as {\tt race} and {\tt gender} but are not necessarily limited to them.

\subsubsection{Categorical Attributes}

For cases where attributes of interest are non-ordinal categorical,
the cartesian product of values on a subset of attributes $\mathbf{x}'\subseteq \mathbf{x}$, form a set of (sub-)groups.
For example, $\{$ {\tt white male}, {\tt white female}, {\tt black male} $,\cdots\}$ are the subgroups defined on the attributes {\tt (race,gender)}.
We refer to the number of attributes used to specify a subgroup as the {\it level} of that subgroup.
For example, the level of the subgroup {\tt white male} is 2, while the level of the subgroup {\tt male} is 1.
We use $\ell(\gee)$, to refer to the level of a subgroup $\gee$.
Similarly, we say a subgroup $\gee'$ is a subset of $\gee$, if the groups specifying $\gee'$ are a superset of the ones for $\gee$. For example {\tt (married white male)} a subset of the more general group {\tt (white male)}. That is, the set of individuals in group {\tt (married white male)} are a subset of {\tt (white male)}.
Moreover, we say a subgroup $\gee$ is a {\it parent} of the subgroup $\gee'$, if $\gee'\subset \gee$ and $\ell(\gee)=\ell(\gee')+1$. For example, the subgroup {\tt (white male)} is a parent of the subgroup {\tt (married white male)}.
We use \textit{patterns} to refer to uncovered subgroups.
A pattern $P$ is a string of $d$ values, where $P[i]$ is either a value from the domain of $x_i$, or it is ``unspecified'', specified with $X$. 
For example, consider a data set with three binary attributes of interest $\mathbf{x}=\{x_1, x_2, x_3\}$. The pattern $P=X01$ specifies all the tuples for which $x_2=0$ and $x_3=1$ ($x_1$ can have any value).
The set of patterns that identify most general uncovered subgroups are called {\it Maximal Uncovered Patterns} (MUPs).

No polynomial time algorithm can guarantee the enumeration of the entire MUPs, however, several algorithms inspired by set enumeration and the Apriori algorithm for association rule mining are proposed to efficiently address this problem~\cite{asudeh2019assessing}.
In this regard, we introduce \textit{Pattern Graph} data structure that exploits the relationship between patterns to do less work than computing all uncovered patterns by removing the non-maximal ones. 
The parent-child relationship between the patterns is represented in a graph that can be used to find better algorithms. 
\textit{Pattern-Breaker} starts from the top of the graph where the general patterns are and moves down by breaking each pattern into more specific ones. If a pattern is uncovered, then all of its descendants are also uncovered and they can not be an MUP, even if they have a parent that is covered. Therefore, this subgraph of the pattern graph can be pruned. 
The issue with \textit{Pattern-Breaker} is that it explores the covered regions of the pattern graph and for the cases where there are a few uncovered patterns, it has to explore a large portion of the exponential-size graph. 
To tackle this, \textit{Pattern-Combiner} algorithm is proposed that performs a bottom-up traversal of the pattern graph. It uses an observation that the coverage of a node at the level of the pattern graph can be computed as the sum of the coverage values of its children. 
The problem with \textit{Pattern-Combiner} is that it traverses over the uncovered nodes first and therefore, it will not perform well for the cases in which most of the nodes in the graph are uncovered. 
In fact, for the cases where most of the MUPs are placed in the middle of the graph, both \textit{Pattern-Breaker} and \textit{Pattern-Combiner} will not be as efficient as they should traverse half of the graph. Therefore, we propose \textit{Deep-Diver}, a search algorithm based on Depth-First-Search that quickly finds the MUPs, and uses them to limit the search space by pruning the nodes both dominating and dominated by the discovered MUPs.

\begin{figure*}[!tb]
    \begin{minipage}[t]{0.31\linewidth}
        \centering
        \includegraphics[width=\textwidth]{submissions/submission1/shahbazi/covcube1.jpg}
        \caption{\small Categorical attributes: the uncovered region of a toy example, as the collection of three MUPs.}
        \label{fig:covcube1}
    \end{minipage}
    \hfill
    \begin{minipage}[t]{0.31\linewidth}
        \centering
        \includegraphics[width=\textwidth]{submissions/submission1/shahbazi/cvrg_2_1.jpg}
        \caption{\small Continuous attributes, 2D: identifying the covered region in the gray Voronoi cell.}
        \label{fig:cvrg_2_1}
    \end{minipage}
    \hfill
    \begin{minipage}[t]{0.31\linewidth}
        \centering
        \includegraphics[width=\textwidth]{submissions/submission1/shahbazi/cvrg_2_2.jpg}
        \caption{ \small Continuous attributes, 2D: Uncovered region marked in red.}
        \label{fig:cvrg_2_2}
    \end{minipage}
\vspace{-5mm}
\end{figure*}

\subsubsection{Continuous Attributes}
Data in the real world often consists of a combination of continuous and discrete values. While simple solutions like binning {\tt age} into {\tt young} and {\tt old} can transform the continuous space into discrete. However, they may lead to coarse groupings that are sensitive to the thresholds chosen. It may be inappropriate to treat a 35-yo as {\tt young} but a 36-yo as {\tt old}. 
Therefore, we extend the notion of coverage to continuous space. Particularly, given data set $\dee$ with $n$ tuples over $d$ attributes, and vicinity radius $\rho$ and coverage threshold $k$, we want to identify the uncovered region -- the universe of uncovered query points.
A query point in continuous data space is covered if there are enough (at least $k$) data points in its $\rho$-vicinity neighborhood. $\rho$-vicinity neighborhood is the circle centered at the query point with radius $\rho$.

Depending on the number of attributes in a data set, we propose two algorithms for identifying uncovered regions in data~\cite{asudeh2021coverage}. 
The first algorithm known as \textit{Uncovered-2D} studies coverage over two-dimensional data sets where $\mathbf{x}=\{x_1,x_2\}$. To find the number of circles that a query point falls into and consequently discover the uncovered region, \textit{Uncovered-2D} makes a connection to $k$-th order Voronoi diagrams.
Consider a data set $\mathcal{D}$ and its corresponding $k$-th order Voronoi diagram. For every tuple $t\in \mathcal{D}$, let $\circ_t$ be the $d$-dimensional sphere ($d$-sphere) with radius $\rho$ centered at $t$.
Consider a $k$-voronoi cell $\mathcal{V}(S)$ in the $k$-th order Voronoi diagram $V_k(\mathcal{D})$.
Any point $q$ inside the intersections of the $d$-spheres of tuples in $S$, i.e. $q\in \underset{\forall t\in S}{\cap ~\circ_t}$, is covered, while all other points in the region are uncovered.
 The algorithm starts by constructing the $k$-th order Voronoi diagram of the data set and then for each Voronoi cell $\mathcal{V}(S)$ in the diagram, it computes the intersection of the circles of the tuples in $S$ and marks the portion of $\mathcal{V}(S)$ that falls outside it as uncovered.
After identifying the uncovered region, a 2D map of $\{x_1,x_2\}$ value combinations is used to report the region to the user.
The algorithm for the 2D case can be extended to the general case by relaxing the assumption on the number of attributes to discover the exact uncovered region, however, due to the curse of dimensionality, the search size space explodes as the number of dimensions increases and as a result, the algorithm will not be practical. Therefore, we propose a randomized approximation algorithm based on the geometric notion of \enet. 
Let $\mathcal{X}$ be a set and $\mathcal{R}$ be a set of subsets of $\mathcal{X}$. A set $\mathcal{N}\subset \mathcal{X}$ is an \enet for $\mathcal{X}$ if for any range $r\in\mathcal{R}$, if  $|r\cap \chi|>\eps|\chi|$, then $r$ contains at least one point of $N$.
The idea, at a high level, is to draw enough random samples from the space of potential query points to form an \enet. 
We then label the sampled query points as $\{-1,+1\}$ depending on whether those are covered or not, and learn the uncovered regions using the samples.

\subsection{Image Data}
Many known incidents of machine failures due to the lack of representation were on image data.
We consider an image data set with a fixed number of low-cardinality sensitive attributes such as {\tt\small race} and {\tt\small gender}. 
It is common that image data sets {\it lack explicit values} for sensitive attributes, which are crucial for coverage identification. An image data set is often a collection of images from different domains with little to no information about their domain and which groups they belong to. As a result, even studying coverage over low-cardinality and categorical attributes of interests is challenging in these cases.

\begin{wrapfigure}{R}{0.42\textwidth}
\centering
\vspace{-3mm}
\scriptsize
\begin{tabular}{|@{}c|@{}c@{}|@{}c@{}|@{}c@{}|} 
 \hline
{\bf data set} & {\bf classifier} & {\bf accuracy} & {\bf precision} \\ 
 &  &  & {\bf on female} \\ \hline
UTKFace:~& DeepFace (opencv) & 93.56 & {52.02}\\\cline{2-4}
({\tt females}=200,& DeepFace (retinaface) & 94.16 & {56.15}\\\cline{2-4}
{\tt males}=2800) & BaseCNN & 97.6 & 74.8\\
\hline
UTKFace:~& DeepFace (opencv) & 96.53 & {\bf 8.0}\\\cline{2-4}
({\tt females}=20,& DeepFace (retinaface) & 96.43 & {\bf 10.09}\\\cline{2-4}
{\tt males}=2980)& BaseCNN & 97.6 & {\bf 21.59}\\
\hline
\end{tabular}
\vspace{-3mm}
\caption{\small ML models' low performance for females in the presence of representation bias.~\cite{mousavi2024data}}\label{fig:mlfails}
\vspace{-3mm}
\end{wrapfigure}

In Figure~\ref{fig:mlfails}, we show that due to the issues such {\it machine bias} and {\it lack of distribution generalizability},
solely relying on state-of-the-art machine learning (ML) techniques fail to effectively identify lack of coverage in image data sets. Therefore, we propose an approach based on combining crowdsouring with ML~\cite{mousavi2024data}. 
Crowdsourcing is particularly promising for image data, for tasks such as image labeling, which, while challenging for the machine, are "easy" for human beings to conduct with minimal error. 

A key observation that enables a cost-effective crowdsourcing approach is that, while studying coverage, we would only like to find out if there are {\it enough tuples from each subgroup}.
Suppose a subgroup is covered if there are $\tau=100$ instances of it in the data set. Assume the (majority) group $\gee_1$ contains $n_1 \gg 100$ objects in the data set. 
To verify that $\gee_1$ is covered, it is enough for the crowd to discover 100 of those objects, not the entire $n_1$. 
Following this, $O(\tau)$ provides a lower bound on the number of crowd tasks required to verify a given group is covered. 
Still, this lower bound only holds for the groups that are covered, i.e., there is at least $\tau$ of those in the data set.
Surprisingly, verifying that a minority group is indeed uncovered is cumbersome, unlike the majority group.
This is because even though discovering $\tau$ objects from a group is enough for verifying that it is covered, one cannot {\it verify} a group is uncovered until there is a chance that the data set might still have enough objects from that group. Thus, assuming a non-zero probability for each unlabeled object to belong to each group, {one might need to ask the crowd to label the entire data set before they can confirm that a specific group is uncovered}.

Our idea for addressing this challenge is to
design {\it a divide and conquer algorithm} that, instead of {point queries}, uses {\it set queries} to iteratively eliminate subsets of data that {does not include any object from the given group}.
At a high level, our idea is to ask a set query from the crowd, inquiring whether the selected set contains at least one object from the given group $\gee$.
The user may provide two responses (yes/no). 
Interestingly, {in either case}, the user response provides valuable information that helps efficiently identify the coverage.
If the answer is ``No'', the set does not include any object from the given group $\gee$. As a result, the algorithm can safely prune the set, asking no further questions about it. In particular, for a group that is not covered, one can expect to see no answers on large set queries helping to prune a significant portion of the data set quickly.
On the other hand, if the answer is ``yes'', the set contains {at least} one object from the group $\gee$. As a result, the algorithm cannot prune the subset since it can have any number (larger than one) of the objects in $\gee$.
At first glance, the queries with yes answers do not provide helpful information as the algorithm cannot prune the subset (hence it needs to divide it into smaller subsets).
However, a key observation is that {the algorithm will only observe a limited number of yes answers} before it stops.
The reason is that the number of set queries with yes answers provides a {lower-bound} on the number of objects from $\gee$ in the data set. As a result, the algorithm can stop as soon as the lower bound reaches $\tau$, knowing that $\gee$ is covered.
The D\&C approach verifies the data coverage for a given group, while our goal is to identify the uncovered regions for a given set of sensitive attributes. The next question is how to utilize this algorithm for efficient coverage identification on different scenarios of sensitive attributes, forming intersectional or non-intersectional groups.
In particular, how can we find maximal uncovered patterns?
Our idea is to apply sampling and aggregate estimation techniques to find the groups that even if merged are likely to still be uncovered. This will help reduce the coverage identification cost by running the D\&C approach for the merged groups once.
 %%%%%%%%%%%%%%%%%%%%%%%%%%%%%%%% RESOLUTION  %%%%%%%%%%%%%%%%%%%%%%%%%%%%%%%%
\section{Resolving Insufficient Representation}\label{sec:resolution}

Data integration~\cite{nargesian2021tailoring,nargesian2022responsible} and data augmentation~\cite{sharma2020data,DBLP:journals/jair/ChawlaBHK02,iosifidis2018dealing,celis2020data} are considered as the primary solutions for reducing data coverage issues in a data set. 
Data integration is promising when external sources of data are available. On the other hand, recent advancements in generative AI and foundation models have enabled efficient and effective augmentation of data sets with synthetic data. 
Therefore, in the following, we review two approaches, one from each category, in the context of lack of coverage resolution.

\subsection{Data Integration}\label{sec:resolution:integration}

Data integration is to consolidate data from different sources into a single, unified view. 
Although it is an effective solution to acquire additional data from different distributions,
there are sampling policy and cost-efficiency concerns that need to be examined.  
Therefore, {\it Data Distribution Tailoring ({\sc DT})} introduces data integration techniques for resolving insufficient representation of subgroups in a data set in the most cost-effective manner~\cite{nargesian2021tailoring}.
A query to {\sc DT} 
consists of a target schema, and a set of group distribution requirements in the form of the minimum counts (e.g., ``{\tt\small 1,000 breast cancer monitoring data in Chicago with at least 30\% label=positive, and at least 20\% black patients}''). 
Collecting a fresh sample from a data view is costly (monetary, human resources, and/or computation cost)~\cite{asudeh2022towards}.
Therefore, {\sc DT} focuses on satisfying the count requirements with minimum cost. 
Given an input query and a lake of available data sources, the first step is to discover a collection of candidate data views that satisfy the target schema.
Each data view $v_i$ is a projection-join $v_i = \Pi\big(D_{i1}\bowtie\cdots\bowtie D_{ik_i} \big)$, where $D_{ij}$ is a data set in a given data lake.
Let us suppose the data views are already discovered.
At a high level, {\sc DT} follows an iterative approach that at each iteration a data view is selected to be queried.
Each query to a data view has a fixed cost and returns a sample that may or may not satisfy the query constraints.
The samples that are either not fresh, or do not satisfy the query are discarded.
Hence, the essential question towards a cost-effective data integration is {\it what data view to query next}.
Depending on the available information about the data sources, various techniques may be employed. 

For the cases when the group distributions are known, the process of collecting the target data set is a sequence of iterative steps, where at every step, the algorithm chooses a data view, queries it, and if the obtained tuple contributes to one of the groups for which the count requirement is not yet fulfilled, it is kept, otherwise discarded. To do so, a {Dynamic Programming (DP)} algorithm is proposed. An optimal source at each iteration minimizes the sum of its sampling cost plus the expected cost of collecting the remaining required groups, based on its sampling outcome.
The DP algorithm, however, has a pseudo-polynomial time complexity. Hence, it quickly becomes intractable for cases where the minimum count requirements for the groups are not small. 
For cases where the (sensitive) attribute of interest is binary, such as (biological) {\tt sex}={\tt \{male, female\}}, and the cost to query data is similar from all sources, it turns out that the optimal strategy is to query the data source with {maximum probability of obtaining a sample from the minority group}.
Expanding the binary-attributes algorithm for non-binary cases, the problem can be modeled as an extension of the ``{\it coupon collector's}'' problem~\cite{motwani1995randomized}, where the goal is to collect $m_i$ instances from each coupon (group) $\gee_i$.
At each iteration, the coupon collector's algorithm identifies a data view as most promising and queries it. In simple terms, a data view with a smaller query cost and a higher chance of obtaining minority groups is more promising.


For the cases where the group distributions are unknown, we model DT as a {\it multi-armed bandit} problem, where every data view is modeled as an arm. 
Every arm has an unknown distribution of different groups while pulling an arm (i.e., querying the corresponding data view) has a cost.
During various iterations, the algorithms pull the arms in an order that its expected total {\it reward} is maximized.
Arguing that the reward of obtaining a tuple from a group is proportional to how rare this group is across different data views, 
we design the reward function based on the expected cost one needs to pay in order to collect a tuple from a specific group.  
As the bandit strategy, we adopt {\it Upper Confidence Bound (UCB)} to balance exploration and exploitation. At every iteration, for every arm, UCB computes confidence intervals for the expected reward and selects the arm with the maximum upper bound of reward to be explored next.

\subsection{Data Augmentation using Foundation Models}

While data integration provides a promising approach for resolving coverage issues in a data set, its effectiveness is limited to the availability of external data sources that are rich enough to find sufficient fresh samples from minority groups. This, however, is not always possible, especially since the minority samples are rare and not easy to obtain.
Fortunately, recent advancements in Generative AI and Foundation Models have enabled synthesizing samples that are otherwise challenging to obtain from the real world.

Therefore, as an alternative approach to data integration, we turn our attention to the Foundation Models and Generative AI for resolving the lack of coverage. 
Particularly, models such as {\sc DALL.E}\footnote{\url{https://openai.com/dall-e-2}} have emerged as powerful tools for generating multi-modal data such as image, audio, and video.
 
We formalize the foundation model \fm as a black-box function with the following inputs, that once queried synthesize an output tuple.
\begin{itemize}
    \item {\bf Prompt}: A natural language description providing instructions on the details of the tuple to be generated. For instance, a prompt for image generation might be ``A realistic photo of a white cat running in a backyard.''
    \item {\bf Guide}: In cases where only a prompt is provided, the foundation model uses its imagination to generate the requested tuple. For the previous example, the prompt of a cat image, the breed, size, background, and other details are generated based on the model's imagination. Alternatively, a guide can be provided to influence the generation process. The guide is formalized as a pair $(t,m)$ where $t$ is a tuple and $m$ is a mask specifying which parts of the guide tuple should be changed. Using the cat example, $t$ can be a cat image and $m$ can specify the foreground to be regenerated.
\end{itemize}

There are multiple challenges towards effective data set augmentations using foundation models. 
First, we have to determine the minimal set of synthetic tuples that once added to the original data set, under-representation issues are resolved.
Second, the generated images should follow the underlying distribution represented in the input data set. Third, the generated tuples should have high quality and look realistic to a human evaluator. Last but not least, given the (often monetary) cost associated with the queries to the foundation model, we should ensure the cost-effectiveness of the data set repair process.

\begin{wrapfigure}{L}{0.45\textwidth}
\centering
\vspace{-3mm}
\scriptsize
    \includegraphics[width=.45\textwidth]{submissions/submission1/shahbazi/enhanced_pipeline.png}
\vspace{-3mm}
\caption{\small Architecture of \fmsystem for image data augmentation for coverage enhancement.}\label{fig:chameleon}
% \vspace{-3mm}
\end{wrapfigure}

\noindent Figure~\ref{fig:chameleon} shows the architecture of our system \fmsystem \cite{chameleon} for coverage enhancement using DALL-E image generator.
To address the first challenge, we define the combinations-selection problem, which minimizes the total number of synthetic tuples for resolving lack of coverage of minorities at the most general level. We show the problem is {\sc NP}-hard, and propose a greedy approximation algorithm for it.
To address the second and third challenges, \fmsystem follows a {\it rejection sampling} strategy.
It views each tuple in the data set $\dee$ as an iid sample from the underlying distribution $\xi$ it represents. It uses the vector representations (embeddings) space to describe the distribution. Then, given a newly generated tuple, it employs the one-class support vector machine (OCSVM) approach proposed by Scholkopf et al.~\cite{scholkopf1999support} to reject the tuple if it does not follow $\xi$.
Moreover, it models the quality evaluation as hypothesis testing and rejects the samples that have a higher chance of being labeled as ``unrealistic'' by a random human evaluator.
Finally, to minimize the number of queries to the foundation model, we provide a guide tuple (and a mask), in addition to the prompt, to the foundation model. We model the guide-selection problem as {\it contextual multi-armed bandit} and propose a solution based on the contextual UCB for it.

Before concluding this section, let us provide some experiment results to demonstrate the effectiveness of data augmentation with \fmsystem. We use FERET DB \cite{phillips1998feret} for this experiment, which comprises 1199 individual images and serves as a standardized facial image database for researchers to develop algorithms and report results. All images in FERET DB share the same dimensions, pose, and facial expression.
First, we identified the (level-1) uncovered ethnicity groups, using the threshold 80. We then used \fmsystem and resolved the lack of coverage issues.
To evaluate the effectiveness of the system, we trained a CNN model to predict the race of each image within this dataset. We then retrained the identical CNN on the repaired training data. Importantly, our test dataset for both experiments remains consistent and is derived from real images.
Table~\ref{tab:lackofcoverage} presents the improvements in precision, recall, and F1 score metrics for under-represented groups after repairing the dataset. The results indicate an enhancement in performance metrics for all under-represented groups following the repair process.

\begin{table}[t]
    \centering
    \caption{Illustrating the effect of lack of coverage repair using \fmsystem on \texttt{FERTDB}}
    \label{tab:lackofcoverage}
    \vspace{-3mm}
    \begin{tabular}{lcccccccc}
        \toprule
         & \multicolumn{4}{c}{\textbf{Classifier Performance on \texttt{FERTDB}}} & \multicolumn{4}{c}{\textbf{Classifier Performance on Repaired}} \\
        \cmidrule(lr){2-5} \cmidrule(lr){6-9}
        \textbf{Ethnicity Groups}& \#Images & Precision & Recall & F1-Score & \#Images & Precision & Recall & F1-Score \\
        \midrule
        Overall          & 756 & 0.81 & 0.75 & 0.78 & 987 & 0.70 & 0.75 & 0.72 \\ \hline
        Black            & 40  & 0.19 & 0.22 & 0.16 & 100 & 0.48 & 0.56 & 0.52 \\
        Hispanic         & 19  & 0.50 & 0.17 & 0.25 & 100 & 0.62 & 0.36 & 0.45 \\
        Middle Eastern   & 10  & 0.00 & 0.00 & 0.00 & 100 & 0.20 & 0.41 & 0.27 \\
        \bottomrule
    \end{tabular}
\end{table}

 %%%%%%%%%%%%%%%%%%%%%%%%%%%%%%%% RELIABILITY  %%%%%%%%%%%%%%%%%%%%%%%%%%%%%%%%
\section{Generating Reliability Warnings}\label{sec:reliability}
% up to 2.5 pages
Interpretability is a necessity for data scientists who develop predictive models for critical decision-making.
In such settings, it is important to provide additional means to support the following question:
{\it is an individual prediction of the model reliable for decision-making?} Our goal is to use the lack of representation to help decision-makers find insights about this critical question.
To further motivate this, let us use the following example:

\vspace{1mm}
\begin{example}\label{ex-0}
{\bf(Part1):} Consider a judge who needs to decide whether to accept or deny a bail request. Using data-driven predictive models is prevalent in such cases for predicting recidivism~\cite{dressel2018accuracy}.
Indeed, such models can be beneficial to help the judge make wise decisions.
Suppose the model predicts the queried individual as high risk (or low risk).
The judge is aware and concerned about the critics surrounding such models.
A major question the judge faces is whether or not they should rely on the prediction outcome to take action for this case.
Furthermore, if, for instance, they decide to ignore the outcome and hence they need to provide a statement supporting their action, what evidence can they provide? 
\end{example}

In line with the recent trend on data-centric AI~\cite{ng2021mlops}, we design {novel approaches}, {complimentary} to the existing work on trustworthy AI~\cite{wing2021trustworthy,kentour2021analysis,liu2021trustworthy,singh2021trustworthy}, to address the aforementioned trust question through the lens of {\it data}.
In particular, unlike existing works that generate trust information from a {\it given \underline{model}}, we associate {\it \underline{data sets} with proper measurements} that specify their {\it the scope of use for predicting future cases}.
We note that a predictive model provides only probabilistic guarantees on the \underline{average} loss over the distribution represented by the data set used for training it.
As a result, these predictions may not be distribution generalizable~\cite{kulynych2022you}.
Consequently, if the query point is {\it not represented} by the data, the guarantees may not hold, hence one cannot rely on the prediction outcome.
Besides, an essential requirement for a learning algorithm is that its training data $\dee$ should represent the underlying distribution $\dist$.
Even if so, the trained model $h$ only provides a probabilistic guarantee on the {expected} loss on random samples from $\dist$.  
A model that performs well on {\it majority} of samples drawn from $\dist$ will have a high performance on average. Still, as we observed in Figure~\ref{fig:mlfails},
its performance for {\it minorities} and points that are not represented is questionable. Let us consider the following toy example:

\begin{figure*}[!b] 
    \begin{minipage}[t]{0.32\linewidth}
        	\centering
        	\includegraphics[width=\textwidth]{submissions/submission1/shahbazi/example_1.png} 
        	\vspace{-9mm}\caption{\small Data set $\dee$ generated using a Gaussian distribution; $x_1$ and $x_2$ are positively correlated}
            \label{fig:ex1:1}
    \end{minipage}
    \hfill
    \begin{minipage}[t]{0.32\linewidth}
        \centering
        	\includegraphics[width =\textwidth]{submissions/submission1/shahbazi/example_2.png} 
        	\vspace{-9mm}\caption{\small The decision boundary of learned model $h$ and query points $\qu^1$ to $\qu^4$}
            \label{fig:ex1:2}
    \end{minipage}
    \hfill
    \begin{minipage}[t]{0.32\linewidth}
        	\centering
        	\includegraphics[width =\textwidth]{submissions/submission1/shahbazi/example_3.png}
        	\vspace{-9mm}\caption{\small Ground-truth boundary, overlaid on the model decision boundary and query points}
            \label{fig:ex1:3}
    \end{minipage}
    \vspace{-5mm}
\end{figure*} 

\vspace{1mm}
\begin{example}\label{ex-1}
Consider a binary classification task where the input space is $\ex=\langle x_1, x_2\rangle$ and the output space is the binary label $y$ with values $\{-1$ (red) $,+1$ (blue)$\}$.
Suppose the underlying data distribution $\dist$ follows a 2D Gaussian, where $x_1$ and $x_2$ 
are positively correlated as shown in Figure~\ref{fig:ex1:1}.
The figure shows the data set $\dee$ drawn independently from the distribution $\dist$, along with their labels as their colors.
Using $\dee$, the prediction model $h$ is constructed as shown in Figure~\ref{fig:ex1:2}. 
The decision boundary is specified in the picture; while any point above the line is predicted as +1, a query point below it is labeled as -1.
The classifier has been evaluated using a test set that is an iid sample set drawn from the underlying data set $\dist$. The accuracy on the test set is high (above 90\%), and hence, the model gets deployed.
We cherry-picked four query points, $\qu^1$ to $\qu^4$, that are also included in Figure~\ref{fig:ex1:2}. Using $h$ for prediction, $h(\qu^1)=-1$, $h(\qu^2)=+1$,  $h(\qu^3)=+1$, and $h(\qu^4)=-1$.
Figure~\ref{fig:ex1:3} adds the ground-truth boundary to the search space, revealing the true label of the query points: every point inside the red circle has the true label $-1$ while any point outside of it is $+1$.
Looking at the figure, $y^1=+1$ while the model predicted it as $h(\qu^1)=-1$.  \hfill$\square$
\end{example}
\vspace{2mm}

Let us take a closer look at the four query points in this example and their placement with regard to the tuples in $\dee$ used for training $h$. 
$\qu^2$ belongs to a {\it dense region} with many training tuples in $\dee$ surrounding it. Besides, all of the tuples in its vicinity have the same label $y=+1$. As a result, one can expect that the model's outcome $h(\qu^2)=+1$ should be a reliable prediction.
Similar to $\qu^2$, $\qu^4$ also belongs to a dense region in $\dee$; however, $\qu^4$ belongs to an {\it uncertain region}, where some of the tuples in its vicinity have a label $y=+1$, and some others have the label $y=-1$. Considering the uncertainty in the vicinity of $\qu^4$, one cannot confidently rely on the outcome of the model $h$. 
On the other hand, the neighbors of $\qu^1$ (resp. $\qu^3$) are not uncertain, all having the label $y=-1$ (resp. $y=+1$).
However, the query points $\qu^1$ and $\qu^3$ are not well represented by $\dee$. In other words, $\qu^1$ and $\qu^3$ are unlikely to be generated according to the underlying distribution $\dist$, represented by $\dee$. As a result, following the no-free-lunch theorem~\cite{kakade2003sample}, one cannot expect the outcome of model $h$ to be reliable for these points.
Looking at the ground-truth boundary in Figure~\ref{fig:ex1:3}, $h$ luckily predicted the outcome for $\qu^3$ correctly, but it was not fortunate to predict the $y^1$ correctly.
Nevertheless, 
since the model is not reliably trained for these points, 
its outcome for these query points is not trustworthy.

From Example~\ref{ex-1}, we observe that the outcome of a model $h$, trained using a data set $\dee$ is not reliable for a query point $\qu$, if:
\begin{itemize}
    \item {\bf Lack of representation:} $\qu$ is not well-represented by $\dee$.
    In such cases, the model has not seen ``enough'' samples similar to $\qu$ to reliably learn and predict the outcome of $\qu$.
    \item {\bf Lack of certainty:} $\qu$ belongs to an uncertain region, where different tuples of $\dee$ in the vicinity of $\qu$ have different target values. $\qu$ belongs to a high-fluctuating area, where tuples in the vicinity of $\qu$ have a wide range of values.
\end{itemize} \vspace{2mm}

\noindent
Based on these two observations, we propose Representation-and-Uncertainty ({\bf RU}) measures.
To identify if a query suffers from uncertainty or lack of representation, one could use a deterministic approach using a fixed threshold. Then if the number of similar samples to (resp. label fluctuation in vicinity of) $\qu$ is larger than the threshold it is considered as unrepresented (resp. uncertain).
This approach, however, would be misleading since two numbers close to the threshold could be treated very differently. Also, all points on each side of the threshold would be considered equally represented (resp., certain). Instead, we consider {\it a randomized approach}, widely popular in the literature, including~\cite{dwork2012fairness}.
That is, instead of using fixed thresholds, a Bernoulli variable (a biased coin) is used that 
assigns $\qu$ as unrepresented (resp., uncertain) based on the number of samples similar to it (resp., its neighborhood uncertainty).
Given a query point $\qu$, let $\pe_o$ be the probability indicating if $\qu$ is not represented and let $\pe_u$ be the probability indicating if $\qu$ belongs to an uncertain region. 
We represent the probability of the Bernoulli variables for lack of representation or uncertainty components as $\pe_o$ and $\pe_u$, respectively. Note that the two Bernoulli variables $\pe_o$ and $\pe_u$ are independent from each other. That simply follows the argument that after specifying the number of similar samples to $\qu$ whether or not it should be considered as unrepresented does not depend on the uncertainty in the neighborhood of $\qu$.

\begin{definition}[\sru]\label{def:sdt}
The \sru is a probabilistic measure that considers the outcome of a model for a query point $\qu$ untrustworthy if $\qu$ is not represented by $\dee$ {\it and} it belongs to an uncertain region.
Formally, the \sru measure is:
\begin{align} 
    \nonumber
    SRU(\qu) &= \pe\big((\qu \mbox{ is outlier}) \wedge (\qu \mbox{ belongs to uncertain region})\big) 
\end{align}
Since $\pe_o$ and $\pe_u$ are independent:

\vspace{-13mm}
\begin{align} \label{eq:strong}
    SRU(\qu) &= \pe_o(\qu) \times \pe_u(\qu)
\end{align}
\end{definition}

\sru raises the warning signal only when the query point fails on {\it both} conditions of being represented by $\dee$ and not belonging to an uncertain region. 
For instance, in Example~\ref{ex-1} none of the query points fail both on representation and on uncertainty; hence neither has a high \sru score.
On the other hand, 
a high \sru score for a query point $\qu$ {\it provides a strong warning signal} that one should perhaps reject the model outcome and not consider it for decision-making.

\sru is a strong signal that raises warnings only for the fearfully concerning cases that fail both on representation and uncertainty.
However, as observed in Example~\ref{ex-1} a query points failing {\it at least} one of these conditions may also not be reliable, at least for critical decision making.
We define the \wru measure to raise a warning for such cases.

\begin{definition}[\wru]\label{def:wdt}
The \wru measure is a probabilistic measure that considers the outcome of a model for a query point $\qu$ untrustworthy if $\qu$ is not represented by $\dee$ {\bf or} it belongs to an uncertain region.
Formally, the \wru is computed as:
\begin{align} \label{eq:weak}
    WRU(\qu) = \pe\big((\qu \mbox{ is outlier}) \vee (\qu \mbox{ belongs to uncertain region})\big) 
    = \pe_o(\qu) + \pe_u(\qu) - \pe_o(\qu) \times \pe_u(\qu)
\end{align}
\end{definition}

Proposing quantitative probabilistic outcomes, \ru measures are interpretable for the users, since beyond the scores, the uncertainty and lack of representation components provide an explanation to justify them. 
Please refer to \cite{techrep} for more details on how to efficiently and effectively compute the representation ($\pe_o$) and uncertainty ($\pe_u$) probabilities, using only $\dee$.
In Example~\ref{ex-0}, let us see how the \ru measures can be helpful.

\noindent{\bf Example 1. (part 2):}
{\it RU measures \underline{raise warning} when
the fitness of the data set used for drawing a prediction is questionable, helping the judge to be cautious when taking action.
Besides, these measures provide \underline{quantitative evidence} to support the judge's action when they decide to ignore a prediction outcome that is not trustworthy.
The judge, for example, can argue to ignore a model outcome for a specific case, based on the insight that 
the model has been built using a
data set that fails to represent the given case.}
\hfill$\square$

Finally, let us demonstrate the efficacy of \ru measures through a series of experiments. Since the \ru measures are {\it data-centric},
those are applicable for both classification and regression tasks, irrespective of the model used.
We use {\it Adult} dataset~\cite{adult} for classification and {\it House Sales in King County} dataset for the validation of regression tasks. From each dataset, we uniformly sample two sets from the underlying distribution. The first set serves as the training set to compute the \ru values, and the second one is used as the test set from which the queries are drawn. We validate our proposal by providing the correlation between the \ru values and the performance of an ML model's prediction on the same data. 

We start by computing the \ru values for all the query points in the test set. Next, we bucketize the query points based on their \ru values in equi-width buckets of width 0.1. We repeat this for both \sru and \wru measures. Next, we train a model on the training data set and predict the target variable for the points in each range of \ru measure. The validation results for the classification task on the {\it Adult} dataset are presented in Figures \ref{fig:exp-adult-sdt} and \ref{fig:exp-adult-wdt}. Each figure corresponds to the accuracy/error measures of the classifier over each bucket of \ru values for \sru and \wru. As the \ru values increase, the accuracy of the model drops while the FPR rises, and therefore, the model fails to capture the ground truth for the points that fall into untrustworthy regions in the data set. By repeating the aforementioned steps for the regression task on the {\it House Sales in King County} dataset, we observe similar results presented in Figures \ref{fig:exp-hs-sdt} and \ref{fig:exp-hs-wdt}. 
As the \ru value increases, the RSS of the regression model follows the same trend denoting that the model fails to perform for tuples with a high \ru value.

\begin{figure}[!tb]
    \begin{minipage}[t]{0.24\linewidth}
        \centering
        \includegraphics[width=\textwidth]{submissions/submission1/shahbazi/sdt_adult.pdf}
        \vspace{-6mm}\caption{\small{\it Adult}, efficacy of \sru  on classification}
        \label{fig:exp-adult-sdt}
    \end{minipage}\hfill
    \begin{minipage}[t]{0.24\linewidth}
        \centering
        \includegraphics[width=\textwidth]{submissions/submission1/shahbazi/wdt_adult.pdf}
        \vspace{-6mm}\caption{\small{\it Adult}, efficacy of \wru  on classification}
        \label{fig:exp-adult-wdt}
    \end{minipage}\hfill
    \begin{minipage}[t]{0.24\linewidth}
        \centering
        \includegraphics[width=\textwidth]{submissions/submission1/shahbazi/sdt_regression_house.pdf}
        \vspace{-6mm}\caption{\small{\it House Sales in King County}, efficacy of \sru on regression}
        \label{fig:exp-hs-sdt}
    \end{minipage}\hfill
    \begin{minipage}[t]{0.24\linewidth}
        \centering
        \includegraphics[width=\textwidth]{submissions/submission1/shahbazi/wdt_regression_house.pdf}
        \vspace{-6mm}\caption{\small{\it House Sales in King County}, efficacy \wru on regression}
        \label{fig:exp-hs-wdt}
    \end{minipage}
\vspace{-5mm}
\end{figure}
 %%%%%%%%%%%%%%%%%%%%%%%%%%%%%%%% RELATED WORK  %%%%%%%%%%%%%%%%%%%%%%%%%%%%%%%%
\section{Related Work}\label{related} 

Bias in data has been looked at for a long time in statistical community~\cite{neyman1936contributions} but social data presents different challenges~\cite{olteanu2019social,fairmlbook,barocas2016big,jk2019bias,drosou2017diversity}.
The diversity and representativeness of data have been widely studied~\cite{drosou2017diversity}, in fields such as social science~\cite{berrey2015enigma, dobbin2016diversity,simpson1949measurement}, political science~\cite{surowiecki2005wisdom}, and information retrieval~\cite{agrawal2009diversifying}. 
Tracing back machine bias to its source, there have been major efforts to identify different types~\cite{mehrabi2021survey, olteanu2019social,friedman1996bias} and sources~\cite{torralba2011unbiased,crawford2013hidden,diakopoulos2015algorithmic} of biases in data. Efforts to satisfy {\it responsible data} requirements~\cite{nargesian2022responsible} extend to various stages of the data analysis pipeline, including data annotation~\cite{li2020towards,lazier2023fairness}, data cleaning and repair~\cite{SalimiRHS19,tae2019data,salimi2020database}, data imputation~\cite{martinez2019fairness}, entity resolution~\cite{shahbazi2023through,fanourakis2023fairer}, data integration~\cite{nargesian2022responsible,nargesian2021tailoring}, etc. 

\paragraph{Data Coverage:}The notion of data coverage has received extensive attention from different angles. Detecting lack of coverage has been studied for datasets with discrete~\cite{asudeh2019assessing} and continuous~\cite{asudeh2021coverage} attributes populated in single or multiple \cite{lin2020identifying} relations.
To resolve insufficient coverage, \cite{accinelli2020coverage, accinelli2021impact,shetiya2022fairness}
consider resolving representation bias in preprocessing pipelines by rewriting queries into the closest operation so that certain subgroups are sufficiently represented in the downstream tasks. Alternatively, ~\cite{asudeh2019assessing,tae2021slice} propose a data collection strategy to acquire as little additional data as possible (to minimize the associated costs) to meet the representation constraints. ~\cite{sharma2020data,iosifidis2018dealing,celis2020data} opt for a data augmentation approach by adding partially altered duplicates of already existing tuples or generating new synthetic entries from existing data. Consequently, the new data set has an equal number of elements for different groups, resulting in potentially resolving the under-representation issues. Finally,  \cite{nargesian2021tailoring} utilizes data integration techniques to consolidate data from different sources into a single dataset to resolve representation bias.
Related works also include ~\cite{chung2019slice,sagadeeva2021sliceline,tae2021slice} that seek to understand if the overall performance of the model fails to reflect and performs poorly on certain slices in the data.
As alternative approaches to measure representation bias, the notion of representation rate~\cite{celis2020data} (a.k.a. equal base rate~\cite{kleinberg2016inherent}) is introduced which compared with coverage, it is more restrictive as it requires almost equal ratios from different groups.
Please refer to \cite{shahbazi2023representation} for a comprehensive survey about representation bias in data. 

\paragraph{ML Reliability:} Model-centric works for uncertainty quantification such as 
probabilistic classifiers~\cite{zadrozny2001obtaining,zadrozny2002transforming,platt1999probabilistic,niculescu2005predicting},
prediction intervals (PIs) \cite{chatfield93predictionintervals,pearce2018high,khosravi2010lower} and conformal predictions (CP)~\cite{angelopoulos2021gentle,shafer2008tutorial} that are used for measuring prediction uncertainty, are built
by maximizing the {\it expected performance} on {\it random} sample from the underlying distribution.
As a result, while providing accurate estimations for the dense regions of data (e.g. majority groups), their estimation accuracy is questionable for the poorly represented regions.
In particular, \cite{angelopoulos2021gentle} recognizes the lack of guarantees in the performance of CP for such regions.
Besides, the bulk of work on trustworthy AI provides information that {\it supports} the outcome of an ML model. For example, existing work on explainable AI, including~\cite{harradon2018causal,ribeiro2016should,gunning2019darpa}, aims to find simple explanations and rules that justify the outcome of a model.
Conversely, we aim to {\it raise warning signals} when the outcome of a model is {\it not} trustworthy. That is, to provide reasons that {\it cast doubt} on the reliability of the model outcome {for a given query point}.

 %%%%%%%%%%%%%%%%%%%%%%%%%%%%%%%% FUTURE  %%%%%%%%%%%%%%%%%%%%%%%%%%%%%%%%
% \vspace{-3mm}
\section{Final Remarks}\label{sec:conclusion}
As Data-centric AI and Responsible AI emerge as focal points in data science research, the development of Data-centric methodologies for ensuring Responsible and Trustworthy AI attracts increasing attention.
While there is some excellent work on responsible data management to achieve this goal, there remain many challenges yet to be addressed.

In this paper, we focused on a crucial aspect of responsible data -- detecting and addressing the under-representation of minorities within a data set.
We formally defined the notion of data coverage and discussed various techniques for (a) identifying lack of representation issues across different data modalities, (b) ensuring proper representation of minorities in data, and (c) limiting the scope-of-use of data sets based on their representation issues by generating proper ({\sc RU}) warning signals.
Even though the research on detecting lack of coverage issues is relatively mature, resolution techniques are still understudied.
Considering the recent advancements in Generative AI, utilizing Foundation Models and Large Language Models, and studying their limitations, for data augmentation to improve the representation of minorities at the data level seems interesting to further explore.

 %%%%%%%%%%%%%%%%%%%%%%%%%%%%%%%% BIB  %%%%%%%%%%%%%%%%%%%%%%%%%%%%%%%%
\bibliographystyle{unsrt}
\small
% \bibliography{ref}
\begin{thebibliography}{10}

\bibitem{asudeh2019assessing}
A.~Asudeh, Z.~Jin, and H.~Jagadish.
\newblock Assessing and remedying coverage for a given dataset.
\newblock In {\em ICDE}, pages 554--565. IEEE, 2019.

\bibitem{shahbazi2023representation}
N.~Shahbazi, Y.~Lin, A.~Asudeh, and H.~Jagadish.
\newblock Representation bias in data: A survey on identification and resolution techniques.
\newblock {\em ACM Computing Surveys}, 2023.

\bibitem{asudeh2021coverage}
A.~Asudeh, N.~Shahbazi, Z.~Jin, and H.~V. Jagadish.
\newblock Identifying insufficient data coverage for ordinal continuous-valued attributes.
\newblock In {\em SIGMOD}. ACM, 2021.

\bibitem{mousavi2024data}
M.~Mousavi, N.~Shahbazi, and A.~Asudeh.
\newblock Data coverage for detecting representation bias in image datasets: {A} crowdsourcing approach.
\newblock In {\em {EDBT}}, pages 47--60, 2024.

\bibitem{nargesian2021tailoring}
F.~Nargesian, A.~Asudeh, and H.~Jagadish.
\newblock Tailoring data source distributions for fairness-aware data integration.
\newblock {\em Proceedings of the VLDB Endowment}, 14(11):2519--2532, 2021.

\bibitem{nargesian2022responsible}
F.~Nargesian, A.~Asudeh, and H.~V. Jagadish.
\newblock Responsible data integration: Next-generation challenges.
\newblock {\em SIGMOD}, 2022.

\bibitem{sharma2020data}
S.~Sharma, Y.~Zhang, J.~M. R{\'\i}os~Aliaga, D.~Bouneffouf, V.~Muthusamy, and K.~R. Varshney.
\newblock Data augmentation for discrimination prevention and bias disambiguation.
\newblock In {\em AIES}, pages 358--364, 2020.

\bibitem{DBLP:journals/jair/ChawlaBHK02}
N.~V. Chawla, K.~W. Bowyer, L.~O. Hall, and W.~P. Kegelmeyer.
\newblock {SMOTE:} synthetic minority over-sampling technique.
\newblock {\em J. Artif. Intell. Res.}, 16:321--357, 2002.

\bibitem{iosifidis2018dealing}
V.~Iosifidis and E.~Ntoutsi.
\newblock Dealing with bias via data augmentation in supervised learning scenarios.
\newblock {\em Jo Bates Paul D. Clough Robert J{\"a}schke}, 24, 2018.

\bibitem{celis2020data}
L.~E. Celis, V.~Keswani, and N.~Vishnoi.
\newblock Data preprocessing to mitigate bias: A maximum entropy based approach.
\newblock In {\em ICML}, pages 1349--1359. PMLR, 2020.

\bibitem{asudeh2022towards}
A.~Asudeh and F.~Nargesian.
\newblock Towards distribution-aware query answering in data markets.
\newblock {\em Proceedings of the VLDB Endowment}, 15(11):3137--3144, 2022.

\bibitem{motwani1995randomized}
R.~Motwani and P.~Raghavan.
\newblock {\em Randomized algorithms}.
\newblock Cambridge university press, 1995.

\bibitem{chameleon}
M.~Erfanian, H.~V. Jagadish, and A.~Asudeh.
\newblock Chameleon: Foundation models for fairness-aware multi-modal data augmentation to enhance coverage of minorities.
\newblock {\em arXiv preprint arXiv:2402.01071}, 2024.

\bibitem{scholkopf1999support}
B.~Sch{\"o}lkopf, R.~C. Williamson, A.~Smola, J.~Shawe-Taylor, and J.~Platt.
\newblock Support vector method for novelty detection.
\newblock {\em NeurIPS}, 12, 1999.

\bibitem{phillips1998feret}
P.~J. Phillips, H.~Wechsler, J.~Huang, and P.~J. Rauss.
\newblock The feret database and evaluation procedure for face-recognition algorithms.
\newblock {\em Image and vision computing}, 16(5):295--306, 1998.

\bibitem{dressel2018accuracy}
J.~Dressel and H.~Farid.
\newblock The accuracy, fairness, and limits of predicting recidivism.
\newblock {\em Science advances}, 4(1):eaao5580, 2018.

\bibitem{ng2021mlops}
A.~Ng.
\newblock Mlops: From model-centric to data-centric {AI}.
\newblock 2021.

\bibitem{wing2021trustworthy}
J.~M. Wing.
\newblock Trustworthy {AI}.
\newblock {\em CACM}, 64(10):64--71, 2021.

\bibitem{kentour2021analysis}
M.~Kentour and J.~Lu.
\newblock Analysis of trustworthiness in machine learning and deep learning.
\newblock {\em InfoComp}, 2021.

\bibitem{liu2021trustworthy}
H.~Liu, Y.~Wang, W.~Fan, X.~Liu, Y.~Li, S.~Jain, A.~K. Jain, and J.~Tang.
\newblock Trustworthy {AI}: A computational perspective.
\newblock {\em arXiv preprint arXiv:2107.06641}, 2021.

\bibitem{singh2021trustworthy}
R.~Singh, M.~Vatsa, and N.~Ratha.
\newblock Trustworthy {AI}.
\newblock In {\em 8th ACM IKDD CODS and 26th COMAD}, pages 449--453. 2021.

\bibitem{kulynych2022you}
B.~Kulynych, Y.-Y. Yang, Y.~Yu, J.~B{\l}asiok, and P.~Nakkiran.
\newblock What you see is what you get: Distributional generalization for algorithm design in deep learning.
\newblock {\em arXiv preprint arXiv:2204.03230}, 2022.

\bibitem{kakade2003sample}
S.~M. Kakade.
\newblock {\em On the sample complexity of reinforcement learning}.
\newblock University of London, University College London (United Kingdom), 2003.

\bibitem{dwork2012fairness}
C.~Dwork, M.~Hardt, T.~Pitassi, O.~Reingold, and R.~Zemel.
\newblock Fairness through awareness.
\newblock In {\em ITCS}, pages 214--226, 2012.

\bibitem{techrep}
N.~Shahbazi and A.~Asudeh.
\newblock Data-centric reliability evaluation of individual predictions.
\newblock {\em CoRR, abs/2204.07682}, 2022.

\bibitem{adult}
M.~Lichman.
\newblock Adult income dataset, {UCI} machine learning repository.
\newblock \url{https://archive.ics.uci.edu/ml/datasets/adult}, 2013.

\bibitem{neyman1936contributions}
J.~Neyman and E.~S. Pearson.
\newblock Contributions to the theory of testing statistical hypotheses.
\newblock {\em Statistical Research Memoirs}, 1936.

\bibitem{olteanu2019social}
A.~Olteanu, C.~Castillo, F.~Diaz, and E.~Kiciman.
\newblock Social data: Biases, methodological pitfalls, and ethical boundaries.
\newblock {\em Frontiers in Big Data}, 2:13, 2019.

\bibitem{fairmlbook}
S.~Barocas, M.~Hardt, and A.~Narayanan.
\newblock Fairness and machine learning: Limitations and opportunities.
\newblock \url{fairmlbook.org}, 2019.

\bibitem{barocas2016big}
S.~Barocas and A.~D. Selbst.
\newblock Big data's disparate impact.
\newblock {\em Calif. L. Rev.}, 104:671, 2016.

\bibitem{jk2019bias}
J.~Kleinberg.
\newblock Fairness, rankings, and behavioral biases.
\newblock FAT*, 2019.

\bibitem{drosou2017diversity}
M.~Drosou, H.~Jagadish, E.~Pitoura, and J.~Stoyanovich.
\newblock Diversity in big data: A review.
\newblock {\em Big data}, 5(2):73--84, 2017.

\bibitem{berrey2015enigma}
E.~Berrey.
\newblock {\em The enigma of diversity: The language of race and the limits of racial justice}.
\newblock University of Chicago Press, 2015.

\bibitem{dobbin2016diversity}
F.~Dobbin and A.~Kalev.
\newblock Why diversity programs fail and what works better.
\newblock {\em Harvard Business Review}, 94(7-8):52--60, 2016.

\bibitem{simpson1949measurement}
E.~H. Simpson.
\newblock Measurement of diversity.
\newblock {\em Nature}, 163(4148), 1949.

\bibitem{surowiecki2005wisdom}
J.~Surowiecki.
\newblock {\em The wisdom of crowds}.
\newblock Anchor, 2005.

\bibitem{agrawal2009diversifying}
R.~Agrawal, S.~Gollapudi, A.~Halverson, and S.~Ieong.
\newblock Diversifying search results.
\newblock In {\em WSDM}, pages 5--14. ACM, 2009.

\bibitem{mehrabi2021survey}
N.~Mehrabi, F.~Morstatter, N.~Saxena, K.~Lerman, and A.~Galstyan.
\newblock A survey on bias and fairness in machine learning.
\newblock {\em ACM Computing Surveys (CSUR)}, 54(6):1--35, 2021.

\bibitem{friedman1996bias}
B.~Friedman and H.~Nissenbaum.
\newblock Bias in computer systems.
\newblock {\em TOIS}, 14(3):330--347, 1996.

\bibitem{torralba2011unbiased}
A.~Torralba and A.~A. Efros.
\newblock Unbiased look at dataset bias.
\newblock In {\em CVPR 2011}, pages 1521--1528. IEEE, 2011.

\bibitem{crawford2013hidden}
K.~Crawford.
\newblock The hidden biases in big data.
\newblock {\em Harvard business review}, 1(4), 2013.

\bibitem{diakopoulos2015algorithmic}
N.~Diakopoulos.
\newblock Algorithmic accountability: Journalistic investigation of computational power structures.
\newblock {\em Digital journalism}, 3(3):398--415, 2015.

\bibitem{li2020towards}
Y.~Li, H.~Sun, and W.~H. Wang.
\newblock Towards fair truth discovery from biased crowdsourced answers.
\newblock In {\em SIGKDD}, pages 599--607, 2020.

\bibitem{lazier2023fairness}
S.~Lazier, S.~Thirumuruganathan, and H.~Anahideh.
\newblock Fairness and bias in truth discovery algorithms: An experimental analysis.
\newblock {\em arXiv preprint arXiv:2304.12573}, 2023.

\bibitem{SalimiRHS19}
B.~Salimi, L.~Rodriguez, B.~Howe, and D.~Suciu.
\newblock Interventional fairness: Causal database repair for algorithmic fairness.
\newblock In {\em {SIGMOD}}, pages 793--810. {ACM}, 2019.

\bibitem{tae2019data}
K.~H. Tae, Y.~Roh, Y.~H. Oh, H.~Kim, and S.~E. Whang.
\newblock Data cleaning for accurate, fair, and robust models: A big data-{AI} integration approach.
\newblock In {\em DEEM workshop}, pages 1--4, 2019.

\bibitem{salimi2020database}
B.~Salimi, B.~Howe, and D.~Suciu.
\newblock Database repair meets algorithmic fairness.
\newblock {\em ACM SIGMOD Record}, 49(1):34--41, 2020.

\bibitem{martinez2019fairness}
F.~Mart{\'\i}nez-Plumed, C.~Ferri, D.~Nieves, and J.~Hern{\'a}ndez-Orallo.
\newblock Fairness and missing values.
\newblock {\em arXiv preprint arXiv:1905.12728}, 2019.

\bibitem{shahbazi2023through}
N.~Shahbazi, N.~Danevski, F.~Nargesian, A.~Asudeh, and D.~Srivastava.
\newblock Through the fairness lens: Experimental analysis and evaluation of entity matching.
\newblock {\em Proceedings of the VLDB Endowment}, 16(11):3279--3292, 2023.

\bibitem{fanourakis2023fairer}
N.~Fanourakis, C.~Kontousias, V.~Efthymiou, V.~Christophides, and D.~Plexousakis.
\newblock Fairer demo: Fairness-aware and explainable entity resolution.
\newblock 2023.

\bibitem{lin2020identifying}
Y.~Lin, Y.~Guan, A.~Asudeh, and H.~Jagadish.
\newblock Identifying insufficient data coverage in databases with multiple relations.
\newblock {\em Proceedings of the VLDB Endowment}, 13(12):2229--2242, 2020.

\bibitem{accinelli2020coverage}
C.~Accinelli, S.~Minisi, and B.~Catania.
\newblock Coverage-based rewriting for data preparation.
\newblock In {\em EDBT Workshops}, 2020.

\bibitem{accinelli2021impact}
C.~Accinelli, B.~Catania, G.~Guerrini, and S.~Minisi.
\newblock The impact of rewriting on coverage constraint satisfaction.
\newblock In {\em EDBT Workshops}, 2021.

\bibitem{shetiya2022fairness}
S.~Shetiya, I.~P. Swift, A.~Asudeh, and G.~Das.
\newblock Fairness-aware range queries for selecting unbiased data.
\newblock In {\em ICDE}. IEEE, 2022.

\bibitem{tae2021slice}
K.~H. Tae and S.~E. Whang.
\newblock Slice tuner: A selective data acquisition framework for accurate and fair machine learning models.
\newblock In {\em SIGMOD}, pages 1771--1783, 2021.

\bibitem{chung2019slice}
Y.~Chung, T.~Kraska, N.~Polyzotis, K.~H. Tae, and S.~E. Whang.
\newblock Slice finder: Automated data slicing for model validation.
\newblock In {\em ICDE}, pages 1550--1553. IEEE, 2019.

\bibitem{sagadeeva2021sliceline}
S.~Sagadeeva and M.~Boehm.
\newblock Sliceline: Fast, linear-algebra-based slice finding for ml model debugging.
\newblock In {\em SIGMOD}, pages 2290--2299, 2021.

\bibitem{kleinberg2016inherent}
J.~Kleinberg, S.~Mullainathan, and M.~Raghavan.
\newblock Inherent trade-offs in the fair determination of risk scores.
\newblock {\em arXiv preprint arXiv:1609.05807}, 2016.

\bibitem{zadrozny2001obtaining}
B.~Zadrozny and C.~Elkan.
\newblock Obtaining calibrated probability estimates from decision trees and naive bayesian classifiers.
\newblock In {\em ICML}, volume~1, pages 609--616. Citeseer, 2001.

\bibitem{zadrozny2002transforming}
B.~Zadrozny and C.~Elkan.
\newblock Transforming classifier scores into accurate multiclass probability estimates.
\newblock In {\em SIGKDD}, pages 694--699, 2002.

\bibitem{platt1999probabilistic}
J.~Platt et~al.
\newblock Probabilistic outputs for support vector machines and comparisons to regularized likelihood methods.
\newblock {\em Advances in large margin classifiers}, 10(3):61--74, 1999.

\bibitem{niculescu2005predicting}
A.~Niculescu-Mizil and R.~Caruana.
\newblock Predicting good probabilities with supervised learning.
\newblock In {\em Proceedings of the 22nd international conference on Machine learning}, pages 625--632, 2005.

\bibitem{chatfield93predictionintervals}
C.~Chatfield.
\newblock Prediction intervals.
\newblock {\em Journal of Business and Economic Statistics}, 11:121--135, 1993.

\bibitem{pearce2018high}
T.~Pearce, A.~Brintrup, M.~Zaki, and A.~Neely.
\newblock High-quality prediction intervals for deep learning: A distribution-free, ensembled approach.
\newblock In {\em International conference on machine learning}, pages 4075--4084. PMLR, 2018.

\bibitem{khosravi2010lower}
A.~Khosravi, S.~Nahavandi, D.~Creighton, and A.~F. Atiya.
\newblock Lower upper bound estimation method for construction of neural network-based prediction intervals.
\newblock {\em IEEE transactions on neural networks}, 22(3):337--346, 2010.

\bibitem{angelopoulos2021gentle}
A.~N. Angelopoulos and S.~Bates.
\newblock A gentle introduction to conformal prediction and distribution-free uncertainty quantification.
\newblock {\em arXiv preprint arXiv:2107.07511}, 2021.

\bibitem{shafer2008tutorial}
G.~Shafer and V.~Vovk.
\newblock A tutorial on conformal prediction.
\newblock {\em Journal of Machine Learning Research}, 9(3), 2008.

\bibitem{harradon2018causal}
M.~Harradon, J.~Druce, and B.~Ruttenberg.
\newblock Causal learning and explanation of deep neural networks via autoencoded activations.
\newblock {\em arXiv preprint arXiv:1802.00541}, 2018.

\bibitem{ribeiro2016should}
M.~T. Ribeiro, S.~Singh, and C.~Guestrin.
\newblock " why should i trust you?" explaining the predictions of any classifier.
\newblock In {\em SIGKDD}, pages 1135--1144, 2016.

\bibitem{gunning2019darpa}
D.~Gunning and D.~Aha.
\newblock Darpa’s explainable artificial intelligence ({XAI}) program.
\newblock {\em AI Magazine}, 40(2):44--58, 2019.

\end{thebibliography}

\end{document}

\end{article}

\begin{article}
{Responsible AI Challenges in End-to-end Machine Learning}
{Steven Euijong Whang, Ki Hyun Tae, Yuji Roh and Geon Heo}
% link to instruction: https://tc.computer.org/tcde/tcde-bulletin-author-instructions/
% \documentclass[11pt,dvipdfm]{article}
\documentclass[11pt]{article}
\usepackage{tabularx}
\usepackage{ragged2e}  % for '\RaggedRight' macro (allows hyphenation)
\usepackage{booktabs}  % for \toprule, \midrule, and \bottomrule macros
\usepackage{textcomp}
\usepackage{amsfonts,amsmath}
\usepackage{deauthor,times}
\usepackage{graphicx} % 
\usepackage{hyperref}
\usepackage{comment}
\graphicspath{{asudeh/}}
\usepackage{soul}
\usepackage{subcaption}
\usepackage{ulem}
\usepackage{wrapfig}
\usepackage{color}
\usepackage{xspace}
\newtheorem{problem}{Problem}

%\DeclareMathOperator*{\argmax}{arg\,max}

%remove the following commands/package b4 submission
\newcommand{\hide}[1]{}
\newcommand{\eat}[1]{}
\newcommand{\resolved}[1]{\hide{#1}}
\newcommand{\abol}[1]{\textcolor{red}{Abol: #1}}
\newcommand{\mahdi}[1]{\textcolor{red}{Mahdi: #1}}
\newcommand{\nima}[1]{\textcolor{red}{Nima: #1}}

\newcommand{\dee}{\mathcal{D}}
\newcommand{\Gee}{\mathcal{G}}
\newcommand{\gee}{\mathbf{g}}
\newcommand{\ee}{\mathbf{e}}
\newcommand{\es}{\mathcal{S}}
\newcommand{\el}{\mathcal{L}}
\newcommand{\xx}{\mathcal{x}}
\newcommand{\dist}{\xi}
\newcommand{\alg}{\mathsf{A}}
\newcommand{\qu}{\mathbf{q}}
\newcommand{\ex}{\mathbf{x}}
\newcommand{\ti}{\mathbf{t}}
\newcommand{\sdt}{\mathsf{SDT}}
\newcommand{\wdt}{\mathsf{WDT}}
\newcommand{\Qu}{\mathbf{Q}}
\newcommand{\pe}{\mathbb{P}}
\newcommand{\megam}{\mathcal{M}}
\newcommand{\eps}{\varepsilon}
\newcommand{\enet}{{$\varepsilon$-{\bf net}}\xspace}
\newcommand{\net}{{\tt net}\xspace}
\newcommand{\vcd}{VC-dimension\xspace}
\newcommand{\at}[1]{{\tt \small #1}\xspace}
\newcommand{\pr}{Pr}

\newcommand{\sharpP}{\mbox{\#P}}
\newcommand{\NP}{\mathsf{NP}}
\newcommand{\LP}{\mathsf{LP}}
\newcommand{\IP}{\mathsf{IP}}
\newcommand{\ru}{{\sc {RU}}\xspace}
\newcommand{\sru}{{\sc {strongRU}}\xspace}
\newcommand{\wru}{{\sc {weakRU}}\xspace}

\newcommand{\fmsystem}{{\sc Chameleon}\xspace}
\newcommand{\fm}{$\mathcal{F}$\xspace}

\newtheorem{experiment}{Experiment}

\begin{document}

\title{Coverage-based Data-centric Approaches for \\Responsible and Trustworthy AI\thanks{This research was supported by the National Science Foundation under grant No. 2107290.}}

\author{
\begin{tabular}[t]{c@{\extracolsep{2.4em}}c@{\extracolsep{2.4em}}c@{\extracolsep{2.3em}}c} 
Nima Shahbazi & Mahdi Erfanian & Abolfazl Asudeh \\ 
University of Illinois Chicago & University of Illinois Chicago & University of Illinois Chicago\\
 nshahb3@uic.edu & merfan2@uic.edu & asudeh@uic.edu
\end{tabular}
}

\maketitle


\begin{abstract}
The grand goal of data-driven decision systems is to help make decisions easier, more accurate, at a higher scale, and also just. However, data-driven algorithms are only as good as the data they work with. Yet, data sets, especially those with social data, often do not represent minorities. The paucity of training data is a perpetual problem for AI, and the outcome of ML models for cases not represented in their training data is often not reliable. 
Hence, without properly addressing the lack of representation issues in data, we cannot expect AI-based societal solutions to have responsible and trustworthy outcomes. 

This paper focuses on data coverage as a data-centric approach for identifying and resolving misrepresentation of minorities in data.
To achieve this goal, we propose novel algorithms that (a) {\it identify} and {\it resolve} insufficient data coverage across data with different modalities and (b) use lack of representation information to generate data-centric {\it reliability warnings}.
 \end{abstract}
 
 %%%%%%%%%%%%%%%%%%%%%%%%%%%%%%%% INTRO  %%%%%%%%%%%%%%%%%%%%%%%%%%%%%%%%
\section{Introduction}\label{sec:intro} % Abstract+Intro: up to 2.5 pages 
Data-driven decision-making has shaped every corner of human life, spanning from autonomous vehicles to healthcare and even predictive policing and criminal justice. A pivotal concern, especially in applications that affect individuals, revolves around the reliability of the decisions rendered by the system.
It is easy to see that the accuracy of a data-driven decision depends, first and foremost, on the data used to make it. Essentially, the system learns the phenomena that data represent. While we may desire that the data should represent the underlying data distribution from which the production data is drawn, this alone may be insufficient, as it merely enables the model to perform well for the average case.
As a result, a model with a high accuracy could fail for specific regions in the data with insufficient representation. These regions may matter because they frequently represent some minority population in society. They could also represent cases that may not happen very often but have a relevant impact on the correctness of a critical decision.
In short, if the data fails to sufficiently represent a specific population, the outcome of the decision system for that population may not be trustworthy.

The phenomenon known as \textit{Representation Bias} can arise from how the data was originally collected, or it could be the result of biases introduced post-collection—whether historically, cognitively, or statistically.

Representation bias is essentially inevitable without a systematic approach to data collection. 
For example, in the context of survey data collection, vital steps involve identifying all populations within the underlying distribution based on desired demographic information and ensuring comprehensive coverage with sufficient samples from each group. 
Even then, only an (uncontrolled) subset of the invitees will opt-in to respond to the survey.
Another challenge lies in the fact that data scientists often lack control over the data collection process, leading to the reliance on ``found data'' in the majority of data-driven systems. Therefore, with no guarantee on the aforementioned steps in the data collection process, the found data is most likely a biased sample.
Acknowledging the potential harms of representation bias, the notion of \textit{Data Coverage}~\cite{asudeh2019assessing,shahbazi2023representation} has been proposed to ensure the adequate representation of minority groups in data sets employed for decision-making and developing sophisticated data science tools. 

Addressing representation issues in data poses various challenges depending on the modality of the data. In this paper, we focus on identifying and resolving lack of coverage issues in data with different modalities.
We start by proposing a variety of techniques (spanning from geometric and combinatorial optimization to crowd-souring) aimed at efficiently detecting insufficient coverage on structured data sets with non-ordinal categorical and continuous attributes, as well as image data sets. Next, we propose a range of approaches grounded in data integration and generative data augmentation to address the lack of coverage by enriching the data sets with more data. However, with limited control over the data collection processes, it could be difficult and expensive to resolve all misrepresentations. 
Since adding more data is not always possible, we proceed to introduce data-centric preventive solutions that warn the user about the reliability of their predictions regarding representation bias issues. These warnings assist users in determining whether they trust the outcomes of the models or exercise caution. 

 %%%%%%%%%%%%%%%%%%%%%%%%%%%%%%%% IDENTIFICATION  %%%%%%%%%%%%%%%%%%%%%%%%%%%%%%%%
\section{Detecting Insufficient Representation of Minorities}\label{sec:identification} %up to 3.5 pages
Representation bias happens when the development (training data) population under-represents 
and subsequently fails to generalize well 
for some parts of the target population, due to historical bias, sampling bias, etc.
The notion of {\it data coverage} has been studied across different settings in \cite{shahbazi2023representation} as a metric to measure representation bias. At a high level, coverage is referred to as having enough similar entries for each object in a data set. 
For a better understanding, let us go over the definition of the generalized notion of coverage:

\begin{definition}[Data Coverage]\label{def:coverage}
Consider a data set $\dee$ with $n$ tuples, each consisting of $d$ attributes of interest $\mathbf{x}=\{x_1, x_2, \cdots,x_d\}$, such as {\tt gender}, {\tt race}, {\tt salary}, {\tt age}, etc, that are used for coverage identification.
The data set also contains target attributes $\mathbf{y} = \{ y_1,\cdots,y_{d'}\}$ that may or may not be considered for the coverage problem.
A query point $q$ is not covered by the data set $\dee$, if there are not ``enough'' data points in $\dee$ that are representative of $q$.
To generalize the notion of coverage, let us define $\gee(q)$ as the universe of tuples that would represent $q$ and let $\gee_\dee(q) = \gee(q)\cap \dee$. In other words, $\gee_\dee(q)$ are the set of tuples in $\dee$ that represent $q$.
Using this notation, we define the coverage of $q$ as the size of $\gee_\dee(q)$. That is,
$cov(q,\dee) = | \gee_\dee(q)|$.
Given a value $\tau$, $q$ is covered if $cov(q,\dee)>\tau$.
Similarly, a group $\gee$ is not covered if $\gee\cap \dee<\tau$.
The {\it uncovered region} in a data set is the collection of groups that are not covered by it.
\end{definition}

\subsection{Structured Data}
In this section, we focus on identifying representation bias in structured data.
Depending on the type of the attributes of interest, we categorize the techniques into two classes based on whether they target the problem for non-ordinal {\it categorical} (e.g. {\tt race}, {\tt gender}) or ordinal {\it continuous} (e.g. {\tt age}) attributes. The attributes of interest considered for representation bias often include sensitive attributes such as {\tt race} and {\tt gender} but are not necessarily limited to them.

\subsubsection{Categorical Attributes}

For cases where attributes of interest are non-ordinal categorical,
the cartesian product of values on a subset of attributes $\mathbf{x}'\subseteq \mathbf{x}$, form a set of (sub-)groups.
For example, $\{$ {\tt white male}, {\tt white female}, {\tt black male} $,\cdots\}$ are the subgroups defined on the attributes {\tt (race,gender)}.
We refer to the number of attributes used to specify a subgroup as the {\it level} of that subgroup.
For example, the level of the subgroup {\tt white male} is 2, while the level of the subgroup {\tt male} is 1.
We use $\ell(\gee)$, to refer to the level of a subgroup $\gee$.
Similarly, we say a subgroup $\gee'$ is a subset of $\gee$, if the groups specifying $\gee'$ are a superset of the ones for $\gee$. For example {\tt (married white male)} a subset of the more general group {\tt (white male)}. That is, the set of individuals in group {\tt (married white male)} are a subset of {\tt (white male)}.
Moreover, we say a subgroup $\gee$ is a {\it parent} of the subgroup $\gee'$, if $\gee'\subset \gee$ and $\ell(\gee)=\ell(\gee')+1$. For example, the subgroup {\tt (white male)} is a parent of the subgroup {\tt (married white male)}.
We use \textit{patterns} to refer to uncovered subgroups.
A pattern $P$ is a string of $d$ values, where $P[i]$ is either a value from the domain of $x_i$, or it is ``unspecified'', specified with $X$. 
For example, consider a data set with three binary attributes of interest $\mathbf{x}=\{x_1, x_2, x_3\}$. The pattern $P=X01$ specifies all the tuples for which $x_2=0$ and $x_3=1$ ($x_1$ can have any value).
The set of patterns that identify most general uncovered subgroups are called {\it Maximal Uncovered Patterns} (MUPs).

No polynomial time algorithm can guarantee the enumeration of the entire MUPs, however, several algorithms inspired by set enumeration and the Apriori algorithm for association rule mining are proposed to efficiently address this problem~\cite{asudeh2019assessing}.
In this regard, we introduce \textit{Pattern Graph} data structure that exploits the relationship between patterns to do less work than computing all uncovered patterns by removing the non-maximal ones. 
The parent-child relationship between the patterns is represented in a graph that can be used to find better algorithms. 
\textit{Pattern-Breaker} starts from the top of the graph where the general patterns are and moves down by breaking each pattern into more specific ones. If a pattern is uncovered, then all of its descendants are also uncovered and they can not be an MUP, even if they have a parent that is covered. Therefore, this subgraph of the pattern graph can be pruned. 
The issue with \textit{Pattern-Breaker} is that it explores the covered regions of the pattern graph and for the cases where there are a few uncovered patterns, it has to explore a large portion of the exponential-size graph. 
To tackle this, \textit{Pattern-Combiner} algorithm is proposed that performs a bottom-up traversal of the pattern graph. It uses an observation that the coverage of a node at the level of the pattern graph can be computed as the sum of the coverage values of its children. 
The problem with \textit{Pattern-Combiner} is that it traverses over the uncovered nodes first and therefore, it will not perform well for the cases in which most of the nodes in the graph are uncovered. 
In fact, for the cases where most of the MUPs are placed in the middle of the graph, both \textit{Pattern-Breaker} and \textit{Pattern-Combiner} will not be as efficient as they should traverse half of the graph. Therefore, we propose \textit{Deep-Diver}, a search algorithm based on Depth-First-Search that quickly finds the MUPs, and uses them to limit the search space by pruning the nodes both dominating and dominated by the discovered MUPs.

\begin{figure*}[!tb]
    \begin{minipage}[t]{0.31\linewidth}
        \centering
        \includegraphics[width=\textwidth]{submissions/submission1/shahbazi/covcube1.jpg}
        \caption{\small Categorical attributes: the uncovered region of a toy example, as the collection of three MUPs.}
        \label{fig:covcube1}
    \end{minipage}
    \hfill
    \begin{minipage}[t]{0.31\linewidth}
        \centering
        \includegraphics[width=\textwidth]{submissions/submission1/shahbazi/cvrg_2_1.jpg}
        \caption{\small Continuous attributes, 2D: identifying the covered region in the gray Voronoi cell.}
        \label{fig:cvrg_2_1}
    \end{minipage}
    \hfill
    \begin{minipage}[t]{0.31\linewidth}
        \centering
        \includegraphics[width=\textwidth]{submissions/submission1/shahbazi/cvrg_2_2.jpg}
        \caption{ \small Continuous attributes, 2D: Uncovered region marked in red.}
        \label{fig:cvrg_2_2}
    \end{minipage}
\vspace{-5mm}
\end{figure*}

\subsubsection{Continuous Attributes}
Data in the real world often consists of a combination of continuous and discrete values. While simple solutions like binning {\tt age} into {\tt young} and {\tt old} can transform the continuous space into discrete. However, they may lead to coarse groupings that are sensitive to the thresholds chosen. It may be inappropriate to treat a 35-yo as {\tt young} but a 36-yo as {\tt old}. 
Therefore, we extend the notion of coverage to continuous space. Particularly, given data set $\dee$ with $n$ tuples over $d$ attributes, and vicinity radius $\rho$ and coverage threshold $k$, we want to identify the uncovered region -- the universe of uncovered query points.
A query point in continuous data space is covered if there are enough (at least $k$) data points in its $\rho$-vicinity neighborhood. $\rho$-vicinity neighborhood is the circle centered at the query point with radius $\rho$.

Depending on the number of attributes in a data set, we propose two algorithms for identifying uncovered regions in data~\cite{asudeh2021coverage}. 
The first algorithm known as \textit{Uncovered-2D} studies coverage over two-dimensional data sets where $\mathbf{x}=\{x_1,x_2\}$. To find the number of circles that a query point falls into and consequently discover the uncovered region, \textit{Uncovered-2D} makes a connection to $k$-th order Voronoi diagrams.
Consider a data set $\mathcal{D}$ and its corresponding $k$-th order Voronoi diagram. For every tuple $t\in \mathcal{D}$, let $\circ_t$ be the $d$-dimensional sphere ($d$-sphere) with radius $\rho$ centered at $t$.
Consider a $k$-voronoi cell $\mathcal{V}(S)$ in the $k$-th order Voronoi diagram $V_k(\mathcal{D})$.
Any point $q$ inside the intersections of the $d$-spheres of tuples in $S$, i.e. $q\in \underset{\forall t\in S}{\cap ~\circ_t}$, is covered, while all other points in the region are uncovered.
 The algorithm starts by constructing the $k$-th order Voronoi diagram of the data set and then for each Voronoi cell $\mathcal{V}(S)$ in the diagram, it computes the intersection of the circles of the tuples in $S$ and marks the portion of $\mathcal{V}(S)$ that falls outside it as uncovered.
After identifying the uncovered region, a 2D map of $\{x_1,x_2\}$ value combinations is used to report the region to the user.
The algorithm for the 2D case can be extended to the general case by relaxing the assumption on the number of attributes to discover the exact uncovered region, however, due to the curse of dimensionality, the search size space explodes as the number of dimensions increases and as a result, the algorithm will not be practical. Therefore, we propose a randomized approximation algorithm based on the geometric notion of \enet. 
Let $\mathcal{X}$ be a set and $\mathcal{R}$ be a set of subsets of $\mathcal{X}$. A set $\mathcal{N}\subset \mathcal{X}$ is an \enet for $\mathcal{X}$ if for any range $r\in\mathcal{R}$, if  $|r\cap \chi|>\eps|\chi|$, then $r$ contains at least one point of $N$.
The idea, at a high level, is to draw enough random samples from the space of potential query points to form an \enet. 
We then label the sampled query points as $\{-1,+1\}$ depending on whether those are covered or not, and learn the uncovered regions using the samples.

\subsection{Image Data}
Many known incidents of machine failures due to the lack of representation were on image data.
We consider an image data set with a fixed number of low-cardinality sensitive attributes such as {\tt\small race} and {\tt\small gender}. 
It is common that image data sets {\it lack explicit values} for sensitive attributes, which are crucial for coverage identification. An image data set is often a collection of images from different domains with little to no information about their domain and which groups they belong to. As a result, even studying coverage over low-cardinality and categorical attributes of interests is challenging in these cases.

\begin{wrapfigure}{R}{0.42\textwidth}
\centering
\vspace{-3mm}
\scriptsize
\begin{tabular}{|@{}c|@{}c@{}|@{}c@{}|@{}c@{}|} 
 \hline
{\bf data set} & {\bf classifier} & {\bf accuracy} & {\bf precision} \\ 
 &  &  & {\bf on female} \\ \hline
UTKFace:~& DeepFace (opencv) & 93.56 & {52.02}\\\cline{2-4}
({\tt females}=200,& DeepFace (retinaface) & 94.16 & {56.15}\\\cline{2-4}
{\tt males}=2800) & BaseCNN & 97.6 & 74.8\\
\hline
UTKFace:~& DeepFace (opencv) & 96.53 & {\bf 8.0}\\\cline{2-4}
({\tt females}=20,& DeepFace (retinaface) & 96.43 & {\bf 10.09}\\\cline{2-4}
{\tt males}=2980)& BaseCNN & 97.6 & {\bf 21.59}\\
\hline
\end{tabular}
\vspace{-3mm}
\caption{\small ML models' low performance for females in the presence of representation bias.~\cite{mousavi2024data}}\label{fig:mlfails}
\vspace{-3mm}
\end{wrapfigure}

In Figure~\ref{fig:mlfails}, we show that due to the issues such {\it machine bias} and {\it lack of distribution generalizability},
solely relying on state-of-the-art machine learning (ML) techniques fail to effectively identify lack of coverage in image data sets. Therefore, we propose an approach based on combining crowdsouring with ML~\cite{mousavi2024data}. 
Crowdsourcing is particularly promising for image data, for tasks such as image labeling, which, while challenging for the machine, are "easy" for human beings to conduct with minimal error. 

A key observation that enables a cost-effective crowdsourcing approach is that, while studying coverage, we would only like to find out if there are {\it enough tuples from each subgroup}.
Suppose a subgroup is covered if there are $\tau=100$ instances of it in the data set. Assume the (majority) group $\gee_1$ contains $n_1 \gg 100$ objects in the data set. 
To verify that $\gee_1$ is covered, it is enough for the crowd to discover 100 of those objects, not the entire $n_1$. 
Following this, $O(\tau)$ provides a lower bound on the number of crowd tasks required to verify a given group is covered. 
Still, this lower bound only holds for the groups that are covered, i.e., there is at least $\tau$ of those in the data set.
Surprisingly, verifying that a minority group is indeed uncovered is cumbersome, unlike the majority group.
This is because even though discovering $\tau$ objects from a group is enough for verifying that it is covered, one cannot {\it verify} a group is uncovered until there is a chance that the data set might still have enough objects from that group. Thus, assuming a non-zero probability for each unlabeled object to belong to each group, {one might need to ask the crowd to label the entire data set before they can confirm that a specific group is uncovered}.

Our idea for addressing this challenge is to
design {\it a divide and conquer algorithm} that, instead of {point queries}, uses {\it set queries} to iteratively eliminate subsets of data that {does not include any object from the given group}.
At a high level, our idea is to ask a set query from the crowd, inquiring whether the selected set contains at least one object from the given group $\gee$.
The user may provide two responses (yes/no). 
Interestingly, {in either case}, the user response provides valuable information that helps efficiently identify the coverage.
If the answer is ``No'', the set does not include any object from the given group $\gee$. As a result, the algorithm can safely prune the set, asking no further questions about it. In particular, for a group that is not covered, one can expect to see no answers on large set queries helping to prune a significant portion of the data set quickly.
On the other hand, if the answer is ``yes'', the set contains {at least} one object from the group $\gee$. As a result, the algorithm cannot prune the subset since it can have any number (larger than one) of the objects in $\gee$.
At first glance, the queries with yes answers do not provide helpful information as the algorithm cannot prune the subset (hence it needs to divide it into smaller subsets).
However, a key observation is that {the algorithm will only observe a limited number of yes answers} before it stops.
The reason is that the number of set queries with yes answers provides a {lower-bound} on the number of objects from $\gee$ in the data set. As a result, the algorithm can stop as soon as the lower bound reaches $\tau$, knowing that $\gee$ is covered.
The D\&C approach verifies the data coverage for a given group, while our goal is to identify the uncovered regions for a given set of sensitive attributes. The next question is how to utilize this algorithm for efficient coverage identification on different scenarios of sensitive attributes, forming intersectional or non-intersectional groups.
In particular, how can we find maximal uncovered patterns?
Our idea is to apply sampling and aggregate estimation techniques to find the groups that even if merged are likely to still be uncovered. This will help reduce the coverage identification cost by running the D\&C approach for the merged groups once.
 %%%%%%%%%%%%%%%%%%%%%%%%%%%%%%%% RESOLUTION  %%%%%%%%%%%%%%%%%%%%%%%%%%%%%%%%
\section{Resolving Insufficient Representation}\label{sec:resolution}

Data integration~\cite{nargesian2021tailoring,nargesian2022responsible} and data augmentation~\cite{sharma2020data,DBLP:journals/jair/ChawlaBHK02,iosifidis2018dealing,celis2020data} are considered as the primary solutions for reducing data coverage issues in a data set. 
Data integration is promising when external sources of data are available. On the other hand, recent advancements in generative AI and foundation models have enabled efficient and effective augmentation of data sets with synthetic data. 
Therefore, in the following, we review two approaches, one from each category, in the context of lack of coverage resolution.

\subsection{Data Integration}\label{sec:resolution:integration}

Data integration is to consolidate data from different sources into a single, unified view. 
Although it is an effective solution to acquire additional data from different distributions,
there are sampling policy and cost-efficiency concerns that need to be examined.  
Therefore, {\it Data Distribution Tailoring ({\sc DT})} introduces data integration techniques for resolving insufficient representation of subgroups in a data set in the most cost-effective manner~\cite{nargesian2021tailoring}.
A query to {\sc DT} 
consists of a target schema, and a set of group distribution requirements in the form of the minimum counts (e.g., ``{\tt\small 1,000 breast cancer monitoring data in Chicago with at least 30\% label=positive, and at least 20\% black patients}''). 
Collecting a fresh sample from a data view is costly (monetary, human resources, and/or computation cost)~\cite{asudeh2022towards}.
Therefore, {\sc DT} focuses on satisfying the count requirements with minimum cost. 
Given an input query and a lake of available data sources, the first step is to discover a collection of candidate data views that satisfy the target schema.
Each data view $v_i$ is a projection-join $v_i = \Pi\big(D_{i1}\bowtie\cdots\bowtie D_{ik_i} \big)$, where $D_{ij}$ is a data set in a given data lake.
Let us suppose the data views are already discovered.
At a high level, {\sc DT} follows an iterative approach that at each iteration a data view is selected to be queried.
Each query to a data view has a fixed cost and returns a sample that may or may not satisfy the query constraints.
The samples that are either not fresh, or do not satisfy the query are discarded.
Hence, the essential question towards a cost-effective data integration is {\it what data view to query next}.
Depending on the available information about the data sources, various techniques may be employed. 

For the cases when the group distributions are known, the process of collecting the target data set is a sequence of iterative steps, where at every step, the algorithm chooses a data view, queries it, and if the obtained tuple contributes to one of the groups for which the count requirement is not yet fulfilled, it is kept, otherwise discarded. To do so, a {Dynamic Programming (DP)} algorithm is proposed. An optimal source at each iteration minimizes the sum of its sampling cost plus the expected cost of collecting the remaining required groups, based on its sampling outcome.
The DP algorithm, however, has a pseudo-polynomial time complexity. Hence, it quickly becomes intractable for cases where the minimum count requirements for the groups are not small. 
For cases where the (sensitive) attribute of interest is binary, such as (biological) {\tt sex}={\tt \{male, female\}}, and the cost to query data is similar from all sources, it turns out that the optimal strategy is to query the data source with {maximum probability of obtaining a sample from the minority group}.
Expanding the binary-attributes algorithm for non-binary cases, the problem can be modeled as an extension of the ``{\it coupon collector's}'' problem~\cite{motwani1995randomized}, where the goal is to collect $m_i$ instances from each coupon (group) $\gee_i$.
At each iteration, the coupon collector's algorithm identifies a data view as most promising and queries it. In simple terms, a data view with a smaller query cost and a higher chance of obtaining minority groups is more promising.


For the cases where the group distributions are unknown, we model DT as a {\it multi-armed bandit} problem, where every data view is modeled as an arm. 
Every arm has an unknown distribution of different groups while pulling an arm (i.e., querying the corresponding data view) has a cost.
During various iterations, the algorithms pull the arms in an order that its expected total {\it reward} is maximized.
Arguing that the reward of obtaining a tuple from a group is proportional to how rare this group is across different data views, 
we design the reward function based on the expected cost one needs to pay in order to collect a tuple from a specific group.  
As the bandit strategy, we adopt {\it Upper Confidence Bound (UCB)} to balance exploration and exploitation. At every iteration, for every arm, UCB computes confidence intervals for the expected reward and selects the arm with the maximum upper bound of reward to be explored next.

\subsection{Data Augmentation using Foundation Models}

While data integration provides a promising approach for resolving coverage issues in a data set, its effectiveness is limited to the availability of external data sources that are rich enough to find sufficient fresh samples from minority groups. This, however, is not always possible, especially since the minority samples are rare and not easy to obtain.
Fortunately, recent advancements in Generative AI and Foundation Models have enabled synthesizing samples that are otherwise challenging to obtain from the real world.

Therefore, as an alternative approach to data integration, we turn our attention to the Foundation Models and Generative AI for resolving the lack of coverage. 
Particularly, models such as {\sc DALL.E}\footnote{\url{https://openai.com/dall-e-2}} have emerged as powerful tools for generating multi-modal data such as image, audio, and video.
 
We formalize the foundation model \fm as a black-box function with the following inputs, that once queried synthesize an output tuple.
\begin{itemize}
    \item {\bf Prompt}: A natural language description providing instructions on the details of the tuple to be generated. For instance, a prompt for image generation might be ``A realistic photo of a white cat running in a backyard.''
    \item {\bf Guide}: In cases where only a prompt is provided, the foundation model uses its imagination to generate the requested tuple. For the previous example, the prompt of a cat image, the breed, size, background, and other details are generated based on the model's imagination. Alternatively, a guide can be provided to influence the generation process. The guide is formalized as a pair $(t,m)$ where $t$ is a tuple and $m$ is a mask specifying which parts of the guide tuple should be changed. Using the cat example, $t$ can be a cat image and $m$ can specify the foreground to be regenerated.
\end{itemize}

There are multiple challenges towards effective data set augmentations using foundation models. 
First, we have to determine the minimal set of synthetic tuples that once added to the original data set, under-representation issues are resolved.
Second, the generated images should follow the underlying distribution represented in the input data set. Third, the generated tuples should have high quality and look realistic to a human evaluator. Last but not least, given the (often monetary) cost associated with the queries to the foundation model, we should ensure the cost-effectiveness of the data set repair process.

\begin{wrapfigure}{L}{0.45\textwidth}
\centering
\vspace{-3mm}
\scriptsize
    \includegraphics[width=.45\textwidth]{submissions/submission1/shahbazi/enhanced_pipeline.png}
\vspace{-3mm}
\caption{\small Architecture of \fmsystem for image data augmentation for coverage enhancement.}\label{fig:chameleon}
% \vspace{-3mm}
\end{wrapfigure}

\noindent Figure~\ref{fig:chameleon} shows the architecture of our system \fmsystem \cite{chameleon} for coverage enhancement using DALL-E image generator.
To address the first challenge, we define the combinations-selection problem, which minimizes the total number of synthetic tuples for resolving lack of coverage of minorities at the most general level. We show the problem is {\sc NP}-hard, and propose a greedy approximation algorithm for it.
To address the second and third challenges, \fmsystem follows a {\it rejection sampling} strategy.
It views each tuple in the data set $\dee$ as an iid sample from the underlying distribution $\xi$ it represents. It uses the vector representations (embeddings) space to describe the distribution. Then, given a newly generated tuple, it employs the one-class support vector machine (OCSVM) approach proposed by Scholkopf et al.~\cite{scholkopf1999support} to reject the tuple if it does not follow $\xi$.
Moreover, it models the quality evaluation as hypothesis testing and rejects the samples that have a higher chance of being labeled as ``unrealistic'' by a random human evaluator.
Finally, to minimize the number of queries to the foundation model, we provide a guide tuple (and a mask), in addition to the prompt, to the foundation model. We model the guide-selection problem as {\it contextual multi-armed bandit} and propose a solution based on the contextual UCB for it.

Before concluding this section, let us provide some experiment results to demonstrate the effectiveness of data augmentation with \fmsystem. We use FERET DB \cite{phillips1998feret} for this experiment, which comprises 1199 individual images and serves as a standardized facial image database for researchers to develop algorithms and report results. All images in FERET DB share the same dimensions, pose, and facial expression.
First, we identified the (level-1) uncovered ethnicity groups, using the threshold 80. We then used \fmsystem and resolved the lack of coverage issues.
To evaluate the effectiveness of the system, we trained a CNN model to predict the race of each image within this dataset. We then retrained the identical CNN on the repaired training data. Importantly, our test dataset for both experiments remains consistent and is derived from real images.
Table~\ref{tab:lackofcoverage} presents the improvements in precision, recall, and F1 score metrics for under-represented groups after repairing the dataset. The results indicate an enhancement in performance metrics for all under-represented groups following the repair process.

\begin{table}[t]
    \centering
    \caption{Illustrating the effect of lack of coverage repair using \fmsystem on \texttt{FERTDB}}
    \label{tab:lackofcoverage}
    \vspace{-3mm}
    \begin{tabular}{lcccccccc}
        \toprule
         & \multicolumn{4}{c}{\textbf{Classifier Performance on \texttt{FERTDB}}} & \multicolumn{4}{c}{\textbf{Classifier Performance on Repaired}} \\
        \cmidrule(lr){2-5} \cmidrule(lr){6-9}
        \textbf{Ethnicity Groups}& \#Images & Precision & Recall & F1-Score & \#Images & Precision & Recall & F1-Score \\
        \midrule
        Overall          & 756 & 0.81 & 0.75 & 0.78 & 987 & 0.70 & 0.75 & 0.72 \\ \hline
        Black            & 40  & 0.19 & 0.22 & 0.16 & 100 & 0.48 & 0.56 & 0.52 \\
        Hispanic         & 19  & 0.50 & 0.17 & 0.25 & 100 & 0.62 & 0.36 & 0.45 \\
        Middle Eastern   & 10  & 0.00 & 0.00 & 0.00 & 100 & 0.20 & 0.41 & 0.27 \\
        \bottomrule
    \end{tabular}
\end{table}

 %%%%%%%%%%%%%%%%%%%%%%%%%%%%%%%% RELIABILITY  %%%%%%%%%%%%%%%%%%%%%%%%%%%%%%%%
\section{Generating Reliability Warnings}\label{sec:reliability}
% up to 2.5 pages
Interpretability is a necessity for data scientists who develop predictive models for critical decision-making.
In such settings, it is important to provide additional means to support the following question:
{\it is an individual prediction of the model reliable for decision-making?} Our goal is to use the lack of representation to help decision-makers find insights about this critical question.
To further motivate this, let us use the following example:

\vspace{1mm}
\begin{example}\label{ex-0}
{\bf(Part1):} Consider a judge who needs to decide whether to accept or deny a bail request. Using data-driven predictive models is prevalent in such cases for predicting recidivism~\cite{dressel2018accuracy}.
Indeed, such models can be beneficial to help the judge make wise decisions.
Suppose the model predicts the queried individual as high risk (or low risk).
The judge is aware and concerned about the critics surrounding such models.
A major question the judge faces is whether or not they should rely on the prediction outcome to take action for this case.
Furthermore, if, for instance, they decide to ignore the outcome and hence they need to provide a statement supporting their action, what evidence can they provide? 
\end{example}

In line with the recent trend on data-centric AI~\cite{ng2021mlops}, we design {novel approaches}, {complimentary} to the existing work on trustworthy AI~\cite{wing2021trustworthy,kentour2021analysis,liu2021trustworthy,singh2021trustworthy}, to address the aforementioned trust question through the lens of {\it data}.
In particular, unlike existing works that generate trust information from a {\it given \underline{model}}, we associate {\it \underline{data sets} with proper measurements} that specify their {\it the scope of use for predicting future cases}.
We note that a predictive model provides only probabilistic guarantees on the \underline{average} loss over the distribution represented by the data set used for training it.
As a result, these predictions may not be distribution generalizable~\cite{kulynych2022you}.
Consequently, if the query point is {\it not represented} by the data, the guarantees may not hold, hence one cannot rely on the prediction outcome.
Besides, an essential requirement for a learning algorithm is that its training data $\dee$ should represent the underlying distribution $\dist$.
Even if so, the trained model $h$ only provides a probabilistic guarantee on the {expected} loss on random samples from $\dist$.  
A model that performs well on {\it majority} of samples drawn from $\dist$ will have a high performance on average. Still, as we observed in Figure~\ref{fig:mlfails},
its performance for {\it minorities} and points that are not represented is questionable. Let us consider the following toy example:

\begin{figure*}[!b] 
    \begin{minipage}[t]{0.32\linewidth}
        	\centering
        	\includegraphics[width=\textwidth]{submissions/submission1/shahbazi/example_1.png} 
        	\vspace{-9mm}\caption{\small Data set $\dee$ generated using a Gaussian distribution; $x_1$ and $x_2$ are positively correlated}
            \label{fig:ex1:1}
    \end{minipage}
    \hfill
    \begin{minipage}[t]{0.32\linewidth}
        \centering
        	\includegraphics[width =\textwidth]{submissions/submission1/shahbazi/example_2.png} 
        	\vspace{-9mm}\caption{\small The decision boundary of learned model $h$ and query points $\qu^1$ to $\qu^4$}
            \label{fig:ex1:2}
    \end{minipage}
    \hfill
    \begin{minipage}[t]{0.32\linewidth}
        	\centering
        	\includegraphics[width =\textwidth]{submissions/submission1/shahbazi/example_3.png}
        	\vspace{-9mm}\caption{\small Ground-truth boundary, overlaid on the model decision boundary and query points}
            \label{fig:ex1:3}
    \end{minipage}
    \vspace{-5mm}
\end{figure*} 

\vspace{1mm}
\begin{example}\label{ex-1}
Consider a binary classification task where the input space is $\ex=\langle x_1, x_2\rangle$ and the output space is the binary label $y$ with values $\{-1$ (red) $,+1$ (blue)$\}$.
Suppose the underlying data distribution $\dist$ follows a 2D Gaussian, where $x_1$ and $x_2$ 
are positively correlated as shown in Figure~\ref{fig:ex1:1}.
The figure shows the data set $\dee$ drawn independently from the distribution $\dist$, along with their labels as their colors.
Using $\dee$, the prediction model $h$ is constructed as shown in Figure~\ref{fig:ex1:2}. 
The decision boundary is specified in the picture; while any point above the line is predicted as +1, a query point below it is labeled as -1.
The classifier has been evaluated using a test set that is an iid sample set drawn from the underlying data set $\dist$. The accuracy on the test set is high (above 90\%), and hence, the model gets deployed.
We cherry-picked four query points, $\qu^1$ to $\qu^4$, that are also included in Figure~\ref{fig:ex1:2}. Using $h$ for prediction, $h(\qu^1)=-1$, $h(\qu^2)=+1$,  $h(\qu^3)=+1$, and $h(\qu^4)=-1$.
Figure~\ref{fig:ex1:3} adds the ground-truth boundary to the search space, revealing the true label of the query points: every point inside the red circle has the true label $-1$ while any point outside of it is $+1$.
Looking at the figure, $y^1=+1$ while the model predicted it as $h(\qu^1)=-1$.  \hfill$\square$
\end{example}
\vspace{2mm}

Let us take a closer look at the four query points in this example and their placement with regard to the tuples in $\dee$ used for training $h$. 
$\qu^2$ belongs to a {\it dense region} with many training tuples in $\dee$ surrounding it. Besides, all of the tuples in its vicinity have the same label $y=+1$. As a result, one can expect that the model's outcome $h(\qu^2)=+1$ should be a reliable prediction.
Similar to $\qu^2$, $\qu^4$ also belongs to a dense region in $\dee$; however, $\qu^4$ belongs to an {\it uncertain region}, where some of the tuples in its vicinity have a label $y=+1$, and some others have the label $y=-1$. Considering the uncertainty in the vicinity of $\qu^4$, one cannot confidently rely on the outcome of the model $h$. 
On the other hand, the neighbors of $\qu^1$ (resp. $\qu^3$) are not uncertain, all having the label $y=-1$ (resp. $y=+1$).
However, the query points $\qu^1$ and $\qu^3$ are not well represented by $\dee$. In other words, $\qu^1$ and $\qu^3$ are unlikely to be generated according to the underlying distribution $\dist$, represented by $\dee$. As a result, following the no-free-lunch theorem~\cite{kakade2003sample}, one cannot expect the outcome of model $h$ to be reliable for these points.
Looking at the ground-truth boundary in Figure~\ref{fig:ex1:3}, $h$ luckily predicted the outcome for $\qu^3$ correctly, but it was not fortunate to predict the $y^1$ correctly.
Nevertheless, 
since the model is not reliably trained for these points, 
its outcome for these query points is not trustworthy.

From Example~\ref{ex-1}, we observe that the outcome of a model $h$, trained using a data set $\dee$ is not reliable for a query point $\qu$, if:
\begin{itemize}
    \item {\bf Lack of representation:} $\qu$ is not well-represented by $\dee$.
    In such cases, the model has not seen ``enough'' samples similar to $\qu$ to reliably learn and predict the outcome of $\qu$.
    \item {\bf Lack of certainty:} $\qu$ belongs to an uncertain region, where different tuples of $\dee$ in the vicinity of $\qu$ have different target values. $\qu$ belongs to a high-fluctuating area, where tuples in the vicinity of $\qu$ have a wide range of values.
\end{itemize} \vspace{2mm}

\noindent
Based on these two observations, we propose Representation-and-Uncertainty ({\bf RU}) measures.
To identify if a query suffers from uncertainty or lack of representation, one could use a deterministic approach using a fixed threshold. Then if the number of similar samples to (resp. label fluctuation in vicinity of) $\qu$ is larger than the threshold it is considered as unrepresented (resp. uncertain).
This approach, however, would be misleading since two numbers close to the threshold could be treated very differently. Also, all points on each side of the threshold would be considered equally represented (resp., certain). Instead, we consider {\it a randomized approach}, widely popular in the literature, including~\cite{dwork2012fairness}.
That is, instead of using fixed thresholds, a Bernoulli variable (a biased coin) is used that 
assigns $\qu$ as unrepresented (resp., uncertain) based on the number of samples similar to it (resp., its neighborhood uncertainty).
Given a query point $\qu$, let $\pe_o$ be the probability indicating if $\qu$ is not represented and let $\pe_u$ be the probability indicating if $\qu$ belongs to an uncertain region. 
We represent the probability of the Bernoulli variables for lack of representation or uncertainty components as $\pe_o$ and $\pe_u$, respectively. Note that the two Bernoulli variables $\pe_o$ and $\pe_u$ are independent from each other. That simply follows the argument that after specifying the number of similar samples to $\qu$ whether or not it should be considered as unrepresented does not depend on the uncertainty in the neighborhood of $\qu$.

\begin{definition}[\sru]\label{def:sdt}
The \sru is a probabilistic measure that considers the outcome of a model for a query point $\qu$ untrustworthy if $\qu$ is not represented by $\dee$ {\it and} it belongs to an uncertain region.
Formally, the \sru measure is:
\begin{align} 
    \nonumber
    SRU(\qu) &= \pe\big((\qu \mbox{ is outlier}) \wedge (\qu \mbox{ belongs to uncertain region})\big) 
\end{align}
Since $\pe_o$ and $\pe_u$ are independent:

\vspace{-13mm}
\begin{align} \label{eq:strong}
    SRU(\qu) &= \pe_o(\qu) \times \pe_u(\qu)
\end{align}
\end{definition}

\sru raises the warning signal only when the query point fails on {\it both} conditions of being represented by $\dee$ and not belonging to an uncertain region. 
For instance, in Example~\ref{ex-1} none of the query points fail both on representation and on uncertainty; hence neither has a high \sru score.
On the other hand, 
a high \sru score for a query point $\qu$ {\it provides a strong warning signal} that one should perhaps reject the model outcome and not consider it for decision-making.

\sru is a strong signal that raises warnings only for the fearfully concerning cases that fail both on representation and uncertainty.
However, as observed in Example~\ref{ex-1} a query points failing {\it at least} one of these conditions may also not be reliable, at least for critical decision making.
We define the \wru measure to raise a warning for such cases.

\begin{definition}[\wru]\label{def:wdt}
The \wru measure is a probabilistic measure that considers the outcome of a model for a query point $\qu$ untrustworthy if $\qu$ is not represented by $\dee$ {\bf or} it belongs to an uncertain region.
Formally, the \wru is computed as:
\begin{align} \label{eq:weak}
    WRU(\qu) = \pe\big((\qu \mbox{ is outlier}) \vee (\qu \mbox{ belongs to uncertain region})\big) 
    = \pe_o(\qu) + \pe_u(\qu) - \pe_o(\qu) \times \pe_u(\qu)
\end{align}
\end{definition}

Proposing quantitative probabilistic outcomes, \ru measures are interpretable for the users, since beyond the scores, the uncertainty and lack of representation components provide an explanation to justify them. 
Please refer to \cite{techrep} for more details on how to efficiently and effectively compute the representation ($\pe_o$) and uncertainty ($\pe_u$) probabilities, using only $\dee$.
In Example~\ref{ex-0}, let us see how the \ru measures can be helpful.

\noindent{\bf Example 1. (part 2):}
{\it RU measures \underline{raise warning} when
the fitness of the data set used for drawing a prediction is questionable, helping the judge to be cautious when taking action.
Besides, these measures provide \underline{quantitative evidence} to support the judge's action when they decide to ignore a prediction outcome that is not trustworthy.
The judge, for example, can argue to ignore a model outcome for a specific case, based on the insight that 
the model has been built using a
data set that fails to represent the given case.}
\hfill$\square$

Finally, let us demonstrate the efficacy of \ru measures through a series of experiments. Since the \ru measures are {\it data-centric},
those are applicable for both classification and regression tasks, irrespective of the model used.
We use {\it Adult} dataset~\cite{adult} for classification and {\it House Sales in King County} dataset for the validation of regression tasks. From each dataset, we uniformly sample two sets from the underlying distribution. The first set serves as the training set to compute the \ru values, and the second one is used as the test set from which the queries are drawn. We validate our proposal by providing the correlation between the \ru values and the performance of an ML model's prediction on the same data. 

We start by computing the \ru values for all the query points in the test set. Next, we bucketize the query points based on their \ru values in equi-width buckets of width 0.1. We repeat this for both \sru and \wru measures. Next, we train a model on the training data set and predict the target variable for the points in each range of \ru measure. The validation results for the classification task on the {\it Adult} dataset are presented in Figures \ref{fig:exp-adult-sdt} and \ref{fig:exp-adult-wdt}. Each figure corresponds to the accuracy/error measures of the classifier over each bucket of \ru values for \sru and \wru. As the \ru values increase, the accuracy of the model drops while the FPR rises, and therefore, the model fails to capture the ground truth for the points that fall into untrustworthy regions in the data set. By repeating the aforementioned steps for the regression task on the {\it House Sales in King County} dataset, we observe similar results presented in Figures \ref{fig:exp-hs-sdt} and \ref{fig:exp-hs-wdt}. 
As the \ru value increases, the RSS of the regression model follows the same trend denoting that the model fails to perform for tuples with a high \ru value.

\begin{figure}[!tb]
    \begin{minipage}[t]{0.24\linewidth}
        \centering
        \includegraphics[width=\textwidth]{submissions/submission1/shahbazi/sdt_adult.pdf}
        \vspace{-6mm}\caption{\small{\it Adult}, efficacy of \sru  on classification}
        \label{fig:exp-adult-sdt}
    \end{minipage}\hfill
    \begin{minipage}[t]{0.24\linewidth}
        \centering
        \includegraphics[width=\textwidth]{submissions/submission1/shahbazi/wdt_adult.pdf}
        \vspace{-6mm}\caption{\small{\it Adult}, efficacy of \wru  on classification}
        \label{fig:exp-adult-wdt}
    \end{minipage}\hfill
    \begin{minipage}[t]{0.24\linewidth}
        \centering
        \includegraphics[width=\textwidth]{submissions/submission1/shahbazi/sdt_regression_house.pdf}
        \vspace{-6mm}\caption{\small{\it House Sales in King County}, efficacy of \sru on regression}
        \label{fig:exp-hs-sdt}
    \end{minipage}\hfill
    \begin{minipage}[t]{0.24\linewidth}
        \centering
        \includegraphics[width=\textwidth]{submissions/submission1/shahbazi/wdt_regression_house.pdf}
        \vspace{-6mm}\caption{\small{\it House Sales in King County}, efficacy \wru on regression}
        \label{fig:exp-hs-wdt}
    \end{minipage}
\vspace{-5mm}
\end{figure}
 %%%%%%%%%%%%%%%%%%%%%%%%%%%%%%%% RELATED WORK  %%%%%%%%%%%%%%%%%%%%%%%%%%%%%%%%
\section{Related Work}\label{related} 

Bias in data has been looked at for a long time in statistical community~\cite{neyman1936contributions} but social data presents different challenges~\cite{olteanu2019social,fairmlbook,barocas2016big,jk2019bias,drosou2017diversity}.
The diversity and representativeness of data have been widely studied~\cite{drosou2017diversity}, in fields such as social science~\cite{berrey2015enigma, dobbin2016diversity,simpson1949measurement}, political science~\cite{surowiecki2005wisdom}, and information retrieval~\cite{agrawal2009diversifying}. 
Tracing back machine bias to its source, there have been major efforts to identify different types~\cite{mehrabi2021survey, olteanu2019social,friedman1996bias} and sources~\cite{torralba2011unbiased,crawford2013hidden,diakopoulos2015algorithmic} of biases in data. Efforts to satisfy {\it responsible data} requirements~\cite{nargesian2022responsible} extend to various stages of the data analysis pipeline, including data annotation~\cite{li2020towards,lazier2023fairness}, data cleaning and repair~\cite{SalimiRHS19,tae2019data,salimi2020database}, data imputation~\cite{martinez2019fairness}, entity resolution~\cite{shahbazi2023through,fanourakis2023fairer}, data integration~\cite{nargesian2022responsible,nargesian2021tailoring}, etc. 

\paragraph{Data Coverage:}The notion of data coverage has received extensive attention from different angles. Detecting lack of coverage has been studied for datasets with discrete~\cite{asudeh2019assessing} and continuous~\cite{asudeh2021coverage} attributes populated in single or multiple \cite{lin2020identifying} relations.
To resolve insufficient coverage, \cite{accinelli2020coverage, accinelli2021impact,shetiya2022fairness}
consider resolving representation bias in preprocessing pipelines by rewriting queries into the closest operation so that certain subgroups are sufficiently represented in the downstream tasks. Alternatively, ~\cite{asudeh2019assessing,tae2021slice} propose a data collection strategy to acquire as little additional data as possible (to minimize the associated costs) to meet the representation constraints. ~\cite{sharma2020data,iosifidis2018dealing,celis2020data} opt for a data augmentation approach by adding partially altered duplicates of already existing tuples or generating new synthetic entries from existing data. Consequently, the new data set has an equal number of elements for different groups, resulting in potentially resolving the under-representation issues. Finally,  \cite{nargesian2021tailoring} utilizes data integration techniques to consolidate data from different sources into a single dataset to resolve representation bias.
Related works also include ~\cite{chung2019slice,sagadeeva2021sliceline,tae2021slice} that seek to understand if the overall performance of the model fails to reflect and performs poorly on certain slices in the data.
As alternative approaches to measure representation bias, the notion of representation rate~\cite{celis2020data} (a.k.a. equal base rate~\cite{kleinberg2016inherent}) is introduced which compared with coverage, it is more restrictive as it requires almost equal ratios from different groups.
Please refer to \cite{shahbazi2023representation} for a comprehensive survey about representation bias in data. 

\paragraph{ML Reliability:} Model-centric works for uncertainty quantification such as 
probabilistic classifiers~\cite{zadrozny2001obtaining,zadrozny2002transforming,platt1999probabilistic,niculescu2005predicting},
prediction intervals (PIs) \cite{chatfield93predictionintervals,pearce2018high,khosravi2010lower} and conformal predictions (CP)~\cite{angelopoulos2021gentle,shafer2008tutorial} that are used for measuring prediction uncertainty, are built
by maximizing the {\it expected performance} on {\it random} sample from the underlying distribution.
As a result, while providing accurate estimations for the dense regions of data (e.g. majority groups), their estimation accuracy is questionable for the poorly represented regions.
In particular, \cite{angelopoulos2021gentle} recognizes the lack of guarantees in the performance of CP for such regions.
Besides, the bulk of work on trustworthy AI provides information that {\it supports} the outcome of an ML model. For example, existing work on explainable AI, including~\cite{harradon2018causal,ribeiro2016should,gunning2019darpa}, aims to find simple explanations and rules that justify the outcome of a model.
Conversely, we aim to {\it raise warning signals} when the outcome of a model is {\it not} trustworthy. That is, to provide reasons that {\it cast doubt} on the reliability of the model outcome {for a given query point}.

 %%%%%%%%%%%%%%%%%%%%%%%%%%%%%%%% FUTURE  %%%%%%%%%%%%%%%%%%%%%%%%%%%%%%%%
% \vspace{-3mm}
\section{Final Remarks}\label{sec:conclusion}
As Data-centric AI and Responsible AI emerge as focal points in data science research, the development of Data-centric methodologies for ensuring Responsible and Trustworthy AI attracts increasing attention.
While there is some excellent work on responsible data management to achieve this goal, there remain many challenges yet to be addressed.

In this paper, we focused on a crucial aspect of responsible data -- detecting and addressing the under-representation of minorities within a data set.
We formally defined the notion of data coverage and discussed various techniques for (a) identifying lack of representation issues across different data modalities, (b) ensuring proper representation of minorities in data, and (c) limiting the scope-of-use of data sets based on their representation issues by generating proper ({\sc RU}) warning signals.
Even though the research on detecting lack of coverage issues is relatively mature, resolution techniques are still understudied.
Considering the recent advancements in Generative AI, utilizing Foundation Models and Large Language Models, and studying their limitations, for data augmentation to improve the representation of minorities at the data level seems interesting to further explore.

 %%%%%%%%%%%%%%%%%%%%%%%%%%%%%%%% BIB  %%%%%%%%%%%%%%%%%%%%%%%%%%%%%%%%
\bibliographystyle{unsrt}
\small
% \bibliography{ref}
\begin{thebibliography}{10}

\bibitem{asudeh2019assessing}
A.~Asudeh, Z.~Jin, and H.~Jagadish.
\newblock Assessing and remedying coverage for a given dataset.
\newblock In {\em ICDE}, pages 554--565. IEEE, 2019.

\bibitem{shahbazi2023representation}
N.~Shahbazi, Y.~Lin, A.~Asudeh, and H.~Jagadish.
\newblock Representation bias in data: A survey on identification and resolution techniques.
\newblock {\em ACM Computing Surveys}, 2023.

\bibitem{asudeh2021coverage}
A.~Asudeh, N.~Shahbazi, Z.~Jin, and H.~V. Jagadish.
\newblock Identifying insufficient data coverage for ordinal continuous-valued attributes.
\newblock In {\em SIGMOD}. ACM, 2021.

\bibitem{mousavi2024data}
M.~Mousavi, N.~Shahbazi, and A.~Asudeh.
\newblock Data coverage for detecting representation bias in image datasets: {A} crowdsourcing approach.
\newblock In {\em {EDBT}}, pages 47--60, 2024.

\bibitem{nargesian2021tailoring}
F.~Nargesian, A.~Asudeh, and H.~Jagadish.
\newblock Tailoring data source distributions for fairness-aware data integration.
\newblock {\em Proceedings of the VLDB Endowment}, 14(11):2519--2532, 2021.

\bibitem{nargesian2022responsible}
F.~Nargesian, A.~Asudeh, and H.~V. Jagadish.
\newblock Responsible data integration: Next-generation challenges.
\newblock {\em SIGMOD}, 2022.

\bibitem{sharma2020data}
S.~Sharma, Y.~Zhang, J.~M. R{\'\i}os~Aliaga, D.~Bouneffouf, V.~Muthusamy, and K.~R. Varshney.
\newblock Data augmentation for discrimination prevention and bias disambiguation.
\newblock In {\em AIES}, pages 358--364, 2020.

\bibitem{DBLP:journals/jair/ChawlaBHK02}
N.~V. Chawla, K.~W. Bowyer, L.~O. Hall, and W.~P. Kegelmeyer.
\newblock {SMOTE:} synthetic minority over-sampling technique.
\newblock {\em J. Artif. Intell. Res.}, 16:321--357, 2002.

\bibitem{iosifidis2018dealing}
V.~Iosifidis and E.~Ntoutsi.
\newblock Dealing with bias via data augmentation in supervised learning scenarios.
\newblock {\em Jo Bates Paul D. Clough Robert J{\"a}schke}, 24, 2018.

\bibitem{celis2020data}
L.~E. Celis, V.~Keswani, and N.~Vishnoi.
\newblock Data preprocessing to mitigate bias: A maximum entropy based approach.
\newblock In {\em ICML}, pages 1349--1359. PMLR, 2020.

\bibitem{asudeh2022towards}
A.~Asudeh and F.~Nargesian.
\newblock Towards distribution-aware query answering in data markets.
\newblock {\em Proceedings of the VLDB Endowment}, 15(11):3137--3144, 2022.

\bibitem{motwani1995randomized}
R.~Motwani and P.~Raghavan.
\newblock {\em Randomized algorithms}.
\newblock Cambridge university press, 1995.

\bibitem{chameleon}
M.~Erfanian, H.~V. Jagadish, and A.~Asudeh.
\newblock Chameleon: Foundation models for fairness-aware multi-modal data augmentation to enhance coverage of minorities.
\newblock {\em arXiv preprint arXiv:2402.01071}, 2024.

\bibitem{scholkopf1999support}
B.~Sch{\"o}lkopf, R.~C. Williamson, A.~Smola, J.~Shawe-Taylor, and J.~Platt.
\newblock Support vector method for novelty detection.
\newblock {\em NeurIPS}, 12, 1999.

\bibitem{phillips1998feret}
P.~J. Phillips, H.~Wechsler, J.~Huang, and P.~J. Rauss.
\newblock The feret database and evaluation procedure for face-recognition algorithms.
\newblock {\em Image and vision computing}, 16(5):295--306, 1998.

\bibitem{dressel2018accuracy}
J.~Dressel and H.~Farid.
\newblock The accuracy, fairness, and limits of predicting recidivism.
\newblock {\em Science advances}, 4(1):eaao5580, 2018.

\bibitem{ng2021mlops}
A.~Ng.
\newblock Mlops: From model-centric to data-centric {AI}.
\newblock 2021.

\bibitem{wing2021trustworthy}
J.~M. Wing.
\newblock Trustworthy {AI}.
\newblock {\em CACM}, 64(10):64--71, 2021.

\bibitem{kentour2021analysis}
M.~Kentour and J.~Lu.
\newblock Analysis of trustworthiness in machine learning and deep learning.
\newblock {\em InfoComp}, 2021.

\bibitem{liu2021trustworthy}
H.~Liu, Y.~Wang, W.~Fan, X.~Liu, Y.~Li, S.~Jain, A.~K. Jain, and J.~Tang.
\newblock Trustworthy {AI}: A computational perspective.
\newblock {\em arXiv preprint arXiv:2107.06641}, 2021.

\bibitem{singh2021trustworthy}
R.~Singh, M.~Vatsa, and N.~Ratha.
\newblock Trustworthy {AI}.
\newblock In {\em 8th ACM IKDD CODS and 26th COMAD}, pages 449--453. 2021.

\bibitem{kulynych2022you}
B.~Kulynych, Y.-Y. Yang, Y.~Yu, J.~B{\l}asiok, and P.~Nakkiran.
\newblock What you see is what you get: Distributional generalization for algorithm design in deep learning.
\newblock {\em arXiv preprint arXiv:2204.03230}, 2022.

\bibitem{kakade2003sample}
S.~M. Kakade.
\newblock {\em On the sample complexity of reinforcement learning}.
\newblock University of London, University College London (United Kingdom), 2003.

\bibitem{dwork2012fairness}
C.~Dwork, M.~Hardt, T.~Pitassi, O.~Reingold, and R.~Zemel.
\newblock Fairness through awareness.
\newblock In {\em ITCS}, pages 214--226, 2012.

\bibitem{techrep}
N.~Shahbazi and A.~Asudeh.
\newblock Data-centric reliability evaluation of individual predictions.
\newblock {\em CoRR, abs/2204.07682}, 2022.

\bibitem{adult}
M.~Lichman.
\newblock Adult income dataset, {UCI} machine learning repository.
\newblock \url{https://archive.ics.uci.edu/ml/datasets/adult}, 2013.

\bibitem{neyman1936contributions}
J.~Neyman and E.~S. Pearson.
\newblock Contributions to the theory of testing statistical hypotheses.
\newblock {\em Statistical Research Memoirs}, 1936.

\bibitem{olteanu2019social}
A.~Olteanu, C.~Castillo, F.~Diaz, and E.~Kiciman.
\newblock Social data: Biases, methodological pitfalls, and ethical boundaries.
\newblock {\em Frontiers in Big Data}, 2:13, 2019.

\bibitem{fairmlbook}
S.~Barocas, M.~Hardt, and A.~Narayanan.
\newblock Fairness and machine learning: Limitations and opportunities.
\newblock \url{fairmlbook.org}, 2019.

\bibitem{barocas2016big}
S.~Barocas and A.~D. Selbst.
\newblock Big data's disparate impact.
\newblock {\em Calif. L. Rev.}, 104:671, 2016.

\bibitem{jk2019bias}
J.~Kleinberg.
\newblock Fairness, rankings, and behavioral biases.
\newblock FAT*, 2019.

\bibitem{drosou2017diversity}
M.~Drosou, H.~Jagadish, E.~Pitoura, and J.~Stoyanovich.
\newblock Diversity in big data: A review.
\newblock {\em Big data}, 5(2):73--84, 2017.

\bibitem{berrey2015enigma}
E.~Berrey.
\newblock {\em The enigma of diversity: The language of race and the limits of racial justice}.
\newblock University of Chicago Press, 2015.

\bibitem{dobbin2016diversity}
F.~Dobbin and A.~Kalev.
\newblock Why diversity programs fail and what works better.
\newblock {\em Harvard Business Review}, 94(7-8):52--60, 2016.

\bibitem{simpson1949measurement}
E.~H. Simpson.
\newblock Measurement of diversity.
\newblock {\em Nature}, 163(4148), 1949.

\bibitem{surowiecki2005wisdom}
J.~Surowiecki.
\newblock {\em The wisdom of crowds}.
\newblock Anchor, 2005.

\bibitem{agrawal2009diversifying}
R.~Agrawal, S.~Gollapudi, A.~Halverson, and S.~Ieong.
\newblock Diversifying search results.
\newblock In {\em WSDM}, pages 5--14. ACM, 2009.

\bibitem{mehrabi2021survey}
N.~Mehrabi, F.~Morstatter, N.~Saxena, K.~Lerman, and A.~Galstyan.
\newblock A survey on bias and fairness in machine learning.
\newblock {\em ACM Computing Surveys (CSUR)}, 54(6):1--35, 2021.

\bibitem{friedman1996bias}
B.~Friedman and H.~Nissenbaum.
\newblock Bias in computer systems.
\newblock {\em TOIS}, 14(3):330--347, 1996.

\bibitem{torralba2011unbiased}
A.~Torralba and A.~A. Efros.
\newblock Unbiased look at dataset bias.
\newblock In {\em CVPR 2011}, pages 1521--1528. IEEE, 2011.

\bibitem{crawford2013hidden}
K.~Crawford.
\newblock The hidden biases in big data.
\newblock {\em Harvard business review}, 1(4), 2013.

\bibitem{diakopoulos2015algorithmic}
N.~Diakopoulos.
\newblock Algorithmic accountability: Journalistic investigation of computational power structures.
\newblock {\em Digital journalism}, 3(3):398--415, 2015.

\bibitem{li2020towards}
Y.~Li, H.~Sun, and W.~H. Wang.
\newblock Towards fair truth discovery from biased crowdsourced answers.
\newblock In {\em SIGKDD}, pages 599--607, 2020.

\bibitem{lazier2023fairness}
S.~Lazier, S.~Thirumuruganathan, and H.~Anahideh.
\newblock Fairness and bias in truth discovery algorithms: An experimental analysis.
\newblock {\em arXiv preprint arXiv:2304.12573}, 2023.

\bibitem{SalimiRHS19}
B.~Salimi, L.~Rodriguez, B.~Howe, and D.~Suciu.
\newblock Interventional fairness: Causal database repair for algorithmic fairness.
\newblock In {\em {SIGMOD}}, pages 793--810. {ACM}, 2019.

\bibitem{tae2019data}
K.~H. Tae, Y.~Roh, Y.~H. Oh, H.~Kim, and S.~E. Whang.
\newblock Data cleaning for accurate, fair, and robust models: A big data-{AI} integration approach.
\newblock In {\em DEEM workshop}, pages 1--4, 2019.

\bibitem{salimi2020database}
B.~Salimi, B.~Howe, and D.~Suciu.
\newblock Database repair meets algorithmic fairness.
\newblock {\em ACM SIGMOD Record}, 49(1):34--41, 2020.

\bibitem{martinez2019fairness}
F.~Mart{\'\i}nez-Plumed, C.~Ferri, D.~Nieves, and J.~Hern{\'a}ndez-Orallo.
\newblock Fairness and missing values.
\newblock {\em arXiv preprint arXiv:1905.12728}, 2019.

\bibitem{shahbazi2023through}
N.~Shahbazi, N.~Danevski, F.~Nargesian, A.~Asudeh, and D.~Srivastava.
\newblock Through the fairness lens: Experimental analysis and evaluation of entity matching.
\newblock {\em Proceedings of the VLDB Endowment}, 16(11):3279--3292, 2023.

\bibitem{fanourakis2023fairer}
N.~Fanourakis, C.~Kontousias, V.~Efthymiou, V.~Christophides, and D.~Plexousakis.
\newblock Fairer demo: Fairness-aware and explainable entity resolution.
\newblock 2023.

\bibitem{lin2020identifying}
Y.~Lin, Y.~Guan, A.~Asudeh, and H.~Jagadish.
\newblock Identifying insufficient data coverage in databases with multiple relations.
\newblock {\em Proceedings of the VLDB Endowment}, 13(12):2229--2242, 2020.

\bibitem{accinelli2020coverage}
C.~Accinelli, S.~Minisi, and B.~Catania.
\newblock Coverage-based rewriting for data preparation.
\newblock In {\em EDBT Workshops}, 2020.

\bibitem{accinelli2021impact}
C.~Accinelli, B.~Catania, G.~Guerrini, and S.~Minisi.
\newblock The impact of rewriting on coverage constraint satisfaction.
\newblock In {\em EDBT Workshops}, 2021.

\bibitem{shetiya2022fairness}
S.~Shetiya, I.~P. Swift, A.~Asudeh, and G.~Das.
\newblock Fairness-aware range queries for selecting unbiased data.
\newblock In {\em ICDE}. IEEE, 2022.

\bibitem{tae2021slice}
K.~H. Tae and S.~E. Whang.
\newblock Slice tuner: A selective data acquisition framework for accurate and fair machine learning models.
\newblock In {\em SIGMOD}, pages 1771--1783, 2021.

\bibitem{chung2019slice}
Y.~Chung, T.~Kraska, N.~Polyzotis, K.~H. Tae, and S.~E. Whang.
\newblock Slice finder: Automated data slicing for model validation.
\newblock In {\em ICDE}, pages 1550--1553. IEEE, 2019.

\bibitem{sagadeeva2021sliceline}
S.~Sagadeeva and M.~Boehm.
\newblock Sliceline: Fast, linear-algebra-based slice finding for ml model debugging.
\newblock In {\em SIGMOD}, pages 2290--2299, 2021.

\bibitem{kleinberg2016inherent}
J.~Kleinberg, S.~Mullainathan, and M.~Raghavan.
\newblock Inherent trade-offs in the fair determination of risk scores.
\newblock {\em arXiv preprint arXiv:1609.05807}, 2016.

\bibitem{zadrozny2001obtaining}
B.~Zadrozny and C.~Elkan.
\newblock Obtaining calibrated probability estimates from decision trees and naive bayesian classifiers.
\newblock In {\em ICML}, volume~1, pages 609--616. Citeseer, 2001.

\bibitem{zadrozny2002transforming}
B.~Zadrozny and C.~Elkan.
\newblock Transforming classifier scores into accurate multiclass probability estimates.
\newblock In {\em SIGKDD}, pages 694--699, 2002.

\bibitem{platt1999probabilistic}
J.~Platt et~al.
\newblock Probabilistic outputs for support vector machines and comparisons to regularized likelihood methods.
\newblock {\em Advances in large margin classifiers}, 10(3):61--74, 1999.

\bibitem{niculescu2005predicting}
A.~Niculescu-Mizil and R.~Caruana.
\newblock Predicting good probabilities with supervised learning.
\newblock In {\em Proceedings of the 22nd international conference on Machine learning}, pages 625--632, 2005.

\bibitem{chatfield93predictionintervals}
C.~Chatfield.
\newblock Prediction intervals.
\newblock {\em Journal of Business and Economic Statistics}, 11:121--135, 1993.

\bibitem{pearce2018high}
T.~Pearce, A.~Brintrup, M.~Zaki, and A.~Neely.
\newblock High-quality prediction intervals for deep learning: A distribution-free, ensembled approach.
\newblock In {\em International conference on machine learning}, pages 4075--4084. PMLR, 2018.

\bibitem{khosravi2010lower}
A.~Khosravi, S.~Nahavandi, D.~Creighton, and A.~F. Atiya.
\newblock Lower upper bound estimation method for construction of neural network-based prediction intervals.
\newblock {\em IEEE transactions on neural networks}, 22(3):337--346, 2010.

\bibitem{angelopoulos2021gentle}
A.~N. Angelopoulos and S.~Bates.
\newblock A gentle introduction to conformal prediction and distribution-free uncertainty quantification.
\newblock {\em arXiv preprint arXiv:2107.07511}, 2021.

\bibitem{shafer2008tutorial}
G.~Shafer and V.~Vovk.
\newblock A tutorial on conformal prediction.
\newblock {\em Journal of Machine Learning Research}, 9(3), 2008.

\bibitem{harradon2018causal}
M.~Harradon, J.~Druce, and B.~Ruttenberg.
\newblock Causal learning and explanation of deep neural networks via autoencoded activations.
\newblock {\em arXiv preprint arXiv:1802.00541}, 2018.

\bibitem{ribeiro2016should}
M.~T. Ribeiro, S.~Singh, and C.~Guestrin.
\newblock " why should i trust you?" explaining the predictions of any classifier.
\newblock In {\em SIGKDD}, pages 1135--1144, 2016.

\bibitem{gunning2019darpa}
D.~Gunning and D.~Aha.
\newblock Darpa’s explainable artificial intelligence ({XAI}) program.
\newblock {\em AI Magazine}, 40(2):44--58, 2019.

\end{thebibliography}

\end{document}

\end{article}



\end{articlesection}

% put the news items below- there can be multiple news sections
% each with its own title
% news will usually have an author as well as a title, 
% e.g. TCDE elections
% news articles are in the same format as letters
% typically, news articles will be stored in a directory called "news"

%\begin{newssection}{News headline}

% insert news items here; news will typically have authors
% see the Sept. 2018 issue for an example

%\begin{news}{news item title}
%{author name}{author affiliation}
%\input{news/news-article.tex}
%\end{news}
%
%\newpage


%\end{newssection}

\begin{callsection}

%  This section will be empty for your version
%
%  Calls for papers section.  Use the callsection environment.
%  Each call for papers is contained in an call environment, where the single 
%  required options to \begin{call} is the name of the conference.
% typically calls are stored in a "calls" directory
%
%\begin{call}{name of conference}
%\centerline{\includegraphics[width=\textwidth, bb= 0 0 590 760]{calls/conference-name.pdf}}
%\end{call}
%\begin{call}{ICDE 2019 Conference}
%\centerline{\includegraphics[width=\textwidth, bb= 0 0 610 790] {../Dec-2018/calls/icde19.pdf}} 
%\centerline{\includegraphics[width=\textwidth, bb= 0 0 590 760] {calls/icde19.pdf}}
%\end{call}
\begin{call}{TCDE Membership Form}
%\centerline{\includegraphics[width=\textwidth, bb= 0 0 610 790]
%\centerline{\includegraphics[width=\textwidth, bb= 0 0 590 760] {../Dec-2018/calls/tcde.pdf}}
\centerline{\includegraphics[width=\textwidth, bb= 0 0 590 760] {../2020-09/calls/tcde.pdf}}
\end{call}

\end{callsection}

\end{bulletin}
\end{document}
