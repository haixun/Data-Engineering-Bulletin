Users author dataframe queries in Jupyter notebooks,  comprising code cells and output from executing these code cells. Figure~\ref{fig:files} shows an example notebook containing dataframe queries on the left. Each code cell contains one or more queries and sometimes ends with a query that outputs results. In this part of Figure~\ref{fig:files}, every cell ends in a query (namely, \code{df1.describe()}, \code{df1.head()}, and \code{df2.describe()}) that outputs results.
Dataframe queries are comprised of operators such as \code{apply} (applying a user defined function on rows/columns), \code{describe} (compute and show summary statistics), and \code{head} (retrieve the top $K$ rows of the dataframe). Operators such as \code{head} and \code{describe}, or simply the dataframe variable itself, are used for inspecting intermediate results. We call these operators \textbf{\textit{interactions}}. 
Users construct queries incrementally by introducing interactions to verify intermediate results. An interaction usually depends on only a subset of the operators specified before it. 
For example, \code{df1.describe()} in Figure~\ref{fig:files} depends only on 
\code{df1 = pd.read\_csv("small\_file")}
but not
\code{df2 = pd.read\_csv("LARGE\_FILE")}.
We call the set of dependencies of an interaction the \textbf{\textit{interaction critical path}}. To show the results of a particular interaction, the operators not on its interaction critical path do not need to be executed even if they were specified before the interaction.

After an interaction, users spend time inspecting the output and authoring new queries based on the output. We call the time between the display of the output and the submission of the next query \textbf{\textit{\thinktime}}, during which the CPU is idle (assuming there are no other processes running on the same server) while the user inspects intermediate results and authors new queries.
We propose \textbf{\textit{opportunistic evaluation}}, an optimization framework that leverages this \thinktime to reduce interactive latency. 
In this framework, the execution of operators that are not on interaction critical paths, which we call \textit{non-critical operators}, are deferred to being evaluated asynchronously during \thinktime to speed up future interactions.
