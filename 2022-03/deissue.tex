\documentclass[11pt]{article}
\usepackage{debulletin,times,epsfig,subfigure,wrapfig,algorithmic,color,boxedminipage,graphicx,url}

% this is the template for an issue of the Data Engineering Bulletin

% all packages used by any paper must be listed here
%\usepackage{hyperref}
%\usepackage{authblk}
%\setlength{\affilsep}{0em}
%\usepackage{inputenc}
%\usepackage{debulletin}
%\usepackage{times}
%\usepackage{graphicx}
%\usepackage{array}
%\usepackage{wrapfig}
%\usepackage[table]{xcolor}
%\usepackage{tcolorbox}
%\usepackage{amssymb}
%%\usepackage[labelfont=bf,labelsep=space,list=true]{subcaption}
%\usepackage{url}
%\usepackage{mathtools, bm}
%\usepackage{float}
%\usepackage{multirow}
%\usepackage{multicol}
%\usepackage{algorithm}
%\usepackage{subfig}
%\usepackage{algpseudocode}
%\setlength{\intextsep}{10pt plus 2pt minus 2pt}
%\hyphenation{finally}
\usepackage[utf8]{inputenc}

\usepackage{amsmath, amssymb, amsfonts}

\usepackage{hyperref}
\usepackage{enumitem}
\usepackage{xspace} 
%\usepackage{xcolor}
\usepackage{tikz}
\usepackage[T1]{fontenc}
\usepackage{beramono}
\usepackage{listings}
\usepackage{xcolor}

% Prov paper
\usepackage{array}
\usepackage{multirow}
\usepackage{multicol}
\usepackage{hhline}
\usepackage{mathrsfs}  
\usepackage[linesnumbered,ruled]{algorithm2e}
%

% strm privacy
\usepackage[english]{babel}
%

% manos
\usepackage{booktabs} 
\usepackage{epstopdf}
\usepackage{verbatimbox}
\usepackage{multirow} 
%

\usepackage{wrapfig}

\usepackage{graphics}
\usepackage{pifont}
%\usepackage{subcaption} % for subtable
%\usepackage{threeparttable} % for tablenotes




\usepackage[numbers]{natbib}
% \documentclass{article}
% Recommended, but optional, packages for figures and better typesetting:
\usepackage{microtype}
\usepackage{graphicx}
% \usepackage{subfigure}
\usepackage{booktabs} % for professional tables
%\usepackage[table,dvipsnames]{xcolor}
\usepackage{epsfig}
\usepackage{pgfplotstable}
\usepackage{pgfplots}
\usepgfplotslibrary{groupplots}
\usepackage{bbm}
\usepackage{booktabs}
\usepackage{verbatim}
\usepackage[T1]{fontenc}
\usepackage{caption}
\usepackage{siunitx}
\usepackage{xspace}
%\usepackage[colorinlistoftodos,textsize=footnotesize]{todonotes}
%\usepackage[utf8]{inputenc}
\usepackage[autostyle, english=american]{csquotes}
\usepackage{breakurl}




\begin{document}


% please enter real date, vol no, issue no
\bulletindate{March 2022}
\bulletinvolume{45}
\bulletinnumber{1}
\bulletinyear{2022}

% these are files that I have- but your part of the issue can be done without
% them
\IEEElogo{cs.pdf}
\insidefrontcover{incvA19.pdf}
%\insidebackcover[ICDE Conference]{./calls/icde-new-a.ps}

\begin{bulletin}

% the above samples assume the issue is generated from a directory structure of the following sort
% major directory name is month and year of issue
% there are sub-directorys for
% letters: directory name is "letters"
% technical articles: a directory per paper, named for an "author"
% news articles: directory name is "news"
% calls: directory name is "calls

%
%  Editor letters section.  Use the lettersection environment.
%  Each letter is contained in a letter environment, where the two required
%  options to \begin{letter} are the author and the address of the author.
%

\begin{lettersection}

% there will be other letters- and a blank page will appear in your document
% but the special issue part will be fine

\begin{letter}{Letter from the Editor-in-Chief}
{Haixun Wang}{Instacart}
\documentclass[11pt]{article} 

\usepackage{deauthor,times,graphicx}
%\usepackage{url}
\usepackage{hyperref}

\begin{document}
How to efficiently and effectively manage large-scale data is a
critical challenge in data management, scientific computing, machine
learning, and many other fields. In this issue, we look into this
problem from two angles.

Gerhard Weikum's opinion piece titled ``Entities with Quantities''
highlights development along the direction of querying the Web as a
database. We have come a long way in keyword based Web search: Today,
all major search engines support entity based question/answering to
certain extent (e.g., returning ``Eiffel Tower'' for query ``the
highest building in Paris''). Weikum is taking one important step
towards the goal of querying the Web as a database. In the article, he
discusses what it takes to find all entities that satisfy a
quantity-based search condition, for example, ``buildings taller than
500m'' or ``runners completing a marathon under 2:10h.''  It is clear
that this requires much advanced data preprocessing (e.g., information
extraction, entity linking, etc.), but more importantly, it requires
that at least part of the data on the entire Web needs to be organized
as a database.

Philippe Bonnet put together the current issue consisting of 5 papers
from leading researchers in the high performance computing and data
management communities on the topic of data management at
Exascale. Advances in exascale computing on petascale supercomputers
are pushing the frontier of scientific computing that requires complex
simulation, benefiting applications ranging from astrophysical
discovery to drug design. But with increasing amounts of data, the gap
between computation and I/O has grown significantly wider, which makes
data management a big challenge. This timely issue answers many
questions in this domain.

\end{document}


\end{letter}
%
\newpage
%
%% your introductory letter goes here
%
%\begin{letter}{Letter from the Special Issue Editor}
\begin{letter}{Letter from the Special Issue Editor} %JF: made it editors, plural
{Sebastian Schelter}{University of Amsterdam \& Ahold Delhaize Research, The Netherlands}
\documentclass[11pt]{article} 

\usepackage{deauthor,times,graphicx}
%\usepackage{url}

\begin{document}


Software systems that store and process personal data have become ubiquitous over the last years and have enabled numerous economic, technological and scientific advances. Unfortunately, the benefits of data-driven analysis and decision making have also been accompanied by several negative developments. Examples include the increased surveillance capabilities of the state~\cite{noplacetohide} and private companies~\cite{surveillancecapitalism}, negative impact on economic inequality~\cite{automatinginequality} and traumatic experiences for individuals~\cite{beyondbroken}. As a reaction, many countries have started to regulate data storage and processing to guarantee and protect the rights of individuals. The most comprehensive such regulation is the 
General Data Protection Regulation (GDPR, {\footnotesize\url{https://gdpr.eu}}) issued by the European Union. 

In this special issue on {\em Directions Towards GDPR-Compliant Data Systems and Applications}, we continue the ongoing discussion in the data management community~\cite{gdpr-benchmark} on how to redesign data systems and applications to be compliant with such regulation. 


\vspace{0.1cm}
\noindent\textbf{Data deletion as a first-class-citizen.} The first three papers of this issue address an important question originating from the ``right-to-be-forgotten'' postulated by GDPR: {\em How can we design efficient data systems that support the timely deletion of data as a first-class citizen?} 
%
The first paper on {\em Disposal by Design} presents a vision for automating data disposal which takes into account processing constraints, regulatory constraints as well as storage constraints, and gives concrete examples from the e-commerce domain, including a suggestion of how to to find summaries of relational data with machine learning. 
%
The second paper on {\em Building Deletion-Compliant Data Systems} argues that the the requirement of timely deletion of user data is becoming central in modern data management scenarios. The authors present a new framework for building deletion-compliant data systems from a holistic perspective, analyse the requirements derived from the new policies, and propose changes in the application and the system layer of data management systems.
%
The third paper called {\em Provenance-based Model Maintenance: Implications for Privacy} focuses on efficient data deletion in a machine learning context. In particular, the authors focus on the extremely challenging problem to refresh existing models after the removal of training samples, which is called ``machine unlearning''. They argue that GDPR regulations imply that the removed samples must be fully erased from the models so that they cannot be leaked to an adversary. The paper reviews two provenance-based solutions and shows how they can guard against ``model inversion attacks", which reconstruct the removed training samples from the updated models after the unlearning process.

\vspace{0.1cm}
\noindent\textbf{Efficient data processing under regulatory constraints.} The subsequent two papers of this issue adress an orthogonal systems-related question originating from GDPR: {\em How can we design efficient data systems that comply with data processing regulations?} 
%
The fourth paper of this issue on {\em Navigating Compliance with Data Transfers in Federated Data Processing} presents work on novel systems and methods for federated data processing, where the processing of geo-distributed data is subjected to data transfer regulations. The authors showcase recent work on compliant geo-distributed data processing and present research challenges and opportunities in making federated data processing systems GDPR-compliant.
%
The fifth paper called {\em Towards Privacy by Design for Data with \MakeUppercase{strm} privacy} discusses the practical challenges of engineering teams to balance privacy and innovation, with respect to effort, data utility and computation costs. The authors argue that current approaches in scalable data systems often treat privacy as an access problem, which is at odds with important legal and design principles. Instead, the authors propose that engineering teams should shift their data privacy efforts to the point of data collection, and discuss an architectural setup for privacy-compliant stream processing applications, which is in production usage.

\vspace{1cm}
\noindent\small{\em{This work was supported by Ahold Delhaize. All content represents the opinion of the author(s), which is not necessarily shared or endorsed by their respective employers and/or sponsors.}}

\begin{thebibliography}{10}
\itemsep=1pt
\begin{small}

\bibitem{gdpr-benchmark}
S.~Supreeth, V.~Banakar, M.~Wasserman, A.~Kumar, and V.~Chidambaram.
\newblock Understanding and Benchmarking the Impact of GDPR on Database Systems.
\newblock Proceedings of the VLDB Endowment 13(7), 2019.

\bibitem{automatinginequality}
V.~Eubanks.
\newblock Automating inequality: How high-tech tools profile, police, and punish the poor. 
\newblock St. Martin's Press, 2018.

\bibitem{noplacetohide}
G.~Greenwald.
\newblock No place to hide: Edward Snowden, the NSA, and the US surveillance state. 
\newblock Macmillan, 2014.

\bibitem{surveillancecapitalism}
S.~Zuboff.
\newblock The age of surveillance capitalism: The fight for a human future at the new frontier of power.
\newblock Profile books,~2019.

\bibitem{beyondbroken}
V.~Warmerdam.
\newblock Beyond Broken - Horrible Remedies for Broken Recommenders.
\newblock Published online at \footnotesize{\texttt{https://koaning.io/posts/beyond-broken/}}, 2021.



\end{small}
\end{thebibliography}



\end{document}

\end{letter}

\end{lettersection}

% put the name of your special issue below

\begin{opinionsection}
\begin{opinion}{Data Errors:  Symptoms,  Causes and  Origins}
{Ihab F. Ilyas, Felix  Naumann}{University  of  Waterloo, University of Potsdam}
\documentclass[11pt]{article}
\usepackage{deauthor,graphicx,times}
\usepackage{cite}

% COMMANDS GO HERE
%please list all commands used in your paper, eliminating duplicates

\newtheorem{Example}{\textbf{Example}}


% Your paper submission must be in a folder named with the
% contact author name.

\graphicspath{{author-name/}}

\begin{document}


%\title{Data Errors: Symptoms, Causes and Origins}

%% \author{
%% Ihab F. Ilyas \\
%% University of Waterloo\\ Canada  \\
%%  ilyas@cs.uwaterloo.ca\\
%% \and Felix Naumann\\
%%  Hasso Plattner Institute, University of Potsdam\\Germany \\
%%  felix.naumann@hpi.de
%% }

%% \maketitle

%\begin{abstract}
%\end{abstract}

%%!TEX root = paper.tex
\section{Introduction: Data Errors and their Root Causes}
\label{sec:intro}

With the recent move towards data-centric AI, data quality is now playing an even bigger role in producing sound and reliable insights, predictions and analytics. While the data management community has been working on the problem of data cleaning for decades, the problem remains very much present. Most efforts have focused on error detection~\cite{DBLP:books/acm/IlyasC19}, attempting to leverage symptoms and the manifestations of these errors in data sets to locate and possibly repair them. 
Indeed, the last few years witnessed significant advances in automating error detection and repairing~\cite{holoclean,holodetect,raha,yeye-unidetect} by probabilistically modelling dirty data sets, and reasoning about error detection and repair as structured prediction problems~\cite{puds,uai_heidari}. In this opinion piece, we present our views on how to further advance the field of data cleaning, and go beyond treating the symptoms of the problem and understand what it takes to treat the causes and the sources of these anomalies and errors.

Tracking errors to their sources is not a new quest of the research community~\cite{DBLP:conf/sigmod/ChalamallaIOP14, DBLP:conf/sigmod/WangM017, DBLP:journals/ftdb/GlavicMR21}. So why are we revisiting it now? %The main observation that motivates this discussion is 
The way current research currently reasons about the root causes of data errors is still, in our opinion, limited. \emph{Tracking errors to sources} has often been framed as  ``computational'' provenance that represents \emph{what} was involved in computing a final data product and possibly \emph{how} this product was computed. The main goal of these provenance-based error tracking systems is projecting errors detected in downstream applications all the way to upstream data sources, where they should be fixed~\cite{DBLP:conf/sigmod/ChalamallaIOP14}.  While the principle is sound, multiple issues often complicate this approach. First, as data processing pipelines become more complex, with cascades of complex machine learning models, capturing this rich provenance information becomes harder, as most if not all of the input is involved in producing the output. Recent progress, however, has been made in tracking responsibility of training data, for example, in the predictions of complex models~\cite{DBLP:journals/corr/abs-2002-08484,pmlr-koh-Liang,DBLP:journals/corr/abs-2202-00622}. Second, even with advances in modelling the responsibility of input sources in the observations in output analytics reports, fixing the sources does not mean that errors have been fixed at their true ``roots'' -- their point of creation; these raw data sources are often the results of other processes not modelled at all in the computational data pipelines, such as grading tasks of humans, sensor readings, and extraction scripts from logs and documents.  

Hence, we argue that effective management of data errors and the next-generation data cleaning systems require a more profound understanding of the root causes data errors. These systems should: (1)~distinguish between \emph{why} errors occur and the processes that generated them in the first place, and \emph{how}  these errors manifest themselves as bad data symptoms (e.g., violations of integrity constraints and appearing as outlying values), and (2)~explicitly model these data generation processes to allow for new process repair actions that go beyond fixing raw data sources. 

% NEED TO INTEGRATE THE FOLLOWING SOMEHOW 
%
%In the absence of any context, a data element\footnote{For now, we formulate our definitions in the most general way by referring to ``data elements'', which can manifest as individual values, rows, columns, objects, graphs, tables or entire datasets. Later, we supply details on this ``what'' question.} cannot constitute an error -- it just ``is''. However, data elements do not exist outside some context: they were somehow produced and shall somehow be consumed. Only this context allows us to identify a data element as erroneous. We can ask \emph{how} the element is an error: how do we notice it as erroneous, what symptoms does it hold? These symptoms exist only in the context of the consumption of the data element. For instance, data might violate some external constraint or assumption. It might be an outlier, compared to other data elements. A relation might be missing some tuples or have some duplicated tuples.

%Given such identified errors, we can ask the obvious next question: \emph{Why} do we observe the error, how did it appear in the data? Possible answers to these questions are the provenance of the data, and the mode in which the data was originally created. For instance, human data entry can create such errors, or faulty sensors produce erroneous data.



%!TEX root = paper.tex
\section{Introduction: Data Errors and their Root Causes}
\label{sec:intro}

With the recent move towards data-centric AI, data quality is now playing an even bigger role in producing sound and reliable insights, predictions and analytics. While the data management community has been working on the problem of data cleaning for decades, the problem remains very much present. Most efforts have focused on error detection~\cite{DBLP:books/acm/IlyasC19}, attempting to leverage symptoms and the manifestations of these errors in data sets to locate and possibly repair them. 
Indeed, the last few years witnessed significant advances in automating error detection and repairing~\cite{holoclean,holodetect,raha,yeye-unidetect} by probabilistically modelling dirty data sets, and reasoning about error detection and repair as structured prediction problems~\cite{puds,uai_heidari}. In this opinion piece, we present our views on how to further advance the field of data cleaning, and go beyond treating the symptoms of the problem and understand what it takes to treat the causes and the sources of these anomalies and errors.

Tracking errors to their sources is not a new quest of the research community~\cite{DBLP:conf/sigmod/ChalamallaIOP14, DBLP:conf/sigmod/WangM017, DBLP:journals/ftdb/GlavicMR21}. So why are we revisiting it now? %The main observation that motivates this discussion is 
The way current research currently reasons about the root causes of data errors is still, in our opinion, limited. \emph{Tracking errors to sources} has often been framed as  ``computational'' provenance that represents \emph{what} was involved in computing a final data product and possibly \emph{how} this product was computed. The main goal of these provenance-based error tracking systems is projecting errors detected in downstream applications all the way to upstream data sources, where they should be fixed~\cite{DBLP:conf/sigmod/ChalamallaIOP14}.  While the principle is sound, multiple issues often complicate this approach. First, as data processing pipelines become more complex, with cascades of complex machine learning models, capturing this rich provenance information becomes harder, as most if not all of the input is involved in producing the output. Recent progress, however, has been made in tracking responsibility of training data, for example, in the predictions of complex models~\cite{DBLP:journals/corr/abs-2002-08484,pmlr-koh-Liang,DBLP:journals/corr/abs-2202-00622}. Second, even with advances in modelling the responsibility of input sources in the observations in output analytics reports, fixing the sources does not mean that errors have been fixed at their true ``roots'' -- their point of creation; these raw data sources are often the results of other processes not modelled at all in the computational data pipelines, such as grading tasks of humans, sensor readings, and extraction scripts from logs and documents.  

Hence, we argue that effective management of data errors and the next-generation data cleaning systems require a more profound understanding of the root causes data errors. These systems should: (1)~distinguish between \emph{why} errors occur and the processes that generated them in the first place, and \emph{how}  these errors manifest themselves as bad data symptoms (e.g., violations of integrity constraints and appearing as outlying values), and (2)~explicitly model these data generation processes to allow for new process repair actions that go beyond fixing raw data sources. 

% NEED TO INTEGRATE THE FOLLOWING SOMEHOW 
%
%In the absence of any context, a data element\footnote{For now, we formulate our definitions in the most general way by referring to ``data elements'', which can manifest as individual values, rows, columns, objects, graphs, tables or entire datasets. Later, we supply details on this ``what'' question.} cannot constitute an error -- it just ``is''. However, data elements do not exist outside some context: they were somehow produced and shall somehow be consumed. Only this context allows us to identify a data element as erroneous. We can ask \emph{how} the element is an error: how do we notice it as erroneous, what symptoms does it hold? These symptoms exist only in the context of the consumption of the data element. For instance, data might violate some external constraint or assumption. It might be an outlier, compared to other data elements. A relation might be missing some tuples or have some duplicated tuples.

%Given such identified errors, we can ask the obvious next question: \emph{Why} do we observe the error, how did it appear in the data? Possible answers to these questions are the provenance of the data, and the mode in which the data was originally created. For instance, human data entry can create such errors, or faulty sensors produce erroneous data.


%%!TEX root = paper.tex
\section{Reasoning about the How: The Symptoms of Errors}
\label{sec:how}

In its most general form, information quality is defined as ``fitness for use''~\cite{Strong96}, i.e., its definition focuses on the use case, the application, the \emph{context} of the data at hand. Most, if not all, error detection methods make heavy use of this context. They search for and focus on the \emph{symptoms} of poor data quality and ask the question \emph{how} a particular data element is an error.
We exemplify this insight with selected examples of traditional error detection problems and their solutions.
\begin{description}
  \item[Outliers] Already by definition, outliers can be recognized only in the context of other data elements -- only these ``inliers'' make some other value an outlier. Typical outlier detection methods create a model to represent normal/typical values and mark as outliers all those values that do not fit the model~\cite{Aggarwal2013}. Whether an outlier is, in fact, an error is application-dependent and user-defined.
  
  \emph{How} is an outlier a data error? It is very different from all \emph{other} data elements, suggesting it is not the intended value for this data item.
  
  \item[Constraint violations] Constraints, such as key-constraints, dependencies, or denial constraints can be used to express the validity of a data instance. These rules are specified by experts or discovered with data profiling methods~\cite{Abedjan2018}. Rarely do they refer to individual data elements; rather they forbid the existence of some elements in the presence of others, such as a key-constraint denying any other record with the same key value. Discovering and cleaning such violations is an active research area~\cite{IlyasC15}.
  
  \emph{How} does a data element violate a constraint? It exists in the presence of some \emph{other} data element. While this is not an error on its own, the collection of these values cannot be part of the intended correct data instance. %, together with which it violates the constraint.
  
  \item[Duplicates] Within a dataset, duplicate records are multiple different representations of the same real-world entity~\cite{Naumann10}. To clean a dataset, such erroneous duplicates must be detected and then merged or eliminated. Identifying a duplicated record is, by definition, possible only by regarding other records, i.e., only in the context of the entire relation. Typical approaches intelligently create duplicate candidate pairs and then determine their similarity to decide whether they are indeed duplicates~\cite{Papadakis21}.
  
  \emph{How} is a duplicate an error? It represents the same real-world object as some \emph{other} data element, with possibly other types of errors causing a different representation.
  
  \item[Missing values] Missing values are easy to detect when they appear as null values in databases or empty strings in files. In more complex cases, ``disguised missing values'' can be recognized only by regarding their context, which usually comprises the other values in the column~\cite{FAHES_18}. 
  The typical means to ``clean'' missing values is to impute their value, again based on their context, usually the non-missing values in a column.
  
  \emph{How} is a missing value an error? It is explicitly represented to indicate an error. However, the difficulty in the case of missing value relates to the interpretation of ``null'' as {\em we don't know the value, but we should} as opposed to schema issues, for example, a relational employees table with some employees who do not have middle names.
\end{description}

Data cleaning, as a means to alleviate the symptoms of poor data quality, is an established and important research and development field, relying heavily on the context of data and its use in applications. Next, we move backwards along the data processing pipeline to explore not these symptoms, but their causes.
%!TEX root = paper.tex
\section{Reasoning about the How: The Symptoms of Errors}
\label{sec:how}

In its most general form, information quality is defined as ``fitness for use''~\cite{Strong96}, i.e., its definition focuses on the use case, the application, the \emph{context} of the data at hand. Most, if not all, error detection methods make heavy use of this context. They search for and focus on the \emph{symptoms} of poor data quality and ask the question \emph{how} a particular data element is an error.
We exemplify this insight with selected examples of traditional error detection problems and their solutions.
\begin{description}
  \item[Outliers] Already by definition, outliers can be recognized only in the context of other data elements -- only these ``inliers'' make some other value an outlier. Typical outlier detection methods create a model to represent normal/typical values and mark as outliers all those values that do not fit the model~\cite{Aggarwal2013}. Whether an outlier is, in fact, an error is application-dependent and user-defined.
  
  \emph{How} is an outlier a data error? It is very different from all \emph{other} data elements, suggesting it is not the intended value for this data item.
  
  \item[Constraint violations] Constraints, such as key-constraints, dependencies, or denial constraints can be used to express the validity of a data instance. These rules are specified by experts or discovered with data profiling methods~\cite{Abedjan2018}. Rarely do they refer to individual data elements; rather they forbid the existence of some elements in the presence of others, such as a key-constraint denying any other record with the same key value. Discovering and cleaning such violations is an active research area~\cite{IlyasC15}.
  
  \emph{How} does a data element violate a constraint? It exists in the presence of some \emph{other} data element. While this is not an error on its own, the collection of these values cannot be part of the intended correct data instance. %, together with which it violates the constraint.
  
  \item[Duplicates] Within a dataset, duplicate records are multiple different representations of the same real-world entity~\cite{Naumann10}. To clean a dataset, such erroneous duplicates must be detected and then merged or eliminated. Identifying a duplicated record is, by definition, possible only by regarding other records, i.e., only in the context of the entire relation. Typical approaches intelligently create duplicate candidate pairs and then determine their similarity to decide whether they are indeed duplicates~\cite{Papadakis21}.
  
  \emph{How} is a duplicate an error? It represents the same real-world object as some \emph{other} data element, with possibly other types of errors causing a different representation.
  
  \item[Missing values] Missing values are easy to detect when they appear as null values in databases or empty strings in files. In more complex cases, ``disguised missing values'' can be recognized only by regarding their context, which usually comprises the other values in the column~\cite{FAHES_18}. 
  The typical means to ``clean'' missing values is to impute their value, again based on their context, usually the non-missing values in a column.
  
  \emph{How} is a missing value an error? It is explicitly represented to indicate an error. However, the difficulty in the case of missing value relates to the interpretation of ``null'' as {\em we don't know the value, but we should} as opposed to schema issues, for example, a relational employees table with some employees who do not have middle names.
\end{description}

Data cleaning, as a means to alleviate the symptoms of poor data quality, is an established and important research and development field, relying heavily on the context of data and its use in applications. Next, we move backwards along the data processing pipeline to explore not these symptoms, but their causes.

%%!TEX root = paper.tex
\section{Reasoning about the Why: The Causes of Errors}
\label{sec:why}

It is time to ask (and answer) the \emph{why} question!
Existing work in the area of data quality, error detection and data cleaning almost exclusively focuses on alleviating the symptoms, rather than removing the cause of the error.
None of the methods asks \emph{why} a particular value is missing, why duplicates exist in the data, why violations occur. Answers as to ``why'' include: faulty (human) data entry, such as missing entries, misplaced values, typos, and vandalism; faulty reading from sensors; missed or not-propagated updates; faulty computations; and misconfigured data pipelines. 

While researchers and practitioners (and medical doctors) will acknowledge the truism that problems are best addressed at their source rather than treating their symptoms, the research community has not adequately addressed this opportunity possibly for several reasons:
\begin{itemize}
    \item In some scenarios, once errors are detected, it is too late -- fixing their cause is futile because the data was intended for a one-time use.
    \item Often, the creation of data is out of the control of the data engineers or data consumers: the data stems from an external source and the data creation can be influenced only through human intervention, such as communicating with the data owners or creators.
    \item Modifying or improving the data creation process is difficult or impossible, for instance due to technical or human limitations: sensors have an inherent error margin; humans are not infallible, etc.
    \item Data processing pipelines have become so complex, that treating the symptoms is the easier short-term goal with quick rewards.
\end{itemize}

Knowledge of the cause of an error and not only its symptom can improve cleaning methods and can help avoid such errors in the first place. To seize this opportunity, multiple challenges must be overcome:
\begin{itemize}
    \item \emph{Modelling} the processes and data generators (including humans) in the system, instead of modelling only the data and errors.

    \item \emph{Detection} of data errors without context and detection of erroneous data processes.
    
    \item Extending the notion of \emph{provenance} to include (possibly faulty) processes, computation (internal provenance), and data generation steps (external provenance). More on this in Section\ref{sec:provenance}.
    
    \item Designing of algorithms and systems to efficiently and effectively trace such extended provenance.
    
    \item Designing \emph{repair} operations for such errors and processes, which need to reach beyond the mere deletion or replacement of data instances that is the currently common approach.

%    \item Algorithms that can point to these new things
\end{itemize}

These challenges can be summarized as creating a more holistic view of data creation and consumption than is currently practiced. Especially the extended notion of provenance deserves a closer look in the next section.

%existing work: Modelling the process or Cleaning actions at the source

%Definition of cause of error: SOME TEXT. Understanding the cause of an error asks ``Why do we observe this error?''

%Raw-data generative process (link to ICDT paper by Ihab, and others)


%``So I'll remove the cause. But not. The symptom!'' Dr.\ Frank N.\ Furter, Rocky Horror Picture Show
%!TEX root = paper.tex
\section{Reasoning about the Why: The Causes of Errors}
\label{sec:why}

It is time to ask (and answer) the \emph{why} question!
Existing work in the area of data quality, error detection and data cleaning almost exclusively focuses on alleviating the symptoms, rather than removing the cause of the error.
None of the methods asks \emph{why} a particular value is missing, why duplicates exist in the data, why violations occur. Answers as to ``why'' include: faulty (human) data entry, such as missing entries, misplaced values, typos, and vandalism; faulty reading from sensors; missed or not-propagated updates; faulty computations; and misconfigured data pipelines. 

While researchers and practitioners (and medical doctors) will acknowledge the truism that problems are best addressed at their source rather than treating their symptoms, the research community has not adequately addressed this opportunity possibly for several reasons:
\begin{itemize}
    \item In some scenarios, once errors are detected, it is too late -- fixing their cause is futile because the data was intended for a one-time use.
    \item Often, the creation of data is out of the control of the data engineers or data consumers: the data stems from an external source and the data creation can be influenced only through human intervention, such as communicating with the data owners or creators.
    \item Modifying or improving the data creation process is difficult or impossible, for instance due to technical or human limitations: sensors have an inherent error margin; humans are not infallible, etc.
    \item Data processing pipelines have become so complex, that treating the symptoms is the easier short-term goal with quick rewards.
\end{itemize}

Knowledge of the cause of an error and not only its symptom can improve cleaning methods and can help avoid such errors in the first place. To seize this opportunity, multiple challenges must be overcome:
\begin{itemize}
    \item \emph{Modelling} the processes and data generators (including humans) in the system, instead of modelling only the data and errors.

    \item \emph{Detection} of data errors without context and detection of erroneous data processes.
    
    \item Extending the notion of \emph{provenance} to include (possibly faulty) processes, computation (internal provenance), and data generation steps (external provenance). More on this in Section\ref{sec:provenance}.
    
    \item Designing of algorithms and systems to efficiently and effectively trace such extended provenance.
    
    \item Designing \emph{repair} operations for such errors and processes, which need to reach beyond the mere deletion or replacement of data instances that is the currently common approach.

%    \item Algorithms that can point to these new things
\end{itemize}

These challenges can be summarized as creating a more holistic view of data creation and consumption than is currently practiced. Especially the extended notion of provenance deserves a closer look in the next section.

%existing work: Modelling the process or Cleaning actions at the source

%Definition of cause of error: SOME TEXT. Understanding the cause of an error asks ``Why do we observe this error?''

%Raw-data generative process (link to ICDT paper by Ihab, and others)


%``So I'll remove the cause. But not. The symptom!'' Dr.\ Frank N.\ Furter, Rocky Horror Picture Show
%%!TEX root = paper.tex
\section{True Data Provenance}
\label{sec:provenance}

Provenance is powerful tool for {\em tracking} data artifacts. In the context of this paper, one might think of it as a way to identify the \emph{where} of the data error's story. Most practical and effective cleaning solutions follow a clean-and-evaluate lifecycle~\cite{Ilyas16DE}, which leverages the computational provenance of data analytics to track data errors to their sources, and attempts to provide explanations that lead to cleaning actions. This typical lifecycle is depicted in Figure~\ref{fig:lifecycle}. 
\begin{figure*}[ht]
  \centering
  \includegraphics[width=0.6\textwidth]{lifecycle}
  \caption{Clean-and-Evaluate Loop~\cite{Ilyas16DE}}
  \label{fig:lifecycle}
\end{figure*}

Provenance and lineage systems focus on describing how the analytical {\em report views} are computed from the sources. For example, Scorpion~\cite{DBLP:journals/pvldb/0002M13}, DBRx~\cite{DBLP:conf/sigmod/ChalamallaIOP14} and QFix~\cite{DBLP:conf/sigmod/WangM017} (and many other followup work) are  solutions that trace back the tuples that contributed to the problems in the target to explain and help fix these errors at data sources. A recent survey summarizes the large body of work in debugging data-driven systems and explain what users see downstream from processing raw data~\cite{DBLP:journals/ftdb/GlavicMR21}. As these processing pipelines become more complex with cascades of large machine learning models, tracing errors in final predictions back to their causes can be very challenging. However, there is recent progress that can help us reason about observations in model predictions and track them back to errors in training data~\cite{DBLP:journals/corr/abs-2202-00622}.

The question becomes: \emph{is explaining errors in final analytics or predictions in terms of data sources enough?} What we refer to as ``raw data sources''  are often cut off the processes that generated these data, such as the human grader that input that data, the extraction script that generated this data from a webpage, or a presentation of the complex data pipeline that ran in a different software stack and generated this source data. From our discussions and involvement with large enterprises over the last decade, we argue that this decoupling is often due to two main reasons:
\begin{itemize}
\item \emph{Difficulty of integrating data processes in provenance systems:} Representing the process that generated the data might require expressive (and hence  complex) provenance systems. For example semiring-based  provenance systems have been extended  to capture information about external inputs (e.g., user choices), and  to capture process executions~\cite{DBLP:journals/vldb/DeutchMT15}; and in the context of scientific workflows,  the need for a control-flow driven workflow provenance model in contrast to the traditional  data-driven execution provenance paradigm has been explored~\cite{DBLP:journals/dke/ButtF21}.

\item \emph{Loss of provenance continuity across systems:}  We might be very careful in collecting and adequately presenting provenance information in the data pipelines we control. However, as the final data product (e.g., predictions, views, aggregates, or transformed data sets) get pushed to the downstream tasks, they are often treated as ``source data'' and downstream pipelines fail to  consume the associated provenance information.
\end{itemize}

Understandably, these are hard problems to tackle and part of the challenge is not even technical and it involves standardizing data provenance representation across business units and different software stacks. However, this might suggest new research directions; for example, we might prefer developing simpler and less expressive provenance models that target interoperability and ease of propagation over representation power of the underlying computations. Another example is that propagating standard meta-data that ties data sources to central data governance and catalogs can be part of the integrity constraints and sanity checks. We suggest also extending meta-data representation of data sources to include {\em repair actions} that reference a controlled vocabulary or a {\em repairing ontology} tapping into the large body of work in work flow and business processes management.





%!TEX root = paper.tex
\section{True Data Provenance}
\label{sec:provenance}

Provenance is powerful tool for {\em tracking} data artifacts. In the context of this paper, one might think of it as a way to identify the \emph{where} of the data error's story. Most practical and effective cleaning solutions follow a clean-and-evaluate lifecycle~\cite{Ilyas16DE}, which leverages the computational provenance of data analytics to track data errors to their sources, and attempts to provide explanations that lead to cleaning actions. This typical lifecycle is depicted in Figure~\ref{fig:lifecycle}. 
\begin{figure*}[ht]
  \centering
  \includegraphics[width=0.6\textwidth]{letters/lifecycle}
  \caption{Clean-and-Evaluate Loop~\cite{Ilyas16DE}}
  \label{fig:lifecycle}
\end{figure*}

Provenance and lineage systems focus on describing how the analytical {\em report views} are computed from the sources. For example, Scorpion~\cite{DBLP:journals/pvldb/0002M13}, DBRx~\cite{DBLP:conf/sigmod/ChalamallaIOP14} and QFix~\cite{DBLP:conf/sigmod/WangM017} (and many other followup work) are  solutions that trace back the tuples that contributed to the problems in the target to explain and help fix these errors at data sources. A recent survey summarizes the large body of work in debugging data-driven systems and explain what users see downstream from processing raw data~\cite{DBLP:journals/ftdb/GlavicMR21}. As these processing pipelines become more complex with cascades of large machine learning models, tracing errors in final predictions back to their causes can be very challenging. However, there is recent progress that can help us reason about observations in model predictions and track them back to errors in training data~\cite{DBLP:journals/corr/abs-2202-00622}.

The question becomes: \emph{is explaining errors in final analytics or predictions in terms of data sources enough?} What we refer to as ``raw data sources''  are often cut off the processes that generated these data, such as the human grader that input that data, the extraction script that generated this data from a webpage, or a presentation of the complex data pipeline that ran in a different software stack and generated this source data. From our discussions and involvement with large enterprises over the last decade, we argue that this decoupling is often due to two main reasons:
\begin{itemize}
\item \emph{Difficulty of integrating data processes in provenance systems:} Representing the process that generated the data might require expressive (and hence  complex) provenance systems. For example semiring-based  provenance systems have been extended  to capture information about external inputs (e.g., user choices), and  to capture process executions~\cite{DBLP:journals/vldb/DeutchMT15}; and in the context of scientific workflows,  the need for a control-flow driven workflow provenance model in contrast to the traditional  data-driven execution provenance paradigm has been explored~\cite{DBLP:journals/dke/ButtF21}.

\item \emph{Loss of provenance continuity across systems:}  We might be very careful in collecting and adequately presenting provenance information in the data pipelines we control. However, as the final data product (e.g., predictions, views, aggregates, or transformed data sets) get pushed to the downstream tasks, they are often treated as ``source data'' and downstream pipelines fail to  consume the associated provenance information.
\end{itemize}

Understandably, these are hard problems to tackle and part of the challenge is not even technical and it involves standardizing data provenance representation across business units and different software stacks. However, this might suggest new research directions; for example, we might prefer developing simpler and less expressive provenance models that target interoperability and ease of propagation over representation power of the underlying computations. Another example is that propagating standard meta-data that ties data sources to central data governance and catalogs can be part of the integrity constraints and sanity checks. We suggest also extending meta-data representation of data sources to include {\em repair actions} that reference a controlled vocabulary or a {\em repairing ontology} tapping into the large body of work in work flow and business processes management.






%\input{5_otherquestions}
%\section{Conclusion}
\label{sec:conc}

To conclude, we suggest opening a new chapter of data quality and data cleaning that understands the entire data processing pipeline, in particular tracing it to the very beginning -- the genesis of the raw data. We have pointed out the challenges, with a focus on a new view of data provenance. 

Having discussed the \emph{how} (symptom), the \emph{why} (cause), and the \emph{where} (via provenance), other questions about errors remain. We have only glossed over the question \emph{what} is erroneous: an individual value, a row, a column, a table, or a process? Our general discussion allows these questions for data model beyond the relational, including tree or graph data, or even images, sound and video. When regarding data as it is created over time, we can ask \emph{when} the data error was introduced, and use data versions to understand the nature of the error~\cite{bleifuss2018exploringchange}. The final question of \emph{who} to blame, we leave to the management sciences.

%Definition of location of error (optional): “Where…?”
%{Reasoning about Where: The }
%WHERE (where in the process/pipeline)

%Ingestion, transformations, predictions, etc 

%Provenance is more relevant (tracking): DBRx, Data Xray, find errors at derivatives 

%relating prediction errors to faulty training data

\section{Conclusion}
\label{sec:conc}

To conclude, we suggest opening a new chapter of data quality and data cleaning that understands the entire data processing pipeline, in particular tracing it to the very beginning -- the genesis of the raw data. We have pointed out the challenges, with a focus on a new view of data provenance. 

Having discussed the \emph{how} (symptom), the \emph{why} (cause), and the \emph{where} (via provenance), other questions about errors remain. We have only glossed over the question \emph{what} is erroneous: an individual value, a row, a column, a table, or a process? Our general discussion allows these questions for data model beyond the relational, including tree or graph data, or even images, sound and video. When regarding data as it is created over time, we can ask \emph{when} the data error was introduced, and use data versions to understand the nature of the error~\cite{bleifuss2018exploringchange}. The final question of \emph{who} to blame, we leave to the management sciences.

%Definition of location of error (optional): “Where…?”
%{Reasoning about Where: The }
%WHERE (where in the process/pipeline)

%Ingestion, transformations, predictions, etc 

%Provenance is more relevant (tracking): DBRx, Data Xray, find errors at derivatives 

%relating prediction errors to faulty training data



\bibliographystyle{plain}
\providecommand{\noopsort}[1]{}
\begin{thebibliography}{10}

\bibitem{Abedjan2018}
Ziawasch Abedjan, Lukasz Golab, Felix Naumann, and Thorsten Papenbrock.
\newblock {\em Data Profiling}.
\newblock Synthesis Lectures on Data Management. Morgan {\&} Claypool
  Publishers, 2018.

\bibitem{Aggarwal2013}
Charu~C. Aggarwal.
\newblock {\em Outlier Analysis}.
\newblock Springer, 2013.

\bibitem{bleifuss2018exploringchange}
Tobias Bleifu\ss, Leon Bornemann, Theodore Johnson, Dmitri~V. Kalashnikov,
  Felix Naumann, and Divesh Srivastava.
\newblock Exploring change - a new dimension of data analytics.
\newblock {\em PVLDB}, 12(2):85--98, 2018.

\bibitem{DBLP:journals/dke/ButtF21}
Anila~Sahar Butt and Peter Fitch.
\newblock A provenance model for control-flow driven scientific workflows.
\newblock {\em Data Knowl. Eng.}, 131-132:101877, 2021.

\bibitem{DBLP:conf/sigmod/ChalamallaIOP14}
Anup Chalamalla, Ihab~F Ilyas, Mourad Ouzzani, and Paolo Papotti.
\newblock Descriptive and prescriptive data cleaning.
\newblock In {\em Proceedings of the International Conference on Management of
  Data (SIGMOD)}, pages 445--456, 2014.

\bibitem{DBLP:journals/vldb/DeutchMT15}
Daniel Deutch, Yuval Moskovitch, and Val Tannen.
\newblock Provenance-based analysis of data-centric processes.
\newblock {\em VLDB Journal}, 24(4):583--607, 2015.

\bibitem{DBLP:journals/ftdb/GlavicMR21}
Boris Glavic, Alexandra Meliou, and Sudeepa Roy.
\newblock Trends in explanations: Understanding and debugging data-driven
  systems.
\newblock {\em Found. Trends Databases}, 11(3):226--318, 2021.

\bibitem{uai_heidari}
Alireza Heidari, Ihab~F. Ilyas, and Theodoros Rekatsinas.
\newblock Approximate inference in structured instances with noisy categorical
  observations.
\newblock In {\em Proceedings of the Conference on Uncertainty in Artificial
  Intelligence (UAI)}, volume 115 of {\em Proceedings of Machine Learning
  Research}, pages 412--421. {AUAI} Press, 2019.

\bibitem{holodetect}
Alireza Heidari, Joshua McGrath, Ihab~F. Ilyas, and Theodoros Rekatsinas.
\newblock Holodetect: Few-shot learning for error detection.
\newblock In {\em Proceedings of the International Conference on Management of
  Data (SIGMOD)}, page 829–846, 2019.

\bibitem{DBLP:journals/corr/abs-2202-00622}
Andrew Ilyas, Sung~Min Park, Logan Engstrom, Guillaume Leclerc, and Aleksander
  Madry.
\newblock Datamodels: Predicting predictions from training data.
\newblock {\em CoRR}, abs/2202.00622, 2022.

\bibitem{Ilyas16DE}
Ihab~F. Ilyas.
\newblock Effective data cleaning with continuous evaluation.
\newblock {\em {IEEE} Data Eng. Bull.}, 39(2):38--46, 2016.

\bibitem{IlyasC15}
Ihab~F. Ilyas and Xu~Chu.
\newblock Trends in cleaning relational data: Consistency and deduplication.
\newblock {\em Found. Trends Databases}, 5(4):281--393, 2015.

\bibitem{DBLP:books/acm/IlyasC19}
Ihab~F. Ilyas and Xu~Chu.
\newblock {\em Data Cleaning}.
\newblock {ACM}, 2019.

\bibitem{pmlr-koh-Liang}
Pang~Wei Koh and Percy Liang.
\newblock Understanding black-box predictions via influence functions.
\newblock In Doina Precup and Yee~Whye Teh, editors, {\em {Proceedings of the
  International Conference on Machine Learning}}, volume~70, pages 1885--1894.
  PMLR, 2017.

\bibitem{raha}
Mohammad Mahdavi, Ziawasch Abedjan, Raul Castro~Fernandez, Samuel Madden,
  Mourad Ouzzani, Michael Stonebraker, and Nan Tang.
\newblock Raha: A configuration-free error detection system.
\newblock In {\em Proceedings of the International Conference on Management of
  Data (SIGMOD)}, page 865–882, 2019.

\bibitem{Naumann10}
Felix Naumann and Melanie Herschel.
\newblock {\em An Introduction to Duplicate Detection}.
\newblock Morgan \& Claypool Publishers, 2010.

\bibitem{Papadakis21}
George Papadakis, Ekaterini Ioannou, Emanouil Thanos, and Themis Palpanas.
\newblock {\em The Four Generations of Entity Resolution}.
\newblock Synthesis Lectures on Data Management. Morgan {\&} Claypool
  Publishers, 2021.

\bibitem{DBLP:journals/corr/abs-2002-08484}
Garima Pruthi, Frederick Liu, Mukund Sundararajan, and Satyen Kale.
\newblock Estimating training data influence by tracking gradient descent.
\newblock {\em CoRR}, abs/2002.08484, 2020.

\bibitem{FAHES_18}
Abdulhakim~A. Qahtan, Ahmed Elmagarmid, Raul Castro~Fernandez, Mourad Ouzzani,
  and Nan Tang.
\newblock {FAHES}: A robust disguised missing values detector.
\newblock In {\em Proceedings of the International Conference on Knowledge
  Discovery and Data Mining (SIGKDD)}, page 2100–2109, 2018.

\bibitem{holoclean}
Theodoros Rekatsinas, Xu~Chu, Ihab~F. Ilyas, and Christopher R\'{e}.
\newblock Holoclean: Holistic data repairs with probabilistic inference.
\newblock {\em PVLDB}, 10(11):1190–1201, August 2017.

\bibitem{puds}
Christopher~De Sa, Ihab~F. Ilyas, Benny Kimelfeld, Christopher R{\'e}, and
  Theodoros Rekatsinas.
\newblock A formal framework for probabilistic unclean databases.
\newblock In {\em Proceedings of the International Conference on Database
  Theory (ICDT)}, pages 6:1--6:18, 2019.

\bibitem{yeye-unidetect}
Pei Wang and Yeye He.
\newblock Uni-detect: {A} unified approach to automated error detection in
  tables.
\newblock In {\em Proceedings of the International Conference on Management of
  Data (SIGMOD)}, pages 811--828. {ACM}, 2019.

\bibitem{Strong96}
Richard~Y.\ Wang and Diane~M.\ Strong.
\newblock Beyond accuracy: What data quality means to data consumers.
\newblock {\em Management of Information Systems}, 12(4):5--34, 1996.

\bibitem{DBLP:conf/sigmod/WangM017}
Xiaolan Wang, Alexandra Meliou, and Eugene Wu.
\newblock Qfix: Diagnosing errors through query histories.
\newblock In {\em Proceedings of the International Conference on Management of
  Data (SIGMOD)}, pages 1369--1384. {ACM}, 2017.

\bibitem{DBLP:journals/pvldb/0002M13}
Eugene Wu and Samuel Madden.
\newblock Scorpion: Explaining away outliers in aggregate queries.
\newblock {\em PVLDB}, 6(8):553--564, 2013.

\end{thebibliography}
%\bibliography{ihab-letter}

%\begin{thebibliography}{10}
%\itemsep=1pt
%\begin{small}
%
%\bibitem{Ilyas2015}
%Ihab F. Ilyas and Xu Chu
%Trends in Cleaning Relational Data
%% publication
%\end{small}
%\end{thebibliography}

\end{document}

\end{opinion}
\end{opinionsection}

\begin{articlesection}{Directions Towards GDPR-Compliant Data Systems and Applications}
%
%  Contributed articles section.  Use the articlesection environment.
%  Each article is contained in an article environment, where the two required
%  options to \begin{article} are the title and author of the article
%

%\makeatletter
%\renewcommand{\AB@affillist}{}
%\renewcommand{\AB@authlist}{}
%\setcounter{authors}{0}
%\makeatother

\begin{article}
{Disposal by Design}
{Susan B. Davidson, Shay Gershtein, Tova Milo, Slava Novgorodov}
\graphicspath{{submissions/disposal-by-design/}}
% link to instruction: https://tc.computer.org/tcde/tcde-bulletin-author-instructions/
% \documentclass[11pt,dvipdfm]{article}
\documentclass[11pt]{article}
\usepackage{tabularx}
\usepackage{ragged2e}  % for '\RaggedRight' macro (allows hyphenation)
\usepackage{booktabs}  % for \toprule, \midrule, and \bottomrule macros
\usepackage{textcomp}
\usepackage{amsfonts,amsmath}
\usepackage{deauthor,times}
\usepackage{graphicx} % 
\usepackage{hyperref}
\usepackage{comment}
\graphicspath{{asudeh/}}
\usepackage{soul}
\usepackage{subcaption}
\usepackage{ulem}
\usepackage{wrapfig}
\usepackage{color}
\usepackage{xspace}
\newtheorem{problem}{Problem}

%\DeclareMathOperator*{\argmax}{arg\,max}

%remove the following commands/package b4 submission
\newcommand{\hide}[1]{}
\newcommand{\eat}[1]{}
\newcommand{\resolved}[1]{\hide{#1}}
\newcommand{\abol}[1]{\textcolor{red}{Abol: #1}}
\newcommand{\mahdi}[1]{\textcolor{red}{Mahdi: #1}}
\newcommand{\nima}[1]{\textcolor{red}{Nima: #1}}

\newcommand{\dee}{\mathcal{D}}
\newcommand{\Gee}{\mathcal{G}}
\newcommand{\gee}{\mathbf{g}}
\newcommand{\ee}{\mathbf{e}}
\newcommand{\es}{\mathcal{S}}
\newcommand{\el}{\mathcal{L}}
\newcommand{\xx}{\mathcal{x}}
\newcommand{\dist}{\xi}
\newcommand{\alg}{\mathsf{A}}
\newcommand{\qu}{\mathbf{q}}
\newcommand{\ex}{\mathbf{x}}
\newcommand{\ti}{\mathbf{t}}
\newcommand{\sdt}{\mathsf{SDT}}
\newcommand{\wdt}{\mathsf{WDT}}
\newcommand{\Qu}{\mathbf{Q}}
\newcommand{\pe}{\mathbb{P}}
\newcommand{\megam}{\mathcal{M}}
\newcommand{\eps}{\varepsilon}
\newcommand{\enet}{{$\varepsilon$-{\bf net}}\xspace}
\newcommand{\net}{{\tt net}\xspace}
\newcommand{\vcd}{VC-dimension\xspace}
\newcommand{\at}[1]{{\tt \small #1}\xspace}
\newcommand{\pr}{Pr}

\newcommand{\sharpP}{\mbox{\#P}}
\newcommand{\NP}{\mathsf{NP}}
\newcommand{\LP}{\mathsf{LP}}
\newcommand{\IP}{\mathsf{IP}}
\newcommand{\ru}{{\sc {RU}}\xspace}
\newcommand{\sru}{{\sc {strongRU}}\xspace}
\newcommand{\wru}{{\sc {weakRU}}\xspace}

\newcommand{\fmsystem}{{\sc Chameleon}\xspace}
\newcommand{\fm}{$\mathcal{F}$\xspace}

\newtheorem{experiment}{Experiment}

\begin{document}

\title{Coverage-based Data-centric Approaches for \\Responsible and Trustworthy AI\thanks{This research was supported by the National Science Foundation under grant No. 2107290.}}

\author{
\begin{tabular}[t]{c@{\extracolsep{2.4em}}c@{\extracolsep{2.4em}}c@{\extracolsep{2.3em}}c} 
Nima Shahbazi & Mahdi Erfanian & Abolfazl Asudeh \\ 
University of Illinois Chicago & University of Illinois Chicago & University of Illinois Chicago\\
 nshahb3@uic.edu & merfan2@uic.edu & asudeh@uic.edu
\end{tabular}
}

\maketitle


\begin{abstract}
The grand goal of data-driven decision systems is to help make decisions easier, more accurate, at a higher scale, and also just. However, data-driven algorithms are only as good as the data they work with. Yet, data sets, especially those with social data, often do not represent minorities. The paucity of training data is a perpetual problem for AI, and the outcome of ML models for cases not represented in their training data is often not reliable. 
Hence, without properly addressing the lack of representation issues in data, we cannot expect AI-based societal solutions to have responsible and trustworthy outcomes. 

This paper focuses on data coverage as a data-centric approach for identifying and resolving misrepresentation of minorities in data.
To achieve this goal, we propose novel algorithms that (a) {\it identify} and {\it resolve} insufficient data coverage across data with different modalities and (b) use lack of representation information to generate data-centric {\it reliability warnings}.
 \end{abstract}
 
 %%%%%%%%%%%%%%%%%%%%%%%%%%%%%%%% INTRO  %%%%%%%%%%%%%%%%%%%%%%%%%%%%%%%%
\section{Introduction}\label{sec:intro} % Abstract+Intro: up to 2.5 pages 
Data-driven decision-making has shaped every corner of human life, spanning from autonomous vehicles to healthcare and even predictive policing and criminal justice. A pivotal concern, especially in applications that affect individuals, revolves around the reliability of the decisions rendered by the system.
It is easy to see that the accuracy of a data-driven decision depends, first and foremost, on the data used to make it. Essentially, the system learns the phenomena that data represent. While we may desire that the data should represent the underlying data distribution from which the production data is drawn, this alone may be insufficient, as it merely enables the model to perform well for the average case.
As a result, a model with a high accuracy could fail for specific regions in the data with insufficient representation. These regions may matter because they frequently represent some minority population in society. They could also represent cases that may not happen very often but have a relevant impact on the correctness of a critical decision.
In short, if the data fails to sufficiently represent a specific population, the outcome of the decision system for that population may not be trustworthy.

The phenomenon known as \textit{Representation Bias} can arise from how the data was originally collected, or it could be the result of biases introduced post-collection—whether historically, cognitively, or statistically.

Representation bias is essentially inevitable without a systematic approach to data collection. 
For example, in the context of survey data collection, vital steps involve identifying all populations within the underlying distribution based on desired demographic information and ensuring comprehensive coverage with sufficient samples from each group. 
Even then, only an (uncontrolled) subset of the invitees will opt-in to respond to the survey.
Another challenge lies in the fact that data scientists often lack control over the data collection process, leading to the reliance on ``found data'' in the majority of data-driven systems. Therefore, with no guarantee on the aforementioned steps in the data collection process, the found data is most likely a biased sample.
Acknowledging the potential harms of representation bias, the notion of \textit{Data Coverage}~\cite{asudeh2019assessing,shahbazi2023representation} has been proposed to ensure the adequate representation of minority groups in data sets employed for decision-making and developing sophisticated data science tools. 

Addressing representation issues in data poses various challenges depending on the modality of the data. In this paper, we focus on identifying and resolving lack of coverage issues in data with different modalities.
We start by proposing a variety of techniques (spanning from geometric and combinatorial optimization to crowd-souring) aimed at efficiently detecting insufficient coverage on structured data sets with non-ordinal categorical and continuous attributes, as well as image data sets. Next, we propose a range of approaches grounded in data integration and generative data augmentation to address the lack of coverage by enriching the data sets with more data. However, with limited control over the data collection processes, it could be difficult and expensive to resolve all misrepresentations. 
Since adding more data is not always possible, we proceed to introduce data-centric preventive solutions that warn the user about the reliability of their predictions regarding representation bias issues. These warnings assist users in determining whether they trust the outcomes of the models or exercise caution. 

 %%%%%%%%%%%%%%%%%%%%%%%%%%%%%%%% IDENTIFICATION  %%%%%%%%%%%%%%%%%%%%%%%%%%%%%%%%
\section{Detecting Insufficient Representation of Minorities}\label{sec:identification} %up to 3.5 pages
Representation bias happens when the development (training data) population under-represents 
and subsequently fails to generalize well 
for some parts of the target population, due to historical bias, sampling bias, etc.
The notion of {\it data coverage} has been studied across different settings in \cite{shahbazi2023representation} as a metric to measure representation bias. At a high level, coverage is referred to as having enough similar entries for each object in a data set. 
For a better understanding, let us go over the definition of the generalized notion of coverage:

\begin{definition}[Data Coverage]\label{def:coverage}
Consider a data set $\dee$ with $n$ tuples, each consisting of $d$ attributes of interest $\mathbf{x}=\{x_1, x_2, \cdots,x_d\}$, such as {\tt gender}, {\tt race}, {\tt salary}, {\tt age}, etc, that are used for coverage identification.
The data set also contains target attributes $\mathbf{y} = \{ y_1,\cdots,y_{d'}\}$ that may or may not be considered for the coverage problem.
A query point $q$ is not covered by the data set $\dee$, if there are not ``enough'' data points in $\dee$ that are representative of $q$.
To generalize the notion of coverage, let us define $\gee(q)$ as the universe of tuples that would represent $q$ and let $\gee_\dee(q) = \gee(q)\cap \dee$. In other words, $\gee_\dee(q)$ are the set of tuples in $\dee$ that represent $q$.
Using this notation, we define the coverage of $q$ as the size of $\gee_\dee(q)$. That is,
$cov(q,\dee) = | \gee_\dee(q)|$.
Given a value $\tau$, $q$ is covered if $cov(q,\dee)>\tau$.
Similarly, a group $\gee$ is not covered if $\gee\cap \dee<\tau$.
The {\it uncovered region} in a data set is the collection of groups that are not covered by it.
\end{definition}

\subsection{Structured Data}
In this section, we focus on identifying representation bias in structured data.
Depending on the type of the attributes of interest, we categorize the techniques into two classes based on whether they target the problem for non-ordinal {\it categorical} (e.g. {\tt race}, {\tt gender}) or ordinal {\it continuous} (e.g. {\tt age}) attributes. The attributes of interest considered for representation bias often include sensitive attributes such as {\tt race} and {\tt gender} but are not necessarily limited to them.

\subsubsection{Categorical Attributes}

For cases where attributes of interest are non-ordinal categorical,
the cartesian product of values on a subset of attributes $\mathbf{x}'\subseteq \mathbf{x}$, form a set of (sub-)groups.
For example, $\{$ {\tt white male}, {\tt white female}, {\tt black male} $,\cdots\}$ are the subgroups defined on the attributes {\tt (race,gender)}.
We refer to the number of attributes used to specify a subgroup as the {\it level} of that subgroup.
For example, the level of the subgroup {\tt white male} is 2, while the level of the subgroup {\tt male} is 1.
We use $\ell(\gee)$, to refer to the level of a subgroup $\gee$.
Similarly, we say a subgroup $\gee'$ is a subset of $\gee$, if the groups specifying $\gee'$ are a superset of the ones for $\gee$. For example {\tt (married white male)} a subset of the more general group {\tt (white male)}. That is, the set of individuals in group {\tt (married white male)} are a subset of {\tt (white male)}.
Moreover, we say a subgroup $\gee$ is a {\it parent} of the subgroup $\gee'$, if $\gee'\subset \gee$ and $\ell(\gee)=\ell(\gee')+1$. For example, the subgroup {\tt (white male)} is a parent of the subgroup {\tt (married white male)}.
We use \textit{patterns} to refer to uncovered subgroups.
A pattern $P$ is a string of $d$ values, where $P[i]$ is either a value from the domain of $x_i$, or it is ``unspecified'', specified with $X$. 
For example, consider a data set with three binary attributes of interest $\mathbf{x}=\{x_1, x_2, x_3\}$. The pattern $P=X01$ specifies all the tuples for which $x_2=0$ and $x_3=1$ ($x_1$ can have any value).
The set of patterns that identify most general uncovered subgroups are called {\it Maximal Uncovered Patterns} (MUPs).

No polynomial time algorithm can guarantee the enumeration of the entire MUPs, however, several algorithms inspired by set enumeration and the Apriori algorithm for association rule mining are proposed to efficiently address this problem~\cite{asudeh2019assessing}.
In this regard, we introduce \textit{Pattern Graph} data structure that exploits the relationship between patterns to do less work than computing all uncovered patterns by removing the non-maximal ones. 
The parent-child relationship between the patterns is represented in a graph that can be used to find better algorithms. 
\textit{Pattern-Breaker} starts from the top of the graph where the general patterns are and moves down by breaking each pattern into more specific ones. If a pattern is uncovered, then all of its descendants are also uncovered and they can not be an MUP, even if they have a parent that is covered. Therefore, this subgraph of the pattern graph can be pruned. 
The issue with \textit{Pattern-Breaker} is that it explores the covered regions of the pattern graph and for the cases where there are a few uncovered patterns, it has to explore a large portion of the exponential-size graph. 
To tackle this, \textit{Pattern-Combiner} algorithm is proposed that performs a bottom-up traversal of the pattern graph. It uses an observation that the coverage of a node at the level of the pattern graph can be computed as the sum of the coverage values of its children. 
The problem with \textit{Pattern-Combiner} is that it traverses over the uncovered nodes first and therefore, it will not perform well for the cases in which most of the nodes in the graph are uncovered. 
In fact, for the cases where most of the MUPs are placed in the middle of the graph, both \textit{Pattern-Breaker} and \textit{Pattern-Combiner} will not be as efficient as they should traverse half of the graph. Therefore, we propose \textit{Deep-Diver}, a search algorithm based on Depth-First-Search that quickly finds the MUPs, and uses them to limit the search space by pruning the nodes both dominating and dominated by the discovered MUPs.

\begin{figure*}[!tb]
    \begin{minipage}[t]{0.31\linewidth}
        \centering
        \includegraphics[width=\textwidth]{submissions/submission1/shahbazi/covcube1.jpg}
        \caption{\small Categorical attributes: the uncovered region of a toy example, as the collection of three MUPs.}
        \label{fig:covcube1}
    \end{minipage}
    \hfill
    \begin{minipage}[t]{0.31\linewidth}
        \centering
        \includegraphics[width=\textwidth]{submissions/submission1/shahbazi/cvrg_2_1.jpg}
        \caption{\small Continuous attributes, 2D: identifying the covered region in the gray Voronoi cell.}
        \label{fig:cvrg_2_1}
    \end{minipage}
    \hfill
    \begin{minipage}[t]{0.31\linewidth}
        \centering
        \includegraphics[width=\textwidth]{submissions/submission1/shahbazi/cvrg_2_2.jpg}
        \caption{ \small Continuous attributes, 2D: Uncovered region marked in red.}
        \label{fig:cvrg_2_2}
    \end{minipage}
\vspace{-5mm}
\end{figure*}

\subsubsection{Continuous Attributes}
Data in the real world often consists of a combination of continuous and discrete values. While simple solutions like binning {\tt age} into {\tt young} and {\tt old} can transform the continuous space into discrete. However, they may lead to coarse groupings that are sensitive to the thresholds chosen. It may be inappropriate to treat a 35-yo as {\tt young} but a 36-yo as {\tt old}. 
Therefore, we extend the notion of coverage to continuous space. Particularly, given data set $\dee$ with $n$ tuples over $d$ attributes, and vicinity radius $\rho$ and coverage threshold $k$, we want to identify the uncovered region -- the universe of uncovered query points.
A query point in continuous data space is covered if there are enough (at least $k$) data points in its $\rho$-vicinity neighborhood. $\rho$-vicinity neighborhood is the circle centered at the query point with radius $\rho$.

Depending on the number of attributes in a data set, we propose two algorithms for identifying uncovered regions in data~\cite{asudeh2021coverage}. 
The first algorithm known as \textit{Uncovered-2D} studies coverage over two-dimensional data sets where $\mathbf{x}=\{x_1,x_2\}$. To find the number of circles that a query point falls into and consequently discover the uncovered region, \textit{Uncovered-2D} makes a connection to $k$-th order Voronoi diagrams.
Consider a data set $\mathcal{D}$ and its corresponding $k$-th order Voronoi diagram. For every tuple $t\in \mathcal{D}$, let $\circ_t$ be the $d$-dimensional sphere ($d$-sphere) with radius $\rho$ centered at $t$.
Consider a $k$-voronoi cell $\mathcal{V}(S)$ in the $k$-th order Voronoi diagram $V_k(\mathcal{D})$.
Any point $q$ inside the intersections of the $d$-spheres of tuples in $S$, i.e. $q\in \underset{\forall t\in S}{\cap ~\circ_t}$, is covered, while all other points in the region are uncovered.
 The algorithm starts by constructing the $k$-th order Voronoi diagram of the data set and then for each Voronoi cell $\mathcal{V}(S)$ in the diagram, it computes the intersection of the circles of the tuples in $S$ and marks the portion of $\mathcal{V}(S)$ that falls outside it as uncovered.
After identifying the uncovered region, a 2D map of $\{x_1,x_2\}$ value combinations is used to report the region to the user.
The algorithm for the 2D case can be extended to the general case by relaxing the assumption on the number of attributes to discover the exact uncovered region, however, due to the curse of dimensionality, the search size space explodes as the number of dimensions increases and as a result, the algorithm will not be practical. Therefore, we propose a randomized approximation algorithm based on the geometric notion of \enet. 
Let $\mathcal{X}$ be a set and $\mathcal{R}$ be a set of subsets of $\mathcal{X}$. A set $\mathcal{N}\subset \mathcal{X}$ is an \enet for $\mathcal{X}$ if for any range $r\in\mathcal{R}$, if  $|r\cap \chi|>\eps|\chi|$, then $r$ contains at least one point of $N$.
The idea, at a high level, is to draw enough random samples from the space of potential query points to form an \enet. 
We then label the sampled query points as $\{-1,+1\}$ depending on whether those are covered or not, and learn the uncovered regions using the samples.

\subsection{Image Data}
Many known incidents of machine failures due to the lack of representation were on image data.
We consider an image data set with a fixed number of low-cardinality sensitive attributes such as {\tt\small race} and {\tt\small gender}. 
It is common that image data sets {\it lack explicit values} for sensitive attributes, which are crucial for coverage identification. An image data set is often a collection of images from different domains with little to no information about their domain and which groups they belong to. As a result, even studying coverage over low-cardinality and categorical attributes of interests is challenging in these cases.

\begin{wrapfigure}{R}{0.42\textwidth}
\centering
\vspace{-3mm}
\scriptsize
\begin{tabular}{|@{}c|@{}c@{}|@{}c@{}|@{}c@{}|} 
 \hline
{\bf data set} & {\bf classifier} & {\bf accuracy} & {\bf precision} \\ 
 &  &  & {\bf on female} \\ \hline
UTKFace:~& DeepFace (opencv) & 93.56 & {52.02}\\\cline{2-4}
({\tt females}=200,& DeepFace (retinaface) & 94.16 & {56.15}\\\cline{2-4}
{\tt males}=2800) & BaseCNN & 97.6 & 74.8\\
\hline
UTKFace:~& DeepFace (opencv) & 96.53 & {\bf 8.0}\\\cline{2-4}
({\tt females}=20,& DeepFace (retinaface) & 96.43 & {\bf 10.09}\\\cline{2-4}
{\tt males}=2980)& BaseCNN & 97.6 & {\bf 21.59}\\
\hline
\end{tabular}
\vspace{-3mm}
\caption{\small ML models' low performance for females in the presence of representation bias.~\cite{mousavi2024data}}\label{fig:mlfails}
\vspace{-3mm}
\end{wrapfigure}

In Figure~\ref{fig:mlfails}, we show that due to the issues such {\it machine bias} and {\it lack of distribution generalizability},
solely relying on state-of-the-art machine learning (ML) techniques fail to effectively identify lack of coverage in image data sets. Therefore, we propose an approach based on combining crowdsouring with ML~\cite{mousavi2024data}. 
Crowdsourcing is particularly promising for image data, for tasks such as image labeling, which, while challenging for the machine, are "easy" for human beings to conduct with minimal error. 

A key observation that enables a cost-effective crowdsourcing approach is that, while studying coverage, we would only like to find out if there are {\it enough tuples from each subgroup}.
Suppose a subgroup is covered if there are $\tau=100$ instances of it in the data set. Assume the (majority) group $\gee_1$ contains $n_1 \gg 100$ objects in the data set. 
To verify that $\gee_1$ is covered, it is enough for the crowd to discover 100 of those objects, not the entire $n_1$. 
Following this, $O(\tau)$ provides a lower bound on the number of crowd tasks required to verify a given group is covered. 
Still, this lower bound only holds for the groups that are covered, i.e., there is at least $\tau$ of those in the data set.
Surprisingly, verifying that a minority group is indeed uncovered is cumbersome, unlike the majority group.
This is because even though discovering $\tau$ objects from a group is enough for verifying that it is covered, one cannot {\it verify} a group is uncovered until there is a chance that the data set might still have enough objects from that group. Thus, assuming a non-zero probability for each unlabeled object to belong to each group, {one might need to ask the crowd to label the entire data set before they can confirm that a specific group is uncovered}.

Our idea for addressing this challenge is to
design {\it a divide and conquer algorithm} that, instead of {point queries}, uses {\it set queries} to iteratively eliminate subsets of data that {does not include any object from the given group}.
At a high level, our idea is to ask a set query from the crowd, inquiring whether the selected set contains at least one object from the given group $\gee$.
The user may provide two responses (yes/no). 
Interestingly, {in either case}, the user response provides valuable information that helps efficiently identify the coverage.
If the answer is ``No'', the set does not include any object from the given group $\gee$. As a result, the algorithm can safely prune the set, asking no further questions about it. In particular, for a group that is not covered, one can expect to see no answers on large set queries helping to prune a significant portion of the data set quickly.
On the other hand, if the answer is ``yes'', the set contains {at least} one object from the group $\gee$. As a result, the algorithm cannot prune the subset since it can have any number (larger than one) of the objects in $\gee$.
At first glance, the queries with yes answers do not provide helpful information as the algorithm cannot prune the subset (hence it needs to divide it into smaller subsets).
However, a key observation is that {the algorithm will only observe a limited number of yes answers} before it stops.
The reason is that the number of set queries with yes answers provides a {lower-bound} on the number of objects from $\gee$ in the data set. As a result, the algorithm can stop as soon as the lower bound reaches $\tau$, knowing that $\gee$ is covered.
The D\&C approach verifies the data coverage for a given group, while our goal is to identify the uncovered regions for a given set of sensitive attributes. The next question is how to utilize this algorithm for efficient coverage identification on different scenarios of sensitive attributes, forming intersectional or non-intersectional groups.
In particular, how can we find maximal uncovered patterns?
Our idea is to apply sampling and aggregate estimation techniques to find the groups that even if merged are likely to still be uncovered. This will help reduce the coverage identification cost by running the D\&C approach for the merged groups once.
 %%%%%%%%%%%%%%%%%%%%%%%%%%%%%%%% RESOLUTION  %%%%%%%%%%%%%%%%%%%%%%%%%%%%%%%%
\section{Resolving Insufficient Representation}\label{sec:resolution}

Data integration~\cite{nargesian2021tailoring,nargesian2022responsible} and data augmentation~\cite{sharma2020data,DBLP:journals/jair/ChawlaBHK02,iosifidis2018dealing,celis2020data} are considered as the primary solutions for reducing data coverage issues in a data set. 
Data integration is promising when external sources of data are available. On the other hand, recent advancements in generative AI and foundation models have enabled efficient and effective augmentation of data sets with synthetic data. 
Therefore, in the following, we review two approaches, one from each category, in the context of lack of coverage resolution.

\subsection{Data Integration}\label{sec:resolution:integration}

Data integration is to consolidate data from different sources into a single, unified view. 
Although it is an effective solution to acquire additional data from different distributions,
there are sampling policy and cost-efficiency concerns that need to be examined.  
Therefore, {\it Data Distribution Tailoring ({\sc DT})} introduces data integration techniques for resolving insufficient representation of subgroups in a data set in the most cost-effective manner~\cite{nargesian2021tailoring}.
A query to {\sc DT} 
consists of a target schema, and a set of group distribution requirements in the form of the minimum counts (e.g., ``{\tt\small 1,000 breast cancer monitoring data in Chicago with at least 30\% label=positive, and at least 20\% black patients}''). 
Collecting a fresh sample from a data view is costly (monetary, human resources, and/or computation cost)~\cite{asudeh2022towards}.
Therefore, {\sc DT} focuses on satisfying the count requirements with minimum cost. 
Given an input query and a lake of available data sources, the first step is to discover a collection of candidate data views that satisfy the target schema.
Each data view $v_i$ is a projection-join $v_i = \Pi\big(D_{i1}\bowtie\cdots\bowtie D_{ik_i} \big)$, where $D_{ij}$ is a data set in a given data lake.
Let us suppose the data views are already discovered.
At a high level, {\sc DT} follows an iterative approach that at each iteration a data view is selected to be queried.
Each query to a data view has a fixed cost and returns a sample that may or may not satisfy the query constraints.
The samples that are either not fresh, or do not satisfy the query are discarded.
Hence, the essential question towards a cost-effective data integration is {\it what data view to query next}.
Depending on the available information about the data sources, various techniques may be employed. 

For the cases when the group distributions are known, the process of collecting the target data set is a sequence of iterative steps, where at every step, the algorithm chooses a data view, queries it, and if the obtained tuple contributes to one of the groups for which the count requirement is not yet fulfilled, it is kept, otherwise discarded. To do so, a {Dynamic Programming (DP)} algorithm is proposed. An optimal source at each iteration minimizes the sum of its sampling cost plus the expected cost of collecting the remaining required groups, based on its sampling outcome.
The DP algorithm, however, has a pseudo-polynomial time complexity. Hence, it quickly becomes intractable for cases where the minimum count requirements for the groups are not small. 
For cases where the (sensitive) attribute of interest is binary, such as (biological) {\tt sex}={\tt \{male, female\}}, and the cost to query data is similar from all sources, it turns out that the optimal strategy is to query the data source with {maximum probability of obtaining a sample from the minority group}.
Expanding the binary-attributes algorithm for non-binary cases, the problem can be modeled as an extension of the ``{\it coupon collector's}'' problem~\cite{motwani1995randomized}, where the goal is to collect $m_i$ instances from each coupon (group) $\gee_i$.
At each iteration, the coupon collector's algorithm identifies a data view as most promising and queries it. In simple terms, a data view with a smaller query cost and a higher chance of obtaining minority groups is more promising.


For the cases where the group distributions are unknown, we model DT as a {\it multi-armed bandit} problem, where every data view is modeled as an arm. 
Every arm has an unknown distribution of different groups while pulling an arm (i.e., querying the corresponding data view) has a cost.
During various iterations, the algorithms pull the arms in an order that its expected total {\it reward} is maximized.
Arguing that the reward of obtaining a tuple from a group is proportional to how rare this group is across different data views, 
we design the reward function based on the expected cost one needs to pay in order to collect a tuple from a specific group.  
As the bandit strategy, we adopt {\it Upper Confidence Bound (UCB)} to balance exploration and exploitation. At every iteration, for every arm, UCB computes confidence intervals for the expected reward and selects the arm with the maximum upper bound of reward to be explored next.

\subsection{Data Augmentation using Foundation Models}

While data integration provides a promising approach for resolving coverage issues in a data set, its effectiveness is limited to the availability of external data sources that are rich enough to find sufficient fresh samples from minority groups. This, however, is not always possible, especially since the minority samples are rare and not easy to obtain.
Fortunately, recent advancements in Generative AI and Foundation Models have enabled synthesizing samples that are otherwise challenging to obtain from the real world.

Therefore, as an alternative approach to data integration, we turn our attention to the Foundation Models and Generative AI for resolving the lack of coverage. 
Particularly, models such as {\sc DALL.E}\footnote{\url{https://openai.com/dall-e-2}} have emerged as powerful tools for generating multi-modal data such as image, audio, and video.
 
We formalize the foundation model \fm as a black-box function with the following inputs, that once queried synthesize an output tuple.
\begin{itemize}
    \item {\bf Prompt}: A natural language description providing instructions on the details of the tuple to be generated. For instance, a prompt for image generation might be ``A realistic photo of a white cat running in a backyard.''
    \item {\bf Guide}: In cases where only a prompt is provided, the foundation model uses its imagination to generate the requested tuple. For the previous example, the prompt of a cat image, the breed, size, background, and other details are generated based on the model's imagination. Alternatively, a guide can be provided to influence the generation process. The guide is formalized as a pair $(t,m)$ where $t$ is a tuple and $m$ is a mask specifying which parts of the guide tuple should be changed. Using the cat example, $t$ can be a cat image and $m$ can specify the foreground to be regenerated.
\end{itemize}

There are multiple challenges towards effective data set augmentations using foundation models. 
First, we have to determine the minimal set of synthetic tuples that once added to the original data set, under-representation issues are resolved.
Second, the generated images should follow the underlying distribution represented in the input data set. Third, the generated tuples should have high quality and look realistic to a human evaluator. Last but not least, given the (often monetary) cost associated with the queries to the foundation model, we should ensure the cost-effectiveness of the data set repair process.

\begin{wrapfigure}{L}{0.45\textwidth}
\centering
\vspace{-3mm}
\scriptsize
    \includegraphics[width=.45\textwidth]{submissions/submission1/shahbazi/enhanced_pipeline.png}
\vspace{-3mm}
\caption{\small Architecture of \fmsystem for image data augmentation for coverage enhancement.}\label{fig:chameleon}
% \vspace{-3mm}
\end{wrapfigure}

\noindent Figure~\ref{fig:chameleon} shows the architecture of our system \fmsystem \cite{chameleon} for coverage enhancement using DALL-E image generator.
To address the first challenge, we define the combinations-selection problem, which minimizes the total number of synthetic tuples for resolving lack of coverage of minorities at the most general level. We show the problem is {\sc NP}-hard, and propose a greedy approximation algorithm for it.
To address the second and third challenges, \fmsystem follows a {\it rejection sampling} strategy.
It views each tuple in the data set $\dee$ as an iid sample from the underlying distribution $\xi$ it represents. It uses the vector representations (embeddings) space to describe the distribution. Then, given a newly generated tuple, it employs the one-class support vector machine (OCSVM) approach proposed by Scholkopf et al.~\cite{scholkopf1999support} to reject the tuple if it does not follow $\xi$.
Moreover, it models the quality evaluation as hypothesis testing and rejects the samples that have a higher chance of being labeled as ``unrealistic'' by a random human evaluator.
Finally, to minimize the number of queries to the foundation model, we provide a guide tuple (and a mask), in addition to the prompt, to the foundation model. We model the guide-selection problem as {\it contextual multi-armed bandit} and propose a solution based on the contextual UCB for it.

Before concluding this section, let us provide some experiment results to demonstrate the effectiveness of data augmentation with \fmsystem. We use FERET DB \cite{phillips1998feret} for this experiment, which comprises 1199 individual images and serves as a standardized facial image database for researchers to develop algorithms and report results. All images in FERET DB share the same dimensions, pose, and facial expression.
First, we identified the (level-1) uncovered ethnicity groups, using the threshold 80. We then used \fmsystem and resolved the lack of coverage issues.
To evaluate the effectiveness of the system, we trained a CNN model to predict the race of each image within this dataset. We then retrained the identical CNN on the repaired training data. Importantly, our test dataset for both experiments remains consistent and is derived from real images.
Table~\ref{tab:lackofcoverage} presents the improvements in precision, recall, and F1 score metrics for under-represented groups after repairing the dataset. The results indicate an enhancement in performance metrics for all under-represented groups following the repair process.

\begin{table}[t]
    \centering
    \caption{Illustrating the effect of lack of coverage repair using \fmsystem on \texttt{FERTDB}}
    \label{tab:lackofcoverage}
    \vspace{-3mm}
    \begin{tabular}{lcccccccc}
        \toprule
         & \multicolumn{4}{c}{\textbf{Classifier Performance on \texttt{FERTDB}}} & \multicolumn{4}{c}{\textbf{Classifier Performance on Repaired}} \\
        \cmidrule(lr){2-5} \cmidrule(lr){6-9}
        \textbf{Ethnicity Groups}& \#Images & Precision & Recall & F1-Score & \#Images & Precision & Recall & F1-Score \\
        \midrule
        Overall          & 756 & 0.81 & 0.75 & 0.78 & 987 & 0.70 & 0.75 & 0.72 \\ \hline
        Black            & 40  & 0.19 & 0.22 & 0.16 & 100 & 0.48 & 0.56 & 0.52 \\
        Hispanic         & 19  & 0.50 & 0.17 & 0.25 & 100 & 0.62 & 0.36 & 0.45 \\
        Middle Eastern   & 10  & 0.00 & 0.00 & 0.00 & 100 & 0.20 & 0.41 & 0.27 \\
        \bottomrule
    \end{tabular}
\end{table}

 %%%%%%%%%%%%%%%%%%%%%%%%%%%%%%%% RELIABILITY  %%%%%%%%%%%%%%%%%%%%%%%%%%%%%%%%
\section{Generating Reliability Warnings}\label{sec:reliability}
% up to 2.5 pages
Interpretability is a necessity for data scientists who develop predictive models for critical decision-making.
In such settings, it is important to provide additional means to support the following question:
{\it is an individual prediction of the model reliable for decision-making?} Our goal is to use the lack of representation to help decision-makers find insights about this critical question.
To further motivate this, let us use the following example:

\vspace{1mm}
\begin{example}\label{ex-0}
{\bf(Part1):} Consider a judge who needs to decide whether to accept or deny a bail request. Using data-driven predictive models is prevalent in such cases for predicting recidivism~\cite{dressel2018accuracy}.
Indeed, such models can be beneficial to help the judge make wise decisions.
Suppose the model predicts the queried individual as high risk (or low risk).
The judge is aware and concerned about the critics surrounding such models.
A major question the judge faces is whether or not they should rely on the prediction outcome to take action for this case.
Furthermore, if, for instance, they decide to ignore the outcome and hence they need to provide a statement supporting their action, what evidence can they provide? 
\end{example}

In line with the recent trend on data-centric AI~\cite{ng2021mlops}, we design {novel approaches}, {complimentary} to the existing work on trustworthy AI~\cite{wing2021trustworthy,kentour2021analysis,liu2021trustworthy,singh2021trustworthy}, to address the aforementioned trust question through the lens of {\it data}.
In particular, unlike existing works that generate trust information from a {\it given \underline{model}}, we associate {\it \underline{data sets} with proper measurements} that specify their {\it the scope of use for predicting future cases}.
We note that a predictive model provides only probabilistic guarantees on the \underline{average} loss over the distribution represented by the data set used for training it.
As a result, these predictions may not be distribution generalizable~\cite{kulynych2022you}.
Consequently, if the query point is {\it not represented} by the data, the guarantees may not hold, hence one cannot rely on the prediction outcome.
Besides, an essential requirement for a learning algorithm is that its training data $\dee$ should represent the underlying distribution $\dist$.
Even if so, the trained model $h$ only provides a probabilistic guarantee on the {expected} loss on random samples from $\dist$.  
A model that performs well on {\it majority} of samples drawn from $\dist$ will have a high performance on average. Still, as we observed in Figure~\ref{fig:mlfails},
its performance for {\it minorities} and points that are not represented is questionable. Let us consider the following toy example:

\begin{figure*}[!b] 
    \begin{minipage}[t]{0.32\linewidth}
        	\centering
        	\includegraphics[width=\textwidth]{submissions/submission1/shahbazi/example_1.png} 
        	\vspace{-9mm}\caption{\small Data set $\dee$ generated using a Gaussian distribution; $x_1$ and $x_2$ are positively correlated}
            \label{fig:ex1:1}
    \end{minipage}
    \hfill
    \begin{minipage}[t]{0.32\linewidth}
        \centering
        	\includegraphics[width =\textwidth]{submissions/submission1/shahbazi/example_2.png} 
        	\vspace{-9mm}\caption{\small The decision boundary of learned model $h$ and query points $\qu^1$ to $\qu^4$}
            \label{fig:ex1:2}
    \end{minipage}
    \hfill
    \begin{minipage}[t]{0.32\linewidth}
        	\centering
        	\includegraphics[width =\textwidth]{submissions/submission1/shahbazi/example_3.png}
        	\vspace{-9mm}\caption{\small Ground-truth boundary, overlaid on the model decision boundary and query points}
            \label{fig:ex1:3}
    \end{minipage}
    \vspace{-5mm}
\end{figure*} 

\vspace{1mm}
\begin{example}\label{ex-1}
Consider a binary classification task where the input space is $\ex=\langle x_1, x_2\rangle$ and the output space is the binary label $y$ with values $\{-1$ (red) $,+1$ (blue)$\}$.
Suppose the underlying data distribution $\dist$ follows a 2D Gaussian, where $x_1$ and $x_2$ 
are positively correlated as shown in Figure~\ref{fig:ex1:1}.
The figure shows the data set $\dee$ drawn independently from the distribution $\dist$, along with their labels as their colors.
Using $\dee$, the prediction model $h$ is constructed as shown in Figure~\ref{fig:ex1:2}. 
The decision boundary is specified in the picture; while any point above the line is predicted as +1, a query point below it is labeled as -1.
The classifier has been evaluated using a test set that is an iid sample set drawn from the underlying data set $\dist$. The accuracy on the test set is high (above 90\%), and hence, the model gets deployed.
We cherry-picked four query points, $\qu^1$ to $\qu^4$, that are also included in Figure~\ref{fig:ex1:2}. Using $h$ for prediction, $h(\qu^1)=-1$, $h(\qu^2)=+1$,  $h(\qu^3)=+1$, and $h(\qu^4)=-1$.
Figure~\ref{fig:ex1:3} adds the ground-truth boundary to the search space, revealing the true label of the query points: every point inside the red circle has the true label $-1$ while any point outside of it is $+1$.
Looking at the figure, $y^1=+1$ while the model predicted it as $h(\qu^1)=-1$.  \hfill$\square$
\end{example}
\vspace{2mm}

Let us take a closer look at the four query points in this example and their placement with regard to the tuples in $\dee$ used for training $h$. 
$\qu^2$ belongs to a {\it dense region} with many training tuples in $\dee$ surrounding it. Besides, all of the tuples in its vicinity have the same label $y=+1$. As a result, one can expect that the model's outcome $h(\qu^2)=+1$ should be a reliable prediction.
Similar to $\qu^2$, $\qu^4$ also belongs to a dense region in $\dee$; however, $\qu^4$ belongs to an {\it uncertain region}, where some of the tuples in its vicinity have a label $y=+1$, and some others have the label $y=-1$. Considering the uncertainty in the vicinity of $\qu^4$, one cannot confidently rely on the outcome of the model $h$. 
On the other hand, the neighbors of $\qu^1$ (resp. $\qu^3$) are not uncertain, all having the label $y=-1$ (resp. $y=+1$).
However, the query points $\qu^1$ and $\qu^3$ are not well represented by $\dee$. In other words, $\qu^1$ and $\qu^3$ are unlikely to be generated according to the underlying distribution $\dist$, represented by $\dee$. As a result, following the no-free-lunch theorem~\cite{kakade2003sample}, one cannot expect the outcome of model $h$ to be reliable for these points.
Looking at the ground-truth boundary in Figure~\ref{fig:ex1:3}, $h$ luckily predicted the outcome for $\qu^3$ correctly, but it was not fortunate to predict the $y^1$ correctly.
Nevertheless, 
since the model is not reliably trained for these points, 
its outcome for these query points is not trustworthy.

From Example~\ref{ex-1}, we observe that the outcome of a model $h$, trained using a data set $\dee$ is not reliable for a query point $\qu$, if:
\begin{itemize}
    \item {\bf Lack of representation:} $\qu$ is not well-represented by $\dee$.
    In such cases, the model has not seen ``enough'' samples similar to $\qu$ to reliably learn and predict the outcome of $\qu$.
    \item {\bf Lack of certainty:} $\qu$ belongs to an uncertain region, where different tuples of $\dee$ in the vicinity of $\qu$ have different target values. $\qu$ belongs to a high-fluctuating area, where tuples in the vicinity of $\qu$ have a wide range of values.
\end{itemize} \vspace{2mm}

\noindent
Based on these two observations, we propose Representation-and-Uncertainty ({\bf RU}) measures.
To identify if a query suffers from uncertainty or lack of representation, one could use a deterministic approach using a fixed threshold. Then if the number of similar samples to (resp. label fluctuation in vicinity of) $\qu$ is larger than the threshold it is considered as unrepresented (resp. uncertain).
This approach, however, would be misleading since two numbers close to the threshold could be treated very differently. Also, all points on each side of the threshold would be considered equally represented (resp., certain). Instead, we consider {\it a randomized approach}, widely popular in the literature, including~\cite{dwork2012fairness}.
That is, instead of using fixed thresholds, a Bernoulli variable (a biased coin) is used that 
assigns $\qu$ as unrepresented (resp., uncertain) based on the number of samples similar to it (resp., its neighborhood uncertainty).
Given a query point $\qu$, let $\pe_o$ be the probability indicating if $\qu$ is not represented and let $\pe_u$ be the probability indicating if $\qu$ belongs to an uncertain region. 
We represent the probability of the Bernoulli variables for lack of representation or uncertainty components as $\pe_o$ and $\pe_u$, respectively. Note that the two Bernoulli variables $\pe_o$ and $\pe_u$ are independent from each other. That simply follows the argument that after specifying the number of similar samples to $\qu$ whether or not it should be considered as unrepresented does not depend on the uncertainty in the neighborhood of $\qu$.

\begin{definition}[\sru]\label{def:sdt}
The \sru is a probabilistic measure that considers the outcome of a model for a query point $\qu$ untrustworthy if $\qu$ is not represented by $\dee$ {\it and} it belongs to an uncertain region.
Formally, the \sru measure is:
\begin{align} 
    \nonumber
    SRU(\qu) &= \pe\big((\qu \mbox{ is outlier}) \wedge (\qu \mbox{ belongs to uncertain region})\big) 
\end{align}
Since $\pe_o$ and $\pe_u$ are independent:

\vspace{-13mm}
\begin{align} \label{eq:strong}
    SRU(\qu) &= \pe_o(\qu) \times \pe_u(\qu)
\end{align}
\end{definition}

\sru raises the warning signal only when the query point fails on {\it both} conditions of being represented by $\dee$ and not belonging to an uncertain region. 
For instance, in Example~\ref{ex-1} none of the query points fail both on representation and on uncertainty; hence neither has a high \sru score.
On the other hand, 
a high \sru score for a query point $\qu$ {\it provides a strong warning signal} that one should perhaps reject the model outcome and not consider it for decision-making.

\sru is a strong signal that raises warnings only for the fearfully concerning cases that fail both on representation and uncertainty.
However, as observed in Example~\ref{ex-1} a query points failing {\it at least} one of these conditions may also not be reliable, at least for critical decision making.
We define the \wru measure to raise a warning for such cases.

\begin{definition}[\wru]\label{def:wdt}
The \wru measure is a probabilistic measure that considers the outcome of a model for a query point $\qu$ untrustworthy if $\qu$ is not represented by $\dee$ {\bf or} it belongs to an uncertain region.
Formally, the \wru is computed as:
\begin{align} \label{eq:weak}
    WRU(\qu) = \pe\big((\qu \mbox{ is outlier}) \vee (\qu \mbox{ belongs to uncertain region})\big) 
    = \pe_o(\qu) + \pe_u(\qu) - \pe_o(\qu) \times \pe_u(\qu)
\end{align}
\end{definition}

Proposing quantitative probabilistic outcomes, \ru measures are interpretable for the users, since beyond the scores, the uncertainty and lack of representation components provide an explanation to justify them. 
Please refer to \cite{techrep} for more details on how to efficiently and effectively compute the representation ($\pe_o$) and uncertainty ($\pe_u$) probabilities, using only $\dee$.
In Example~\ref{ex-0}, let us see how the \ru measures can be helpful.

\noindent{\bf Example 1. (part 2):}
{\it RU measures \underline{raise warning} when
the fitness of the data set used for drawing a prediction is questionable, helping the judge to be cautious when taking action.
Besides, these measures provide \underline{quantitative evidence} to support the judge's action when they decide to ignore a prediction outcome that is not trustworthy.
The judge, for example, can argue to ignore a model outcome for a specific case, based on the insight that 
the model has been built using a
data set that fails to represent the given case.}
\hfill$\square$

Finally, let us demonstrate the efficacy of \ru measures through a series of experiments. Since the \ru measures are {\it data-centric},
those are applicable for both classification and regression tasks, irrespective of the model used.
We use {\it Adult} dataset~\cite{adult} for classification and {\it House Sales in King County} dataset for the validation of regression tasks. From each dataset, we uniformly sample two sets from the underlying distribution. The first set serves as the training set to compute the \ru values, and the second one is used as the test set from which the queries are drawn. We validate our proposal by providing the correlation between the \ru values and the performance of an ML model's prediction on the same data. 

We start by computing the \ru values for all the query points in the test set. Next, we bucketize the query points based on their \ru values in equi-width buckets of width 0.1. We repeat this for both \sru and \wru measures. Next, we train a model on the training data set and predict the target variable for the points in each range of \ru measure. The validation results for the classification task on the {\it Adult} dataset are presented in Figures \ref{fig:exp-adult-sdt} and \ref{fig:exp-adult-wdt}. Each figure corresponds to the accuracy/error measures of the classifier over each bucket of \ru values for \sru and \wru. As the \ru values increase, the accuracy of the model drops while the FPR rises, and therefore, the model fails to capture the ground truth for the points that fall into untrustworthy regions in the data set. By repeating the aforementioned steps for the regression task on the {\it House Sales in King County} dataset, we observe similar results presented in Figures \ref{fig:exp-hs-sdt} and \ref{fig:exp-hs-wdt}. 
As the \ru value increases, the RSS of the regression model follows the same trend denoting that the model fails to perform for tuples with a high \ru value.

\begin{figure}[!tb]
    \begin{minipage}[t]{0.24\linewidth}
        \centering
        \includegraphics[width=\textwidth]{submissions/submission1/shahbazi/sdt_adult.pdf}
        \vspace{-6mm}\caption{\small{\it Adult}, efficacy of \sru  on classification}
        \label{fig:exp-adult-sdt}
    \end{minipage}\hfill
    \begin{minipage}[t]{0.24\linewidth}
        \centering
        \includegraphics[width=\textwidth]{submissions/submission1/shahbazi/wdt_adult.pdf}
        \vspace{-6mm}\caption{\small{\it Adult}, efficacy of \wru  on classification}
        \label{fig:exp-adult-wdt}
    \end{minipage}\hfill
    \begin{minipage}[t]{0.24\linewidth}
        \centering
        \includegraphics[width=\textwidth]{submissions/submission1/shahbazi/sdt_regression_house.pdf}
        \vspace{-6mm}\caption{\small{\it House Sales in King County}, efficacy of \sru on regression}
        \label{fig:exp-hs-sdt}
    \end{minipage}\hfill
    \begin{minipage}[t]{0.24\linewidth}
        \centering
        \includegraphics[width=\textwidth]{submissions/submission1/shahbazi/wdt_regression_house.pdf}
        \vspace{-6mm}\caption{\small{\it House Sales in King County}, efficacy \wru on regression}
        \label{fig:exp-hs-wdt}
    \end{minipage}
\vspace{-5mm}
\end{figure}
 %%%%%%%%%%%%%%%%%%%%%%%%%%%%%%%% RELATED WORK  %%%%%%%%%%%%%%%%%%%%%%%%%%%%%%%%
\section{Related Work}\label{related} 

Bias in data has been looked at for a long time in statistical community~\cite{neyman1936contributions} but social data presents different challenges~\cite{olteanu2019social,fairmlbook,barocas2016big,jk2019bias,drosou2017diversity}.
The diversity and representativeness of data have been widely studied~\cite{drosou2017diversity}, in fields such as social science~\cite{berrey2015enigma, dobbin2016diversity,simpson1949measurement}, political science~\cite{surowiecki2005wisdom}, and information retrieval~\cite{agrawal2009diversifying}. 
Tracing back machine bias to its source, there have been major efforts to identify different types~\cite{mehrabi2021survey, olteanu2019social,friedman1996bias} and sources~\cite{torralba2011unbiased,crawford2013hidden,diakopoulos2015algorithmic} of biases in data. Efforts to satisfy {\it responsible data} requirements~\cite{nargesian2022responsible} extend to various stages of the data analysis pipeline, including data annotation~\cite{li2020towards,lazier2023fairness}, data cleaning and repair~\cite{SalimiRHS19,tae2019data,salimi2020database}, data imputation~\cite{martinez2019fairness}, entity resolution~\cite{shahbazi2023through,fanourakis2023fairer}, data integration~\cite{nargesian2022responsible,nargesian2021tailoring}, etc. 

\paragraph{Data Coverage:}The notion of data coverage has received extensive attention from different angles. Detecting lack of coverage has been studied for datasets with discrete~\cite{asudeh2019assessing} and continuous~\cite{asudeh2021coverage} attributes populated in single or multiple \cite{lin2020identifying} relations.
To resolve insufficient coverage, \cite{accinelli2020coverage, accinelli2021impact,shetiya2022fairness}
consider resolving representation bias in preprocessing pipelines by rewriting queries into the closest operation so that certain subgroups are sufficiently represented in the downstream tasks. Alternatively, ~\cite{asudeh2019assessing,tae2021slice} propose a data collection strategy to acquire as little additional data as possible (to minimize the associated costs) to meet the representation constraints. ~\cite{sharma2020data,iosifidis2018dealing,celis2020data} opt for a data augmentation approach by adding partially altered duplicates of already existing tuples or generating new synthetic entries from existing data. Consequently, the new data set has an equal number of elements for different groups, resulting in potentially resolving the under-representation issues. Finally,  \cite{nargesian2021tailoring} utilizes data integration techniques to consolidate data from different sources into a single dataset to resolve representation bias.
Related works also include ~\cite{chung2019slice,sagadeeva2021sliceline,tae2021slice} that seek to understand if the overall performance of the model fails to reflect and performs poorly on certain slices in the data.
As alternative approaches to measure representation bias, the notion of representation rate~\cite{celis2020data} (a.k.a. equal base rate~\cite{kleinberg2016inherent}) is introduced which compared with coverage, it is more restrictive as it requires almost equal ratios from different groups.
Please refer to \cite{shahbazi2023representation} for a comprehensive survey about representation bias in data. 

\paragraph{ML Reliability:} Model-centric works for uncertainty quantification such as 
probabilistic classifiers~\cite{zadrozny2001obtaining,zadrozny2002transforming,platt1999probabilistic,niculescu2005predicting},
prediction intervals (PIs) \cite{chatfield93predictionintervals,pearce2018high,khosravi2010lower} and conformal predictions (CP)~\cite{angelopoulos2021gentle,shafer2008tutorial} that are used for measuring prediction uncertainty, are built
by maximizing the {\it expected performance} on {\it random} sample from the underlying distribution.
As a result, while providing accurate estimations for the dense regions of data (e.g. majority groups), their estimation accuracy is questionable for the poorly represented regions.
In particular, \cite{angelopoulos2021gentle} recognizes the lack of guarantees in the performance of CP for such regions.
Besides, the bulk of work on trustworthy AI provides information that {\it supports} the outcome of an ML model. For example, existing work on explainable AI, including~\cite{harradon2018causal,ribeiro2016should,gunning2019darpa}, aims to find simple explanations and rules that justify the outcome of a model.
Conversely, we aim to {\it raise warning signals} when the outcome of a model is {\it not} trustworthy. That is, to provide reasons that {\it cast doubt} on the reliability of the model outcome {for a given query point}.

 %%%%%%%%%%%%%%%%%%%%%%%%%%%%%%%% FUTURE  %%%%%%%%%%%%%%%%%%%%%%%%%%%%%%%%
% \vspace{-3mm}
\section{Final Remarks}\label{sec:conclusion}
As Data-centric AI and Responsible AI emerge as focal points in data science research, the development of Data-centric methodologies for ensuring Responsible and Trustworthy AI attracts increasing attention.
While there is some excellent work on responsible data management to achieve this goal, there remain many challenges yet to be addressed.

In this paper, we focused on a crucial aspect of responsible data -- detecting and addressing the under-representation of minorities within a data set.
We formally defined the notion of data coverage and discussed various techniques for (a) identifying lack of representation issues across different data modalities, (b) ensuring proper representation of minorities in data, and (c) limiting the scope-of-use of data sets based on their representation issues by generating proper ({\sc RU}) warning signals.
Even though the research on detecting lack of coverage issues is relatively mature, resolution techniques are still understudied.
Considering the recent advancements in Generative AI, utilizing Foundation Models and Large Language Models, and studying their limitations, for data augmentation to improve the representation of minorities at the data level seems interesting to further explore.

 %%%%%%%%%%%%%%%%%%%%%%%%%%%%%%%% BIB  %%%%%%%%%%%%%%%%%%%%%%%%%%%%%%%%
\bibliographystyle{unsrt}
\small
% \bibliography{ref}
\begin{thebibliography}{10}

\bibitem{asudeh2019assessing}
A.~Asudeh, Z.~Jin, and H.~Jagadish.
\newblock Assessing and remedying coverage for a given dataset.
\newblock In {\em ICDE}, pages 554--565. IEEE, 2019.

\bibitem{shahbazi2023representation}
N.~Shahbazi, Y.~Lin, A.~Asudeh, and H.~Jagadish.
\newblock Representation bias in data: A survey on identification and resolution techniques.
\newblock {\em ACM Computing Surveys}, 2023.

\bibitem{asudeh2021coverage}
A.~Asudeh, N.~Shahbazi, Z.~Jin, and H.~V. Jagadish.
\newblock Identifying insufficient data coverage for ordinal continuous-valued attributes.
\newblock In {\em SIGMOD}. ACM, 2021.

\bibitem{mousavi2024data}
M.~Mousavi, N.~Shahbazi, and A.~Asudeh.
\newblock Data coverage for detecting representation bias in image datasets: {A} crowdsourcing approach.
\newblock In {\em {EDBT}}, pages 47--60, 2024.

\bibitem{nargesian2021tailoring}
F.~Nargesian, A.~Asudeh, and H.~Jagadish.
\newblock Tailoring data source distributions for fairness-aware data integration.
\newblock {\em Proceedings of the VLDB Endowment}, 14(11):2519--2532, 2021.

\bibitem{nargesian2022responsible}
F.~Nargesian, A.~Asudeh, and H.~V. Jagadish.
\newblock Responsible data integration: Next-generation challenges.
\newblock {\em SIGMOD}, 2022.

\bibitem{sharma2020data}
S.~Sharma, Y.~Zhang, J.~M. R{\'\i}os~Aliaga, D.~Bouneffouf, V.~Muthusamy, and K.~R. Varshney.
\newblock Data augmentation for discrimination prevention and bias disambiguation.
\newblock In {\em AIES}, pages 358--364, 2020.

\bibitem{DBLP:journals/jair/ChawlaBHK02}
N.~V. Chawla, K.~W. Bowyer, L.~O. Hall, and W.~P. Kegelmeyer.
\newblock {SMOTE:} synthetic minority over-sampling technique.
\newblock {\em J. Artif. Intell. Res.}, 16:321--357, 2002.

\bibitem{iosifidis2018dealing}
V.~Iosifidis and E.~Ntoutsi.
\newblock Dealing with bias via data augmentation in supervised learning scenarios.
\newblock {\em Jo Bates Paul D. Clough Robert J{\"a}schke}, 24, 2018.

\bibitem{celis2020data}
L.~E. Celis, V.~Keswani, and N.~Vishnoi.
\newblock Data preprocessing to mitigate bias: A maximum entropy based approach.
\newblock In {\em ICML}, pages 1349--1359. PMLR, 2020.

\bibitem{asudeh2022towards}
A.~Asudeh and F.~Nargesian.
\newblock Towards distribution-aware query answering in data markets.
\newblock {\em Proceedings of the VLDB Endowment}, 15(11):3137--3144, 2022.

\bibitem{motwani1995randomized}
R.~Motwani and P.~Raghavan.
\newblock {\em Randomized algorithms}.
\newblock Cambridge university press, 1995.

\bibitem{chameleon}
M.~Erfanian, H.~V. Jagadish, and A.~Asudeh.
\newblock Chameleon: Foundation models for fairness-aware multi-modal data augmentation to enhance coverage of minorities.
\newblock {\em arXiv preprint arXiv:2402.01071}, 2024.

\bibitem{scholkopf1999support}
B.~Sch{\"o}lkopf, R.~C. Williamson, A.~Smola, J.~Shawe-Taylor, and J.~Platt.
\newblock Support vector method for novelty detection.
\newblock {\em NeurIPS}, 12, 1999.

\bibitem{phillips1998feret}
P.~J. Phillips, H.~Wechsler, J.~Huang, and P.~J. Rauss.
\newblock The feret database and evaluation procedure for face-recognition algorithms.
\newblock {\em Image and vision computing}, 16(5):295--306, 1998.

\bibitem{dressel2018accuracy}
J.~Dressel and H.~Farid.
\newblock The accuracy, fairness, and limits of predicting recidivism.
\newblock {\em Science advances}, 4(1):eaao5580, 2018.

\bibitem{ng2021mlops}
A.~Ng.
\newblock Mlops: From model-centric to data-centric {AI}.
\newblock 2021.

\bibitem{wing2021trustworthy}
J.~M. Wing.
\newblock Trustworthy {AI}.
\newblock {\em CACM}, 64(10):64--71, 2021.

\bibitem{kentour2021analysis}
M.~Kentour and J.~Lu.
\newblock Analysis of trustworthiness in machine learning and deep learning.
\newblock {\em InfoComp}, 2021.

\bibitem{liu2021trustworthy}
H.~Liu, Y.~Wang, W.~Fan, X.~Liu, Y.~Li, S.~Jain, A.~K. Jain, and J.~Tang.
\newblock Trustworthy {AI}: A computational perspective.
\newblock {\em arXiv preprint arXiv:2107.06641}, 2021.

\bibitem{singh2021trustworthy}
R.~Singh, M.~Vatsa, and N.~Ratha.
\newblock Trustworthy {AI}.
\newblock In {\em 8th ACM IKDD CODS and 26th COMAD}, pages 449--453. 2021.

\bibitem{kulynych2022you}
B.~Kulynych, Y.-Y. Yang, Y.~Yu, J.~B{\l}asiok, and P.~Nakkiran.
\newblock What you see is what you get: Distributional generalization for algorithm design in deep learning.
\newblock {\em arXiv preprint arXiv:2204.03230}, 2022.

\bibitem{kakade2003sample}
S.~M. Kakade.
\newblock {\em On the sample complexity of reinforcement learning}.
\newblock University of London, University College London (United Kingdom), 2003.

\bibitem{dwork2012fairness}
C.~Dwork, M.~Hardt, T.~Pitassi, O.~Reingold, and R.~Zemel.
\newblock Fairness through awareness.
\newblock In {\em ITCS}, pages 214--226, 2012.

\bibitem{techrep}
N.~Shahbazi and A.~Asudeh.
\newblock Data-centric reliability evaluation of individual predictions.
\newblock {\em CoRR, abs/2204.07682}, 2022.

\bibitem{adult}
M.~Lichman.
\newblock Adult income dataset, {UCI} machine learning repository.
\newblock \url{https://archive.ics.uci.edu/ml/datasets/adult}, 2013.

\bibitem{neyman1936contributions}
J.~Neyman and E.~S. Pearson.
\newblock Contributions to the theory of testing statistical hypotheses.
\newblock {\em Statistical Research Memoirs}, 1936.

\bibitem{olteanu2019social}
A.~Olteanu, C.~Castillo, F.~Diaz, and E.~Kiciman.
\newblock Social data: Biases, methodological pitfalls, and ethical boundaries.
\newblock {\em Frontiers in Big Data}, 2:13, 2019.

\bibitem{fairmlbook}
S.~Barocas, M.~Hardt, and A.~Narayanan.
\newblock Fairness and machine learning: Limitations and opportunities.
\newblock \url{fairmlbook.org}, 2019.

\bibitem{barocas2016big}
S.~Barocas and A.~D. Selbst.
\newblock Big data's disparate impact.
\newblock {\em Calif. L. Rev.}, 104:671, 2016.

\bibitem{jk2019bias}
J.~Kleinberg.
\newblock Fairness, rankings, and behavioral biases.
\newblock FAT*, 2019.

\bibitem{drosou2017diversity}
M.~Drosou, H.~Jagadish, E.~Pitoura, and J.~Stoyanovich.
\newblock Diversity in big data: A review.
\newblock {\em Big data}, 5(2):73--84, 2017.

\bibitem{berrey2015enigma}
E.~Berrey.
\newblock {\em The enigma of diversity: The language of race and the limits of racial justice}.
\newblock University of Chicago Press, 2015.

\bibitem{dobbin2016diversity}
F.~Dobbin and A.~Kalev.
\newblock Why diversity programs fail and what works better.
\newblock {\em Harvard Business Review}, 94(7-8):52--60, 2016.

\bibitem{simpson1949measurement}
E.~H. Simpson.
\newblock Measurement of diversity.
\newblock {\em Nature}, 163(4148), 1949.

\bibitem{surowiecki2005wisdom}
J.~Surowiecki.
\newblock {\em The wisdom of crowds}.
\newblock Anchor, 2005.

\bibitem{agrawal2009diversifying}
R.~Agrawal, S.~Gollapudi, A.~Halverson, and S.~Ieong.
\newblock Diversifying search results.
\newblock In {\em WSDM}, pages 5--14. ACM, 2009.

\bibitem{mehrabi2021survey}
N.~Mehrabi, F.~Morstatter, N.~Saxena, K.~Lerman, and A.~Galstyan.
\newblock A survey on bias and fairness in machine learning.
\newblock {\em ACM Computing Surveys (CSUR)}, 54(6):1--35, 2021.

\bibitem{friedman1996bias}
B.~Friedman and H.~Nissenbaum.
\newblock Bias in computer systems.
\newblock {\em TOIS}, 14(3):330--347, 1996.

\bibitem{torralba2011unbiased}
A.~Torralba and A.~A. Efros.
\newblock Unbiased look at dataset bias.
\newblock In {\em CVPR 2011}, pages 1521--1528. IEEE, 2011.

\bibitem{crawford2013hidden}
K.~Crawford.
\newblock The hidden biases in big data.
\newblock {\em Harvard business review}, 1(4), 2013.

\bibitem{diakopoulos2015algorithmic}
N.~Diakopoulos.
\newblock Algorithmic accountability: Journalistic investigation of computational power structures.
\newblock {\em Digital journalism}, 3(3):398--415, 2015.

\bibitem{li2020towards}
Y.~Li, H.~Sun, and W.~H. Wang.
\newblock Towards fair truth discovery from biased crowdsourced answers.
\newblock In {\em SIGKDD}, pages 599--607, 2020.

\bibitem{lazier2023fairness}
S.~Lazier, S.~Thirumuruganathan, and H.~Anahideh.
\newblock Fairness and bias in truth discovery algorithms: An experimental analysis.
\newblock {\em arXiv preprint arXiv:2304.12573}, 2023.

\bibitem{SalimiRHS19}
B.~Salimi, L.~Rodriguez, B.~Howe, and D.~Suciu.
\newblock Interventional fairness: Causal database repair for algorithmic fairness.
\newblock In {\em {SIGMOD}}, pages 793--810. {ACM}, 2019.

\bibitem{tae2019data}
K.~H. Tae, Y.~Roh, Y.~H. Oh, H.~Kim, and S.~E. Whang.
\newblock Data cleaning for accurate, fair, and robust models: A big data-{AI} integration approach.
\newblock In {\em DEEM workshop}, pages 1--4, 2019.

\bibitem{salimi2020database}
B.~Salimi, B.~Howe, and D.~Suciu.
\newblock Database repair meets algorithmic fairness.
\newblock {\em ACM SIGMOD Record}, 49(1):34--41, 2020.

\bibitem{martinez2019fairness}
F.~Mart{\'\i}nez-Plumed, C.~Ferri, D.~Nieves, and J.~Hern{\'a}ndez-Orallo.
\newblock Fairness and missing values.
\newblock {\em arXiv preprint arXiv:1905.12728}, 2019.

\bibitem{shahbazi2023through}
N.~Shahbazi, N.~Danevski, F.~Nargesian, A.~Asudeh, and D.~Srivastava.
\newblock Through the fairness lens: Experimental analysis and evaluation of entity matching.
\newblock {\em Proceedings of the VLDB Endowment}, 16(11):3279--3292, 2023.

\bibitem{fanourakis2023fairer}
N.~Fanourakis, C.~Kontousias, V.~Efthymiou, V.~Christophides, and D.~Plexousakis.
\newblock Fairer demo: Fairness-aware and explainable entity resolution.
\newblock 2023.

\bibitem{lin2020identifying}
Y.~Lin, Y.~Guan, A.~Asudeh, and H.~Jagadish.
\newblock Identifying insufficient data coverage in databases with multiple relations.
\newblock {\em Proceedings of the VLDB Endowment}, 13(12):2229--2242, 2020.

\bibitem{accinelli2020coverage}
C.~Accinelli, S.~Minisi, and B.~Catania.
\newblock Coverage-based rewriting for data preparation.
\newblock In {\em EDBT Workshops}, 2020.

\bibitem{accinelli2021impact}
C.~Accinelli, B.~Catania, G.~Guerrini, and S.~Minisi.
\newblock The impact of rewriting on coverage constraint satisfaction.
\newblock In {\em EDBT Workshops}, 2021.

\bibitem{shetiya2022fairness}
S.~Shetiya, I.~P. Swift, A.~Asudeh, and G.~Das.
\newblock Fairness-aware range queries for selecting unbiased data.
\newblock In {\em ICDE}. IEEE, 2022.

\bibitem{tae2021slice}
K.~H. Tae and S.~E. Whang.
\newblock Slice tuner: A selective data acquisition framework for accurate and fair machine learning models.
\newblock In {\em SIGMOD}, pages 1771--1783, 2021.

\bibitem{chung2019slice}
Y.~Chung, T.~Kraska, N.~Polyzotis, K.~H. Tae, and S.~E. Whang.
\newblock Slice finder: Automated data slicing for model validation.
\newblock In {\em ICDE}, pages 1550--1553. IEEE, 2019.

\bibitem{sagadeeva2021sliceline}
S.~Sagadeeva and M.~Boehm.
\newblock Sliceline: Fast, linear-algebra-based slice finding for ml model debugging.
\newblock In {\em SIGMOD}, pages 2290--2299, 2021.

\bibitem{kleinberg2016inherent}
J.~Kleinberg, S.~Mullainathan, and M.~Raghavan.
\newblock Inherent trade-offs in the fair determination of risk scores.
\newblock {\em arXiv preprint arXiv:1609.05807}, 2016.

\bibitem{zadrozny2001obtaining}
B.~Zadrozny and C.~Elkan.
\newblock Obtaining calibrated probability estimates from decision trees and naive bayesian classifiers.
\newblock In {\em ICML}, volume~1, pages 609--616. Citeseer, 2001.

\bibitem{zadrozny2002transforming}
B.~Zadrozny and C.~Elkan.
\newblock Transforming classifier scores into accurate multiclass probability estimates.
\newblock In {\em SIGKDD}, pages 694--699, 2002.

\bibitem{platt1999probabilistic}
J.~Platt et~al.
\newblock Probabilistic outputs for support vector machines and comparisons to regularized likelihood methods.
\newblock {\em Advances in large margin classifiers}, 10(3):61--74, 1999.

\bibitem{niculescu2005predicting}
A.~Niculescu-Mizil and R.~Caruana.
\newblock Predicting good probabilities with supervised learning.
\newblock In {\em Proceedings of the 22nd international conference on Machine learning}, pages 625--632, 2005.

\bibitem{chatfield93predictionintervals}
C.~Chatfield.
\newblock Prediction intervals.
\newblock {\em Journal of Business and Economic Statistics}, 11:121--135, 1993.

\bibitem{pearce2018high}
T.~Pearce, A.~Brintrup, M.~Zaki, and A.~Neely.
\newblock High-quality prediction intervals for deep learning: A distribution-free, ensembled approach.
\newblock In {\em International conference on machine learning}, pages 4075--4084. PMLR, 2018.

\bibitem{khosravi2010lower}
A.~Khosravi, S.~Nahavandi, D.~Creighton, and A.~F. Atiya.
\newblock Lower upper bound estimation method for construction of neural network-based prediction intervals.
\newblock {\em IEEE transactions on neural networks}, 22(3):337--346, 2010.

\bibitem{angelopoulos2021gentle}
A.~N. Angelopoulos and S.~Bates.
\newblock A gentle introduction to conformal prediction and distribution-free uncertainty quantification.
\newblock {\em arXiv preprint arXiv:2107.07511}, 2021.

\bibitem{shafer2008tutorial}
G.~Shafer and V.~Vovk.
\newblock A tutorial on conformal prediction.
\newblock {\em Journal of Machine Learning Research}, 9(3), 2008.

\bibitem{harradon2018causal}
M.~Harradon, J.~Druce, and B.~Ruttenberg.
\newblock Causal learning and explanation of deep neural networks via autoencoded activations.
\newblock {\em arXiv preprint arXiv:1802.00541}, 2018.

\bibitem{ribeiro2016should}
M.~T. Ribeiro, S.~Singh, and C.~Guestrin.
\newblock " why should i trust you?" explaining the predictions of any classifier.
\newblock In {\em SIGKDD}, pages 1135--1144, 2016.

\bibitem{gunning2019darpa}
D.~Gunning and D.~Aha.
\newblock Darpa’s explainable artificial intelligence ({XAI}) program.
\newblock {\em AI Magazine}, 40(2):44--58, 2019.

\end{thebibliography}

\end{document}

\end{article}

\begin{article}
{Building Deletion-Compliant Data Systems}
{Manos Athanassoulis, Subhadeep Sarkar, Tarikul Islam Papon, Zichen Zhu, Dimitris Staratzis}
\graphicspath{{submissions/building-deletion-compliant-data-systems/}}
\documentclass[11pt,dvipdfmx]{article}


\usepackage{deauthor,times,graphicx}
\usepackage{booktabs} 
\usepackage{amsmath}
\usepackage{epstopdf}
\usepackage{verbatimbox}
\usepackage{multirow} 

%\graphicspath{{athanassoulis/}}


\newcommand\Paragraph[1]{\vspace{0.02in}  \noindent \textbf{#1.}}
\newcommand\Paragraphqit[1]{\vspace{0.02in}  \noindent \textit{#1?}}
\newcommand\Paragraphbit[1]{\vspace{0.02in}  \noindent \textbf{\textit{#1.}}}

\begin{document}



\title{Building Deletion-Compliant Data Systems}


\author{
Manos Athanassoulis, Subhadeep Sarkar, Tarikul Islam Papon, Zichen Zhu, Dimitris Staratzis
\medskip\\
Boston University
}



\maketitle


\begin{abstract}
Most modern data systems have been designed with two goals in mind -- fast ingestion and 
low-latency query processing. The first goal has led to the development of a plethora of 
write-optimized data stores that employ the \textit{out-of-place} paradigm. 
Due to their write-optimized design, out-of-place data systems perform deletes 
\textit{logically} via invalidation, and retain the invalid data for arbitrarily long. 
However, due to the recent enactment of new data privacy regulations, 
the requirement of \textit{timely deletion of user data} has become central. 
% One primary focus of privacy regulations is to establish the user's 
The \textit{right to be 
forgotten} (in EU's GDPR), \textit{right to delete} (in California's CCPA and CPRA), or 
\textit{deletion right} (in Virginia's VCDPA) mandates that service providers persistently delete a user's data within a pre-set time duration. 
Logical deletion in out-of-place data systems, however, does not offer guarantees for 
\textit{timely and persistent deletion}, and attempting to enforce it using existing tools 
leads to poor performance and increased operational costs. 

\hspace{0.02in} In this paper, we present a new framework for building \textit{deletion-compliant
data systems} from a holistic perspective. We analyze the new regulations and the 
requirements derived from the new policies, and we propose changes in the application and
the system layer of data management. We outline the new types of deletion requests that need to be supported, the query language
modifications needed to be able to request for timely persistent data deletion, and the system-level 
changes needed to realize timely and persistent deletes. 
The proposed framework for deletion compliance lays the groundwork for a new class of data 
systems that can offer system-level guarantees for user data privacy. We present recent results 
spanning all layers of the framework: the requirements and
the application layer target any database system, while the system layer discussion is geared 
towards out-of-place systems. Finally, we conclude with a discussion on next steps
and open challenges on building deletion-compliant data systems.

\end{abstract}





\section{Introduction}
\label{sec:introduction}


Data-intensive social and commercial applications and new advancements in domains like Internet-of-things, edge computing, 5G communications, and autonomous vehicles, generate a vast amount of \textit{personal data} processed by several data companies~\cite{Cisco2018,Gartner2017}. 
The increasing demand for efficient collection, storage, and processing of user data over the past two decades, has driven the development of data systems that can \textit{sustain high ingestion rates} without compromising the ability to \textit{access and analyze the data quickly}. 

\newpage
\Paragraph{Out-of-Place Systems}
The need for optimizing data ingestion while maintaining efficient data access has led
to the prominence of the \textit{out-of-place} paradigm, which fulfills these goals by
minimizing the interference between reads and writes. Today, several commercial relational and 
array-based data 
stores~\cite{Athanassoulis2011,Deng2020,Farber2012,Heman2010,Idreos2012,Kang2016,Lamb2012,Sadoghi2016,Stonebraker2005} and NoSQL data 
stores~\cite{ApacheAccumulo,ApacheCassandra,ApacheHBase,DeCandia2007,FacebookRocksDB,Golan-Gueta2015,Huang2019,Sears2012} have adopted the out-of-place paradigm. 

\textit{Relational and Array-based Systems}. 
Relational systems that buffer updates before applying them lazily on the base data, essentially, follow the out-of-place paradigm. 
The columnar and array data stores implemented by Vertica~\cite{Lamb2012,Stonebraker2005}, 
SciDB~\cite{Paradigm4,Stonebraker2013}, and TileDB~\cite{Papadopoulos2016,TileDB} use an in-memory storage component that stores incoming inserts, updates, and deletes out of place, and 
applies the changes lazily on the disk-resident data. Similarly, the state-of-the-art column-store system
MonetDB~\cite{Idreos2012} uses an in-memory positional index for incoming 
data~\cite{Heman2010}, and SAP HANA uses
a delta store per table to facilitate fast ingestion without affecting its 
read-optimized data layout~\cite{Farber2012}. 
Finally, several research-prototype systems
use a separate delta store on faster storage (e.g., SSDs/NVM) to 
offer efficient access to incoming 
data \cite{Athanassoulis2011,Athanassoulis2015,Deng2020,Kang2016,Sadoghi2016}.

\textit{NoSQL Systems}.
More than relational systems, production-grade NoSQL key-value stores predominantly employ the 
out-of-place paradigm, frequently based on the log-structured merge (LSM) paradigm.
An LSM-tree is a heavily write-optimized out-of-place data structure that maintains several on-disk components, which can be viewed as several out-of-place delta stores~\cite{ONeil1996,Dayan2017,Idreos2019,Luo2020b,Sarkar2022b,Zheng2018}.
Key-value stores such as RocksDB~\cite{Dong2017,FacebookRocksDB} and  LevelDB~\cite{GoogleLevelDB} at Facebook, BigTable~\cite{Chang2006} at Google, X-Engine~\cite{Huang2019,Yang2020} at Alibaba, Voldemort~\cite{LinkedInVoldemort} at LinkedIn, DynamoDB~\cite{DeCandia2007} at Amazon, Cassandra~\cite{ApacheCassandra}, HBase~\cite{ApacheHBase}, and Accumulo~\cite{ApacheAccumulo} at Apache, and bLSM~\cite{Sears2012} and cLSM~\cite{Golan-Gueta2015} at Yahoo are based on the log-structured merge (LSM) paradigm. 
Other out-of-place architectures employed by NoSQL systems are B$^+$-tree, B$^\epsilon$-tree, and fractal tree-based storage engines with \emph{buffering support}, such as COLA~\cite{Bender2000}, TokuDB~\cite{Kuszmaul2014}, and 
B\textit{e}rtFS~\cite{Bender2015,Jannen2015}. 

\textit{Cloud-based Systems}.
Cloud-based systems naturally employ the out-of-place paradigm as they rely on the 
immutability of cloud storage. Hence, systems like Amazon 
Redshift~\cite{AmazonRedshift,Gupta2015}, Cloud Data Platform~\cite{Dageville2016} at 
Snowflake, and Delta Lake~\cite{Databricks,Databricks2021} at Databricks employ variations
of the out-of-place paradigm in the interest of performance. 
Deletes and updates are initially performed logically and are gradually propagated to 
persistent media through periodic merging with base data. 

\Paragraph{Deletes in Out-of-Place Systems}
A key property of out-of-place systems is that they treat deletes (and updates) similarly to 
inserts, i.e., instead of deleting (updating) entries in-place, they insert a new version of 
the entry to be deleted that \textit{logically invalidates} the target entries. 
These special entries that are responsible for logical deletes are termed \textit{delete 
markers}~\cite{Lamb2012} or \textit{tombstones}~\cite{Dong2017,Sarkar2020}. 

Logical data deletion is a quintessential out-of-place operation, but it does not guarantee 
\emph{purging} of the data under deletion within a definite timeframe. Rather, the data is marked as
invalid; essentially, \emph{not accessible} to external users. In practice, logically 
deleted entries are kept for arbitrarily long in the system, since the time to definitively 
delete the data (termed \emph{persistent deletion}) depends on the state of the system, 
and not on when the user request expects the data to be deleted~\cite{Sarkar2020}.
In fact, most out-of-place data stores are built with the underlying assumption of 
\textit{perpetual data retention} in order to gain more insights from the user and 
organizational data~\cite{Whittaker2019}, hence \emph{timely persistent deletion} has not
been part of their design goals. In addition to deletes, logical updates in out-of-place 
systems are applied lazily too, however, the implications of out-of-place deletes are 
critical in terms of the privacy regulations, and thus, are our focus. 

\subsection{Problem: The Privacy Concern}
\Paragraph{Cost of Logical Deletes} Logical deletes and updates in out-of-place systems boost ingestion performance, however, they come at a significant cost.  
In fact, when tasked with deleting user data persistently in a timely manner, out-of-place systems suffer both in terms of (a) data privacy protection and (b) the overall system performance. 
Such systems are designed to retain the logically invalidated data indefinitely, and the time required for persistent removal of the physical data entries depends on (i) the data layout, (ii) the data re-organization policy (e.g., node splitting/merging in B-trees, compaction in LSM-trees, consolidation in TileDB), (iii) the design of the storage engine (such as the fanout of a tree and the size of a database), and (iv) the composition and distribution of the workload -- factors that are \textit{beyond the control of the application and system developers or administrators}. 
Thus, most out-of-place systems are unable to provide any latency guarantees for persistent deletion of user data~\cite{Sarkar2020}. 


\Paragraph{The Legal Frontier}
In recent years, a number of government-driven efforts across the globe unfolded, 
aiming to protect the privacy of user data and give back to the users the control of their 
personal data. On the legal side, regulations such as the EU's GDPR~\cite{GDPR}, California's 
CCPA~\cite{CCPA2018} and CPRA~\cite{CPRA}, and Virginia's 
VCDPA~\cite{VCPDA} have been introduced, which mandate that data companies ensure \textit{privacy through deletion}~\cite{Shastri2019,Shastri2021}. 
GDPR's \textit{right to be forgotten}, CCPA and CPRA's \textit{right to delete}, and the \textit{deletion right} in VCDPA particularly focus on \textit{persistent deletion of user data on-demand and in a timely manner}~\cite{Ambrose2013,Deshpande2018,Goddard2017,Jones2012,Sarkar2018,Shastri2019,Schwarzkopf2019,Tsesis2014}. 

\Paragraph{The Technological Roadblock} Treating \emph{deletes} as \textit{first-class citizens} is new for the data systems community, and it would require a significant amount of work to transform classical systems to be efficient deletion-wise. 
Even today, it continues to be a critical technological challenge for the biggest of data companies using state-of-the-art storage engines to demonstrate compliance with the deletion regulations and to efficiently delete user data on-demand~\cite{Shah2019,Shastri2020,Shastri2021}. 
To translate this into numbers, between January 2020 and January 2022, the penalties under GDPR paid by data companies amounted to more than \$1B, which includes large contributions from companies such as Amazon (\$877M), WhatsApp (\$255M), Google Ireland (\$102M), and Facebook (\$68M), H\&M (\$41M), British Airways (\$26M), and Marriot (\$23M)~\cite{Piper2022,DLAPiper2020,Tessian2022}. 
Thus, to demonstrate compliance, many companies end up performing expensive database-wide consolidations periodically (e.g., every few weeks), to ensure timely persistent deletion of user data~\cite{Sarkar2020,Sarkar2021c}. 
Such operations are remarkably expensive in terms of time and money, cause undesirable latency spikes, and hence, should be avoided. 

\begin{figure*}[tb]
    \centering
        \includegraphics[scale=0.29]{figs/layered_problem_scenario.pdf}
    \vspace{-0.25in} 
    \caption{The four layers of deletion-compliant data systems.}
     \vspace{-0.1in}
    \label{fig: vision}
\end{figure*}

 
\subsection{Deletion-Compliant Data Systems}
In this paper, we present our vision and first results on designing data systems that ensure data privacy through timely and persistent deletion of user data. 
Existing efforts that attempt to delete user/application data on-demand suffer in terms of performance as the underlying data layout and data management mechanisms are ill-suited for the purpose. 
We identify the missing links, in terms of technological tools, both at the application level and the system level, and we propose a hierarchical framework that enables our vision of privacy through deletion in out-of-place data systems (Figure~\ref{fig: vision}).



\newpage
\Paragraph{Roadmap} 
The privacy through deletion framework is a roadmap toward building deletion-compliant data systems. 
We begin by outlining the challenges associated with each layer of our vision, i.e., in the context of (i)~translating the legal mandates to user requirements, (ii) expressing the user requirements through a declarative API, and (iii) realizing the application-level requirements at the system level.
Next, we identify and categorize the different classes of user requests for deletes in light of the legal regulations. 
Based on this, we present the challenges associated with transforming the classes of deletion requests into application-level specifications, and we propose an SQL extension as an example that can be extended to other query languages to support deletion of user data \textit{periodically} and \textit{on-demand}. 
Further, we outline the design and tools necessary at the system level to support the application-level requirements. 
Finally, we conclude with a discussion on how the proposed framework drives us toward building deletion-compliant data systems, and what further research challenges remain open to fully realize this vision.






%
 
\section{From Regulation to Practice}
\label{sec:legal}


The legal landscape for data privacy has changed drastically over the past few years, and governments across countries, as well as across different states in the US, have enforced acts and regulations to control the consumption of user data by service providers and give back to the users the control of their personal data. 
Translating the new regulations to new \emph{user-data privacy-compliant system behavior}
still faces significant challenges. In this section, we present in more detail the 
requirements from the regulation point of view, and we showcase through three realistic
scenarios the limitations of the state-of-the-art data systems when tasked to implement
these requirements. 


\subsection{Regulations on Timely Data Deletion}
While the new regulations propose an array of new requirements, we particularly focus on the legal 
policies concerning data retention and data deletion, with the objective of ensuring 
\textit{privacy through deletion}. 

\Paragraph{Right to be Forgotten, \textit{EU GDPR}}
The General Data Protection Regulation (GDPR) has revolutionized the data privacy and security landscape across the European Union countries~\cite{GDPR}. 
One of the fundamental changes introduced through the GDPR (over the older Data Protection Act (DPA) that it replaced), is the \textit{right to be forgotten}, which empowers the users with the \emph{right} to request a service provider to delete all their personal data persistently from its domain. 
Such deletion requests may be presented either up-front or on-demand. 
The service provider must comply with those requests, unless it has compelling reasons for acting otherwise (Art. 17(3)). 


\Paragraph{Right to Delete, \textit{CCPA}, \textit{CPRA}}
The California Customer Protection Act (CCPA), which will eventually be replaced by the California Privacy Rights Act (CPRA) in 2023, allows the users/consumers in California to request from service providers to permanently delete all data personal to the user~\cite{CCPA2018,CPRA}. 
Under CCPA and CPRA, the service providers must acknowledge such a user request within $10$ days, and respond to the request within $45$ business days~\cite{Brown2021}. 
Persistent deletion must be performed by removing the target data across all domains, barring archive and backup systems, along with data anonymization as required. 

\Paragraph{Right to Delete, \textit{VCDPA}}
Similarly to CCPA, the Virginia Consumer Data Protection Act (VCDPA) empowers users in Virginia to exercise their right to delete their personal data from a provider's domain~\cite{VCPDA}. 
VCDPA requires the service providers to serve a delete-request from a user within $45$ business days~\cite{Brown2021}. 


\Paragraph{Right to be Forgotten, \textit{UK GDPR, DPA}} 
The UK GDPR, along with the Data Protection Act (DPA) 2018 provides the country's citizens with similar rights about personal data deletion as the EU GDPR. 
The users are allowed to express their deletion preference verbally or in writing, to which the service providers must respond within $30$ days~\cite{UKGDPR,DPA2018}. 



\Paragraph{Other Efforts}  
Among other countries, Argentina~\cite{Carter2013,Pardo2020}, Singapore~\cite{Chik2013}, India~\cite{Kittane2021}, Canada~\cite{PIPEDA2019}, and South Korea~\cite{Brown2016} have some implementation of the right to deletion as a part of their respective privacy protection acts.



\subsection{Limitations of the State of the Art} 
In light of the deletion regulations, we now present three real-life scenarios to highlight why state-of-the-art data systems are ill-equipped to support deletes efficiently without hurting performance. 
We do so by identifying the missing links in different hierarchical levels of the proposed privacy-through-deletion framework.
Below, we illustrate (i) that the users are unable to express their preferences about deleting their personal data, (ii) why it is difficult for application developers to support the deletion requests from the users, and (iii) why it is difficult to realize persistent deletes in a timely manner in live production servers.


\textit{Scenario 1}: \textit{Alice} is a user of a smart-home ecosystem, \textit{HomeComp}, which provides real-time services including video surveillance, remote temperature, and illumination control. 
\textit{Alice} enjoys the services of \textit{HomeComp}, but concerned about her personal data privacy, she wants \textit{HomeComp} to permanently delete all her data older than $30$ days on a rolling basis.  

\Paragraphqit{The problem}
Like most service providers, \textit{HomeComp}'s data model is built around the assumption of perpetual data retention; deletion of user data needs a human-in-the-loop that performs the necessary actions.
Thus, \textit{HomeComp} does not allow its user to request for rolling timestamp-based data deletion. 

\textit{Scenario 2}: \textit{StreamEra} is a company that provides real-time insights for data streams, and allows its users to request on-demand deletion of their personal data, as it is bound by the \emph{right to be forgotten}.
\textit{StreamEra} uses an SQL-based wrapper on top of its storage layer.

\Paragraphqit{The problem} 
While \textit{StreamEra} wants to serve its users by ensuring timely persistent deletion of their personal data, SQL does not provide support for such an operation.
Instead, the backend engineers implement the user-requested deletion functionality at the application level in an ad-hoc manner as it is not native to SQL.

\textit{Scenario 3}: A cloud-based online data analysis company \textit{ClouData}, stores user data using immutable files within its HTAP data store. 
\textit{ClouData} is bound by the \emph{right to be forgotten}, and thus, has to delete all user data that are older than $D$ days.

\Paragraphqit{The problem} As the data organization on disk is not based on the ingestion timestamp and aims to accelerate read queries, it uses the most frequently
queried attribute to partition. Hence, the objects qualifying for a timestamp-based
deletion may be dispersed within the data store. 
As in-place deletion is not supported due to immutability, state-of-the-art data stores periodically consolidate the entire data set to delete all invalid entries. 
Ensuring privacy via this approach is costly in terms of disk writes and overall accesses, and causes latency spikes leading to performance unpredictability. 

\Paragraph{Other Challenges of Logical Deletes} In addition to not complying with regulatory requirements, 
logical deletes may cause more hurdles. Specifically,
by retaining invalidated data (that should not be used anymore), a data company:

\begin{enumerate}
	\item Wastes storage space and energy on data that cannot exploit in any way. Further, data maintenance results in additional write amplification that wears off the underlying storage devices \cite{Athanassoulis2016}.
	\item Risks that a security leak will reveal user data that users expect to be deleted~\cite{Piper2022}.
	\item Hurts read performance, as its data management layer uses metadata and indexes for all data regardless of whether they are invalidated~\cite{Sarkar2020}.
\end{enumerate}






\newpage 
\section{Privacy Through Timely Deletion}
\label{sec:vision}


We now outline our vision toward developing deletion-aware data systems, which by design, are capable of \emph{deleting user data persistently, and in an efficient and timely manner}. 
Toward this, we introduce a new set of application level and system level tools that capture, transform, and realize the user-requirements for deletes. 

Figure~\ref{fig: vision} shows our four-layered approach. The first step is the
\textit{policy layer}, implemented by the governments, that enact specific clauses to protect
data privacy through deletion. The second layer is the \emph{requirements layer} that 
translates the regulations into application requirements. Next, we have the \emph{application
layer} that proposes the necessary changes in query languages to allow applications to
easily express their constraints. Finally, the \emph{system
layer} implements efficient means for data deletion and demonstrates regulation compliance.


\subsection{Requirements Layer}
\vspace{-0.05in}

\Paragraph{Challenge} 
This layer analyzes the regulations from the policy layer and categorizes the various requests the
user \emph{should be} able to make on the application layer. The impact of the newly enacted policies
is on various aspects including accountability (audit), security (protect data access), and right of access
(efficient accessing) \cite{Shah2019,Shastri2021}. In this work, we focus on \emph{storage limitation} 
(``data should not be stored beyond its purpose''), the \emph{right to be forgotten} (``find and
delete groups of data'')~\cite{Shah2019}, and how to transform them into concrete requirements.



\Paragraph{Types of Deletion Requests} 
We codify the two types of data deletion requests as requirements for (a)~retention-driven \emph{rolling 
deletion} and (b)~\emph{on-demand deletion} both with a timely constraint~\cite{Sarkar2020}, as illustrated 
in the second part of Figure~\ref{fig: vision}.

\Paragraphbit{Retention-driven deletes} 
In cases that the purpose of storing the data has expired, a rolling deletion should take place, which
will ensure that the underlying data management solutions persistently delete this data, based on a
pre-set \textit{retention duration}. This duration can be governed by legislation, the specific application,
or even user preference, hence it has to be tunable. To abide by the policies, the data management
layer has to permanently delete expired items within a specific timeframe, provided by the service-level
agreement (SLA) between the user and the service providers.

\Paragraphbit{Deletion on-demand} 
The regulations for deletes also allow users to submit on-demand deletion requests for any personal data,
upon which the service provider has to delete user data persistently. 
On-demand deletion requests can be submitted through an API provided by the service provider, and upon 
submission, all data for a user are purged persistently within a threshold period. 
This threshold for persistent deletion is also set by the provider following the regulations and is 
agreed upon in the form of an SLA-clause. 


\subsection{Application Layer}
\vspace{-0.05in}

\Paragraph{Challenge} With the deletion-related regulations translated to deletion requirements, the next step 
is to transform them into a format that is interpretable by the application layer. The interface of
data stores is typically declarative query languages (e.g., SQL, GraphQL, DMX, LINQ, and N1QL) that support expressing complex queries as well as
inserting new data, updates, and deletes. The missing link here
is that state-of-the-art query languages do not have support for data deletion based on retention and
does not have a way to express the timely deletion requirements. 


\Paragraph{Extending SQL}
Hence, to implement the deletion requirements, we propose an extension to SQL~\cite{Sarkar2022} that includes
support for timely deletion both in the data definition (DDL) and the data 
manipulation (DML) parts of the language, as summarized in the third part of 
Figure~\ref{fig: vision}. 
The objective of the SQL extension is three-fold. 

\begin{itemize} \vspace*{-0.5mm}
	\item[1.] To support \textit{retention-driven deletion}, we augment both the \texttt{CREATE TABLE} and \texttt{INSERT INTO} statements so that a relational table can be associated with a number of options for specific time-to-live (TTL). Every data object is bound to a specific TTL according to the application SLA or to user preference. 
	\item[2.] To ensure \textit{timely persistence of on-demand deletion requests}, we augment the \texttt{CREATE TABLE} and \texttt{DELETE FROM} statements to allow a relational table to support a predetermined set of timely deletion guarantees, and each deletion to select the level of service to which it adheres.
	\item[3.] Finally, we extend the \texttt{CREATE TABLE}, \texttt{INSERT INTO} and \texttt{DELETE FROM} statements to support \emph{arbitrary delete thresholds} for retention duration and deletion persistence.
\end{itemize}

 


\Paragraphbit{Enabling retention-driven deletes}
To support retention-driven deletes, we extend \texttt{CREATE TABLE} to allow an
application developer to specify several levels of retention duration as a table property.

\begin{verbnobox}[\fontsize{9pt}{10pt}\selectfont]
    CREATE TABLE R (column1 type1, column2 type2, ...) 
    WITH RET_DUR FIXED (t1 <ret1>, t2 <ret2>, ...);
\end{verbnobox}
The above \texttt{CREATE TABLE} statement creates a table \texttt{R} that supports 
retention-based deletes with specific retention duration of \texttt{ret1}, 
\texttt{ret2}, etc, which are mapped to symbolic representations \texttt{t1}, 
\texttt{t2}, etc.
In general, the \texttt{WITH RET\_DUR} clause is an optional clause when creating
a new table, and will be necessary only for tables that need to support deletes with 
predefined retention duration values. In such cases, each \texttt{INSERT} statement can use 
(up to) one of the predefined retention duration values, say \texttt{t1} to classify the 
specific object as one to be deleted after \texttt{ret1} time.
For example, a table that is configured to support retention duration of 30 days and 60 days (\texttt{CREATE TABLE R (...) WITH RET\_DUR FIXED (t1 '30 days', t2 '60 days');}), 
can only receive inserts with retention duration \texttt{t1} or \texttt{t2}.
An ingestion without a retention period explicitly mentioned, is kept perpetually 
following the logic of a classical insert. 
The syntax of an insert, now, has the 
optional \texttt{WITH RET\_DUR} clause as follows.
\begin{verbnobox}[\fontsize{9pt}{10pt}\selectfont]
 INSERT INTO R (val1, val2, ...) WITH RET_DUR t<i>;
\end{verbnobox}


\Paragraphbit{Support for arbitrary retention duration}
To support arbitrary retention duration, we further add the \texttt{ARBITRARY} 
keyword to both the \texttt{CREATE TABLE} and \texttt{INSERT} statements.
The support for arbitrary retention duration is necessary particularly for systems
in a distributed setting that replicate data across physical data stores in different 
geolocations, each bound by different regulatory requirements. 
The full syntax of the proposed SQL extension for retention-based deletion is below.
\begin{verbnobox}[\fontsize{9pt}{10pt}\selectfont]
 CREATE TABLE R (column1 type1, column2 type2, ...) 
 WITH RET_DUR {ARBITRARY | FIXED (t1 <ret1>, t2 <ret2>, ...)};
\end{verbnobox}
\begin{verbnobox}[\fontsize{9pt}{10pt}\selectfont]
 INSERT INTO R (val1, val2, ...) WITH RET_DUR { <t> | t<i> } ;
\end{verbnobox} 
Note that having a \emph{pre-defined set of retention duration values} provides more 
information to the system compared to allowing arbitrary duration. As a result, it 
\emph{allows the system to better prepare} to offer efficient retention-driven 
deletes. Conversely, we expect that data stores that aim to support arbitrary 
retention duration will face increased system-level challenges.


\Paragraphbit{Enabling timely on-demand deletion}
We further propose to augment SQL to express timely on-demand deletion. To do so, 
we introduce the concept of \textit{delete persistence threshold} 
(DPT)~\cite{Sarkar2020}, which denotes the maximum delay between a logical delete and 
its persistence. Every relational table can be associated with several such thresholds
that are defined from the legal constraints or based on user preference.
Similarly to retention-driven deletes, we also extend SQL to support arbitrary DPTs 
when the DPTs are not specified \textit{a priori}. 
Below, we outline the modifications to the DDL and DML parts of SQL to support on-demand timely deletion requests.
\begin{verbnobox}[\fontsize{9pt}{10pt}\selectfont]
 CREATE TABLE S (column1 type1, column2 type2, ...) 
 WITH DPT {ARBITRARY | FIXED (d1 <dpt1>, d2 <dpt2>, ...)};
\end{verbnobox}
\begin{verbnobox}[\fontsize{9pt}{10pt}\selectfont]
 DELETE FROM S WHERE (...) WITH DPT { <d> | d<i> };
\end{verbnobox}
Table \texttt{S} can support several DPTs (e.g., \texttt{dpt1}, \texttt{dpt2}) as long 
as the DPTs are pre-determined. Applications can trigger on-demand deletion 
with any such DPT through the \texttt{DELETE} command. 
Similarly to retention-driven deletes, timely persistent deletion of data on-demand is easier to handle from a storage engine if the DPTs supported are known \textit{a priori} during the table creation.

\Paragraph{Putting everything together}
With the proposed SQL extensions, a relational table can now support multiple 
(pre-defined or arbitrary) thresholds for both retention-based and on-demand deletes, 
with the following \texttt{CREATE TABLE} statement. 
\begin{verbnobox}[\fontsize{9pt}{10pt}\selectfont]
 CREATE TABLE T (column1 type1, column2 type2, ...) 
 WITH RET_DUR {ARBITRARY | FIXED (t1 <ret1>, t2 <ret2>, ...)};
 WITH DPT {ARBITRARY | FIXED (d1 <dpt1>, d2 <dpt2>, ...)};
\end{verbnobox}
Note that typically retention-based deletes come from the \emph{application requirements},  and 
on-demand deletion requests are issued \emph{by the user}. However, in both cases,
the deletes have to happen \emph{timely} as per the regulatory requirements.
The proposed changes in SQL are not enough to guarantee that the system will deliver on the
need for timely and persistent data deletion. Rather, they create the interface for data systems so that
need for timely deletion. Rather, they create the interface for data systems so that
the users and applications can express the deletion requests which is enforced by the regulations. The data deletion \emph{per se} is realized at the \textit{system level}, and we will next discuss advances and challenges on that front.


\subsection{System Layer}
\label{subsec:system-level}

With the requirement analysis and the declarative interface in place, the users
and the applications can express all the mandated deletion requests and the
underlying system is now tasked with implementing them. Before discussing the 
challenges of implementing timely retention-based and on-demand deletes, we discuss
the taxonomy of system-level deletes the application layer may initiate. In other
words, we want to understand what delete patterns may be generated at the system
layer. 

\subsubsection{Taxonomy of Deletes at the System Layer}
The behavior of a low-level delete operation depends on (i) the logical organization 
and physical layout of the data, and (ii) the attribute based on which the deletion
requests are issued. To better understand this \emph{delete design space}, we
classify different delete operations across two dimensions: (a) deletes on 
\emph{primary} vs. \emph{secondary} attributes, and (b) deletes based
on a single value of the delete attribute, termed \emph{point deletes}, vs. 
deletes on a range of the delete attribute, termed \emph{range 
deletes}~\cite{Sarkar2020}.
Table \ref{tab:summary} summarizes the state of the art in out-of-place
systems for different delete operations, their performance impact, and their 
at-large implications.

\begin{table}[h]
    \centering
   {
   \scriptsize
        \begin{tabular}{c|cc|cc}
        \toprule
        \multicolumn{1}{c}{\multirow{2}{*}{\begin{tabular}[c]{@{}c@{}}\textbf{Delete Workloads} \end{tabular}}}  & \multicolumn{2}{c}{\textbf{Primary Deletes}}   & \multicolumn{2}{c}{\textbf{Secondary Deletes}}                    \\
        \multicolumn{1}{c}{}                                                                    & \multicolumn{1}{c}{Point}        & \multicolumn{1}{c}{Range}   & \multicolumn{1}{c}{Point}        & \multicolumn{1}{c}{Range}    \\
        \midrule        
        \multirow{1}{*}{State-of-the-art}     
            & \multirow{1}{*}{insert point tombstones}    
            & \multirow{1}{*}{insert range tombstones}        
            & \multirow{2}{*}{not supported}  
            & \multirow{2}{*}{full-tree compaction}         
        \\  implementation & (\textit{logical}) 
            & (\textit{logical}) 
            & 
            & \\
        \midrule        
        \multirow{1}{*}{Point query}     
            & \multirow{1}{*}{search for key; stop}    
            & \multirow{1}{*}{search for key; compare fetched key}        
            & \multirow{1}{*}{N/A}  
            & \multirow{1}{*}{N/A}         \\
            path
            & if a tombstone is found
            & with the histogram (discard if invalidated)
            &
            & \\
        \midrule        
        \multirow{1}{*}{Range query}     
            & \multirow{1}{*}{merge qualifying sorted runs;}    
            & \multirow{1}{*}{merge qualifying sorted runs;}        
            & \multirow{1}{*}{N/A}  
            & \multirow{1}{*}{N/A}         \\
            path
            & discard on the fly if TS exist
            & check each value against histogram
            & 
            & \\
        \midrule        
        \multirow{4}{*}{Implications}     
            & \multirow{1}{*}{unbounded persistence latency}    
            & \multirow{1}{*}{unbounded persistence latency}        
            & \multirow{1}{*}{}  
            & \multirow{1}{*}{huge latency spikes}         
        \\  & high space amplification 
            & high space amplification
            & N/A
            & high write amplification
        \\  & high write amplification
            & high write amplification
            & 
            & superfluous reads from disk
        \\  & 
            & severely affects read performance 
            & 
            & \\
        \bottomrule
        \end{tabular}
   }
   \vspace{-0.1in}
    \caption{Implications of logical deletes on performance in state-of-the-art out-of-place data stores.} \label{tab:summary}
    \vspace{-0.1in}
\end{table}

\Paragraph{Primary Deletes} In out-of-place systems, data is ultimately organized based on a so-called
\emph{sort key} (for example, the key of the key-value pairs in an LSM-tree). The
sort key often is the primary key of the database, hence, a majority of delete 
operations can be expressed as deletes based on the sort key, or \emph{primary
deletes}. Note that even deletes on other attributes may be preferable to be
converted to primary deletes if there is a secondary index. Both point and range
primary deletes use the notion of a \emph{delete marker} or a \emph{tombstone}
that is inserted in the data collection (on the deleted sort key) and invalidates 
prior version of the key(s), and they are very common in real workloads~\cite{Cao2020}. 
Primary deletes can be triggered either by user activity (i.e., on-demand) or by automated processes (e.g., data migration). 

\Paragraphbit{Implications and Challenges} When an out-of-place system has files that contain
tombstones, then both point and range query paths are affected. Specifically, since the system
accesses files with decreasing age (i.e., the most recent ones first) when 
looking for a key (\emph{point query}), it will also be looking for tombstones. If a 
tombstone is found,
the search will terminate because all other (older) instances of that key are 
invalid. 
In the presence of range deletes, the implementation is more complex as it is hard
to use per-file range delete tombstones~\cite{Madan2018}. Instead, a database-wide histogram
of deleted ranges is maintained and every query compares against this histogram before 
proceeding~\cite{Sarkar2020}.
When considering \emph{range queries}, all the sorted files have to be merged 
and the deleted entries are discarded on the fly by comparing against the
visited point and range tombstones~\cite{Callaghan2020}. 

The implications of primary deletes are multi-fold. First, while out-of-place 
systems support deletes, any deletion is logical, and there is no \emph{a priori}
bound on the delete persistence latency. Second, by maintaining both the tombstones
and the invalid entries for arbitrarily long time, the systems pay in terms of 
increased space amplification. Thirdly, by reorganizing data including invalid 
entries and tombstones, we further pay in terms of increased write amplification.
Note that space and write amplification \cite{Athanassoulis2016,Dong2017} are two 
fundamental sources of cost when deploying data system. Finally, while range 
tombstones are used to offer the range delete functionality, they are rather 
cumbersome and impact the read performance severely~\cite{Callaghan2020,Madan2018}.

\Paragraph{Secondary Deletes} In some other cases, we may need to organize data 
based on a sort key, but we have a majority of deletes on a different attribute.
Note that if we have individual deletes on a different attribute, the most 
prudent approach is to guarantee that we have a secondary index and transform a
secondary delete in one (or more) primary point delete, hence in
Table~\ref{tab:summary}, we see the lack of support for point secondary deletes.  
However, in some cases, we may have long range deletes on a secondary attribute. 
For example, when working on a window of the most recent data we can 
repetitively delete data based on a timestamp. A similar case is the 
retention-based deletes introduced earlier. 

\Paragraphbit{Implications and Challenges} Secondary range deletes are not 
native in out-of-place systems, since the underlying data is organized based on
the sort key. While converting them to a collection of point primary deletes might
work in several cases, it will overload the system with tombstones. Instead, 
several systems opt to perform a full-database merging and re-writing periodically
to fulfill any secondary range deletion constraints they might have to follow. This approach leads
to significant write amplification, superfluous data accesses, and a large 
penalty in terms of latency spikes on the workload during this merging.

\Paragraph{From Delete Requirements to Delete Types} The delete taxonomy at the
system level helps us map the delete requirements to low-level data operations. 
A retention-driven deletion is typically modeled as a secondary range delete, and
if the delete range has few objects (i.e., low selectivity), it can be implemented as a collection of
primary point deletes. On the other hand, on-demand deletion is typically 
implemented as a primary point or range delete and there are several performance 
challenges to be addressed.

\subsubsection{Realizing Timely Deletes}
Timely data deletion while respecting the retention SLAs without hurting the system 
performance is a key challenge. The efficiency of deletion depends on the schema and 
the physical data layout, the data re-organization strategy, the workload, and the 
design of the storage engine. The right-most part of Figure~\ref{fig: vision}
outlines the design changes needed and the input parameters used to co-optimize the 
performance of a storage engine while ensuring timely delete persistence. We
now discuss how both classes of deletes can be realized efficiently through 
modifications in the design of out-of-place storage engines, focusing on 
LSM-bases storage engines.

\Paragraph{Realizing Primary Deletes}
An LSM-based storage engine implements a primary delete by inserting a tombstone
on the desired key (or key-range) with a DPT associated. The application-provided
DPT is an indicator for how long a logical delete may live before its persistence,
that is, before purging any invalidated versions of the key(s) under deletion. The
tombstone representation is augmented with additional metadata, e.g., an extra byte 
to account for $128$ possible different DPTs in the same table. 
During its natural course of data re-organization the storage engine checks for any 
data blocks (i.e., pages, files, or sorted runs) with an expired TTL and 
consolidates them to ensure timely persistence. This data consolidation in LSM-based
data stores is called \emph{compaction}~\cite{Sarkar2022a,Sarkar2021c} and is the process of
selecting some components of the database (files, sorted runs) to merge and discard
invalid entries. At any point of time an LSM-based system has several files that 
may be compacted (it can be in the order of several thousands) so the decision which files to 
compact is a crucial decision. In general, the decision is based on read query 
metrics, however, in Lethe~\cite{Sarkar2020} we propose a new approach that 
prioritizes compactions of files depending on the age of the tombstones they contain.
Specifically, we assign different TTLs based on the level of the underlying LSM-tree 
and when there is a tombstone with an expired TTL we select the file that contains
it for compaction, as shown in Figure~\ref{fig:ec}. A key decision is how to ensure
that the multi-step merging of tombstones will always respect the 
application-defined DPT. This is ensured by assigning a different TTL to each 
tombstone after every compaction in a way that the sum of all its TTL amounts to the 
desired DPT. 


\begin{figure}[t]
    \centering  
        \includegraphics[width=\textwidth]{figs/effacingcompaction.pdf} 
        \vspace{-0.25in}        
    \caption{FADE persists tombstones within DPT, thus, improving overall performance.}
    \label{fig:ec} 
\vspace{0.2in}        
\end{figure} 

\begin{figure}[t]
    \centering  
        \includegraphics[width=\textwidth]{figs/storage_layout.pdf} 
   \vspace{-0.25in}        
    \caption{KiWi stores data in an interweaved fashion on the sort and delete key to facilitate efficient secondary range deletes while offering competitive  read performance.}
    \label{fig:layout} 
   \vspace{-0.15in}        
\end{figure} 


\Paragraph{Realizing Secondary Deletes}
As we discussed above, several instances of secondary deletes can be realized as
a collection of primary point deletes. However, when we are frequently tasked to
deleted a range of values based on a secondary attribute, we can achieve something
significantly better. In particular, a new weaved data layout between the original
sort key and the (secondary) delete key can offer much more efficient and timely
secondary range deletion while maintaining competitive read performance. The key idea is
to create a nested data organization that alternates between organizing data
based on the sort key (to facilitate good search performance) and based on the
delete key (to allow for consecutive chunks of data to be deleted at a time). 

This approach is implemented in the KiWi data layout~\cite{Sarkar2020} as shown in
Figure~\ref{fig:layout}. The core idea is that while the major components of the
database (files) are organized based on the sort key, every file is composed of
delete tiles that are internally organized based on the delete key,
partitioning the data accordingly. Lastly, each data page is again organized on
the search key to facilitate efficient in-memory search. The benefit for this
weaved data layout is that in the case of secondary range deletes, we can discard
entire groups of pages at a time, signaling the file system to reclaim this page
instantly, essentially converting the secondary delete to a page reclamation action
that has very low latency compared to a full database reorganization. In the 
worst case, we will have to in-place edit a few pages at the edge of the range, 
which is a tunable parameter that controls the maximum secondary deletion 
persistence latency as a tradeoff vs. read performance. 

\Paragraph{Evaluation} 
The approaches presented above for timely deletion were implemented as part of the
LSM-based system Lethe~\cite{Sarkar2020}, and achieved efficient timely
deletion respecting predetermined guarantees. The left hand-side of Figure~\ref{fig:results} shows the CDF of the 
tombstone age while varying the desired DPT to 16\%, 25\%, and 50\% of the duration
of the experiment. The colored areas correspond to the number of cumulative 
tombstones for the corresponding age on the x-axis, while the horizontal dotted-line
is the desired DPT. We observe that Lethe was able to always deliver the
requested DPT. The gray area corresponds to the age of tombstones of the state of
the art, where no DPT is imposed and deletes are not persisted timely. Notably, we also measured that enforcing the 
desired DPT shows benefits in terms of access time because the amount of invalid 
data was reduced. Similarly, we saw benefits in space amplification, and only marginal
cost increase in amortized write amplification.

The right hand-side of Figure~\ref{fig:results} shows the fraction of fully dropped
pages during a range delete as we vary the size of the delete tiles. We observe
that the fraction of pages fully dropped increases with the delete tile size, allowing
for efficient reclamation of the invalid data. Conversely, the read queries become
more expensive as we allow for more page drops, so the ideal delete tile size should
be tuned based on the workload. 


\begin{figure}[t]
    \centering  
        \includegraphics[width=\textwidth]{figs/lethe_figures.pdf} 
   \vspace{-0.25in}        
    \caption{Lethe ensures timely persistence of logically invalidated data within LSM-based out-of-place data systems for both primary and secondary classes of deletes.}
    \label{fig:results} 
   \vspace{-0.15in}        
\end{figure} 

\newpage
\Paragraph{Deletes in a Complex Data Model} The previous discussion focuses on
handling deletes in a per-instance manner without considering multiple copies of the
data in a more complex setting. Cohn-Gordon \textit{et al.}~\cite{Cohn-Gordon2020} proposed the deletion 
framework \textit{DELF} that ensures reliable data deletion from an online social 
network (OSN). 
\textit{DELF} enables detection of inconsistent data deletion in OSNs and also 
facilitates data recovery in cases where user data was incorrectly deleted. 
Minaei \textit{et al.}~\cite{Minaei2019} proposed a framework for persistently deleting all instances 
of user data in presence of observers, thereby, ensuring privacy through timely 
content concealment and removal. 





 
\section{Challenges and Opportunities}
\label{sec:challenges}
\vspace{-0.075in}



In Section~\ref{sec:vision}, we outlined the steps taken to realize the four-layered
vision of delete-compliant data systems presented in Figure~\ref{fig: vision}, 
however, there are still open questions and challenges for such systems which pose
opportunities for further innovative systems research.



\Paragraph{Device-level Deletion}
Data management solutions rely on storage devices and treat them as black boxes.
However, deleting data persistently at the device level and from data archives is an open 
technological challenge. Current endeavors in this direction are mainly focused on 
encryption-based solutions~\cite{Kissel2014,Koppel2013,Li2019a}.
Nevertheless, retention-based deletes entail persistent deletion of a ``quantum'' of data 
(e.g., the data ingested in a day) posing the following challenges for 
encryption-based solutions.
First, it is hard to \textit{determine the encryption granularity} while minimizing 
the encrypt/decrypt overhead. Second, with several data streams for different users/applications (and thus, bound by different SLAs), 
it is hard to \textit{manage the encryption keys efficiently and in a scalable manner}.
Third, efficient and scalable \textit{deletion from archives and backup stores on-demand} is hard to be supported by encryption-based deletion as the encrypt/decrypt cost and the fine encryption granularity adds prohibitive overheads.
Finally, from a legislation point-of-view it is not yet clear whether encrypting
and discarding the key is an accepted form of deletion.

When considering a system-level deletion similar approach to the one presented in 
Section~\ref{subsec:system-level} storage devices are essentially one more level of 
managing data at the physical layer and similar approaches have to be implemented
in the file system or the file and data systems have to be developed in tandem.

\Paragraph{Cloud-Level Deletion}
Further, operating on the cloud, data systems use virtualized devices and
object storage which is even more abstract hiding the details of how the 
low-level device and page management is taking place. Offering guarantees for
timely data deletion in virtualized storage will require a similar multi-layered
approach where the file system and the device firmware will expose knobs to allow
the application on top to request specific page reclamation properties. 





\Paragraph{Deletion in Distributed/Federated Computing Environment} 
With more and more data stores being transformed to cloud-based stores, user data 
may be collected, processed, and stored across multiple domains, spread across 
different geographic locations~\cite{Sarkar2018}. 
With different geographic locations being bound by different privacy regulations, we 
need to design \textit{solutions to ensure consistency for persistent deletion} of 
user data. 
Our intuition is that existing solutions for data stream-tainting~\cite{Enck2010}, 
cross-domain data tracing~\cite{Demsky2011,Herbster2016}, and related data 
provenance solutions~\cite{Buneman2018,Hasan2019} can be useful to address this 
problem.



\Paragraph{Compliance Demonstration} 
Last but certainly not least, data systems have to be able to prove compliance
when audited. The natural way to do so now is via log auditing, however, a more
light-weight algorithmic way for providing this will benefit both systems and users.
Inspecting logs and the underlying data is a time-consuming process and the 
long-term goal of the community should be to design system-level tools that can 
verifiably prove compliance with the privacy regulations. 
One interesting development in this direction is the evolution of security-driven 
operating systems, such as seL4~\cite{Klein2010,SeL4}. 
Another approach that can be taken is to show that the codebase of a data system
has the necessary code-paths for timely deletion via static and dynamic analysis.
An open challenge is to develop static and dynamic analysis tools that can prove
that a system deletes data respecting the timely deletion requirements set.









 
\section{Conclusion}
\label{sec:conclusion}
\vspace{-0.075in}

In this paper, we highlight that the recently enacted regulations  mandate
new data deletion requirements, requiring a new breed of data systems to support
them. We show that existing state-of-the-art out-of-place systems are ill-equipped
for this task, and we present a four-layered approach towards building the necessary
infrastructure. We present recent work on that front, and we conclude by discussing
several open research challenges.

\Paragraph{Acknowledgments} This work was partially funded by National
Science Foundation under Grant No. IIS-1850202 and a Facebook
Faculty Research Award.




 

\begin{thebibliography}{10}
\itemsep=1pt
\begin{small}


  \bibitem{UKGDPR}
  {Right to erasure}.
  \newblock {\em
    https://ico.org.uk/for-organisations/guide-to-data-protection/guide-to-the-general-data-protection-regulation-gdpr/individual-rights/right-to-erasure/}.
  
  \bibitem{GDPR}
  {Regulation (EU) 2016/679 of the European Parliament and of the council of 27
    April 2016 on the protection of natural persons with regard to the processing
    of personal data and on the free movement of such data, and repealing
    Directive 95/46/EC}.
  \newblock {\em Official Journal of the European Union (Legislative Acts)},
    pages L119/1 -- L119/88, 2016.
  
  \bibitem{CCPA2018}
  {California Consumer Privacy Act}.
  \newblock {\em Assembly Bill No. 375, Chapter 55}, 2018.
  
  \bibitem{DPA2018}
  {Data Protection Act 2018}.
  \newblock {\em
    https://www.legislation.gov.uk/ukpga/2018/12/pdfs/ukpga{\_}20180012{\_}en.pdf},
    2018.
  
  \bibitem{PIPEDA2019}
  {PIPEDA in brief}.
  \newblock {\em
    https://www.priv.gc.ca/en/privacy-topics/privacy-laws-in-canada/the-personal-information-protection-and-electronic-documents-act-pipeda/pipeda{\_}brief/},
    2019.
  
  \bibitem{CPRA}
  {The California Privacy Rights Act of 2020}.
  \newblock {\em https://thecpra.org/}, 2020.
  
  \bibitem{VCPDA}
  {Virginia Consumer Data Protection Act}.
  \newblock {\em
    https://www.sullcrom.com/files/upload/SC-Publication-Virginia-Second-State-Enact-Privacy-Legislation.pdf},
    2021.
  
  \bibitem{AmazonRedshift}
  Amazon.
  \newblock {Redshift}.
  \newblock {\em https://aws.amazon.com/redshift/}.
  
  \bibitem{Ambrose2013}
  M.~L. Ambrose and J.~Ausloos.
  \newblock {The Right to Be Forgotten Across the Pond}.
  \newblock {\em Journal of Information Policy}, 3:1--23, 2013.
  
  \bibitem{ApacheAccumulo}
  Apache.
  \newblock {Accumulo}.
  \newblock {\em https://accumulo.apache.org/}.
  
  \bibitem{ApacheHBase}
  Apache.
  \newblock {HBase}.
  \newblock {\em http://hbase.apache.org/}.
  
  \bibitem{ApacheCassandra}
  Apache.
  \newblock {Cassandra}.
  \newblock {\em http://cassandra.apache.org}, 2021.
  
  \bibitem{Athanassoulis2011}
  M.~Athanassoulis, S.~Chen, A.~Ailamaki, P.~B. Gibbons, and R.~Stoica.
  \newblock {MaSM: Efficient Online Updates in Data Warehouses}.
  \newblock In {\em Proceedings of the ACM SIGMOD International Conference on
    Management of Data}, pages 865--876, 2011.
  
  \bibitem{Athanassoulis2015}
  M.~Athanassoulis, S.~Chen, A.~Ailamaki, P.~B. Gibbons, and R.~Stoica.
  \newblock {Online Updates on Data Warehouses via Judicious Use of Solid-State
    Storage}.
  \newblock {\em ACM Transactions on Database Systems (TODS)}, 40(1), 2015.
  
  \bibitem{Athanassoulis2016}
  M.~Athanassoulis, M.~S. Kester, L.~M. Maas, R.~Stoica, S.~Idreos, A.~Ailamaki,
    and M.~Callaghan.
  \newblock {Designing Access Methods: The RUM Conjecture}.
  \newblock In {\em Proceedings of the International Conference on Extending
    Database Technology (EDBT)}, pages 461--466, 2016.
  
  \bibitem{Bender2000}
  M.~A. Bender, E.~D. Demaine, and M.~Farach-Colton.
  \newblock {Cache-Oblivious B-Trees}.
  \newblock In {\em Proceedings of the Annual Symposium on Foundations of
    Computer Science (FOCS)}, pages 399--409, 2000.
  
  \bibitem{Bender2015}
  M.~A. Bender, M.~Farach-Colton, W.~Jannen, R.~Johnson, B.~C. Kuszmaul, D.~E.
    Porter, J.~Yuan, and Y.~Zhan.
  \newblock {An Introduction to B$\epsilon$-trees and Write-Optimization}.
  \newblock {\em White Paper}, 2015.
  
  \bibitem{Brown2016}
  C.~T. Brown and T.~D. Manoranjan.
  \newblock {South Korea Releases Guidance on Right to Be Forgotten}.
  \newblock {\em
    https://www.lexology.com/library/detail.aspx?g=21be3837-0c43-4047-b8b5-9e863960b0b9},
    2016.
  
  \bibitem{Brown2021}
  G.~A. Brown.
  \newblock {Consumers' "Right to Delete" under US State Privacy Laws}.
  \newblock {\em
    https://www.securityprivacybytes.com/2021/03/consumers-right-to-delete-under-us-state-privacy-laws/},
    2021.
  
  \bibitem{Buneman2018}
  P.~Buneman and W.-C. Tan.
  \newblock {Data Provenance: What next?}
  \newblock {\em SIGMOD Rec.}, 47(3):5--16, 2018.
  
  \bibitem{Callaghan2020}
  M.~Callaghan.
  \newblock {Deletes are fast and slow in an LSM}.
  \newblock {\em
    http://smalldatum.blogspot.com/2020/01/deletes-are-fast-and-slow-in-lsm.html},
    2020.
  
  \bibitem{Cao2020}
  Z.~Cao, S.~Dong, S.~Vemuri, and D.~H.~C. Du.
  \newblock {Characterizing, Modeling, and Benchmarking RocksDB Key-Value
    Workloads at Facebook}.
  \newblock In {\em Proceedings of the USENIX Conference on File and Storage
    Technologies (FAST)}, pages 209--223, 2020.
  
  \bibitem{Carter2013}
  E.~L. Carter.
  \newblock {Argentina's Right to be Forgotten}.
  \newblock {\em Emory International Law Review}, 27(1), 2013.
  
  \bibitem{Chang2006}
  F.~Chang, J.~Dean, S.~Ghemawat, W.~C. Hsieh, D.~A. Wallach, M.~Burrows,
    T.~Chandra, A.~Fikes, and R.~E. Gruber.
  \newblock {Bigtable: A Distributed Storage System for Structured Data}.
  \newblock In {\em Proceedings of the USENIX Symposium on Operating Systems
    Design and Implementation (OSDI)}, pages 205--218, 2006.
  
  \bibitem{Chik2013}
  W.~B. Chik.
  \newblock {The Singapore Personal Data Protection Act and an assessment of
    future trends in data privacy reform}.
  \newblock {\em Comput. Law Secur. Rev.}, 29(5):554--575, 2013.
  
  \bibitem{Cisco2018}
  Cisco.
  \newblock {Cisco Global Cloud Index: Forecast and Methodology, 2016–2021}.
  \newblock {\em White Paper}, 2018.
  
  \bibitem{Cohn-Gordon2020}
  K.~Cohn-Gordon, G.~Damaskinos, D.~Neto, J.~Cordova, B.~Reitz, B.~Strahs,
    D.~Obenshain, P.~Pearce, I.~Papagiannis, and A.~Media.
  \newblock {DELF: Safeguarding deletion correctness in Online Social Networks}.
  \newblock In {\em 29th USENIX Security Symposium, USENIX Security 2020, August
    12-14, 2020}, 2020.
  
  \bibitem{Dageville2016}
  B.~Dageville, T.~Cruanes, M.~Zukowski, V.~Antonov, A.~Avanes, J.~Bock,
    J.~Claybaugh, D.~Engovatov, M.~Hentschel, J.~Huang, A.~W. Lee, A.~Motivala,
    A.~Q. Munir, S.~Pelley, P.~Povinec, G.~Rahn, S.~Triantafyllis, and
    P.~Unterbrunner.
  \newblock {The Snowflake Elastic Data Warehouse}.
  \newblock In {\em Proceedings of the ACM SIGMOD International Conference on
    Management of Data}, pages 215--226, 2016.
  
  \bibitem{Databricks}
  Databricks.
  \newblock {Online reference}.
  \newblock {\em https://databricks.com/}.
  
  \bibitem{Databricks2021}
  Databricks.
  \newblock {Table deletes, updates, and merges}.
  \newblock {\em
    https://docs.databricks.com/delta/delta-update.html{\#}delete-from-a-table},
    2021.
  
  \bibitem{Dayan2017}
  N.~Dayan, M.~Athanassoulis, and S.~Idreos.
  \newblock {Monkey: Optimal Navigable Key-Value Store}.
  \newblock In {\em Proceedings of the ACM SIGMOD International Conference on
    Management of Data}, pages 79--94, 2017.
  
  \bibitem{DeCandia2007}
  G.~DeCandia, D.~Hastorun, M.~Jampani, G.~Kakulapati, A.~Lakshman, A.~Pilchin,
    S.~Sivasubramanian, P.~Vosshall, and W.~Vogels.
  \newblock {Dynamo: Amazon's Highly Available Key-value Store}.
  \newblock {\em ACM SIGOPS Operating Systems Review}, 41(6):205--220, 2007.
  
  \bibitem{Demsky2011}
  B.~Demsky.
  \newblock {Cross-application data provenance and policy enforcement}.
  \newblock {\em ACM Transactions on Information Systems (TOIS)},
    14(1):6:1----6:22, 2011.
  
  \bibitem{Deng2020}
  F.~Deng, Q.~Cao, S.~Wang, S.~Liu, J.~Yao, Y.~Dong, and P.~Yang.
  \newblock {SeRW: Adaptively Separating Read and Write upon SSDs of Hybrid
    Storage Server in Clouds}.
  \newblock In {\em Proceedings of the International Conference on Parallel
    Processing (ICPP)}, pages 76:1----76:11, 2020.
  
  \bibitem{Deshpande2018}
  A.~Deshpande and A.~Machanavajjhala.
  \newblock {ACM SIGMOD Blog: Privacy Challenges in the Post-GDPR World: A Data
    Management Perspective}.
  \newblock {\em http://wp.sigmod.org/?p=2554}, 2018.
  
  \bibitem{Dong2017}
  S.~Dong, M.~Callaghan, L.~Galanis, D.~Borthakur, T.~Savor, and M.~Strum.
  \newblock {Optimizing Space Amplification in RocksDB}.
  \newblock In {\em Proceedings of the Biennial Conference on Innovative Data
    Systems Research (CIDR)}, 2017.
  
  \bibitem{Enck2010}
  W.~Enck, P.~Gilbert, B.-G. Chun, L.~P. Cox, J.~Jung, P.~D. McDaniel, and
    A.~Sheth.
  \newblock {TaintDroid: An Information-Flow Tracking System for Realtime Privacy
    Monitoring on Smartphones}.
  \newblock In {\em Proceedings of the USENIX Symposium on Operating Systems
    Design and Implementation (OSDI)}, pages 393--407, 2010.
  
  \bibitem{FacebookRocksDB}
  Facebook.
  \newblock {RocksDB}.
  \newblock {\em https://github.com/facebook/rocksdb}, 2021.
  
  \bibitem{Farber2012}
  F.~F{\"{a}}rber, N.~May, W.~Lehner, P.~Gro{\ss}e, I.~M{\"{u}}ller, H.~Rauhe,
    and J.~Dees.
  \newblock {The SAP HANA Database -- An Architecture Overview}.
  \newblock {\em IEEE Data Engineering Bulletin}, 35(1):28--33, 2012.
  
  \bibitem{Gartner2017}
  Gartner.
  \newblock {Gartner Says 8.4 Billion Connected ``Things" Will Be in Use in 2017,
    Up 31 Percent From 2016}.
  \newblock https://tinyurl.com/Gartner2020, 2017.
  
  \bibitem{Goddard2017}
  M.~Goddard.
  \newblock {The EU General Data Protection Regulation (GDPR): European
    Regulation that has a Global Impact}.
  \newblock {\em International Journal of Market Research}, 59(6):703--705, 2017.
  
  \bibitem{Golan-Gueta2015}
  G.~Golan-Gueta, E.~Bortnikov, E.~Hillel, and I.~Keidar.
  \newblock {Scaling Concurrent Log-Structured Data Stores}.
  \newblock In {\em Proceedings of the ACM European Conference on Computer
    Systems (EuroSys)}, pages 32:1--32:14, 2015.
  
  \bibitem{GoogleLevelDB}
  Google.
  \newblock {LevelDB}.
  \newblock {\em https://github.com/google/leveldb/}, 2021.
  
  \bibitem{Gupta2015}
  A.~Gupta, D.~Agarwal, D.~Tan, J.~Kulesza, R.~Pathak, S.~Stefani, and
    V.~Srinivasan.
  \newblock {Amazon Redshift and the Case for Simpler Data Warehouses}.
  \newblock In {\em Proceedings of the ACM SIGMOD International Conference on
    Management of Data}, pages 1917--1923, 2015.
  
  \bibitem{Hasan2019}
  S.~S. Hasan, N.~H. Sultan, and F.~A. Barbhuiya.
  \newblock {Cloud Data Provenance using IPFS and Blockchain Technology}.
  \newblock In {\em Proceedings of the International Workshop on Security in
    Cloud Computing (SCC)}, pages 5--12, 2019.
  
  \bibitem{Heman2010}
  S.~H{\'{e}}man, M.~Zukowski, and N.~J. Nes.
  \newblock {Positional Update Handling in Column Stores}.
  \newblock In {\em Proceedings of the ACM SIGMOD International Conference on
    Management of Data}, pages 543--554, 2010.
  
  \newpage
  \bibitem{Herbster2016}
  R.~Herbster, S.~DellaTorre, P.~Druschel, and B.~Bhattacharjee.
  \newblock {Privacy Capsules: Preventing Information Leaks by Mobile Apps}.
  \newblock In {\em Proceedings of the 14th Annual International Conference on
    Mobile Systems, Applications, and Services, MobiSys 2016, Singapore, June
    26-30, 2016}, pages 399--411, 2016.
  
  \bibitem{Huang2019}
  G.~Huang, X.~Cheng, J.~Wang, Y.~Wang, D.~He, T.~Zhang, F.~Li, S.~Wang, W.~Cao,
    and Q.~Li.
  \newblock {X-Engine: An Optimized Storage Engine for Large-scale E-commerce
    Transaction Processing}.
  \newblock In {\em Proceedings of the ACM SIGMOD International Conference on
    Management of Data}, pages 651--665, 2019.
  
  \bibitem{Idreos2019}
  S.~Idreos, N.~Dayan, W.~Qin, M.~Akmanalp, S.~Hilgard, A.~Ross, J.~Lennon,
    V.~Jain, H.~Gupta, D.~Li, and Z.~Zhu.
  \newblock {Design Continuums and the Path Toward Self-Designing Key-Value
    Stores that Know and Learn}.
  \newblock In {\em Proceedings of the Biennial Conference on Innovative Data
    Systems Research (CIDR)}, 2019.
  
  \bibitem{Idreos2012}
  S.~Idreos, F.~Groffen, N.~Nes, S.~Manegold, K.~S. Mullender, and M.~L. Kersten.
  \newblock {MonetDB: Two Decades of Research in Column-oriented Database
    Architectures}.
  \newblock {\em IEEE Data Engineering Bulletin}, 35(1):40--45, 2012.
  
  \bibitem{Jannen2015}
  W.~Jannen, J.~Yuan, Y.~Zhan, A.~Akshintala, J.~Esmet, Y.~Jiao, A.~Mittal,
    P.~Pandey, P.~Reddy, L.~Walsh, M.~A. Bender, M.~Farach-Colton, R.~Johnson,
    B.~C. Kuszmaul, and D.~E. Porter.
  \newblock {BetrFS: A Right-optimized Write-optimized File System}.
  \newblock In {\em Proceedings of the USENIX Conference on File and Storage
    Technologies (FAST)}, pages 301--315, 2015.
  
  \bibitem{Jones2012}
  M.~L. Jones.
  \newblock {It's About Time: Privacy, Information Lifecycles, and the Right to
    Be Forgotten}.
  \newblock {\em Stanford Technology Law Review}, 16(2):54, 2012.
  
  \bibitem{Kang2016}
  W.-H. Kang, S.-W. Lee, and B.~Moon.
  \newblock {Flash as cache extension for online transactional workloads}.
  \newblock {\em The VLDB Journal}, 25(5):673--694, 2016.
  
  \bibitem{Kissel2014}
  R.~Kissel, A.~Regenscheid, M.~Scholl, and K.~Stine.
  \newblock {Guidelines for Media Sanitization}.
  \newblock {\em NIST Special Publication 800-88}, 2014.
  
  \bibitem{Kittane2021}
  P.~Kittane, I.~S. Charles, A.~Kamath, and G.~Gokhale.
  \newblock {Privacy and Data Protection -- India Wrap 2020}.
  \newblock {\em The National Law Review}, XI(162), 2021.
  
  \bibitem{Klein2010}
  G.~Klein, J.~Andronick, K.~Elphinstone, G.~Heiser, D.~Cock, P.~Derrin,
    D.~Elkaduwe, K.~Engelhardt, R.~Kolanski, M.~Norrish, T.~Sewell, H.~Tuch, and
    S.~Winwood.
  \newblock {seL4: formal verification of an operating-system kernel}.
  \newblock {\em Communications of the ACM}, 53(6):107--115, 2010.
  
  \bibitem{Koppel2013}
  B.~K{\"{o}}ppel and S.~Neuhaus.
  \newblock {Analysis of a hardware security module's high-availability setting}.
  \newblock {\em IEEE Security Privacy}, 11(3):77--80, 2013.
  
  \bibitem{Kuszmaul2014}
  B.~C. Kuszmaul.
  \newblock {A Comparison of Fractal Trees to Log-Structured Merge (LSM) Trees}.
  \newblock {\em Tokutek White Paper}, 2014.
  
  \bibitem{Lamb2012}
  A.~Lamb, M.~Fuller, and R.~Varadarajan.
  \newblock {The Vertica Analytic Database: C-Store 7 Years Later}.
  \newblock {\em Proceedings of the VLDB Endowment}, 5(12):1790--1801, 2012.
  
  \bibitem{Li2019a}
  B.~Li and D.~H.~C. Du.
  \newblock {TASecure: Temperature-Aware Secure Deletion Scheme for Solid State
    Drives}.
  \newblock In {\em Proceedings of the Great Lakes Symposium on VLSI (GLSVLSI)},
    pages 275--278, 2019.
  
  \bibitem{LinkedInVoldemort}
  LinkedIn.
  \newblock {Voldemort}.
  \newblock {\em http://www.project-voldemort.com}.
  
  \bibitem{Luo2020b}
  C.~Luo and M.~J. Carey.
  \newblock {LSM-based Storage Techniques: A Survey}.
  \newblock {\em The VLDB Journal}, 29(1):393--418, 2020.
  
  \bibitem{Madan2018}
  A.~Madan and A.~Kryczka.
  \newblock {DeleteRange: A New Native RocksDB Operation}.
  \newblock {\em https://rocksdb.org/blog/2018/11/21/delete-range.html}, 2018.
  
  \bibitem{Piper2022}
  R.~McKean, E.~Kurowska-Tober, and H.~Waem.
  \newblock {DLA Piper GDPR fines and data breach survey: January 2022}.
  \newblock {\em
    https://www.dlapiper.com/en/us/insights/publications/2022/1/dla-piper-gdpr-fines-and-data-breach-survey-2022/},
    2022.
  
  \bibitem{Minaei2019}
  M.~Minaei, M.~Mondal, P.~Loiseau, K.~P. Gummadi, and A.~Kate.
  \newblock {Lethe: Conceal Content Deletion from Persistent Observers}.
  \newblock {\em Proceedings on Privacy Enhancing Technologies (PoPET)},
    2019(1):206--226, 2019.
  
  \bibitem{ONeil1996}
  P.~E. O'Neil, E.~Cheng, D.~Gawlick, and E.~J. O'Neil.
  \newblock {The log-structured merge-tree (LSM-tree)}.
  \newblock {\em Acta Informatica}, 33(4):351--385, 1996.
  
  \bibitem{Papadopoulos2016}
  S.~Papadopoulos, K.~Datta, S.~Madden, and T.~Mattson.
  \newblock {The TileDB Array Data Storage Manager}.
  \newblock {\em Proceedings of the VLDB Endowment}, 10(4):349--360, 2016.
  
  \bibitem{Paradigm4}
  Paradigm4.
  \newblock {Online reference}.
  \newblock {\em https://www.paradigm4.com/}.
  
  \bibitem{Pardo2020}
  D.~Pardo.
  \newblock {First Decision on the "Right to be Forgotten" in Argentina}.
  \newblock {\em
    https://scholarlycommons.law.emory.edu/cgi/viewcontent.cgi?article=1097{\&}context=eilr},
    2020.
  
  \bibitem{DLAPiper2020}
  D.~Piper.
  \newblock {GDPR Data Breach Survey 2020}.
  \newblock {\em
    https://www.dlapiper.com/en/us/insights/publications/2020/01/gdpr-data-breach-survey-2020/},
    2020.
  
  \bibitem{Sadoghi2016}
  M.~Sadoghi, K.~A. Ross, M.~Canim, and B.~Bhattacharjee.
  \newblock {Exploiting SSDs in operational multiversion databases}.
  \newblock {\em The VLDB Journal}, 25(5):651--672, 2016.
  
  \bibitem{Sarkar2022a}
  S.~Sarkar, K.~Chen, Z.~Zhu, and M.~Athanassoulis.
  \newblock {Compactionary: A Dictionary for LSM Compactions}.
  \newblock In {\em Proceedings of the ACM SIGMOD International Conference on
    Management of Data}, 2022.
  
  \bibitem{Sarkar2022b}
  S.~Sarkar and M.~Athanassoulis.
  \newblock {Dissecting, Designing, and Optimizing LSM-based Data Stores}.
  \newblock In {\em Proceedings of the ACM SIGMOD International Conference on
    Management of Data}, 2022.
  
  \bibitem{Sarkar2022}
  S.~Sarkar and M.~Athanassoulis.
  \newblock {Query Language Support for Timely Data Deletion}.
  \newblock In {\em Proceedings of the International Conference on Extending
    Database Technology (EDBT)}, 2022.
  
  \bibitem{Sarkar2018}
  S.~Sarkar, J.-P. Ban{\^{a}}tre, L.~Rilling, and C.~Morin.
  \newblock {Towards Enforcement of the EU GDPR: Enabling Data Erasure}.
  \newblock In {\em Proceedings of the IEEE International Conference of Internet
    of Things (iThings)}, pages 1--8, 2018.
  
  \bibitem{Sarkar2020}
  S.~Sarkar, T.~I. Papon, D.~Staratzis, and M.~Athanassoulis.
  \newblock {Lethe: A Tunable Delete-Aware LSM Engine}.
  \newblock In {\em Proceedings of the ACM SIGMOD International Conference on
    Management of Data}, pages 893--908, 2020.
  
  \bibitem{Sarkar2021c}
  S.~Sarkar, D.~Staratzis, Z.~Zhu, and M.~Athanassoulis.
  \newblock {Constructing and Analyzing the LSM Compaction Design Space}.
  \newblock {\em Proceedings of the VLDB Endowment}, 14(11):2216--2229, 2021.
  
  \bibitem{Schwarzkopf2019}
  M.~Schwarzkopf, E.~Kohler, M.~F. Kaashoek, and R.~T. Morris.
  \newblock {Position: GDPR Compliance by Construction}.
  \newblock In {\em Selected Papers from VLDB Workshop on Polystore Systems for
    Heterogeneous Data in Multiple Databases with Privacy and Security Assurances
    (POLY)}, volume 11721 of {\em Lecture Notes in Computer Science}, pages
    39--53, 2019.
  
  \bibitem{Sears2012}
  R.~Sears and R.~Ramakrishnan.
  \newblock {bLSM: A General Purpose Log Structured Merge Tree}.
  \newblock In {\em Proceedings of the ACM SIGMOD International Conference on
    Management of Data}, pages 217--228, 2012.
  
  \bibitem{SeL4}
  SeL4.
  \newblock {Online reference}.
  \newblock {\em https://sel4.systems}.
  
  \bibitem{Shah2019}
  A.~Shah, V.~Banakar, S.~Shastri, M.~Wasserman, and V.~Chidambaram.
  \newblock {Analyzing the Impact of GDPR on Storage Systems}.
  \newblock In {\em Proceedings of the USENIX Conference on Hot Topics in Storage
    and File Systems (HotStorage)}, 2019.
  
  \bibitem{Shastri2020}
  S.~Shastri, V.~Banakar, M.~Wasserman, A.~Kumar, and V.~Chidambaram.
  \newblock {Understanding and Benchmarking the Impact of GDPR on Database
    Systems}.
  \newblock {\em Proceedings of the VLDB Endowment}, 13(7):1064--1077, 2020.
  
  \bibitem{Shastri2019}
  S.~Shastri, M.~Wasserman, and V.~Chidambaram.
  \newblock {The Seven Sins of Personal-Data Processing Systems under GDPR}.
  \newblock In {\em Proceedings of USENIX Workshop on Hot Topics in Cloud
    Computing (HotCloud)}, 2019.
  
  \bibitem{Shastri2021}
  S.~Shastri, M.~Wasserman, and V.~Chidambaram.
  \newblock {GDPR anti-patterns}.
  \newblock {\em Communications of the ACM}, 64(2):59--65, 2021.
  
  \bibitem{Stonebraker2005}
  M.~Stonebraker, D.~J. Abadi, A.~Batkin, X.~Chen, M.~Cherniack, M.~Ferreira,
    E.~Lau, A.~Lin, S.~R. Madden, E.~J. O'Neil, P.~E. O'Neil, A.~Rasin, N.~Tran,
    and S.~Zdonik.
  \newblock {C-Store: A Column-oriented DBMS}.
  \newblock In {\em Proceedings of the International Conference on Very Large
    Data Bases (VLDB)}, pages 553--564, 2005.
  
  \bibitem{Stonebraker2013}
  M.~Stonebraker, J.~Duggan, L.~Battle, and O.~Papaemmanouil.
  \newblock {SciDB DBMS Research at M.I.T}.
  \newblock {\em IEEE Data Eng. Bull.}, 36(4):21--30, 2013.
  
  \bibitem{Tessian2022}
  Tessian.
  \newblock {25 Biggest GDPR Fines So Far (2019, 2020, 2021, 2022)}.
  \newblock {\em https://www.tessian.com/blog/biggest-gdpr-fines-2020/}, 2022.
  
  \bibitem{TileDB}
  TileDB.
  \newblock {Online reference}.
  \newblock {\em https://tiledb.io}.
  
  \bibitem{Tsesis2014}
  A.~Tsesis.
  \newblock {The Right to be Forgotten and Erasure: Privacy, Data Brokers, and
    the Indefinite Retention of Data}.
  \newblock {\em Wake Forest Law Review}, 48:51, 2014.
  
  \bibitem{Whittaker2019}
  Z.~Whittaker and N.~Lomas.
  \newblock {Even years later, Twitter doesn't delete your direct messages}.
  \newblock {\em https://techcrunch.com/2019/02/15/twitter-direct-messages/},
    2019.
  
  \bibitem{Yang2020}
  L.~Yang, H.~Wu, T.~Zhang, X.~Cheng, F.~Li, L.~Zou, Y.~Wang, R.~Chen, J.~Wang,
    and G.~Huang.
  \newblock {Leaper: A Learned Prefetcher for Cache Invalidation in LSM-tree
    based Storage Engines}.
  \newblock {\em Proceedings of the VLDB Endowment}, 13(11):1976--1989, 2020.
  
  \bibitem{Zheng2018}
  Q.~Zheng, C.~D. Cranor, D.~Guo, G.~R. Ganger, G.~Amvrosiadis, G.~A. Gibson,
    B.~W. Settlemyer, G.~Grider, and F.~Guo.
  \newblock {Scaling Embedded In-Situ Indexing with DeltaFS}.
  \newblock In {\em Proceedings of the ACM/IEEE International Conference for High
    Performance Computing, Networking, Storage and Analysis (SC)}, pages
    3:1--3:15, 2018.

\end{small}
\end{thebibliography}


\end{document}

\end{article}

\begin{article}
{Provenance-based Model Maintenance: Implications for Privacy}
{Yinjun Wu, Val Tannen, Susan B. Davidson}
\graphicspath{{submissions/provenance-based-model-maintenance-implications-for-privacy/}}
% link to instruction: https://tc.computer.org/tcde/tcde-bulletin-author-instructions/
% \documentclass[11pt,dvipdfm]{article}
\documentclass[11pt]{article}
\usepackage{tabularx}
\usepackage{ragged2e}  % for '\RaggedRight' macro (allows hyphenation)
\usepackage{booktabs}  % for \toprule, \midrule, and \bottomrule macros
\usepackage{textcomp}
\usepackage{amsfonts,amsmath}
\usepackage{deauthor,times}
\usepackage{graphicx} % 
\usepackage{hyperref}
\usepackage{comment}
\graphicspath{{asudeh/}}
\usepackage{soul}
\usepackage{subcaption}
\usepackage{ulem}
\usepackage{wrapfig}
\usepackage{color}
\usepackage{xspace}
\newtheorem{problem}{Problem}

%\DeclareMathOperator*{\argmax}{arg\,max}

%remove the following commands/package b4 submission
\newcommand{\hide}[1]{}
\newcommand{\eat}[1]{}
\newcommand{\resolved}[1]{\hide{#1}}
\newcommand{\abol}[1]{\textcolor{red}{Abol: #1}}
\newcommand{\mahdi}[1]{\textcolor{red}{Mahdi: #1}}
\newcommand{\nima}[1]{\textcolor{red}{Nima: #1}}

\newcommand{\dee}{\mathcal{D}}
\newcommand{\Gee}{\mathcal{G}}
\newcommand{\gee}{\mathbf{g}}
\newcommand{\ee}{\mathbf{e}}
\newcommand{\es}{\mathcal{S}}
\newcommand{\el}{\mathcal{L}}
\newcommand{\xx}{\mathcal{x}}
\newcommand{\dist}{\xi}
\newcommand{\alg}{\mathsf{A}}
\newcommand{\qu}{\mathbf{q}}
\newcommand{\ex}{\mathbf{x}}
\newcommand{\ti}{\mathbf{t}}
\newcommand{\sdt}{\mathsf{SDT}}
\newcommand{\wdt}{\mathsf{WDT}}
\newcommand{\Qu}{\mathbf{Q}}
\newcommand{\pe}{\mathbb{P}}
\newcommand{\megam}{\mathcal{M}}
\newcommand{\eps}{\varepsilon}
\newcommand{\enet}{{$\varepsilon$-{\bf net}}\xspace}
\newcommand{\net}{{\tt net}\xspace}
\newcommand{\vcd}{VC-dimension\xspace}
\newcommand{\at}[1]{{\tt \small #1}\xspace}
\newcommand{\pr}{Pr}

\newcommand{\sharpP}{\mbox{\#P}}
\newcommand{\NP}{\mathsf{NP}}
\newcommand{\LP}{\mathsf{LP}}
\newcommand{\IP}{\mathsf{IP}}
\newcommand{\ru}{{\sc {RU}}\xspace}
\newcommand{\sru}{{\sc {strongRU}}\xspace}
\newcommand{\wru}{{\sc {weakRU}}\xspace}

\newcommand{\fmsystem}{{\sc Chameleon}\xspace}
\newcommand{\fm}{$\mathcal{F}$\xspace}

\newtheorem{experiment}{Experiment}

\begin{document}

\title{Coverage-based Data-centric Approaches for \\Responsible and Trustworthy AI\thanks{This research was supported by the National Science Foundation under grant No. 2107290.}}

\author{
\begin{tabular}[t]{c@{\extracolsep{2.4em}}c@{\extracolsep{2.4em}}c@{\extracolsep{2.3em}}c} 
Nima Shahbazi & Mahdi Erfanian & Abolfazl Asudeh \\ 
University of Illinois Chicago & University of Illinois Chicago & University of Illinois Chicago\\
 nshahb3@uic.edu & merfan2@uic.edu & asudeh@uic.edu
\end{tabular}
}

\maketitle


\begin{abstract}
The grand goal of data-driven decision systems is to help make decisions easier, more accurate, at a higher scale, and also just. However, data-driven algorithms are only as good as the data they work with. Yet, data sets, especially those with social data, often do not represent minorities. The paucity of training data is a perpetual problem for AI, and the outcome of ML models for cases not represented in their training data is often not reliable. 
Hence, without properly addressing the lack of representation issues in data, we cannot expect AI-based societal solutions to have responsible and trustworthy outcomes. 

This paper focuses on data coverage as a data-centric approach for identifying and resolving misrepresentation of minorities in data.
To achieve this goal, we propose novel algorithms that (a) {\it identify} and {\it resolve} insufficient data coverage across data with different modalities and (b) use lack of representation information to generate data-centric {\it reliability warnings}.
 \end{abstract}
 
 %%%%%%%%%%%%%%%%%%%%%%%%%%%%%%%% INTRO  %%%%%%%%%%%%%%%%%%%%%%%%%%%%%%%%
\section{Introduction}\label{sec:intro} % Abstract+Intro: up to 2.5 pages 
Data-driven decision-making has shaped every corner of human life, spanning from autonomous vehicles to healthcare and even predictive policing and criminal justice. A pivotal concern, especially in applications that affect individuals, revolves around the reliability of the decisions rendered by the system.
It is easy to see that the accuracy of a data-driven decision depends, first and foremost, on the data used to make it. Essentially, the system learns the phenomena that data represent. While we may desire that the data should represent the underlying data distribution from which the production data is drawn, this alone may be insufficient, as it merely enables the model to perform well for the average case.
As a result, a model with a high accuracy could fail for specific regions in the data with insufficient representation. These regions may matter because they frequently represent some minority population in society. They could also represent cases that may not happen very often but have a relevant impact on the correctness of a critical decision.
In short, if the data fails to sufficiently represent a specific population, the outcome of the decision system for that population may not be trustworthy.

The phenomenon known as \textit{Representation Bias} can arise from how the data was originally collected, or it could be the result of biases introduced post-collection—whether historically, cognitively, or statistically.

Representation bias is essentially inevitable without a systematic approach to data collection. 
For example, in the context of survey data collection, vital steps involve identifying all populations within the underlying distribution based on desired demographic information and ensuring comprehensive coverage with sufficient samples from each group. 
Even then, only an (uncontrolled) subset of the invitees will opt-in to respond to the survey.
Another challenge lies in the fact that data scientists often lack control over the data collection process, leading to the reliance on ``found data'' in the majority of data-driven systems. Therefore, with no guarantee on the aforementioned steps in the data collection process, the found data is most likely a biased sample.
Acknowledging the potential harms of representation bias, the notion of \textit{Data Coverage}~\cite{asudeh2019assessing,shahbazi2023representation} has been proposed to ensure the adequate representation of minority groups in data sets employed for decision-making and developing sophisticated data science tools. 

Addressing representation issues in data poses various challenges depending on the modality of the data. In this paper, we focus on identifying and resolving lack of coverage issues in data with different modalities.
We start by proposing a variety of techniques (spanning from geometric and combinatorial optimization to crowd-souring) aimed at efficiently detecting insufficient coverage on structured data sets with non-ordinal categorical and continuous attributes, as well as image data sets. Next, we propose a range of approaches grounded in data integration and generative data augmentation to address the lack of coverage by enriching the data sets with more data. However, with limited control over the data collection processes, it could be difficult and expensive to resolve all misrepresentations. 
Since adding more data is not always possible, we proceed to introduce data-centric preventive solutions that warn the user about the reliability of their predictions regarding representation bias issues. These warnings assist users in determining whether they trust the outcomes of the models or exercise caution. 

 %%%%%%%%%%%%%%%%%%%%%%%%%%%%%%%% IDENTIFICATION  %%%%%%%%%%%%%%%%%%%%%%%%%%%%%%%%
\section{Detecting Insufficient Representation of Minorities}\label{sec:identification} %up to 3.5 pages
Representation bias happens when the development (training data) population under-represents 
and subsequently fails to generalize well 
for some parts of the target population, due to historical bias, sampling bias, etc.
The notion of {\it data coverage} has been studied across different settings in \cite{shahbazi2023representation} as a metric to measure representation bias. At a high level, coverage is referred to as having enough similar entries for each object in a data set. 
For a better understanding, let us go over the definition of the generalized notion of coverage:

\begin{definition}[Data Coverage]\label{def:coverage}
Consider a data set $\dee$ with $n$ tuples, each consisting of $d$ attributes of interest $\mathbf{x}=\{x_1, x_2, \cdots,x_d\}$, such as {\tt gender}, {\tt race}, {\tt salary}, {\tt age}, etc, that are used for coverage identification.
The data set also contains target attributes $\mathbf{y} = \{ y_1,\cdots,y_{d'}\}$ that may or may not be considered for the coverage problem.
A query point $q$ is not covered by the data set $\dee$, if there are not ``enough'' data points in $\dee$ that are representative of $q$.
To generalize the notion of coverage, let us define $\gee(q)$ as the universe of tuples that would represent $q$ and let $\gee_\dee(q) = \gee(q)\cap \dee$. In other words, $\gee_\dee(q)$ are the set of tuples in $\dee$ that represent $q$.
Using this notation, we define the coverage of $q$ as the size of $\gee_\dee(q)$. That is,
$cov(q,\dee) = | \gee_\dee(q)|$.
Given a value $\tau$, $q$ is covered if $cov(q,\dee)>\tau$.
Similarly, a group $\gee$ is not covered if $\gee\cap \dee<\tau$.
The {\it uncovered region} in a data set is the collection of groups that are not covered by it.
\end{definition}

\subsection{Structured Data}
In this section, we focus on identifying representation bias in structured data.
Depending on the type of the attributes of interest, we categorize the techniques into two classes based on whether they target the problem for non-ordinal {\it categorical} (e.g. {\tt race}, {\tt gender}) or ordinal {\it continuous} (e.g. {\tt age}) attributes. The attributes of interest considered for representation bias often include sensitive attributes such as {\tt race} and {\tt gender} but are not necessarily limited to them.

\subsubsection{Categorical Attributes}

For cases where attributes of interest are non-ordinal categorical,
the cartesian product of values on a subset of attributes $\mathbf{x}'\subseteq \mathbf{x}$, form a set of (sub-)groups.
For example, $\{$ {\tt white male}, {\tt white female}, {\tt black male} $,\cdots\}$ are the subgroups defined on the attributes {\tt (race,gender)}.
We refer to the number of attributes used to specify a subgroup as the {\it level} of that subgroup.
For example, the level of the subgroup {\tt white male} is 2, while the level of the subgroup {\tt male} is 1.
We use $\ell(\gee)$, to refer to the level of a subgroup $\gee$.
Similarly, we say a subgroup $\gee'$ is a subset of $\gee$, if the groups specifying $\gee'$ are a superset of the ones for $\gee$. For example {\tt (married white male)} a subset of the more general group {\tt (white male)}. That is, the set of individuals in group {\tt (married white male)} are a subset of {\tt (white male)}.
Moreover, we say a subgroup $\gee$ is a {\it parent} of the subgroup $\gee'$, if $\gee'\subset \gee$ and $\ell(\gee)=\ell(\gee')+1$. For example, the subgroup {\tt (white male)} is a parent of the subgroup {\tt (married white male)}.
We use \textit{patterns} to refer to uncovered subgroups.
A pattern $P$ is a string of $d$ values, where $P[i]$ is either a value from the domain of $x_i$, or it is ``unspecified'', specified with $X$. 
For example, consider a data set with three binary attributes of interest $\mathbf{x}=\{x_1, x_2, x_3\}$. The pattern $P=X01$ specifies all the tuples for which $x_2=0$ and $x_3=1$ ($x_1$ can have any value).
The set of patterns that identify most general uncovered subgroups are called {\it Maximal Uncovered Patterns} (MUPs).

No polynomial time algorithm can guarantee the enumeration of the entire MUPs, however, several algorithms inspired by set enumeration and the Apriori algorithm for association rule mining are proposed to efficiently address this problem~\cite{asudeh2019assessing}.
In this regard, we introduce \textit{Pattern Graph} data structure that exploits the relationship between patterns to do less work than computing all uncovered patterns by removing the non-maximal ones. 
The parent-child relationship between the patterns is represented in a graph that can be used to find better algorithms. 
\textit{Pattern-Breaker} starts from the top of the graph where the general patterns are and moves down by breaking each pattern into more specific ones. If a pattern is uncovered, then all of its descendants are also uncovered and they can not be an MUP, even if they have a parent that is covered. Therefore, this subgraph of the pattern graph can be pruned. 
The issue with \textit{Pattern-Breaker} is that it explores the covered regions of the pattern graph and for the cases where there are a few uncovered patterns, it has to explore a large portion of the exponential-size graph. 
To tackle this, \textit{Pattern-Combiner} algorithm is proposed that performs a bottom-up traversal of the pattern graph. It uses an observation that the coverage of a node at the level of the pattern graph can be computed as the sum of the coverage values of its children. 
The problem with \textit{Pattern-Combiner} is that it traverses over the uncovered nodes first and therefore, it will not perform well for the cases in which most of the nodes in the graph are uncovered. 
In fact, for the cases where most of the MUPs are placed in the middle of the graph, both \textit{Pattern-Breaker} and \textit{Pattern-Combiner} will not be as efficient as they should traverse half of the graph. Therefore, we propose \textit{Deep-Diver}, a search algorithm based on Depth-First-Search that quickly finds the MUPs, and uses them to limit the search space by pruning the nodes both dominating and dominated by the discovered MUPs.

\begin{figure*}[!tb]
    \begin{minipage}[t]{0.31\linewidth}
        \centering
        \includegraphics[width=\textwidth]{submissions/submission1/shahbazi/covcube1.jpg}
        \caption{\small Categorical attributes: the uncovered region of a toy example, as the collection of three MUPs.}
        \label{fig:covcube1}
    \end{minipage}
    \hfill
    \begin{minipage}[t]{0.31\linewidth}
        \centering
        \includegraphics[width=\textwidth]{submissions/submission1/shahbazi/cvrg_2_1.jpg}
        \caption{\small Continuous attributes, 2D: identifying the covered region in the gray Voronoi cell.}
        \label{fig:cvrg_2_1}
    \end{minipage}
    \hfill
    \begin{minipage}[t]{0.31\linewidth}
        \centering
        \includegraphics[width=\textwidth]{submissions/submission1/shahbazi/cvrg_2_2.jpg}
        \caption{ \small Continuous attributes, 2D: Uncovered region marked in red.}
        \label{fig:cvrg_2_2}
    \end{minipage}
\vspace{-5mm}
\end{figure*}

\subsubsection{Continuous Attributes}
Data in the real world often consists of a combination of continuous and discrete values. While simple solutions like binning {\tt age} into {\tt young} and {\tt old} can transform the continuous space into discrete. However, they may lead to coarse groupings that are sensitive to the thresholds chosen. It may be inappropriate to treat a 35-yo as {\tt young} but a 36-yo as {\tt old}. 
Therefore, we extend the notion of coverage to continuous space. Particularly, given data set $\dee$ with $n$ tuples over $d$ attributes, and vicinity radius $\rho$ and coverage threshold $k$, we want to identify the uncovered region -- the universe of uncovered query points.
A query point in continuous data space is covered if there are enough (at least $k$) data points in its $\rho$-vicinity neighborhood. $\rho$-vicinity neighborhood is the circle centered at the query point with radius $\rho$.

Depending on the number of attributes in a data set, we propose two algorithms for identifying uncovered regions in data~\cite{asudeh2021coverage}. 
The first algorithm known as \textit{Uncovered-2D} studies coverage over two-dimensional data sets where $\mathbf{x}=\{x_1,x_2\}$. To find the number of circles that a query point falls into and consequently discover the uncovered region, \textit{Uncovered-2D} makes a connection to $k$-th order Voronoi diagrams.
Consider a data set $\mathcal{D}$ and its corresponding $k$-th order Voronoi diagram. For every tuple $t\in \mathcal{D}$, let $\circ_t$ be the $d$-dimensional sphere ($d$-sphere) with radius $\rho$ centered at $t$.
Consider a $k$-voronoi cell $\mathcal{V}(S)$ in the $k$-th order Voronoi diagram $V_k(\mathcal{D})$.
Any point $q$ inside the intersections of the $d$-spheres of tuples in $S$, i.e. $q\in \underset{\forall t\in S}{\cap ~\circ_t}$, is covered, while all other points in the region are uncovered.
 The algorithm starts by constructing the $k$-th order Voronoi diagram of the data set and then for each Voronoi cell $\mathcal{V}(S)$ in the diagram, it computes the intersection of the circles of the tuples in $S$ and marks the portion of $\mathcal{V}(S)$ that falls outside it as uncovered.
After identifying the uncovered region, a 2D map of $\{x_1,x_2\}$ value combinations is used to report the region to the user.
The algorithm for the 2D case can be extended to the general case by relaxing the assumption on the number of attributes to discover the exact uncovered region, however, due to the curse of dimensionality, the search size space explodes as the number of dimensions increases and as a result, the algorithm will not be practical. Therefore, we propose a randomized approximation algorithm based on the geometric notion of \enet. 
Let $\mathcal{X}$ be a set and $\mathcal{R}$ be a set of subsets of $\mathcal{X}$. A set $\mathcal{N}\subset \mathcal{X}$ is an \enet for $\mathcal{X}$ if for any range $r\in\mathcal{R}$, if  $|r\cap \chi|>\eps|\chi|$, then $r$ contains at least one point of $N$.
The idea, at a high level, is to draw enough random samples from the space of potential query points to form an \enet. 
We then label the sampled query points as $\{-1,+1\}$ depending on whether those are covered or not, and learn the uncovered regions using the samples.

\subsection{Image Data}
Many known incidents of machine failures due to the lack of representation were on image data.
We consider an image data set with a fixed number of low-cardinality sensitive attributes such as {\tt\small race} and {\tt\small gender}. 
It is common that image data sets {\it lack explicit values} for sensitive attributes, which are crucial for coverage identification. An image data set is often a collection of images from different domains with little to no information about their domain and which groups they belong to. As a result, even studying coverage over low-cardinality and categorical attributes of interests is challenging in these cases.

\begin{wrapfigure}{R}{0.42\textwidth}
\centering
\vspace{-3mm}
\scriptsize
\begin{tabular}{|@{}c|@{}c@{}|@{}c@{}|@{}c@{}|} 
 \hline
{\bf data set} & {\bf classifier} & {\bf accuracy} & {\bf precision} \\ 
 &  &  & {\bf on female} \\ \hline
UTKFace:~& DeepFace (opencv) & 93.56 & {52.02}\\\cline{2-4}
({\tt females}=200,& DeepFace (retinaface) & 94.16 & {56.15}\\\cline{2-4}
{\tt males}=2800) & BaseCNN & 97.6 & 74.8\\
\hline
UTKFace:~& DeepFace (opencv) & 96.53 & {\bf 8.0}\\\cline{2-4}
({\tt females}=20,& DeepFace (retinaface) & 96.43 & {\bf 10.09}\\\cline{2-4}
{\tt males}=2980)& BaseCNN & 97.6 & {\bf 21.59}\\
\hline
\end{tabular}
\vspace{-3mm}
\caption{\small ML models' low performance for females in the presence of representation bias.~\cite{mousavi2024data}}\label{fig:mlfails}
\vspace{-3mm}
\end{wrapfigure}

In Figure~\ref{fig:mlfails}, we show that due to the issues such {\it machine bias} and {\it lack of distribution generalizability},
solely relying on state-of-the-art machine learning (ML) techniques fail to effectively identify lack of coverage in image data sets. Therefore, we propose an approach based on combining crowdsouring with ML~\cite{mousavi2024data}. 
Crowdsourcing is particularly promising for image data, for tasks such as image labeling, which, while challenging for the machine, are "easy" for human beings to conduct with minimal error. 

A key observation that enables a cost-effective crowdsourcing approach is that, while studying coverage, we would only like to find out if there are {\it enough tuples from each subgroup}.
Suppose a subgroup is covered if there are $\tau=100$ instances of it in the data set. Assume the (majority) group $\gee_1$ contains $n_1 \gg 100$ objects in the data set. 
To verify that $\gee_1$ is covered, it is enough for the crowd to discover 100 of those objects, not the entire $n_1$. 
Following this, $O(\tau)$ provides a lower bound on the number of crowd tasks required to verify a given group is covered. 
Still, this lower bound only holds for the groups that are covered, i.e., there is at least $\tau$ of those in the data set.
Surprisingly, verifying that a minority group is indeed uncovered is cumbersome, unlike the majority group.
This is because even though discovering $\tau$ objects from a group is enough for verifying that it is covered, one cannot {\it verify} a group is uncovered until there is a chance that the data set might still have enough objects from that group. Thus, assuming a non-zero probability for each unlabeled object to belong to each group, {one might need to ask the crowd to label the entire data set before they can confirm that a specific group is uncovered}.

Our idea for addressing this challenge is to
design {\it a divide and conquer algorithm} that, instead of {point queries}, uses {\it set queries} to iteratively eliminate subsets of data that {does not include any object from the given group}.
At a high level, our idea is to ask a set query from the crowd, inquiring whether the selected set contains at least one object from the given group $\gee$.
The user may provide two responses (yes/no). 
Interestingly, {in either case}, the user response provides valuable information that helps efficiently identify the coverage.
If the answer is ``No'', the set does not include any object from the given group $\gee$. As a result, the algorithm can safely prune the set, asking no further questions about it. In particular, for a group that is not covered, one can expect to see no answers on large set queries helping to prune a significant portion of the data set quickly.
On the other hand, if the answer is ``yes'', the set contains {at least} one object from the group $\gee$. As a result, the algorithm cannot prune the subset since it can have any number (larger than one) of the objects in $\gee$.
At first glance, the queries with yes answers do not provide helpful information as the algorithm cannot prune the subset (hence it needs to divide it into smaller subsets).
However, a key observation is that {the algorithm will only observe a limited number of yes answers} before it stops.
The reason is that the number of set queries with yes answers provides a {lower-bound} on the number of objects from $\gee$ in the data set. As a result, the algorithm can stop as soon as the lower bound reaches $\tau$, knowing that $\gee$ is covered.
The D\&C approach verifies the data coverage for a given group, while our goal is to identify the uncovered regions for a given set of sensitive attributes. The next question is how to utilize this algorithm for efficient coverage identification on different scenarios of sensitive attributes, forming intersectional or non-intersectional groups.
In particular, how can we find maximal uncovered patterns?
Our idea is to apply sampling and aggregate estimation techniques to find the groups that even if merged are likely to still be uncovered. This will help reduce the coverage identification cost by running the D\&C approach for the merged groups once.
 %%%%%%%%%%%%%%%%%%%%%%%%%%%%%%%% RESOLUTION  %%%%%%%%%%%%%%%%%%%%%%%%%%%%%%%%
\section{Resolving Insufficient Representation}\label{sec:resolution}

Data integration~\cite{nargesian2021tailoring,nargesian2022responsible} and data augmentation~\cite{sharma2020data,DBLP:journals/jair/ChawlaBHK02,iosifidis2018dealing,celis2020data} are considered as the primary solutions for reducing data coverage issues in a data set. 
Data integration is promising when external sources of data are available. On the other hand, recent advancements in generative AI and foundation models have enabled efficient and effective augmentation of data sets with synthetic data. 
Therefore, in the following, we review two approaches, one from each category, in the context of lack of coverage resolution.

\subsection{Data Integration}\label{sec:resolution:integration}

Data integration is to consolidate data from different sources into a single, unified view. 
Although it is an effective solution to acquire additional data from different distributions,
there are sampling policy and cost-efficiency concerns that need to be examined.  
Therefore, {\it Data Distribution Tailoring ({\sc DT})} introduces data integration techniques for resolving insufficient representation of subgroups in a data set in the most cost-effective manner~\cite{nargesian2021tailoring}.
A query to {\sc DT} 
consists of a target schema, and a set of group distribution requirements in the form of the minimum counts (e.g., ``{\tt\small 1,000 breast cancer monitoring data in Chicago with at least 30\% label=positive, and at least 20\% black patients}''). 
Collecting a fresh sample from a data view is costly (monetary, human resources, and/or computation cost)~\cite{asudeh2022towards}.
Therefore, {\sc DT} focuses on satisfying the count requirements with minimum cost. 
Given an input query and a lake of available data sources, the first step is to discover a collection of candidate data views that satisfy the target schema.
Each data view $v_i$ is a projection-join $v_i = \Pi\big(D_{i1}\bowtie\cdots\bowtie D_{ik_i} \big)$, where $D_{ij}$ is a data set in a given data lake.
Let us suppose the data views are already discovered.
At a high level, {\sc DT} follows an iterative approach that at each iteration a data view is selected to be queried.
Each query to a data view has a fixed cost and returns a sample that may or may not satisfy the query constraints.
The samples that are either not fresh, or do not satisfy the query are discarded.
Hence, the essential question towards a cost-effective data integration is {\it what data view to query next}.
Depending on the available information about the data sources, various techniques may be employed. 

For the cases when the group distributions are known, the process of collecting the target data set is a sequence of iterative steps, where at every step, the algorithm chooses a data view, queries it, and if the obtained tuple contributes to one of the groups for which the count requirement is not yet fulfilled, it is kept, otherwise discarded. To do so, a {Dynamic Programming (DP)} algorithm is proposed. An optimal source at each iteration minimizes the sum of its sampling cost plus the expected cost of collecting the remaining required groups, based on its sampling outcome.
The DP algorithm, however, has a pseudo-polynomial time complexity. Hence, it quickly becomes intractable for cases where the minimum count requirements for the groups are not small. 
For cases where the (sensitive) attribute of interest is binary, such as (biological) {\tt sex}={\tt \{male, female\}}, and the cost to query data is similar from all sources, it turns out that the optimal strategy is to query the data source with {maximum probability of obtaining a sample from the minority group}.
Expanding the binary-attributes algorithm for non-binary cases, the problem can be modeled as an extension of the ``{\it coupon collector's}'' problem~\cite{motwani1995randomized}, where the goal is to collect $m_i$ instances from each coupon (group) $\gee_i$.
At each iteration, the coupon collector's algorithm identifies a data view as most promising and queries it. In simple terms, a data view with a smaller query cost and a higher chance of obtaining minority groups is more promising.


For the cases where the group distributions are unknown, we model DT as a {\it multi-armed bandit} problem, where every data view is modeled as an arm. 
Every arm has an unknown distribution of different groups while pulling an arm (i.e., querying the corresponding data view) has a cost.
During various iterations, the algorithms pull the arms in an order that its expected total {\it reward} is maximized.
Arguing that the reward of obtaining a tuple from a group is proportional to how rare this group is across different data views, 
we design the reward function based on the expected cost one needs to pay in order to collect a tuple from a specific group.  
As the bandit strategy, we adopt {\it Upper Confidence Bound (UCB)} to balance exploration and exploitation. At every iteration, for every arm, UCB computes confidence intervals for the expected reward and selects the arm with the maximum upper bound of reward to be explored next.

\subsection{Data Augmentation using Foundation Models}

While data integration provides a promising approach for resolving coverage issues in a data set, its effectiveness is limited to the availability of external data sources that are rich enough to find sufficient fresh samples from minority groups. This, however, is not always possible, especially since the minority samples are rare and not easy to obtain.
Fortunately, recent advancements in Generative AI and Foundation Models have enabled synthesizing samples that are otherwise challenging to obtain from the real world.

Therefore, as an alternative approach to data integration, we turn our attention to the Foundation Models and Generative AI for resolving the lack of coverage. 
Particularly, models such as {\sc DALL.E}\footnote{\url{https://openai.com/dall-e-2}} have emerged as powerful tools for generating multi-modal data such as image, audio, and video.
 
We formalize the foundation model \fm as a black-box function with the following inputs, that once queried synthesize an output tuple.
\begin{itemize}
    \item {\bf Prompt}: A natural language description providing instructions on the details of the tuple to be generated. For instance, a prompt for image generation might be ``A realistic photo of a white cat running in a backyard.''
    \item {\bf Guide}: In cases where only a prompt is provided, the foundation model uses its imagination to generate the requested tuple. For the previous example, the prompt of a cat image, the breed, size, background, and other details are generated based on the model's imagination. Alternatively, a guide can be provided to influence the generation process. The guide is formalized as a pair $(t,m)$ where $t$ is a tuple and $m$ is a mask specifying which parts of the guide tuple should be changed. Using the cat example, $t$ can be a cat image and $m$ can specify the foreground to be regenerated.
\end{itemize}

There are multiple challenges towards effective data set augmentations using foundation models. 
First, we have to determine the minimal set of synthetic tuples that once added to the original data set, under-representation issues are resolved.
Second, the generated images should follow the underlying distribution represented in the input data set. Third, the generated tuples should have high quality and look realistic to a human evaluator. Last but not least, given the (often monetary) cost associated with the queries to the foundation model, we should ensure the cost-effectiveness of the data set repair process.

\begin{wrapfigure}{L}{0.45\textwidth}
\centering
\vspace{-3mm}
\scriptsize
    \includegraphics[width=.45\textwidth]{submissions/submission1/shahbazi/enhanced_pipeline.png}
\vspace{-3mm}
\caption{\small Architecture of \fmsystem for image data augmentation for coverage enhancement.}\label{fig:chameleon}
% \vspace{-3mm}
\end{wrapfigure}

\noindent Figure~\ref{fig:chameleon} shows the architecture of our system \fmsystem \cite{chameleon} for coverage enhancement using DALL-E image generator.
To address the first challenge, we define the combinations-selection problem, which minimizes the total number of synthetic tuples for resolving lack of coverage of minorities at the most general level. We show the problem is {\sc NP}-hard, and propose a greedy approximation algorithm for it.
To address the second and third challenges, \fmsystem follows a {\it rejection sampling} strategy.
It views each tuple in the data set $\dee$ as an iid sample from the underlying distribution $\xi$ it represents. It uses the vector representations (embeddings) space to describe the distribution. Then, given a newly generated tuple, it employs the one-class support vector machine (OCSVM) approach proposed by Scholkopf et al.~\cite{scholkopf1999support} to reject the tuple if it does not follow $\xi$.
Moreover, it models the quality evaluation as hypothesis testing and rejects the samples that have a higher chance of being labeled as ``unrealistic'' by a random human evaluator.
Finally, to minimize the number of queries to the foundation model, we provide a guide tuple (and a mask), in addition to the prompt, to the foundation model. We model the guide-selection problem as {\it contextual multi-armed bandit} and propose a solution based on the contextual UCB for it.

Before concluding this section, let us provide some experiment results to demonstrate the effectiveness of data augmentation with \fmsystem. We use FERET DB \cite{phillips1998feret} for this experiment, which comprises 1199 individual images and serves as a standardized facial image database for researchers to develop algorithms and report results. All images in FERET DB share the same dimensions, pose, and facial expression.
First, we identified the (level-1) uncovered ethnicity groups, using the threshold 80. We then used \fmsystem and resolved the lack of coverage issues.
To evaluate the effectiveness of the system, we trained a CNN model to predict the race of each image within this dataset. We then retrained the identical CNN on the repaired training data. Importantly, our test dataset for both experiments remains consistent and is derived from real images.
Table~\ref{tab:lackofcoverage} presents the improvements in precision, recall, and F1 score metrics for under-represented groups after repairing the dataset. The results indicate an enhancement in performance metrics for all under-represented groups following the repair process.

\begin{table}[t]
    \centering
    \caption{Illustrating the effect of lack of coverage repair using \fmsystem on \texttt{FERTDB}}
    \label{tab:lackofcoverage}
    \vspace{-3mm}
    \begin{tabular}{lcccccccc}
        \toprule
         & \multicolumn{4}{c}{\textbf{Classifier Performance on \texttt{FERTDB}}} & \multicolumn{4}{c}{\textbf{Classifier Performance on Repaired}} \\
        \cmidrule(lr){2-5} \cmidrule(lr){6-9}
        \textbf{Ethnicity Groups}& \#Images & Precision & Recall & F1-Score & \#Images & Precision & Recall & F1-Score \\
        \midrule
        Overall          & 756 & 0.81 & 0.75 & 0.78 & 987 & 0.70 & 0.75 & 0.72 \\ \hline
        Black            & 40  & 0.19 & 0.22 & 0.16 & 100 & 0.48 & 0.56 & 0.52 \\
        Hispanic         & 19  & 0.50 & 0.17 & 0.25 & 100 & 0.62 & 0.36 & 0.45 \\
        Middle Eastern   & 10  & 0.00 & 0.00 & 0.00 & 100 & 0.20 & 0.41 & 0.27 \\
        \bottomrule
    \end{tabular}
\end{table}

 %%%%%%%%%%%%%%%%%%%%%%%%%%%%%%%% RELIABILITY  %%%%%%%%%%%%%%%%%%%%%%%%%%%%%%%%
\section{Generating Reliability Warnings}\label{sec:reliability}
% up to 2.5 pages
Interpretability is a necessity for data scientists who develop predictive models for critical decision-making.
In such settings, it is important to provide additional means to support the following question:
{\it is an individual prediction of the model reliable for decision-making?} Our goal is to use the lack of representation to help decision-makers find insights about this critical question.
To further motivate this, let us use the following example:

\vspace{1mm}
\begin{example}\label{ex-0}
{\bf(Part1):} Consider a judge who needs to decide whether to accept or deny a bail request. Using data-driven predictive models is prevalent in such cases for predicting recidivism~\cite{dressel2018accuracy}.
Indeed, such models can be beneficial to help the judge make wise decisions.
Suppose the model predicts the queried individual as high risk (or low risk).
The judge is aware and concerned about the critics surrounding such models.
A major question the judge faces is whether or not they should rely on the prediction outcome to take action for this case.
Furthermore, if, for instance, they decide to ignore the outcome and hence they need to provide a statement supporting their action, what evidence can they provide? 
\end{example}

In line with the recent trend on data-centric AI~\cite{ng2021mlops}, we design {novel approaches}, {complimentary} to the existing work on trustworthy AI~\cite{wing2021trustworthy,kentour2021analysis,liu2021trustworthy,singh2021trustworthy}, to address the aforementioned trust question through the lens of {\it data}.
In particular, unlike existing works that generate trust information from a {\it given \underline{model}}, we associate {\it \underline{data sets} with proper measurements} that specify their {\it the scope of use for predicting future cases}.
We note that a predictive model provides only probabilistic guarantees on the \underline{average} loss over the distribution represented by the data set used for training it.
As a result, these predictions may not be distribution generalizable~\cite{kulynych2022you}.
Consequently, if the query point is {\it not represented} by the data, the guarantees may not hold, hence one cannot rely on the prediction outcome.
Besides, an essential requirement for a learning algorithm is that its training data $\dee$ should represent the underlying distribution $\dist$.
Even if so, the trained model $h$ only provides a probabilistic guarantee on the {expected} loss on random samples from $\dist$.  
A model that performs well on {\it majority} of samples drawn from $\dist$ will have a high performance on average. Still, as we observed in Figure~\ref{fig:mlfails},
its performance for {\it minorities} and points that are not represented is questionable. Let us consider the following toy example:

\begin{figure*}[!b] 
    \begin{minipage}[t]{0.32\linewidth}
        	\centering
        	\includegraphics[width=\textwidth]{submissions/submission1/shahbazi/example_1.png} 
        	\vspace{-9mm}\caption{\small Data set $\dee$ generated using a Gaussian distribution; $x_1$ and $x_2$ are positively correlated}
            \label{fig:ex1:1}
    \end{minipage}
    \hfill
    \begin{minipage}[t]{0.32\linewidth}
        \centering
        	\includegraphics[width =\textwidth]{submissions/submission1/shahbazi/example_2.png} 
        	\vspace{-9mm}\caption{\small The decision boundary of learned model $h$ and query points $\qu^1$ to $\qu^4$}
            \label{fig:ex1:2}
    \end{minipage}
    \hfill
    \begin{minipage}[t]{0.32\linewidth}
        	\centering
        	\includegraphics[width =\textwidth]{submissions/submission1/shahbazi/example_3.png}
        	\vspace{-9mm}\caption{\small Ground-truth boundary, overlaid on the model decision boundary and query points}
            \label{fig:ex1:3}
    \end{minipage}
    \vspace{-5mm}
\end{figure*} 

\vspace{1mm}
\begin{example}\label{ex-1}
Consider a binary classification task where the input space is $\ex=\langle x_1, x_2\rangle$ and the output space is the binary label $y$ with values $\{-1$ (red) $,+1$ (blue)$\}$.
Suppose the underlying data distribution $\dist$ follows a 2D Gaussian, where $x_1$ and $x_2$ 
are positively correlated as shown in Figure~\ref{fig:ex1:1}.
The figure shows the data set $\dee$ drawn independently from the distribution $\dist$, along with their labels as their colors.
Using $\dee$, the prediction model $h$ is constructed as shown in Figure~\ref{fig:ex1:2}. 
The decision boundary is specified in the picture; while any point above the line is predicted as +1, a query point below it is labeled as -1.
The classifier has been evaluated using a test set that is an iid sample set drawn from the underlying data set $\dist$. The accuracy on the test set is high (above 90\%), and hence, the model gets deployed.
We cherry-picked four query points, $\qu^1$ to $\qu^4$, that are also included in Figure~\ref{fig:ex1:2}. Using $h$ for prediction, $h(\qu^1)=-1$, $h(\qu^2)=+1$,  $h(\qu^3)=+1$, and $h(\qu^4)=-1$.
Figure~\ref{fig:ex1:3} adds the ground-truth boundary to the search space, revealing the true label of the query points: every point inside the red circle has the true label $-1$ while any point outside of it is $+1$.
Looking at the figure, $y^1=+1$ while the model predicted it as $h(\qu^1)=-1$.  \hfill$\square$
\end{example}
\vspace{2mm}

Let us take a closer look at the four query points in this example and their placement with regard to the tuples in $\dee$ used for training $h$. 
$\qu^2$ belongs to a {\it dense region} with many training tuples in $\dee$ surrounding it. Besides, all of the tuples in its vicinity have the same label $y=+1$. As a result, one can expect that the model's outcome $h(\qu^2)=+1$ should be a reliable prediction.
Similar to $\qu^2$, $\qu^4$ also belongs to a dense region in $\dee$; however, $\qu^4$ belongs to an {\it uncertain region}, where some of the tuples in its vicinity have a label $y=+1$, and some others have the label $y=-1$. Considering the uncertainty in the vicinity of $\qu^4$, one cannot confidently rely on the outcome of the model $h$. 
On the other hand, the neighbors of $\qu^1$ (resp. $\qu^3$) are not uncertain, all having the label $y=-1$ (resp. $y=+1$).
However, the query points $\qu^1$ and $\qu^3$ are not well represented by $\dee$. In other words, $\qu^1$ and $\qu^3$ are unlikely to be generated according to the underlying distribution $\dist$, represented by $\dee$. As a result, following the no-free-lunch theorem~\cite{kakade2003sample}, one cannot expect the outcome of model $h$ to be reliable for these points.
Looking at the ground-truth boundary in Figure~\ref{fig:ex1:3}, $h$ luckily predicted the outcome for $\qu^3$ correctly, but it was not fortunate to predict the $y^1$ correctly.
Nevertheless, 
since the model is not reliably trained for these points, 
its outcome for these query points is not trustworthy.

From Example~\ref{ex-1}, we observe that the outcome of a model $h$, trained using a data set $\dee$ is not reliable for a query point $\qu$, if:
\begin{itemize}
    \item {\bf Lack of representation:} $\qu$ is not well-represented by $\dee$.
    In such cases, the model has not seen ``enough'' samples similar to $\qu$ to reliably learn and predict the outcome of $\qu$.
    \item {\bf Lack of certainty:} $\qu$ belongs to an uncertain region, where different tuples of $\dee$ in the vicinity of $\qu$ have different target values. $\qu$ belongs to a high-fluctuating area, where tuples in the vicinity of $\qu$ have a wide range of values.
\end{itemize} \vspace{2mm}

\noindent
Based on these two observations, we propose Representation-and-Uncertainty ({\bf RU}) measures.
To identify if a query suffers from uncertainty or lack of representation, one could use a deterministic approach using a fixed threshold. Then if the number of similar samples to (resp. label fluctuation in vicinity of) $\qu$ is larger than the threshold it is considered as unrepresented (resp. uncertain).
This approach, however, would be misleading since two numbers close to the threshold could be treated very differently. Also, all points on each side of the threshold would be considered equally represented (resp., certain). Instead, we consider {\it a randomized approach}, widely popular in the literature, including~\cite{dwork2012fairness}.
That is, instead of using fixed thresholds, a Bernoulli variable (a biased coin) is used that 
assigns $\qu$ as unrepresented (resp., uncertain) based on the number of samples similar to it (resp., its neighborhood uncertainty).
Given a query point $\qu$, let $\pe_o$ be the probability indicating if $\qu$ is not represented and let $\pe_u$ be the probability indicating if $\qu$ belongs to an uncertain region. 
We represent the probability of the Bernoulli variables for lack of representation or uncertainty components as $\pe_o$ and $\pe_u$, respectively. Note that the two Bernoulli variables $\pe_o$ and $\pe_u$ are independent from each other. That simply follows the argument that after specifying the number of similar samples to $\qu$ whether or not it should be considered as unrepresented does not depend on the uncertainty in the neighborhood of $\qu$.

\begin{definition}[\sru]\label{def:sdt}
The \sru is a probabilistic measure that considers the outcome of a model for a query point $\qu$ untrustworthy if $\qu$ is not represented by $\dee$ {\it and} it belongs to an uncertain region.
Formally, the \sru measure is:
\begin{align} 
    \nonumber
    SRU(\qu) &= \pe\big((\qu \mbox{ is outlier}) \wedge (\qu \mbox{ belongs to uncertain region})\big) 
\end{align}
Since $\pe_o$ and $\pe_u$ are independent:

\vspace{-13mm}
\begin{align} \label{eq:strong}
    SRU(\qu) &= \pe_o(\qu) \times \pe_u(\qu)
\end{align}
\end{definition}

\sru raises the warning signal only when the query point fails on {\it both} conditions of being represented by $\dee$ and not belonging to an uncertain region. 
For instance, in Example~\ref{ex-1} none of the query points fail both on representation and on uncertainty; hence neither has a high \sru score.
On the other hand, 
a high \sru score for a query point $\qu$ {\it provides a strong warning signal} that one should perhaps reject the model outcome and not consider it for decision-making.

\sru is a strong signal that raises warnings only for the fearfully concerning cases that fail both on representation and uncertainty.
However, as observed in Example~\ref{ex-1} a query points failing {\it at least} one of these conditions may also not be reliable, at least for critical decision making.
We define the \wru measure to raise a warning for such cases.

\begin{definition}[\wru]\label{def:wdt}
The \wru measure is a probabilistic measure that considers the outcome of a model for a query point $\qu$ untrustworthy if $\qu$ is not represented by $\dee$ {\bf or} it belongs to an uncertain region.
Formally, the \wru is computed as:
\begin{align} \label{eq:weak}
    WRU(\qu) = \pe\big((\qu \mbox{ is outlier}) \vee (\qu \mbox{ belongs to uncertain region})\big) 
    = \pe_o(\qu) + \pe_u(\qu) - \pe_o(\qu) \times \pe_u(\qu)
\end{align}
\end{definition}

Proposing quantitative probabilistic outcomes, \ru measures are interpretable for the users, since beyond the scores, the uncertainty and lack of representation components provide an explanation to justify them. 
Please refer to \cite{techrep} for more details on how to efficiently and effectively compute the representation ($\pe_o$) and uncertainty ($\pe_u$) probabilities, using only $\dee$.
In Example~\ref{ex-0}, let us see how the \ru measures can be helpful.

\noindent{\bf Example 1. (part 2):}
{\it RU measures \underline{raise warning} when
the fitness of the data set used for drawing a prediction is questionable, helping the judge to be cautious when taking action.
Besides, these measures provide \underline{quantitative evidence} to support the judge's action when they decide to ignore a prediction outcome that is not trustworthy.
The judge, for example, can argue to ignore a model outcome for a specific case, based on the insight that 
the model has been built using a
data set that fails to represent the given case.}
\hfill$\square$

Finally, let us demonstrate the efficacy of \ru measures through a series of experiments. Since the \ru measures are {\it data-centric},
those are applicable for both classification and regression tasks, irrespective of the model used.
We use {\it Adult} dataset~\cite{adult} for classification and {\it House Sales in King County} dataset for the validation of regression tasks. From each dataset, we uniformly sample two sets from the underlying distribution. The first set serves as the training set to compute the \ru values, and the second one is used as the test set from which the queries are drawn. We validate our proposal by providing the correlation between the \ru values and the performance of an ML model's prediction on the same data. 

We start by computing the \ru values for all the query points in the test set. Next, we bucketize the query points based on their \ru values in equi-width buckets of width 0.1. We repeat this for both \sru and \wru measures. Next, we train a model on the training data set and predict the target variable for the points in each range of \ru measure. The validation results for the classification task on the {\it Adult} dataset are presented in Figures \ref{fig:exp-adult-sdt} and \ref{fig:exp-adult-wdt}. Each figure corresponds to the accuracy/error measures of the classifier over each bucket of \ru values for \sru and \wru. As the \ru values increase, the accuracy of the model drops while the FPR rises, and therefore, the model fails to capture the ground truth for the points that fall into untrustworthy regions in the data set. By repeating the aforementioned steps for the regression task on the {\it House Sales in King County} dataset, we observe similar results presented in Figures \ref{fig:exp-hs-sdt} and \ref{fig:exp-hs-wdt}. 
As the \ru value increases, the RSS of the regression model follows the same trend denoting that the model fails to perform for tuples with a high \ru value.

\begin{figure}[!tb]
    \begin{minipage}[t]{0.24\linewidth}
        \centering
        \includegraphics[width=\textwidth]{submissions/submission1/shahbazi/sdt_adult.pdf}
        \vspace{-6mm}\caption{\small{\it Adult}, efficacy of \sru  on classification}
        \label{fig:exp-adult-sdt}
    \end{minipage}\hfill
    \begin{minipage}[t]{0.24\linewidth}
        \centering
        \includegraphics[width=\textwidth]{submissions/submission1/shahbazi/wdt_adult.pdf}
        \vspace{-6mm}\caption{\small{\it Adult}, efficacy of \wru  on classification}
        \label{fig:exp-adult-wdt}
    \end{minipage}\hfill
    \begin{minipage}[t]{0.24\linewidth}
        \centering
        \includegraphics[width=\textwidth]{submissions/submission1/shahbazi/sdt_regression_house.pdf}
        \vspace{-6mm}\caption{\small{\it House Sales in King County}, efficacy of \sru on regression}
        \label{fig:exp-hs-sdt}
    \end{minipage}\hfill
    \begin{minipage}[t]{0.24\linewidth}
        \centering
        \includegraphics[width=\textwidth]{submissions/submission1/shahbazi/wdt_regression_house.pdf}
        \vspace{-6mm}\caption{\small{\it House Sales in King County}, efficacy \wru on regression}
        \label{fig:exp-hs-wdt}
    \end{minipage}
\vspace{-5mm}
\end{figure}
 %%%%%%%%%%%%%%%%%%%%%%%%%%%%%%%% RELATED WORK  %%%%%%%%%%%%%%%%%%%%%%%%%%%%%%%%
\section{Related Work}\label{related} 

Bias in data has been looked at for a long time in statistical community~\cite{neyman1936contributions} but social data presents different challenges~\cite{olteanu2019social,fairmlbook,barocas2016big,jk2019bias,drosou2017diversity}.
The diversity and representativeness of data have been widely studied~\cite{drosou2017diversity}, in fields such as social science~\cite{berrey2015enigma, dobbin2016diversity,simpson1949measurement}, political science~\cite{surowiecki2005wisdom}, and information retrieval~\cite{agrawal2009diversifying}. 
Tracing back machine bias to its source, there have been major efforts to identify different types~\cite{mehrabi2021survey, olteanu2019social,friedman1996bias} and sources~\cite{torralba2011unbiased,crawford2013hidden,diakopoulos2015algorithmic} of biases in data. Efforts to satisfy {\it responsible data} requirements~\cite{nargesian2022responsible} extend to various stages of the data analysis pipeline, including data annotation~\cite{li2020towards,lazier2023fairness}, data cleaning and repair~\cite{SalimiRHS19,tae2019data,salimi2020database}, data imputation~\cite{martinez2019fairness}, entity resolution~\cite{shahbazi2023through,fanourakis2023fairer}, data integration~\cite{nargesian2022responsible,nargesian2021tailoring}, etc. 

\paragraph{Data Coverage:}The notion of data coverage has received extensive attention from different angles. Detecting lack of coverage has been studied for datasets with discrete~\cite{asudeh2019assessing} and continuous~\cite{asudeh2021coverage} attributes populated in single or multiple \cite{lin2020identifying} relations.
To resolve insufficient coverage, \cite{accinelli2020coverage, accinelli2021impact,shetiya2022fairness}
consider resolving representation bias in preprocessing pipelines by rewriting queries into the closest operation so that certain subgroups are sufficiently represented in the downstream tasks. Alternatively, ~\cite{asudeh2019assessing,tae2021slice} propose a data collection strategy to acquire as little additional data as possible (to minimize the associated costs) to meet the representation constraints. ~\cite{sharma2020data,iosifidis2018dealing,celis2020data} opt for a data augmentation approach by adding partially altered duplicates of already existing tuples or generating new synthetic entries from existing data. Consequently, the new data set has an equal number of elements for different groups, resulting in potentially resolving the under-representation issues. Finally,  \cite{nargesian2021tailoring} utilizes data integration techniques to consolidate data from different sources into a single dataset to resolve representation bias.
Related works also include ~\cite{chung2019slice,sagadeeva2021sliceline,tae2021slice} that seek to understand if the overall performance of the model fails to reflect and performs poorly on certain slices in the data.
As alternative approaches to measure representation bias, the notion of representation rate~\cite{celis2020data} (a.k.a. equal base rate~\cite{kleinberg2016inherent}) is introduced which compared with coverage, it is more restrictive as it requires almost equal ratios from different groups.
Please refer to \cite{shahbazi2023representation} for a comprehensive survey about representation bias in data. 

\paragraph{ML Reliability:} Model-centric works for uncertainty quantification such as 
probabilistic classifiers~\cite{zadrozny2001obtaining,zadrozny2002transforming,platt1999probabilistic,niculescu2005predicting},
prediction intervals (PIs) \cite{chatfield93predictionintervals,pearce2018high,khosravi2010lower} and conformal predictions (CP)~\cite{angelopoulos2021gentle,shafer2008tutorial} that are used for measuring prediction uncertainty, are built
by maximizing the {\it expected performance} on {\it random} sample from the underlying distribution.
As a result, while providing accurate estimations for the dense regions of data (e.g. majority groups), their estimation accuracy is questionable for the poorly represented regions.
In particular, \cite{angelopoulos2021gentle} recognizes the lack of guarantees in the performance of CP for such regions.
Besides, the bulk of work on trustworthy AI provides information that {\it supports} the outcome of an ML model. For example, existing work on explainable AI, including~\cite{harradon2018causal,ribeiro2016should,gunning2019darpa}, aims to find simple explanations and rules that justify the outcome of a model.
Conversely, we aim to {\it raise warning signals} when the outcome of a model is {\it not} trustworthy. That is, to provide reasons that {\it cast doubt} on the reliability of the model outcome {for a given query point}.

 %%%%%%%%%%%%%%%%%%%%%%%%%%%%%%%% FUTURE  %%%%%%%%%%%%%%%%%%%%%%%%%%%%%%%%
% \vspace{-3mm}
\section{Final Remarks}\label{sec:conclusion}
As Data-centric AI and Responsible AI emerge as focal points in data science research, the development of Data-centric methodologies for ensuring Responsible and Trustworthy AI attracts increasing attention.
While there is some excellent work on responsible data management to achieve this goal, there remain many challenges yet to be addressed.

In this paper, we focused on a crucial aspect of responsible data -- detecting and addressing the under-representation of minorities within a data set.
We formally defined the notion of data coverage and discussed various techniques for (a) identifying lack of representation issues across different data modalities, (b) ensuring proper representation of minorities in data, and (c) limiting the scope-of-use of data sets based on their representation issues by generating proper ({\sc RU}) warning signals.
Even though the research on detecting lack of coverage issues is relatively mature, resolution techniques are still understudied.
Considering the recent advancements in Generative AI, utilizing Foundation Models and Large Language Models, and studying their limitations, for data augmentation to improve the representation of minorities at the data level seems interesting to further explore.

 %%%%%%%%%%%%%%%%%%%%%%%%%%%%%%%% BIB  %%%%%%%%%%%%%%%%%%%%%%%%%%%%%%%%
\bibliographystyle{unsrt}
\small
% \bibliography{ref}
\begin{thebibliography}{10}

\bibitem{asudeh2019assessing}
A.~Asudeh, Z.~Jin, and H.~Jagadish.
\newblock Assessing and remedying coverage for a given dataset.
\newblock In {\em ICDE}, pages 554--565. IEEE, 2019.

\bibitem{shahbazi2023representation}
N.~Shahbazi, Y.~Lin, A.~Asudeh, and H.~Jagadish.
\newblock Representation bias in data: A survey on identification and resolution techniques.
\newblock {\em ACM Computing Surveys}, 2023.

\bibitem{asudeh2021coverage}
A.~Asudeh, N.~Shahbazi, Z.~Jin, and H.~V. Jagadish.
\newblock Identifying insufficient data coverage for ordinal continuous-valued attributes.
\newblock In {\em SIGMOD}. ACM, 2021.

\bibitem{mousavi2024data}
M.~Mousavi, N.~Shahbazi, and A.~Asudeh.
\newblock Data coverage for detecting representation bias in image datasets: {A} crowdsourcing approach.
\newblock In {\em {EDBT}}, pages 47--60, 2024.

\bibitem{nargesian2021tailoring}
F.~Nargesian, A.~Asudeh, and H.~Jagadish.
\newblock Tailoring data source distributions for fairness-aware data integration.
\newblock {\em Proceedings of the VLDB Endowment}, 14(11):2519--2532, 2021.

\bibitem{nargesian2022responsible}
F.~Nargesian, A.~Asudeh, and H.~V. Jagadish.
\newblock Responsible data integration: Next-generation challenges.
\newblock {\em SIGMOD}, 2022.

\bibitem{sharma2020data}
S.~Sharma, Y.~Zhang, J.~M. R{\'\i}os~Aliaga, D.~Bouneffouf, V.~Muthusamy, and K.~R. Varshney.
\newblock Data augmentation for discrimination prevention and bias disambiguation.
\newblock In {\em AIES}, pages 358--364, 2020.

\bibitem{DBLP:journals/jair/ChawlaBHK02}
N.~V. Chawla, K.~W. Bowyer, L.~O. Hall, and W.~P. Kegelmeyer.
\newblock {SMOTE:} synthetic minority over-sampling technique.
\newblock {\em J. Artif. Intell. Res.}, 16:321--357, 2002.

\bibitem{iosifidis2018dealing}
V.~Iosifidis and E.~Ntoutsi.
\newblock Dealing with bias via data augmentation in supervised learning scenarios.
\newblock {\em Jo Bates Paul D. Clough Robert J{\"a}schke}, 24, 2018.

\bibitem{celis2020data}
L.~E. Celis, V.~Keswani, and N.~Vishnoi.
\newblock Data preprocessing to mitigate bias: A maximum entropy based approach.
\newblock In {\em ICML}, pages 1349--1359. PMLR, 2020.

\bibitem{asudeh2022towards}
A.~Asudeh and F.~Nargesian.
\newblock Towards distribution-aware query answering in data markets.
\newblock {\em Proceedings of the VLDB Endowment}, 15(11):3137--3144, 2022.

\bibitem{motwani1995randomized}
R.~Motwani and P.~Raghavan.
\newblock {\em Randomized algorithms}.
\newblock Cambridge university press, 1995.

\bibitem{chameleon}
M.~Erfanian, H.~V. Jagadish, and A.~Asudeh.
\newblock Chameleon: Foundation models for fairness-aware multi-modal data augmentation to enhance coverage of minorities.
\newblock {\em arXiv preprint arXiv:2402.01071}, 2024.

\bibitem{scholkopf1999support}
B.~Sch{\"o}lkopf, R.~C. Williamson, A.~Smola, J.~Shawe-Taylor, and J.~Platt.
\newblock Support vector method for novelty detection.
\newblock {\em NeurIPS}, 12, 1999.

\bibitem{phillips1998feret}
P.~J. Phillips, H.~Wechsler, J.~Huang, and P.~J. Rauss.
\newblock The feret database and evaluation procedure for face-recognition algorithms.
\newblock {\em Image and vision computing}, 16(5):295--306, 1998.

\bibitem{dressel2018accuracy}
J.~Dressel and H.~Farid.
\newblock The accuracy, fairness, and limits of predicting recidivism.
\newblock {\em Science advances}, 4(1):eaao5580, 2018.

\bibitem{ng2021mlops}
A.~Ng.
\newblock Mlops: From model-centric to data-centric {AI}.
\newblock 2021.

\bibitem{wing2021trustworthy}
J.~M. Wing.
\newblock Trustworthy {AI}.
\newblock {\em CACM}, 64(10):64--71, 2021.

\bibitem{kentour2021analysis}
M.~Kentour and J.~Lu.
\newblock Analysis of trustworthiness in machine learning and deep learning.
\newblock {\em InfoComp}, 2021.

\bibitem{liu2021trustworthy}
H.~Liu, Y.~Wang, W.~Fan, X.~Liu, Y.~Li, S.~Jain, A.~K. Jain, and J.~Tang.
\newblock Trustworthy {AI}: A computational perspective.
\newblock {\em arXiv preprint arXiv:2107.06641}, 2021.

\bibitem{singh2021trustworthy}
R.~Singh, M.~Vatsa, and N.~Ratha.
\newblock Trustworthy {AI}.
\newblock In {\em 8th ACM IKDD CODS and 26th COMAD}, pages 449--453. 2021.

\bibitem{kulynych2022you}
B.~Kulynych, Y.-Y. Yang, Y.~Yu, J.~B{\l}asiok, and P.~Nakkiran.
\newblock What you see is what you get: Distributional generalization for algorithm design in deep learning.
\newblock {\em arXiv preprint arXiv:2204.03230}, 2022.

\bibitem{kakade2003sample}
S.~M. Kakade.
\newblock {\em On the sample complexity of reinforcement learning}.
\newblock University of London, University College London (United Kingdom), 2003.

\bibitem{dwork2012fairness}
C.~Dwork, M.~Hardt, T.~Pitassi, O.~Reingold, and R.~Zemel.
\newblock Fairness through awareness.
\newblock In {\em ITCS}, pages 214--226, 2012.

\bibitem{techrep}
N.~Shahbazi and A.~Asudeh.
\newblock Data-centric reliability evaluation of individual predictions.
\newblock {\em CoRR, abs/2204.07682}, 2022.

\bibitem{adult}
M.~Lichman.
\newblock Adult income dataset, {UCI} machine learning repository.
\newblock \url{https://archive.ics.uci.edu/ml/datasets/adult}, 2013.

\bibitem{neyman1936contributions}
J.~Neyman and E.~S. Pearson.
\newblock Contributions to the theory of testing statistical hypotheses.
\newblock {\em Statistical Research Memoirs}, 1936.

\bibitem{olteanu2019social}
A.~Olteanu, C.~Castillo, F.~Diaz, and E.~Kiciman.
\newblock Social data: Biases, methodological pitfalls, and ethical boundaries.
\newblock {\em Frontiers in Big Data}, 2:13, 2019.

\bibitem{fairmlbook}
S.~Barocas, M.~Hardt, and A.~Narayanan.
\newblock Fairness and machine learning: Limitations and opportunities.
\newblock \url{fairmlbook.org}, 2019.

\bibitem{barocas2016big}
S.~Barocas and A.~D. Selbst.
\newblock Big data's disparate impact.
\newblock {\em Calif. L. Rev.}, 104:671, 2016.

\bibitem{jk2019bias}
J.~Kleinberg.
\newblock Fairness, rankings, and behavioral biases.
\newblock FAT*, 2019.

\bibitem{drosou2017diversity}
M.~Drosou, H.~Jagadish, E.~Pitoura, and J.~Stoyanovich.
\newblock Diversity in big data: A review.
\newblock {\em Big data}, 5(2):73--84, 2017.

\bibitem{berrey2015enigma}
E.~Berrey.
\newblock {\em The enigma of diversity: The language of race and the limits of racial justice}.
\newblock University of Chicago Press, 2015.

\bibitem{dobbin2016diversity}
F.~Dobbin and A.~Kalev.
\newblock Why diversity programs fail and what works better.
\newblock {\em Harvard Business Review}, 94(7-8):52--60, 2016.

\bibitem{simpson1949measurement}
E.~H. Simpson.
\newblock Measurement of diversity.
\newblock {\em Nature}, 163(4148), 1949.

\bibitem{surowiecki2005wisdom}
J.~Surowiecki.
\newblock {\em The wisdom of crowds}.
\newblock Anchor, 2005.

\bibitem{agrawal2009diversifying}
R.~Agrawal, S.~Gollapudi, A.~Halverson, and S.~Ieong.
\newblock Diversifying search results.
\newblock In {\em WSDM}, pages 5--14. ACM, 2009.

\bibitem{mehrabi2021survey}
N.~Mehrabi, F.~Morstatter, N.~Saxena, K.~Lerman, and A.~Galstyan.
\newblock A survey on bias and fairness in machine learning.
\newblock {\em ACM Computing Surveys (CSUR)}, 54(6):1--35, 2021.

\bibitem{friedman1996bias}
B.~Friedman and H.~Nissenbaum.
\newblock Bias in computer systems.
\newblock {\em TOIS}, 14(3):330--347, 1996.

\bibitem{torralba2011unbiased}
A.~Torralba and A.~A. Efros.
\newblock Unbiased look at dataset bias.
\newblock In {\em CVPR 2011}, pages 1521--1528. IEEE, 2011.

\bibitem{crawford2013hidden}
K.~Crawford.
\newblock The hidden biases in big data.
\newblock {\em Harvard business review}, 1(4), 2013.

\bibitem{diakopoulos2015algorithmic}
N.~Diakopoulos.
\newblock Algorithmic accountability: Journalistic investigation of computational power structures.
\newblock {\em Digital journalism}, 3(3):398--415, 2015.

\bibitem{li2020towards}
Y.~Li, H.~Sun, and W.~H. Wang.
\newblock Towards fair truth discovery from biased crowdsourced answers.
\newblock In {\em SIGKDD}, pages 599--607, 2020.

\bibitem{lazier2023fairness}
S.~Lazier, S.~Thirumuruganathan, and H.~Anahideh.
\newblock Fairness and bias in truth discovery algorithms: An experimental analysis.
\newblock {\em arXiv preprint arXiv:2304.12573}, 2023.

\bibitem{SalimiRHS19}
B.~Salimi, L.~Rodriguez, B.~Howe, and D.~Suciu.
\newblock Interventional fairness: Causal database repair for algorithmic fairness.
\newblock In {\em {SIGMOD}}, pages 793--810. {ACM}, 2019.

\bibitem{tae2019data}
K.~H. Tae, Y.~Roh, Y.~H. Oh, H.~Kim, and S.~E. Whang.
\newblock Data cleaning for accurate, fair, and robust models: A big data-{AI} integration approach.
\newblock In {\em DEEM workshop}, pages 1--4, 2019.

\bibitem{salimi2020database}
B.~Salimi, B.~Howe, and D.~Suciu.
\newblock Database repair meets algorithmic fairness.
\newblock {\em ACM SIGMOD Record}, 49(1):34--41, 2020.

\bibitem{martinez2019fairness}
F.~Mart{\'\i}nez-Plumed, C.~Ferri, D.~Nieves, and J.~Hern{\'a}ndez-Orallo.
\newblock Fairness and missing values.
\newblock {\em arXiv preprint arXiv:1905.12728}, 2019.

\bibitem{shahbazi2023through}
N.~Shahbazi, N.~Danevski, F.~Nargesian, A.~Asudeh, and D.~Srivastava.
\newblock Through the fairness lens: Experimental analysis and evaluation of entity matching.
\newblock {\em Proceedings of the VLDB Endowment}, 16(11):3279--3292, 2023.

\bibitem{fanourakis2023fairer}
N.~Fanourakis, C.~Kontousias, V.~Efthymiou, V.~Christophides, and D.~Plexousakis.
\newblock Fairer demo: Fairness-aware and explainable entity resolution.
\newblock 2023.

\bibitem{lin2020identifying}
Y.~Lin, Y.~Guan, A.~Asudeh, and H.~Jagadish.
\newblock Identifying insufficient data coverage in databases with multiple relations.
\newblock {\em Proceedings of the VLDB Endowment}, 13(12):2229--2242, 2020.

\bibitem{accinelli2020coverage}
C.~Accinelli, S.~Minisi, and B.~Catania.
\newblock Coverage-based rewriting for data preparation.
\newblock In {\em EDBT Workshops}, 2020.

\bibitem{accinelli2021impact}
C.~Accinelli, B.~Catania, G.~Guerrini, and S.~Minisi.
\newblock The impact of rewriting on coverage constraint satisfaction.
\newblock In {\em EDBT Workshops}, 2021.

\bibitem{shetiya2022fairness}
S.~Shetiya, I.~P. Swift, A.~Asudeh, and G.~Das.
\newblock Fairness-aware range queries for selecting unbiased data.
\newblock In {\em ICDE}. IEEE, 2022.

\bibitem{tae2021slice}
K.~H. Tae and S.~E. Whang.
\newblock Slice tuner: A selective data acquisition framework for accurate and fair machine learning models.
\newblock In {\em SIGMOD}, pages 1771--1783, 2021.

\bibitem{chung2019slice}
Y.~Chung, T.~Kraska, N.~Polyzotis, K.~H. Tae, and S.~E. Whang.
\newblock Slice finder: Automated data slicing for model validation.
\newblock In {\em ICDE}, pages 1550--1553. IEEE, 2019.

\bibitem{sagadeeva2021sliceline}
S.~Sagadeeva and M.~Boehm.
\newblock Sliceline: Fast, linear-algebra-based slice finding for ml model debugging.
\newblock In {\em SIGMOD}, pages 2290--2299, 2021.

\bibitem{kleinberg2016inherent}
J.~Kleinberg, S.~Mullainathan, and M.~Raghavan.
\newblock Inherent trade-offs in the fair determination of risk scores.
\newblock {\em arXiv preprint arXiv:1609.05807}, 2016.

\bibitem{zadrozny2001obtaining}
B.~Zadrozny and C.~Elkan.
\newblock Obtaining calibrated probability estimates from decision trees and naive bayesian classifiers.
\newblock In {\em ICML}, volume~1, pages 609--616. Citeseer, 2001.

\bibitem{zadrozny2002transforming}
B.~Zadrozny and C.~Elkan.
\newblock Transforming classifier scores into accurate multiclass probability estimates.
\newblock In {\em SIGKDD}, pages 694--699, 2002.

\bibitem{platt1999probabilistic}
J.~Platt et~al.
\newblock Probabilistic outputs for support vector machines and comparisons to regularized likelihood methods.
\newblock {\em Advances in large margin classifiers}, 10(3):61--74, 1999.

\bibitem{niculescu2005predicting}
A.~Niculescu-Mizil and R.~Caruana.
\newblock Predicting good probabilities with supervised learning.
\newblock In {\em Proceedings of the 22nd international conference on Machine learning}, pages 625--632, 2005.

\bibitem{chatfield93predictionintervals}
C.~Chatfield.
\newblock Prediction intervals.
\newblock {\em Journal of Business and Economic Statistics}, 11:121--135, 1993.

\bibitem{pearce2018high}
T.~Pearce, A.~Brintrup, M.~Zaki, and A.~Neely.
\newblock High-quality prediction intervals for deep learning: A distribution-free, ensembled approach.
\newblock In {\em International conference on machine learning}, pages 4075--4084. PMLR, 2018.

\bibitem{khosravi2010lower}
A.~Khosravi, S.~Nahavandi, D.~Creighton, and A.~F. Atiya.
\newblock Lower upper bound estimation method for construction of neural network-based prediction intervals.
\newblock {\em IEEE transactions on neural networks}, 22(3):337--346, 2010.

\bibitem{angelopoulos2021gentle}
A.~N. Angelopoulos and S.~Bates.
\newblock A gentle introduction to conformal prediction and distribution-free uncertainty quantification.
\newblock {\em arXiv preprint arXiv:2107.07511}, 2021.

\bibitem{shafer2008tutorial}
G.~Shafer and V.~Vovk.
\newblock A tutorial on conformal prediction.
\newblock {\em Journal of Machine Learning Research}, 9(3), 2008.

\bibitem{harradon2018causal}
M.~Harradon, J.~Druce, and B.~Ruttenberg.
\newblock Causal learning and explanation of deep neural networks via autoencoded activations.
\newblock {\em arXiv preprint arXiv:1802.00541}, 2018.

\bibitem{ribeiro2016should}
M.~T. Ribeiro, S.~Singh, and C.~Guestrin.
\newblock " why should i trust you?" explaining the predictions of any classifier.
\newblock In {\em SIGKDD}, pages 1135--1144, 2016.

\bibitem{gunning2019darpa}
D.~Gunning and D.~Aha.
\newblock Darpa’s explainable artificial intelligence ({XAI}) program.
\newblock {\em AI Magazine}, 40(2):44--58, 2019.

\end{thebibliography}

\end{document}

\end{article}

\begin{article}
{Navigating Compliance with Data Transfers in Federated Data Processing}
{Kaustubh Beedkar, Jorge Quian\'e, Volker Markl}
\graphicspath{{submissions/navigating-compliance-with-data-transfers-in-federated-data-processing/}}
% link to instruction: https://tc.computer.org/tcde/tcde-bulletin-author-instructions/
% \documentclass[11pt,dvipdfm]{article}
\documentclass[11pt]{article}
\usepackage{tabularx}
\usepackage{ragged2e}  % for '\RaggedRight' macro (allows hyphenation)
\usepackage{booktabs}  % for \toprule, \midrule, and \bottomrule macros
\usepackage{textcomp}
\usepackage{amsfonts,amsmath}
\usepackage{deauthor,times}
\usepackage{graphicx} % 
\usepackage{hyperref}
\usepackage{comment}
\graphicspath{{asudeh/}}
\usepackage{soul}
\usepackage{subcaption}
\usepackage{ulem}
\usepackage{wrapfig}
\usepackage{color}
\usepackage{xspace}
\newtheorem{problem}{Problem}

%\DeclareMathOperator*{\argmax}{arg\,max}

%remove the following commands/package b4 submission
\newcommand{\hide}[1]{}
\newcommand{\eat}[1]{}
\newcommand{\resolved}[1]{\hide{#1}}
\newcommand{\abol}[1]{\textcolor{red}{Abol: #1}}
\newcommand{\mahdi}[1]{\textcolor{red}{Mahdi: #1}}
\newcommand{\nima}[1]{\textcolor{red}{Nima: #1}}

\newcommand{\dee}{\mathcal{D}}
\newcommand{\Gee}{\mathcal{G}}
\newcommand{\gee}{\mathbf{g}}
\newcommand{\ee}{\mathbf{e}}
\newcommand{\es}{\mathcal{S}}
\newcommand{\el}{\mathcal{L}}
\newcommand{\xx}{\mathcal{x}}
\newcommand{\dist}{\xi}
\newcommand{\alg}{\mathsf{A}}
\newcommand{\qu}{\mathbf{q}}
\newcommand{\ex}{\mathbf{x}}
\newcommand{\ti}{\mathbf{t}}
\newcommand{\sdt}{\mathsf{SDT}}
\newcommand{\wdt}{\mathsf{WDT}}
\newcommand{\Qu}{\mathbf{Q}}
\newcommand{\pe}{\mathbb{P}}
\newcommand{\megam}{\mathcal{M}}
\newcommand{\eps}{\varepsilon}
\newcommand{\enet}{{$\varepsilon$-{\bf net}}\xspace}
\newcommand{\net}{{\tt net}\xspace}
\newcommand{\vcd}{VC-dimension\xspace}
\newcommand{\at}[1]{{\tt \small #1}\xspace}
\newcommand{\pr}{Pr}

\newcommand{\sharpP}{\mbox{\#P}}
\newcommand{\NP}{\mathsf{NP}}
\newcommand{\LP}{\mathsf{LP}}
\newcommand{\IP}{\mathsf{IP}}
\newcommand{\ru}{{\sc {RU}}\xspace}
\newcommand{\sru}{{\sc {strongRU}}\xspace}
\newcommand{\wru}{{\sc {weakRU}}\xspace}

\newcommand{\fmsystem}{{\sc Chameleon}\xspace}
\newcommand{\fm}{$\mathcal{F}$\xspace}

\newtheorem{experiment}{Experiment}

\begin{document}

\title{Coverage-based Data-centric Approaches for \\Responsible and Trustworthy AI\thanks{This research was supported by the National Science Foundation under grant No. 2107290.}}

\author{
\begin{tabular}[t]{c@{\extracolsep{2.4em}}c@{\extracolsep{2.4em}}c@{\extracolsep{2.3em}}c} 
Nima Shahbazi & Mahdi Erfanian & Abolfazl Asudeh \\ 
University of Illinois Chicago & University of Illinois Chicago & University of Illinois Chicago\\
 nshahb3@uic.edu & merfan2@uic.edu & asudeh@uic.edu
\end{tabular}
}

\maketitle


\begin{abstract}
The grand goal of data-driven decision systems is to help make decisions easier, more accurate, at a higher scale, and also just. However, data-driven algorithms are only as good as the data they work with. Yet, data sets, especially those with social data, often do not represent minorities. The paucity of training data is a perpetual problem for AI, and the outcome of ML models for cases not represented in their training data is often not reliable. 
Hence, without properly addressing the lack of representation issues in data, we cannot expect AI-based societal solutions to have responsible and trustworthy outcomes. 

This paper focuses on data coverage as a data-centric approach for identifying and resolving misrepresentation of minorities in data.
To achieve this goal, we propose novel algorithms that (a) {\it identify} and {\it resolve} insufficient data coverage across data with different modalities and (b) use lack of representation information to generate data-centric {\it reliability warnings}.
 \end{abstract}
 
 %%%%%%%%%%%%%%%%%%%%%%%%%%%%%%%% INTRO  %%%%%%%%%%%%%%%%%%%%%%%%%%%%%%%%
\section{Introduction}\label{sec:intro} % Abstract+Intro: up to 2.5 pages 
Data-driven decision-making has shaped every corner of human life, spanning from autonomous vehicles to healthcare and even predictive policing and criminal justice. A pivotal concern, especially in applications that affect individuals, revolves around the reliability of the decisions rendered by the system.
It is easy to see that the accuracy of a data-driven decision depends, first and foremost, on the data used to make it. Essentially, the system learns the phenomena that data represent. While we may desire that the data should represent the underlying data distribution from which the production data is drawn, this alone may be insufficient, as it merely enables the model to perform well for the average case.
As a result, a model with a high accuracy could fail for specific regions in the data with insufficient representation. These regions may matter because they frequently represent some minority population in society. They could also represent cases that may not happen very often but have a relevant impact on the correctness of a critical decision.
In short, if the data fails to sufficiently represent a specific population, the outcome of the decision system for that population may not be trustworthy.

The phenomenon known as \textit{Representation Bias} can arise from how the data was originally collected, or it could be the result of biases introduced post-collection—whether historically, cognitively, or statistically.

Representation bias is essentially inevitable without a systematic approach to data collection. 
For example, in the context of survey data collection, vital steps involve identifying all populations within the underlying distribution based on desired demographic information and ensuring comprehensive coverage with sufficient samples from each group. 
Even then, only an (uncontrolled) subset of the invitees will opt-in to respond to the survey.
Another challenge lies in the fact that data scientists often lack control over the data collection process, leading to the reliance on ``found data'' in the majority of data-driven systems. Therefore, with no guarantee on the aforementioned steps in the data collection process, the found data is most likely a biased sample.
Acknowledging the potential harms of representation bias, the notion of \textit{Data Coverage}~\cite{asudeh2019assessing,shahbazi2023representation} has been proposed to ensure the adequate representation of minority groups in data sets employed for decision-making and developing sophisticated data science tools. 

Addressing representation issues in data poses various challenges depending on the modality of the data. In this paper, we focus on identifying and resolving lack of coverage issues in data with different modalities.
We start by proposing a variety of techniques (spanning from geometric and combinatorial optimization to crowd-souring) aimed at efficiently detecting insufficient coverage on structured data sets with non-ordinal categorical and continuous attributes, as well as image data sets. Next, we propose a range of approaches grounded in data integration and generative data augmentation to address the lack of coverage by enriching the data sets with more data. However, with limited control over the data collection processes, it could be difficult and expensive to resolve all misrepresentations. 
Since adding more data is not always possible, we proceed to introduce data-centric preventive solutions that warn the user about the reliability of their predictions regarding representation bias issues. These warnings assist users in determining whether they trust the outcomes of the models or exercise caution. 

 %%%%%%%%%%%%%%%%%%%%%%%%%%%%%%%% IDENTIFICATION  %%%%%%%%%%%%%%%%%%%%%%%%%%%%%%%%
\section{Detecting Insufficient Representation of Minorities}\label{sec:identification} %up to 3.5 pages
Representation bias happens when the development (training data) population under-represents 
and subsequently fails to generalize well 
for some parts of the target population, due to historical bias, sampling bias, etc.
The notion of {\it data coverage} has been studied across different settings in \cite{shahbazi2023representation} as a metric to measure representation bias. At a high level, coverage is referred to as having enough similar entries for each object in a data set. 
For a better understanding, let us go over the definition of the generalized notion of coverage:

\begin{definition}[Data Coverage]\label{def:coverage}
Consider a data set $\dee$ with $n$ tuples, each consisting of $d$ attributes of interest $\mathbf{x}=\{x_1, x_2, \cdots,x_d\}$, such as {\tt gender}, {\tt race}, {\tt salary}, {\tt age}, etc, that are used for coverage identification.
The data set also contains target attributes $\mathbf{y} = \{ y_1,\cdots,y_{d'}\}$ that may or may not be considered for the coverage problem.
A query point $q$ is not covered by the data set $\dee$, if there are not ``enough'' data points in $\dee$ that are representative of $q$.
To generalize the notion of coverage, let us define $\gee(q)$ as the universe of tuples that would represent $q$ and let $\gee_\dee(q) = \gee(q)\cap \dee$. In other words, $\gee_\dee(q)$ are the set of tuples in $\dee$ that represent $q$.
Using this notation, we define the coverage of $q$ as the size of $\gee_\dee(q)$. That is,
$cov(q,\dee) = | \gee_\dee(q)|$.
Given a value $\tau$, $q$ is covered if $cov(q,\dee)>\tau$.
Similarly, a group $\gee$ is not covered if $\gee\cap \dee<\tau$.
The {\it uncovered region} in a data set is the collection of groups that are not covered by it.
\end{definition}

\subsection{Structured Data}
In this section, we focus on identifying representation bias in structured data.
Depending on the type of the attributes of interest, we categorize the techniques into two classes based on whether they target the problem for non-ordinal {\it categorical} (e.g. {\tt race}, {\tt gender}) or ordinal {\it continuous} (e.g. {\tt age}) attributes. The attributes of interest considered for representation bias often include sensitive attributes such as {\tt race} and {\tt gender} but are not necessarily limited to them.

\subsubsection{Categorical Attributes}

For cases where attributes of interest are non-ordinal categorical,
the cartesian product of values on a subset of attributes $\mathbf{x}'\subseteq \mathbf{x}$, form a set of (sub-)groups.
For example, $\{$ {\tt white male}, {\tt white female}, {\tt black male} $,\cdots\}$ are the subgroups defined on the attributes {\tt (race,gender)}.
We refer to the number of attributes used to specify a subgroup as the {\it level} of that subgroup.
For example, the level of the subgroup {\tt white male} is 2, while the level of the subgroup {\tt male} is 1.
We use $\ell(\gee)$, to refer to the level of a subgroup $\gee$.
Similarly, we say a subgroup $\gee'$ is a subset of $\gee$, if the groups specifying $\gee'$ are a superset of the ones for $\gee$. For example {\tt (married white male)} a subset of the more general group {\tt (white male)}. That is, the set of individuals in group {\tt (married white male)} are a subset of {\tt (white male)}.
Moreover, we say a subgroup $\gee$ is a {\it parent} of the subgroup $\gee'$, if $\gee'\subset \gee$ and $\ell(\gee)=\ell(\gee')+1$. For example, the subgroup {\tt (white male)} is a parent of the subgroup {\tt (married white male)}.
We use \textit{patterns} to refer to uncovered subgroups.
A pattern $P$ is a string of $d$ values, where $P[i]$ is either a value from the domain of $x_i$, or it is ``unspecified'', specified with $X$. 
For example, consider a data set with three binary attributes of interest $\mathbf{x}=\{x_1, x_2, x_3\}$. The pattern $P=X01$ specifies all the tuples for which $x_2=0$ and $x_3=1$ ($x_1$ can have any value).
The set of patterns that identify most general uncovered subgroups are called {\it Maximal Uncovered Patterns} (MUPs).

No polynomial time algorithm can guarantee the enumeration of the entire MUPs, however, several algorithms inspired by set enumeration and the Apriori algorithm for association rule mining are proposed to efficiently address this problem~\cite{asudeh2019assessing}.
In this regard, we introduce \textit{Pattern Graph} data structure that exploits the relationship between patterns to do less work than computing all uncovered patterns by removing the non-maximal ones. 
The parent-child relationship between the patterns is represented in a graph that can be used to find better algorithms. 
\textit{Pattern-Breaker} starts from the top of the graph where the general patterns are and moves down by breaking each pattern into more specific ones. If a pattern is uncovered, then all of its descendants are also uncovered and they can not be an MUP, even if they have a parent that is covered. Therefore, this subgraph of the pattern graph can be pruned. 
The issue with \textit{Pattern-Breaker} is that it explores the covered regions of the pattern graph and for the cases where there are a few uncovered patterns, it has to explore a large portion of the exponential-size graph. 
To tackle this, \textit{Pattern-Combiner} algorithm is proposed that performs a bottom-up traversal of the pattern graph. It uses an observation that the coverage of a node at the level of the pattern graph can be computed as the sum of the coverage values of its children. 
The problem with \textit{Pattern-Combiner} is that it traverses over the uncovered nodes first and therefore, it will not perform well for the cases in which most of the nodes in the graph are uncovered. 
In fact, for the cases where most of the MUPs are placed in the middle of the graph, both \textit{Pattern-Breaker} and \textit{Pattern-Combiner} will not be as efficient as they should traverse half of the graph. Therefore, we propose \textit{Deep-Diver}, a search algorithm based on Depth-First-Search that quickly finds the MUPs, and uses them to limit the search space by pruning the nodes both dominating and dominated by the discovered MUPs.

\begin{figure*}[!tb]
    \begin{minipage}[t]{0.31\linewidth}
        \centering
        \includegraphics[width=\textwidth]{submissions/submission1/shahbazi/covcube1.jpg}
        \caption{\small Categorical attributes: the uncovered region of a toy example, as the collection of three MUPs.}
        \label{fig:covcube1}
    \end{minipage}
    \hfill
    \begin{minipage}[t]{0.31\linewidth}
        \centering
        \includegraphics[width=\textwidth]{submissions/submission1/shahbazi/cvrg_2_1.jpg}
        \caption{\small Continuous attributes, 2D: identifying the covered region in the gray Voronoi cell.}
        \label{fig:cvrg_2_1}
    \end{minipage}
    \hfill
    \begin{minipage}[t]{0.31\linewidth}
        \centering
        \includegraphics[width=\textwidth]{submissions/submission1/shahbazi/cvrg_2_2.jpg}
        \caption{ \small Continuous attributes, 2D: Uncovered region marked in red.}
        \label{fig:cvrg_2_2}
    \end{minipage}
\vspace{-5mm}
\end{figure*}

\subsubsection{Continuous Attributes}
Data in the real world often consists of a combination of continuous and discrete values. While simple solutions like binning {\tt age} into {\tt young} and {\tt old} can transform the continuous space into discrete. However, they may lead to coarse groupings that are sensitive to the thresholds chosen. It may be inappropriate to treat a 35-yo as {\tt young} but a 36-yo as {\tt old}. 
Therefore, we extend the notion of coverage to continuous space. Particularly, given data set $\dee$ with $n$ tuples over $d$ attributes, and vicinity radius $\rho$ and coverage threshold $k$, we want to identify the uncovered region -- the universe of uncovered query points.
A query point in continuous data space is covered if there are enough (at least $k$) data points in its $\rho$-vicinity neighborhood. $\rho$-vicinity neighborhood is the circle centered at the query point with radius $\rho$.

Depending on the number of attributes in a data set, we propose two algorithms for identifying uncovered regions in data~\cite{asudeh2021coverage}. 
The first algorithm known as \textit{Uncovered-2D} studies coverage over two-dimensional data sets where $\mathbf{x}=\{x_1,x_2\}$. To find the number of circles that a query point falls into and consequently discover the uncovered region, \textit{Uncovered-2D} makes a connection to $k$-th order Voronoi diagrams.
Consider a data set $\mathcal{D}$ and its corresponding $k$-th order Voronoi diagram. For every tuple $t\in \mathcal{D}$, let $\circ_t$ be the $d$-dimensional sphere ($d$-sphere) with radius $\rho$ centered at $t$.
Consider a $k$-voronoi cell $\mathcal{V}(S)$ in the $k$-th order Voronoi diagram $V_k(\mathcal{D})$.
Any point $q$ inside the intersections of the $d$-spheres of tuples in $S$, i.e. $q\in \underset{\forall t\in S}{\cap ~\circ_t}$, is covered, while all other points in the region are uncovered.
 The algorithm starts by constructing the $k$-th order Voronoi diagram of the data set and then for each Voronoi cell $\mathcal{V}(S)$ in the diagram, it computes the intersection of the circles of the tuples in $S$ and marks the portion of $\mathcal{V}(S)$ that falls outside it as uncovered.
After identifying the uncovered region, a 2D map of $\{x_1,x_2\}$ value combinations is used to report the region to the user.
The algorithm for the 2D case can be extended to the general case by relaxing the assumption on the number of attributes to discover the exact uncovered region, however, due to the curse of dimensionality, the search size space explodes as the number of dimensions increases and as a result, the algorithm will not be practical. Therefore, we propose a randomized approximation algorithm based on the geometric notion of \enet. 
Let $\mathcal{X}$ be a set and $\mathcal{R}$ be a set of subsets of $\mathcal{X}$. A set $\mathcal{N}\subset \mathcal{X}$ is an \enet for $\mathcal{X}$ if for any range $r\in\mathcal{R}$, if  $|r\cap \chi|>\eps|\chi|$, then $r$ contains at least one point of $N$.
The idea, at a high level, is to draw enough random samples from the space of potential query points to form an \enet. 
We then label the sampled query points as $\{-1,+1\}$ depending on whether those are covered or not, and learn the uncovered regions using the samples.

\subsection{Image Data}
Many known incidents of machine failures due to the lack of representation were on image data.
We consider an image data set with a fixed number of low-cardinality sensitive attributes such as {\tt\small race} and {\tt\small gender}. 
It is common that image data sets {\it lack explicit values} for sensitive attributes, which are crucial for coverage identification. An image data set is often a collection of images from different domains with little to no information about their domain and which groups they belong to. As a result, even studying coverage over low-cardinality and categorical attributes of interests is challenging in these cases.

\begin{wrapfigure}{R}{0.42\textwidth}
\centering
\vspace{-3mm}
\scriptsize
\begin{tabular}{|@{}c|@{}c@{}|@{}c@{}|@{}c@{}|} 
 \hline
{\bf data set} & {\bf classifier} & {\bf accuracy} & {\bf precision} \\ 
 &  &  & {\bf on female} \\ \hline
UTKFace:~& DeepFace (opencv) & 93.56 & {52.02}\\\cline{2-4}
({\tt females}=200,& DeepFace (retinaface) & 94.16 & {56.15}\\\cline{2-4}
{\tt males}=2800) & BaseCNN & 97.6 & 74.8\\
\hline
UTKFace:~& DeepFace (opencv) & 96.53 & {\bf 8.0}\\\cline{2-4}
({\tt females}=20,& DeepFace (retinaface) & 96.43 & {\bf 10.09}\\\cline{2-4}
{\tt males}=2980)& BaseCNN & 97.6 & {\bf 21.59}\\
\hline
\end{tabular}
\vspace{-3mm}
\caption{\small ML models' low performance for females in the presence of representation bias.~\cite{mousavi2024data}}\label{fig:mlfails}
\vspace{-3mm}
\end{wrapfigure}

In Figure~\ref{fig:mlfails}, we show that due to the issues such {\it machine bias} and {\it lack of distribution generalizability},
solely relying on state-of-the-art machine learning (ML) techniques fail to effectively identify lack of coverage in image data sets. Therefore, we propose an approach based on combining crowdsouring with ML~\cite{mousavi2024data}. 
Crowdsourcing is particularly promising for image data, for tasks such as image labeling, which, while challenging for the machine, are "easy" for human beings to conduct with minimal error. 

A key observation that enables a cost-effective crowdsourcing approach is that, while studying coverage, we would only like to find out if there are {\it enough tuples from each subgroup}.
Suppose a subgroup is covered if there are $\tau=100$ instances of it in the data set. Assume the (majority) group $\gee_1$ contains $n_1 \gg 100$ objects in the data set. 
To verify that $\gee_1$ is covered, it is enough for the crowd to discover 100 of those objects, not the entire $n_1$. 
Following this, $O(\tau)$ provides a lower bound on the number of crowd tasks required to verify a given group is covered. 
Still, this lower bound only holds for the groups that are covered, i.e., there is at least $\tau$ of those in the data set.
Surprisingly, verifying that a minority group is indeed uncovered is cumbersome, unlike the majority group.
This is because even though discovering $\tau$ objects from a group is enough for verifying that it is covered, one cannot {\it verify} a group is uncovered until there is a chance that the data set might still have enough objects from that group. Thus, assuming a non-zero probability for each unlabeled object to belong to each group, {one might need to ask the crowd to label the entire data set before they can confirm that a specific group is uncovered}.

Our idea for addressing this challenge is to
design {\it a divide and conquer algorithm} that, instead of {point queries}, uses {\it set queries} to iteratively eliminate subsets of data that {does not include any object from the given group}.
At a high level, our idea is to ask a set query from the crowd, inquiring whether the selected set contains at least one object from the given group $\gee$.
The user may provide two responses (yes/no). 
Interestingly, {in either case}, the user response provides valuable information that helps efficiently identify the coverage.
If the answer is ``No'', the set does not include any object from the given group $\gee$. As a result, the algorithm can safely prune the set, asking no further questions about it. In particular, for a group that is not covered, one can expect to see no answers on large set queries helping to prune a significant portion of the data set quickly.
On the other hand, if the answer is ``yes'', the set contains {at least} one object from the group $\gee$. As a result, the algorithm cannot prune the subset since it can have any number (larger than one) of the objects in $\gee$.
At first glance, the queries with yes answers do not provide helpful information as the algorithm cannot prune the subset (hence it needs to divide it into smaller subsets).
However, a key observation is that {the algorithm will only observe a limited number of yes answers} before it stops.
The reason is that the number of set queries with yes answers provides a {lower-bound} on the number of objects from $\gee$ in the data set. As a result, the algorithm can stop as soon as the lower bound reaches $\tau$, knowing that $\gee$ is covered.
The D\&C approach verifies the data coverage for a given group, while our goal is to identify the uncovered regions for a given set of sensitive attributes. The next question is how to utilize this algorithm for efficient coverage identification on different scenarios of sensitive attributes, forming intersectional or non-intersectional groups.
In particular, how can we find maximal uncovered patterns?
Our idea is to apply sampling and aggregate estimation techniques to find the groups that even if merged are likely to still be uncovered. This will help reduce the coverage identification cost by running the D\&C approach for the merged groups once.
 %%%%%%%%%%%%%%%%%%%%%%%%%%%%%%%% RESOLUTION  %%%%%%%%%%%%%%%%%%%%%%%%%%%%%%%%
\section{Resolving Insufficient Representation}\label{sec:resolution}

Data integration~\cite{nargesian2021tailoring,nargesian2022responsible} and data augmentation~\cite{sharma2020data,DBLP:journals/jair/ChawlaBHK02,iosifidis2018dealing,celis2020data} are considered as the primary solutions for reducing data coverage issues in a data set. 
Data integration is promising when external sources of data are available. On the other hand, recent advancements in generative AI and foundation models have enabled efficient and effective augmentation of data sets with synthetic data. 
Therefore, in the following, we review two approaches, one from each category, in the context of lack of coverage resolution.

\subsection{Data Integration}\label{sec:resolution:integration}

Data integration is to consolidate data from different sources into a single, unified view. 
Although it is an effective solution to acquire additional data from different distributions,
there are sampling policy and cost-efficiency concerns that need to be examined.  
Therefore, {\it Data Distribution Tailoring ({\sc DT})} introduces data integration techniques for resolving insufficient representation of subgroups in a data set in the most cost-effective manner~\cite{nargesian2021tailoring}.
A query to {\sc DT} 
consists of a target schema, and a set of group distribution requirements in the form of the minimum counts (e.g., ``{\tt\small 1,000 breast cancer monitoring data in Chicago with at least 30\% label=positive, and at least 20\% black patients}''). 
Collecting a fresh sample from a data view is costly (monetary, human resources, and/or computation cost)~\cite{asudeh2022towards}.
Therefore, {\sc DT} focuses on satisfying the count requirements with minimum cost. 
Given an input query and a lake of available data sources, the first step is to discover a collection of candidate data views that satisfy the target schema.
Each data view $v_i$ is a projection-join $v_i = \Pi\big(D_{i1}\bowtie\cdots\bowtie D_{ik_i} \big)$, where $D_{ij}$ is a data set in a given data lake.
Let us suppose the data views are already discovered.
At a high level, {\sc DT} follows an iterative approach that at each iteration a data view is selected to be queried.
Each query to a data view has a fixed cost and returns a sample that may or may not satisfy the query constraints.
The samples that are either not fresh, or do not satisfy the query are discarded.
Hence, the essential question towards a cost-effective data integration is {\it what data view to query next}.
Depending on the available information about the data sources, various techniques may be employed. 

For the cases when the group distributions are known, the process of collecting the target data set is a sequence of iterative steps, where at every step, the algorithm chooses a data view, queries it, and if the obtained tuple contributes to one of the groups for which the count requirement is not yet fulfilled, it is kept, otherwise discarded. To do so, a {Dynamic Programming (DP)} algorithm is proposed. An optimal source at each iteration minimizes the sum of its sampling cost plus the expected cost of collecting the remaining required groups, based on its sampling outcome.
The DP algorithm, however, has a pseudo-polynomial time complexity. Hence, it quickly becomes intractable for cases where the minimum count requirements for the groups are not small. 
For cases where the (sensitive) attribute of interest is binary, such as (biological) {\tt sex}={\tt \{male, female\}}, and the cost to query data is similar from all sources, it turns out that the optimal strategy is to query the data source with {maximum probability of obtaining a sample from the minority group}.
Expanding the binary-attributes algorithm for non-binary cases, the problem can be modeled as an extension of the ``{\it coupon collector's}'' problem~\cite{motwani1995randomized}, where the goal is to collect $m_i$ instances from each coupon (group) $\gee_i$.
At each iteration, the coupon collector's algorithm identifies a data view as most promising and queries it. In simple terms, a data view with a smaller query cost and a higher chance of obtaining minority groups is more promising.


For the cases where the group distributions are unknown, we model DT as a {\it multi-armed bandit} problem, where every data view is modeled as an arm. 
Every arm has an unknown distribution of different groups while pulling an arm (i.e., querying the corresponding data view) has a cost.
During various iterations, the algorithms pull the arms in an order that its expected total {\it reward} is maximized.
Arguing that the reward of obtaining a tuple from a group is proportional to how rare this group is across different data views, 
we design the reward function based on the expected cost one needs to pay in order to collect a tuple from a specific group.  
As the bandit strategy, we adopt {\it Upper Confidence Bound (UCB)} to balance exploration and exploitation. At every iteration, for every arm, UCB computes confidence intervals for the expected reward and selects the arm with the maximum upper bound of reward to be explored next.

\subsection{Data Augmentation using Foundation Models}

While data integration provides a promising approach for resolving coverage issues in a data set, its effectiveness is limited to the availability of external data sources that are rich enough to find sufficient fresh samples from minority groups. This, however, is not always possible, especially since the minority samples are rare and not easy to obtain.
Fortunately, recent advancements in Generative AI and Foundation Models have enabled synthesizing samples that are otherwise challenging to obtain from the real world.

Therefore, as an alternative approach to data integration, we turn our attention to the Foundation Models and Generative AI for resolving the lack of coverage. 
Particularly, models such as {\sc DALL.E}\footnote{\url{https://openai.com/dall-e-2}} have emerged as powerful tools for generating multi-modal data such as image, audio, and video.
 
We formalize the foundation model \fm as a black-box function with the following inputs, that once queried synthesize an output tuple.
\begin{itemize}
    \item {\bf Prompt}: A natural language description providing instructions on the details of the tuple to be generated. For instance, a prompt for image generation might be ``A realistic photo of a white cat running in a backyard.''
    \item {\bf Guide}: In cases where only a prompt is provided, the foundation model uses its imagination to generate the requested tuple. For the previous example, the prompt of a cat image, the breed, size, background, and other details are generated based on the model's imagination. Alternatively, a guide can be provided to influence the generation process. The guide is formalized as a pair $(t,m)$ where $t$ is a tuple and $m$ is a mask specifying which parts of the guide tuple should be changed. Using the cat example, $t$ can be a cat image and $m$ can specify the foreground to be regenerated.
\end{itemize}

There are multiple challenges towards effective data set augmentations using foundation models. 
First, we have to determine the minimal set of synthetic tuples that once added to the original data set, under-representation issues are resolved.
Second, the generated images should follow the underlying distribution represented in the input data set. Third, the generated tuples should have high quality and look realistic to a human evaluator. Last but not least, given the (often monetary) cost associated with the queries to the foundation model, we should ensure the cost-effectiveness of the data set repair process.

\begin{wrapfigure}{L}{0.45\textwidth}
\centering
\vspace{-3mm}
\scriptsize
    \includegraphics[width=.45\textwidth]{submissions/submission1/shahbazi/enhanced_pipeline.png}
\vspace{-3mm}
\caption{\small Architecture of \fmsystem for image data augmentation for coverage enhancement.}\label{fig:chameleon}
% \vspace{-3mm}
\end{wrapfigure}

\noindent Figure~\ref{fig:chameleon} shows the architecture of our system \fmsystem \cite{chameleon} for coverage enhancement using DALL-E image generator.
To address the first challenge, we define the combinations-selection problem, which minimizes the total number of synthetic tuples for resolving lack of coverage of minorities at the most general level. We show the problem is {\sc NP}-hard, and propose a greedy approximation algorithm for it.
To address the second and third challenges, \fmsystem follows a {\it rejection sampling} strategy.
It views each tuple in the data set $\dee$ as an iid sample from the underlying distribution $\xi$ it represents. It uses the vector representations (embeddings) space to describe the distribution. Then, given a newly generated tuple, it employs the one-class support vector machine (OCSVM) approach proposed by Scholkopf et al.~\cite{scholkopf1999support} to reject the tuple if it does not follow $\xi$.
Moreover, it models the quality evaluation as hypothesis testing and rejects the samples that have a higher chance of being labeled as ``unrealistic'' by a random human evaluator.
Finally, to minimize the number of queries to the foundation model, we provide a guide tuple (and a mask), in addition to the prompt, to the foundation model. We model the guide-selection problem as {\it contextual multi-armed bandit} and propose a solution based on the contextual UCB for it.

Before concluding this section, let us provide some experiment results to demonstrate the effectiveness of data augmentation with \fmsystem. We use FERET DB \cite{phillips1998feret} for this experiment, which comprises 1199 individual images and serves as a standardized facial image database for researchers to develop algorithms and report results. All images in FERET DB share the same dimensions, pose, and facial expression.
First, we identified the (level-1) uncovered ethnicity groups, using the threshold 80. We then used \fmsystem and resolved the lack of coverage issues.
To evaluate the effectiveness of the system, we trained a CNN model to predict the race of each image within this dataset. We then retrained the identical CNN on the repaired training data. Importantly, our test dataset for both experiments remains consistent and is derived from real images.
Table~\ref{tab:lackofcoverage} presents the improvements in precision, recall, and F1 score metrics for under-represented groups after repairing the dataset. The results indicate an enhancement in performance metrics for all under-represented groups following the repair process.

\begin{table}[t]
    \centering
    \caption{Illustrating the effect of lack of coverage repair using \fmsystem on \texttt{FERTDB}}
    \label{tab:lackofcoverage}
    \vspace{-3mm}
    \begin{tabular}{lcccccccc}
        \toprule
         & \multicolumn{4}{c}{\textbf{Classifier Performance on \texttt{FERTDB}}} & \multicolumn{4}{c}{\textbf{Classifier Performance on Repaired}} \\
        \cmidrule(lr){2-5} \cmidrule(lr){6-9}
        \textbf{Ethnicity Groups}& \#Images & Precision & Recall & F1-Score & \#Images & Precision & Recall & F1-Score \\
        \midrule
        Overall          & 756 & 0.81 & 0.75 & 0.78 & 987 & 0.70 & 0.75 & 0.72 \\ \hline
        Black            & 40  & 0.19 & 0.22 & 0.16 & 100 & 0.48 & 0.56 & 0.52 \\
        Hispanic         & 19  & 0.50 & 0.17 & 0.25 & 100 & 0.62 & 0.36 & 0.45 \\
        Middle Eastern   & 10  & 0.00 & 0.00 & 0.00 & 100 & 0.20 & 0.41 & 0.27 \\
        \bottomrule
    \end{tabular}
\end{table}

 %%%%%%%%%%%%%%%%%%%%%%%%%%%%%%%% RELIABILITY  %%%%%%%%%%%%%%%%%%%%%%%%%%%%%%%%
\section{Generating Reliability Warnings}\label{sec:reliability}
% up to 2.5 pages
Interpretability is a necessity for data scientists who develop predictive models for critical decision-making.
In such settings, it is important to provide additional means to support the following question:
{\it is an individual prediction of the model reliable for decision-making?} Our goal is to use the lack of representation to help decision-makers find insights about this critical question.
To further motivate this, let us use the following example:

\vspace{1mm}
\begin{example}\label{ex-0}
{\bf(Part1):} Consider a judge who needs to decide whether to accept or deny a bail request. Using data-driven predictive models is prevalent in such cases for predicting recidivism~\cite{dressel2018accuracy}.
Indeed, such models can be beneficial to help the judge make wise decisions.
Suppose the model predicts the queried individual as high risk (or low risk).
The judge is aware and concerned about the critics surrounding such models.
A major question the judge faces is whether or not they should rely on the prediction outcome to take action for this case.
Furthermore, if, for instance, they decide to ignore the outcome and hence they need to provide a statement supporting their action, what evidence can they provide? 
\end{example}

In line with the recent trend on data-centric AI~\cite{ng2021mlops}, we design {novel approaches}, {complimentary} to the existing work on trustworthy AI~\cite{wing2021trustworthy,kentour2021analysis,liu2021trustworthy,singh2021trustworthy}, to address the aforementioned trust question through the lens of {\it data}.
In particular, unlike existing works that generate trust information from a {\it given \underline{model}}, we associate {\it \underline{data sets} with proper measurements} that specify their {\it the scope of use for predicting future cases}.
We note that a predictive model provides only probabilistic guarantees on the \underline{average} loss over the distribution represented by the data set used for training it.
As a result, these predictions may not be distribution generalizable~\cite{kulynych2022you}.
Consequently, if the query point is {\it not represented} by the data, the guarantees may not hold, hence one cannot rely on the prediction outcome.
Besides, an essential requirement for a learning algorithm is that its training data $\dee$ should represent the underlying distribution $\dist$.
Even if so, the trained model $h$ only provides a probabilistic guarantee on the {expected} loss on random samples from $\dist$.  
A model that performs well on {\it majority} of samples drawn from $\dist$ will have a high performance on average. Still, as we observed in Figure~\ref{fig:mlfails},
its performance for {\it minorities} and points that are not represented is questionable. Let us consider the following toy example:

\begin{figure*}[!b] 
    \begin{minipage}[t]{0.32\linewidth}
        	\centering
        	\includegraphics[width=\textwidth]{submissions/submission1/shahbazi/example_1.png} 
        	\vspace{-9mm}\caption{\small Data set $\dee$ generated using a Gaussian distribution; $x_1$ and $x_2$ are positively correlated}
            \label{fig:ex1:1}
    \end{minipage}
    \hfill
    \begin{minipage}[t]{0.32\linewidth}
        \centering
        	\includegraphics[width =\textwidth]{submissions/submission1/shahbazi/example_2.png} 
        	\vspace{-9mm}\caption{\small The decision boundary of learned model $h$ and query points $\qu^1$ to $\qu^4$}
            \label{fig:ex1:2}
    \end{minipage}
    \hfill
    \begin{minipage}[t]{0.32\linewidth}
        	\centering
        	\includegraphics[width =\textwidth]{submissions/submission1/shahbazi/example_3.png}
        	\vspace{-9mm}\caption{\small Ground-truth boundary, overlaid on the model decision boundary and query points}
            \label{fig:ex1:3}
    \end{minipage}
    \vspace{-5mm}
\end{figure*} 

\vspace{1mm}
\begin{example}\label{ex-1}
Consider a binary classification task where the input space is $\ex=\langle x_1, x_2\rangle$ and the output space is the binary label $y$ with values $\{-1$ (red) $,+1$ (blue)$\}$.
Suppose the underlying data distribution $\dist$ follows a 2D Gaussian, where $x_1$ and $x_2$ 
are positively correlated as shown in Figure~\ref{fig:ex1:1}.
The figure shows the data set $\dee$ drawn independently from the distribution $\dist$, along with their labels as their colors.
Using $\dee$, the prediction model $h$ is constructed as shown in Figure~\ref{fig:ex1:2}. 
The decision boundary is specified in the picture; while any point above the line is predicted as +1, a query point below it is labeled as -1.
The classifier has been evaluated using a test set that is an iid sample set drawn from the underlying data set $\dist$. The accuracy on the test set is high (above 90\%), and hence, the model gets deployed.
We cherry-picked four query points, $\qu^1$ to $\qu^4$, that are also included in Figure~\ref{fig:ex1:2}. Using $h$ for prediction, $h(\qu^1)=-1$, $h(\qu^2)=+1$,  $h(\qu^3)=+1$, and $h(\qu^4)=-1$.
Figure~\ref{fig:ex1:3} adds the ground-truth boundary to the search space, revealing the true label of the query points: every point inside the red circle has the true label $-1$ while any point outside of it is $+1$.
Looking at the figure, $y^1=+1$ while the model predicted it as $h(\qu^1)=-1$.  \hfill$\square$
\end{example}
\vspace{2mm}

Let us take a closer look at the four query points in this example and their placement with regard to the tuples in $\dee$ used for training $h$. 
$\qu^2$ belongs to a {\it dense region} with many training tuples in $\dee$ surrounding it. Besides, all of the tuples in its vicinity have the same label $y=+1$. As a result, one can expect that the model's outcome $h(\qu^2)=+1$ should be a reliable prediction.
Similar to $\qu^2$, $\qu^4$ also belongs to a dense region in $\dee$; however, $\qu^4$ belongs to an {\it uncertain region}, where some of the tuples in its vicinity have a label $y=+1$, and some others have the label $y=-1$. Considering the uncertainty in the vicinity of $\qu^4$, one cannot confidently rely on the outcome of the model $h$. 
On the other hand, the neighbors of $\qu^1$ (resp. $\qu^3$) are not uncertain, all having the label $y=-1$ (resp. $y=+1$).
However, the query points $\qu^1$ and $\qu^3$ are not well represented by $\dee$. In other words, $\qu^1$ and $\qu^3$ are unlikely to be generated according to the underlying distribution $\dist$, represented by $\dee$. As a result, following the no-free-lunch theorem~\cite{kakade2003sample}, one cannot expect the outcome of model $h$ to be reliable for these points.
Looking at the ground-truth boundary in Figure~\ref{fig:ex1:3}, $h$ luckily predicted the outcome for $\qu^3$ correctly, but it was not fortunate to predict the $y^1$ correctly.
Nevertheless, 
since the model is not reliably trained for these points, 
its outcome for these query points is not trustworthy.

From Example~\ref{ex-1}, we observe that the outcome of a model $h$, trained using a data set $\dee$ is not reliable for a query point $\qu$, if:
\begin{itemize}
    \item {\bf Lack of representation:} $\qu$ is not well-represented by $\dee$.
    In such cases, the model has not seen ``enough'' samples similar to $\qu$ to reliably learn and predict the outcome of $\qu$.
    \item {\bf Lack of certainty:} $\qu$ belongs to an uncertain region, where different tuples of $\dee$ in the vicinity of $\qu$ have different target values. $\qu$ belongs to a high-fluctuating area, where tuples in the vicinity of $\qu$ have a wide range of values.
\end{itemize} \vspace{2mm}

\noindent
Based on these two observations, we propose Representation-and-Uncertainty ({\bf RU}) measures.
To identify if a query suffers from uncertainty or lack of representation, one could use a deterministic approach using a fixed threshold. Then if the number of similar samples to (resp. label fluctuation in vicinity of) $\qu$ is larger than the threshold it is considered as unrepresented (resp. uncertain).
This approach, however, would be misleading since two numbers close to the threshold could be treated very differently. Also, all points on each side of the threshold would be considered equally represented (resp., certain). Instead, we consider {\it a randomized approach}, widely popular in the literature, including~\cite{dwork2012fairness}.
That is, instead of using fixed thresholds, a Bernoulli variable (a biased coin) is used that 
assigns $\qu$ as unrepresented (resp., uncertain) based on the number of samples similar to it (resp., its neighborhood uncertainty).
Given a query point $\qu$, let $\pe_o$ be the probability indicating if $\qu$ is not represented and let $\pe_u$ be the probability indicating if $\qu$ belongs to an uncertain region. 
We represent the probability of the Bernoulli variables for lack of representation or uncertainty components as $\pe_o$ and $\pe_u$, respectively. Note that the two Bernoulli variables $\pe_o$ and $\pe_u$ are independent from each other. That simply follows the argument that after specifying the number of similar samples to $\qu$ whether or not it should be considered as unrepresented does not depend on the uncertainty in the neighborhood of $\qu$.

\begin{definition}[\sru]\label{def:sdt}
The \sru is a probabilistic measure that considers the outcome of a model for a query point $\qu$ untrustworthy if $\qu$ is not represented by $\dee$ {\it and} it belongs to an uncertain region.
Formally, the \sru measure is:
\begin{align} 
    \nonumber
    SRU(\qu) &= \pe\big((\qu \mbox{ is outlier}) \wedge (\qu \mbox{ belongs to uncertain region})\big) 
\end{align}
Since $\pe_o$ and $\pe_u$ are independent:

\vspace{-13mm}
\begin{align} \label{eq:strong}
    SRU(\qu) &= \pe_o(\qu) \times \pe_u(\qu)
\end{align}
\end{definition}

\sru raises the warning signal only when the query point fails on {\it both} conditions of being represented by $\dee$ and not belonging to an uncertain region. 
For instance, in Example~\ref{ex-1} none of the query points fail both on representation and on uncertainty; hence neither has a high \sru score.
On the other hand, 
a high \sru score for a query point $\qu$ {\it provides a strong warning signal} that one should perhaps reject the model outcome and not consider it for decision-making.

\sru is a strong signal that raises warnings only for the fearfully concerning cases that fail both on representation and uncertainty.
However, as observed in Example~\ref{ex-1} a query points failing {\it at least} one of these conditions may also not be reliable, at least for critical decision making.
We define the \wru measure to raise a warning for such cases.

\begin{definition}[\wru]\label{def:wdt}
The \wru measure is a probabilistic measure that considers the outcome of a model for a query point $\qu$ untrustworthy if $\qu$ is not represented by $\dee$ {\bf or} it belongs to an uncertain region.
Formally, the \wru is computed as:
\begin{align} \label{eq:weak}
    WRU(\qu) = \pe\big((\qu \mbox{ is outlier}) \vee (\qu \mbox{ belongs to uncertain region})\big) 
    = \pe_o(\qu) + \pe_u(\qu) - \pe_o(\qu) \times \pe_u(\qu)
\end{align}
\end{definition}

Proposing quantitative probabilistic outcomes, \ru measures are interpretable for the users, since beyond the scores, the uncertainty and lack of representation components provide an explanation to justify them. 
Please refer to \cite{techrep} for more details on how to efficiently and effectively compute the representation ($\pe_o$) and uncertainty ($\pe_u$) probabilities, using only $\dee$.
In Example~\ref{ex-0}, let us see how the \ru measures can be helpful.

\noindent{\bf Example 1. (part 2):}
{\it RU measures \underline{raise warning} when
the fitness of the data set used for drawing a prediction is questionable, helping the judge to be cautious when taking action.
Besides, these measures provide \underline{quantitative evidence} to support the judge's action when they decide to ignore a prediction outcome that is not trustworthy.
The judge, for example, can argue to ignore a model outcome for a specific case, based on the insight that 
the model has been built using a
data set that fails to represent the given case.}
\hfill$\square$

Finally, let us demonstrate the efficacy of \ru measures through a series of experiments. Since the \ru measures are {\it data-centric},
those are applicable for both classification and regression tasks, irrespective of the model used.
We use {\it Adult} dataset~\cite{adult} for classification and {\it House Sales in King County} dataset for the validation of regression tasks. From each dataset, we uniformly sample two sets from the underlying distribution. The first set serves as the training set to compute the \ru values, and the second one is used as the test set from which the queries are drawn. We validate our proposal by providing the correlation between the \ru values and the performance of an ML model's prediction on the same data. 

We start by computing the \ru values for all the query points in the test set. Next, we bucketize the query points based on their \ru values in equi-width buckets of width 0.1. We repeat this for both \sru and \wru measures. Next, we train a model on the training data set and predict the target variable for the points in each range of \ru measure. The validation results for the classification task on the {\it Adult} dataset are presented in Figures \ref{fig:exp-adult-sdt} and \ref{fig:exp-adult-wdt}. Each figure corresponds to the accuracy/error measures of the classifier over each bucket of \ru values for \sru and \wru. As the \ru values increase, the accuracy of the model drops while the FPR rises, and therefore, the model fails to capture the ground truth for the points that fall into untrustworthy regions in the data set. By repeating the aforementioned steps for the regression task on the {\it House Sales in King County} dataset, we observe similar results presented in Figures \ref{fig:exp-hs-sdt} and \ref{fig:exp-hs-wdt}. 
As the \ru value increases, the RSS of the regression model follows the same trend denoting that the model fails to perform for tuples with a high \ru value.

\begin{figure}[!tb]
    \begin{minipage}[t]{0.24\linewidth}
        \centering
        \includegraphics[width=\textwidth]{submissions/submission1/shahbazi/sdt_adult.pdf}
        \vspace{-6mm}\caption{\small{\it Adult}, efficacy of \sru  on classification}
        \label{fig:exp-adult-sdt}
    \end{minipage}\hfill
    \begin{minipage}[t]{0.24\linewidth}
        \centering
        \includegraphics[width=\textwidth]{submissions/submission1/shahbazi/wdt_adult.pdf}
        \vspace{-6mm}\caption{\small{\it Adult}, efficacy of \wru  on classification}
        \label{fig:exp-adult-wdt}
    \end{minipage}\hfill
    \begin{minipage}[t]{0.24\linewidth}
        \centering
        \includegraphics[width=\textwidth]{submissions/submission1/shahbazi/sdt_regression_house.pdf}
        \vspace{-6mm}\caption{\small{\it House Sales in King County}, efficacy of \sru on regression}
        \label{fig:exp-hs-sdt}
    \end{minipage}\hfill
    \begin{minipage}[t]{0.24\linewidth}
        \centering
        \includegraphics[width=\textwidth]{submissions/submission1/shahbazi/wdt_regression_house.pdf}
        \vspace{-6mm}\caption{\small{\it House Sales in King County}, efficacy \wru on regression}
        \label{fig:exp-hs-wdt}
    \end{minipage}
\vspace{-5mm}
\end{figure}
 %%%%%%%%%%%%%%%%%%%%%%%%%%%%%%%% RELATED WORK  %%%%%%%%%%%%%%%%%%%%%%%%%%%%%%%%
\section{Related Work}\label{related} 

Bias in data has been looked at for a long time in statistical community~\cite{neyman1936contributions} but social data presents different challenges~\cite{olteanu2019social,fairmlbook,barocas2016big,jk2019bias,drosou2017diversity}.
The diversity and representativeness of data have been widely studied~\cite{drosou2017diversity}, in fields such as social science~\cite{berrey2015enigma, dobbin2016diversity,simpson1949measurement}, political science~\cite{surowiecki2005wisdom}, and information retrieval~\cite{agrawal2009diversifying}. 
Tracing back machine bias to its source, there have been major efforts to identify different types~\cite{mehrabi2021survey, olteanu2019social,friedman1996bias} and sources~\cite{torralba2011unbiased,crawford2013hidden,diakopoulos2015algorithmic} of biases in data. Efforts to satisfy {\it responsible data} requirements~\cite{nargesian2022responsible} extend to various stages of the data analysis pipeline, including data annotation~\cite{li2020towards,lazier2023fairness}, data cleaning and repair~\cite{SalimiRHS19,tae2019data,salimi2020database}, data imputation~\cite{martinez2019fairness}, entity resolution~\cite{shahbazi2023through,fanourakis2023fairer}, data integration~\cite{nargesian2022responsible,nargesian2021tailoring}, etc. 

\paragraph{Data Coverage:}The notion of data coverage has received extensive attention from different angles. Detecting lack of coverage has been studied for datasets with discrete~\cite{asudeh2019assessing} and continuous~\cite{asudeh2021coverage} attributes populated in single or multiple \cite{lin2020identifying} relations.
To resolve insufficient coverage, \cite{accinelli2020coverage, accinelli2021impact,shetiya2022fairness}
consider resolving representation bias in preprocessing pipelines by rewriting queries into the closest operation so that certain subgroups are sufficiently represented in the downstream tasks. Alternatively, ~\cite{asudeh2019assessing,tae2021slice} propose a data collection strategy to acquire as little additional data as possible (to minimize the associated costs) to meet the representation constraints. ~\cite{sharma2020data,iosifidis2018dealing,celis2020data} opt for a data augmentation approach by adding partially altered duplicates of already existing tuples or generating new synthetic entries from existing data. Consequently, the new data set has an equal number of elements for different groups, resulting in potentially resolving the under-representation issues. Finally,  \cite{nargesian2021tailoring} utilizes data integration techniques to consolidate data from different sources into a single dataset to resolve representation bias.
Related works also include ~\cite{chung2019slice,sagadeeva2021sliceline,tae2021slice} that seek to understand if the overall performance of the model fails to reflect and performs poorly on certain slices in the data.
As alternative approaches to measure representation bias, the notion of representation rate~\cite{celis2020data} (a.k.a. equal base rate~\cite{kleinberg2016inherent}) is introduced which compared with coverage, it is more restrictive as it requires almost equal ratios from different groups.
Please refer to \cite{shahbazi2023representation} for a comprehensive survey about representation bias in data. 

\paragraph{ML Reliability:} Model-centric works for uncertainty quantification such as 
probabilistic classifiers~\cite{zadrozny2001obtaining,zadrozny2002transforming,platt1999probabilistic,niculescu2005predicting},
prediction intervals (PIs) \cite{chatfield93predictionintervals,pearce2018high,khosravi2010lower} and conformal predictions (CP)~\cite{angelopoulos2021gentle,shafer2008tutorial} that are used for measuring prediction uncertainty, are built
by maximizing the {\it expected performance} on {\it random} sample from the underlying distribution.
As a result, while providing accurate estimations for the dense regions of data (e.g. majority groups), their estimation accuracy is questionable for the poorly represented regions.
In particular, \cite{angelopoulos2021gentle} recognizes the lack of guarantees in the performance of CP for such regions.
Besides, the bulk of work on trustworthy AI provides information that {\it supports} the outcome of an ML model. For example, existing work on explainable AI, including~\cite{harradon2018causal,ribeiro2016should,gunning2019darpa}, aims to find simple explanations and rules that justify the outcome of a model.
Conversely, we aim to {\it raise warning signals} when the outcome of a model is {\it not} trustworthy. That is, to provide reasons that {\it cast doubt} on the reliability of the model outcome {for a given query point}.

 %%%%%%%%%%%%%%%%%%%%%%%%%%%%%%%% FUTURE  %%%%%%%%%%%%%%%%%%%%%%%%%%%%%%%%
% \vspace{-3mm}
\section{Final Remarks}\label{sec:conclusion}
As Data-centric AI and Responsible AI emerge as focal points in data science research, the development of Data-centric methodologies for ensuring Responsible and Trustworthy AI attracts increasing attention.
While there is some excellent work on responsible data management to achieve this goal, there remain many challenges yet to be addressed.

In this paper, we focused on a crucial aspect of responsible data -- detecting and addressing the under-representation of minorities within a data set.
We formally defined the notion of data coverage and discussed various techniques for (a) identifying lack of representation issues across different data modalities, (b) ensuring proper representation of minorities in data, and (c) limiting the scope-of-use of data sets based on their representation issues by generating proper ({\sc RU}) warning signals.
Even though the research on detecting lack of coverage issues is relatively mature, resolution techniques are still understudied.
Considering the recent advancements in Generative AI, utilizing Foundation Models and Large Language Models, and studying their limitations, for data augmentation to improve the representation of minorities at the data level seems interesting to further explore.

 %%%%%%%%%%%%%%%%%%%%%%%%%%%%%%%% BIB  %%%%%%%%%%%%%%%%%%%%%%%%%%%%%%%%
\bibliographystyle{unsrt}
\small
% \bibliography{ref}
\begin{thebibliography}{10}

\bibitem{asudeh2019assessing}
A.~Asudeh, Z.~Jin, and H.~Jagadish.
\newblock Assessing and remedying coverage for a given dataset.
\newblock In {\em ICDE}, pages 554--565. IEEE, 2019.

\bibitem{shahbazi2023representation}
N.~Shahbazi, Y.~Lin, A.~Asudeh, and H.~Jagadish.
\newblock Representation bias in data: A survey on identification and resolution techniques.
\newblock {\em ACM Computing Surveys}, 2023.

\bibitem{asudeh2021coverage}
A.~Asudeh, N.~Shahbazi, Z.~Jin, and H.~V. Jagadish.
\newblock Identifying insufficient data coverage for ordinal continuous-valued attributes.
\newblock In {\em SIGMOD}. ACM, 2021.

\bibitem{mousavi2024data}
M.~Mousavi, N.~Shahbazi, and A.~Asudeh.
\newblock Data coverage for detecting representation bias in image datasets: {A} crowdsourcing approach.
\newblock In {\em {EDBT}}, pages 47--60, 2024.

\bibitem{nargesian2021tailoring}
F.~Nargesian, A.~Asudeh, and H.~Jagadish.
\newblock Tailoring data source distributions for fairness-aware data integration.
\newblock {\em Proceedings of the VLDB Endowment}, 14(11):2519--2532, 2021.

\bibitem{nargesian2022responsible}
F.~Nargesian, A.~Asudeh, and H.~V. Jagadish.
\newblock Responsible data integration: Next-generation challenges.
\newblock {\em SIGMOD}, 2022.

\bibitem{sharma2020data}
S.~Sharma, Y.~Zhang, J.~M. R{\'\i}os~Aliaga, D.~Bouneffouf, V.~Muthusamy, and K.~R. Varshney.
\newblock Data augmentation for discrimination prevention and bias disambiguation.
\newblock In {\em AIES}, pages 358--364, 2020.

\bibitem{DBLP:journals/jair/ChawlaBHK02}
N.~V. Chawla, K.~W. Bowyer, L.~O. Hall, and W.~P. Kegelmeyer.
\newblock {SMOTE:} synthetic minority over-sampling technique.
\newblock {\em J. Artif. Intell. Res.}, 16:321--357, 2002.

\bibitem{iosifidis2018dealing}
V.~Iosifidis and E.~Ntoutsi.
\newblock Dealing with bias via data augmentation in supervised learning scenarios.
\newblock {\em Jo Bates Paul D. Clough Robert J{\"a}schke}, 24, 2018.

\bibitem{celis2020data}
L.~E. Celis, V.~Keswani, and N.~Vishnoi.
\newblock Data preprocessing to mitigate bias: A maximum entropy based approach.
\newblock In {\em ICML}, pages 1349--1359. PMLR, 2020.

\bibitem{asudeh2022towards}
A.~Asudeh and F.~Nargesian.
\newblock Towards distribution-aware query answering in data markets.
\newblock {\em Proceedings of the VLDB Endowment}, 15(11):3137--3144, 2022.

\bibitem{motwani1995randomized}
R.~Motwani and P.~Raghavan.
\newblock {\em Randomized algorithms}.
\newblock Cambridge university press, 1995.

\bibitem{chameleon}
M.~Erfanian, H.~V. Jagadish, and A.~Asudeh.
\newblock Chameleon: Foundation models for fairness-aware multi-modal data augmentation to enhance coverage of minorities.
\newblock {\em arXiv preprint arXiv:2402.01071}, 2024.

\bibitem{scholkopf1999support}
B.~Sch{\"o}lkopf, R.~C. Williamson, A.~Smola, J.~Shawe-Taylor, and J.~Platt.
\newblock Support vector method for novelty detection.
\newblock {\em NeurIPS}, 12, 1999.

\bibitem{phillips1998feret}
P.~J. Phillips, H.~Wechsler, J.~Huang, and P.~J. Rauss.
\newblock The feret database and evaluation procedure for face-recognition algorithms.
\newblock {\em Image and vision computing}, 16(5):295--306, 1998.

\bibitem{dressel2018accuracy}
J.~Dressel and H.~Farid.
\newblock The accuracy, fairness, and limits of predicting recidivism.
\newblock {\em Science advances}, 4(1):eaao5580, 2018.

\bibitem{ng2021mlops}
A.~Ng.
\newblock Mlops: From model-centric to data-centric {AI}.
\newblock 2021.

\bibitem{wing2021trustworthy}
J.~M. Wing.
\newblock Trustworthy {AI}.
\newblock {\em CACM}, 64(10):64--71, 2021.

\bibitem{kentour2021analysis}
M.~Kentour and J.~Lu.
\newblock Analysis of trustworthiness in machine learning and deep learning.
\newblock {\em InfoComp}, 2021.

\bibitem{liu2021trustworthy}
H.~Liu, Y.~Wang, W.~Fan, X.~Liu, Y.~Li, S.~Jain, A.~K. Jain, and J.~Tang.
\newblock Trustworthy {AI}: A computational perspective.
\newblock {\em arXiv preprint arXiv:2107.06641}, 2021.

\bibitem{singh2021trustworthy}
R.~Singh, M.~Vatsa, and N.~Ratha.
\newblock Trustworthy {AI}.
\newblock In {\em 8th ACM IKDD CODS and 26th COMAD}, pages 449--453. 2021.

\bibitem{kulynych2022you}
B.~Kulynych, Y.-Y. Yang, Y.~Yu, J.~B{\l}asiok, and P.~Nakkiran.
\newblock What you see is what you get: Distributional generalization for algorithm design in deep learning.
\newblock {\em arXiv preprint arXiv:2204.03230}, 2022.

\bibitem{kakade2003sample}
S.~M. Kakade.
\newblock {\em On the sample complexity of reinforcement learning}.
\newblock University of London, University College London (United Kingdom), 2003.

\bibitem{dwork2012fairness}
C.~Dwork, M.~Hardt, T.~Pitassi, O.~Reingold, and R.~Zemel.
\newblock Fairness through awareness.
\newblock In {\em ITCS}, pages 214--226, 2012.

\bibitem{techrep}
N.~Shahbazi and A.~Asudeh.
\newblock Data-centric reliability evaluation of individual predictions.
\newblock {\em CoRR, abs/2204.07682}, 2022.

\bibitem{adult}
M.~Lichman.
\newblock Adult income dataset, {UCI} machine learning repository.
\newblock \url{https://archive.ics.uci.edu/ml/datasets/adult}, 2013.

\bibitem{neyman1936contributions}
J.~Neyman and E.~S. Pearson.
\newblock Contributions to the theory of testing statistical hypotheses.
\newblock {\em Statistical Research Memoirs}, 1936.

\bibitem{olteanu2019social}
A.~Olteanu, C.~Castillo, F.~Diaz, and E.~Kiciman.
\newblock Social data: Biases, methodological pitfalls, and ethical boundaries.
\newblock {\em Frontiers in Big Data}, 2:13, 2019.

\bibitem{fairmlbook}
S.~Barocas, M.~Hardt, and A.~Narayanan.
\newblock Fairness and machine learning: Limitations and opportunities.
\newblock \url{fairmlbook.org}, 2019.

\bibitem{barocas2016big}
S.~Barocas and A.~D. Selbst.
\newblock Big data's disparate impact.
\newblock {\em Calif. L. Rev.}, 104:671, 2016.

\bibitem{jk2019bias}
J.~Kleinberg.
\newblock Fairness, rankings, and behavioral biases.
\newblock FAT*, 2019.

\bibitem{drosou2017diversity}
M.~Drosou, H.~Jagadish, E.~Pitoura, and J.~Stoyanovich.
\newblock Diversity in big data: A review.
\newblock {\em Big data}, 5(2):73--84, 2017.

\bibitem{berrey2015enigma}
E.~Berrey.
\newblock {\em The enigma of diversity: The language of race and the limits of racial justice}.
\newblock University of Chicago Press, 2015.

\bibitem{dobbin2016diversity}
F.~Dobbin and A.~Kalev.
\newblock Why diversity programs fail and what works better.
\newblock {\em Harvard Business Review}, 94(7-8):52--60, 2016.

\bibitem{simpson1949measurement}
E.~H. Simpson.
\newblock Measurement of diversity.
\newblock {\em Nature}, 163(4148), 1949.

\bibitem{surowiecki2005wisdom}
J.~Surowiecki.
\newblock {\em The wisdom of crowds}.
\newblock Anchor, 2005.

\bibitem{agrawal2009diversifying}
R.~Agrawal, S.~Gollapudi, A.~Halverson, and S.~Ieong.
\newblock Diversifying search results.
\newblock In {\em WSDM}, pages 5--14. ACM, 2009.

\bibitem{mehrabi2021survey}
N.~Mehrabi, F.~Morstatter, N.~Saxena, K.~Lerman, and A.~Galstyan.
\newblock A survey on bias and fairness in machine learning.
\newblock {\em ACM Computing Surveys (CSUR)}, 54(6):1--35, 2021.

\bibitem{friedman1996bias}
B.~Friedman and H.~Nissenbaum.
\newblock Bias in computer systems.
\newblock {\em TOIS}, 14(3):330--347, 1996.

\bibitem{torralba2011unbiased}
A.~Torralba and A.~A. Efros.
\newblock Unbiased look at dataset bias.
\newblock In {\em CVPR 2011}, pages 1521--1528. IEEE, 2011.

\bibitem{crawford2013hidden}
K.~Crawford.
\newblock The hidden biases in big data.
\newblock {\em Harvard business review}, 1(4), 2013.

\bibitem{diakopoulos2015algorithmic}
N.~Diakopoulos.
\newblock Algorithmic accountability: Journalistic investigation of computational power structures.
\newblock {\em Digital journalism}, 3(3):398--415, 2015.

\bibitem{li2020towards}
Y.~Li, H.~Sun, and W.~H. Wang.
\newblock Towards fair truth discovery from biased crowdsourced answers.
\newblock In {\em SIGKDD}, pages 599--607, 2020.

\bibitem{lazier2023fairness}
S.~Lazier, S.~Thirumuruganathan, and H.~Anahideh.
\newblock Fairness and bias in truth discovery algorithms: An experimental analysis.
\newblock {\em arXiv preprint arXiv:2304.12573}, 2023.

\bibitem{SalimiRHS19}
B.~Salimi, L.~Rodriguez, B.~Howe, and D.~Suciu.
\newblock Interventional fairness: Causal database repair for algorithmic fairness.
\newblock In {\em {SIGMOD}}, pages 793--810. {ACM}, 2019.

\bibitem{tae2019data}
K.~H. Tae, Y.~Roh, Y.~H. Oh, H.~Kim, and S.~E. Whang.
\newblock Data cleaning for accurate, fair, and robust models: A big data-{AI} integration approach.
\newblock In {\em DEEM workshop}, pages 1--4, 2019.

\bibitem{salimi2020database}
B.~Salimi, B.~Howe, and D.~Suciu.
\newblock Database repair meets algorithmic fairness.
\newblock {\em ACM SIGMOD Record}, 49(1):34--41, 2020.

\bibitem{martinez2019fairness}
F.~Mart{\'\i}nez-Plumed, C.~Ferri, D.~Nieves, and J.~Hern{\'a}ndez-Orallo.
\newblock Fairness and missing values.
\newblock {\em arXiv preprint arXiv:1905.12728}, 2019.

\bibitem{shahbazi2023through}
N.~Shahbazi, N.~Danevski, F.~Nargesian, A.~Asudeh, and D.~Srivastava.
\newblock Through the fairness lens: Experimental analysis and evaluation of entity matching.
\newblock {\em Proceedings of the VLDB Endowment}, 16(11):3279--3292, 2023.

\bibitem{fanourakis2023fairer}
N.~Fanourakis, C.~Kontousias, V.~Efthymiou, V.~Christophides, and D.~Plexousakis.
\newblock Fairer demo: Fairness-aware and explainable entity resolution.
\newblock 2023.

\bibitem{lin2020identifying}
Y.~Lin, Y.~Guan, A.~Asudeh, and H.~Jagadish.
\newblock Identifying insufficient data coverage in databases with multiple relations.
\newblock {\em Proceedings of the VLDB Endowment}, 13(12):2229--2242, 2020.

\bibitem{accinelli2020coverage}
C.~Accinelli, S.~Minisi, and B.~Catania.
\newblock Coverage-based rewriting for data preparation.
\newblock In {\em EDBT Workshops}, 2020.

\bibitem{accinelli2021impact}
C.~Accinelli, B.~Catania, G.~Guerrini, and S.~Minisi.
\newblock The impact of rewriting on coverage constraint satisfaction.
\newblock In {\em EDBT Workshops}, 2021.

\bibitem{shetiya2022fairness}
S.~Shetiya, I.~P. Swift, A.~Asudeh, and G.~Das.
\newblock Fairness-aware range queries for selecting unbiased data.
\newblock In {\em ICDE}. IEEE, 2022.

\bibitem{tae2021slice}
K.~H. Tae and S.~E. Whang.
\newblock Slice tuner: A selective data acquisition framework for accurate and fair machine learning models.
\newblock In {\em SIGMOD}, pages 1771--1783, 2021.

\bibitem{chung2019slice}
Y.~Chung, T.~Kraska, N.~Polyzotis, K.~H. Tae, and S.~E. Whang.
\newblock Slice finder: Automated data slicing for model validation.
\newblock In {\em ICDE}, pages 1550--1553. IEEE, 2019.

\bibitem{sagadeeva2021sliceline}
S.~Sagadeeva and M.~Boehm.
\newblock Sliceline: Fast, linear-algebra-based slice finding for ml model debugging.
\newblock In {\em SIGMOD}, pages 2290--2299, 2021.

\bibitem{kleinberg2016inherent}
J.~Kleinberg, S.~Mullainathan, and M.~Raghavan.
\newblock Inherent trade-offs in the fair determination of risk scores.
\newblock {\em arXiv preprint arXiv:1609.05807}, 2016.

\bibitem{zadrozny2001obtaining}
B.~Zadrozny and C.~Elkan.
\newblock Obtaining calibrated probability estimates from decision trees and naive bayesian classifiers.
\newblock In {\em ICML}, volume~1, pages 609--616. Citeseer, 2001.

\bibitem{zadrozny2002transforming}
B.~Zadrozny and C.~Elkan.
\newblock Transforming classifier scores into accurate multiclass probability estimates.
\newblock In {\em SIGKDD}, pages 694--699, 2002.

\bibitem{platt1999probabilistic}
J.~Platt et~al.
\newblock Probabilistic outputs for support vector machines and comparisons to regularized likelihood methods.
\newblock {\em Advances in large margin classifiers}, 10(3):61--74, 1999.

\bibitem{niculescu2005predicting}
A.~Niculescu-Mizil and R.~Caruana.
\newblock Predicting good probabilities with supervised learning.
\newblock In {\em Proceedings of the 22nd international conference on Machine learning}, pages 625--632, 2005.

\bibitem{chatfield93predictionintervals}
C.~Chatfield.
\newblock Prediction intervals.
\newblock {\em Journal of Business and Economic Statistics}, 11:121--135, 1993.

\bibitem{pearce2018high}
T.~Pearce, A.~Brintrup, M.~Zaki, and A.~Neely.
\newblock High-quality prediction intervals for deep learning: A distribution-free, ensembled approach.
\newblock In {\em International conference on machine learning}, pages 4075--4084. PMLR, 2018.

\bibitem{khosravi2010lower}
A.~Khosravi, S.~Nahavandi, D.~Creighton, and A.~F. Atiya.
\newblock Lower upper bound estimation method for construction of neural network-based prediction intervals.
\newblock {\em IEEE transactions on neural networks}, 22(3):337--346, 2010.

\bibitem{angelopoulos2021gentle}
A.~N. Angelopoulos and S.~Bates.
\newblock A gentle introduction to conformal prediction and distribution-free uncertainty quantification.
\newblock {\em arXiv preprint arXiv:2107.07511}, 2021.

\bibitem{shafer2008tutorial}
G.~Shafer and V.~Vovk.
\newblock A tutorial on conformal prediction.
\newblock {\em Journal of Machine Learning Research}, 9(3), 2008.

\bibitem{harradon2018causal}
M.~Harradon, J.~Druce, and B.~Ruttenberg.
\newblock Causal learning and explanation of deep neural networks via autoencoded activations.
\newblock {\em arXiv preprint arXiv:1802.00541}, 2018.

\bibitem{ribeiro2016should}
M.~T. Ribeiro, S.~Singh, and C.~Guestrin.
\newblock " why should i trust you?" explaining the predictions of any classifier.
\newblock In {\em SIGKDD}, pages 1135--1144, 2016.

\bibitem{gunning2019darpa}
D.~Gunning and D.~Aha.
\newblock Darpa’s explainable artificial intelligence ({XAI}) program.
\newblock {\em AI Magazine}, 40(2):44--58, 2019.

\end{thebibliography}

\end{document}

\end{article}

\begin{article}
{Towards Privacy by Design for Data with \MakeUppercase{strm} privacy}
{Bart van Deenen, Pim Nauts, Robin Trietsch, Bart Voorn}
\graphicspath{{submissions/towards-privacy-by-design-for-data-with-strm-privacy/}}
% link to instruction: https://tc.computer.org/tcde/tcde-bulletin-author-instructions/
% \documentclass[11pt,dvipdfm]{article}
\documentclass[11pt]{article}
\usepackage{tabularx}
\usepackage{ragged2e}  % for '\RaggedRight' macro (allows hyphenation)
\usepackage{booktabs}  % for \toprule, \midrule, and \bottomrule macros
\usepackage{textcomp}
\usepackage{amsfonts,amsmath}
\usepackage{deauthor,times}
\usepackage{graphicx} % 
\usepackage{hyperref}
\usepackage{comment}
\graphicspath{{asudeh/}}
\usepackage{soul}
\usepackage{subcaption}
\usepackage{ulem}
\usepackage{wrapfig}
\usepackage{color}
\usepackage{xspace}
\newtheorem{problem}{Problem}

%\DeclareMathOperator*{\argmax}{arg\,max}

%remove the following commands/package b4 submission
\newcommand{\hide}[1]{}
\newcommand{\eat}[1]{}
\newcommand{\resolved}[1]{\hide{#1}}
\newcommand{\abol}[1]{\textcolor{red}{Abol: #1}}
\newcommand{\mahdi}[1]{\textcolor{red}{Mahdi: #1}}
\newcommand{\nima}[1]{\textcolor{red}{Nima: #1}}

\newcommand{\dee}{\mathcal{D}}
\newcommand{\Gee}{\mathcal{G}}
\newcommand{\gee}{\mathbf{g}}
\newcommand{\ee}{\mathbf{e}}
\newcommand{\es}{\mathcal{S}}
\newcommand{\el}{\mathcal{L}}
\newcommand{\xx}{\mathcal{x}}
\newcommand{\dist}{\xi}
\newcommand{\alg}{\mathsf{A}}
\newcommand{\qu}{\mathbf{q}}
\newcommand{\ex}{\mathbf{x}}
\newcommand{\ti}{\mathbf{t}}
\newcommand{\sdt}{\mathsf{SDT}}
\newcommand{\wdt}{\mathsf{WDT}}
\newcommand{\Qu}{\mathbf{Q}}
\newcommand{\pe}{\mathbb{P}}
\newcommand{\megam}{\mathcal{M}}
\newcommand{\eps}{\varepsilon}
\newcommand{\enet}{{$\varepsilon$-{\bf net}}\xspace}
\newcommand{\net}{{\tt net}\xspace}
\newcommand{\vcd}{VC-dimension\xspace}
\newcommand{\at}[1]{{\tt \small #1}\xspace}
\newcommand{\pr}{Pr}

\newcommand{\sharpP}{\mbox{\#P}}
\newcommand{\NP}{\mathsf{NP}}
\newcommand{\LP}{\mathsf{LP}}
\newcommand{\IP}{\mathsf{IP}}
\newcommand{\ru}{{\sc {RU}}\xspace}
\newcommand{\sru}{{\sc {strongRU}}\xspace}
\newcommand{\wru}{{\sc {weakRU}}\xspace}

\newcommand{\fmsystem}{{\sc Chameleon}\xspace}
\newcommand{\fm}{$\mathcal{F}$\xspace}

\newtheorem{experiment}{Experiment}

\begin{document}

\title{Coverage-based Data-centric Approaches for \\Responsible and Trustworthy AI\thanks{This research was supported by the National Science Foundation under grant No. 2107290.}}

\author{
\begin{tabular}[t]{c@{\extracolsep{2.4em}}c@{\extracolsep{2.4em}}c@{\extracolsep{2.3em}}c} 
Nima Shahbazi & Mahdi Erfanian & Abolfazl Asudeh \\ 
University of Illinois Chicago & University of Illinois Chicago & University of Illinois Chicago\\
 nshahb3@uic.edu & merfan2@uic.edu & asudeh@uic.edu
\end{tabular}
}

\maketitle


\begin{abstract}
The grand goal of data-driven decision systems is to help make decisions easier, more accurate, at a higher scale, and also just. However, data-driven algorithms are only as good as the data they work with. Yet, data sets, especially those with social data, often do not represent minorities. The paucity of training data is a perpetual problem for AI, and the outcome of ML models for cases not represented in their training data is often not reliable. 
Hence, without properly addressing the lack of representation issues in data, we cannot expect AI-based societal solutions to have responsible and trustworthy outcomes. 

This paper focuses on data coverage as a data-centric approach for identifying and resolving misrepresentation of minorities in data.
To achieve this goal, we propose novel algorithms that (a) {\it identify} and {\it resolve} insufficient data coverage across data with different modalities and (b) use lack of representation information to generate data-centric {\it reliability warnings}.
 \end{abstract}
 
 %%%%%%%%%%%%%%%%%%%%%%%%%%%%%%%% INTRO  %%%%%%%%%%%%%%%%%%%%%%%%%%%%%%%%
\section{Introduction}\label{sec:intro} % Abstract+Intro: up to 2.5 pages 
Data-driven decision-making has shaped every corner of human life, spanning from autonomous vehicles to healthcare and even predictive policing and criminal justice. A pivotal concern, especially in applications that affect individuals, revolves around the reliability of the decisions rendered by the system.
It is easy to see that the accuracy of a data-driven decision depends, first and foremost, on the data used to make it. Essentially, the system learns the phenomena that data represent. While we may desire that the data should represent the underlying data distribution from which the production data is drawn, this alone may be insufficient, as it merely enables the model to perform well for the average case.
As a result, a model with a high accuracy could fail for specific regions in the data with insufficient representation. These regions may matter because they frequently represent some minority population in society. They could also represent cases that may not happen very often but have a relevant impact on the correctness of a critical decision.
In short, if the data fails to sufficiently represent a specific population, the outcome of the decision system for that population may not be trustworthy.

The phenomenon known as \textit{Representation Bias} can arise from how the data was originally collected, or it could be the result of biases introduced post-collection—whether historically, cognitively, or statistically.

Representation bias is essentially inevitable without a systematic approach to data collection. 
For example, in the context of survey data collection, vital steps involve identifying all populations within the underlying distribution based on desired demographic information and ensuring comprehensive coverage with sufficient samples from each group. 
Even then, only an (uncontrolled) subset of the invitees will opt-in to respond to the survey.
Another challenge lies in the fact that data scientists often lack control over the data collection process, leading to the reliance on ``found data'' in the majority of data-driven systems. Therefore, with no guarantee on the aforementioned steps in the data collection process, the found data is most likely a biased sample.
Acknowledging the potential harms of representation bias, the notion of \textit{Data Coverage}~\cite{asudeh2019assessing,shahbazi2023representation} has been proposed to ensure the adequate representation of minority groups in data sets employed for decision-making and developing sophisticated data science tools. 

Addressing representation issues in data poses various challenges depending on the modality of the data. In this paper, we focus on identifying and resolving lack of coverage issues in data with different modalities.
We start by proposing a variety of techniques (spanning from geometric and combinatorial optimization to crowd-souring) aimed at efficiently detecting insufficient coverage on structured data sets with non-ordinal categorical and continuous attributes, as well as image data sets. Next, we propose a range of approaches grounded in data integration and generative data augmentation to address the lack of coverage by enriching the data sets with more data. However, with limited control over the data collection processes, it could be difficult and expensive to resolve all misrepresentations. 
Since adding more data is not always possible, we proceed to introduce data-centric preventive solutions that warn the user about the reliability of their predictions regarding representation bias issues. These warnings assist users in determining whether they trust the outcomes of the models or exercise caution. 

 %%%%%%%%%%%%%%%%%%%%%%%%%%%%%%%% IDENTIFICATION  %%%%%%%%%%%%%%%%%%%%%%%%%%%%%%%%
\section{Detecting Insufficient Representation of Minorities}\label{sec:identification} %up to 3.5 pages
Representation bias happens when the development (training data) population under-represents 
and subsequently fails to generalize well 
for some parts of the target population, due to historical bias, sampling bias, etc.
The notion of {\it data coverage} has been studied across different settings in \cite{shahbazi2023representation} as a metric to measure representation bias. At a high level, coverage is referred to as having enough similar entries for each object in a data set. 
For a better understanding, let us go over the definition of the generalized notion of coverage:

\begin{definition}[Data Coverage]\label{def:coverage}
Consider a data set $\dee$ with $n$ tuples, each consisting of $d$ attributes of interest $\mathbf{x}=\{x_1, x_2, \cdots,x_d\}$, such as {\tt gender}, {\tt race}, {\tt salary}, {\tt age}, etc, that are used for coverage identification.
The data set also contains target attributes $\mathbf{y} = \{ y_1,\cdots,y_{d'}\}$ that may or may not be considered for the coverage problem.
A query point $q$ is not covered by the data set $\dee$, if there are not ``enough'' data points in $\dee$ that are representative of $q$.
To generalize the notion of coverage, let us define $\gee(q)$ as the universe of tuples that would represent $q$ and let $\gee_\dee(q) = \gee(q)\cap \dee$. In other words, $\gee_\dee(q)$ are the set of tuples in $\dee$ that represent $q$.
Using this notation, we define the coverage of $q$ as the size of $\gee_\dee(q)$. That is,
$cov(q,\dee) = | \gee_\dee(q)|$.
Given a value $\tau$, $q$ is covered if $cov(q,\dee)>\tau$.
Similarly, a group $\gee$ is not covered if $\gee\cap \dee<\tau$.
The {\it uncovered region} in a data set is the collection of groups that are not covered by it.
\end{definition}

\subsection{Structured Data}
In this section, we focus on identifying representation bias in structured data.
Depending on the type of the attributes of interest, we categorize the techniques into two classes based on whether they target the problem for non-ordinal {\it categorical} (e.g. {\tt race}, {\tt gender}) or ordinal {\it continuous} (e.g. {\tt age}) attributes. The attributes of interest considered for representation bias often include sensitive attributes such as {\tt race} and {\tt gender} but are not necessarily limited to them.

\subsubsection{Categorical Attributes}

For cases where attributes of interest are non-ordinal categorical,
the cartesian product of values on a subset of attributes $\mathbf{x}'\subseteq \mathbf{x}$, form a set of (sub-)groups.
For example, $\{$ {\tt white male}, {\tt white female}, {\tt black male} $,\cdots\}$ are the subgroups defined on the attributes {\tt (race,gender)}.
We refer to the number of attributes used to specify a subgroup as the {\it level} of that subgroup.
For example, the level of the subgroup {\tt white male} is 2, while the level of the subgroup {\tt male} is 1.
We use $\ell(\gee)$, to refer to the level of a subgroup $\gee$.
Similarly, we say a subgroup $\gee'$ is a subset of $\gee$, if the groups specifying $\gee'$ are a superset of the ones for $\gee$. For example {\tt (married white male)} a subset of the more general group {\tt (white male)}. That is, the set of individuals in group {\tt (married white male)} are a subset of {\tt (white male)}.
Moreover, we say a subgroup $\gee$ is a {\it parent} of the subgroup $\gee'$, if $\gee'\subset \gee$ and $\ell(\gee)=\ell(\gee')+1$. For example, the subgroup {\tt (white male)} is a parent of the subgroup {\tt (married white male)}.
We use \textit{patterns} to refer to uncovered subgroups.
A pattern $P$ is a string of $d$ values, where $P[i]$ is either a value from the domain of $x_i$, or it is ``unspecified'', specified with $X$. 
For example, consider a data set with three binary attributes of interest $\mathbf{x}=\{x_1, x_2, x_3\}$. The pattern $P=X01$ specifies all the tuples for which $x_2=0$ and $x_3=1$ ($x_1$ can have any value).
The set of patterns that identify most general uncovered subgroups are called {\it Maximal Uncovered Patterns} (MUPs).

No polynomial time algorithm can guarantee the enumeration of the entire MUPs, however, several algorithms inspired by set enumeration and the Apriori algorithm for association rule mining are proposed to efficiently address this problem~\cite{asudeh2019assessing}.
In this regard, we introduce \textit{Pattern Graph} data structure that exploits the relationship between patterns to do less work than computing all uncovered patterns by removing the non-maximal ones. 
The parent-child relationship between the patterns is represented in a graph that can be used to find better algorithms. 
\textit{Pattern-Breaker} starts from the top of the graph where the general patterns are and moves down by breaking each pattern into more specific ones. If a pattern is uncovered, then all of its descendants are also uncovered and they can not be an MUP, even if they have a parent that is covered. Therefore, this subgraph of the pattern graph can be pruned. 
The issue with \textit{Pattern-Breaker} is that it explores the covered regions of the pattern graph and for the cases where there are a few uncovered patterns, it has to explore a large portion of the exponential-size graph. 
To tackle this, \textit{Pattern-Combiner} algorithm is proposed that performs a bottom-up traversal of the pattern graph. It uses an observation that the coverage of a node at the level of the pattern graph can be computed as the sum of the coverage values of its children. 
The problem with \textit{Pattern-Combiner} is that it traverses over the uncovered nodes first and therefore, it will not perform well for the cases in which most of the nodes in the graph are uncovered. 
In fact, for the cases where most of the MUPs are placed in the middle of the graph, both \textit{Pattern-Breaker} and \textit{Pattern-Combiner} will not be as efficient as they should traverse half of the graph. Therefore, we propose \textit{Deep-Diver}, a search algorithm based on Depth-First-Search that quickly finds the MUPs, and uses them to limit the search space by pruning the nodes both dominating and dominated by the discovered MUPs.

\begin{figure*}[!tb]
    \begin{minipage}[t]{0.31\linewidth}
        \centering
        \includegraphics[width=\textwidth]{submissions/submission1/shahbazi/covcube1.jpg}
        \caption{\small Categorical attributes: the uncovered region of a toy example, as the collection of three MUPs.}
        \label{fig:covcube1}
    \end{minipage}
    \hfill
    \begin{minipage}[t]{0.31\linewidth}
        \centering
        \includegraphics[width=\textwidth]{submissions/submission1/shahbazi/cvrg_2_1.jpg}
        \caption{\small Continuous attributes, 2D: identifying the covered region in the gray Voronoi cell.}
        \label{fig:cvrg_2_1}
    \end{minipage}
    \hfill
    \begin{minipage}[t]{0.31\linewidth}
        \centering
        \includegraphics[width=\textwidth]{submissions/submission1/shahbazi/cvrg_2_2.jpg}
        \caption{ \small Continuous attributes, 2D: Uncovered region marked in red.}
        \label{fig:cvrg_2_2}
    \end{minipage}
\vspace{-5mm}
\end{figure*}

\subsubsection{Continuous Attributes}
Data in the real world often consists of a combination of continuous and discrete values. While simple solutions like binning {\tt age} into {\tt young} and {\tt old} can transform the continuous space into discrete. However, they may lead to coarse groupings that are sensitive to the thresholds chosen. It may be inappropriate to treat a 35-yo as {\tt young} but a 36-yo as {\tt old}. 
Therefore, we extend the notion of coverage to continuous space. Particularly, given data set $\dee$ with $n$ tuples over $d$ attributes, and vicinity radius $\rho$ and coverage threshold $k$, we want to identify the uncovered region -- the universe of uncovered query points.
A query point in continuous data space is covered if there are enough (at least $k$) data points in its $\rho$-vicinity neighborhood. $\rho$-vicinity neighborhood is the circle centered at the query point with radius $\rho$.

Depending on the number of attributes in a data set, we propose two algorithms for identifying uncovered regions in data~\cite{asudeh2021coverage}. 
The first algorithm known as \textit{Uncovered-2D} studies coverage over two-dimensional data sets where $\mathbf{x}=\{x_1,x_2\}$. To find the number of circles that a query point falls into and consequently discover the uncovered region, \textit{Uncovered-2D} makes a connection to $k$-th order Voronoi diagrams.
Consider a data set $\mathcal{D}$ and its corresponding $k$-th order Voronoi diagram. For every tuple $t\in \mathcal{D}$, let $\circ_t$ be the $d$-dimensional sphere ($d$-sphere) with radius $\rho$ centered at $t$.
Consider a $k$-voronoi cell $\mathcal{V}(S)$ in the $k$-th order Voronoi diagram $V_k(\mathcal{D})$.
Any point $q$ inside the intersections of the $d$-spheres of tuples in $S$, i.e. $q\in \underset{\forall t\in S}{\cap ~\circ_t}$, is covered, while all other points in the region are uncovered.
 The algorithm starts by constructing the $k$-th order Voronoi diagram of the data set and then for each Voronoi cell $\mathcal{V}(S)$ in the diagram, it computes the intersection of the circles of the tuples in $S$ and marks the portion of $\mathcal{V}(S)$ that falls outside it as uncovered.
After identifying the uncovered region, a 2D map of $\{x_1,x_2\}$ value combinations is used to report the region to the user.
The algorithm for the 2D case can be extended to the general case by relaxing the assumption on the number of attributes to discover the exact uncovered region, however, due to the curse of dimensionality, the search size space explodes as the number of dimensions increases and as a result, the algorithm will not be practical. Therefore, we propose a randomized approximation algorithm based on the geometric notion of \enet. 
Let $\mathcal{X}$ be a set and $\mathcal{R}$ be a set of subsets of $\mathcal{X}$. A set $\mathcal{N}\subset \mathcal{X}$ is an \enet for $\mathcal{X}$ if for any range $r\in\mathcal{R}$, if  $|r\cap \chi|>\eps|\chi|$, then $r$ contains at least one point of $N$.
The idea, at a high level, is to draw enough random samples from the space of potential query points to form an \enet. 
We then label the sampled query points as $\{-1,+1\}$ depending on whether those are covered or not, and learn the uncovered regions using the samples.

\subsection{Image Data}
Many known incidents of machine failures due to the lack of representation were on image data.
We consider an image data set with a fixed number of low-cardinality sensitive attributes such as {\tt\small race} and {\tt\small gender}. 
It is common that image data sets {\it lack explicit values} for sensitive attributes, which are crucial for coverage identification. An image data set is often a collection of images from different domains with little to no information about their domain and which groups they belong to. As a result, even studying coverage over low-cardinality and categorical attributes of interests is challenging in these cases.

\begin{wrapfigure}{R}{0.42\textwidth}
\centering
\vspace{-3mm}
\scriptsize
\begin{tabular}{|@{}c|@{}c@{}|@{}c@{}|@{}c@{}|} 
 \hline
{\bf data set} & {\bf classifier} & {\bf accuracy} & {\bf precision} \\ 
 &  &  & {\bf on female} \\ \hline
UTKFace:~& DeepFace (opencv) & 93.56 & {52.02}\\\cline{2-4}
({\tt females}=200,& DeepFace (retinaface) & 94.16 & {56.15}\\\cline{2-4}
{\tt males}=2800) & BaseCNN & 97.6 & 74.8\\
\hline
UTKFace:~& DeepFace (opencv) & 96.53 & {\bf 8.0}\\\cline{2-4}
({\tt females}=20,& DeepFace (retinaface) & 96.43 & {\bf 10.09}\\\cline{2-4}
{\tt males}=2980)& BaseCNN & 97.6 & {\bf 21.59}\\
\hline
\end{tabular}
\vspace{-3mm}
\caption{\small ML models' low performance for females in the presence of representation bias.~\cite{mousavi2024data}}\label{fig:mlfails}
\vspace{-3mm}
\end{wrapfigure}

In Figure~\ref{fig:mlfails}, we show that due to the issues such {\it machine bias} and {\it lack of distribution generalizability},
solely relying on state-of-the-art machine learning (ML) techniques fail to effectively identify lack of coverage in image data sets. Therefore, we propose an approach based on combining crowdsouring with ML~\cite{mousavi2024data}. 
Crowdsourcing is particularly promising for image data, for tasks such as image labeling, which, while challenging for the machine, are "easy" for human beings to conduct with minimal error. 

A key observation that enables a cost-effective crowdsourcing approach is that, while studying coverage, we would only like to find out if there are {\it enough tuples from each subgroup}.
Suppose a subgroup is covered if there are $\tau=100$ instances of it in the data set. Assume the (majority) group $\gee_1$ contains $n_1 \gg 100$ objects in the data set. 
To verify that $\gee_1$ is covered, it is enough for the crowd to discover 100 of those objects, not the entire $n_1$. 
Following this, $O(\tau)$ provides a lower bound on the number of crowd tasks required to verify a given group is covered. 
Still, this lower bound only holds for the groups that are covered, i.e., there is at least $\tau$ of those in the data set.
Surprisingly, verifying that a minority group is indeed uncovered is cumbersome, unlike the majority group.
This is because even though discovering $\tau$ objects from a group is enough for verifying that it is covered, one cannot {\it verify} a group is uncovered until there is a chance that the data set might still have enough objects from that group. Thus, assuming a non-zero probability for each unlabeled object to belong to each group, {one might need to ask the crowd to label the entire data set before they can confirm that a specific group is uncovered}.

Our idea for addressing this challenge is to
design {\it a divide and conquer algorithm} that, instead of {point queries}, uses {\it set queries} to iteratively eliminate subsets of data that {does not include any object from the given group}.
At a high level, our idea is to ask a set query from the crowd, inquiring whether the selected set contains at least one object from the given group $\gee$.
The user may provide two responses (yes/no). 
Interestingly, {in either case}, the user response provides valuable information that helps efficiently identify the coverage.
If the answer is ``No'', the set does not include any object from the given group $\gee$. As a result, the algorithm can safely prune the set, asking no further questions about it. In particular, for a group that is not covered, one can expect to see no answers on large set queries helping to prune a significant portion of the data set quickly.
On the other hand, if the answer is ``yes'', the set contains {at least} one object from the group $\gee$. As a result, the algorithm cannot prune the subset since it can have any number (larger than one) of the objects in $\gee$.
At first glance, the queries with yes answers do not provide helpful information as the algorithm cannot prune the subset (hence it needs to divide it into smaller subsets).
However, a key observation is that {the algorithm will only observe a limited number of yes answers} before it stops.
The reason is that the number of set queries with yes answers provides a {lower-bound} on the number of objects from $\gee$ in the data set. As a result, the algorithm can stop as soon as the lower bound reaches $\tau$, knowing that $\gee$ is covered.
The D\&C approach verifies the data coverage for a given group, while our goal is to identify the uncovered regions for a given set of sensitive attributes. The next question is how to utilize this algorithm for efficient coverage identification on different scenarios of sensitive attributes, forming intersectional or non-intersectional groups.
In particular, how can we find maximal uncovered patterns?
Our idea is to apply sampling and aggregate estimation techniques to find the groups that even if merged are likely to still be uncovered. This will help reduce the coverage identification cost by running the D\&C approach for the merged groups once.
 %%%%%%%%%%%%%%%%%%%%%%%%%%%%%%%% RESOLUTION  %%%%%%%%%%%%%%%%%%%%%%%%%%%%%%%%
\section{Resolving Insufficient Representation}\label{sec:resolution}

Data integration~\cite{nargesian2021tailoring,nargesian2022responsible} and data augmentation~\cite{sharma2020data,DBLP:journals/jair/ChawlaBHK02,iosifidis2018dealing,celis2020data} are considered as the primary solutions for reducing data coverage issues in a data set. 
Data integration is promising when external sources of data are available. On the other hand, recent advancements in generative AI and foundation models have enabled efficient and effective augmentation of data sets with synthetic data. 
Therefore, in the following, we review two approaches, one from each category, in the context of lack of coverage resolution.

\subsection{Data Integration}\label{sec:resolution:integration}

Data integration is to consolidate data from different sources into a single, unified view. 
Although it is an effective solution to acquire additional data from different distributions,
there are sampling policy and cost-efficiency concerns that need to be examined.  
Therefore, {\it Data Distribution Tailoring ({\sc DT})} introduces data integration techniques for resolving insufficient representation of subgroups in a data set in the most cost-effective manner~\cite{nargesian2021tailoring}.
A query to {\sc DT} 
consists of a target schema, and a set of group distribution requirements in the form of the minimum counts (e.g., ``{\tt\small 1,000 breast cancer monitoring data in Chicago with at least 30\% label=positive, and at least 20\% black patients}''). 
Collecting a fresh sample from a data view is costly (monetary, human resources, and/or computation cost)~\cite{asudeh2022towards}.
Therefore, {\sc DT} focuses on satisfying the count requirements with minimum cost. 
Given an input query and a lake of available data sources, the first step is to discover a collection of candidate data views that satisfy the target schema.
Each data view $v_i$ is a projection-join $v_i = \Pi\big(D_{i1}\bowtie\cdots\bowtie D_{ik_i} \big)$, where $D_{ij}$ is a data set in a given data lake.
Let us suppose the data views are already discovered.
At a high level, {\sc DT} follows an iterative approach that at each iteration a data view is selected to be queried.
Each query to a data view has a fixed cost and returns a sample that may or may not satisfy the query constraints.
The samples that are either not fresh, or do not satisfy the query are discarded.
Hence, the essential question towards a cost-effective data integration is {\it what data view to query next}.
Depending on the available information about the data sources, various techniques may be employed. 

For the cases when the group distributions are known, the process of collecting the target data set is a sequence of iterative steps, where at every step, the algorithm chooses a data view, queries it, and if the obtained tuple contributes to one of the groups for which the count requirement is not yet fulfilled, it is kept, otherwise discarded. To do so, a {Dynamic Programming (DP)} algorithm is proposed. An optimal source at each iteration minimizes the sum of its sampling cost plus the expected cost of collecting the remaining required groups, based on its sampling outcome.
The DP algorithm, however, has a pseudo-polynomial time complexity. Hence, it quickly becomes intractable for cases where the minimum count requirements for the groups are not small. 
For cases where the (sensitive) attribute of interest is binary, such as (biological) {\tt sex}={\tt \{male, female\}}, and the cost to query data is similar from all sources, it turns out that the optimal strategy is to query the data source with {maximum probability of obtaining a sample from the minority group}.
Expanding the binary-attributes algorithm for non-binary cases, the problem can be modeled as an extension of the ``{\it coupon collector's}'' problem~\cite{motwani1995randomized}, where the goal is to collect $m_i$ instances from each coupon (group) $\gee_i$.
At each iteration, the coupon collector's algorithm identifies a data view as most promising and queries it. In simple terms, a data view with a smaller query cost and a higher chance of obtaining minority groups is more promising.


For the cases where the group distributions are unknown, we model DT as a {\it multi-armed bandit} problem, where every data view is modeled as an arm. 
Every arm has an unknown distribution of different groups while pulling an arm (i.e., querying the corresponding data view) has a cost.
During various iterations, the algorithms pull the arms in an order that its expected total {\it reward} is maximized.
Arguing that the reward of obtaining a tuple from a group is proportional to how rare this group is across different data views, 
we design the reward function based on the expected cost one needs to pay in order to collect a tuple from a specific group.  
As the bandit strategy, we adopt {\it Upper Confidence Bound (UCB)} to balance exploration and exploitation. At every iteration, for every arm, UCB computes confidence intervals for the expected reward and selects the arm with the maximum upper bound of reward to be explored next.

\subsection{Data Augmentation using Foundation Models}

While data integration provides a promising approach for resolving coverage issues in a data set, its effectiveness is limited to the availability of external data sources that are rich enough to find sufficient fresh samples from minority groups. This, however, is not always possible, especially since the minority samples are rare and not easy to obtain.
Fortunately, recent advancements in Generative AI and Foundation Models have enabled synthesizing samples that are otherwise challenging to obtain from the real world.

Therefore, as an alternative approach to data integration, we turn our attention to the Foundation Models and Generative AI for resolving the lack of coverage. 
Particularly, models such as {\sc DALL.E}\footnote{\url{https://openai.com/dall-e-2}} have emerged as powerful tools for generating multi-modal data such as image, audio, and video.
 
We formalize the foundation model \fm as a black-box function with the following inputs, that once queried synthesize an output tuple.
\begin{itemize}
    \item {\bf Prompt}: A natural language description providing instructions on the details of the tuple to be generated. For instance, a prompt for image generation might be ``A realistic photo of a white cat running in a backyard.''
    \item {\bf Guide}: In cases where only a prompt is provided, the foundation model uses its imagination to generate the requested tuple. For the previous example, the prompt of a cat image, the breed, size, background, and other details are generated based on the model's imagination. Alternatively, a guide can be provided to influence the generation process. The guide is formalized as a pair $(t,m)$ where $t$ is a tuple and $m$ is a mask specifying which parts of the guide tuple should be changed. Using the cat example, $t$ can be a cat image and $m$ can specify the foreground to be regenerated.
\end{itemize}

There are multiple challenges towards effective data set augmentations using foundation models. 
First, we have to determine the minimal set of synthetic tuples that once added to the original data set, under-representation issues are resolved.
Second, the generated images should follow the underlying distribution represented in the input data set. Third, the generated tuples should have high quality and look realistic to a human evaluator. Last but not least, given the (often monetary) cost associated with the queries to the foundation model, we should ensure the cost-effectiveness of the data set repair process.

\begin{wrapfigure}{L}{0.45\textwidth}
\centering
\vspace{-3mm}
\scriptsize
    \includegraphics[width=.45\textwidth]{submissions/submission1/shahbazi/enhanced_pipeline.png}
\vspace{-3mm}
\caption{\small Architecture of \fmsystem for image data augmentation for coverage enhancement.}\label{fig:chameleon}
% \vspace{-3mm}
\end{wrapfigure}

\noindent Figure~\ref{fig:chameleon} shows the architecture of our system \fmsystem \cite{chameleon} for coverage enhancement using DALL-E image generator.
To address the first challenge, we define the combinations-selection problem, which minimizes the total number of synthetic tuples for resolving lack of coverage of minorities at the most general level. We show the problem is {\sc NP}-hard, and propose a greedy approximation algorithm for it.
To address the second and third challenges, \fmsystem follows a {\it rejection sampling} strategy.
It views each tuple in the data set $\dee$ as an iid sample from the underlying distribution $\xi$ it represents. It uses the vector representations (embeddings) space to describe the distribution. Then, given a newly generated tuple, it employs the one-class support vector machine (OCSVM) approach proposed by Scholkopf et al.~\cite{scholkopf1999support} to reject the tuple if it does not follow $\xi$.
Moreover, it models the quality evaluation as hypothesis testing and rejects the samples that have a higher chance of being labeled as ``unrealistic'' by a random human evaluator.
Finally, to minimize the number of queries to the foundation model, we provide a guide tuple (and a mask), in addition to the prompt, to the foundation model. We model the guide-selection problem as {\it contextual multi-armed bandit} and propose a solution based on the contextual UCB for it.

Before concluding this section, let us provide some experiment results to demonstrate the effectiveness of data augmentation with \fmsystem. We use FERET DB \cite{phillips1998feret} for this experiment, which comprises 1199 individual images and serves as a standardized facial image database for researchers to develop algorithms and report results. All images in FERET DB share the same dimensions, pose, and facial expression.
First, we identified the (level-1) uncovered ethnicity groups, using the threshold 80. We then used \fmsystem and resolved the lack of coverage issues.
To evaluate the effectiveness of the system, we trained a CNN model to predict the race of each image within this dataset. We then retrained the identical CNN on the repaired training data. Importantly, our test dataset for both experiments remains consistent and is derived from real images.
Table~\ref{tab:lackofcoverage} presents the improvements in precision, recall, and F1 score metrics for under-represented groups after repairing the dataset. The results indicate an enhancement in performance metrics for all under-represented groups following the repair process.

\begin{table}[t]
    \centering
    \caption{Illustrating the effect of lack of coverage repair using \fmsystem on \texttt{FERTDB}}
    \label{tab:lackofcoverage}
    \vspace{-3mm}
    \begin{tabular}{lcccccccc}
        \toprule
         & \multicolumn{4}{c}{\textbf{Classifier Performance on \texttt{FERTDB}}} & \multicolumn{4}{c}{\textbf{Classifier Performance on Repaired}} \\
        \cmidrule(lr){2-5} \cmidrule(lr){6-9}
        \textbf{Ethnicity Groups}& \#Images & Precision & Recall & F1-Score & \#Images & Precision & Recall & F1-Score \\
        \midrule
        Overall          & 756 & 0.81 & 0.75 & 0.78 & 987 & 0.70 & 0.75 & 0.72 \\ \hline
        Black            & 40  & 0.19 & 0.22 & 0.16 & 100 & 0.48 & 0.56 & 0.52 \\
        Hispanic         & 19  & 0.50 & 0.17 & 0.25 & 100 & 0.62 & 0.36 & 0.45 \\
        Middle Eastern   & 10  & 0.00 & 0.00 & 0.00 & 100 & 0.20 & 0.41 & 0.27 \\
        \bottomrule
    \end{tabular}
\end{table}

 %%%%%%%%%%%%%%%%%%%%%%%%%%%%%%%% RELIABILITY  %%%%%%%%%%%%%%%%%%%%%%%%%%%%%%%%
\section{Generating Reliability Warnings}\label{sec:reliability}
% up to 2.5 pages
Interpretability is a necessity for data scientists who develop predictive models for critical decision-making.
In such settings, it is important to provide additional means to support the following question:
{\it is an individual prediction of the model reliable for decision-making?} Our goal is to use the lack of representation to help decision-makers find insights about this critical question.
To further motivate this, let us use the following example:

\vspace{1mm}
\begin{example}\label{ex-0}
{\bf(Part1):} Consider a judge who needs to decide whether to accept or deny a bail request. Using data-driven predictive models is prevalent in such cases for predicting recidivism~\cite{dressel2018accuracy}.
Indeed, such models can be beneficial to help the judge make wise decisions.
Suppose the model predicts the queried individual as high risk (or low risk).
The judge is aware and concerned about the critics surrounding such models.
A major question the judge faces is whether or not they should rely on the prediction outcome to take action for this case.
Furthermore, if, for instance, they decide to ignore the outcome and hence they need to provide a statement supporting their action, what evidence can they provide? 
\end{example}

In line with the recent trend on data-centric AI~\cite{ng2021mlops}, we design {novel approaches}, {complimentary} to the existing work on trustworthy AI~\cite{wing2021trustworthy,kentour2021analysis,liu2021trustworthy,singh2021trustworthy}, to address the aforementioned trust question through the lens of {\it data}.
In particular, unlike existing works that generate trust information from a {\it given \underline{model}}, we associate {\it \underline{data sets} with proper measurements} that specify their {\it the scope of use for predicting future cases}.
We note that a predictive model provides only probabilistic guarantees on the \underline{average} loss over the distribution represented by the data set used for training it.
As a result, these predictions may not be distribution generalizable~\cite{kulynych2022you}.
Consequently, if the query point is {\it not represented} by the data, the guarantees may not hold, hence one cannot rely on the prediction outcome.
Besides, an essential requirement for a learning algorithm is that its training data $\dee$ should represent the underlying distribution $\dist$.
Even if so, the trained model $h$ only provides a probabilistic guarantee on the {expected} loss on random samples from $\dist$.  
A model that performs well on {\it majority} of samples drawn from $\dist$ will have a high performance on average. Still, as we observed in Figure~\ref{fig:mlfails},
its performance for {\it minorities} and points that are not represented is questionable. Let us consider the following toy example:

\begin{figure*}[!b] 
    \begin{minipage}[t]{0.32\linewidth}
        	\centering
        	\includegraphics[width=\textwidth]{submissions/submission1/shahbazi/example_1.png} 
        	\vspace{-9mm}\caption{\small Data set $\dee$ generated using a Gaussian distribution; $x_1$ and $x_2$ are positively correlated}
            \label{fig:ex1:1}
    \end{minipage}
    \hfill
    \begin{minipage}[t]{0.32\linewidth}
        \centering
        	\includegraphics[width =\textwidth]{submissions/submission1/shahbazi/example_2.png} 
        	\vspace{-9mm}\caption{\small The decision boundary of learned model $h$ and query points $\qu^1$ to $\qu^4$}
            \label{fig:ex1:2}
    \end{minipage}
    \hfill
    \begin{minipage}[t]{0.32\linewidth}
        	\centering
        	\includegraphics[width =\textwidth]{submissions/submission1/shahbazi/example_3.png}
        	\vspace{-9mm}\caption{\small Ground-truth boundary, overlaid on the model decision boundary and query points}
            \label{fig:ex1:3}
    \end{minipage}
    \vspace{-5mm}
\end{figure*} 

\vspace{1mm}
\begin{example}\label{ex-1}
Consider a binary classification task where the input space is $\ex=\langle x_1, x_2\rangle$ and the output space is the binary label $y$ with values $\{-1$ (red) $,+1$ (blue)$\}$.
Suppose the underlying data distribution $\dist$ follows a 2D Gaussian, where $x_1$ and $x_2$ 
are positively correlated as shown in Figure~\ref{fig:ex1:1}.
The figure shows the data set $\dee$ drawn independently from the distribution $\dist$, along with their labels as their colors.
Using $\dee$, the prediction model $h$ is constructed as shown in Figure~\ref{fig:ex1:2}. 
The decision boundary is specified in the picture; while any point above the line is predicted as +1, a query point below it is labeled as -1.
The classifier has been evaluated using a test set that is an iid sample set drawn from the underlying data set $\dist$. The accuracy on the test set is high (above 90\%), and hence, the model gets deployed.
We cherry-picked four query points, $\qu^1$ to $\qu^4$, that are also included in Figure~\ref{fig:ex1:2}. Using $h$ for prediction, $h(\qu^1)=-1$, $h(\qu^2)=+1$,  $h(\qu^3)=+1$, and $h(\qu^4)=-1$.
Figure~\ref{fig:ex1:3} adds the ground-truth boundary to the search space, revealing the true label of the query points: every point inside the red circle has the true label $-1$ while any point outside of it is $+1$.
Looking at the figure, $y^1=+1$ while the model predicted it as $h(\qu^1)=-1$.  \hfill$\square$
\end{example}
\vspace{2mm}

Let us take a closer look at the four query points in this example and their placement with regard to the tuples in $\dee$ used for training $h$. 
$\qu^2$ belongs to a {\it dense region} with many training tuples in $\dee$ surrounding it. Besides, all of the tuples in its vicinity have the same label $y=+1$. As a result, one can expect that the model's outcome $h(\qu^2)=+1$ should be a reliable prediction.
Similar to $\qu^2$, $\qu^4$ also belongs to a dense region in $\dee$; however, $\qu^4$ belongs to an {\it uncertain region}, where some of the tuples in its vicinity have a label $y=+1$, and some others have the label $y=-1$. Considering the uncertainty in the vicinity of $\qu^4$, one cannot confidently rely on the outcome of the model $h$. 
On the other hand, the neighbors of $\qu^1$ (resp. $\qu^3$) are not uncertain, all having the label $y=-1$ (resp. $y=+1$).
However, the query points $\qu^1$ and $\qu^3$ are not well represented by $\dee$. In other words, $\qu^1$ and $\qu^3$ are unlikely to be generated according to the underlying distribution $\dist$, represented by $\dee$. As a result, following the no-free-lunch theorem~\cite{kakade2003sample}, one cannot expect the outcome of model $h$ to be reliable for these points.
Looking at the ground-truth boundary in Figure~\ref{fig:ex1:3}, $h$ luckily predicted the outcome for $\qu^3$ correctly, but it was not fortunate to predict the $y^1$ correctly.
Nevertheless, 
since the model is not reliably trained for these points, 
its outcome for these query points is not trustworthy.

From Example~\ref{ex-1}, we observe that the outcome of a model $h$, trained using a data set $\dee$ is not reliable for a query point $\qu$, if:
\begin{itemize}
    \item {\bf Lack of representation:} $\qu$ is not well-represented by $\dee$.
    In such cases, the model has not seen ``enough'' samples similar to $\qu$ to reliably learn and predict the outcome of $\qu$.
    \item {\bf Lack of certainty:} $\qu$ belongs to an uncertain region, where different tuples of $\dee$ in the vicinity of $\qu$ have different target values. $\qu$ belongs to a high-fluctuating area, where tuples in the vicinity of $\qu$ have a wide range of values.
\end{itemize} \vspace{2mm}

\noindent
Based on these two observations, we propose Representation-and-Uncertainty ({\bf RU}) measures.
To identify if a query suffers from uncertainty or lack of representation, one could use a deterministic approach using a fixed threshold. Then if the number of similar samples to (resp. label fluctuation in vicinity of) $\qu$ is larger than the threshold it is considered as unrepresented (resp. uncertain).
This approach, however, would be misleading since two numbers close to the threshold could be treated very differently. Also, all points on each side of the threshold would be considered equally represented (resp., certain). Instead, we consider {\it a randomized approach}, widely popular in the literature, including~\cite{dwork2012fairness}.
That is, instead of using fixed thresholds, a Bernoulli variable (a biased coin) is used that 
assigns $\qu$ as unrepresented (resp., uncertain) based on the number of samples similar to it (resp., its neighborhood uncertainty).
Given a query point $\qu$, let $\pe_o$ be the probability indicating if $\qu$ is not represented and let $\pe_u$ be the probability indicating if $\qu$ belongs to an uncertain region. 
We represent the probability of the Bernoulli variables for lack of representation or uncertainty components as $\pe_o$ and $\pe_u$, respectively. Note that the two Bernoulli variables $\pe_o$ and $\pe_u$ are independent from each other. That simply follows the argument that after specifying the number of similar samples to $\qu$ whether or not it should be considered as unrepresented does not depend on the uncertainty in the neighborhood of $\qu$.

\begin{definition}[\sru]\label{def:sdt}
The \sru is a probabilistic measure that considers the outcome of a model for a query point $\qu$ untrustworthy if $\qu$ is not represented by $\dee$ {\it and} it belongs to an uncertain region.
Formally, the \sru measure is:
\begin{align} 
    \nonumber
    SRU(\qu) &= \pe\big((\qu \mbox{ is outlier}) \wedge (\qu \mbox{ belongs to uncertain region})\big) 
\end{align}
Since $\pe_o$ and $\pe_u$ are independent:

\vspace{-13mm}
\begin{align} \label{eq:strong}
    SRU(\qu) &= \pe_o(\qu) \times \pe_u(\qu)
\end{align}
\end{definition}

\sru raises the warning signal only when the query point fails on {\it both} conditions of being represented by $\dee$ and not belonging to an uncertain region. 
For instance, in Example~\ref{ex-1} none of the query points fail both on representation and on uncertainty; hence neither has a high \sru score.
On the other hand, 
a high \sru score for a query point $\qu$ {\it provides a strong warning signal} that one should perhaps reject the model outcome and not consider it for decision-making.

\sru is a strong signal that raises warnings only for the fearfully concerning cases that fail both on representation and uncertainty.
However, as observed in Example~\ref{ex-1} a query points failing {\it at least} one of these conditions may also not be reliable, at least for critical decision making.
We define the \wru measure to raise a warning for such cases.

\begin{definition}[\wru]\label{def:wdt}
The \wru measure is a probabilistic measure that considers the outcome of a model for a query point $\qu$ untrustworthy if $\qu$ is not represented by $\dee$ {\bf or} it belongs to an uncertain region.
Formally, the \wru is computed as:
\begin{align} \label{eq:weak}
    WRU(\qu) = \pe\big((\qu \mbox{ is outlier}) \vee (\qu \mbox{ belongs to uncertain region})\big) 
    = \pe_o(\qu) + \pe_u(\qu) - \pe_o(\qu) \times \pe_u(\qu)
\end{align}
\end{definition}

Proposing quantitative probabilistic outcomes, \ru measures are interpretable for the users, since beyond the scores, the uncertainty and lack of representation components provide an explanation to justify them. 
Please refer to \cite{techrep} for more details on how to efficiently and effectively compute the representation ($\pe_o$) and uncertainty ($\pe_u$) probabilities, using only $\dee$.
In Example~\ref{ex-0}, let us see how the \ru measures can be helpful.

\noindent{\bf Example 1. (part 2):}
{\it RU measures \underline{raise warning} when
the fitness of the data set used for drawing a prediction is questionable, helping the judge to be cautious when taking action.
Besides, these measures provide \underline{quantitative evidence} to support the judge's action when they decide to ignore a prediction outcome that is not trustworthy.
The judge, for example, can argue to ignore a model outcome for a specific case, based on the insight that 
the model has been built using a
data set that fails to represent the given case.}
\hfill$\square$

Finally, let us demonstrate the efficacy of \ru measures through a series of experiments. Since the \ru measures are {\it data-centric},
those are applicable for both classification and regression tasks, irrespective of the model used.
We use {\it Adult} dataset~\cite{adult} for classification and {\it House Sales in King County} dataset for the validation of regression tasks. From each dataset, we uniformly sample two sets from the underlying distribution. The first set serves as the training set to compute the \ru values, and the second one is used as the test set from which the queries are drawn. We validate our proposal by providing the correlation between the \ru values and the performance of an ML model's prediction on the same data. 

We start by computing the \ru values for all the query points in the test set. Next, we bucketize the query points based on their \ru values in equi-width buckets of width 0.1. We repeat this for both \sru and \wru measures. Next, we train a model on the training data set and predict the target variable for the points in each range of \ru measure. The validation results for the classification task on the {\it Adult} dataset are presented in Figures \ref{fig:exp-adult-sdt} and \ref{fig:exp-adult-wdt}. Each figure corresponds to the accuracy/error measures of the classifier over each bucket of \ru values for \sru and \wru. As the \ru values increase, the accuracy of the model drops while the FPR rises, and therefore, the model fails to capture the ground truth for the points that fall into untrustworthy regions in the data set. By repeating the aforementioned steps for the regression task on the {\it House Sales in King County} dataset, we observe similar results presented in Figures \ref{fig:exp-hs-sdt} and \ref{fig:exp-hs-wdt}. 
As the \ru value increases, the RSS of the regression model follows the same trend denoting that the model fails to perform for tuples with a high \ru value.

\begin{figure}[!tb]
    \begin{minipage}[t]{0.24\linewidth}
        \centering
        \includegraphics[width=\textwidth]{submissions/submission1/shahbazi/sdt_adult.pdf}
        \vspace{-6mm}\caption{\small{\it Adult}, efficacy of \sru  on classification}
        \label{fig:exp-adult-sdt}
    \end{minipage}\hfill
    \begin{minipage}[t]{0.24\linewidth}
        \centering
        \includegraphics[width=\textwidth]{submissions/submission1/shahbazi/wdt_adult.pdf}
        \vspace{-6mm}\caption{\small{\it Adult}, efficacy of \wru  on classification}
        \label{fig:exp-adult-wdt}
    \end{minipage}\hfill
    \begin{minipage}[t]{0.24\linewidth}
        \centering
        \includegraphics[width=\textwidth]{submissions/submission1/shahbazi/sdt_regression_house.pdf}
        \vspace{-6mm}\caption{\small{\it House Sales in King County}, efficacy of \sru on regression}
        \label{fig:exp-hs-sdt}
    \end{minipage}\hfill
    \begin{minipage}[t]{0.24\linewidth}
        \centering
        \includegraphics[width=\textwidth]{submissions/submission1/shahbazi/wdt_regression_house.pdf}
        \vspace{-6mm}\caption{\small{\it House Sales in King County}, efficacy \wru on regression}
        \label{fig:exp-hs-wdt}
    \end{minipage}
\vspace{-5mm}
\end{figure}
 %%%%%%%%%%%%%%%%%%%%%%%%%%%%%%%% RELATED WORK  %%%%%%%%%%%%%%%%%%%%%%%%%%%%%%%%
\section{Related Work}\label{related} 

Bias in data has been looked at for a long time in statistical community~\cite{neyman1936contributions} but social data presents different challenges~\cite{olteanu2019social,fairmlbook,barocas2016big,jk2019bias,drosou2017diversity}.
The diversity and representativeness of data have been widely studied~\cite{drosou2017diversity}, in fields such as social science~\cite{berrey2015enigma, dobbin2016diversity,simpson1949measurement}, political science~\cite{surowiecki2005wisdom}, and information retrieval~\cite{agrawal2009diversifying}. 
Tracing back machine bias to its source, there have been major efforts to identify different types~\cite{mehrabi2021survey, olteanu2019social,friedman1996bias} and sources~\cite{torralba2011unbiased,crawford2013hidden,diakopoulos2015algorithmic} of biases in data. Efforts to satisfy {\it responsible data} requirements~\cite{nargesian2022responsible} extend to various stages of the data analysis pipeline, including data annotation~\cite{li2020towards,lazier2023fairness}, data cleaning and repair~\cite{SalimiRHS19,tae2019data,salimi2020database}, data imputation~\cite{martinez2019fairness}, entity resolution~\cite{shahbazi2023through,fanourakis2023fairer}, data integration~\cite{nargesian2022responsible,nargesian2021tailoring}, etc. 

\paragraph{Data Coverage:}The notion of data coverage has received extensive attention from different angles. Detecting lack of coverage has been studied for datasets with discrete~\cite{asudeh2019assessing} and continuous~\cite{asudeh2021coverage} attributes populated in single or multiple \cite{lin2020identifying} relations.
To resolve insufficient coverage, \cite{accinelli2020coverage, accinelli2021impact,shetiya2022fairness}
consider resolving representation bias in preprocessing pipelines by rewriting queries into the closest operation so that certain subgroups are sufficiently represented in the downstream tasks. Alternatively, ~\cite{asudeh2019assessing,tae2021slice} propose a data collection strategy to acquire as little additional data as possible (to minimize the associated costs) to meet the representation constraints. ~\cite{sharma2020data,iosifidis2018dealing,celis2020data} opt for a data augmentation approach by adding partially altered duplicates of already existing tuples or generating new synthetic entries from existing data. Consequently, the new data set has an equal number of elements for different groups, resulting in potentially resolving the under-representation issues. Finally,  \cite{nargesian2021tailoring} utilizes data integration techniques to consolidate data from different sources into a single dataset to resolve representation bias.
Related works also include ~\cite{chung2019slice,sagadeeva2021sliceline,tae2021slice} that seek to understand if the overall performance of the model fails to reflect and performs poorly on certain slices in the data.
As alternative approaches to measure representation bias, the notion of representation rate~\cite{celis2020data} (a.k.a. equal base rate~\cite{kleinberg2016inherent}) is introduced which compared with coverage, it is more restrictive as it requires almost equal ratios from different groups.
Please refer to \cite{shahbazi2023representation} for a comprehensive survey about representation bias in data. 

\paragraph{ML Reliability:} Model-centric works for uncertainty quantification such as 
probabilistic classifiers~\cite{zadrozny2001obtaining,zadrozny2002transforming,platt1999probabilistic,niculescu2005predicting},
prediction intervals (PIs) \cite{chatfield93predictionintervals,pearce2018high,khosravi2010lower} and conformal predictions (CP)~\cite{angelopoulos2021gentle,shafer2008tutorial} that are used for measuring prediction uncertainty, are built
by maximizing the {\it expected performance} on {\it random} sample from the underlying distribution.
As a result, while providing accurate estimations for the dense regions of data (e.g. majority groups), their estimation accuracy is questionable for the poorly represented regions.
In particular, \cite{angelopoulos2021gentle} recognizes the lack of guarantees in the performance of CP for such regions.
Besides, the bulk of work on trustworthy AI provides information that {\it supports} the outcome of an ML model. For example, existing work on explainable AI, including~\cite{harradon2018causal,ribeiro2016should,gunning2019darpa}, aims to find simple explanations and rules that justify the outcome of a model.
Conversely, we aim to {\it raise warning signals} when the outcome of a model is {\it not} trustworthy. That is, to provide reasons that {\it cast doubt} on the reliability of the model outcome {for a given query point}.

 %%%%%%%%%%%%%%%%%%%%%%%%%%%%%%%% FUTURE  %%%%%%%%%%%%%%%%%%%%%%%%%%%%%%%%
% \vspace{-3mm}
\section{Final Remarks}\label{sec:conclusion}
As Data-centric AI and Responsible AI emerge as focal points in data science research, the development of Data-centric methodologies for ensuring Responsible and Trustworthy AI attracts increasing attention.
While there is some excellent work on responsible data management to achieve this goal, there remain many challenges yet to be addressed.

In this paper, we focused on a crucial aspect of responsible data -- detecting and addressing the under-representation of minorities within a data set.
We formally defined the notion of data coverage and discussed various techniques for (a) identifying lack of representation issues across different data modalities, (b) ensuring proper representation of minorities in data, and (c) limiting the scope-of-use of data sets based on their representation issues by generating proper ({\sc RU}) warning signals.
Even though the research on detecting lack of coverage issues is relatively mature, resolution techniques are still understudied.
Considering the recent advancements in Generative AI, utilizing Foundation Models and Large Language Models, and studying their limitations, for data augmentation to improve the representation of minorities at the data level seems interesting to further explore.

 %%%%%%%%%%%%%%%%%%%%%%%%%%%%%%%% BIB  %%%%%%%%%%%%%%%%%%%%%%%%%%%%%%%%
\bibliographystyle{unsrt}
\small
% \bibliography{ref}
\begin{thebibliography}{10}

\bibitem{asudeh2019assessing}
A.~Asudeh, Z.~Jin, and H.~Jagadish.
\newblock Assessing and remedying coverage for a given dataset.
\newblock In {\em ICDE}, pages 554--565. IEEE, 2019.

\bibitem{shahbazi2023representation}
N.~Shahbazi, Y.~Lin, A.~Asudeh, and H.~Jagadish.
\newblock Representation bias in data: A survey on identification and resolution techniques.
\newblock {\em ACM Computing Surveys}, 2023.

\bibitem{asudeh2021coverage}
A.~Asudeh, N.~Shahbazi, Z.~Jin, and H.~V. Jagadish.
\newblock Identifying insufficient data coverage for ordinal continuous-valued attributes.
\newblock In {\em SIGMOD}. ACM, 2021.

\bibitem{mousavi2024data}
M.~Mousavi, N.~Shahbazi, and A.~Asudeh.
\newblock Data coverage for detecting representation bias in image datasets: {A} crowdsourcing approach.
\newblock In {\em {EDBT}}, pages 47--60, 2024.

\bibitem{nargesian2021tailoring}
F.~Nargesian, A.~Asudeh, and H.~Jagadish.
\newblock Tailoring data source distributions for fairness-aware data integration.
\newblock {\em Proceedings of the VLDB Endowment}, 14(11):2519--2532, 2021.

\bibitem{nargesian2022responsible}
F.~Nargesian, A.~Asudeh, and H.~V. Jagadish.
\newblock Responsible data integration: Next-generation challenges.
\newblock {\em SIGMOD}, 2022.

\bibitem{sharma2020data}
S.~Sharma, Y.~Zhang, J.~M. R{\'\i}os~Aliaga, D.~Bouneffouf, V.~Muthusamy, and K.~R. Varshney.
\newblock Data augmentation for discrimination prevention and bias disambiguation.
\newblock In {\em AIES}, pages 358--364, 2020.

\bibitem{DBLP:journals/jair/ChawlaBHK02}
N.~V. Chawla, K.~W. Bowyer, L.~O. Hall, and W.~P. Kegelmeyer.
\newblock {SMOTE:} synthetic minority over-sampling technique.
\newblock {\em J. Artif. Intell. Res.}, 16:321--357, 2002.

\bibitem{iosifidis2018dealing}
V.~Iosifidis and E.~Ntoutsi.
\newblock Dealing with bias via data augmentation in supervised learning scenarios.
\newblock {\em Jo Bates Paul D. Clough Robert J{\"a}schke}, 24, 2018.

\bibitem{celis2020data}
L.~E. Celis, V.~Keswani, and N.~Vishnoi.
\newblock Data preprocessing to mitigate bias: A maximum entropy based approach.
\newblock In {\em ICML}, pages 1349--1359. PMLR, 2020.

\bibitem{asudeh2022towards}
A.~Asudeh and F.~Nargesian.
\newblock Towards distribution-aware query answering in data markets.
\newblock {\em Proceedings of the VLDB Endowment}, 15(11):3137--3144, 2022.

\bibitem{motwani1995randomized}
R.~Motwani and P.~Raghavan.
\newblock {\em Randomized algorithms}.
\newblock Cambridge university press, 1995.

\bibitem{chameleon}
M.~Erfanian, H.~V. Jagadish, and A.~Asudeh.
\newblock Chameleon: Foundation models for fairness-aware multi-modal data augmentation to enhance coverage of minorities.
\newblock {\em arXiv preprint arXiv:2402.01071}, 2024.

\bibitem{scholkopf1999support}
B.~Sch{\"o}lkopf, R.~C. Williamson, A.~Smola, J.~Shawe-Taylor, and J.~Platt.
\newblock Support vector method for novelty detection.
\newblock {\em NeurIPS}, 12, 1999.

\bibitem{phillips1998feret}
P.~J. Phillips, H.~Wechsler, J.~Huang, and P.~J. Rauss.
\newblock The feret database and evaluation procedure for face-recognition algorithms.
\newblock {\em Image and vision computing}, 16(5):295--306, 1998.

\bibitem{dressel2018accuracy}
J.~Dressel and H.~Farid.
\newblock The accuracy, fairness, and limits of predicting recidivism.
\newblock {\em Science advances}, 4(1):eaao5580, 2018.

\bibitem{ng2021mlops}
A.~Ng.
\newblock Mlops: From model-centric to data-centric {AI}.
\newblock 2021.

\bibitem{wing2021trustworthy}
J.~M. Wing.
\newblock Trustworthy {AI}.
\newblock {\em CACM}, 64(10):64--71, 2021.

\bibitem{kentour2021analysis}
M.~Kentour and J.~Lu.
\newblock Analysis of trustworthiness in machine learning and deep learning.
\newblock {\em InfoComp}, 2021.

\bibitem{liu2021trustworthy}
H.~Liu, Y.~Wang, W.~Fan, X.~Liu, Y.~Li, S.~Jain, A.~K. Jain, and J.~Tang.
\newblock Trustworthy {AI}: A computational perspective.
\newblock {\em arXiv preprint arXiv:2107.06641}, 2021.

\bibitem{singh2021trustworthy}
R.~Singh, M.~Vatsa, and N.~Ratha.
\newblock Trustworthy {AI}.
\newblock In {\em 8th ACM IKDD CODS and 26th COMAD}, pages 449--453. 2021.

\bibitem{kulynych2022you}
B.~Kulynych, Y.-Y. Yang, Y.~Yu, J.~B{\l}asiok, and P.~Nakkiran.
\newblock What you see is what you get: Distributional generalization for algorithm design in deep learning.
\newblock {\em arXiv preprint arXiv:2204.03230}, 2022.

\bibitem{kakade2003sample}
S.~M. Kakade.
\newblock {\em On the sample complexity of reinforcement learning}.
\newblock University of London, University College London (United Kingdom), 2003.

\bibitem{dwork2012fairness}
C.~Dwork, M.~Hardt, T.~Pitassi, O.~Reingold, and R.~Zemel.
\newblock Fairness through awareness.
\newblock In {\em ITCS}, pages 214--226, 2012.

\bibitem{techrep}
N.~Shahbazi and A.~Asudeh.
\newblock Data-centric reliability evaluation of individual predictions.
\newblock {\em CoRR, abs/2204.07682}, 2022.

\bibitem{adult}
M.~Lichman.
\newblock Adult income dataset, {UCI} machine learning repository.
\newblock \url{https://archive.ics.uci.edu/ml/datasets/adult}, 2013.

\bibitem{neyman1936contributions}
J.~Neyman and E.~S. Pearson.
\newblock Contributions to the theory of testing statistical hypotheses.
\newblock {\em Statistical Research Memoirs}, 1936.

\bibitem{olteanu2019social}
A.~Olteanu, C.~Castillo, F.~Diaz, and E.~Kiciman.
\newblock Social data: Biases, methodological pitfalls, and ethical boundaries.
\newblock {\em Frontiers in Big Data}, 2:13, 2019.

\bibitem{fairmlbook}
S.~Barocas, M.~Hardt, and A.~Narayanan.
\newblock Fairness and machine learning: Limitations and opportunities.
\newblock \url{fairmlbook.org}, 2019.

\bibitem{barocas2016big}
S.~Barocas and A.~D. Selbst.
\newblock Big data's disparate impact.
\newblock {\em Calif. L. Rev.}, 104:671, 2016.

\bibitem{jk2019bias}
J.~Kleinberg.
\newblock Fairness, rankings, and behavioral biases.
\newblock FAT*, 2019.

\bibitem{drosou2017diversity}
M.~Drosou, H.~Jagadish, E.~Pitoura, and J.~Stoyanovich.
\newblock Diversity in big data: A review.
\newblock {\em Big data}, 5(2):73--84, 2017.

\bibitem{berrey2015enigma}
E.~Berrey.
\newblock {\em The enigma of diversity: The language of race and the limits of racial justice}.
\newblock University of Chicago Press, 2015.

\bibitem{dobbin2016diversity}
F.~Dobbin and A.~Kalev.
\newblock Why diversity programs fail and what works better.
\newblock {\em Harvard Business Review}, 94(7-8):52--60, 2016.

\bibitem{simpson1949measurement}
E.~H. Simpson.
\newblock Measurement of diversity.
\newblock {\em Nature}, 163(4148), 1949.

\bibitem{surowiecki2005wisdom}
J.~Surowiecki.
\newblock {\em The wisdom of crowds}.
\newblock Anchor, 2005.

\bibitem{agrawal2009diversifying}
R.~Agrawal, S.~Gollapudi, A.~Halverson, and S.~Ieong.
\newblock Diversifying search results.
\newblock In {\em WSDM}, pages 5--14. ACM, 2009.

\bibitem{mehrabi2021survey}
N.~Mehrabi, F.~Morstatter, N.~Saxena, K.~Lerman, and A.~Galstyan.
\newblock A survey on bias and fairness in machine learning.
\newblock {\em ACM Computing Surveys (CSUR)}, 54(6):1--35, 2021.

\bibitem{friedman1996bias}
B.~Friedman and H.~Nissenbaum.
\newblock Bias in computer systems.
\newblock {\em TOIS}, 14(3):330--347, 1996.

\bibitem{torralba2011unbiased}
A.~Torralba and A.~A. Efros.
\newblock Unbiased look at dataset bias.
\newblock In {\em CVPR 2011}, pages 1521--1528. IEEE, 2011.

\bibitem{crawford2013hidden}
K.~Crawford.
\newblock The hidden biases in big data.
\newblock {\em Harvard business review}, 1(4), 2013.

\bibitem{diakopoulos2015algorithmic}
N.~Diakopoulos.
\newblock Algorithmic accountability: Journalistic investigation of computational power structures.
\newblock {\em Digital journalism}, 3(3):398--415, 2015.

\bibitem{li2020towards}
Y.~Li, H.~Sun, and W.~H. Wang.
\newblock Towards fair truth discovery from biased crowdsourced answers.
\newblock In {\em SIGKDD}, pages 599--607, 2020.

\bibitem{lazier2023fairness}
S.~Lazier, S.~Thirumuruganathan, and H.~Anahideh.
\newblock Fairness and bias in truth discovery algorithms: An experimental analysis.
\newblock {\em arXiv preprint arXiv:2304.12573}, 2023.

\bibitem{SalimiRHS19}
B.~Salimi, L.~Rodriguez, B.~Howe, and D.~Suciu.
\newblock Interventional fairness: Causal database repair for algorithmic fairness.
\newblock In {\em {SIGMOD}}, pages 793--810. {ACM}, 2019.

\bibitem{tae2019data}
K.~H. Tae, Y.~Roh, Y.~H. Oh, H.~Kim, and S.~E. Whang.
\newblock Data cleaning for accurate, fair, and robust models: A big data-{AI} integration approach.
\newblock In {\em DEEM workshop}, pages 1--4, 2019.

\bibitem{salimi2020database}
B.~Salimi, B.~Howe, and D.~Suciu.
\newblock Database repair meets algorithmic fairness.
\newblock {\em ACM SIGMOD Record}, 49(1):34--41, 2020.

\bibitem{martinez2019fairness}
F.~Mart{\'\i}nez-Plumed, C.~Ferri, D.~Nieves, and J.~Hern{\'a}ndez-Orallo.
\newblock Fairness and missing values.
\newblock {\em arXiv preprint arXiv:1905.12728}, 2019.

\bibitem{shahbazi2023through}
N.~Shahbazi, N.~Danevski, F.~Nargesian, A.~Asudeh, and D.~Srivastava.
\newblock Through the fairness lens: Experimental analysis and evaluation of entity matching.
\newblock {\em Proceedings of the VLDB Endowment}, 16(11):3279--3292, 2023.

\bibitem{fanourakis2023fairer}
N.~Fanourakis, C.~Kontousias, V.~Efthymiou, V.~Christophides, and D.~Plexousakis.
\newblock Fairer demo: Fairness-aware and explainable entity resolution.
\newblock 2023.

\bibitem{lin2020identifying}
Y.~Lin, Y.~Guan, A.~Asudeh, and H.~Jagadish.
\newblock Identifying insufficient data coverage in databases with multiple relations.
\newblock {\em Proceedings of the VLDB Endowment}, 13(12):2229--2242, 2020.

\bibitem{accinelli2020coverage}
C.~Accinelli, S.~Minisi, and B.~Catania.
\newblock Coverage-based rewriting for data preparation.
\newblock In {\em EDBT Workshops}, 2020.

\bibitem{accinelli2021impact}
C.~Accinelli, B.~Catania, G.~Guerrini, and S.~Minisi.
\newblock The impact of rewriting on coverage constraint satisfaction.
\newblock In {\em EDBT Workshops}, 2021.

\bibitem{shetiya2022fairness}
S.~Shetiya, I.~P. Swift, A.~Asudeh, and G.~Das.
\newblock Fairness-aware range queries for selecting unbiased data.
\newblock In {\em ICDE}. IEEE, 2022.

\bibitem{tae2021slice}
K.~H. Tae and S.~E. Whang.
\newblock Slice tuner: A selective data acquisition framework for accurate and fair machine learning models.
\newblock In {\em SIGMOD}, pages 1771--1783, 2021.

\bibitem{chung2019slice}
Y.~Chung, T.~Kraska, N.~Polyzotis, K.~H. Tae, and S.~E. Whang.
\newblock Slice finder: Automated data slicing for model validation.
\newblock In {\em ICDE}, pages 1550--1553. IEEE, 2019.

\bibitem{sagadeeva2021sliceline}
S.~Sagadeeva and M.~Boehm.
\newblock Sliceline: Fast, linear-algebra-based slice finding for ml model debugging.
\newblock In {\em SIGMOD}, pages 2290--2299, 2021.

\bibitem{kleinberg2016inherent}
J.~Kleinberg, S.~Mullainathan, and M.~Raghavan.
\newblock Inherent trade-offs in the fair determination of risk scores.
\newblock {\em arXiv preprint arXiv:1609.05807}, 2016.

\bibitem{zadrozny2001obtaining}
B.~Zadrozny and C.~Elkan.
\newblock Obtaining calibrated probability estimates from decision trees and naive bayesian classifiers.
\newblock In {\em ICML}, volume~1, pages 609--616. Citeseer, 2001.

\bibitem{zadrozny2002transforming}
B.~Zadrozny and C.~Elkan.
\newblock Transforming classifier scores into accurate multiclass probability estimates.
\newblock In {\em SIGKDD}, pages 694--699, 2002.

\bibitem{platt1999probabilistic}
J.~Platt et~al.
\newblock Probabilistic outputs for support vector machines and comparisons to regularized likelihood methods.
\newblock {\em Advances in large margin classifiers}, 10(3):61--74, 1999.

\bibitem{niculescu2005predicting}
A.~Niculescu-Mizil and R.~Caruana.
\newblock Predicting good probabilities with supervised learning.
\newblock In {\em Proceedings of the 22nd international conference on Machine learning}, pages 625--632, 2005.

\bibitem{chatfield93predictionintervals}
C.~Chatfield.
\newblock Prediction intervals.
\newblock {\em Journal of Business and Economic Statistics}, 11:121--135, 1993.

\bibitem{pearce2018high}
T.~Pearce, A.~Brintrup, M.~Zaki, and A.~Neely.
\newblock High-quality prediction intervals for deep learning: A distribution-free, ensembled approach.
\newblock In {\em International conference on machine learning}, pages 4075--4084. PMLR, 2018.

\bibitem{khosravi2010lower}
A.~Khosravi, S.~Nahavandi, D.~Creighton, and A.~F. Atiya.
\newblock Lower upper bound estimation method for construction of neural network-based prediction intervals.
\newblock {\em IEEE transactions on neural networks}, 22(3):337--346, 2010.

\bibitem{angelopoulos2021gentle}
A.~N. Angelopoulos and S.~Bates.
\newblock A gentle introduction to conformal prediction and distribution-free uncertainty quantification.
\newblock {\em arXiv preprint arXiv:2107.07511}, 2021.

\bibitem{shafer2008tutorial}
G.~Shafer and V.~Vovk.
\newblock A tutorial on conformal prediction.
\newblock {\em Journal of Machine Learning Research}, 9(3), 2008.

\bibitem{harradon2018causal}
M.~Harradon, J.~Druce, and B.~Ruttenberg.
\newblock Causal learning and explanation of deep neural networks via autoencoded activations.
\newblock {\em arXiv preprint arXiv:1802.00541}, 2018.

\bibitem{ribeiro2016should}
M.~T. Ribeiro, S.~Singh, and C.~Guestrin.
\newblock " why should i trust you?" explaining the predictions of any classifier.
\newblock In {\em SIGKDD}, pages 1135--1144, 2016.

\bibitem{gunning2019darpa}
D.~Gunning and D.~Aha.
\newblock Darpa’s explainable artificial intelligence ({XAI}) program.
\newblock {\em AI Magazine}, 40(2):44--58, 2019.

\end{thebibliography}

\end{document}

\end{article}








\end{articlesection}

% put the news items below- there can be multiple news sections
% each with its own title
% news will usually have an author as well as a title, 
% e.g. TCDE elections
% news articles are in the same format as letters
% typically, news articles will be stored in a directory called "news"

%\begin{newssection}{News headline}

% insert news items here; news will typically have authors
% see the Sept. 2018 issue for an example

%\begin{news}{news item title}
%{author name}{author affiliation}
%\input{news/news-article.tex}
%\end{news}
%
%\newpage


%\end{newssection}

\begin{callsection}

%  This section will be empty for your version
%
%  Calls for papers section.  Use the callsection environment.
%  Each call for papers is contained in an call environment, where the single 
%  required options to \begin{call} is the name of the conference.
% typically calls are stored in a "calls" directory
%
%\begin{call}{name of conference}
%\centerline{\includegraphics[width=\textwidth, bb= 0 0 590 760]{calls/conference-name.pdf}}
%\end{call}
%\begin{call}{ICDE 2019 Conference}
%\centerline{\includegraphics[width=\textwidth, bb= 0 0 610 790] {../Dec-2018/calls/icde19.pdf}} 
%\centerline{\includegraphics[width=\textwidth, bb= 0 0 590 760] {calls/icde19.pdf}}
%\end{call}
\begin{call}{TCDE Membership Form}
%\centerline{\includegraphics[width=\textwidth, bb= 0 0 610 790]
%\centerline{\includegraphics[width=\textwidth, bb= 0 0 590 760] {../Dec-2018/calls/tcde.pdf}}
\centerline{\includegraphics[width=\textwidth, bb= 0 0 590 760] {../2020-09/calls/tcde.pdf}}
\end{call}

\end{callsection}

\end{bulletin}
\end{document}
