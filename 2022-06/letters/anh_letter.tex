\documentclass[11pt]{article} 

\usepackage{deauthor,times,graphicx}
%\usepackage{url}

\begin{document}
%There is a convergence of databases and blockchain systems. 
%The former contain techniques that store and
%manage data efficiently and at scale, while the latter combine ideas from distributed systems, security, and
%data management to protect data and computation from malicious attackers. 
The success of blockchains prompts the database community to revisit the
trade-offs between security and performance in data management systems.  In
fact, database researchers in the past few years have made significant contributions to the
understanding and advancement of blockchains.  This issue focuses on 
systems that recently emerged (or was resurrected) at the intersection of
blockchains and databases. We call them {\em transparent databases}, which
provide data security through transparency. In particular, these systems enable
the users to securely verify that both the data and its history have not been
tampered with. They achieve transparency by maintaining data in an append-only
ledger (a core data structure in blockchains), and protecting the ledger with
authenticated data
structures such as Merkle trees (core data structures in both blockchains and secure outsourced databases).
Although the security community have been deploying similar systems specifically for public key
infrastructure, for example, key transparency and certificate transparency, our community's interest in
transparent databases stems from the challenges in building general-purpose, high-performance systems that
solve real-world data management problems. This issue contains perspectives from expert researchers working
on this topic. They share their views on the state-of-the-art, the use cases, and the future research
directions.  

The issue opens with a contribution from Henry F. Korth, in which he reminds us of how transparency is often
at odds with privacy, and more importantly, their trade-offs are made for us by a trusted party. He explains
how blockchains, which removes the trust on opaque institutions, can revolutionize most data-driven
applications. He highlights two building blocks that are vital to such revolutions: Merkle trees and zero
knowledge proofs. The former allows for selective disclosure of information, and the latter for proving
correct execution without revealing the data.  When combined, they enable not only integrity protection of
data and computation, but also of the data provenance. The second paper, by Zhe Peng, Jianliang Xu, Haibo Hu,
and Lei Chen, demonstrates how these techniques can be used to give data owners control over their data. The
authors present a timely example of COVID-19 data sharing, in which users want fine-grained control of what
data to share and with whom. For this
use case, a Merkle tree is built over the user data and its root is published on a blockchain. To selectively
share some functions on some data, the user constructs a Merkle proof for the data, and a zero-knowledge proof
showing that the output is computed correctly on the input whose Merkle root is on the blockchain. 

Blockchains are an important component of transparent databases, because at very least they can serve as a
public bulletin board where commitments are stored. The next two papers describe the latest techniques for
improving performance and security of blockchains. Junchao Chen, Suyash Gupta, Sajjad Rahnama, and Mohammad
Sadoghi, present interesting insights on the advantages and limitations of two types of consensus protocols.
On the one hand, Byzantine Fault Tolerance  (BFT) protocols have high performance, but require 
strong identities, and they can be broken when the attacker steals more than $f$ private
keys. On the other hand, Proof of Work (PoW) protocols are harder to break, but they are unsustainable. The
authors then propose a new protocol, called Proof of Collaboration, that aims to have the best of both worlds.
Deepal Tennakoon and Vincent Gramoli discuss the state-of-the-art on blockchain sharding --- the popular
database technique in which the data is partitioned into multiple shards. Sharding helps scale blockchain
throughputs by distributing the works. However, the key challenge in blockchain sharding, which does not exist
in traditional database settings, is the presence of malicious attackers that
influence shard assignments in order to insert themselves to target shards. If successful, the attackers can
break the fault tolerance threshold of the target shard and subsequently break the security of the
blockchain. The authors explain how {\em probabilistic sharding} relies on trusted sources of randomness
to avoid such attacks. They propose another layer of defense, which is to make sharding transparent
such that users can verify how the shard is formed. 

While transparent databases can be built directly on existing blockchains that are mainly designed for
cryptocurrency or assets management applications, the next paper by Dumitrel Loghin describes another approach
based on {\em blockchain databases}. Such systems integrate blockchain and database features, and are  
classified into permissioned blockchains, hybrid systems, and ledger databases. They share a similar
architecture that consists of a ledger storage for data history, a database storage for the states, and a
broadcasting service for coordination. Hybrid systems adopt either an out-of-blockchain database design, in
which the system starts with a blockchain and builds database features on top of it, or an out-of-database
blockchain design, in which the system starts with a database and builds blockchain features to it. The author
compares the performance of different systems and shows that ledger databases achieve the
highest throughputs. The last paper, by Meihui Zhang, Cong
Yue, Changhao Zhu, and Ziyue Zhong, provides in-depth analysis of ledger databases. The authors identify a
number of design choices that impact the overall security and performance. They then
propose a benchmarking framework, named LedgerBench, for comparing different systems. The framework contains
workloads that stress the unique features of ledger databases such as verification and auditing. It has
flexible APIs such that new workloads and systems can be easily added. This paper also presents interesting
experimental results on their implementations of three commercial systems: LedgerDB from Alibaba, QLDB from
Amazon, and SQL Ledger from Microsoft. One of the main findings is that updating the ledger and verification
constitute the main performance bottleneck.  

It is a real pleasure working with the authors for this issue. Their insights are refreshing and I believe
the readers will learn much from reading their contributions.  
\end{document}
