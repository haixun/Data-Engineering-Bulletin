\section{Proof-of-Concept Evaluation}
We now present an initial evaluation of our vision of \DualChain{} architecture,
which we implement in the open-sourced \ResDB{} fabric~\cite{poe,geobft,rcc, 
ringbft}.\footnote{For our proof-of-concept of \DualChain{} 
architecture, we employ an experimental version of \ResDB{} with the codename of {\em NexRes} 
(Next Generation \ResDB{}); this is an architectural rewrite of the \ResDB{} 3.0 
(the latest stable version).}

\begin{figure*}
    \centering
    \setlength{\tabcolsep}{1pt}
    \begin{tabular}{cc@{\quad}cc}
        \pbfttput&
        \pbftlat\\
    \end{tabular}
    \vspace{-3mm}
    \caption{Evaluating peak throughput and commitment latency attained by \pbft{} consensus in \DualChain{}.}
    \vspace{-3mm}
    \label{fig:performance_pbft}
\end{figure*}

\begin{figure*}
    \centering
    \setlength{\tabcolsep}{1pt}
    \begin{tabular}{cc@{\quad}cc}
        \EvalBatchTput{(a)}&
        \EvalBatchLatency{(b)}&
        \EvalMinerTput{(c)}&
        \EvalMinerLatency{(d)}\\
    \end{tabular}
    \vspace{-3mm}
    \caption{Evaluating \PoC{} throughput and average settlement latency with 
    different difficulty (D), miners (M) and block batch size (B).}
    \vspace{-3mm}
    \label{fig:performance_poc}
\end{figure*}

{\bf Experimental Setup.}
We use Oracle Cloud Infrastructure's VM.Standard2.8 architecture to deploy miners 
and replicas ($16$ cores, \SI{8.2}{Gbps} bandwidth, $120$ GB Memory). In our 
experiments, we generate $400$ million client requests of size \SI{64}{B} each,
while client response has size \SI{17}{B}. We average results over three runs.
We require clients to sign their messages using \texttt{ED25519} while replicas use 
\texttt{CMAC}.

{\bf Batching.}
We employ the standard practice of batching client transactions, which we refer to 
\textit{transaction batching}, to optimize the \pbft{} consensus. Additionally, during 
the \PoC{} consensus, each miner aggregates multiple batches from \pbft{} consensus, '
which we refer to \textit{block batching}, prior to mining. 

 
{\bf \pbft{} Scaling.}
In Figure~\ref{fig:performance_pbft}, we gauge the peak throughput and 
\textit{commitment latency} incurred by the \pbft{} consensus of our \DualChain{} 
architecture. This is an important metric as it informs the rate at which we 
can reply to the clients. For this experiment, we increase the number of replicas 
from $16$ to $120$ and require the primary to process a batch of $100$ 
transactions per consensus; transaction batch size is set to $100$. 
As expected, on increasing the number of replicas, there is a drop in peak 
throughput and an increase in incurred latency, which remains in the subsecond 
range. This phenomenon occurs because, at each setting, we are approximately 
hitting the network bandwidth; as the number of replicas increases, more messages 
are communicated per consensus. In summary, the peak throughput reaches well 
over \SI{800}{k} transactions/second at $16$ replicas while sustaining over 
\SI{100}{k} transactions/second even when scaling to as many as $120$ replicas.


{\bf \PoC{} Scaling.}
Next, we study the impact of our \PoC{} consensus protocol on the \DualChain{} 
architecture. We deploy $120$ replicas for running \PBFT{} consensus. For the 
\PoC{} setting, we set the solution space parameter to $42$ nonce bits and 
split it into equal disjoint slices based on the number of miners. Notice that 
the difficulty of each problem is the number of leading zeros in the hash. In 
Figure~\ref{fig:performance_poc}, we present our results; here $D$ refers to 
difficulty (with the default of $D=9$), $M$ refers to the number of miners 
(with the default of $M=120$), and $B$ refers to the number of batches in a block. 
As stated earlier, for every $B$ blocks produced by \PBFT{} a single block is 
notarized and minted by \PoC{}.
 
In Figures~\ref{fig:performance_poc}(a) and~\ref{fig:performance_poc}(b), 
we fix the number of miners to $120$, which allows creating $120$ equal slices, and 
increase the block size from \SI{120}{k} to \SI{280}{k}. We test at two 
difficulty levels: $D=8$ and $D=9$. Our results indicate that $D=8$ is 
relatively easy, due to which each miner has to perform a smaller amount 
of work. As a result, any increase in batch size does not increase peak 
throughput. Hence, we test on $D=9$ at which mining smaller batch size 
impacts the system throughput as miners have to participate in a larger 
number of consensus rounds. On further increasing the block size, 
we observe that the throughput hits the \PBFT{}'s peak as desired. Thus, 
notarizing the blocks by \PoC{} no longer hinders the system throughput, 
it only prolongs the \textit{settlement latency} as expected. 
%
When examining the end-to-end system throughput of \DualChain{} which 
includes both \PBFT{} and \PoC{}, we observe a sustained throughput 
of \SI{105}{k} with a commitment latency of $1$ second and the settlement 
latency of $198$ seconds when the batch size is set to \SI{200}{k}. 
{\em Note:} We could not test at difficulty beyond $D=9$ as each nonce 
computation became prohibitively time- and resource-intensive given 
our available commodity hardware.

In Figures~\ref{fig:performance_poc}(c) and~\ref{fig:performance_poc}(d), 
we gauge the performance of \PoC{} when there are $64$ to $120$ miners. 
For these experiments, we also test at three different block batch sizes. 
As expected, the degree of collaborative mining is directly proportional 
to the number of miners. As we increase the number of miners, there is an
increase in peak throughput and a decrease in the settlement latency.


%In our experiments, we realize \mo{as expected not we realized} that when 
%the mining time is always smaller than the transaction producing time, 
%the miners will be always waiting for the transactions before they start 
%the \PoW{}. In this case, the \PoW{} throughput will be the same as the 
%\PBFT{} throughput. For example, it is much easier to find out the 
%solutions when we set the difficulty as eight. When we set up BBS larger 
%than 80K which will need 1.5 minute to fulfill in our \PBFT{} setting 
%but the mining latency is less than 1.5 minutes, the miners are idle 
%at most of the time. The throughput is consistence.  Thus, the throughput 
%will always be around (see  $\ref{fig:performance_poc}$). We also found 
%when we increase BBS to 120K, the throughput for difficulty nine will 
%also be around 100k. Due to the limitation of computation resource, 
%we are not able to evaluate the performance of difficulty ten.
%
%\mo{add the precise stats that justify our choice, for example, on 
%average solving puzzle with 9 leading zeros requires 3.724789943 minutes with 
%2 minutes standard deviation, which is why this block range make 
%sense} 
 

%\mo{for caption text, please see our earlier paper for more standard 
%formats.}
