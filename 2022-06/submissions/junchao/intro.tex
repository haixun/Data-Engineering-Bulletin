\section{Introduction}  
\label{s:intro}
The past decade has observed a surge in the design and deployment of 
decentralized systems. A key reason for this surge is the growing desire in 
the society to have self-governing democratic financial systems that are not 
under the control of a privileged set of entities. A central control often 
translates to a forced trust model with limited provision to support 
transparency and accountability. The adoption of Blockchain, for example, 
is a by-product of the ability to break away from the forced-central control 
in a trust-worthy fashion~\cite{blockchain-book}. The emerging blockchain 
platforms facilitate a reliable execution of any digital contracts 
(i.e., transactions) in a decentralized manner despite the existence of malicious 
actors. At the core of any blockchain platform is a Byzantine 
fault-tolerant (\BFT{}) consensus protocol and a tamper-proof replicated 
ledger~\cite{bedrock,blockchain-book,scalable-ledger}. The \BFT{} 
protocol helps to achieve {\em consensus} on the order of incoming client 
requests among all the replicas, while the ledger logs this agreement. 

Traditional \BFT{} protocols expect a {\em permissioned} system where the 
identities of all the replicas (i.e., participants) are known prior to any 
consensus as they rely on having a verifiable voting right in a democratic 
setting. These protocols rely on a {\em communication-oriented} consensus 
model, where all the participants exchange endorsements across multiple 
rounds before they can reach a decision~\cite{sharper,pbftj,ahl,poe,rcc,geobft,flexitrust,ringbft,mirbft,basil,hotstuff}.
In these protocols, a system of $\n{}$ replicas can reach a common decision 
if at most $\f{}$ of them are malicious, such that $\n{} \ge 3\f{}+1$. 
The $\n{}$ parties are said to reach a decision when at least a majority 
of honest parties agrees to that decision. This decision is logged by 
requiring all the agreeing parties to {\em sign} the decision. Hence, the 
reached decision is considered {\em tamper-proof} because it has support 
of a majority of honest participants.

Despite being around for more than two decades, traditional \BFT{} protocols 
did not see any major practical applications until the introduction of 
blockchain technology. We attribute {\em two} key factors for this lack of 
adoption. (i) To ensure that the malicious participants do not spawn multiple 
identities, these \BFT{} protocols need an authority (i.e., {\em a forced 
trust gateway}) to verify and register every participant to verify every 
vote~\cite{sybil-attack}; some participants may find this intrusive if they 
do not want to reveal their personal information. (ii) To overwrite the ledger, 
malicious participants just require access to the private keys of honest 
participants. In a sense, the proof of the validity of the ledger is not 
self-contained, and it operates on the assumption that the private-keys 
are kept safe externally indefinitely.

To resolve these challenges, initial blockchain platforms such as 
Bitcoin~\cite{bitcoin} and Ethereum~\cite{ether} offer a {\em permissionless} 
model of consensus. These systems employ the {\em Proof-of-Work} (\PoW{}) 
protocol~\cite{bitcoin,ether}, which follows a {\em computation-oriented} 
consensus model and requires all the participants to compete with each other 
and try to solve a complex puzzle. Whichever participant solves the puzzle 
first gets to add a new entry ({\em block}) to the ledger. As a result, 
\PoW{} protocol eliminates the three challenges seen by traditional \BFT{} 
protocols. (i) Malicious participants can spawn multiple identities, but what 
actually matters is the available compute power. (ii) Each block includes 
the hash of the previous block; overwriting the ledger requires recomputing 
all the blocks making it computationally infeasible. (iii) Since reaching the 
consensus is based on presenting the proof of work that is embedded on the 
ledger (i.e., self-contained), there is no longer any need for external 
safe-keeping of private keys to sign endorsements.

These properties offered by \PoW{} protocol help blockchain platforms to 
design a {\em decentralized economy}, where any person can participate in the 
consensus process, and the economy has a self-generating currency to monetize 
its participants. Monetizing the participants is necessary as the \PoW{} 
protocol expects the participants to spend their resources to solve a complex 
puzzle. Clients of the Bitcoin platform, create transactions that exchange 
Bitcoins and send them to the participants (miners) in the \PoW{} protocol. 
These miners check if the transaction is valid; the client has sufficient 
Bitcoins to transfer. If the transaction is valid, they run \PoW{} protocol to include 
this transaction in the ledger. The winning miner of \PoW{} gets a portion of 
the client's Bitcoin as {\em fees}, while the mining process (\PoW{}) mints  
new tokens to fund the economy. This new token is transferred to the winning 
miner's account and is recorded as a transaction in the block.

The key challenge with platforms like Bitcoin is their {\em practicality}. 
These platforms have abysmally low throughput in the order of $10$ transactions 
per second in part due to inadequate choice of small block sizes. Furthermore, as 
more miners join the network, the complexity of the puzzle has to be increased. 
For example, the complexity of the current Bitcoin puzzle is so high that the 
miners work in large groups to have any positive probability of creating the 
next winning block~\cite{blockchain-book}. Moreover, as miners are competing 
with each other, it leads to massive wastage of computational resources (energy) 
as only the winning miner's efforts are recorded and rewarded. This results in an unsustainable ecosystem~\cite{badcoin,badbadcoin}.

We observe these challenges in the designs of existing \BFT{} protocols and 
blockchain platforms and envision a \DualChain{} system that learns from these 
models and eliminates their key challenges. Essentially, we aim to establish a 
new research agenda; a new field of hybrid consensus protocols that depart from 
competitive consensus to a collaborative consensus that is both resilient and 
sustainable. Our \DualChain{} architecture takes a step in this direction by 
running two consensuses on each client transaction while ensuring there is no 
increase in the latency observed by the client. Each client request is first 
ordered through a state-of-the-art \BFT{} consensus protocol ({\em commitment}), 
subsequently, this ordered request is engraved into the ledger through the 
\PoW-style consensus ({\em settlement}). Specifically, \DualChain{} causes no 
increase in commitment latency while improving the settlement latency observed 
by existing protocols. Ordering the client transaction through a \BFT{} consensus 
protocol first allows our \DualChain{} system to guarantee the following benefits: 
(i) clients receive low-latency responses, and (ii) \PoW{} participants no longer 
need to compete, resulting in a high-throughput sustainable chain. As a result, 
instead of employing the \PoW{} for consensus, we design a novel protocol that 
allows miners to collaborate. We refer to this paradigm as 
{\em Power-of-Collaboration} (\PoC{}). 

Our \PoC{} protocol splits the complex puzzle into disjoint slices and requires 
each miner to work on a distinct slice. This slice distribution significantly 
reduces the resource wastage and provides each honest miner with a reward 
for each new transaction added to the ledger. As each ledger entry is added 
collaboratively, any malicious entity that wishes to overwrite the ledger 
needs to match the computational power of all the existing miners making it 
practically impossible. These features of our \DualChain{} system make it 
lucrative; its design is the bedrock for a secure and efficient decentralized 
economy.
