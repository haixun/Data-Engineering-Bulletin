%!TEX root=../2022_IEEE_DEB_Vizier.tex
%%%%%%%%%%%%%%%%%%%%%%%%%%%%%%%%%%%%%%%%%%%%%%%%%%%%%%%%%%%%%%%%%%%%%%%%%%%%%%%%
\section{Versioned Data Artifact Store}
\label{sec:data-artifacts}

Steps in Vizier workflows create, read, and update \textit{``data artifacts''} or artifacts for short.
To maximize interoperability between cells, Vizier establishes standards for how these artifacts are serialized, which we now discuss. Versions of artifacts are immutable. 
An update to an artifact creates a new object representing the updated artifact. Immutability greatly simplifies handling of state in Vizier and enables us to share artifact versions across multiple revisions of a workflow.

\mypara{Dataset Artifacts}
Tabular data is represented by Vizier as a Spark dataframe~\cite{AX15}, a logical encoding of how a dataset is derived from source data.
The choice to store datasets through a logical representation (specifying the computation instead of its result) is driven largely by the need for managing annotations (requirement \textbf{A3} which is implemented by rewriting the computation to handle annotations), but also results in lower space consumption.
% Because data frames are not serializable, Vizier instead stores dataframe constructors: standardized logic for instantiating a data frame, e.g. by transforming or applying other artifacts.

% Vizier's dataframe constructors are a close analog of SparkML's \texttt{Pipeline}s, which provide a serializable representation of data frame transformation logic.
% Although Vizier provides a pipeline data frame constructor, Pipelines are strictly a representation of transformations, normally requiring glue code to apply the pipeline to an independently loaded dataset;
% Constructors completely capture the logic needed to instantiate a dataframe.

\mypara{Parameter Artifacts}
A parameter artifact can be any primitive value of any data type supported by Apache Spark; We note that this includes simple nested collections like Arrays and Maps.
This type of artifact provides a way to parameterize scripts and other system components, particularly for non-technical users.
% Vizier provides a ``Set Parameter'' command that allows users to specify parameters;
and parameter artifacts are used to pass simple data (e.g., simple constants in Python) between cells.

In addition to these two artifact types, Vizier also supports plots (stored in the vegalite~ \cite{DBLP:journals/tvcg/SatyanarayanMWH17} format), blobs and uninterpreted files, and language-specific constructs, e.g., Python function definitions. For lack of space, we do not further  discuss these other artifact types.

% \mypara{Chart Artifacts}
% Although Vizier provides means for plot generation (e.g., Bokeh, Matplotlib) through messages emitted by a cell, a standardized vegalite~\cite{DBLP:journals/tvcg/SatyanarayanMWH17} artifact format allows multiple cells to collaboratively edit a plot.
% For example, a ``Chart'' command makes it easy for users to quickly visualize a dataset, and then a Spark or Python cell can refine the resulting visualization by tweaking labels, adding markers, or adjusting colors.

% \mypara{Blob and File Artifacts}
% As a fallback to other data formats, Vizier allows cells to export state through a simple blob format.
% For example, Python falls back to encoding global state in `pickle' format, which is stored as a simple blob.
% The blob store is encoded as an in-memory array; If the artifact is large enough to warrant disk IO (e.g., a large model), requires a directory structure (e.g., parquet files), or is otherwise tied to the filesystem, Vizier provide a file artifact format.
% File artifacts are stored in a unique location tied to the artifact identifier.

% \mypara{Function Artifacts}
% Symbols representing language-specific constructs like functions, classes, and imported modules are encoded as raw code.
% Each artifact stores condains code that defines a single symbol under a specific name.
% At time of writing, support for function artifacts requires explicit cross-platform implementation support, although efforts are under way to provide a standardized cross-language interface.




%%% Local Variables:
%%% mode: latex
%%% TeX-master: "../2022_IEEE_DEB_Vizier"
%%% End:
