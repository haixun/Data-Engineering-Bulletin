
% VLDB template version of 2020-08-03 enhances the ACM template, version 1.7.0:
% https://www.acm.org/publications/proceedings-template
% The ACM Latex guide provides further information about the ACM template

% Regular Research Papers (up to 12 pages excluding references)
\documentclass[sigconf, nonacm]{acmart}

%% The following content must be adapted for the final version
% paper-specific
\newcommand\vldbdoi{XX.XX/XXX.XX}
\newcommand\vldbpages{XXX-XXX}
% issue-specific
\newcommand\vldbvolume{14}
\newcommand\vldbissue{1}
\newcommand\vldbyear{2020}
% should be fine as it is
\newcommand\vldbauthors{\authors}
\newcommand\vldbtitle{\shorttitle} 
% leave empty if no availability url should be set
\newcommand\vldbavailabilityurl{URL_TO_YOUR_ARTIFACTS}
% whether page numbers should be shown or not, use 'plain' for review versions, 'empty' for camera ready
\newcommand\vldbpagestyle{plain} 

% \usepackage[numbers,sort]{natbib}


\newcommand{\fix}{\marginpar{FIX}}
\newcommand{\new}{\marginpar{NEW}}
\newcommand{\yym}[1]{{\color{blue}{[(YYM): #1]}}}
\newcommand{\xz}[1]{{\color{red}{[(XZ): #1]}}}
\newcommand{\lsx}[1]{{\color{cyan}{[LSX: #1]}}}

% \usepackage[sort]{natbib}
\usepackage{pifont}
\usepackage{amsmath,bm}
\usepackage{amsthm}
\usepackage{bbm}
\usepackage{amsthm}
\usepackage{pifont}
\usepackage[ruled,linesnumbered]{algorithm2e}
\newtheorem{definition}{Definition}[section]
\newtheorem{assumption}{Assumption}
\usepackage{wrapfig}
\def\dname{CMLP}

\usepackage{wrapfig}
\usepackage{tabu}
\urlstyle{same}
\usepackage{xcolor}
\usepackage[normalem]{ulem}
\usepackage[noend]{algpseudocode}
\usepackage{stfloats}
% new package
\usepackage{multirow}
\usepackage{caption}
\usepackage{float}
\usepackage{microtype}
\newlength\savewidth\newcommand\shline{\noalign{\global\savewidth\arrayrulewidth
  \global\arrayrulewidth 1pt}\hline\noalign{\global\arrayrulewidth\savewidth}}

\begin{document}
\title{Distilling Causal Metaknowledge from Knowledge Graphs}

\author{Yuan Meng$^1$, Yancheng Dong$^1$, Shixuan Liu$^2$, Xinzhu Ma$^3$, Chaohao Yuan$^1$, Yue He$^1$, Furui Liu$^4$, Pei Jian$^5$, Peng Cui$^1$}
\email{{yuanmeng, dongyc20, yuanch22, heyue18 }@mails.tsinghua.edu.cn, 
liushixuan@nudt.edu.cn
}
\email{xinzhu.ma@sydney.edu.au, 104256204@qq.com,  jpei@cs.sfu.ca,  cuip@tsinghua.edu.cn}
\affiliation{$^1$ Tsinghua University $^2$National University of Defense Technology 
\\  $^3$ The University of Sydney  $^4$ Huawei Technologies Ltd.  $^5$ Simon Fraser University}







\begin{abstract}
The recent surge in blockchain applications has accelerated the research in 
designing efficient decentralized currencies. Building a decentralized economy 
on the traditional byzantine fault-tolerant (\BFT{}) protocols or the Proof-of-Work 
(\PoW{}) consensus protocol is inadequate as the immutability of the ledger created 
by the former is at the mercy of the long-term safe-keeping of private keys of 
participants, while the latter yields an extremely inefficient and environmentally 
unsustainable consensus. To ameliorate this situation, we envision the design of 
our \DualChain{} architecture, which offers the best of both worlds. Our \DualChain{} 
design runs a traditional \BFT{} protocol to commit client transactions and employs 
our novel Power-of-Collaboration (\PoC{}) protocol to notarize the \BFT{} chain. 
Unlike \PoW{}, our \PoC{} protocol advocates for participants to work together 
collaboratively instead of competing (often selfishly), which results in a 
safe, high-throughput, and resource-efficient consensus design.


\end{abstract}

\maketitle
\section{Introduction}
\label{section:introduction}
\section{Introduction}

Information Retrieval (IR) involves retrieving a set of candidates from a large document collection
given a user query. The retrieved candidates may be further reranked to bring the most relevant ones to the top, constituting a typical retrieve-and-rerank (R\&R) framework \cite{wang2018evidence, hu2019retrieve}.
Reranking generally improves the ranks of relevant candidates among those retrieved, thus improving on metrics such as Mean Reciprocal Rank (MRR) \cite{Craswell2009} and Normalized Discounted Cumulative Gain (nDCG) \cite{jarvelin2002cumulated}, which assign better scores when relevant results are ranked higher. 
However, retrieval metrics like Recall@K, which mainly evaluate the presence of relevant candidates in the top $K$ retrieved results, remain unaffected.
Increasing Recall@K can be key, especially when the retrieved results are used in downstream knowledge-intensive tasks \cite{petroni2021kilt} such as open-domain question answering \cite{chen2017reading, chen2020open, gangi2021synthetic}, fact-checking \cite{thorne2018fever}, entity linking \cite{hoffart2011robust,sil2013re,sil2018neural} and dialog generation \cite{dinan2018wizard, komeili2022internet}.

Most existing neural IR methods use a dual-encoder retriever \cite{karpukhin2020dense, khattab2020colbert} and a subsequent cross-encoder reranker \cite{nogueira2019passage}. 
Dual-encoder\footnote{We use the terms bi-encoder and dual-encoder interchangeably in this paper.} models leverage separate query and passage encoders and perform a late interaction between the query and passage output representations. This enables them to perform inference at scale as passage representations can be pre-computed. Cross-encoder models, on the other hand, accept the query and the passage together as input, leaving out scope for pre-computation. The cross-encoder typically provides better ranking than the dual-encoder---thanks to its more elaborate computation of query-passage similarity informed by cross-attention---but is limited to seeing only the retrieved candidates in an R\&R
framework.


\begin{wrapfigure}{r}{0.42\linewidth}
    \centering
    \includegraphics[width=1.0\linewidth]{submissions/Revanth2024/figures/cross_encoder_feedback_2.png}
    \caption{\textsc{ReFIT}: The proposed method for reranker relevance feedback. We introduce an inference-time distillation process (step 3) into the traditional retrieve-and-rerank framework (steps 1 and 2) to compute a new query vector, which improves recall when used for a second retrieval step (step 4).}
    \label{fig:overall_framework}
    \vspace{-1em}
\end{wrapfigure}

Since the more sophisticated reranker often generalizes better at passage scoring than the simpler, but more efficient retriever, here we propose to use relevance feedback from the former to improve the quality of query representations for the latter directly \textit{at inference}.
Concretely, after the R\&R pipeline is invoked for a test instance, we update the retriever's corresponding query vector by minimizing a distillation loss that brings its score distribution over the retrieved passages closer to that of the reranker.
The new query vector is then used to retrieve documents for the second time. 
This process effectively teaches the retriever how to rank passages like the reranker---a stronger model---for the given test instance.
Our approach, \textsc{ReFIT}\footnote{\textsc{ReFIT} stands for \textbf{Re}ranker \textbf{F}eedback at \textbf{I}nference \textbf{T}ime}, is lightweight as only the output query vectors (and no model parameters) are updated, ensuring comparable inference-time latency when incorporated into the R\&R framework. 
Figure \ref{fig:overall_framework} shows a schematic diagram of our approach, which introduces a distillation and a second retrieval step into the R\&R framework.
By operating exclusively in the representation space---as we only update the query vectors---our framework yields a parameter-free and architecture-agnostic solution, thereby providing flexibility along important application dimensions, e.g., the language, domain, and modality of retrieval. 
We empirically demonstrate this effect by showing improvements in retrieval on multiple English domains, across 26 languages in multilingual and cross-lingual settings, and in different modalities such as text and video retrieval.
 

Our main contributions are as follows:
\begin{itemize}
    \item We propose \textsc{ReFIT}, an inference-time mechanism to improve the recall of retrieval in IR using relevance feedback from a reranker.
    \item Empirically, \textsc{ReFIT} improves retrieval performance in multi-domain, multilingual, cross-lingual and multi-modal evaluation.
    \item The proposed distillation step is fast, considerably increasing recall without any loss in ranking performance over a standard R\&R pipeline with comparable latency.
\end{itemize}









\section{Preliminaries and Problem Statement}
\label{section:model}
\subsection{Definitions and Notations}

In this paper, we follow the definition of knowledge graph as in~\cite{ji2021survey}:
\begin{definitionnew}
A \textbf{Knowledge Graph (KG)} is defined as $\mathcal{G}=(\mathcal{E},\mathcal{R},\mathcal{F})$, where
$\mathcal{E}$, $\mathcal{R}$ and $\mathcal{F}$ are sets of entities, relations and facts, respectively.
Every fact is a triple $(e_h,R,e_t) \in \mathcal{F}$, where $e_h, e_t\in \mathcal{E}$ and $R \in \mathcal{R}$ are head entity, tail entity and the relation between entities, respectively. Without loss of generality, we simultaneously represent a fact as $R(e_h,e_t)$.
\end{definitionnew}

%Here we introduce a similar data structure with KG: heterogeneous information graph, following the definition in ~\cite{sun2011pathsim}:

%\begin{definition}
%A \textbf{heterogeneous information network (HIN)} \cite{sun2011pathsim}is a directed graph $G=(V, E, \Phi, \Psi)$, where $V$ is the set of nodes (or entities) of the graph; $E \subseteq V \times V$ is the set of edges connecting the nodes in $V$; and $\Phi$ and $\Psi$ are functions for labeling nodes and edges. We have $\Phi: V \rightarrow \mathcal{A}$, where $\mathcal{A}$ is the set of node classes, and $\Psi: E \rightarrow \mathcal{B}$, where $\mathcal{B}$ is the set of edge types.
%\end{definition}

%Given a complex heterogeneous information network, it is necessary to provide its meta level (i.e., schema-level), which is called network schema.

%\begin{definition}
%A \textbf{network schema}\cite{sun2011pathsim} is a meta template for a HIN $G=(V, E)$ with the object type mapping $\phi: V \rightarrow \mathcal{A}$ and the link mapping $\psi: E \rightarrow \mathcal{R}$, which is a directed graph defined over object types $\mathcal{A}$, with edges as relations from $\mathcal{R}$, denoted as $T_G=(\mathcal{A}, \mathcal{R})$.
%\end{definition}

%Generally, both KG and HIN are graph-based models for representing relational data.
%The HIN provides this constraint via labeling function $\Phi$, requiring each node to correspond to a single node type, whereas the description of node types is not necessary for the general KG.
%The relationship between KG and HIN is a subject of debate in academia.
%A recently published article by


%\begin{definition}
%A \textbf{meta-path}\cite{sun2011pathsim} $\mathcal{P}$ is a path defined on the graph of network schema $T_G=(\mathcal{A}, \mathcal{R})$, and is denoted in the form of $A_1 \stackrel{R_1}{\longrightarrow} A_2 \stackrel{R_2}{\longrightarrow} \ldots \stackrel{R_l}{\longrightarrow} A_{l+1}$, which defines a composite relation $R=R_1 \circ R_2 \circ \ldots \circ R_l$ between type $A_1$ and $A_{l+1}$, where - denotes the composition operator on relations.
%\end{definition}




%A Knowledge graph(KG) $\mathcal{G}=(\mathcal{E},\mathcal{R},\mathcal{T})$ can be regarded as a kind of heterogeneous information network\lsx{?}, where
%$\mathcal{E}$
% $\mathcal{E} = \{e_1,e_2,\cdots,e_{|\mathcal{E}|}\}$
%is the \textit{entity} set,
% of $\mathcal{G}$
% and $|\mathcal{E}|$ is the number of the entities in $\mathcal{G}$;
%$\mathcal{R}$
% $\mathcal{R} = \{R_1,R_2,\cdots,R_{|\mathcal{R}|}\}$
%is the \textit{relation} set,
% in $\mathcal{G}$, which includes $|\mathcal{R}|$ different types of relations;
%and $\mathcal{T} \in \mathcal{E} \times \mathcal{R} \times \mathcal{E}$ is the \textit{fact} set, where the facts are represented as triples.
%A triple can be denoted as $<e_h,R,e_t>$ or $R(e_h,e_t)$,
%where $e_h$ is the head entity, $e_t$ is the tail entity, and these two entities are connected by the relation $R$ to form a \textit{fact} in $\mathcal{G}$.
% Under the closed world assumption (CWA)
% ~\cite{hogan2021knowledge}, the fact set includes all the existing relationships between any entity pair from $\mathcal{R} \times \mathcal{R}$.
First-order logic (FOL) offers a pivotal way to represent real-world knowledge for reasoning. Horn rules, as a special and typical case of FOL rules, propose to represent a target relation by a body of conjunctive relations.

\begin{definitionnew}
\label{def:horn_rule}
A \textbf{Horn Rule}, generally chain-like, is given as,
\begin{equation*}
    R_h(x,y) \leftarrow R_{b_1}(x,z_1) \circ \cdots \circ R_{b_l}(z_{l-1},y)
\end{equation*}
where, $R_h(x,y)$ signifies the rule head (target relation) that we wishes to reason and $R_{b_1}(x,z_1) \circ \cdots \circ R_{b_l}(z_{l-1},y)$ is the rule body (relation path). For simplicity, we denote a Horn rule as $R_h:\mathbf{R_b}$, where $\mathbf{R_b}=[R_{b_1}, \cdots, R_{b_l}]$. To reason $R_h$, the size of the rule space is $|\mathcal{B}_h|$.
Every \textbf{closed path} of such Horn rule is required to: 1) connect $(x,y)$ via the rule body, which is a sequence of relations $\mathbf{R_b}$, and 2) ensure $(x,y)$ are accessible directly via the target relation $R_h$. Closed paths are also known as \textbf{rule instances}.
\end{definitionnew}

%Each KG corresponds to a concept set $\mathcal{C}$, and each concept $C \in \mathcal{C}$ corresponds to a concept-specific entity set  $\mathcal{E}_C \in \mathcal{E}$.
%Therefore the concept can be treated as the label of the entity class.


\subsection{Problem Statement}
The goal of this work is to learn the \textbf{causal rule} that is formalized as the horn rule.
Specifically, the objective of traditional logical rule learning is to assign a plausibility score $\mathbf{S}(R_h|\mathbf{R_b})$ to each rule in the discovered rule space, which can be subsequently aggregated to answer queries about the KG.
Currently, plausibility scores are defined over closed paths (e.g., the PCA confidence for AMIE~\cite{galarraga_amie_2013}), which are correlational observations.
We have demonstrated that these scores are prone to spurious correlations and therefore result in inaccurate predictions under OoD settings, in Sec. \ref{section:introduction}.
Therefore the other aim is to give a plausibility score based on causal strength.


In this paper, the causal rules are mined for link prediction in KG.
We follow the commonly accepted problem definition of link prediction in KG~\cite{rossi2021knowledge,tiwari2021revisiting}:
given an observed KG $\mathcal{G}$ with missing facts, our goal is to predict the correct entity for an given query $(e_h,R_h,?)$ (or $(?,R_h,e_t)$).
%For simplicity, we refer to the known entity in the prediction as source entity and the entity to be predicted as target entity.
% Specifically, the objective of logical rule learning is to assign a plausibility score $\mathbf{S}(R_h|\mathbf{R_b})$ to each rule in the discovered rule space, which are subsequently aggregated to rank all possible answers. Currently, plausibility score are defined over closed paths (e.g., the PCA confidence for AMIE~\cite{galarraga_amie_2013}), which are associational observations. We elaborate these scores are prone to spurious correlations and therefore result in inaccurate predictions under OoD settings, in Sec. \ref{section:model}. 


% introduce the representation 
% \section{Structural Causal Knowledge Model}

\section{Proposed Method: ~\dname}
\label{section:model}
In this section, we introduce the proposed approach~\dname~ which learns causal rules for KG link prediction.
\dname~ first transforms the relational data into the propositional data to conduct statistical analysis (Sec. \ref{sec:tabularnizar}).
Then \dname~ presents a local causality identification algorithm based on the $d$-seperation criterion to efficiently mines interpretable causal rules (Sec. \ref{sec:discovery}).
Finally, a specific causation-based score is applied in predictor to answer the queries with learned causal rules (Sec. \ref{sec:link_pred}).
The pipeline of \dname~is illustrated in Fig. \ref{fig:framwork}.
% the proposed CFLP first transforms the input data from graph representation into tabular representation for better  (Sec. \ref{sec:tabularnizar}). Then we design a local causality identification algorithm based on the $d$-seperation criterion to efficiently mines interpretable causal rules (Sec. \ref{sec:discovery}).
% Finally, a specific link prediction approach is applied to integrates the weight information  derived from the causality test (Sec. \ref{sec:link_pred}).
\begin{figure*}[htbp]
\begin{center}
\includegraphics[width=18cm]{submissions/causal-meta-knowledge/figures/cmlp.png}
\end{center}
\caption{The framework of~\dname. Particularly, CFLP first transforms the relational data into propositional data for better statistical analysis. Then it mines interpretable causal rules, which can be interpreted as a kind of metaknowlege\cite{evans2011metaknowledge}.
Finally, a plausibiliy score derived from the causality test is applied in predictor to rank the answers of the given query.}
\label{fig:framwork}
\end{figure*}
\begin{figure*}[hbp]
\begin{center}
\includegraphics[width=14cm]{submissions/causal-meta-knowledge/figures/transformer.jpg}
\end{center}
\caption{The process of knowledge graph transformer}
\label{fig:tabular}
\vspace{-0.6cm}
\end{figure*}

\subsection{Knowledge Graph Transformer}
\label{sec:tabularnizar}
% The inference rules are generally defined at the concept level, and the link prediction tasks are operated at the entity level.
% For example, based on the inference rule of relation \texttt{Treats(Compound, Disease)}, we can conduct a specific query $(?, Treat, Nicotine Dependence)$.
% In~\dname, to learn the conceptual causal rules, we first introduce the possible causes of a conceptual fact.
Traditional causal discovery algorithms are defined on propositional data, with well-defined variables and samples, which do not exist in relational data like KGs.
Therefore, we give the definition and scope of the variables we study in the causal discovery phase by mapping the potential causes and queried relations into variables.
Then we give the practical approach for transforming KG into tabular data, whose horizontal axis are the variables we defined.
The process is shown schematically in Fig.~\ref{fig:tabular}



\subsubsection{Causal variables in KG}
The causal rule can be interpreted as a description of causal relationship between the body and the head.
Naturally, we formulate variables based on the elements of rules.
% Due to the difficulty of data acquisition, for knowledge graphs extracted from texts, there is often no corresponding attribute information of entities corresponding to them.
% The connection information between a pair of entities is the only source we can access to infer the missing link between them.
% For example, the genetic connection structure is
%$Compound \stackrel{Bind}{\longrightarrow} Gene \stackrel{Associate}{\longrightarrow} Disease$.
% $Bind(Compound, Gene) \circ Associate(Gene, Disease)$.

% \yym{a sample to explain this model}
% \yym{rewrite}
% Relational causal model (RCM) is a Baysian network, which encodes the causal dependency for the relational data.
% RCM is designed for the relational database, in which the entity classes and relations are represented as tables with attributes.
% RCM depends on the known relational skeleton, which includes all the relations between entities, to infer the unknown attributes.
% However, in practice, the relations among entities are missing and are treated as learned target in some application, such as link prediction task for KG.
% And normally, the attributes of entity classes are lack in the real KG.
% Therefore in this section, we will build a causal model for the relations in KG without information of the entities' attributes.
% With the relational skeleton, RCM can represents the conditional independent relationship among the attributes of the entity class.

% We first introduce several concepts and definitions, which are the components of our model.

\noindent
% \begin{definition}
% \textbf{Knowledge Graph Schema(KGS.)}
% A KGS $S=(\mathcal{C}^S, \mathcal{R}^S, \mathcal{V},\mathcal{B} )$ is a directed graph, defined on a KG $\mathcal{G}=(\mathcal{E},\mathcal{R},\mathcal{T})$ with relation set $\mathcal{R}^S \in \mathcal{R}$,
% where each vertice $v \in \mathcal{V}$ corresponds to a concept $C \in \mathcal{C}^S$ and each edge $b \in \mathcal{B}$ corresponds to a relation $R  \in \mathcal{R}^S$.
% \end{definition}

%\begin{definition}
%A \textbf{meta structure} connecting source node $n_h$ and target node $n_t$
%$M=(\mathcal{C}_M, \mathcal{R}_M, \mathcal{V},\mathcal{B}, n_h, n_t)$ is a directed graph, defined on a KG $\mathcal{G}=(\mathcal{E},\mathcal{R},\mathcal{T})$ with concept set $\mathcal{C}$, where $\mathcal{V}$ and $\mathcal{B}$ are node set and edge set, respectively, and each node $v \in \mathcal{V}$ corresponds to a concept $C \in \mathcal{C}_M$ each edge $b \in \mathcal{B}$ corresponds to a relation $R  \in \mathcal{R}_M$.

% with relation set $\mathcal{R}_M \in \mathcal{R}$,
% where each vertice $v \in \mathcal{V}$ corresponds to a concept $C \in \mathcal{C}^S$ and each edge $b \in \mathcal{B}$ corresponds to a relation $R  \in \mathcal{R}^S$.
%\end{definition}
% A relation path is a template and proxy of sub-KG that describes a kind of topological relationship between a pair of concepts.
% Therefore, the bodies of rules can be conceived as topology attributes for all concept pairs.
% And the head of the rule is also a special attribute.
% Here we give the some officially definitions to introduce the assumption on the causes of concept-level links in KG.
%The conceptual triple is a kind of meta structure.
%There is a slightly difference with the past definition of meta structure \cite{huang2016meta}: we do not restrict the meta structure to be a directed acyclic graph (DAG), because the direction of the connection from head entity to tail entity may not be unidirectional or directed under different relational semantics, such as $<Compound,Associate,Disease>$\lsx{?}

%\begin{definition}
%A \textbf{Rule-induced Variable} $X=(C_h,C_t).M$ is defined on two concepts $C_h$ and $C_t$ and a meta structure $M$, where $C_h$ and $C_t$ correspond to $n_s$ and $n_t$ in $M$, respectively.
%\end{definition}

\begin{definitionnew}
For entity pair $(x,y)$, its \textbf{Rule-induced Variable} $X_k=f_{\mathcal{G}}(x,y|\mathbf{R_b}^k)$, where $\mathbf{R_b}^k$ corresponds to the rule body in the $k$-th rule $R_h:\mathbf{R_b}^k (k \in \{1, ..., |\mathcal{B}_h|\})$. The assignment function $f_{\mathcal{G}}(\cdot|\mathbf{R_b})$ can be either connectivity feature or path count for $\mathbf{R_b}$ in KG $\mathcal{G}$.
\end{definitionnew}

The head of rule also induces a special variable $Y=f_{\mathcal{G}}(x,y|R_h)$.
In the real link prediction task, the queries are normally on a specific relation, such as \texttt{Treat} in drug repurposing.
Therefore, the causal rule mining problem is to discovery the causal relationship between variables $X_k=f_{\mathcal{G}}(x,y|\mathbf{R_b}^k), k \in \{1, ..., |\mathcal{B}_h|\}$ and variable  $Y=f_{\mathcal{G}}(x,y|R_h)$.
% With the definition of variables, the causal rule discovery is transformed to the traditional causal discovery task
Then we introduce the practical approach that we transform the KG into tabular data for causality analysis.

% In this paper, we treat the \textit{entity pair} as the study unit, since we focus on the generation mechanism of the relations in KG without the attributes of a single entity and each relation must involve two entities.
% The concept pair label the classes of the entity pair, and the meta structure $M$ reflect the connection characteristics between the two entities. Besides, We assume the concept-level connection information of given concept pair as the factors, which potentially \textit{cause} the specific relation between these two concepts.

%\begin{assumption}
%\textbf{Candidate causes of conceptual triples:}
%\label{ass:candidata}
%given a variable $(C_h,C_t).M_q$ defined by a queried conceptual triple $M_q=<C_h,R,C_t>$, the candidate causes of it are any variables $(C_h^c,C_t^c).M_c$ defined by the meta structure $M_c=(\mathcal{C}_M, \mathcal{R}_M, \mathcal{V},\mathcal{B}, n_h, n_t)$, where $C_h^c=C_h$ and $C_h^t=C_t$.
% and $C_1^{Ca} \cap C_1 \neq \emptyset, C_2^{Ca} \cap C_2 \neq \emptyset$.
%\end{assumption}

% \begin{assumption}
% \textbf{Candidate causes of rule-induced variables:}
% \label{ass:candidata}
% given a variable $X_k=f_{\mathcal{G}}(x,y|\mathbf{R_b}^k)$ defined by the $k$-th rule to reason queried relation $R_th$, the candidate causes of it are any variables $X_j=f_{\mathcal{G}}(x,y|\mathbf{R_b}^j)$ defined by other rule $j(j\neq k)$ in the rule space.
% % and $C_1^{Ca} \cap C_1 \neq \emptyset, C_2^{Ca} \cap C_2 \neq \emptyset$.
% \end{assumption}

% A KGS-based variable $X=f(\mathcal{G},S,e^h,e^t)$ is a function of two entity variables $e^h,e^t$, with each entity variable corresponding to two nodes of KGS $S=(\mathcal{C}^S, \mathcal{R}^S, \mathcal{V},\mathcal{B} )$, where the KGS is defined on KG $\mathcal{G}$.


% \begin{itemize}
%     \item a set of meta structural attributes$(C_h,C_t).\mathcal{M}$
%     \item a set of parents $\text{Pa}((C_h,C_t).M)=\{(U_1,\dots,U_l\}$, where each $U_i$ has the form $C_h,C_t).M_i$;
%     \item a conditional probability distribution (CPD) that represents $P(C_h,C_t).M| C_h,C_t).M)$
% \end{itemize}
% A KGS-based dependency model $\mathcal{M}_\theta$ has two parts:

% 1. The structure $\mathcal{M}=(\mathcal{V},\mathcal{D})$: a directed graph with each node corresponding to a KGS-based variables $X \in \mathcal{V}$ and each edge corresponding to a schema-level dependency defined between two variables in $V$.

% 2. The parameters $\theta$: a conditional probability distribution
% \begin{equation}
% \begin{aligned}
% P(X=f(\mathcal{G},S^X,e^h,e^t)|\text{~parents}(X))
% \end{aligned}
% \end{equation}
% for each KGS-based variable with the entity variables and $\text{parents}(X)=\{Y|Y=f(\mathcal{G},S^Y,e^h,e^t)\}$ is the set of parent KGS-based variables, which have the same entity variable pair with $X$.




% Given the entity set, if we can model the





% \noindent
% \begin{definition}
% \textbf{KGS-based variable}
% A KGS-based variable $X=f(\mathcal{G},S,e^h,e^t)$ is a function of two entity variables $e^h,e^t$, with each entity variable corresponding to two nodes of KGS $S=(\mathcal{C}^S, \mathcal{R}^S, \mathcal{V},\mathcal{B} )$, where the KGS is defined on KG $\mathcal{G}$.
% \end{definition}

% The mapping function $f$ is as following:
% $$
% f(\mathcal{G},S,E^h,E^t)=\left\{
% \begin{aligned}
%     1  & \text{~~if there is an instance path of KGS~}
%        S \text{~with~} e^h,e^t \text{in} \ \mathcal{G}.\\
%     0 & \text{~~otherwise}.
% \end{aligned}
% \right.
% $$


% \begin{definition}
% \textbf{Schema-level dependency}
% A schema-level dependency $Y \to X$, defined between two KGS-based variables $X=f(\mathcal{G},S^X,e^h,e^t)$ and $Y=f(\mathcal{G},S^Y,e^h,e^t)$, is a directed probabilistic dependence from  $Y$ to $X$, which have the same entity variables.
% \end{definition}

% \begin{definition}
% \textbf{KGS-based dependency model}
% A KGS-based dependency model $\mathcal{M}_\theta$ has two parts:

% 1. The structure $\mathcal{M}=(\mathcal{V},\mathcal{D})$: a directed graph with each node corresponding to a KGS-based variables $X \in \mathcal{V}$ and each edge corresponding to a schema-level dependency defined between two variables in $V$.

% 2. The parameters $\theta$: a conditional probability distribution
% \begin{equation}
% \begin{aligned}
% P(X=f(\mathcal{G},S^X,e^h,e^t)|\text{~parents}(X))
% \end{aligned}
% \end{equation}
% for each KGS-based variable with the entity variables and $\text{parents}(X)=\{Y|Y=f(\mathcal{G},S^Y,e^h,e^t)\}$ is the set of parent KGS-based variables, which have the same entity variable pair with $X$.

% \end{definition}


% \noindent
% \begin{definition}
% \textbf{Instances of KGS-based variable}
% A instance of a KGS-based variable $X(S,v^h,v^t)$ is an entity pair $(e^h, e^t)$, with $e^h$ belonging to the concept which $v^h$ corresponds to and $e^t$ belonging to the concept which $v^t$ corresponds to.
% \end{definition}






% In the last section, we introduce the SPRM, which is a Bayesian model defined on the meta structural variables.
% In this section, we will presents the practical method to learn the SPRM $\Pi$ from a given KG.
% Especially, we focus on the learning of the structure $\mathcal{S}$.
% Because structure $\mathcal{S}$ is the fundamental of parameter $\mathcal{S}$ in SPRM, and the exploration of model parameters demands assumptions on the probability distribution followed by the data, hence limiting the applicability of the model.
% In the next section, we will show the structural information is already available for downstream tasks such as link prediction.
% The architecture of the structure learning method is shown in Fig~\ref{fig:framwork}.


% \subsection{Knowledge Graph Tabularnizar}

% \begin{wrapfigure}{r}{0.2\texotwidth}
% \includegraphics[width=0.2\textwidth]{./fig/path_examplev2.pdf}
% \caption{An illustrative KGS, which includes four concepts and three relations.}
% \label{fig:path_example}
% \end{wrapfigure}


% The traditional causal discovery algorithm aims to learn the whole causal structure, without any prior knowledge about the skeleton of the structure or the orientation of the causal relationship.
% which includes all causal relationships between the involved variable.
% The causal structure can be represented as a DAG, where each node indicates an involved variable.
% Discovering causal relationships from observational data
% by a set of causal relationships among a set of variables, and the causal discovery is normally regarded as a the problem of learning the \textit{whole} causal structure from observational data in the prior works.
% However, the fundamental tasks in KG, such as KG completion and KG reasoning, mainly concern the single relation.
% Therefore, the main goal of rule mining is to find the body structure for a given head relation.
% Consequently, in this paper, we only need to solve a \textit{local} causal discovery problem, which is to find all the direct causes for a given KGS-based variable.
% Here we propose an efficient causal rule discovery method, \textit{\dname}, which performs the following steps:
\subsubsection{Transforming knowledge graph into propositional data}

% \begin{figure}[t]
% \centering
% \includegraphics[width=0.45\linewidth]{./figures/path_examplev2.pdf}
% \caption{An illustrative meta structure, which includes four concepts and three relations.}
% \label{fig:path_example}
% \end{figure}

(1) Step-1: Searching candidate causes $X$.
According to definition~\ref{def:horn_rule}, any rule-induced variable $X$, which is defined on entity pair$(x,y)$ and seeks to help reasoning over $R_h$, is a valid candidate cause for $Y=f_{\mathcal{G}}(x,y|R_h)$.
% Further we observed the following phenomena:
% \begin{itemize}
% \item[1)] Any graph contains two specific nodes can be represented as a path between them (duplicate nodes are permitted).
% For example, as shown in Fig.\ref{fig:path_example}, the structure between $C_1$ and $C_2$ can be determined by the path $C_1 \stackrel{R_1}{\longrightarrow} C_2 \stackrel{R_3}{\longrightarrow} C_4 \stackrel{R_3}{\longleftarrow} C_2 \stackrel{R_2}{\longrightarrow} C_3$.
% \item[2)] A complex meta structure can be considered as a combination of multiple meta paths.
% For example, the meta structure $M_N$ in Fig.~\ref{fig:tabular} can be decomposed into two meta path(shown in Fig.~\ref{fig:decompose}).
% The causal function $(C_h,C_t).M_q=f((C_h,C_t).M_N)$ can be represented by
% $(C_h,C_t).M_q=f((C_h,C_t).M_{N_1},(C_h,C_t).M_{N_2})$
% \begin{figure}[t]
% \centering
% \includegraphics[width=0.8\linewidth]{./figures/decompose.pdf}
% \caption{Two meta paths decomposed from the meta structure $M_N$ in Fig.~\ref{fig:tabular}.}
% \label{fig:decompose}
% \end{figure}
% \begin{equation}
% \label{eq:decompose_meta_path}
% \begin{aligned}
% M_{N_1}:Compound1 \stackrel{Bind}{\longrightarrow}Gene1\stackrel{Participate}{\longrightarrow} Biological Process1 \stackrel{Participate}{\longleftarrow} Gene2\stackrel{Associate}{\longrightarrow} Disease1 \\
% M_{N_2}:Compound1 \stackrel{Bind}{\longrightarrow}Gene1\stackrel{Upregulate}{\longrightarrow} Compound1 \stackrel{Treat}{\longleftarrow}Disease1.
% \end{aligned}
% \end{equation}
% $M_{N_1}:Compound1 \stackrel{Bind}{\longrightarrow}Gene1\stackrel{Participate}{\longrightarrow} Biological Process1 \stackrel{Participate}{\longleftarrow} Gene2\stackrel{Associate}{\longrightarrow} Disease1$ and $M_{N_2}:Compound1 \stackrel{Bind}{\longrightarrow}Gene1\stackrel{Upregulate}{\longrightarrow} Compound1 \stackrel{Treat}{\longleftarrow}Disease1$.
% \item[2)]  There are many well-studied path finding algorithms, which can search the paths under different types of constraints, such as Dijkstra’s algorithm~\cite{lanning2014dijkstra}, A* search~\cite{cui2011based}, best-first search~\cite{heusner2018best}, etc.
% These off-the-shelf methods can be directly adopted to our framework.
% In the experiments, we adopt the best-first search algorithm.
% \item[3)] Lots of existing rule mining methods~\cite{sadeghian2019drum,yang2017differentiable,ho2018rule} designed for the \textit{closed rules}, meaning that each entity set appears in at least two edges of the rule.
% It renders the path-like graph structure in most cases.
% \end{itemize}
So we find all the candidate causes by searching all the paths between entity pairs $(x, y)$, which have the relation $R_h$ between them.
There are many well-studied path finding algorithms, which can search the paths under different types of constraints, such as Dijkstra’s algorithm~\cite{lanning2014dijkstra}, A* search~\cite{cui2011based}, best-first search~\cite{heusner2018best}, etc.
% These off-the-shelf methods can be directly adopted to our framework.
In the experiments, we adopt the best-first search algorithm.
Since the number of candidate causes can be the power level of the number of relation types, we require that the length of the path is no more than $\ell$, where $\ell$ is the hyper-parameters.
In the experiments of this paper, we set $\ell$ as 3.
% need to be supported by at least $a_{sup}$ entity pair $(e^h,e^t)$ in the training KG, and the length of the path is no more than $\ell$, where $a_{sup}$ and $\ell$ are the hyper-parameters.
(2) Step-2: generating samples.
% In tabular data, variables have distinct physical meanings, and the values of each instance are accessed via data collection.
% There is no explicit quantitative information inside triple-based knowledge graphs.
% For further causality learning, the assignment rules for the variables specified by structural attributes must be agreed upon in order to generate a numerical sample set that can be statistically analyzed.
In this paper, we use the connectivity as the assignment function to get quantitative samples.
\noindent
\begin{definitionnew} the
\textbf{binary assignment function of rule-induced variable} is as following:
$$
f_\mathcal{G}(e_h,e_t|\mathbf{R_b}^k)=\mathbbm 1_{\rm con}(e_h,e_t|\mathbf{R_b}^k)
$$
where $\mathbbm 1_{\rm con}(e_h,e_t|\mathbf{R_b}^k) \in \{0,1\}$ checks whether there exists a path instance of $\mathbf{R_b^k}$ between $e_h$ and $e_t$ in KG $\mathcal{G}$.
\end{definitionnew}

% \begin{definition} the
% \textbf{binary assignment function of meta structural attribute-defined variable} is as following:
% $$
% f_{\mathcal{G},(C_h,C_t).M}(e_h,e_t)=\left\{
% \begin{aligned}
%     1  & \text{~~if condition one is satisfied}. \\
%     0 & \text{~~otherwise}.
% \end{aligned}
% \text{for~} e_h \in
%     \mathcal{E}_h \text{~and~} e_t \in \mathcal{E}_t\\
% \right.
% $$
% where condition one is there is an meta structure instance of $M$ between $e_h$ and $e_t$ in KG $\mathcal{G}$.
% \end{definition}

In this assignment function, we consider whether two entities can be connected via a relation path, instead of the entities or number of the connection paths.
There are two main reasons for this design:
(1) We expect that the mined causal relationship can be generalized to any dataset in this domain.
% Without the attribute of the entities, we can not model the entities based on the their common features.
Thus, if we want to distinguish different entities which instantiate the meta structure, we need to build a multinomial model for all possible entities.
The multinomial would be infeasibly large.
And our model can not be applied to any scenario which contain an unseen entity.
(2) this function can be seen as an aggregation function to summary the connection information between entities.
The aggregation function is very common in the causal relation model~\cite{maier2010learning,lee2016learning,lee2020towards,salimi2020causal}.
With the aggregation function, we can build a concise and expressive model.
Since the only thing we need is whether the entities are connected.
Based on this assignment function, by sampling entity pairs in the training KG and querying the corresponding variable values, we can obtain tabular data for causal analysis.




% (2) Step-2: Refinement of the Identical Relations.
% In KG, there may be some identical relations, even though they have different relation names.
% For example, Wife(A,B) $\leftrightarrow$ Husband(B,A), if A is the wife of B, then B must be the husband of A.
% However, they will lead the invalid independence test in the following causal discovery step, even though these two relations have very strong causal relationship with each other.
% In particular, based on the causal variable and sample definitions in KG (Sec.~\ref{sec:v_a_s}), these two relations are the same variables for the causal discovery method, since the values of their samples are the same all the time.
% When one relation is treated as the conditional variable in the independent test of the other one, the conditional independence (CI) test $CI(X,Y|X)$ will be judged as independent.
% So for an analyzed KGS $S_E$, we search all the identical KGSs in the input KG and temporarily remove them from the candidate cause set in the independent test period.
% The causal rules which include the identical KGSs will have the highest weight, when they are applied into the downstream tasks.
\subsection{Causal MetaKnowlege Discovery via $d$-seperation Criterion}
\label{sec:discovery}

The $d$-separation criteria~\cite{glymour2016causal} (see Definition 3.4) is a sufficient and necessary condition for the compatibility of a probability distribution with a causal model in the form of a directed acyclic graph (DAG).
It states that a joint probability distribution of a set of random variables is compatible with the DAG (each node represents one of the given variables and each arrow represents the possibility of causal influence) if and only if the distribution satisfies a set of conditional independence relations encoded in the structure of the DAG.
Therefore, $d$-separation is widely used in the algorithms in discovering causal structure\cite{giudice2022dual,sondhi2019reduced,gerhardus2020high}.

\begin{definitionnew}
\textbf{d-seperation.} A path $p$ is blocked by a set of nodes $Z$ if and only if:

1.$p$ contains a chain of nodes $A \to B\to C$ or a fork $A \gets B\to C$ such that the middle node $B$ is in $Z$ (i.e., $B$ is conditioned on), or:

2. $p$ contains a collider $A \to B \gets C$ such that the collision node $B$ is not in $Z$, and no descendant of $B$ is in $Z$.

If $~Z$ blocks every path between two nodes $X$ and $Y$, then $X$ and $Y$ are $d$-separated, conditional on $Z$, and thus are independent conditional on $Z$.
\end{definitionnew}


\begin{algorithm2e}
\caption{Local causal metaknowledge discovery}
\label{alg:pc-like}
\KwIn{$Y$ and $\{y_i\}, i=1,\dots,N$ : variable and samples of queried variable $(C_h,C_t).M_q$ ;
 $\mathcal{X}^{Ca} = \{X_k\}, k=1,\dots,K$ and $\{\{x_i\}_k\}, i=1,\dots,N$: variables and samples of candidate causes; }
\KwOut{causes $\mathcal{X}^{C}$ of $Y$}
level $d \gets 0$\;
\While{$d<= |\mathcal{X}^{Ca}|-1$}{

\For{each $X_k \in  \mathcal{X}^{Ca}$}{
    \For{each subset $\mathcal{Z} \in \mathcal{X}^{Ca} \backslash \{ X_k\}$ and $|\mathcal{Z}|=d$}{
    Test CI($X_k,Y|\mathcal{Z}$)\;
    \If{CI($X_k,Y|\mathcal{Z}$)}{
        Test CI($\mathcal{Z},Y|X_k$) {(Reverse CI test.)} \;
        \If{not CI($\mathcal{Z},Y|X_k$)}{
            Remove $X_k$ from $\mathcal{X}^{Ca}$\;
            Break\;
        }
    }
}
}
$d \gets d+1$\;
}
$\mathcal{X}^{C} = \mathcal{X}^{Ca}$
\end{algorithm2e}

In this work, we design an efficient causal metaknowledge discovery algorithm based on $d$-separation.
With $d$-separation, we can get the following conclusion: given any set of variables $Z$, where $Z$ does not include $X$, $X$ is not independent of its parent node ({i.e.} direct cause).
% \yym{theorem}
Based on this conclusion, we can obtain a criterion for determining the direct cause of variable $X$.
Furthermore, we design the following local causal metaknowledge discovery algorithm (Algo.~\ref{alg:pc-like}) for the queried variable $Y=f_{\mathcal{G}}(x,y|R_h)$.
We only mine the direct cause of $Y$, instead of the entire causal structure of variable set $\mathcal{X}^{Ca} \cup \{Y\}$.
Particularly, given a queried variable $Y=f_{\mathcal{G}}(x,y|R_h)$, for each candidate cause in $\mathcal{X}^{Ca}$~(denoted as variable $X_k$),
the proposed algorithm decides whether $X_j$ should be retained in candidate causes set $\mathcal{X}^{Ca}$ by testing the independence of $X_k$ and $Y$ conditioning on a subset $\mathcal{Z}$ of $\mathcal{X}^{Ca}\backslash \{X_k\}$.
The conditional independent(CI) tests are organised by levels (based on the size $d$ of the conditioning sets).
At the first level ($d = 0$), all pairs of variables are tested conditioning on the empty set.
Some of the candidate causes would be removed and the algorithm only tests the remaining candidate causes in the next level ($d = 1$).
The size of the conditioning set, $d$, is progressively increased (by one) at each new level until $d$ is greater than $|\mathcal{X}^{Ca}|-1$.
Each corresponding relation path of $X \in \mathcal{X}^C$ construct a valid rule to predict the relation $R_h$ in $Y$.

\begin{figure}[t]
\vspace{0cm}
\centering
\includegraphics[width=8.5cm]{submissions/causal-meta-knowledge/figures/bidirection.jpg}

\caption{An example of bidirectional causal relationship, which may lead wrong results.}
\label{fig:bidirection}

\end{figure}

It is noteworthy that we add the reverse CI test in Algo.~\ref{alg:pc-like} (line 7) to avoid the impact of redundant relations in KGs.
For example, $Compound1 \stackrel{Resembles}{\longrightarrow}Compound2$ and $Compound1 \stackrel{Binds}{\longrightarrow}Gene1\stackrel{Binds}{\longleftarrow}Compound2$ express the similar message, which could lead the invalid independence test, as shown in Fig.~\ref{fig:bidirection}.
% \xz{As shown in Fig. xx}, the CI test results
% $Y=(C_h,C_t).M_q  \perp  X_1=(C_h,C_t).M_1| X_2=(C_h,C_t).M_2$  and $Y\perp  X_2|X_1$, where $C_h$ is Compound1, $C_t$ is Disease1, $M_q=Compound1 \stackrel{Treats}{\longrightarrow}Disease1$, $M_1= Compound1 \stackrel{Resembles}{\longrightarrow}Compound2 \stackrel{Treats}{\longrightarrow}Disease1$ and  $M_2= Compound1 \stackrel{Binds}{\longrightarrow}Gene1\stackrel{Binds}{\longleftarrow}Compound2 \stackrel{Treats}{\longrightarrow}Disease1$.
It will lead both $X_1$ and $X_2$ are removed from the candidate cause set of queried variable $Y$, even though they have very strong causal relationship with the drug treatment of diseases.
Consequently, we use the reverse CI test to avoid this issue.
In particular, if $X_j$ and $Y$ are judged to be independent conditioning on $\mathcal{Z}$, we will examine the independence between $\mathcal{Z}$ and $Y$ conditioning on $\mathcal{X}_j$.
When the result of the additional test is negative, $X_j$ will be removed from $\mathcal{X}^{Ca}$.
In this paper, we adopt SCI method~\cite{marx2019testing} as the independent test method in the experiments, which works well on limited samples and discrete variables.

\subsection{Link Prediction based on Explainable Causal Metaknowledge}
\label{sec:link_pred}

The approach for link prediction based on interpretable rules tends to generate corresponding weights in the rule mining phase. By accumulating the weights of the rules satisfied by each predicted entity, a score of the predicted entities can be generated, and then the results are ranked based on this score.
Here we first introduce how to generate rule weights under the causal model and then describe the approach for link prediction based on generated weights.

\begin{figure}[t]
\vspace{0cm}
\centering
\includegraphics[width=8.5cm]{submissions/causal-meta-knowledge/figures/direct_cause.jpg}
% \vspace{-3pt}
\caption{An example of non-independent rule-induced variables, which are both the causes of queried relation.}
\label{fig:direct_cause_example}
% \vspace{-5pt}
\end{figure}

\noindent
\textbf{Weights of rules based on conditional dependency.}
In Algo.~\ref{alg:pc-like}, we discover the direct causes by the non-independence relationship between the candidate meta structures and the queried meta structures.
It is important to note that the meta structures of $\mathcal{X}^C$ are not independent to each other.
Fig.~\ref{fig:direct_cause_example} gives an example for this case. Specifically, $X_1$ and $X_2$ are both causes of $Y$.
Since $X_1$ is also a cause of $X_2$, if we directly calculate the causal strength between $X_2$ and $Y$, it is inevitable that $w_2$ will contain the causal effects that arise from $X_1$ along the path $X_1\to X_2 \to Y$.
Therefore, in order to better measure the importance of each causal rule and to avoid double-counted in the calculation of each proposed entity's score, we adopt the minimal conditional dependence as a measure of the importance of causal rules:

\begin{equation}
\label{eq:weight}
\begin{aligned}
w_j  = \min(\{ dependence(X_j,Y|\mathcal{Z}) \})
\\
\text{~for any subset~} \mathcal{Z} \in \mathcal{X}^{Ca}\backslash \{X_j\},
\end{aligned}
\end{equation}
where $w_j$  is the rule weight of the meta structure in $X_j$.
In this paper, we use the $SCI_f(X,Y|Z)$ in SCI independence test~\cite{marx2019testing} as the dependence score in Eq.~\ref{eq:weight}, which can be get in the process of causal rules discovery.
The higher of $SCI_f(X,Y|Z)$, the stronger the dependency.

\noindent
\textbf{Score function of entity results.}
Because of the incompleteness nature of KGs, open world assumption (OWA)~\cite{ji2021survey} is often considered on real datasets.
Under the OWA, the SUM function are usually adopted to calculate the ranking score of the predicted entity $e_h$ in link prediction task $(?,R_h,e_t)$:
\begin{equation}
\label{eq:link-prediction-sum}
\begin{aligned}
S_{R_q}^{sum}=\sum_{i=1}^K \tilde{w}_i Q_{i},.
\end{aligned}
\end{equation}
where $K$ is the number of causal rules, $\tilde{w}_i$ is the normalized weight.
$Q_i=1$ when the body of the $i$-th causal rule holds for the entity pair ($e_h, e_t$), otherwise $Q_i=0$.
This approach focuses on the entities supported by multiple rules and does not use the non-existent relations between entity pairs, since the unreliable negative samples under OWA.
In this paper, Eq.~\ref{eq:link-prediction-sum} is used in the link predictions on real data.
For KG under closed world assumption(CWA)~\cite{ji2021survey}, the negative facts are also reliable, therefore we design a new function to apply the rules in the link prediction task.
Particularly, given an query $(?,R_h,e_t)$, the score of the triple $(e_h,R_h,e_t)$ is true can be formulated as:
\begin{equation}
\label{eq:link-prediction-avg}
\begin{aligned}
S_{R_q}^{avg} = \sum_i^K \tilde{w}_i \big(Q_i \bar{Y}_{X_i=1} + (1-Q_i) \bar{Y}_{X_i=0}\big),
\end{aligned}
\end{equation}
where $K$ is the number of causal rules for the queried relation, $\tilde{w}_i$ is the normalized weight for the $i$-th result rule.
$\bar{Y}_{X_i=1}$ denotes the proportion of the queried relation to be true when the body of the $i$-th causal rule is true in the training data, and $\bar{Y}_{X_i=0}$ denotes the proportion of the queried relation to be true when the body of the $i$-th causal rule is false.
$Q_i=1$ when the body of the $i$-th causal rule holds for the entity pair ($e_h$, $e_t$), otherwise $Q_i=0$.
The results will be ranked by $S_{R_q}$ of each valid $e_t$.
In this paper, Eq.~\ref{eq:link-prediction-avg} is used in the link predictions on simulation data.


% For the link prediction $<?,R,e_t>$, there are two ways to calculate the ranking score of the predicted entity $e_h$:
% \begin{itemize}
% %     \item MAX: this approach highlights the role of the highest-impact rule.
% % \begin{equation}
% % \label{eq:link-prediction-max}
% % \begin{aligned}
% % S_{R_q}^{max}=\max \left\{\tilde{w}_{1} Q_{1}, \tilde{w}_{2} Q_{2}, \ldots, \tilde{w}_{K} Q_{K}\right\}.
% % \end{aligned}

%  \item SUM: this approach focuses on the results supported by multiple rules.
% \begin{equation}
% \label{eq:link-prediction-sum}
% \begin{aligned}
% S_{R_q}^{sum}=\sum_{i=1}^K \tilde{w}_i Q_{i},.
% \end{aligned}
% \end{equation}
% where $K$ is the number of causal rules, $\tilde{w}_i$ is the normalized weight.
% $Q_i=1$ when the body of the $i$-th causal rule holds for the entity pair ($e_h, e_t$), otherwise $Q_i=0$.
% \item AVG: \xz{for a $e_t$, this approach considers both the prediction of a rule on the target relation when it is satisfied and unsatisfied.}

% \begin{equation}
% \label{eq:link-prediction-avg}
% \begin{aligned}
% S_{R_q}^{avg} = \sum_i^K \tilde{w}_i \big(Q_i \bar{Y}_{X_i=1} + (1-Q_i) \bar{Y}_{X_i=0}\big),
% \end{aligned}
% \end{equation}
% where $\bar{Y}_{X_i=1}$ denotes the proportion of the queried relation to be true when the body of the $i$-th causal rule is true in the training data, and $\bar{Y}_{X_i=0}$ denotes the proportion of the queried relation to be true when the body of the $i$-th causal rule is false.
% \end{itemize}
% These ranking functions show their unique advantages under different application scenarios, and we provide thorough discussions for them based on the experimental results. 


\section{EXPERIMENTAL STUDY}
\label{section:experiment}
\section{Experiments}
\label{experiment}

In this section, we provide experimental results of FedAQ in homogeneous local data distribution settings. We compare FedAQ with other quantization-based federated optimization algorithms, FedPAQ \cite{reisizadeh2020fedpaq} and FedCOMGATE \cite{haddadpour2021federated}. FedAvg \cite{mcmahan2017communication} and FedAC \cite{yuan2020federated}, federated optimization algorithms without quantization, are also our baselines. We empirically validate the performance of 5 algorithms on classical classification tasks on MNIST\cite{lecun1998mnist} and CIFAR-10\cite{krizhevsky2009learning} datasets in the distributed learning environment. We consider three objective functions i) A strongly convex objective of $l_2$-regularized logistic regression model on the MNIST dataset, ii) A non convex objective of training a multilayer perceptron on the MNIST data, and iii) A non convex objective of training a convolution neural network (CNN) on the CIFAR-10 dataset. %The details of the implementation environment, datasets, training models, hyperparameter choices, quantization bits, and new time metric are elaborated in Appx.~\ref{app:experimental_setup}.

% \begin{figure*}[!htbp]
%     \centering
%     % Figure 0
%     \begin{subfigure}[b]{0.31\textwidth}
%     \includegraphics[width=\textwidth]{figure/loss_iid_comm_str_cvx.png}
%     %\caption{DCGAN}
%     \end{subfigure}
%     % Figure 1
%     \begin{subfigure}[b]{0.31\textwidth}
%     \includegraphics[width=\textwidth]{figure/loss_iid_bits_str_cvx.png}
%     %\caption{DCGAN}
%     \end{subfigure}
%     %\quad
%     % Figure 2
%     \begin{subfigure}[b]{0.31\textwidth}
%     \includegraphics[width=\textwidth]{figure/loss_iid_time_str_cvx.png}
%     %\caption{OKGAN}
%     \end{subfigure}

%     \setcounter{subfigure}{0}
%     % Figure 0
%     \begin{subfigure}[b]{0.31\textwidth}
%     \includegraphics[width=\textwidth]{figure/loss_iid_comm_localstep_100_2.png}
%     %\caption{DCGAN}
%     \end{subfigure}
%     % Figure 1
%     \begin{subfigure}[b]{0.31\textwidth}
%     \includegraphics[width=\textwidth]{figure/loss_iid_bits_localstep_100_2.png}
%     %\caption{DCGAN}
%     \end{subfigure}
%     %\quad
%     % Figure 2
%     \begin{subfigure}[b]{0.31\textwidth}
%     \includegraphics[width=\textwidth]{figure/loss_iid_time_localstep_100_2.png}
%     %\caption{OKGAN}
%     \end{subfigure}
%     \caption{Comparing FedAQ with FedAvg, FedPAQ, FedCOMGATE, and FedAC on MNIST with Strongly Convex Settings (first row) and Non-Convex Settings (second row). We observe how the global training loss changes across communication rounds (first column), communicated bits (second column), and human time (third column). FedAQ-I(8bits) and FedAQ(4bits) respectively outperform other algorithms for strongly convex settings and non-convex settings. FedAQ(4bits) sends the same number of communicated bits as FedPAQ(8bits) and FedCOMGATE(8bits) in each communication round, which indicates a fair comparison (See Quantization bits in Appx.~\ref{app:experimental_setup}).}
%     \label{graph_in_main_body}
% \end{figure*}

\subsection{Experimental Setup}
\label{experimental_setup}

\paragraph{Implementation Environment.} We follow the implementation setup in \cite{haddadpour2021federated}. We use the Distributed library of PyTorch to implement our algorithm because this library allows us to simulate real-world communication and distributed training. The 18 cores of Intel Xeon E5-2676 CPU are used as computing sources. Each core is considered as one local client. We use 16 cores for strongly convex MNIST, 18 cores for the non-convex MNIST, and 8 cores for the CIFAR-10. For MNIST, the strongly convex experiment and the non-convex one respectively run for 300 rounds of communication with 20 local updates and 50 rounds of communication with 100 local updates. The CIFAR-10 experiment runs for 100 rounds of communication with 100 local updates.

\paragraph{Datasets.} For image classification tasks, we choose two main classical image datasets: MNIST and CIFAR-10. Since we assume homogeneous settings, data is distributed homogeneously among clients, which also means each device has access to all 10 classes.

% \paragraph{Training Models.} For MNIST, we use a $l_2$-regularized logistic regression model for the strongly convex case and a multilayer perceptron (MLP) with two hidden layers for the non-convex case. For CIFAR-10, we use a Convolutional Neural Network (CNN). Here, we note that the number of parameters in a neural network model is directly related to the number of communicated bits. We discuss more on this in Appx.~\ref{app:NN_comm_bits}.

\paragraph{Hyperparameter Choice.} The important hyperparmeters in our experiments are learning rates for each algorithm. For the client learning rate $\eta$, we respectively use 0.002, 0.1, and 0.01 for strongly convex MNIST, non-convex MNIST, and CIFAR-10 for all algorithms. For FedAQ and FedAC, once we set the value of $\mu$, other hyperparameters ($\gamma, \alpha, \beta$) are automatically determined (See condition set (\ref{parameter_FedAQ}) and (\ref{parameter2_FedAQ})). Thus, we choose 0.1, 0.01, and 0.2 for $\mu$ value for strongly convex MNIST, non-convex MNIST, and CIFAR-10. Since too large $\mu$ leads to slow convergence and too small $\mu$ leads to unstable training, we get these $\mu$ values by tuning $\mu$ appropriately. FedCOMGATE has a server learning rate, and we set this value as 1 for all experiments.

\paragraph{Quantization Bits.} We have three quantization-based federated algorithms: FedAQ, FedPAQ, FedCOMGATE. We quantize the updates from 32 bits to 8 bits for all quantization-based algorithms in both MNIST and CIFAR-10. Additionally, particularly for FedAQ in non-convex experiments, we consider 4 bits quantization as well. Since FedAQ sends twice as many messages as FedPAQ or FedCOMGATE at every synchronization when we use 8 bits quantization for all cases, we apply 4 bits quantization to FedAQ to let FedAQ send the same amount of information in each communication round as other quantization-based algorithms for a fair comparison.

\paragraph{New Time Metric.} In our experiments, communication between CPU cores is very fast, so it is hard to say that the environment of our experiments fully reflects the real-world federated learning when there is a heavy communication burden. Thus, we use a linear model to estimate the execution time $T_{\textrm{round}}(\mathcal{A})$ between two consecutive communication rounds for real federated learning scenarios \cite{wang2021field}.
\begin{align*}
    &T_{\textrm{round}}(\mathcal{A}) = T_{\textrm{comm}}(\mathcal{A})+T_{\textrm{comp}}(\mathcal{A}), & &T_{\textrm{comm}}(\mathcal{A}) = \frac{S_{\textrm{down}(\mathcal{A})}}{B_{\textrm{down}}} + \frac{S_{\textrm{up}(\mathcal{A})}}{B_{\textrm{up}}} \\
    &T_{\textrm{comp}}(\mathcal{A}) = \max_j T_{\textrm{client}}^j(\mathcal{A}) + T_{\textrm{server}}(\mathcal{A}), & &T_{\textrm{client}}^j(\mathcal{A}) = R_{\textrm{comp}}T_{\textrm{sim}}^j (\mathcal{A}) + C_{\textrm{comp}}
\end{align*}
Since $T_{\textrm{server}}(\mathcal{A})$ is relatively smaller than $T_{\textrm{client}}^j(\mathcal{A})$, we ignore $T_{\textrm{server}}(\mathcal{A})$ in our experiments. We get client download size $S_{\textrm{down}(\mathcal{A})}$ and upload size $S_{\textrm{up}(\mathcal{A})}$ from the number of neural network parameters. $\max_j T_{\textrm{sim}}^j(\mathcal{A})$ is the computation time in our simulation.
\begin{align*}
    B_{\textrm{down}} \sim 0.75 \textrm{MB/secs},\textrm{ } B_{\textrm{up}} \sim 0.25 \textrm{B/secs},\textrm{ } R_{\textrm{comp}} \sim 7,\textrm{ } C_{\textrm{comp}} \sim 10 \textrm{secs}
\end{align*}
\cite{wang2021field} estimate each value of the above parameters from a real world cross-device FL system. The upload bandwidth $B_{\textrm{up}}$ is generally smaller than download bandwidth $B_{\textrm{down}}$. We define human time as the parallel time estimated by this new time metric.

\subsubsection{Training Models}

For MNIST, we use a $l_2$-regularized logistic regression model for the strongly convex case and a multilayer perceptron (MLP) with two hidden layers for the non-convex case. For CIFAR-10, we use a Convolutional Neural Network (CNN). Here, we note that the number of parameters in a neural network model is directly related to the number of communicated bits. We discuss more details as follows.

\paragraph{MLP Model for MNIST.} We use a multilayer perceptron (MLP) with two hidden layers. Each hidden layer consists of 200 neurons with ReLU activations. Thus, we compute the total number of parameters in this MLP model as below.
\begin{align*}
    (\# \textrm{ of MLP parameters) } &= (\# \textrm{ of input features) } \times (\# \textrm{ of neurons in the 1st layer}) \\
    &+ (\# \textrm{ of neurons in the 1st layer) } \times (\# \textrm{ of neurons in the 2nd layer}) \\
    &+ (\# \textrm{ of neurons in the 2nd layer) } \times (\# \textrm{ of MNIST classes}) \\
    &+ (\# \textrm{ of neurons in the 1st layer) } + (\# \textrm{ of neurons in the 2nd layer) } \\
    &+ (\# \textrm{ of MNIST classes}) \\
    &= 28 \times 28 \times 200 + 200 \times 200 + 200 \times 10 + 200 + 200 + 10 = 199210
\end{align*}
Finally, we derive $S_\textrm{up}(\mathcal{A}) (= S_\textrm{down}(\mathcal{A})$), defined in \cref{experimental_setup} (New time metric), by using the above fact. We use 32 bits floating-point if there is no quantization.
\begin{align*}
    S_\textrm{up}(\mathcal{A}) &= (\# \textrm{ of device) } \times (\# \textrm{ of MLP parameters) } \times (\# \textrm{ of bits)} \\
    &= 18 \times 199210 \times 32 = 114744960
\end{align*}
The FedAvg algorithm follows the above calculation. If we use 8 bits quantization for FedPAQ, FedCOMGATE, and FedAQ, ($\#$ of bits) in the above equation will respectively be  8, 8, and 16. Since FedAQ sends twice as many messages as others at every communication round, ($\#$ of bits) for FedAQ is 16. Similarly, ($\#$ of bits) for FedAC, which has no quantization, is 64.

\paragraph{CNN Model for CIFAR-10.} We use a CNN model, which consists of two 2-dimensional convolutional layers, two max pooling layers, and two fully connected layers. The ReLU activations are used in this CNN model. Let's clarify ($\#$ of input channel, $\#$ of output channel, kernel size, stride) for convolutional layers. We respectively use (3, 20, 5, 1), (20, 50, 5, 1) for the 1st and 2nd convolutional layer. Let's denote each convolutional layer and fully connected layer as CONV1, CONV2, FC3, FC4. At first, the activation shape of input layer for CIFAR-10 is (32, 32, 3). Then, we get the activation shape after CONV1 and the number of parameters for CONV1.
\begin{align*}
    (\textrm{width of activation shape) } &= \frac{\textrm{(width of previous activation shape) } - \textrm{kernel size} + 1}{\textrm{stride}} \\
    &= \frac{32-5+1}{1} = 28 \textrm{ } \Rightarrow \textrm{ activation shape} = (28, 28, 20) \\
    (\# \textrm{ of CONV1 parameters) } &= \Big(\textrm{kernel size } \times \textrm{ kernel size } \\
    &\times (\# \textrm{ of filters in the previous layer) }+1 \Big) \\
    &\times (\# \textrm{ of filters in the current layer}) \\
    &= (5 \times 5 \times 3 + 1) \times 20 = 1520
\end{align*}
The activation shape becomes (14, 14, 20) after max pooling. There are no learnable parameters in pooling layers. We do similar calculation for CONV2.
\begin{align*}
    (\textrm{width of activation shape) } &= \frac{\textrm{(width of previous activation shape) } - \textrm{kernel size} + 1}{\textrm{stride}} \\
    &= \frac{14-5+1}{1} = 10 \textrm{ } \Rightarrow \textrm{ activation shape} = (10, 10, 50) \\
    (\# \textrm{ of CONV2 parameters) } &= \Big(\textrm{kernel size } \times \textrm{ kernel size } \times (\# \textrm{ of filters in the previous layer) }\\
    &+1\Big) \times (\# \textrm{ of filters in the current layer}) \\
    &= (5 \times 5 \times 20 + 1) \times 50 = 25050
\end{align*}
The activation shape becomes (5, 5, 50) after second max pooling. Then, we calculate the number of parameters in FC3 and FC4 similar to the MLP case.
\begin{align*}
    (\# \textrm{ of FC3 parameters }) &= (5 \times 5 \times 50) \times 512 + 512 = 640512 \\
    (\# \textrm{ of FC4 parameters }) &= 512 \times 10 + 10 = 5130
\end{align*}
Thus, the total number of parameters in this CNN model is
\begin{align*}
    (\# \textrm{ of CNN parameters) } &= (\# \textrm{ of CONV1 parameters) } + (\# \textrm{ of CONV2 parameters) } \\
    &+ (\# \textrm{ of FC3 parameters) } + (\# \textrm{ of FC4 parameters) } \\
    &= 1520 + 25050 + 640512 + 5130 = 672212
\end{align*}
Finally, we derive $S_\textrm{up}(\mathcal{A}) (= S_\textrm{down}(\mathcal{A})$) in this case.
\begin{align*}
    S_\textrm{up}(\mathcal{A}) &= (\# \textrm{ of device) } \times (\# \textrm{ of CNN parameters) } \times (\# \textrm{ of bits)} \\
    &= 8 \times 672212 \times 32 = 172086272
\end{align*}
We can do the similar discussion in the MLP case when it comes to applying this to quantization-based federated optimization algorithms.

\subsection{Experimental Results}
\label{experimental_results}

In our experiments on both MNIST and CIFAR-10, we verify how the global training loss and test accuracy of five algorithms change with respect to communication rounds, the number of bits communicated between one client and the server during the uplink, and human time. We provide both qualitative analysis and quantitative results for plots.

\subsubsection{Qualitative Analysis}
\label{qualitative_analysis}

\paragraph{Strongly Convex Case.} In this experiment, we compare FedAQ under the condition set (\ref{parameter_FedAQ}) and set (\ref{parameter2_FedAQ}) with FedAvg, FedPAQ, FedCOMGATE, and FedAC-I. We denote each FedAQ as FedAQ-I and FedAQ-II. As we observe the theoretical benefits of FedAQ over other methods in \cref{convergence_analysis}, FedAQ-I outperforms all other quantization-based federated optimization algorithms and FedAC-I in all plots (See each first row of Figure \ref{graph_in_main_body}, \ref{mnist_graph}). However, although FedAQ-II shows the fast convergence speed, the training process is unstable. Thus, we only use FedAQ-I for further non-convex experiments. FedAC and FedAQ in non-convex experiments indicate FedAC-I and FedAQ-I.

\paragraph{Non-Convex Case.} Each second row of Figure \ref{graph_in_main_body}, \ref{mnist_graph}, and Figure \ref{cifar10_graph} clearly demonstrates that FedAQ with 4 bits quantization outperforms other algorithms in all plots. In terms of communication rounds, accelerated algorithms, FedAQ and FedAC, converge faster than other algorithms. We also observe that quantization does not lead to slower convergence, which means we can apply an efficient quantization scheme to make communication efficient FL systems without sacrificing convergence speed. The plots related to communicated bits are helpful to interpret how algorithms work well in situations with heavy communication. FedAQ with 8 bits quantization shows comparable performance relative to FedPAQ and FedCOMGATE with the help of acceleration, even though FedAQ sends more updates during every synchronization. When we use 4 bits quantization for FedAQ to make the number of communicated bits the same for all quantization-based algorithms during synchronization, FedAQ shows a much faster convergence speed with regard to the number of communicated bits. However, plots of communicated bits fail to reflect how algorithms converge in real estimated time for FL scenarios, which consists of both communication and computation. Thus, we further analyze algorithms with human time. We observe that FedAQ with 8 quantization bits performs slightly better than FedPAQ and FedCOMGATE for both MNIST and CIFAR-10. This occurs because while all quantization-based algorithms send the same number of communicated bits, the number of communication rounds for FedAQ is much smaller than others. Then, this also indicates that FedAQ takes less computation time than other methods while reaching the same accuracy.

\subsubsection{Quantitative Results}
\label{app:quantitative_graphs}

We provide quantitative results to help readers understand plots better. To be specific, for all plots, we observe the number of communication rounds, the number of communicated bits, and the human time required to achieve a particular test accuracy by each federated optimization algorithm.

For the strongly convex experiment on MNIST (See the first row of Figure \ref{mnist_graph}), the number of communication rounds required to achieve 90.28\% test accuracy by FedAvg, FedPAQ(8bits), FedCOMGATE(8bits), FedAC-I, FedAQ-I(8bits), FedAQ-II(8bits) are respectively 217, 216, 260, 28, 26, 99. The number of communicated bits required to achieve the same accuracy are respectively 5.4e7, 1.4e7, 1.6e7, 1.4e7, 3.3e6, 1.2e7. Lastly, the required human time are respectively 3220s, 2760s, 3336s, 484s, 344s, 1323s. In this experiment, FedAQ-I(8bits) requires the smallest number of communication rounds, the smallest number of communicated bits, and the shortest human time to achieve the same test accuracy. These experimental results support the validity of our theoretical analysis on strongly convex cases.

For the non-convex experiment on MNIST (See the second row of Figure \ref{mnist_graph}), the number of communication rounds required to achieve 97.6\% test accuracy by FedAvg, FedPAQ(8bits), FedCOMGATE(8bits), FedAC, FedAQ(8bits), FedAQ(4bits) are respectively 23, 48, 38, 18, 18, 16. The number of communicated bits required to achieve the same accuracy are respectively 1.5e8, 7.6e7, 6.1e7, 2.3e8, 5.7e7, 2.5e7. Finally, the required human time are respectively 2424s, 2311s, 1834s, 3327s, 1248s, 805s. Thus, we conclude that FedAQ(4bits) outperforms other algorithms, and even FedAQ(8bits) needs smaller number of communicated bits/less human time to achieve the goal accuracy than FedPAQ(8bits)/FedCOMGATE(8bits).

For the non-convex experiment on CIFAR-10 (See Figure \ref{cifar10_graph}), the number of communication rounds required to achieve 65.4\% test accuracy by FedAvg, FedPAQ(8bits), FedCOMGATE(8bits), FedAC, FedAQ(8bits), FedAQ(4bits) are respectively 98, 89, 95, 49, 50, 48. The number of communicated bits required to achieve the same accuracy are respectively 2.1e9, 4.8e8, 5.1e8, 2.1e9, 5.4e8, 2.6e8. Finally, the required human time are respectively 31798s, 11526s, 12240s, 28720s, 9902s, 6464s. As with the non-convex experiment on MNIST, FedAQ(4bits) outperforms other algorithms, and even FedAQ(8bits) requires less human time to achieve the same accuracy than FedPAQ(8bits)/FedCOMGATE(8bits).

\begin{remark}
Our current experimental setup only allows us to scale the number of clients up to the number of CPU cores in our machine. Since FedAQ achieves linear speed up in the number of workers with much fewer communication rounds than other quantization based methods, we expect FedAQ to outperform other methods by an even larger margin as we scale the number of workers.
\end{remark}

\begin{figure*}[!htbp]
    \centering
    % Figure 0
    \begin{subfigure}[]{
    \includegraphics[width=0.31\textwidth]{submissions/YeojoonYoun/figure/loss_iid_comm_str_cvx.png}
    %\caption{DCGAN}
    }
    \end{subfigure}
    % Figure 1
    \begin{subfigure}[]{
    \includegraphics[width=0.31\textwidth]{submissions/YeojoonYoun/figure/loss_iid_bits_str_cvx.png}
    %\caption{DCGAN}
    }
    \end{subfigure}
    %\quad
    % Figure 2
    \begin{subfigure}[]{
    \includegraphics[width=0.31\textwidth]{submissions/YeojoonYoun/figure/loss_iid_time_str_cvx.png}
    %\caption{OKGAN}
    }
    \end{subfigure}

    \setcounter{subfigure}{0}
    % Figure 0
    \begin{subfigure}[]{
    \includegraphics[width=0.31\textwidth]{submissions/YeojoonYoun/figure/loss_iid_comm_localstep_100_2.png}
    %\caption{DCGAN}
    }
    \end{subfigure}
    % Figure 1
    \begin{subfigure}[]{
    \includegraphics[width=0.31\textwidth]{submissions/YeojoonYoun/figure/loss_iid_bits_localstep_100_2.png}
    %\caption{DCGAN}
    }
    \end{subfigure}
    %\quad
    % Figure 2
    \begin{subfigure}[]{
    \includegraphics[width=0.31\textwidth]{submissions/YeojoonYoun/figure/loss_iid_time_localstep_100_2.png}
    %\caption{OKGAN}
    }
    \end{subfigure}
    \caption{Comparing FedAQ with FedAvg, FedPAQ, FedCOMGATE, and FedAC on MNIST with Strongly Convex Settings (first row) and Non-Convex Settings (second row). We observe how the global training loss changes across communication rounds (first column), communicated bits (second column), and human time (third column). FedAQ-I(8bits) and FedAQ(4bits) respectively outperform other algorithms for strongly convex settings and non-convex settings. FedAQ(4bits) sends the same number of communicated bits as FedPAQ(8bits) and FedCOMGATE(8bits) in each communication round, which indicates a fair comparison (See Quantization bits in \cref{experimental_setup}).}
    \label{graph_in_main_body}
\end{figure*}

\begin{figure*}[hbt!]%[!htbp]
    \centering
    % Figure 0
    \begin{subfigure}[]{
    \includegraphics[width=0.31\textwidth]{submissions/YeojoonYoun/figure/accuracy_iid_comm_str_cvx.png}
    %\caption{DCGAN}
    }
    \end{subfigure}
    % Figure 1
    \begin{subfigure}[]{
    \includegraphics[width=0.31\textwidth]{submissions/YeojoonYoun/figure/accuracy_iid_bits_str_cvx.png}
    %\caption{DCGAN}
    }
    \end{subfigure}
    %\quad
    % Figure 2
    \begin{subfigure}[]{
    \includegraphics[width=0.31\textwidth]{submissions/YeojoonYoun/figure/accuracy_iid_time_str_cvx.png}
    %\caption{OKGAN}
    }
    \end{subfigure}

    \setcounter{subfigure}{0}
    % Figure 0
    \begin{subfigure}[]{
    \includegraphics[width=0.31\textwidth]{submissions/YeojoonYoun/figure/accuracy_iid_comm_localstep_100_2.png}
    %\caption{DCGAN}
    }
    \end{subfigure}
    % Figure 1
    \begin{subfigure}[]{
    \includegraphics[width=0.31\textwidth]{submissions/YeojoonYoun/figure/accuracy_iid_bits_localstep_100_2.png}
    %\caption{DCGAN}
    }
    \end{subfigure}
    %\quad
    % Figure 2
    \begin{subfigure}[]{
    \includegraphics[width=0.31\textwidth]{submissions/YeojoonYoun/figure/accuracy_iid_time_localstep_100_2.png}
    %\caption{OKGAN}
    }
    \end{subfigure}
    \caption{Comparing FedAQ with FedAvg, FedPAQ, FedCOMGATE, and FedAC on MNIST with Strongly Convex Settings (first row) and Non-Convex Settings (second row). We observe how the test accuracy changes across communication rounds (first column), communicated bits (second column), and human time (third column). FedAQ-I outperforms other algorithms in all plots for strongly convex settings. Moreover, FedAQ(4bits) outperforms other algorithms in all plots for non-convex settings.}
    \label{mnist_graph}
\end{figure*}
%\FloatBarrier

\begin{figure*}[hbt!]%[!htbp]
    \centering
    % Figure 0
    \begin{subfigure}[]{
    \includegraphics[width=0.31\textwidth]{submissions/YeojoonYoun/figure/loss_iid_comm_cnn_step100.png}
    %\caption{DCGAN}
    }
    \end{subfigure}
    % Figure 1
    \begin{subfigure}[]{
    \includegraphics[width=0.31\textwidth]{submissions/YeojoonYoun/figure/loss_iid_bits_cnn_step100.png}
    %\caption{DCGAN}
    }
    \end{subfigure}
    %\quad
    % Figure 2
    \begin{subfigure}[]{
    \includegraphics[width=0.31\textwidth]{submissions/YeojoonYoun/figure/loss_iid_time_cnn_step100.png}
    %\caption{OKGAN}
    }
    \end{subfigure}

    \setcounter{subfigure}{0}
    % Figure 0
    \begin{subfigure}[]{
    \includegraphics[width=0.31\textwidth]{submissions/YeojoonYoun/figure/accuracy_iid_comm_cnn_step100.png}
    %\caption{DCGAN}
    }
    \end{subfigure}
    % Figure 1
    \begin{subfigure}[]{
    \includegraphics[width=0.31\textwidth]{submissions/YeojoonYoun/figure/accuracy_iid_bits_cnn_step100.png}
    %\caption{DCGAN}
    }
    \end{subfigure}
    %\quad
    % Figure 2
    \begin{subfigure}[]{
    \includegraphics[width=0.31\textwidth]{submissions/YeojoonYoun/figure/accuracy_iid_time_cnn_step100.png}
    %\caption{OKGAN}
    }
    \end{subfigure}
    \caption{Comparing FedAQ with FedAvg, FedPAQ, FedCOMGATE, and FedAC on CIFAR-10. We observe how the global training loss and test accuracy change across communication rounds (first column), communicated bits (second column), and human time (third column). We use a CNN model for CIFAR-10. Similar to the MNIST experiment, FedAQ (4 bits) outperforms all other algorithms in every case.}
    \label{cifar10_graph}
\end{figure*}



\section{Related Work}
\label{section:related}
\section{Related Work}
\label{sec:related-work}
\noindent \textbf{Pseudo-relevance feedback:} Our method has similarities with %the existing approach of 
Pseudo-Relevance Feedback (PRF) \cite{rocchio1971relevance, lv2009adaptive, li2022does} in IR: \cite{bendersky2011parameterized, xu2017quary} use the retrieved documents to improve sparse approaches via query expansion or query term reweighting, \cite{li2018nprf, zheng2020bert} score similarity between a target document and a top-ranked feedback document, while \cite{yu2021improving} train a separate query encoder that computes a new query embedding using the retrieved documents as additional input. In contrast, our approach does not require customized training feedback models or availability of explicit feedback data, as we improve the query vector by directly distilling from the reranker's output within an R\&R framework. %\pradeep{Why is our approach better?} 

Further, previous approaches to PRF have been dependent on the choice of retriever architecture and language; \cite{yu2021improving}'s PRF model is tied to the retriever used, \cite{chandradevan2022learning} explore cross-lingual relevance feedback, but require feedback documents in target language and thereby could only apply to three languages, while \cite{li2022interpolate} explore interpolating relevance feedback between dense and sparse approaches.
On the other hand, our approach is independent of the choice of the retriever and reranker architecture, and can be used for neural retrieval in any domain, language or modality. \\

\noindent \textbf{Distillation in Neural IR:} Existing approaches primarily leverage reranker feedback \textit{during training} of the dual-encoder retriever, to sample better negatives \cite{qu2021rocketqa}, for standard knowledge distillation of the cross-attention scores \cite{izacard2020distilling}, to train smaller and more efficient rankers by distilling larger models \cite{hofstatter2020improving}, or to align the geometry of dual-encoder embeddings with that from cross-encoders \cite{wang2021enhancing}. Instead, we leverage distillation at inference time, updating only the query representation to replicate the cross-encoder’s scores for the corresponding test instance.
A key implication of this design choice is that unlike existing methods, we keep the retriever parameters unchanged, meaning \textsc{ReFIT} can be incorporated out-of-the-box into any neural R\&R framework. In contrast, extending training-time distillation to new languages or modalities would require re-training the bi-encoder.

More recently, \textsc{TouR}~\cite{sung2023optimizing} has proposed test-time optimization of query representations with two variants: \textsc{TouR}$_{\text{hard}}$ and  \textsc{TouR}$_{\text{soft}}$. 
\textsc{TouR}$_{\text{hard}}$ optimizes the marginal likelihood of a small set of (pseudo) positive contexts.
\textsc{ReFIT} shares similarities with \textsc{TouR}$_{\text{soft}}$, which uses the normalized scores of a cross-encoder over the retrieved results as soft labels.
Crucially, \textsc{TouR} relies on multiple iterations of relevance feedback via distillation, where each iteration runs until the top-1 retrieval result has the highest reranker score (in \textsc{TouR}$_{\text{soft}}$) or is a pseudo-positive (in \textsc{TouR}$_{\text{hard}}$).
This makes inference highly computationally expensive, as each additional iteration involves labeling top-$K$ retrieval results with a reranker and then retrieving again.
\textsc{ReFIT} improves efficiency over \textsc{TouR} by requiring only a single iteration of feedback that simply updates the query vector for longer, foregoing additional retrieval and reranking steps. More specifics on the inference process of the two methods can be found in \S{\ref{sec:tour_comparison}}.
\textsc{TouR} was evaluated only on English phrase and passage retrieval tasks, while we demonstrate \textsc{ReFIT}'s effectiveness in multidomain, multilingual and multimodal settings, with an empirical comparison with \textsc{TouR} in \S{\ref{sec:tour_comparison}}.

\section{Conclusion and Future work}
\label{section:conclusion}
\vspace{-1em}
\section{Conclusion}\label{sec:conclusion}

We have designed and implemented \Hammer and \name to enhance the system efficiency of vector search. We have followed the state-of-the-art vector search algorithm, but at every level, we have introduced innovations that stretch prior methods and tailor our system for the newer hardware setting of multi-core processors and heterogeneous memory. 
Our innovations, such as intra-query path-wise parallelism, staged expansion, redundancy-aware synchronization, and HM-aware index construction, improve the computational and memory efficiency of large-scale vector search. Many of these methods should work well in other settings, such as SSD-based methods and compression-based indices. Finally, we propose several open research directions from a system perspective that have the potential to further improve large-scale vector search. We hope these directions will inspire new research that can advance vector search and make it more efficient, robust, and easy-to-use.


\bibliographystyle{ACM-Reference-Format}
\setcitestyle{numbers,sort&compress}
\bibliography{sample}

\end{document}
\endinput


% %%
% %% The abstract is a short summary of the work to be presented in the
% %% article.
% \begin{abstract}
% Praesent imperdiet, lacus nec varius placerat, est ex eleifend justo, a vulputate leo massa consectetur nunc. Donec posuere in mi ut tempus. Pellentesque sem odio, faucibus non mi in, laoreet maximus arcu. In hac habitasse platea dictumst. Nunc euismod neque eu urna accumsan, vitae vehicula metus tincidunt. Maecenas congue tortor nec varius pellentesque. Pellentesque bibendum libero ac dignissim euismod. Aliquam justo ante, pretium vel mollis sed, consectetur accumsan nibh. Nulla sit amet sollicitudin est. Etiam ullamcorper diam a sapien lacinia faucibus.
% \end{abstract}

% \maketitle

% %%% do not modify the following VLDB block %%
% %%% VLDB block start %%%
% \pagestyle{\vldbpagestyle}
% \begingroup\small\noindent\raggedright\textbf{PVLDB Reference Format:}\\
% \vldbauthors. \vldbtitle. PVLDB, \vldbvolume(\vldbissue): \vldbpages, \vldbyear.\\
% \href{https://doi.org/\vldbdoi}{doi:\vldbdoi}
% \endgroup
% \begingroup
% \renewcommand\thefootnote{}\footnote{\noindent
% This work is licensed under the Creative Commons BY-NC-ND 4.0 International License. Visit \url{https://creativecommons.org/licenses/by-nc-nd/4.0/} to view a copy of this license. For any use beyond those covered by this license, obtain permission by emailing \href{mailto:info@vldb.org}{info@vldb.org}. Copyright is held by the owner/author(s). Publication rights licensed to the VLDB Endowment. \\
% \raggedright Proceedings of the VLDB Endowment, Vol. \vldbvolume, No. \vldbissue\ %
% ISSN 2150-8097. \\
% \href{https://doi.org/\vldbdoi}{doi:\vldbdoi} \\
% }\addtocounter{footnote}{-1}\endgroup
% %%% VLDB block end %%%

% %%% do not modify the following VLDB block %%
% %%% VLDB block start %%%
% \ifdefempty{\vldbavailabilityurl}{}{
% \vspace{.3cm}
% \begingroup\small\noindent\raggedright\textbf{PVLDB Artifact Availability:}\\
% The source code, data, and/or other artifacts have been made available at \url{\vldbavailabilityurl}.
% \endgroup
% }
% %%% VLDB block end %%%

% \section{Introduction}

% Lorem ipsum dolor sit amet, consectetur adipiscing elit. Suspendisse a arcu quis arcu malesuada ultricies vitae in felis. Curabitur porta lacus at felis viverra hendrerit in non eros. Nam tempus tincidunt metus vitae fermentum. Donec sed risus felis. Cras luctus massa elementum, semper urna vel, efficitur ipsum. Morbi at tellus libero.

% Praesent imperdiet, lacus nec varius placerat, est ex eleifend justo, a vulputate leo massa consectetur nunc. Donec posuere in mi ut tempus. Pellentesque sem odio, faucibus non mi in, laoreet maximus arcu. In hac habitasse platea dictumst. Nunc euismod neque eu urna accumsan, vitae vehicula metus tincidunt. Maecenas congue tortor nec varius pellentesque. Pellentesque bibendum libero ac dignissim euismod. Aliquam justo ante, pretium vel mollis sed, consectetur accumsan nibh. Nulla sit amet sollicitudin est. 

% \section{Core Structural Elements}

% Nulla placerat feugiat augue, id blandit urna pretium nec. Nulla velit sem, tempor vel mauris ut, porta commodo quam. Donec lectus erat, sodales eu mauris eu, fringilla vestibulum nisl. Morbi viverra tellus id lorem faucibus cursus. Quisque et orci in est faucibus semper vel a turpis. Vivamus posuere sed ligula et. 

% \subsection{Figures}

% Aliquam justo ante, pretium vel mollis sed, consectetur accumsan nibh. Nulla sit amet sollicitudin est. Etiam ullamcorper diam a sapien lacinia faucibus. Duis vulputate, nisl nec tincidunt volutpat, erat orci eleifend diam, eget semper risus est eget nisl. Donec non odio id neque pharetra ultrices sit amet id purus. Nulla non dictum tellus, id ullamcorper libero. Curabitur vitae nulla dapibus, ornare dolor in, efficitur enim. Cras fermentum facilisis elit vitae egestas. Nam vulputate est non tellus efficitur pharetra. Vestibulum ligula est, varius in suscipit vel, porttitor id massa. Nulla placerat feugiat augue, id blandit urna pretium nec. Nulla velit sem, tempor vel mauris ut, porta commodo quam \autoref{fig:duck}.

% \begin{figure}
%   \centering
%   \includegraphics[width=\linewidth]{figures/duck}
%   \caption{An illustration of a Mallard Duck. Picture from Mabel Osgood Wright, \textit{Birdcraft}, published 1897.}
%   \label{fig:duck}
% \end{figure}

% \begin{table*}[t]
%   \caption{A double column table.}
%   \label{tab:commands}
%   \begin{tabular}{ccl}
%     \toprule
%     A Wide Command Column & A Random Number & Comments\\
%     \midrule
%     \verb|\tabular| & 100& The content of a table \\
%     \verb|\table|  & 300 & For floating tables within a single column\\
%     \verb|\table*| & 400 & For wider floating tables that span two columns\\
%     \bottomrule
%   \end{tabular}
% \end{table*}

% \subsection{Tables}

% Curabitur vitae nulla dapibus, ornare dolor in, efficitur enim. Cras fermentum facilisis elit vitae egestas. Mauris porta, neque non rutrum efficitur, odio odio faucibus tortor, vitae imperdiet metus quam vitae eros. Proin porta dictum accumsan \autoref{tab:commands}.

% Duis cursus maximus facilisis. Integer euismod, purus et condimentum suscipit, augue turpis euismod libero, ac porttitor tellus neque eu enim. Nam vulputate est non tellus efficitur pharetra. Aenean molestie tristique venenatis. Nam congue pulvinar vehicula. Duis lacinia mollis purus, ac aliquet arcu dignissim ac \autoref{tab:freq}. 

% \begin{table}[hb]% h asks to places the floating element [h]ere.
%   \caption{Frequency of Special Characters}
%   \label{tab:freq}
%   \begin{tabular}{ccl}
%     \toprule
%     Non-English or Math & Frequency & Comments\\
%     \midrule
%     \O & 1 in 1000& For Swedish names\\
%     $\pi$ & 1 in 5 & Common in math\\
%     \$ & 4 in 5 & Used in business\\
%     $\Psi^2_1$ & 1 in 40\,000 & Unexplained usage\\
%   \bottomrule
% \end{tabular}
% \end{table}

% Nulla sit amet enim tortor. Ut non felis lectus. Aenean quis felis faucibus, efficitur magna vitae. Curabitur ut mauris vel augue tempor suscipit eget eget lacus. Sed pulvinar lobortis dictum. Aliquam dapibus a velit.

% \subsection{Listings and Styles}

% Aenean malesuada fringilla felis, vel hendrerit enim feugiat et. Proin dictum ante nec tortor bibendum viverra. Curabitur non nibh ut mauris egestas ultrices consequat non odio.

% \begin{itemize}
% \item Duis lacinia mollis purus, ac aliquet arcu dignissim ac. Vivamus accumsan sollicitudin dui, sed porta sem consequat.
% \item Curabitur ut mauris vel augue tempor suscipit eget eget lacus. Sed pulvinar lobortis dictum. Aliquam dapibus a velit.
% \item Curabitur vitae nulla dapibus, ornare dolor in, efficitur enim.
% \end{itemize}

% Ut sagittis, massa nec rhoncus dignissim, urna ipsum vestibulum odio, ac dapibus massa lorem a dui. Nulla sit amet enim tortor. Ut non felis lectus. Aenean quis felis faucibus, efficitur magna vitae. 

% \begin{enumerate}
% \item Duis lacinia mollis purus, ac aliquet arcu dignissim ac. Vivamus accumsan sollicitudin dui, sed porta sem consequat.
% \item Curabitur ut mauris vel augue tempor suscipit eget eget lacus. Sed pulvinar lobortis dictum. Aliquam dapibus a velit.
% \item Curabitur vitae nulla dapibus, ornare dolor in, efficitur enim.
% \end{enumerate}

% Cras fermentum facilisis elit vitae egestas. Mauris porta, neque non rutrum efficitur, odio odio faucibus tortor, vitae imperdiet metus quam vitae eros. Proin porta dictum accumsan. Aliquam dapibus a velit. Curabitur vitae nulla dapibus, ornare dolor in, efficitur enim. Ut maximus mi id arcu ultricies feugiat. Phasellus facilisis purus ac ipsum varius bibendum.

% \subsection{Math and Equations}

% Curabitur vitae nulla dapibus, ornare dolor in, efficitur enim. Cras fermentum facilisis elit vitae egestas. Nam vulputate est non tellus efficitur pharetra. Vestibulum ligula est, varius in suscipit vel, porttitor id massa. Cras facilisis suscipit orci, ac tincidunt erat.
% \begin{equation}
%   \lim_{n\rightarrow \infty}x=0
% \end{equation}

% Sed pulvinar lobortis dictum. Aliquam dapibus a velit porttitor ultrices. Ut maximus mi id arcu ultricies feugiat. Phasellus facilisis purus ac ipsum varius bibendum. Aenean a quam at massa efficitur tincidunt facilisis sit amet felis. 
% \begin{displaymath}
%   \sum_{i=0}^{\infty} x + 1
% \end{displaymath}

% Suspendisse molestie ultricies tincidunt. Praesent metus ex, tempus quis gravida nec, consequat id arcu. Donec maximus fermentum nulla quis maximus.
% \begin{equation}
%   \sum_{i=0}^{\infty}x_i=\int_{0}^{\pi+2} f
% \end{equation}

% Curabitur vitae nulla dapibus, ornare dolor in, efficitur enim. Cras fermentum facilisis elit vitae egestas. Nam vulputate est non tellus efficitur pharetra. Vestibulum ligula est, varius in suscipit vel, porttitor id massa. Cras facilisis suscipit orci, ac tincidunt erat.

% \section{Citations}

% Some examples of references. A paginated journal article~\cite{Abril07}, an enumerated journal article~\cite{Cohen07}, a reference to an entire issue~\cite{JCohen96}, a monograph (whole book) ~\cite{Kosiur01}, a monograph/whole book in a series (see 2a in spec. document)~\cite{Harel79}, a divisible-book such as an anthology or compilation~\cite{Editor00} followed by the same example, however we only output the series if the volume number is given~\cite{Editor00a} (so Editor00a's series should NOT be present since it has no vol. no.), a chapter in a divisible book~\cite{Spector90}, a chapter in a divisible book in a series~\cite{Douglass98}, a multi-volume work as book~\cite{Knuth97}, an article in a proceedings (of a conference, symposium, workshop for example) (paginated proceedings article)~\cite{Andler79}, a proceedings article with all possible elements~\cite{Smith10}, an example of an enumerated proceedings article~\cite{VanGundy07}, an informally published work~\cite{Harel78}, a doctoral dissertation~\cite{Clarkson85}, a master's thesis~\cite{anisi03}, an finally two online documents or world wide web resources~\cite{Thornburg01, Ablamowicz07}.

% \begin{acks}
%  This work was supported by the [...] Research Fund of [...] (Number [...]). Additional funding was provided by [...] and [...]. We also thank [...] for contributing [...].
% \end{acks}

%\clearpage


