

Knowledge graphs are ubiquitous data structure which have been used in many applications. 
Knowledge graph reasoning aims to discover or infer knowledge based on existing information in the knowledge graph.
However, most of the existing works belong to {\em point-wise} approaches, which perform reasoning w.r.t. a single piece of clue.
Comparative reasoning over knowledge graph focuses on inferring commonality and inconsistency with respect to multiple pieces of clues which is a new research direction and can be applied to many applications. 
In this paper, we formally give the definition of comparative reasoning and propose several different methods to tackle comparative reasoning in both pairwise and collective cases. 
The idea of the proposed methods is that we find a knowledge segment from the knowledge graph to best represent the semantic meaning of the given claim, and reasons according to it.
We perform extensive empirical evaluations on real-world datasets to demonstrate that the proposed methods have good performances.



