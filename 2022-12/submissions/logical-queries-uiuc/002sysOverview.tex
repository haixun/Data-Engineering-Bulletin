


\hide{
\begin{table}[]
	\centering
	\footnotesize
	\caption{Notations and definitions}
	\begin{tabular}{|c|c|}
		\hline
		{\bf Symbols}       & {\bf Definition}                \\ \hline
		$\mathcal{G}$=\{$V^G$, $E^G$, $L^G$\} & a knowledge graph  \\ \hline
		$v_i$         & the $i^\textrm{th}$ entity/node in knowledge graph \\ \hline
		$r_i$         & the $i^\textrm{th}$ relation/edge in knowledge graph \\ \hline
		$e_i$         & the $i^\textrm{th}$ given by the user   \\ \hline
		$KS_i$         & knowledge segment $i$ \\ \hline
		${Q}$=\{$V^Q$, $E^Q$, $L^Q$\} & an attributed query graph \\ \hline
		$A_i$         & adjacency matrix of $KS_i$ \\ \hline
		$N_i$         & {attribute matrix of $KS_i$, the $j^\textrm{th}$ row} \\  
		 ~ & {denotes the attribute vector of node $j$ in $KS_i$} \\ \hline
		$A_{\times}$      & kronecker product of $A_1$ and $A_2$      \\ \hline
		$N^l$      & diagonal matrix of the $l^\textrm{th}$ node attribute  \\ \hline
		$N_{\times}$      &  combined node attribute matrix  \\ \hline
		$S^{i,j}$          & single entry matrix $S^{i,j}(i,j)=1$ and zeros elsewhere  \\ \hline
	\end{tabular}
\label{notation}
\vspace{-1.5\baselineskip}
\end{table}
}


\begin{table}[]
	\centering
	\footnotesize
	\caption{Notations and definitions}
	\setlength\tabcolsep{1.5pt}
	\begin{tabular}{|c|c|c|c|}
		\hline
		{\bf Symbols}       & {\bf Definition}     & {\bf Symbols}       & {\bf Definition}              \\ \hline
		$\mathcal{G}$=\{$V^G$, $E^G$, $L^G$\} & a knowledge graph  & $v_i$         & the $i^\textrm{th}$ entity/node in knowledge graph \\ \hline
		$r_i$         & the $i^\textrm{th}$ relation/edge in knowledge graph & $e_i$         & the $i^\textrm{th}$ given by the user   \\ \hline
		
		${Q}$=\{$V^Q$, $E^Q$, $L^Q$\} & an attributed query graph &  $KS_i$         & knowledge segment $i$  \\ \hline
		
		$A_i$         & adjacency matrix of $KS_i$   &  $A_{\times}$      & kronecker product of $A_1$ and $A_2$      \\ \hline
		$N^l$      & diagonal matrix of the $l^\textrm{th}$ node attribute  & $N_{\times}$      &  combined node attribute matrix  \\ \hline
		
		
		$N_i$         & {attribute matrix of $KS_i$, the $j^\textrm{th}$ row} &  $S^{i,j}$          & single entry matrix $S^{i,j}(i,j)=1$ and zeros elsewhere  \\ 
		 ~ & {denotes the attribute vector of node $j$ in $KS_i$} & & \\ \hline
		
	\end{tabular}
\label{notation}
\end{table}

In this section, we  first introduce the  symbols that will be used throughout the paper, then we introduce other important concepts and formally define the comparative reasoning problem.

Table ~\ref{notation} gives the main notations used throughout this paper. 
A knowledge graph can be denoted as $\mathcal{G}=(V, R, E)$ where $V = \{v_1, v_2, ..., v_n\}$ is the set of nodes/entities, $R = \{r_1, r_2, ..., r_m\}$ is the set of relations and $E$ is the set of triples.
Each triple in the knowledge graph can be denoted as $(h, r, t)$ where $h \in V$ is the head (i.e., subject) of the triple, $t \in V$ is the tail (i.e., object) of the triple and $r \in R$ is the edge (i.e., relation, predicate) of the triple which connects the head $h$ to the tail $t$. 

Given multiple pieces of clues, the goal of comparative reasoning is to infer commonality and/or inconsistency among them.  
If the given information is a pair of clues, we call it pairwise comparative reasoning or pairwise fact checking. The goal is to deduce whether these two clues are coherent or not. 
If the given information is a connected query graph, the goal is to detect whether there is inconsistency inside the given graph. The problem is called collective comparative reasoning or collective fact checking.
Different from traditional point-wise reasoning methods, comparative reasoning can unveil some subtle patterns which point-wise approaches may overlook.
Take knowledge graph based fact checking as an example, considering two claims/triples: 
({\tt Barack Obama}, {\tt graduatedFrom}, {\tt Harvard University}) and
({\tt Barack Obama}, {\tt majorIn}, {\tt Political Science}).
Even each clue/claim is true,  but if we check them together at the same time, we can see that they cannot be both true. This is because {\tt Barack Obama} majored in {\tt law} instead of {\tt Political Science} when he studied at {\tt Harvard University}. So, we might fail to detect the inconsistency between them without appropriately examining different clues/claims together.

To facilitate comparative reasoning, how to utilize the background information in knowledge graph is an important problem.
If we can find a subgraph in the knowledge graph which can best express the semantic meaning of each input clue, the hidden conflicts can be easier to detect. Ideally, this subgraph should contain all the meaningful/important entities and relations in the knowledge graph which 
are related to the given clue.
We call this subgraph {\em knowledge segment}, 
which is formally defined as follows.
\begin{definition}{\textbf{Knowledge Segment (KS for short)}} is a connection subgraph of the knowledge graph that best describes the semantic context of a piece of given clue (i.e., a node, a triple or a query graph).\vspace{-0.5\baselineskip}
\end{definition}

Figure ~\ref{inconsistency} gives an example of { knowledge segment}. Given two clues which are two query graphs extracted from the text and image respective, their corresponding knowledge segments are shown in the bottom of the Figure. As we can see, expressing the given clues with { knowledge segments} can help us detect the inconsistency without difficulty. 


%When reasoning among multiple pieces of clues according to their knowledge segments, 

Given the knowledge segments of multiple pieces of clues,
comparative reasoning aims to infer the commonality and inconsistency among these knowledge segments to make decision.
For pairwise case, the commonality refers to the common elements of these two knowledge segments. The inconsistency includes any elements that are contradicts with each other. 
Assuming the two given clues are two edges/triples: $E^Q_1 = <{\tt s_1}, {\tt p_1}, {\tt o_1}>$ and $E^Q_2 = <{\tt s_2}, {\tt p_2}, {\tt o_2}>$ where ${\tt s_1}, {\tt o_1}, {\tt s_2}, {\tt o_2} \in V$ and ${\tt p_1}, {\tt p_2} \in E$. We denote their corresponding knowledge segments as $KS_1$ for $E^Q_1$ and $KS_2$ for $E^Q_2$, respectively. 
The commonality and inconsistency between these two knowledge segments are defined as follows. 
\begin{definition}{\textbf{Commonality.}}
Given two triples ($E^Q_1$ and $E^Q_2$) and their knowledge segments ($KS_1$ and $KS_2$), the commonality of these two triples refers to the shared nodes and edges between $E^Q_1$ and $E^Q_2$, as well as the shared nodes and edges between $KS_1$ and $KS_2$: $((V^{KS_1} \cap V^{KS_2}) \cup (V^{Q_1} \cap V^{Q_2}) , (E^{KS_1} \cap E^{KS_2}) \cup (E^{Q_1} \cap E^{Q_2})  )$.
\end{definition}
\vspace{-0.8\baselineskip}
\begin{definition}{\textbf{Inconsistency.}}
Given two knowledge segments $KS_1$ and $KS_2$, 
the inconsistency between these two knowledge segments refers to any element (node, node attribute or edge) in $KS_1$ and $KS_2$ that contradicts with each other. 
\end{definition}

Different from pairwise comparative reasoning, collective comparative reasoning aims to find the commonality and/or inconsistency inside a query graph which consists of a set of inter-connected edges/triples. The corresponding definition is given below.
\begin{definition}{\textbf{Collective Commonality.}}
For each edge $E^Q_i$ in a query graph $Q$, let $KS_i$ be its knowledge segment. % extracted from the knowledge graph. 
The collective commonality between any triple pair in the query graph is the intersection of their knowledge segments.
\end{definition}
\vspace{-0.8\baselineskip}
\begin{definition}{\textbf{Collective Inconsistency.}}
For each edge $E^Q_i$ in a query graph $Q$, let $KS_i$ be its knowledge segment. % extracted from the knowledge graph. 
The collective inconsistency refers to any elements (node or edge or node attribute) in these knowledge segments that contradict with each other.
\end{definition}


%Multi-pieces comparative reasoning aims at checking whether multiple pieces of clues are inconsistent. 
Given the above notation and information, the problem of comparative reasoning is formal defined as: 

\begin{pro-stat}{Pairwise Comparative Reasoning: }
	
	\textbf{Given:} (1) A knowledge graph $\mathcal{G}$, (2) two triples $E^Q_1$ and $E^Q_2$;
	
	\textbf{Output:} A binary decision regarding the consistency of $E^Q_1$ and $E^Q_2$.
	
\end{pro-stat}




\begin{pro-stat}{Collective Comparative Reasoning: }
	
	\textbf{Given:} (1) A knowledge graph $\mathcal{G}$, (2) a query graph ${Q}$;
	
	\textbf{Output:}  A binary decision regarding the consistency of ${Q}$.
	
\end{pro-stat}





