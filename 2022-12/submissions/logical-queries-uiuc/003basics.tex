


In this section, we introduce how to extra knowledge segment to best express the semantic meaning of a given claim. We first introduce how to transform the knowledge graph into a relation specified weighted graph, then introduce how to extract Edge-specific Knowledge Segment and Subgraph-specific Knowledge Segment from it.


Generally speaking, given a clue (e.g., a triple or a query graph) from the user, 
the goal of knowledge segment extraction is to extra a subgraph which can best express the semantic meaning of the given clue.
Many existing methods have been proposed to  extract a concise subgraph from the source node of the querying edge to its target node in weighted or unweighted graphs. For example, multi-hop method ~\cite{knowledge-path}, minimum cost maximum flow method ~\cite{Shiralkar2017}, {\em K-}simple shortest paths based method ~\cite{Freitas} or connection subgraph  ~\cite{Faloutsos2004}, ~\cite{Koren2006MEP}, ~\cite{conan} extraction methods.


However, these methods do not directly apply to knowledge graphs because the edges (i.e., predicates) of a knowledge graph have specific semantic meanings (e.g., types, relations). To address this issue, we seek to convert the knowledge graph to a weighted graph by designing a predicate-predicate similarity measure for knowledge segment extraction. 


\subsection{Predicate-Predicate Similarity}

In order to transform the knowledge graph into weighted graph, 
We propose to use a TF-IDF based method to measure the similarity between different predicates, and transfer the knowledge graph into a weighted graph whose edge weight represents the similarity between the edge predicate and query predicate. 
The key idea behind TF-IDF based method is that we can treat each triple in the knowledge graph and its adjacent neighboring triples as a document, and use a TF-IDF like weighting strategy to calculate the predicate similarity. For example, predicate ${\tt receiveDegreeFrom}$ may have neighbor predicates $ {\tt major}$ and ${\tt graduateFrom}$. These predicates should have high similarity with each other. 

To be specific, we use the knowledge graph to build a co-occurrence matrix of predicates, and calculate their similarity by a TF-IDF like weighting strategy as follows. Let $i,j$ denote two different predicates. We define the $\textrm{TF}$ between two predicates as $\textrm{TF}(i, j) = \log(1 + C(i,j) {w}(j))$, where $C(i,j)$ is the co-occurrence of predicate $i$ and $j$. The $\textrm{IDF}$ is defined as $\textrm{IDF}(j) = \log \frac{|M|}{|\{i : C(i,j)>0\}|}$, 
where $M$ is the number of predicates in the knowledge graph. Then, we build a TF-IDF weighted co-occurrence matrix $U$ as $U(i, j) = \textrm{TF}(i, j) \times IDF(j)$.
Finally, the similarity of two predicates is defined as $\textrm{Sim(i, j)} = \textrm{Cosine}(U_i, U_j)$, 
where $U_i$ and $U_j$ are the $i^{th}$ row and $j^{th}$ row of $U$, respectively.

For the predicate-predicate similarity,
suppose we want to calculate the similarity between {\tt major} and {\tt study}. Both {\tt major} and {\tt study}  have only one adjacent neighboring predicate {\tt graduate}. This means that for any predicate $i \neq {\tt graduate}$, $U({\tt major}, i) = U({\tt study}, i) = 0$.
Since $\textrm{E}({\tt graduate}) = 0$, we have ${w}({\tt graduate}) = 2\sigma(\infty) - 1 = 1$. We have \textrm{TF}({\tt major}, {\tt graduate}) = \textrm{TF}({\tt study}, {\tt graduate}) = $\log(1 + 1 \times 1) = 1$, and $U({\tt major}, {\tt graduate}) = U({\tt study}, {\tt graduate}) = IDF({\tt graduate}) = \log\frac{8}{4} = 1$. If we compare the two vectors, $U_{{\tt major}}$ and $U_{{\tt study}}$, we find that they are the same. Therefore, we have that $\textrm{Sim}({\tt major}, {\tt study}) = 1$.


\subsection{Edge-specific Knowledge Segment}~\label{basic:edge}\vspace{-1\baselineskip}

Edge-specific knowledge segment extraction aims at finding a knowledge segment to best characterize the semantic context of the given edge (i.e., a triple). 
Several connection subgraph extraction methods exist for a weighted graph, e.g., ~\cite{Tong2006CSP} uses a random walk with restart based method to find an approximate subgraph; ~\cite{Koren2006MEP} uses maximal network flow to find a subgraph and ~\cite{Freitas} aims to find a denser local graph partitions. In this paper,
after transforming the knowledge graph into a weighted graph, we find {\em k-}simple shortest paths~\cite{Koren2006MEP} from the subject to the object of the given query edge as its knowledge segment. 



\subsection{Subgraph-specific Knowledge Segment}~\label{basic:subgraph}

Following the idea of edge-specific knowledge segment extraction, we extract a knowledge segment for each edge in the given subgraph and we call the graph which contains all the edge-specific knowledge segments subgraph-specific knowledge segment.
In other words, a subgraph-specific knowledge segment consists of multiple inter-linked edge-specific knowledge segments (i.e., one edge-specific knowledge segment for each edge of the input query subgraph). 

The subgraph-specific knowledge segment provides richer semantics, including both the semantics for each edge of the query graph and the semantics for the relationship between different edges of the input query graph. 







