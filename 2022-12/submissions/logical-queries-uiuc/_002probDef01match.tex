

Knowledge graph searching or graph querying in knowledge graph has been studied for a long time. 
The knowledge graph searching mainly focus on given a query graph, query pattern or SPARQL query which specified by users based on their own knowledge, finding a matching subgraph in the underlying knowledge graph, or given a NLP query, constructing a corresponding query graph and finding the matching subgraphs. 
In this paper, the query can be a node, and edge or a semantic query graph. The definition of the semantic query graph is given below.

\begin{definition}{Semantic Query Graph}
The semantic query graph can be denoted as $Q=(V, E)$ where $V$ is the set of entities in the query graph, and $E$ is the set of predicates in the query graph.
\end{definition}

Also different traditional knowledge graph searching algorithm, Given a semantic query graph $Q$, we hope to find a knowledge segment for the query graph. More specifically speaking, for each triple in the query graph $Q$, We will reason over the underlying knowledge graph to extract a semantic connection subgraph from it, with the subject and object nodes as the source and target nodes. We call the result as knowledge segment which is defined as below.

\begin{definition}{Knowledge Segment/Semantic subgraph}
Give a knowledge graph $G$ and semantic query $Q$, the knowledge segment can be denoted as $KS_Q = (V, E)$ which satisfies that $V_Q \in V_K$ and for each $e \in E_Q$, it has a correspond subgraph $G_s$. And $KS_Q = \cup G_s$.
\end{definition}


The main function of our system for knowledge graph searching can be divided into three parts: (1) given a node, finding its corresponding entity in the knowledge graph \textbf{G}; (2) given a edge, finding its corresponding knowledge Segment in the knowledge graph; (3) given a semantic subgraph \textbf{Q}, finding its corresponding knowledge segment in the knowledge graph.

%\lliu{Hi professor Tong, According to the results returned by Google, knowledge graph reasoning can be divided into 4 parts: missing entity predication, missing link predication, rules learning and fact prediction (Given a triple, predict whether it is true or fale).} 
%\hh{good, summarize these tasks in Table 2}

%\lliu{This is the definition from paper "A review: Knowledge reasoning over knowledge graph".  Knowledge reasoning over knowledge graphs aims to identify errors and infer new conclusions from existing data. New relations among entities can be derived through knowledge reasoning and can feed back to enrich the knowledge graphs, and then support the advanced applications. }

In summary, we could define our subgraph searching problem as below.


\begin{pro-stat}{Knowledge Graph Searching}
	
	\textbf{Given:} (1) A underlying knowledge graph; (2) A single node, a single edge or a semantic query graph.
	
	\textbf{Find:} (1) A list of node; (2) A Knowledge Segment.
	
\end{pro-stat}

