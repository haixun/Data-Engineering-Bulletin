


Knowledge graph reasoning has been studied for a long time, the goal of knowledge graph reasoning is to discover new knowledge or detect errors from the existing data. Currently, the application of knowledge graph reasoing can be divided into 4 parts: missing entity predication, missing link predication, rules learning and fact prediction (Given a triple, predict whether it is true or fale). Different from these methods, in our system, we focuses on knowledge graph comparative reasoning. More specifically speaking, given multiple piece of information the user has (multiple query nodes/edges/subgraphs), we aim to find their commonality (the intersection/AND), the difference (the diff) and the union. Or given the same input, but multiple KGs, we will compare and reason the search results from different KGs. Comparative knowledge graph reasoning has a lot of applications. One of the most useful application is detecting the fack news.

The idea behind comparative reasoning is that if two knowledge segments are common. They should have high similarity. If there are some inconsistencies, after we alter the inconsistent elements which could be node, edge and/or attributes, the similarity between these two knowledge segments could significantly improved.
Formally, the problem can be defined as below.
\begin{definition}{Comparative knowledge graph reasoning:}
Given two knowledge segements $KS_1$ and $KS_2$, found the common node set $VS$ and edge set $ES$ which satisfy $(VS, ES) = KS_1 \cap KS_2$. and find the elements that could significantly improve the similarity between two knowledge segments.
\end{definition}







%\hh{some additional functions we can consider (we might not implement them in this version of the system): (1) sense-making (given multiple query nodes, find groups between them, and for each group, find the best connection within the group ~\cite{Leman2013SIAM}, (2) given a single node/edge and multiple KGs, find commonality and differences in different KGs}





%\hh{let us try to have a table to summarize the different functions and whether any existing system and/or our system support it: first column: input; second column: output: third column: system XYZ~\cite{}, our method, etc. this will help differentiate our system to the existing works and also guide us which functions we should put more emphasis on}


%\hh{we can discuss this more -- one possible distinctive feature that differentiate us from the existing work might be the reasoning/comparison capability of our system. the existing work mostly focus on finding things that the user wants (given a query graph or spark query, find a match; given a NLP query, construct  the corresponding query graph. in contrast, in our system, given multiple piece of information the user has (multiple query nodes/edges/subgraphs, we will be able to find their commonality (the intersection/AND), the difference (the diff) and the union. or given the same input, but multiple KGs, we will compare and reason the search results from different KGs}






