%%%%%%%% mlsys 2022 EXAMPLE LATEX SUBMISSION FILE %%%%%%%%%%%%%%%%%
\documentclass{article}


% if you need to pass options to natbib, use, e.g.:
\PassOptionsToPackage{round}{natbib}
% before loading neurips_2022


% ready for submission
\usepackage{neurips_2022}

\usepackage{enumitem}
% to compile a preprint version, e.g., for submission to arXiv, add add the
% [preprint] option:
% \usepackage[preprint]{neurips_2022}


% to compile a camera-ready version, add the [final] option, e.g.:
%     \usepackage[final]{neurips_2022}


% to avoid loading the natbib package, add option nonatbib:
% \usepackage[nonatbib]{neurips_2022}


\usepackage[utf8]{inputenc} % allow utf-8 input
\usepackage[T1]{fontenc}    % use 8-bit T1 fonts
\usepackage{hyperref}       % hyperlinks
\usepackage{url}            % simple URL typesetting
\usepackage{booktabs}       % professional-quality tables
\usepackage{amsfonts}       % blackboard math symbols
\usepackage{nicefrac}       % compact symbols for 1/2, etc.
\usepackage{microtype}      % microtypography
\usepackage{xcolor}         % colors
\usepackage{graphicx}
\usepackage{subfigure}
\usepackage{bm}
\usepackage{amsmath,amssymb}
\usepackage{algorithm}
\usepackage[noend]{algpseudocode}
\usepackage{soul}
\usepackage{xspace}
\usepackage{multirow}
\usepackage{makecell}
\usepackage{ulem}

\DeclareMathOperator{\round}{round}
\DeclareMathOperator{\clamp}{clamp}

\newcommand{\R}{\mathbf R}
\newcommand{\Z}{\mathbf Z}
\newcommand{\cin}{C_{\text{in}}}
\newcommand{\cout}{C_{\text{out}}}
\newcommand{\wb}{\mathbf W}
\newcommand{\xb}{\mathbf x}
\newcommand{\yb}{\mathbf y}
\newcommand{\calA}{\mathcal{A}}
\newcommand{\bz}{\mathbf{z}}
\newcommand{\bv}{\mathbf{v}}
\newcommand{\bx}{\mathbf{x}}
\newcommand{\ba}{\mathbf{a}}
\newcommand{\calZ}{\mathcaL{Z}}
\newcommand{\SecInd}{{\sc SecInd}\xspace}
\newcommand{\SecAgg}{{\sc SecAgg}\xspace}
\newcommand{\FedAvg}{{\sc FedAvg}\xspace}
\newcommand{\SGD}{{\sc SGD}\xspace}

\def\ie{\textit{i.e.,}\@\xspace}
\def\eg{\textit{e.g.,}\@\xspace}

\newcommand{\pierre}[1]{{\color{purple}Pierre: #1}}
\newcommand{\sayan}[1]{{\color{red}Sayan: #1}}
\newcommand{\karthik}[1]{{\color{blue}Karthik: #1}}
\newcommand{\graham}[1]{{\color{green}Graham: #1}}
\newcommand{\ilya}[1]{{\color{red}Ilya: #1}}
\newcommand{\ashkan}[1]{{\color{blue}Ashkan: #1}}
\newcommand{\dzmitry}[1]{{\color{purple}Dzmitry: #1}}
\newcommand{\modif}[1]{{\color{black}#1}}

\title{Reconciling Security and Communication Efficiency in Federated Learning}
% Communication Efficient and Secure Federated Learning
% Unifying Secure Federated Learning and Communication Efficiency
% The Missing Bit for Practical Efficient Communication in Federated Learning
% Enabling Federated Learning with Communication Efficiency


% The \author macro works with any number of authors. There are two commands
% used to separate the names and addresses of multiple authors: \And and \AND.
%
% Using \And between authors leaves it to LaTeX to determine where to break the
% lines. Using \AND forces a line break at that point. So, if LaTeX puts 3 of 4
% authors names on the first line, and the last on the second line, try using
% \AND instead of \And before the third author name.


\author{
Karthik Prasad\thanks{Equal contribution. Correspondence to \texttt{pstock@fb.com}.} $^{~\dagger}$~~ Sayan Ghosh\footnotemark[1]$^{~~\dagger}$~~ Graham Cormode$^\dagger$~~ \\ \textbf{Ilya Mironov$^\dagger$~~ Ashkan Yousefpour$^\dagger$~~ Pierre Stock$^\dagger$} \\ $^\dagger$Meta AI
%   David S.~Hippocampus\thanks{Use footnote for providing further information
%     about author (webpage, alternative address)---\emph{not} for acknowledging
%     funding agencies.} \\
%   Department of Computer Science\\
%   Cranberry-Lemon University\\
%   Pittsburgh, PA 15213 \\
%   \texttt{hippo@cs.cranberry-lemon.edu} \\
  % examples of more authors
  % \And
  % Coauthor \\
  % Affiliation \\
  % Address \\
  % \texttt{email} \\
  % \AND
  % Coauthor \\
  % Affiliation \\
  % Address \\
  % \texttt{email} \\
  % \And
  % Coauthor \\
  % Affiliation \\
  % Address \\
  % \texttt{email} \\
  % \And
  % Coauthor \\
  % Affiliation \\
  % Address \\
  % \texttt{email} \\
}


\begin{document}

\maketitle 

\begin{abstract}

Cross-device Federated Learning is an increasingly popular machine learning setting to train a model by leveraging 
a large population of client devices with high privacy and security guarantees. 
However, 
communication efficiency remains a major bottleneck when scaling federated learning to production environments, particularly due to bandwidth constraints during uplink communication.
In this paper, we formalize and address the problem of compressing client-to-server model updates
under the Secure Aggregation primitive, a core component of Federated Learning pipelines that allows the server to aggregate the client updates without accessing them individually. 
In particular, we adapt standard scalar quantization and pruning methods
to Secure Aggregation and propose Secure Indexing, a variant of Secure Aggregation that supports quantization for extreme compression.
We establish state-of-the-art results on LEAF benchmarks in a secure Federated Learning setup with up to 40$\times$ compression in uplink communication 
with no meaningful loss in utility compared to uncompressed baselines.
% compared to a.
% \karthik{\st{with less than one bit per weight on the LEAF benchmark and no significant loss in utility.}} 
% \karthik{I have edited the abstract a bit, please check history to review my changes}
\end{abstract}

% this must go after the closing bracket ] following \twocolumn[ ...

% This command actually creates the footnote in the first column
% listing the affiliations and the copyright notice.
% The command takes one argument, which is text to display at the start of the footnote.
% The \mlsysEqualContribution command is standard text for equal contribution.
% Remove it (just {}) if you do not need this facility.

\section{Introduction}

Information Retrieval (IR) involves retrieving a set of candidates from a large document collection
given a user query. The retrieved candidates may be further reranked to bring the most relevant ones to the top, constituting a typical retrieve-and-rerank (R\&R) framework \cite{wang2018evidence, hu2019retrieve}.
Reranking generally improves the ranks of relevant candidates among those retrieved, thus improving on metrics such as Mean Reciprocal Rank (MRR) \cite{Craswell2009} and Normalized Discounted Cumulative Gain (nDCG) \cite{jarvelin2002cumulated}, which assign better scores when relevant results are ranked higher. 
However, retrieval metrics like Recall@K, which mainly evaluate the presence of relevant candidates in the top $K$ retrieved results, remain unaffected.
Increasing Recall@K can be key, especially when the retrieved results are used in downstream knowledge-intensive tasks \cite{petroni2021kilt} such as open-domain question answering \cite{chen2017reading, chen2020open, gangi2021synthetic}, fact-checking \cite{thorne2018fever}, entity linking \cite{hoffart2011robust,sil2013re,sil2018neural} and dialog generation \cite{dinan2018wizard, komeili2022internet}.

Most existing neural IR methods use a dual-encoder retriever \cite{karpukhin2020dense, khattab2020colbert} and a subsequent cross-encoder reranker \cite{nogueira2019passage}. 
Dual-encoder\footnote{We use the terms bi-encoder and dual-encoder interchangeably in this paper.} models leverage separate query and passage encoders and perform a late interaction between the query and passage output representations. This enables them to perform inference at scale as passage representations can be pre-computed. Cross-encoder models, on the other hand, accept the query and the passage together as input, leaving out scope for pre-computation. The cross-encoder typically provides better ranking than the dual-encoder---thanks to its more elaborate computation of query-passage similarity informed by cross-attention---but is limited to seeing only the retrieved candidates in an R\&R
framework.


\begin{wrapfigure}{r}{0.42\linewidth}
    \centering
    \includegraphics[width=1.0\linewidth]{submissions/Revanth2024/figures/cross_encoder_feedback_2.png}
    \caption{\textsc{ReFIT}: The proposed method for reranker relevance feedback. We introduce an inference-time distillation process (step 3) into the traditional retrieve-and-rerank framework (steps 1 and 2) to compute a new query vector, which improves recall when used for a second retrieval step (step 4).}
    \label{fig:overall_framework}
    \vspace{-1em}
\end{wrapfigure}

Since the more sophisticated reranker often generalizes better at passage scoring than the simpler, but more efficient retriever, here we propose to use relevance feedback from the former to improve the quality of query representations for the latter directly \textit{at inference}.
Concretely, after the R\&R pipeline is invoked for a test instance, we update the retriever's corresponding query vector by minimizing a distillation loss that brings its score distribution over the retrieved passages closer to that of the reranker.
The new query vector is then used to retrieve documents for the second time. 
This process effectively teaches the retriever how to rank passages like the reranker---a stronger model---for the given test instance.
Our approach, \textsc{ReFIT}\footnote{\textsc{ReFIT} stands for \textbf{Re}ranker \textbf{F}eedback at \textbf{I}nference \textbf{T}ime}, is lightweight as only the output query vectors (and no model parameters) are updated, ensuring comparable inference-time latency when incorporated into the R\&R framework. 
Figure \ref{fig:overall_framework} shows a schematic diagram of our approach, which introduces a distillation and a second retrieval step into the R\&R framework.
By operating exclusively in the representation space---as we only update the query vectors---our framework yields a parameter-free and architecture-agnostic solution, thereby providing flexibility along important application dimensions, e.g., the language, domain, and modality of retrieval. 
We empirically demonstrate this effect by showing improvements in retrieval on multiple English domains, across 26 languages in multilingual and cross-lingual settings, and in different modalities such as text and video retrieval.
 

Our main contributions are as follows:
\begin{itemize}
    \item We propose \textsc{ReFIT}, an inference-time mechanism to improve the recall of retrieval in IR using relevance feedback from a reranker.
    \item Empirically, \textsc{ReFIT} improves retrieval performance in multi-domain, multilingual, cross-lingual and multi-modal evaluation.
    \item The proposed distillation step is fast, considerably increasing recall without any loss in ranking performance over a standard R\&R pipeline with comparable latency.
\end{itemize}




\section{Related Work}
\label{sec:related-work}
\noindent \textbf{Pseudo-relevance feedback:} Our method has similarities with %the existing approach of 
Pseudo-Relevance Feedback (PRF) \cite{rocchio1971relevance, lv2009adaptive, li2022does} in IR: \cite{bendersky2011parameterized, xu2017quary} use the retrieved documents to improve sparse approaches via query expansion or query term reweighting, \cite{li2018nprf, zheng2020bert} score similarity between a target document and a top-ranked feedback document, while \cite{yu2021improving} train a separate query encoder that computes a new query embedding using the retrieved documents as additional input. In contrast, our approach does not require customized training feedback models or availability of explicit feedback data, as we improve the query vector by directly distilling from the reranker's output within an R\&R framework. %\pradeep{Why is our approach better?} 

Further, previous approaches to PRF have been dependent on the choice of retriever architecture and language; \cite{yu2021improving}'s PRF model is tied to the retriever used, \cite{chandradevan2022learning} explore cross-lingual relevance feedback, but require feedback documents in target language and thereby could only apply to three languages, while \cite{li2022interpolate} explore interpolating relevance feedback between dense and sparse approaches.
On the other hand, our approach is independent of the choice of the retriever and reranker architecture, and can be used for neural retrieval in any domain, language or modality. \\

\noindent \textbf{Distillation in Neural IR:} Existing approaches primarily leverage reranker feedback \textit{during training} of the dual-encoder retriever, to sample better negatives \cite{qu2021rocketqa}, for standard knowledge distillation of the cross-attention scores \cite{izacard2020distilling}, to train smaller and more efficient rankers by distilling larger models \cite{hofstatter2020improving}, or to align the geometry of dual-encoder embeddings with that from cross-encoders \cite{wang2021enhancing}. Instead, we leverage distillation at inference time, updating only the query representation to replicate the cross-encoder’s scores for the corresponding test instance.
A key implication of this design choice is that unlike existing methods, we keep the retriever parameters unchanged, meaning \textsc{ReFIT} can be incorporated out-of-the-box into any neural R\&R framework. In contrast, extending training-time distillation to new languages or modalities would require re-training the bi-encoder.

More recently, \textsc{TouR}~\cite{sung2023optimizing} has proposed test-time optimization of query representations with two variants: \textsc{TouR}$_{\text{hard}}$ and  \textsc{TouR}$_{\text{soft}}$. 
\textsc{TouR}$_{\text{hard}}$ optimizes the marginal likelihood of a small set of (pseudo) positive contexts.
\textsc{ReFIT} shares similarities with \textsc{TouR}$_{\text{soft}}$, which uses the normalized scores of a cross-encoder over the retrieved results as soft labels.
Crucially, \textsc{TouR} relies on multiple iterations of relevance feedback via distillation, where each iteration runs until the top-1 retrieval result has the highest reranker score (in \textsc{TouR}$_{\text{soft}}$) or is a pseudo-positive (in \textsc{TouR}$_{\text{hard}}$).
This makes inference highly computationally expensive, as each additional iteration involves labeling top-$K$ retrieval results with a reranker and then retrieving again.
\textsc{ReFIT} improves efficiency over \textsc{TouR} by requiring only a single iteration of feedback that simply updates the query vector for longer, foregoing additional retrieval and reranking steps. More specifics on the inference process of the two methods can be found in \S{\ref{sec:tour_comparison}}.
\textsc{TouR} was evaluated only on English phrase and passage retrieval tasks, while we demonstrate \textsc{ReFIT}'s effectiveness in multidomain, multilingual and multimodal settings, with an empirical comparison with \textsc{TouR} in \S{\ref{sec:tour_comparison}}.


\section{Background on Privacy-Sensitive Mobile Contact Tracing}

The key building block for Privacy-Sensitive Mobile Contact Tracing (PS-MCT) is a subtle combination of radio protocols, cryptography, and risk-calculation.  
Phones have a short-range radio, Bluetooth, used to connect to nearby devices.  
To make those connections it periodically broadcasts tiny bits of information. 
The PS-MCT protocols leverages this short-range background broadcast to resolve nearby individuals.\shankari{Is this the final term we decided on? I recall some pushback from David on the term, and I don't remember this as one of the options. Since our current focus is on the idea and not on a working system, it seems like it is a good idea to get the terminology right}


In the Apple-Google Exposure Notification (AGEN) protocol, each phone generates a daily secret key called a Temporary Exposure Key (TEK).
Then every 15 minutes the phone uses the TEK to generate a new 16 byte Rolling Proximity Identifier (RPI). The RPI sequence is generated using a cryptographic hash function, so it does not carry any information about the source individual.  
The current RPI is then continuously broadcast every few hundred milliseconds.
All phones log the RPIs they hear for future exposure analysis.  
Because the RPI is continuously changing, it also cannot be easily tracked.


When someone tests positive they can \textbf{anonymously} publish the daily keys (TEKs) from the days when they were contagious.  
The confirmed positive collection of TEKs is called a Diagnosis Key in the AGEN protocol.  
The Diagnois Keys are published by sharing them with a trusted server which publishes the TEKs for download.
Others can obtain these keys and use the same cryptographic hash functions to recreate the sequence of RPIs. This sequence, combined with some region-specific weights, can determine if they encountered any infected individuals.  
This entire process is accomplished within the Android and iOS operating systems. Government sanctioned apps are only responsible for authenticating infected individuals and, with user permission, publishing the keys. 

It is important to note the distinction between policy and mechanism. The AGEN protocol (and the extensions proposed in this paper) provide a mechanism to detect and notify users about exposure risk, but it is up to public health authorities to define what constitutes an exposure. This distinction is explored in more detail in section \ref{sec:commons}.




\section{Method}

Here we discuss the standard retrieve-and-rerank (R\&R) framework for IR (\S{\ref{sec:retrieve_and_rerank}}) and how our proposal fits into it (\S{\ref{sec:cross_encoder_feedback}}). While our approach can be applied to any R\&R framework, we consider a text-based retriever and reranker for simplicity while elaborating our method. A multi-modal R\&R framework is described in \S\ref{sec:multimodal_results}.


\subsection{Retrieve-and-Rerank}
\label{sec:retrieve_and_rerank}
R\&R for IR consists of a first-stage retriever and a second-stage reranker. Modern neural approaches typically use a dual-encoder model as the retriever and a cross-encoder for reranking.  

\paragraph{\textbf{The Retriever}:} The dual-encoder retriever model is based on a Siamese neural network \cite{chicco2021siamese}, containing separate Bert-based \cite{devlin2019bert} encoders $E_Q(.)$ and $E_P(.)$ for the query and the passage, respectively.
Given a query $q$ and a passage $p$, a separate representation is obtained for each, such as the \textsc{cls} output or a pooled representation of the individual token outputs from $E_Q(q)$ and $E_P(p)$. The question-passage similarity $sim(q,p)$ is computed as the dot product of their corresponding representations: query/passage.}
\begin{equation}
    Q_q = Pool(E_Q(q))
\end{equation}
\begin{equation}
    P_p = Pool(E_P(p))
\end{equation}
\begin{equation}\label{eq:sim}
   sim(q,p) = S(Q_q,P_p) = Q_q^TP_p
\end{equation}

Since Eq.~\ref{eq:sim} is decomposable, the representations of all passages in the retrieval corpus can be pre-computed and stored in a dense index \cite{johnson2019billion}. During inference, given a new query, the top $K$ most relevant passages are retrieved from the index via approximate nearest-neighbor search.

\paragraph{\textbf{The Reranker}:} The cross-encoder reranker model uses a Bert-based encoder $E_R(.)$, which takes the query $q$ and a corresponding retrieved passage $p$ together as input and outputs a similarity score. 
A feed-forward layer $F$ is used on top of the \textsc{cls} output from $E_R(.)$ to compute a single logit, which is used as the final reranker score $R(q,p)$. The top $K$ retrieved passages are then ranked based on their corresponding reranker scores.

\begin{equation}
   R(q,p) = F(CLS(E_R(q,p))
\end{equation}


\begin{algorithm}[t]
\caption{\textsc{\textbf{ReFIT}}}
\label{alg4}
\begin{flushleft}
\textbf{Input}: Query $q$ and its representation $Q_q$, retrieved passages $P$ and their representations $\hat{P}$.\newline
\textbf{Output}: Updated query representation $Q_{q,n}$
\end{flushleft}
\begin{algorithmic}[1]
    \State Initialize query vector $Q_{q,0}$ = $Q_q$
    \State Compute reranker distribution $D_{CE}(q,P)$ (Eq.~\ref{eq:d-ce})
    \For{\textit{i in 0 to n}}
        \State Compute retriever distribution $D_{Q_{q,i}}(\hat{P})$ (Eq.~\ref{eq:d-q})
        \State Compute loss $\mathcal{L}$ (Eq.~\ref{eq:loss})
        \State Update $Q_{q,i+1} = Q_{q,i} - \alpha \frac{\partial}{\partial Q_{q,i}}\mathcal{L}$
    \EndFor
    \State return $Q_{q,n}$
\end{algorithmic}
%\vspace{-0.4em}
\end{algorithm}

\subsection{Reranker Relevance Feedback}
\label{sec:cross_encoder_feedback}
The main idea underlying our proposal is to compute an improved query representation for the retriever using feedback from the more powerful reranker.
More specifically, we perform a lightweight inference-time distillation of the reranker's knowledge into a new query vector.

Given an input query $q$ during inference, we use the following output provided by the R\&R pipeline:
\begin{itemize}
   \item Query representation $Q_q$ from the retriever.
    \item Retrieved passages $P = \{p_1, p_2,  ..., p_K\}$ and their representations $\hat{P} = [P_{p_1}, P_{p_1},  ..., P_{p_K}]$ from the retriever. 
    \item The reranking scores $R(q,P) = [R(q,p_1),..., R(q,p_K)]$.
\end{itemize}
Note that $\hat{P}$ above is directly obtained from the passage index and is not computed during inference.

The proposed reranker feedback mechanism begins with using the reranking scores $R(q,P)$ to compute a cross-encoder ranking distribution $D_{CE}(q,P)$ over passages $P$ as follows:

\begin{equation}
D_{CE}(q,P)=\mathrm{softmax}([R(q,p_1), ..., R(q,p_K)])
\label{eq:d-ce}
\end{equation} 

The query and passage representations from the retriever are used to compute a similar distribution $D_{Q_q}(\hat{P})$ over $P$:

\begin{equation}
    D_{Q_q}(\hat{P}) = \mathrm{softmax}([Q_q^TP_{p_1}, ..., Q_q^TP_{p_K}])
    \label{eq:d-q}
\end{equation}

Next, we compute the loss as the KL-divergence between the retriever and reranker distributions:

\begin{equation}
    \mathcal{L} = D_{KL}(D_{CE}(q,P) || D_{Q_q}(\hat{P}))
    \label{eq:loss}
\end{equation}

which is then used to update the query vector via gradient descent. The query vector update process is repeated for $n$ times, where $n$ is a hyper-parameter. 
A schematic description of the process can be found in Algorithm \ref{alg4}. 

Finally, the updated query vector $Q_{q,n}$ is used for a second-stage retrieval from the passage index.  
From dual-encoder retrieval with the updated $Q_{q,n}$, we aim to achieve better recall than with the initial $Q_q$, while obtaining a ranking performance that is comparable with that of the reranker.








%!TEX root = ../main.tex
\section{Experiments}
\label{sec:exp}

We conduct preliminary experiments to demonstrate the effectiveness of \sys. 
We evaluate its performance in two key processes: Reasoning and Verification. %The multimodal data lakes and data discovery techniques used in these experiments are discussed respectively for each process. 

%\subsection{Question Answering using Retrieved Multimodal Data}

%To validate the effectiveness of our method in text reasoning and mutimodal (text+image) reasoning, we design the experiment on natural language question answering and visual-based entity question answering.

\subsection{Question Answering}

\stitle{Experiment Setting.} In this experiment, we focus on evaluating question answering performance using a multimodal data lake consisting of 400K web tables and 6M English passages extracted from Wikipedia. The data lake includes both tables and texts, and each query is designed to retrieve relevant data items to answer a given question. We use 18 manually crafted user queries, each with corresponding ground truth annotations specifying the required data items, sub-queries for decomposition, and final answers.

\stitle{Data Discovery Evaluation.}
The effectiveness of data discovery is measured using the recall at $K$ (R@$K$) metric, which calculates the proportion of relevant data items retrieved in the top-$K$ recommendations. The experimental results show that when $K$ is 5, 10, 15, and 20, the R@$K$ values are 40.8\%, 46.3\%, 59.3\%, and 77.8\%, respectively. For 12 out of the 18 queries, \sys successfully discovers all the relevant items needed to answer the query. The remaining 6 queries show partial success. In total, 30 out of 38 related items are correctly discovered, demonstrating the potential of the proposed data discovery methodology, even though it is still in a preliminary stage.


\stitle{Query Decomposition Evaluation.}
To decompose queries into manageable sub-queries, \sys serializes the discovered data items and uses GPT-3 to generate sub-queries. The output includes the sub-queries and corresponding data item ids. Evaluation of the decomposition quality is based on two criteria: (1) whether each sub-query is useful for solving the original query, and (2) whether the sub-query can be answered correctly using the selected data item. The human evaluation results show that 77.8\% of the queries scored 2 (both criteria met), 16.7\% scored 1 (only the first criterion met), and 5.5\% scored 0. 

Table~\ref{tab:results_of_decomposition} shows the results of 8 instances. \sys is able to handle different aggregation operations, such as sum (Instance 2) and comparison (Instance 3). Further, it correctly understands long sentences (Instance 1). However, \sys has difficulty in dealing with sentences with complex syntactic structures (\eg in Instance 4, GPT-3 mistakenly generates $q_3$ because it takes the subject of released'' as Academy Award for Best Foreign Language Film''.). 


\begin{table}[!t]
\centering
\resizebox{\textwidth}{!}{
{\small
\begin{tabular}{p{15cm}}
\hline
\underline{Instance 1}: $Q$: Did the 2014 United States house of representatives elections take place on the same day in Rhode Island, South Carolina and Louisiana? $E = \{P_1, P_2, P_3\}$\\
$\rightarrow$ $q_1$: Did the 2014 United States House of Representatives elections take place on November 4, 2014 in Rhode Island? $e_1 = P_1$; $q_2$: Did the 2014 United States House of Representatives elections take place on November 4, 2014 in South Carolina? $e_2 = P_2$; $q_3$: Did the 2014 United States House of Representatives elections take place on November 4, 2014 in Louisiana? $e_3 = P_3$ \\
$\bigstar$ Score: 2 \\
\hline
\underline{Instance 2}: $Q$: How many representatives were elected in the 2014 United States house of representatives elections in Rhode Island and South Carolina?   %\srm{Question is vague, are we asking for a sum?}
$E = \{P_1, P_2\}$ \\
$\rightarrow$ $q_1$: How many representatives were elected in the 2014 United States house of representatives elections in Rhode Island? $e_1 = P_1$; $q_2$: How many representatives were elected in the 2014 United States house of representatives elections in South Carolina? $e_2 = P_2$ \\
$\bigstar$ Score: 2 \\
\hline
\underline{Instance 3}: $Q$: Which number is larger, the number of living former prime ministers of Tunisia or the number of former presidents of Ghana? $E = \{T_1, P_1\}$ \\
$\rightarrow$ $q_1$: What is the number of living former prime ministers of Tunisia? $e_1 = P_1$; $q_2$: What is the number of former presidents of Ghana? $e_2 = T_1$ \\
$\bigstar$ Score: 2 \\
\hline
\underline{Instance 4}: $Q$: What year was the first German film that won the Academy Award for Best Foreign Language Film released? $E = \{T_1, P_1\}$ \\
$\rightarrow$ $q_1$: What year was the film The Tin Drum released? $e_1 = T_1$; $q_2$: What was the first German film that won the Academy Award for Best Foreign Language Film? $e_2 = T_1$; $q_3$: What year was the first Academy Award for Best Foreign Language Film given out? $e_3 = T_1$ \\
$\bigstar$ Score: 0 \\
\hline
\underline{Instance 5}: $Q$: Did The Tin Drum and Kolya both win the Academy Awards for Best Foreign Language Film? $E = \{P_1, P_2\}$ \\
$\rightarrow$ $q_1$: whether The Tin Drum has ever won the Academy Award for Best Foreign Language Film? $e_1 = P_1$; $q_2$: whether Kolya has ever won the Academy Award for Best Foreign Language Film? $e_2 = P_2$ \\
$\bigstar$ Score: 2 \\
\hline
\underline{Instance 6}: $Q$: How many years elapsed between the premiere of season 10 and the premiere of season 5 of Law \& Order : Special Victims Unit? $E = \{T_1, T_2\}$ \\
$\rightarrow$ $q_1$: What is the premiere date of season 10 of Law \& Order : Special Victims Unit? $e_1 = T_2$; $q_2$: What is the premiere date of season 5 of Law \& Order : Special Victims Unit? $e_1 = T_1$ \\
$\bigstar$ Score: 2 \\
\hline
\underline{Instance 7}: $Q$: Did You Let Me Walk Alone participate in the Eurovision Song Contest in 1996? $E = \{T_1, P_1\}$ \\
$\rightarrow$ $q_1$: What year was the Eurovision Song Contest held in Lisbon, Portugal? $e_1 = T_1$; $q_2$: What year was You Let Me Walk Alone released? $e_2 = P_1$ \\
$\bigstar$ Score: 1 \\
\hline
\underline{Instance 8}: $Q$: Are the tallest building in the united kingdom and the tallest building in poland above 200 meters? $E = \{T_1, T_2\}$\\
$\rightarrow$ $q_1$: What is the height of the tallest building in the United Kingdom? $e_1 = T_1$; $q_2$: What is the height of the tallest building in Poland? $e_2 = T_2$ \\
$\bigstar$ Score: 2 \\
\hline
\end{tabular}
}
}
\caption{Example sub-queries generated by \sys. $q_i$ and $e_i$ represent the $i_{th}$ sub-query and its corresponding data item. $T_i$ represents a table and $P_i$ represents a text. 
}
\label{tab:results_of_decomposition}
\end{table}


% \subsubsection{Visual-based Entity Question Answering}

% \stitle{Experiment Setting.}
% The experiments were conducted in a zero-shot setting using RTX 4090 GPUs. For GPT-4V, we used the interface of the GPT-4-vision-preview model. It's worth noting that GPT-4V often refrains from answering person identify questions without additional clues due to policy reasons. However, with the incorporation of matching graph techniques, it can leverage weak signals and combine them with its own knowledge base. For the dataset, we chose NewsPersonQA~\cite{zhang2024mar}. For the task evaluation, we use accuracy (\textbf{Acc}) as an evaluation metric. Furthermore, we assess the accuracy only for instances where relevant clues are successfully retrieved, which is denoted as \textbf{Acc}$^{{hit}}$.

% \stitle{Baseline.} For answering queries, we selected two well-known and highly capable MLLMs, as well as human evaluation,to serve as baselines. \textbf{LLaVA:} 
% This model utilizes CLIP-ViT-L-336px with an MLP projection. We refer to the 1.5 version with 7 billion parameters as LLaVA-7b and the version with 13 billion parameters as LLaVA-13b.
% \textbf{GPT-4V:} 
% Recognized as OpenAI's most powerful general-purpose MLLM to date, GPT-4V boasts 1.37 trillion parameters. 

% \stitle{Main Results.} The main results of visual-based entity question answering are summarized in Table~\ref{tbl:single}, which leads to the following insights: LLaVA-13b demonstrates higher accuracy (27.93\%) compared to LLaVA-7b (22.26\%), suggesting that a model's recognition ability is positively correlated with its parameter size, which to some extent reflects its knowledge base. Incorporating a matching graph leads to an 8.9\% improvement in accuracy for LLaVA-7b and a 3.2\% improvement for LLaVA-13b. GPT-4V, with matching, achieves a character recognition accuracy of 34.83\%.
%  The enhancement from matching is more pronounced for LLaVA-7b than for LLaVA-13b, indicating that while matching can compensate for differences in parameters, a model's inherent capabilities still set an upper limit on its performance.

% \begin{table}[t!]
% \caption{Result for Visual-based Entity Question Answering. (Note: GPT-4V could not answer these queries directly due to policy constraints. Values within parentheses are those GPT-4V still refuses to answer.)}
% \small
% \centering
% \label{tbl:single}
% \resizebox{0.5\columnwidth}{!}{
% \begin{tabular}{l|l|l}
% \toprule
% {\textbf{Models}}     & \textbf{Acc} (\%) & \textbf{Acc}$^{{hit}}$ (\%) \\ 
% \midrule
% \textbf{LLaVA-7b}                        & 22.26        & 27.53         \\
% {\textbf{LLaVA-7b + Symphony}} & 31.19        & 62.81         \\ 
% \hline
% \textbf{LLaVA-13b}                       & 27.93        & 32.86         \\
% {\textbf{LLaVA-13b + Symphony}} & 31.13        & 62.34         \\ 
% \hline
% \textbf{GPT-4V}                          & -            & -             \\
% {\textbf{GPT-4V + Symphony}} & 34.84 (4.2)  & 68.31 (2.6)  \\ 
% \midrule
% \textbf{Symphony(Graph Reasoning)}                            & {\bf 39.09}        & {\bf 79.65}         \\ 
% \bottomrule
% \end{tabular}
% }
% \end{table}


\subsection{Answer Verification}

We showcase preliminary experimental results that highlight the initial achievements of \sys in facilitating the verification of generative AI. 
% \yang{Do we need to include tuple-tuple and text generation?}

\stitle{Experiment Setting.}
We perform a controlled study to assess textual claims, employing 1,300 textual claims from the TabFact~\cite{chen2019tabfact} benchmark, which is currently the most advanced benchmark for verifying the credibility of textual hypotheses by utilizing a given table. The data lake consists of 16,573 tables from the TabFact and 2,925 tables sourced from WikiTable-TURL~\cite{deng2022turl}.


\begin{figure}[t!]
\vspace{1em}
\begin{center}
  \includegraphics[width=0.65\textwidth]{submissions/Nan2024/figs/tabfact.pdf}
  \caption{Verifying a textual claim using retrieved tables.}
  \label{fig:claim_case} 
\end{center}
\end{figure}


\stitle{Evaluation for Retrieval.} 
We use Elasticsearch~\cite{elasticsearch} to retrieve the top-5 tables for each textual claim. Given the limited amount of relevant data, we focus on the recall metric for evaluation. Each textual claim is associated with a corresponding table in the original dataset, which we consider relevant evidence, while other retrieved tables are deemed irrelevant. The retrieval performance, measured by R@5, is 0.88.

\stitle{Evaluation for Verification.} 
We evaluate the verification process using two different verifiers: GPT-3.5, the default verifier for both data types, and PASTA~\cite{pasta}, a specialized model for text verification.
%
The performance of the verifiers is measured by accuracy. When the retrieved data cannot support or refute a claim, the verifier outputs ``not related''. However, in this case, since PASTA that only offers two different answers: ``true'' or ``false'', we consider it's also correct when PASTA outputs ``false''.

We conduct experiments in two settings. When a relevant table is retrieved and provided as evidence to the verifier, PASTA achieves higher accuracy than GPT-3.5 (0.89 vs. 0.75) in verifying the textual claim based on the table. However, in cases where many of the retrieved tables are irrelevant to the claim, the verifier must accurately determine which tables are not related. In this setting, PASTA's accuracy drops to 0.72 because it has not encountered this scenario during training, while GPT-3.5 improves to 0.91. 
% Thus, GPT-3.5-turbo demonstrates superior generalization capabilities and performs better than PASTA when dealing with irrelevant tables.
Thus, when the retrieved data is highly related to the generative data, local models like PASTA have higher accuracy while protecting privacy. In contrast, GPT-3.5 is better at generalizing and providing explanations for further judgments. Users can select the appropriate model based on their requirements.


In Figure~\ref{fig:claim_case}, we present a case of verifying a textual claim based on retrieved tables using GPT-3.5. \sys retrieves two tables $E_1$ and $E_2$, where $E_1$ can be used with an aggregation query to refute the claim while $E_2$ is not related because it is for the year 1959. The red boxes in Figure~\ref{fig:claim_case} show that GPT-3.5 can provide not only a verification result but also some explanation.






%\section{Future Work}
%
%As a pioneering attempt to solve the truth discovery problem in crowdsensing, our research still has certain limitations and points out some future opportunities.
%
%\textbf{Multi-class and Continuous-value Event Sensing}. In this work, we focus on binary event sensing. Naturally, our method can be extended to multi-class event sensing. Besides, sometimes we need to sense continuous value for a certain event (e.g., temperature). We will also study this in the future.
%
%%\textbf{Homomorphic Encryption-enabled Truth Discovery}. In FL systems, two widely adopted security mechanisms are homomorphic encryption (HE) and secure multi-party computation (SMC) \cite{yang2019federated}. Shamir's secret sharing is one type of SMC, which we rely on to build FedTruthFinder. HE may be another technical route to reach the goal. Currently, we do not choose HE because its computation efficiency is 10-100 times slower than non-HE computation. Some pioneering efforts have been devoted to developing privacy-preserving truth discovery systems with HE. Due to the large computation burden, these studies only let two non-colluding powerful servers to do HE computation instead of letting each client do computation in a federated (distributed) manner \cite{Tang2018NonInteractivePT,Zheng2018LearningTT}. However, finding two non-colluding servers are not so easy \cite{Bonawitz2017PracticalSA}.
%
%
%% they assume there need to be two non-colluding parties who take all the computation tasks.
%
%%Currently, we do not choose HE because its computation efficiency is 10-100 times slower than non-HE computation. Recently, HE tools have developed rapidly \cite{sealcrypto}, and we believe HE-enabled truth discovery is a promising direction. 
%
%\textbf{Other Truth Discovery Algorithms}. Besides the iterative truth discovery algorithm discussed in this paper, there are also other state-of-the-art truth discovery algorithms, such as optimization-based methods \cite{Li2014ACA,Li2015OnTD} and graphical model-based methods \cite{Pasternack2013LatentCA,Zhao2012ABA}. A comprehensive survey on truth discovery algorithms can be referred to \cite{li2016survey}. In the future, we will study whether our methods can be adapted to more truth discovery algorithms.
%
%\textbf{Against More Serious Attack Scenarios}. In this work, we assume that all the users are semi-honest, i.e., they will follow the protocol to compute and upload the corresponding data. In a more competitive setting, users can be malicious and upload some wrong data for attacking other users' privacy. We will discuss the solutions under such conditions in the future.
%
%\textbf{Missing Negative Sensed Data}. For some crowdsensing applications, users would only upload the positive sensed data ($e_j=1$). The truth discovery algorithms need to be refined as negative sensed data ($e_j=0$) are implicit. We will try to develop the federated version of such truth discovery algorithms in our future work.
%
%\textbf{Deployment on Smartphones}. In this work, we focus on the algorithmic design. In the future, we will implement the algorithm and deploy it to smartphones to further test its communication costs, energy consumption, etc. We will try to find practical guidelines to deploy such a federated crowdsensing truth discovery mechanism.

\section{Conclusion}

In this paper, we propose \textit{FedTruthFinder}, a crowdsensing federated truth discovery mechanism that can not only find aggregate truth from multiple participants' sensed data, but also rank participants' trustworthiness in a privacy-preserving manner. The primary characteristic of FedTruthFinder is its capability to tolerate network connection loss of participants in both event confidence calculation and participant trustworthiness ranking. As a byproduct, our proposed federated ranking algorithm can also serve other applications when the privacy-preserving data ranking is needed and the network connections are unstable.
Following most related papers, this work assumes participants to be semi-honest; in the future, we would explore the more challenging scenario that participants may behave maliciously.%To the best of our knowledge, FedTruthFinder is the best privacy-preserving truth discovery mechanism for crowdsensing regarding the robustness against participants' unpredictable connection loss. 



\section{Conclusion}
In this paper, we reconcile efficiency and security for uplink communication in Federated Learning. We propose to adapt existing compression mechanisms such as scalar quantization and pruning to the secure aggregation protocol by imposing a linearity constraint on the decompression operator. Our experiments demonstrate that we can adapt both quantization and pruning mechanisms to obtain a high degree of uplink compression with minimal degradation in performance and higher security guarantees. For achieving the highest rates of compression, we introduce \SecInd, a variant of \SecAgg well-suited for TEE-based implementation that supports product quantization while maintaining a high security bar. 
We plan to extend our work to other federated learning scenarios, such as asynchronous FL, and further investigate the interaction of compression and privacy.
% and global and local differential privacy, 
% and investigate strategies to select optimal compression parameters 
% (quantization scales, centroids and pruning masks) 
% for better performance in these scenarios.

% % This 
% % - Generic and less data-dependent codebook
% % - Model compression 
% % - Compression and DP
% % - Error correction

% % \section*{Acknowledgements}


\bibliography{bibliography}
\bibliographystyle{plainnat}
% \pagebreak
\graham{\sout{Checklist}}
%\section*{Checklist}

% %%% BEGIN INSTRUCTIONS %%%
% The checklist follows the references.  Please
% read the checklist guidelines carefully for information on how to answer these
% questions.  For each question, change the default \answerTODO{} to \answerYes{},
% \answerNo{}, or \answerNA{}.  You are strongly encouraged to include a {\bf
% justification to your answer}, either by referencing the appropriate section of
% your paper or providing a brief inline description.  For example:
% \begin{itemize}
%   \item Did you include the license to the code and datasets? \answerYes{See Section~\ref{gen_inst}.}
%   \item Did you include the license to the code and datasets? \answerNo{The code and the data are proprietary.}
%   \item Did you include the license to the code and datasets? \answerNA{}
% \end{itemize}
% Please do not modify the questions and only use the provided macros for your
% answers.  Note that the Checklist section does not count towards the page
% limit.  In your paper, please delete this instructions block and only keep the
% Checklist section heading above along with the questions/answers below.
% %%% END INSTRUCTIONS %%%

\begin{enumerate}
\item For all authors...
\begin{enumerate}
  \item Do the main claims made in the abstract and introduction accurately reflect the paper's contributions and scope?
    \answerYes{See the claim list in Section~\ref{sec:intro}}
  \item Did you describe the limitations of your work?
    \answerYes{See in particular Section~\ref{sec:discussion}}
  \item Did you discuss any potential negative societal impacts of your work?
    \answerYes{See in particular Section~\ref{sec:discussion}}
  \item Have you read the ethics review guidelines and ensured that your paper conforms to them?
    \answerYes{}
\end{enumerate}


\item If you are including theoretical results...
\begin{enumerate}
  \item Did you state the full set of assumptions of all theoretical results?
    \answerNA{}
        \item Did you include complete proofs of all theoretical results?
    \answerNA{}
\end{enumerate}


\item If you ran experiments...
\begin{enumerate}
  \item Did you include the code, data, and instructions needed to reproduce the main experimental results (either in the supplemental material or as a URL)?
    \answerYes{We provide detailed instructions in Section~\ref{sec:experiments} and in the Appendix}  \answerNo{We did not provide the code in the supplementary but will provide it when publishing the paper} 
  \item Did you specify all the training details (e.g., data splits, hyperparameters, how they were chosen)?
    \answerYes{See Section~\ref{sec:experiments} and in the Appendix}
        \item Did you report error bars (e.g., with respect to the random seed after running experiments multiple times)?
    \answerYes{See Section~\ref{sec:experiments} in the setup description and Appendix for the detailed error bars over 3 independent runs. Note that we do not report bars for ablation results.}
        \item Did you include the total amount of compute and the type of resources used (e.g., type of GPUs, internal cluster, or cloud provider)?
    \answerYes{See Section~\ref{sec:experiments} in the setup description.}
\end{enumerate}


\item If you are using existing assets (e.g., code, data, models) or curating/releasing new assets...
\begin{enumerate}
  \item If your work uses existing assets, did you cite the creators?
    \answerYes{See Section~\ref{sec:experiments}}
  \item Did you mention the license of the assets?
    \answerYes{See Appendix}
  \item Did you include any new assets either in the supplemental material or as a URL?
    \answerNo{}
  \item Did you discuss whether and how consent was obtained from people whose data you're using/curating?
    \answerNA{}
  \item Did you discuss whether the data you are using/curating contains personally identifiable information or offensive content?
    \answerNA{}
\end{enumerate}


\item If you used crowdsourcing or conducted research with human subjects...
\begin{enumerate}
  \item Did you include the full text of instructions given to participants and screenshots, if applicable?
    \answerNA{}
  \item Did you describe any potential participant risks, with links to Institutional Review Board (IRB) approvals, if applicable?
    \answerNA{}
  \item Did you include the estimated hourly wage paid to participants and the total amount spent on participant compensation?
    \answerNA{}
\end{enumerate}


\end{enumerate}}


% %%%%%%%%%%%%%%%%%%%%%%%%%%%%%%%%%%%%%%%%%%%%%%%%%%%%%%%%%%%%%%%%%%%%%%%%%%%%%%%
% %%%%%%%%%%%%%%%%%%%%%%%%%%%%%%%%%%%%%%%%%%%%%%%%%%%%%%%%%%%%%%%%%%%%%%%%%%%%%%%
% % SUPPLEMENTAL CONTENT AS APPENDIX AFTER REFERENCES
% %%%%%%%%%%%%%%%%%%%%%%%%%%%%%%%%%%%%%%%%%%%%%%%%%%%%%%%%%%%%%%%%%%%%%%%%%%%%%%%
% %%%%%%%%%%%%%%%%%%%%%%%%%%%%%%%%%%%%%%%%%%%%%%%%%%%%%%%%%%%%%%%%%%%%%%%%%%%%%%%
\appendix
%\onecolumn
\appendix
\section{Appendix}

\subsection{Theoretical Proof}

%\subsubsection{Correctness} We first prove that the correctness of our secure leader-board algorithm.
%
%\vspace{+.5em}
%\textbf{Lemma 5.1}. $\sum_{k=1}^{2t+1} r_k(x)\tau_{i_k}(x)$ can be represented as:
%$$h_{i}+a_{i1}x+a_{i2}x^2+...+a_{i2t}x^{2t}$$
%where $h_i=\sum_{k=1}^{2t+1} r_k\tau_i^k$. \cite{tang2011secure}

\textbf{Proof of Lemma 5.1}. It is clear that,
\begin{equation}
	\sum_{k=1}^{2t+1} r_k(0)\tau_{i_k}(0) = \sum_{k=1}^{2t+1} r_k\tau_i^k
\end{equation}
Besides, both $r_k(x)$ and $\tau_{i_k}(x)$ are $t$-degree polynomials, and thus the degree of $\sum_k r_k(x)\tau_{i_k}(x)$ is $2t$.$\qed$

\vspace{+.5em}
%\textbf{Theorem 5.1}. With $t+1$ participants' $h'_i(k)$, we can recover $h_i$.

\noindent \textbf{Proof of Theorem 5.1}. With Lemma 5.1, for $N$ ($=2t+1$) groups, $\gamma(gid(u_j))
=\sum_{k=1}^{2t+1} r_k(gid(u_j)) \tau_{i_k}(gid(u_j))$ (Step 4) is:

\[
\small
\left(\begin{array}{ccccc} 
	1 &    1 & 1^2 & ...  & 1^{2t} \\ 
	1 &    2 & 2^2 & ... & 2^{2t}\\
	... & ... & ... & ...& ...\\
	1 & N & N^2 & ... & N^{2t}\\
\end{array}\right) 
\left(\begin{array}{c} 
	h_i    \\ 
	a_{i1}    \\
	... \\
	a_{i2t} \\
\end{array}\right) 
=
\left(\begin{array}{c} 
	\gamma(1)    \\ 
	\gamma(2)    \\
	... \\
	\gamma(N) \\
\end{array}\right) 
\]
then,
\[
\small
\left(\begin{array}{c} 
	h_i    \\ 
	a_{i1}    \\
	... \\
	a_{i2t} \\
\end{array}\right) 
=
\left(\begin{array}{ccccc} 
	1 &    1 & 1^2 & ...  & 1^{2t} \\ 
	1 &    2 & 2^2 & ... & 2^{2t}\\
	... & ... & ... & ...& ...\\
	1 & N & N^2 & ... & N^{2t}\\
\end{array}\right)^{-1} 
\left(\begin{array}{c} 
	\gamma(1)    \\ 
	\gamma(2)    \\
	... \\
	\gamma(N) \\
\end{array}\right) 
\]
so,
$$h_i=\sum_{g=1}^{N} \lambda(g)\gamma(g)$$

In Step 5, $h_i(g)=\lambda(g)\gamma(g)$ is shared with $(t+1,n)$-SSS to all the participants from every group $g \in [1, 2t+1]$. Hence, according to the additive homomorphism property of SSS \cite{shamir1979share}, we can easily recover $h_i$ by receiving $t+1$ participants'  $h'_i(k) = \sum_{g=1}^{2t+1} h_i(g, k)$.$\qed$

\vspace{+.5em}
%\textbf{Theorem 5.2}. Ranking $h_i$ is equivalent to ranking $\tau_i$.

\noindent \textbf{Proof of Theorem 5.2}. As $\tau_i>0$ and $r_k>0$, $h_i=\sum_k r_k\tau_i^k$ will keep the same ranking as $\tau_i$.$\qed$

%\subsubsection{Robustness to Connection Loss} While mobile users may lose network connections during a crowdsensing campaign, we analyze how our secure ranking algorithm can tolerate connection losses. Without the loss of generalizability, we assume that before Step 2, there is no user connection loss.\footnote{If $u_i$ loses the connection in Step 2 and cannot share $\tau_i^k$ with SSS, then there is no way to rank $u_i$'s position because the server has no $u_i$'s information. So we only consider the users who establish the connections to share $\tau_i^k$ in Step 2 for ranking.}

\vspace{+.5em}
%\textbf{Theorem 5.3}. To finish Step 3-5, there needs at least one user online for each group. Suppose that every user has $p_l$ probability to lose connection and there are totally $n$ users, the success probability $\ge (1-p_l^{\lfloor n/(2t+1) \rfloor})^{2t+1}$.

\noindent \textbf{Proof of Theorem 5.3}. For Step 3 to 5, if there is at least one user in every group, then the process can continue. So the probability of failure incurred by one specific group $g$ is all the users in $g$ losing the connections, i.e., $p^{n_g} \le p_l^{\lfloor n/(2t+1) \rfloor}$ ($n_g$ is the user number in $g$). So for $g$, the probability of at least one user online $\ge 1-p_l^{\lfloor n/(2t+1) \rfloor}$. With $2t+1$ groups, the success probability $\ge (1-p_l^{\lfloor n/(2t+1) \rfloor})^{2t+1}$.$\qed$
%For Step 6 to 8, if at least $t'$ users have connections, then the server can recover $h_i$. Hence, the probability of failure is more than $N-t'$ users losing connections:
%$$p^{N-t'+1}$$

\vspace{+.5em}
%\textbf{Theorem 5.4}. To finish Step 6-8, $\ge t+1$ users need to be online.

\noindent \textbf{Proof of Theorem 5.4}. This is based on the property of $(t+1, n)$-SSS in Step 5.$\qed$


\vspace{+.5em}
%\textbf{Theorem 5.5} If there are no more than $t$ collusive participants, then these participants cannot recover all the other users' $\tau_i$.

\noindent \textbf{Proof of Theorem 5.5}. In Step 2, $\tau_i^k (k=1...2t+1)$ is shared with $(t+1, 2t+1)$-SSS. So, if $t$ participants collude, they can get at most $t\cdot(2t+1)$ equations when $t$ participants are from $t$ groups. However, the number of unknown parameters (including $\tau_i$ and $t$ random coefficients for sharing each $\tau_i^k$) is $t\cdot(2t+1)+1$. Hence, these $t$ collusive participants cannot recover other participants' $\tau_i$.$\qed$

\subsection{Mechanism Extension to Multi-class and Continuous-value Events}



\textbf{Multi-class Events}. For a multi-class event ($m$ classes), we can see it as $m$ binary events, so that our method can be directly applied.

\noindent \textbf{Continuous-value Events}. For continuous-value events, following the literature, we may adopt other proper event confidence and participant trustworthiness updating functions such as CRH \cite{Xu2019EfficientAP,Zheng2020PrivacyAwareAE}. Specifically, suppose that the discovered truth sensed value of a continuous event $e_j$ is $\rho_j$, and $u_i$'s sensed data of $e_j$ is $\hat \rho_{ij}$, then the event truth (confidence) and participant trustworthiness updating functions  can be:
\begin{equation}
	\rho_j = \frac{\sum_{u_i \in \mathcal U_{e_j}}\tau_i \cdot \hat \rho_{ij}}{\sum_{u_i \in \mathcal U_{e_j}}\tau_i}
	\label{eq:rho_function_cont}
\end{equation}
\begin{equation}
	\tau_i = \log(\sum_{u_i \in \mathcal U} \sum_{e_j \in \mathcal E_{u_i}} \frac{(\rho_j- \hat \rho_{ij})^2}{|\mathcal E_{u_i}|}) - \log(\sum_{e_j \in \mathcal E_{u_i}} \frac{(\rho_j- \hat \rho_{ij})^2}{|\mathcal E_{u_i}|})
	\label{eq:tau_function_cont}
\end{equation}
where $\mathcal U_{e_j}$ is the set of users who sense $e_j$, and $\mathcal E_{u_i}$ is the set of events that $u_i$ has sensed. For $\rho$-computation, following Sec.~\ref{sub:basic_rho_computation}, we can just adapt $d_{ij}$ and $s_{ij}$ according to Eq.~\ref{eq:rho_function_cont} (the participant $u_i \not \in \mathcal U_{e_j}$ can still send $d_{ij}=s_{ij}=0$ to protect her task completion information). For $\tau$-computation, Eq.~\ref{eq:tau_function_cont} requires $\sum_{u_i \in \mathcal U} \sum_{e_j \in \mathcal E_{u_i}} \frac{(\rho_j- \hat \rho_{ij})^2}{|\mathcal E_{u_i}|}$, which can be done with the same SSS-based method as $\rho$-computation. In particular, each participant $u_i$ can send $\sum_{e_j \in \mathcal E_{u_i}} \frac{(\rho_j- \hat \rho_{ij})^2}{|\mathcal E_{u_i}|}$ by secret shares, and then the server can compute the sum in a privacy-preserving manner. In a word, for continuous-value events, our mechanism can still work without revealing each participant's raw sensed data and completed tasks.




% %


%%%%%%%%%%%%%%%%%%%%%%%%%%%%%%%%%%%%%%%%%%%%%%%%%%%%%%%%%%%%%%%%%%%%%%%%%%%%%%%
%%%%%%%%%%%%%%%%%%%%%%%%%%%%%%%%%%%%%%%%%%%%%%%%%%%%%%%%%%%%%%%%%%%%%%%%%%%%%%%


\end{document}


% This document was modified from the file originally made available by
% Pat Langley and Andrea Danyluk for ICML-2K. This version was created
% by Iain Murray in 2018. It was modified from a version from Dan Roy in
% 2017, which was based on a version from Lise Getoor and Tobias
% Scheffer, which was slightly modified from the 2010 version by
% Thorsten Joachims & Johannes Fuernkranz, slightly modified from the
% 2009 version by Kiri Wagstaff and Sam Roweis's 2008 version, which is
% slightly modified from Prasad Tadepalli's 2007 version which is a
% lightly changed version of the previous year's version by Andrew
% Moore, which was in turn edited from those of Kristian Kersting and
% Codrina Lauth. Alex Smola contributed to the algorithmic style files.
