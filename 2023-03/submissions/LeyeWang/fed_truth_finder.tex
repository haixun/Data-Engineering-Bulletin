\documentclass[11pt]{article}
\usepackage{subcaption}

\usepackage[utf8]{inputenc}
\usepackage{deauthor}
\usepackage{times,graphicx}
\usepackage{url}

% user packages
\usepackage{todonotes}
\usepackage{pifont}
\newcommand{\cmark}{\ding{51}}
\newcommand{\xmark}{\ding{55}}
\usepackage{multirow}
\usepackage{booktabs}
\usepackage{caption,subcaption}
\usepackage{graphicx}
\usepackage{natbib}
\usepackage{amsmath}

\title{Federated Truth Discovery for Mobile Crowdsensing with Privacy-Preserving Trustworthiness Assessment}

\author{Leye Wang$^{1,2}$, Guanghong Fan$^{1,2}$, Xiao Han$^{3,}$\footnote{Corresponding author} \\
 \small $^1$Key Lab of High Confidence Software Technologies (Peking University), Ministry of Education, China \\
\small$^2$School of Computer Science, Peking University, China\\
\small$^3$School of Information Management and Engineering, Shanghai University of Finance and Economics, China\\
\texttt{\small leyewang@pku.edu.cn, fgh@stu.pku.edu.cn, xiaohan@mail.shufe.edu.cn}
}

\begin{document}
	
\maketitle

\begin{abstract}
With the prevalence of smart mobile devices empowered by considerable sensing capabilities, crowdsensing has become one promising way to sense urban phenomena (e.g., traffic and environment) at a large scale. In crowdsensing, a fundamental issue is discovering the truth from participants' noisy sensed data. Traditionally, participants need to upload their raw sensed data with locations for truth discovery, but this may leak participants' private information such as home and work locations. In this paper, we propose a federated truth discovery method that can learn the truth without collecting participants' sensed data and locations. Our method ensures that the obtained truth quality has no performance loss compared to the original truth discovery method if all the participants keep online; even if some participants lose connections unpredictably, our method can still learn the truth based on rest participants' data.
Meanwhile, as participants' sensed data are unknown to the server, it is hard for the crowdsensing organizer to justify each participant's sensing trustworthiness. This brings difficulties to crowdsensing management such as participant recruitment and incentive allocation. We further propose a federated ranking mechanism to generate a leader-board for participants' trustworthiness, which can also tolerate participants' connection loss. Both theoretical analysis and real-data empirical evaluations have been done to verify the effectiveness of FedTruthFinder.

\end{abstract}


\section{Introduction}
\label{seC:intro}
% Interpretability and nutritional labels, generally

An essential ingredient of successful machine-assisted decision-making, particularly in high-stakes decisions, is interpretability --– allowing humans to understand, trust and, if necessary, contest, the computational process and its outcomes.   These decision-making processes are typically complex:  carried out in multiple steps, employing models with many hidden assumptions, and relying on datasets that are often repurposed --- used outside of the original context for which they were intended.\footnote{See Section 1.4 of Salganik's ``Bit by Bit''~\cite{salganik} for a discussion of data repurposing in the Digital Age, which he aptly describes as "mixing readymades with custommades.''}  In response, humans need to be able to determine the ``fitness for use'' of a given model or dataset, and to assess the methodology that was used to produce it.  

To address this need, we propose to develop interpretability and transparency tools based on the concept of a {\em nutritional label}, drawing an analogy to the food industry, where simple, standard labels convey information about the ingredients and production processes. Short of setting up a chemistry lab, the consumer would otherwise have no access to this information. Similarly, consumers of data products cannot be expected to reproduce the computational procedures just to understand fitness for their use.   Nutritional labels, in contrast, are designed to support specific decisions by the consumer rather than completeness of information.  A number of proposals for hand-designed nutritional labels for data, methods, or both have been suggested in the literature\cite{DBLP:journals/corr/abs-1803-09010,DBLP:journals/corr/abs-1805-03677,DBLP:conf/fat/MitchellWZBVHSR19}; we advocate deriving such labels automatically or semi-automatically as a side effect of the computational process itself, embodying the paradigm of {\em interpretability-by-design}. 

Interpretability means different things to different stakeholders, including individuals being affected by decisions, individuals making decisions with the help of machines, policy makers, regulators, auditors, vendors, data scientists who develop and deploy the systems, and members of the general public.  Designers of nutritional labels must therefore consider {\em what} they are explaining,  {\em to whom}, and {\em for what purpose}.  In the remainder of this section, we will briefly describe two regulatory frameworks that mandate interpretability of data collection and processing to members of the general public, auditors, and regulators,  where nutritional labels offer a compelling solution (Section~\ref{sec:intro:reg}).  We then discuss interpretability requirements in data sharing, particularly when data is altered to protect privacy or mitigate bias (Section~\ref{sec:intro:synth}).

\subsection{Regulatory Requirements for Interpretability}
\label{sec:intro:reg}

The European Union recently enacted a sweeping regulatory framework known as the General Data Protection Regulation, or the GDPR~\cite{gdpr}.  The regulation was adopted in April 2016, and became enforceable about two years later, on May 25, 2018.  The GDPR aims to protect the rights and freedoms of natural persons with regard to how their personal data is processed, moved, and exchanged (Article 1).  The GDPR is broad in scope, and applies to ``the processing of personal data wholly or partly by automated means'' (Article 2), both in the private sector and in the public sector.  Personal data is broadly construed, and refers to any information relating to an identified or identifiable natural person, called the {\em data subject} (Article 4).  

According to Article 4, lawful processing of data is predicated on the data subject's {\em informed consent}, stating whether their personal data can be used, and for what purpose (Articles 6, 7).
Further,  data subjects have {\em the right to be informed} about the collection and use of their data.~\footnote{\url{https://gdpr-info.eu/issues/right-to-be-informed/}}
Providing insight to data subjects about the collection and use of their data requires technical methods  that support interpretability.  

Regulatory frameworks that mandate interpretability are also starting to emerge in the US.  New York City was the first US municipality to pass a law (Local Law 49 of 2018)~\cite{Vacca}, requiring that a task force be put in place to survey the current use of ``automated decision systems'' (ADS) in city agencies. ADS are defined as ``computerized implementations of algorithms, including those derived from machine learning or other data processing or artificial intelligence techniques, which are used to make or assist in making decisions.''   The task force is developing recommendations for enacting algorithmic transparency by the agencies, and will propose procedures for: (i) requesting and receiving an explanation of an algorithmic decision affecting an individual (Section 3 (b) of Local Law 49); (ii) interrogating ADS for bias and discrimination against members of legally protected groups, and addressing instances in which a person is harmed based on membership in such groups (Sections 3 (c) and (d)); (iii) and assessing how ADS function and are used, and archiving the systems together with the data they use (Sections 3 (e) and (f)).

Other government entities in the US are following suit.  Vermont is convening an Artificial Intelligence Task Force to ``... make recommendations on the responsible growth of Vermont’s emerging technology markets, the use of artificial intelligence in State government, and State regulation of the artificial intelligence field.''~\cite{Vermont}.  Idaho’s legislature has passed a law that eliminates trade secret protections for algorithmic systems used in criminal justice~\cite{Idaho}.  In early April 2019, Senators Booker and Wyden introduced the Algorithmic Accountability Act of 2019 to the US Congress~\cite{BookerWydenClarke}. The Act, if passed, would use ``automated decision systems impact assessment'' to address and remedy harms caused by algorithmic systems to federally protected classes of people. The act empowers the Federal Trade Commission to issue regulations requiring larger companies to conduct impact assessments of their algorithmic systems.

The use of nutritional labels in response to these and similar regulatory requirements can benefit a variety of stakeholders.  The designer of a data-driven algorithmic method may use them to validate assumptions, check legal compliance, and tune parameters.  Government agencies may exchange labels to coordinate service delivery, for example when working to address the opioid epidemic, where  at least three sectors must coordinate: health care, criminal justice, and emergency housing, implying a global optimization problem to assign resources to patients effectively, fairly and transparently. The general public may review labels to hold agencies accountable to their commitment to equitable resource distribution. 


\subsection{Interpretability with Semi-synthetic Data}
\label{sec:intro:synth}

%Datasets are now increasingly used to train models to make decisions once made by humans.  In these automated systems, biases in the data are propagated and amplified with no human in the loop.  The bias, and the effect of the bias on the quality of decisions made, is not easily detectable due to the relative opacity of the system.  

A central issue in machine-assisted decision-making is its reliance on historical data, which often embeds results of historical discrimination, also known as {\em structural bias}.   As we have seen time and time again, models trained on data will appear to work well, but will silently and dangerously reinforce discrimination~\cite{propublicaJ,amazon_hiring,amazon_delivery}.  Worse yet, these models will legitimize the bias --- ``the computer said so.''  Nutritional labels for data and models are designed specifically to mitigate the harms implied by these scenarios, in contrast to the more general concept of ``data about data.''

Good datasets drive research: they inform new methods, focus attention on important problems, promote a culture of reproducibility, and facilitate communication across discipline boundaries.  But research-ready datasets are scarce due to the high potential for misuse. Researchers, analysts, and practitioners therefore too often find themselves compelled to use the data they have on hand rather than the data they would (or should) like to use.  For example, aggregate usage patterns of ride hailing services may overestimate demand in early-adopter (\ie wealthy) neighborhoods, creating a feedback loop that reduces service in poorer neighborhoods, which in turn reduces usage.  In this example, and in many others, there is a need to alter the input dataset to achieve specific properties in the output, while preserving all other relevant properties.  We refer to such altered datasets as \textit{semi-synthetic}.

Recent examples of methods that produce semi-synthetic data include database repair for causal fairness~\cite{DBLP:conf/sigmod/SalimiRHS19}, database augmentation for coverage enhancement~\cite{DBLP:conf/icde/AsudehJJ19}, and privacy-preserving and bias-correcting data release~\cite{DBLP:conf/ssdbm/PingSH17,DBLP:conf/vldb/RodriguezSPSH18}. A semi-synthetic datasets may be altered in different ways.  Noise may be added to it to protect privacy, or statistical bias may be removed or deliberately introduced.  When a dataset of this kind is released, its composition and the process by which it was derived must be made interpretable to a data scientist, helping determine fitness for use.  For example, datasets repaired for racial bias are unsuitable for studying discrimination mitigation methods, while datasets with bias deliberately introduced are less appropriate for research unrelated to fairness.   This gives another compelling use case for nutritional labels.

%To make our discussion more concrete, let us consider data scientists who must identify datasets appropriate for their task.  This is particularly important when semi-synthetic datasets are being released, to which noise is added to protect privacy, or statistical bias is removed or deliberately introduced.  For example, datasets repaired for racial bias are unsuitable for studying discrimination mitigation methods, while datasets with bias deliberately introduced are less appropriate for research unrelated to fairness.  



\section{Preliminary: Truth Discovery}
\label{sec:preliminary}

\begin{figure}
	\centering
	\includegraphics[width=.5\linewidth]{submissions/LeyeWang/fig/truthfinder.pdf}
	\caption{Overview of Iterative Truth Discovery}
	\label{fig:truthfinder}
	\vspace{-1em}
\end{figure}


Truth discovery algorithms usually follow an iterative method to calibrate user trustworthiness and data confidence alternatively until convergence \citep{yin2008truth}. Figure~\ref{fig:truthfinder} shows the framework of iterative truth discovery methods. In this paper, for clarity, we assume that sensed data is a binary spatial event. That is, for a specific location, the sensed data can be 1 or 0. Our method can be easily extended to multi-class and continuous-value events (see Appendix).

As shown in Figure~\ref{fig:truthfinder}, first, participants upload all of their sensed data and locations $\mathcal E_i$  to the central server. The central server would assign an initial trustworthiness score $\tau_i$ to each participant $u_i$ (e.g., 0.9 by assuming that 90\% of the sensed data are accurate). Then, for each sensed event $e_j$, the truth discovery algorithm will calculate its confidence $\rho_j$ (i.e., the probability of $e_j = 1$) by considering the users who have sensed $e_j$ as:
\begin{equation}
\rho_j = F_\rho(\mathcal U_{j,1}, \mathcal U_{j,0})
\label{eq:event_confidence}
\end{equation}
where $\mathcal U_{j,k}$ is the users who have sensed the event $e_j$ with the reported data $k$; $F_\rho$ is an event confidence calculation function which we will elaborate on later.

With $\rho_j$ for each event $e_j$, we can then update the trustworthiness score $\tau_i$ of each participant $u_i$ by:
\begin{equation}
\tau_i = F_\tau(\mathcal E_{i,1}, \mathcal E_{i,0})
\end{equation}
where $\mathcal E_{i,k}$ is the users' sensed event set with the reported data $k$; $F_\tau$ is a user trustworthiness calculation function which we will elaborate on later.

Once $\tau_i$ is updated for each user $u_i$, we can continue updating $\rho_j$ for each event $e_j$ according to Eq.~\ref{eq:event_confidence}, and so on, leading to an alternative updating process for both $\tau_i$ and $\rho_j$. This process can be terminated after a fixed number of iterations or until convergence. Next, we elaborate on the common choices of $F_\rho$ and $F_\tau$ in literature.

\textbf{Sum Function}

An intuitive selection of the updating functions of $F_\rho$ and $F_\tau$ is the weighted sum:
\begin{equation}
\rho_j = F_\rho(\mathcal U_{j,1}, \mathcal U_{j,0}) = \frac{\sum_{u_i \in \mathcal U_{j,1}} \tau_i}{\sum_{u_i \in \mathcal U_{j,1}} \tau_i + \sum_{u_k \in \mathcal U_{j,0}} \tau_k}
\label{eq:rho_function_sum}
\end{equation}
\begin{equation}
\tau_i = F_\tau(\mathcal E_{i,1}, \mathcal E_{i,0}) = \frac{\sum_{e_j \in \mathcal E_{i,1}} \rho_j + \sum_{e_k \in \mathcal E_{i,0}} 1-\rho_k}{|\mathcal E_{i,1}|+|\mathcal E_{i,0}|}
\label{eq:tau_function}
\end{equation}

\textbf{Logistic Function}

Another widely used updating function is the Logistic function \citep{yin2008truth}. Its basic idea is seeing every user independently, so that the probability of event happening, i.e., $e_j=1$, can be formulated as:
\begin{equation}
	\rho_j = 1 - \prod_{u_i \in \mathcal U_{j,1}} (1-\tau_i)
\end{equation}
As $1-\tau_i$ may often be small and multiplying many of them may lead to underflow, prior studies proposed to use the logarithm to define a log-trustworthiness score of $u_i$ as \citep{yin2008truth}:
\begin{equation}
	\tau_i^* = - \ln(1-\tau_i)
\end{equation}
Similarly, a log-confidence score of event $e_j$ is defined as:
\begin{equation}
	\rho_j^* = - \ln(1-\rho_i)
\end{equation}
Then, we can infer
\begin{equation}
	\rho_j^* = \sum_{u_i \in \mathcal U_{j,1}} \tau_i^*
\end{equation}
The above equation does not consider the users' trustworthiness who report $e_j=0$, and thus we refine it:
\begin{equation}
	\rho_j^* = \sum_{u_i \in \mathcal U_{j,1}} \tau_i^* - \sum_{u_k \in \mathcal U_{j,0}} \tau_k^*
\end{equation}
Finally, a logistic function is used to calculate the final confidence $\rho_j$ of event $e_j$ \citep{yin2008truth}:
\begin{equation}
	\rho_j = F_\rho(\mathcal U_{j,1}, \mathcal U_{j,0}) = (1+e^{-\rho_j^*})^{-1}
	\label{eq:rho_function_log}
\end{equation}
$\tau_i$ is updated same as Eq.~\ref{eq:tau_function}. 
\section{Method}

Here we discuss the standard retrieve-and-rerank (R\&R) framework for IR (\S{\ref{sec:retrieve_and_rerank}}) and how our proposal fits into it (\S{\ref{sec:cross_encoder_feedback}}). While our approach can be applied to any R\&R framework, we consider a text-based retriever and reranker for simplicity while elaborating our method. A multi-modal R\&R framework is described in \S\ref{sec:multimodal_results}.


\subsection{Retrieve-and-Rerank}
\label{sec:retrieve_and_rerank}
R\&R for IR consists of a first-stage retriever and a second-stage reranker. Modern neural approaches typically use a dual-encoder model as the retriever and a cross-encoder for reranking.  

\paragraph{\textbf{The Retriever}:} The dual-encoder retriever model is based on a Siamese neural network \cite{chicco2021siamese}, containing separate Bert-based \cite{devlin2019bert} encoders $E_Q(.)$ and $E_P(.)$ for the query and the passage, respectively.
Given a query $q$ and a passage $p$, a separate representation is obtained for each, such as the \textsc{cls} output or a pooled representation of the individual token outputs from $E_Q(q)$ and $E_P(p)$. The question-passage similarity $sim(q,p)$ is computed as the dot product of their corresponding representations: query/passage.}
\begin{equation}
    Q_q = Pool(E_Q(q))
\end{equation}
\begin{equation}
    P_p = Pool(E_P(p))
\end{equation}
\begin{equation}\label{eq:sim}
   sim(q,p) = S(Q_q,P_p) = Q_q^TP_p
\end{equation}

Since Eq.~\ref{eq:sim} is decomposable, the representations of all passages in the retrieval corpus can be pre-computed and stored in a dense index \cite{johnson2019billion}. During inference, given a new query, the top $K$ most relevant passages are retrieved from the index via approximate nearest-neighbor search.

\paragraph{\textbf{The Reranker}:} The cross-encoder reranker model uses a Bert-based encoder $E_R(.)$, which takes the query $q$ and a corresponding retrieved passage $p$ together as input and outputs a similarity score. 
A feed-forward layer $F$ is used on top of the \textsc{cls} output from $E_R(.)$ to compute a single logit, which is used as the final reranker score $R(q,p)$. The top $K$ retrieved passages are then ranked based on their corresponding reranker scores.

\begin{equation}
   R(q,p) = F(CLS(E_R(q,p))
\end{equation}


\begin{algorithm}[t]
\caption{\textsc{\textbf{ReFIT}}}
\label{alg4}
\begin{flushleft}
\textbf{Input}: Query $q$ and its representation $Q_q$, retrieved passages $P$ and their representations $\hat{P}$.\newline
\textbf{Output}: Updated query representation $Q_{q,n}$
\end{flushleft}
\begin{algorithmic}[1]
    \State Initialize query vector $Q_{q,0}$ = $Q_q$
    \State Compute reranker distribution $D_{CE}(q,P)$ (Eq.~\ref{eq:d-ce})
    \For{\textit{i in 0 to n}}
        \State Compute retriever distribution $D_{Q_{q,i}}(\hat{P})$ (Eq.~\ref{eq:d-q})
        \State Compute loss $\mathcal{L}$ (Eq.~\ref{eq:loss})
        \State Update $Q_{q,i+1} = Q_{q,i} - \alpha \frac{\partial}{\partial Q_{q,i}}\mathcal{L}$
    \EndFor
    \State return $Q_{q,n}$
\end{algorithmic}
%\vspace{-0.4em}
\end{algorithm}

\subsection{Reranker Relevance Feedback}
\label{sec:cross_encoder_feedback}
The main idea underlying our proposal is to compute an improved query representation for the retriever using feedback from the more powerful reranker.
More specifically, we perform a lightweight inference-time distillation of the reranker's knowledge into a new query vector.

Given an input query $q$ during inference, we use the following output provided by the R\&R pipeline:
\begin{itemize}
   \item Query representation $Q_q$ from the retriever.
    \item Retrieved passages $P = \{p_1, p_2,  ..., p_K\}$ and their representations $\hat{P} = [P_{p_1}, P_{p_1},  ..., P_{p_K}]$ from the retriever. 
    \item The reranking scores $R(q,P) = [R(q,p_1),..., R(q,p_K)]$.
\end{itemize}
Note that $\hat{P}$ above is directly obtained from the passage index and is not computed during inference.

The proposed reranker feedback mechanism begins with using the reranking scores $R(q,P)$ to compute a cross-encoder ranking distribution $D_{CE}(q,P)$ over passages $P$ as follows:

\begin{equation}
D_{CE}(q,P)=\mathrm{softmax}([R(q,p_1), ..., R(q,p_K)])
\label{eq:d-ce}
\end{equation} 

The query and passage representations from the retriever are used to compute a similar distribution $D_{Q_q}(\hat{P})$ over $P$:

\begin{equation}
    D_{Q_q}(\hat{P}) = \mathrm{softmax}([Q_q^TP_{p_1}, ..., Q_q^TP_{p_K}])
    \label{eq:d-q}
\end{equation}

Next, we compute the loss as the KL-divergence between the retriever and reranker distributions:

\begin{equation}
    \mathcal{L} = D_{KL}(D_{CE}(q,P) || D_{Q_q}(\hat{P}))
    \label{eq:loss}
\end{equation}

which is then used to update the query vector via gradient descent. The query vector update process is repeated for $n$ times, where $n$ is a hyper-parameter. 
A schematic description of the process can be found in Algorithm \ref{alg4}. 

Finally, the updated query vector $Q_{q,n}$ is used for a second-stage retrieval from the passage index.  
From dual-encoder retrieval with the updated $Q_{q,n}$, we aim to achieve better recall than with the initial $Q_q$, while obtaining a ranking performance that is comparable with that of the reranker.







\section{Federated Trustworthiness Rank}
\label{sec:trust_ranking}

While FedTruthFinder learns the integrated event truth in a privacy-preserving manner, it brings a challenge in justifying participants' trustworthiness. For example, to incentivize the crowdsensing participants, it is a common strategy to pay the high-trustworthy participants (i.e., high-quality sensing results) with higher incentives. However, in FedTruthFinder, the sensing quality of each participant, i.e., the trustworthiness score $\tau_i$ is kept at each participant side and unknown to the server. Hence, how to assess participants' trustworthiness is required and challenging for FedTruthFinder.

In this section, we first illustrate a concrete case to describe that $\tau_i$ cannot be directly known to the server, otherwise the server may infer which event $u_i$ has sensed. As $\tau_i$ cannot be known to the server, we then design a secure ranking algorithm to let the server know every participant $u_i$'s ranking position of $\tau_i$ among all the participants without leaking $\tau_i$. Based on the ranked positions, the crowdsensing organizer can enable certain trustworthiness-aware incentive mechanisms, e.g., rewarding high-position participants with bonus, which can incentivize participants to compete with each other to get more high-quality sensed data \citep{Reddy2010ExaminingMF}.

\subsection{Privacy Leakage by Trustworthiness $\tau_i$}

Here, we illustrate an example to show the risk of revealing $\tau_i$ to the server for leaking participant $u_i$'s privacy.

Without the loss of generality, we assume that $u_1$'s $\tau_1 = 0.9$, and other $u_i$'s $\tau_i < 0.9\ (i\not=1)$. Suppose that one event $e_j$'s $\rho_j = 0.9$ after truth discovery, then we can easily infer that $u_1$ has sensed the event $e_j$ and the sensed result is $1$. This reveals the fact that $u_1$ has visited the location of $e_j$, leaking $u_1$'s location privacy.

Hence, participants cannot directly upload their $\tau_i$ to the server for incentive allocation. Next, we design a privacy-preserving method to enable trustworthiness-aware incentive allocation. 

\subsection{Secure Trustworthiness Leader-board}

While revealing $\tau_i$ may leak participants' private information, we propose a secure ranking algorithm to learn a  leader-board regarding participants' trustworthiness for facilitating trustworthiness-aware incentive allocation.

Secure ranking algorithms have been studied for decades; however, prior studies cannot be directly applied in our scenario for two reasons. First, the communication overheads are usually high. Second, prior studies mostly assume that all the network connections are stable for all the parties, but this is unrealistic for crowdsensing. 

Our secure ranking algorithm generally follows the design of \citet{tang2011secure}. However, the original design \citep{tang2011secure} cannot tolerate any participants to lose the network connections. We thus enhance it to ensure that the ranking algorithm can still work when certain participants lose connections.
The major steps of our federated trustworthiness leader-board generation mechanism are:

\textbf{Step 1}. First, we categorize all the participants into $(2t+1)$ groups, and thus each group includes $n/(2t+1)$ participants. We denote $gid(u)$ to refer to the group ID of participant $u$.

\textbf{Step 2}. For each user $u_i$, she shares $\tau_i$, $\tau_i^2$, ... , $\tau_i^{2t+1}$ with $(t+1, 2t+1)$-SSS to all the user groups. Specifically, a user $u_j$ will receive the share piece regarding $gid(u_j)$, denoted as $\tau_{i_1}(gid(u_j))$, $\tau_{i_2}(gid(u_j))$, ... $\tau_{i_{2t+1}}(gid(u_j))$ for $\tau_i$, $\tau_i^2$, ... , $\tau_i^{2t+1}$, respectively.

\textbf{Step 3}. For each user group $g_k$, it generates a random number $r_k (>0)$ and shares $r_k$ with $(t+1, 2t+1)$-SSS to all the user groups. That is, $u_j$ will receive $r_k$'s share regarding $gid(u_j)$, denoted as $r_k(gid(u_j))$.

\textbf{Step 4}. For each participant $u_j$, she calculates the following number with the $\tau_{i_k}(gid(u_j))$ received from $u_i$:
\begin{align}
h_{i}(gid(u_j)) &= \lambda(gid(u_j)) \sum_{k=1}^{2t+1} r_k(gid(u_j)) \tau_{i_k}(gid(u_j)) \\ &= \lambda(gid(u_j)) \gamma(gid(u_j))
\end{align}
where
\[
\left(\begin{array}{ccccc} 
	1 &    1 & 1^2 & ...  & 1^{2t} \\ 
	1 &    2 & 2^2 & ... & 2^{2t}\\
	... & ... & ... & ...& ...\\
	1 & 2t+1 & (2t+1)^2 & ... & (2t+1)^{2t}\\
\end{array}\right)^{-1} 
\]
\[
=\left(\begin{array}{ccccc} 
	\lambda(1) &   \lambda(2) & \lambda(3) & ...  & \lambda(2t+1) \\ 
	... & ... & ... & ...& ...\\
	... & ... & ... & ...& ...\\
	... & ... & ... & ...& ...
\end{array}\right) 
\]


\textbf{Step 5}. For each user group, we randomly select one participant $u_j$ to share $\{h_{i}(gid(u_j))| i \in [1,n]\}$ with $(t+1, n)$-SSS to all the $n$ participants. Each user $u_k$'s received shares from all the groups are denoted as $\{h_i(g, k)| i \in [1,n], g \in [1, 2t+1]\}$.

\textbf{Step 6}. For each participant $u_k$, she computes:
\begin{equation}
h'_i(k) = \sum_{g=1}^{2t+1} h_i(g, k), \quad \forall i \in [1,n]
\end{equation}
Each $u_k$ sends $\{h'_i(k)| i\in[1,n]\}$ to the server.

\textbf{Step 7}. After receiving at least $t+1$ participants' $\{h'_i(k)| i\in[1,n]\}$, the server can recover:
\begin{equation}
	h_i = \sum_{k=1}^{2t+1} r_k \tau_i^k, \quad \forall i \in [1,n]
\end{equation}

\textbf{Step 8}. The server ranks $u_i$ according to $h_i$ and the ranked list is the leader-board regarding trustworthiness $\tau_i$.

Note that same as $\rho$-computation, we do not need to establish the peer-to-peer communication channels between every two participant clients and can use the crowdsensing server for coordination. To avoid redundancy, readers can refer to Sec.~\ref{subsub:server_coordination} for details.

\textbf{Remark on our novelty}. The key improvement of our secure ranking algorithm compared to \citet{tang2011secure} is the enhanced robustness against participants' connection loss. In \citet{tang2011secure}, every participant holds a $r_i$ and we will randomly select $2t+1$ participants to share their $r_i$ (Step 3) and $h_i$ (Step 5). This process is easy to break if a selected online user (Step 3) loses the connection in Step 5. Our proposed algorithm first constructs user groups so that we only need at least one participant online in each group for both Step 3 and 5, reducing the failure possibility incurred by connection loss. It is worth noting that this algorithm can not only rank crowdsensing participants' trustworthiness, but also be applied to many other applications when privacy-preserving data ranking is needed under unstable network connections.

\textbf{Remark on the ranked measurements}. In the previous algorithm description, we suppose that $\tau_i$ needs to be ranked. In practice, crowdsensing organizers can use the same secure ranking mechanism to rank other key measurements of participants (e.g., the number of sensed events) to design better incentive mechanisms or participant recruitment strategies.

\subsection{Theoretical Analysis}
\label{sub:theoretical_analysis_2}

All the proofs are illustrated in Appendix.

\subsubsection{Correctness} We first prove the correctness of our algorithm.

\vspace{+.5em}
\textbf{Lemma 5.1}. $\sum_{k=1}^{2t+1} r_k(x)\tau_{i_k}(x)$ can be represented as:
$$h_{i}+a_{i1}x+a_{i2}x^2+...+a_{i2t}x^{2t}$$
where $h_i=\sum_{k=1}^{2t+1} r_k\tau_i^k$. \citep{tang2011secure}
 
\vspace{+.5em}
\textbf{Theorem 5.1}. With $t+1$ participants' $h'_i(k)$, we can recover $h_i$.

\begin{figure*}[t!]%[tbhp]
	\centering
		\begin{subfigure}[t]{.325\linewidth}
			\includegraphics[width=1\linewidth]{./fig/data_num_cl0.01.PNG}
			\caption{$p_l=0.01$}
			\label{fig:cl0.01}
		\end{subfigure}
		\begin{subfigure}[t]{.325\linewidth}
			\includegraphics[width=1\linewidth]{./fig/data_num_cl0.05.PNG}
			\caption{$p_l=0.05$}
			\label{fig:chicago}
		\end{subfigure}
		\begin{subfigure}[t]{.325\linewidth}
			\includegraphics[width=1\linewidth]{./fig/data_num_cl0.1.PNG}
			\caption{$p_l=0.1$}
			\label{fig:dc}
		\end{subfigure}
		\caption{Number of data for each event's truth discovery by iterations.}
		\label{fig:num_sensed_data}
\end{figure*}

\begin{figure*}
	\centering
		\begin{subfigure}[t]{.3\linewidth}
			\includegraphics[width=1\linewidth]{./fig/failure_cl0.05.PNG}
			\caption{$p_l=0.05$}
		\end{subfigure}
		\begin{subfigure}[t]{.3\linewidth}
			\includegraphics[width=1\linewidth]{./fig/failure_cl0.1.PNG}
			\caption{$p_l=0.1$}
		\end{subfigure}
		\caption{Failure probability of truth discovery.}
		\label{fig:failure}
\end{figure*}

\vspace{+.5em}
\textbf{Theorem 5.2}. Ranking $h_i$ is equivalent to ranking $\tau_i$.


\subsubsection{Robustness to Connection Loss} We analyze how our secure ranking algorithm can tolerate connection losses. We assume that before Step 2, there is no user connection loss.\footnote{If $u_i$ loses the connection in Step 2 and cannot share $\tau_i^k$ with SSS, then there is no way to rank $u_i$'s position because the server has no $u_i$'s information.}

\vspace{+.5em}
\textbf{Theorem 5.3}. To finish Step 3-5, there needs at least one user online for each group. Suppose that every user has $p_l$ probability to lose connection and there are $n$ users, the success probability $\ge (1-p_l^{\lfloor n/(2t+1) \rfloor})^{2t+1}$.


\vspace{+.5em}
\textbf{Theorem 5.4}. To finish Step 6-8, $\ge t+1$ users need to be online.



\subsubsection{Security} Here, we analyze the security of our mechanism.

\vspace{+.5em}
\textbf{Theorem 5.5} If there are no more than $t$ collusive participants, then these participants cannot recover all the other users' $\tau_i$.






\subsubsection{Complexity} We analyze the algorithm from communication and computation complexity perspectives for participant clients.

\textbf{Communication Complexity - $O(tn)$}. In Step 2, the communication overhead of one participant to send $\tau_i,\tau_i^2,...,\tau_i^{2t+1}$ is $O(t^2)$, while each user received data is $O(tn)$. In Step 3, the complexity is $O(t)$. In Step 5, for sending data, the complexity is $O(n)$; for receiving data, the complexity is $O(tn)$. In Step 7, the complexity is $O(n)$. Combing them together, the communication complexity of the whole process is $O(tn)$ as $t<n$.


\textbf{Computation Complexity - $O(tn)$}. The main computation processes of each client include (1) calculating secret shares for $\tau_i,\tau_i^2,...,\tau_i^{2t+1}$ with $(t+1, 2t+1)$-SSS in Step 2, which is $O(t^2)$, (2) calculating secret shares of $r_k$ in Step 3, which is $O(t)$, (3) computing  $h_i$ in Step 4, which is $O(tn)$, and (4) calculating $h_i'$ in Step 6, which is $O(tn)$. Hence, the final computation complexity is $O(tn)$.

\section{Experiments}
\label{experiment}

In this section, we provide experimental results of FedAQ in homogeneous local data distribution settings. We compare FedAQ with other quantization-based federated optimization algorithms, FedPAQ \cite{reisizadeh2020fedpaq} and FedCOMGATE \cite{haddadpour2021federated}. FedAvg \cite{mcmahan2017communication} and FedAC \cite{yuan2020federated}, federated optimization algorithms without quantization, are also our baselines. We empirically validate the performance of 5 algorithms on classical classification tasks on MNIST\cite{lecun1998mnist} and CIFAR-10\cite{krizhevsky2009learning} datasets in the distributed learning environment. We consider three objective functions i) A strongly convex objective of $l_2$-regularized logistic regression model on the MNIST dataset, ii) A non convex objective of training a multilayer perceptron on the MNIST data, and iii) A non convex objective of training a convolution neural network (CNN) on the CIFAR-10 dataset. %The details of the implementation environment, datasets, training models, hyperparameter choices, quantization bits, and new time metric are elaborated in Appx.~\ref{app:experimental_setup}.

% \begin{figure*}[!htbp]
%     \centering
%     % Figure 0
%     \begin{subfigure}[b]{0.31\textwidth}
%     \includegraphics[width=\textwidth]{figure/loss_iid_comm_str_cvx.png}
%     %\caption{DCGAN}
%     \end{subfigure}
%     % Figure 1
%     \begin{subfigure}[b]{0.31\textwidth}
%     \includegraphics[width=\textwidth]{figure/loss_iid_bits_str_cvx.png}
%     %\caption{DCGAN}
%     \end{subfigure}
%     %\quad
%     % Figure 2
%     \begin{subfigure}[b]{0.31\textwidth}
%     \includegraphics[width=\textwidth]{figure/loss_iid_time_str_cvx.png}
%     %\caption{OKGAN}
%     \end{subfigure}

%     \setcounter{subfigure}{0}
%     % Figure 0
%     \begin{subfigure}[b]{0.31\textwidth}
%     \includegraphics[width=\textwidth]{figure/loss_iid_comm_localstep_100_2.png}
%     %\caption{DCGAN}
%     \end{subfigure}
%     % Figure 1
%     \begin{subfigure}[b]{0.31\textwidth}
%     \includegraphics[width=\textwidth]{figure/loss_iid_bits_localstep_100_2.png}
%     %\caption{DCGAN}
%     \end{subfigure}
%     %\quad
%     % Figure 2
%     \begin{subfigure}[b]{0.31\textwidth}
%     \includegraphics[width=\textwidth]{figure/loss_iid_time_localstep_100_2.png}
%     %\caption{OKGAN}
%     \end{subfigure}
%     \caption{Comparing FedAQ with FedAvg, FedPAQ, FedCOMGATE, and FedAC on MNIST with Strongly Convex Settings (first row) and Non-Convex Settings (second row). We observe how the global training loss changes across communication rounds (first column), communicated bits (second column), and human time (third column). FedAQ-I(8bits) and FedAQ(4bits) respectively outperform other algorithms for strongly convex settings and non-convex settings. FedAQ(4bits) sends the same number of communicated bits as FedPAQ(8bits) and FedCOMGATE(8bits) in each communication round, which indicates a fair comparison (See Quantization bits in Appx.~\ref{app:experimental_setup}).}
%     \label{graph_in_main_body}
% \end{figure*}

\subsection{Experimental Setup}
\label{experimental_setup}

\paragraph{Implementation Environment.} We follow the implementation setup in \cite{haddadpour2021federated}. We use the Distributed library of PyTorch to implement our algorithm because this library allows us to simulate real-world communication and distributed training. The 18 cores of Intel Xeon E5-2676 CPU are used as computing sources. Each core is considered as one local client. We use 16 cores for strongly convex MNIST, 18 cores for the non-convex MNIST, and 8 cores for the CIFAR-10. For MNIST, the strongly convex experiment and the non-convex one respectively run for 300 rounds of communication with 20 local updates and 50 rounds of communication with 100 local updates. The CIFAR-10 experiment runs for 100 rounds of communication with 100 local updates.

\paragraph{Datasets.} For image classification tasks, we choose two main classical image datasets: MNIST and CIFAR-10. Since we assume homogeneous settings, data is distributed homogeneously among clients, which also means each device has access to all 10 classes.

% \paragraph{Training Models.} For MNIST, we use a $l_2$-regularized logistic regression model for the strongly convex case and a multilayer perceptron (MLP) with two hidden layers for the non-convex case. For CIFAR-10, we use a Convolutional Neural Network (CNN). Here, we note that the number of parameters in a neural network model is directly related to the number of communicated bits. We discuss more on this in Appx.~\ref{app:NN_comm_bits}.

\paragraph{Hyperparameter Choice.} The important hyperparmeters in our experiments are learning rates for each algorithm. For the client learning rate $\eta$, we respectively use 0.002, 0.1, and 0.01 for strongly convex MNIST, non-convex MNIST, and CIFAR-10 for all algorithms. For FedAQ and FedAC, once we set the value of $\mu$, other hyperparameters ($\gamma, \alpha, \beta$) are automatically determined (See condition set (\ref{parameter_FedAQ}) and (\ref{parameter2_FedAQ})). Thus, we choose 0.1, 0.01, and 0.2 for $\mu$ value for strongly convex MNIST, non-convex MNIST, and CIFAR-10. Since too large $\mu$ leads to slow convergence and too small $\mu$ leads to unstable training, we get these $\mu$ values by tuning $\mu$ appropriately. FedCOMGATE has a server learning rate, and we set this value as 1 for all experiments.

\paragraph{Quantization Bits.} We have three quantization-based federated algorithms: FedAQ, FedPAQ, FedCOMGATE. We quantize the updates from 32 bits to 8 bits for all quantization-based algorithms in both MNIST and CIFAR-10. Additionally, particularly for FedAQ in non-convex experiments, we consider 4 bits quantization as well. Since FedAQ sends twice as many messages as FedPAQ or FedCOMGATE at every synchronization when we use 8 bits quantization for all cases, we apply 4 bits quantization to FedAQ to let FedAQ send the same amount of information in each communication round as other quantization-based algorithms for a fair comparison.

\paragraph{New Time Metric.} In our experiments, communication between CPU cores is very fast, so it is hard to say that the environment of our experiments fully reflects the real-world federated learning when there is a heavy communication burden. Thus, we use a linear model to estimate the execution time $T_{\textrm{round}}(\mathcal{A})$ between two consecutive communication rounds for real federated learning scenarios \cite{wang2021field}.
\begin{align*}
    &T_{\textrm{round}}(\mathcal{A}) = T_{\textrm{comm}}(\mathcal{A})+T_{\textrm{comp}}(\mathcal{A}), & &T_{\textrm{comm}}(\mathcal{A}) = \frac{S_{\textrm{down}(\mathcal{A})}}{B_{\textrm{down}}} + \frac{S_{\textrm{up}(\mathcal{A})}}{B_{\textrm{up}}} \\
    &T_{\textrm{comp}}(\mathcal{A}) = \max_j T_{\textrm{client}}^j(\mathcal{A}) + T_{\textrm{server}}(\mathcal{A}), & &T_{\textrm{client}}^j(\mathcal{A}) = R_{\textrm{comp}}T_{\textrm{sim}}^j (\mathcal{A}) + C_{\textrm{comp}}
\end{align*}
Since $T_{\textrm{server}}(\mathcal{A})$ is relatively smaller than $T_{\textrm{client}}^j(\mathcal{A})$, we ignore $T_{\textrm{server}}(\mathcal{A})$ in our experiments. We get client download size $S_{\textrm{down}(\mathcal{A})}$ and upload size $S_{\textrm{up}(\mathcal{A})}$ from the number of neural network parameters. $\max_j T_{\textrm{sim}}^j(\mathcal{A})$ is the computation time in our simulation.
\begin{align*}
    B_{\textrm{down}} \sim 0.75 \textrm{MB/secs},\textrm{ } B_{\textrm{up}} \sim 0.25 \textrm{B/secs},\textrm{ } R_{\textrm{comp}} \sim 7,\textrm{ } C_{\textrm{comp}} \sim 10 \textrm{secs}
\end{align*}
\cite{wang2021field} estimate each value of the above parameters from a real world cross-device FL system. The upload bandwidth $B_{\textrm{up}}$ is generally smaller than download bandwidth $B_{\textrm{down}}$. We define human time as the parallel time estimated by this new time metric.

\subsubsection{Training Models}

For MNIST, we use a $l_2$-regularized logistic regression model for the strongly convex case and a multilayer perceptron (MLP) with two hidden layers for the non-convex case. For CIFAR-10, we use a Convolutional Neural Network (CNN). Here, we note that the number of parameters in a neural network model is directly related to the number of communicated bits. We discuss more details as follows.

\paragraph{MLP Model for MNIST.} We use a multilayer perceptron (MLP) with two hidden layers. Each hidden layer consists of 200 neurons with ReLU activations. Thus, we compute the total number of parameters in this MLP model as below.
\begin{align*}
    (\# \textrm{ of MLP parameters) } &= (\# \textrm{ of input features) } \times (\# \textrm{ of neurons in the 1st layer}) \\
    &+ (\# \textrm{ of neurons in the 1st layer) } \times (\# \textrm{ of neurons in the 2nd layer}) \\
    &+ (\# \textrm{ of neurons in the 2nd layer) } \times (\# \textrm{ of MNIST classes}) \\
    &+ (\# \textrm{ of neurons in the 1st layer) } + (\# \textrm{ of neurons in the 2nd layer) } \\
    &+ (\# \textrm{ of MNIST classes}) \\
    &= 28 \times 28 \times 200 + 200 \times 200 + 200 \times 10 + 200 + 200 + 10 = 199210
\end{align*}
Finally, we derive $S_\textrm{up}(\mathcal{A}) (= S_\textrm{down}(\mathcal{A})$), defined in \cref{experimental_setup} (New time metric), by using the above fact. We use 32 bits floating-point if there is no quantization.
\begin{align*}
    S_\textrm{up}(\mathcal{A}) &= (\# \textrm{ of device) } \times (\# \textrm{ of MLP parameters) } \times (\# \textrm{ of bits)} \\
    &= 18 \times 199210 \times 32 = 114744960
\end{align*}
The FedAvg algorithm follows the above calculation. If we use 8 bits quantization for FedPAQ, FedCOMGATE, and FedAQ, ($\#$ of bits) in the above equation will respectively be  8, 8, and 16. Since FedAQ sends twice as many messages as others at every communication round, ($\#$ of bits) for FedAQ is 16. Similarly, ($\#$ of bits) for FedAC, which has no quantization, is 64.

\paragraph{CNN Model for CIFAR-10.} We use a CNN model, which consists of two 2-dimensional convolutional layers, two max pooling layers, and two fully connected layers. The ReLU activations are used in this CNN model. Let's clarify ($\#$ of input channel, $\#$ of output channel, kernel size, stride) for convolutional layers. We respectively use (3, 20, 5, 1), (20, 50, 5, 1) for the 1st and 2nd convolutional layer. Let's denote each convolutional layer and fully connected layer as CONV1, CONV2, FC3, FC4. At first, the activation shape of input layer for CIFAR-10 is (32, 32, 3). Then, we get the activation shape after CONV1 and the number of parameters for CONV1.
\begin{align*}
    (\textrm{width of activation shape) } &= \frac{\textrm{(width of previous activation shape) } - \textrm{kernel size} + 1}{\textrm{stride}} \\
    &= \frac{32-5+1}{1} = 28 \textrm{ } \Rightarrow \textrm{ activation shape} = (28, 28, 20) \\
    (\# \textrm{ of CONV1 parameters) } &= \Big(\textrm{kernel size } \times \textrm{ kernel size } \\
    &\times (\# \textrm{ of filters in the previous layer) }+1 \Big) \\
    &\times (\# \textrm{ of filters in the current layer}) \\
    &= (5 \times 5 \times 3 + 1) \times 20 = 1520
\end{align*}
The activation shape becomes (14, 14, 20) after max pooling. There are no learnable parameters in pooling layers. We do similar calculation for CONV2.
\begin{align*}
    (\textrm{width of activation shape) } &= \frac{\textrm{(width of previous activation shape) } - \textrm{kernel size} + 1}{\textrm{stride}} \\
    &= \frac{14-5+1}{1} = 10 \textrm{ } \Rightarrow \textrm{ activation shape} = (10, 10, 50) \\
    (\# \textrm{ of CONV2 parameters) } &= \Big(\textrm{kernel size } \times \textrm{ kernel size } \times (\# \textrm{ of filters in the previous layer) }\\
    &+1\Big) \times (\# \textrm{ of filters in the current layer}) \\
    &= (5 \times 5 \times 20 + 1) \times 50 = 25050
\end{align*}
The activation shape becomes (5, 5, 50) after second max pooling. Then, we calculate the number of parameters in FC3 and FC4 similar to the MLP case.
\begin{align*}
    (\# \textrm{ of FC3 parameters }) &= (5 \times 5 \times 50) \times 512 + 512 = 640512 \\
    (\# \textrm{ of FC4 parameters }) &= 512 \times 10 + 10 = 5130
\end{align*}
Thus, the total number of parameters in this CNN model is
\begin{align*}
    (\# \textrm{ of CNN parameters) } &= (\# \textrm{ of CONV1 parameters) } + (\# \textrm{ of CONV2 parameters) } \\
    &+ (\# \textrm{ of FC3 parameters) } + (\# \textrm{ of FC4 parameters) } \\
    &= 1520 + 25050 + 640512 + 5130 = 672212
\end{align*}
Finally, we derive $S_\textrm{up}(\mathcal{A}) (= S_\textrm{down}(\mathcal{A})$) in this case.
\begin{align*}
    S_\textrm{up}(\mathcal{A}) &= (\# \textrm{ of device) } \times (\# \textrm{ of CNN parameters) } \times (\# \textrm{ of bits)} \\
    &= 8 \times 672212 \times 32 = 172086272
\end{align*}
We can do the similar discussion in the MLP case when it comes to applying this to quantization-based federated optimization algorithms.

\subsection{Experimental Results}
\label{experimental_results}

In our experiments on both MNIST and CIFAR-10, we verify how the global training loss and test accuracy of five algorithms change with respect to communication rounds, the number of bits communicated between one client and the server during the uplink, and human time. We provide both qualitative analysis and quantitative results for plots.

\subsubsection{Qualitative Analysis}
\label{qualitative_analysis}

\paragraph{Strongly Convex Case.} In this experiment, we compare FedAQ under the condition set (\ref{parameter_FedAQ}) and set (\ref{parameter2_FedAQ}) with FedAvg, FedPAQ, FedCOMGATE, and FedAC-I. We denote each FedAQ as FedAQ-I and FedAQ-II. As we observe the theoretical benefits of FedAQ over other methods in \cref{convergence_analysis}, FedAQ-I outperforms all other quantization-based federated optimization algorithms and FedAC-I in all plots (See each first row of Figure \ref{graph_in_main_body}, \ref{mnist_graph}). However, although FedAQ-II shows the fast convergence speed, the training process is unstable. Thus, we only use FedAQ-I for further non-convex experiments. FedAC and FedAQ in non-convex experiments indicate FedAC-I and FedAQ-I.

\paragraph{Non-Convex Case.} Each second row of Figure \ref{graph_in_main_body}, \ref{mnist_graph}, and Figure \ref{cifar10_graph} clearly demonstrates that FedAQ with 4 bits quantization outperforms other algorithms in all plots. In terms of communication rounds, accelerated algorithms, FedAQ and FedAC, converge faster than other algorithms. We also observe that quantization does not lead to slower convergence, which means we can apply an efficient quantization scheme to make communication efficient FL systems without sacrificing convergence speed. The plots related to communicated bits are helpful to interpret how algorithms work well in situations with heavy communication. FedAQ with 8 bits quantization shows comparable performance relative to FedPAQ and FedCOMGATE with the help of acceleration, even though FedAQ sends more updates during every synchronization. When we use 4 bits quantization for FedAQ to make the number of communicated bits the same for all quantization-based algorithms during synchronization, FedAQ shows a much faster convergence speed with regard to the number of communicated bits. However, plots of communicated bits fail to reflect how algorithms converge in real estimated time for FL scenarios, which consists of both communication and computation. Thus, we further analyze algorithms with human time. We observe that FedAQ with 8 quantization bits performs slightly better than FedPAQ and FedCOMGATE for both MNIST and CIFAR-10. This occurs because while all quantization-based algorithms send the same number of communicated bits, the number of communication rounds for FedAQ is much smaller than others. Then, this also indicates that FedAQ takes less computation time than other methods while reaching the same accuracy.

\subsubsection{Quantitative Results}
\label{app:quantitative_graphs}

We provide quantitative results to help readers understand plots better. To be specific, for all plots, we observe the number of communication rounds, the number of communicated bits, and the human time required to achieve a particular test accuracy by each federated optimization algorithm.

For the strongly convex experiment on MNIST (See the first row of Figure \ref{mnist_graph}), the number of communication rounds required to achieve 90.28\% test accuracy by FedAvg, FedPAQ(8bits), FedCOMGATE(8bits), FedAC-I, FedAQ-I(8bits), FedAQ-II(8bits) are respectively 217, 216, 260, 28, 26, 99. The number of communicated bits required to achieve the same accuracy are respectively 5.4e7, 1.4e7, 1.6e7, 1.4e7, 3.3e6, 1.2e7. Lastly, the required human time are respectively 3220s, 2760s, 3336s, 484s, 344s, 1323s. In this experiment, FedAQ-I(8bits) requires the smallest number of communication rounds, the smallest number of communicated bits, and the shortest human time to achieve the same test accuracy. These experimental results support the validity of our theoretical analysis on strongly convex cases.

For the non-convex experiment on MNIST (See the second row of Figure \ref{mnist_graph}), the number of communication rounds required to achieve 97.6\% test accuracy by FedAvg, FedPAQ(8bits), FedCOMGATE(8bits), FedAC, FedAQ(8bits), FedAQ(4bits) are respectively 23, 48, 38, 18, 18, 16. The number of communicated bits required to achieve the same accuracy are respectively 1.5e8, 7.6e7, 6.1e7, 2.3e8, 5.7e7, 2.5e7. Finally, the required human time are respectively 2424s, 2311s, 1834s, 3327s, 1248s, 805s. Thus, we conclude that FedAQ(4bits) outperforms other algorithms, and even FedAQ(8bits) needs smaller number of communicated bits/less human time to achieve the goal accuracy than FedPAQ(8bits)/FedCOMGATE(8bits).

For the non-convex experiment on CIFAR-10 (See Figure \ref{cifar10_graph}), the number of communication rounds required to achieve 65.4\% test accuracy by FedAvg, FedPAQ(8bits), FedCOMGATE(8bits), FedAC, FedAQ(8bits), FedAQ(4bits) are respectively 98, 89, 95, 49, 50, 48. The number of communicated bits required to achieve the same accuracy are respectively 2.1e9, 4.8e8, 5.1e8, 2.1e9, 5.4e8, 2.6e8. Finally, the required human time are respectively 31798s, 11526s, 12240s, 28720s, 9902s, 6464s. As with the non-convex experiment on MNIST, FedAQ(4bits) outperforms other algorithms, and even FedAQ(8bits) requires less human time to achieve the same accuracy than FedPAQ(8bits)/FedCOMGATE(8bits).

\begin{remark}
Our current experimental setup only allows us to scale the number of clients up to the number of CPU cores in our machine. Since FedAQ achieves linear speed up in the number of workers with much fewer communication rounds than other quantization based methods, we expect FedAQ to outperform other methods by an even larger margin as we scale the number of workers.
\end{remark}

\begin{figure*}[!htbp]
    \centering
    % Figure 0
    \begin{subfigure}[]{
    \includegraphics[width=0.31\textwidth]{submissions/YeojoonYoun/figure/loss_iid_comm_str_cvx.png}
    %\caption{DCGAN}
    }
    \end{subfigure}
    % Figure 1
    \begin{subfigure}[]{
    \includegraphics[width=0.31\textwidth]{submissions/YeojoonYoun/figure/loss_iid_bits_str_cvx.png}
    %\caption{DCGAN}
    }
    \end{subfigure}
    %\quad
    % Figure 2
    \begin{subfigure}[]{
    \includegraphics[width=0.31\textwidth]{submissions/YeojoonYoun/figure/loss_iid_time_str_cvx.png}
    %\caption{OKGAN}
    }
    \end{subfigure}

    \setcounter{subfigure}{0}
    % Figure 0
    \begin{subfigure}[]{
    \includegraphics[width=0.31\textwidth]{submissions/YeojoonYoun/figure/loss_iid_comm_localstep_100_2.png}
    %\caption{DCGAN}
    }
    \end{subfigure}
    % Figure 1
    \begin{subfigure}[]{
    \includegraphics[width=0.31\textwidth]{submissions/YeojoonYoun/figure/loss_iid_bits_localstep_100_2.png}
    %\caption{DCGAN}
    }
    \end{subfigure}
    %\quad
    % Figure 2
    \begin{subfigure}[]{
    \includegraphics[width=0.31\textwidth]{submissions/YeojoonYoun/figure/loss_iid_time_localstep_100_2.png}
    %\caption{OKGAN}
    }
    \end{subfigure}
    \caption{Comparing FedAQ with FedAvg, FedPAQ, FedCOMGATE, and FedAC on MNIST with Strongly Convex Settings (first row) and Non-Convex Settings (second row). We observe how the global training loss changes across communication rounds (first column), communicated bits (second column), and human time (third column). FedAQ-I(8bits) and FedAQ(4bits) respectively outperform other algorithms for strongly convex settings and non-convex settings. FedAQ(4bits) sends the same number of communicated bits as FedPAQ(8bits) and FedCOMGATE(8bits) in each communication round, which indicates a fair comparison (See Quantization bits in \cref{experimental_setup}).}
    \label{graph_in_main_body}
\end{figure*}

\begin{figure*}[hbt!]%[!htbp]
    \centering
    % Figure 0
    \begin{subfigure}[]{
    \includegraphics[width=0.31\textwidth]{submissions/YeojoonYoun/figure/accuracy_iid_comm_str_cvx.png}
    %\caption{DCGAN}
    }
    \end{subfigure}
    % Figure 1
    \begin{subfigure}[]{
    \includegraphics[width=0.31\textwidth]{submissions/YeojoonYoun/figure/accuracy_iid_bits_str_cvx.png}
    %\caption{DCGAN}
    }
    \end{subfigure}
    %\quad
    % Figure 2
    \begin{subfigure}[]{
    \includegraphics[width=0.31\textwidth]{submissions/YeojoonYoun/figure/accuracy_iid_time_str_cvx.png}
    %\caption{OKGAN}
    }
    \end{subfigure}

    \setcounter{subfigure}{0}
    % Figure 0
    \begin{subfigure}[]{
    \includegraphics[width=0.31\textwidth]{submissions/YeojoonYoun/figure/accuracy_iid_comm_localstep_100_2.png}
    %\caption{DCGAN}
    }
    \end{subfigure}
    % Figure 1
    \begin{subfigure}[]{
    \includegraphics[width=0.31\textwidth]{submissions/YeojoonYoun/figure/accuracy_iid_bits_localstep_100_2.png}
    %\caption{DCGAN}
    }
    \end{subfigure}
    %\quad
    % Figure 2
    \begin{subfigure}[]{
    \includegraphics[width=0.31\textwidth]{submissions/YeojoonYoun/figure/accuracy_iid_time_localstep_100_2.png}
    %\caption{OKGAN}
    }
    \end{subfigure}
    \caption{Comparing FedAQ with FedAvg, FedPAQ, FedCOMGATE, and FedAC on MNIST with Strongly Convex Settings (first row) and Non-Convex Settings (second row). We observe how the test accuracy changes across communication rounds (first column), communicated bits (second column), and human time (third column). FedAQ-I outperforms other algorithms in all plots for strongly convex settings. Moreover, FedAQ(4bits) outperforms other algorithms in all plots for non-convex settings.}
    \label{mnist_graph}
\end{figure*}
%\FloatBarrier

\begin{figure*}[hbt!]%[!htbp]
    \centering
    % Figure 0
    \begin{subfigure}[]{
    \includegraphics[width=0.31\textwidth]{submissions/YeojoonYoun/figure/loss_iid_comm_cnn_step100.png}
    %\caption{DCGAN}
    }
    \end{subfigure}
    % Figure 1
    \begin{subfigure}[]{
    \includegraphics[width=0.31\textwidth]{submissions/YeojoonYoun/figure/loss_iid_bits_cnn_step100.png}
    %\caption{DCGAN}
    }
    \end{subfigure}
    %\quad
    % Figure 2
    \begin{subfigure}[]{
    \includegraphics[width=0.31\textwidth]{submissions/YeojoonYoun/figure/loss_iid_time_cnn_step100.png}
    %\caption{OKGAN}
    }
    \end{subfigure}

    \setcounter{subfigure}{0}
    % Figure 0
    \begin{subfigure}[]{
    \includegraphics[width=0.31\textwidth]{submissions/YeojoonYoun/figure/accuracy_iid_comm_cnn_step100.png}
    %\caption{DCGAN}
    }
    \end{subfigure}
    % Figure 1
    \begin{subfigure}[]{
    \includegraphics[width=0.31\textwidth]{submissions/YeojoonYoun/figure/accuracy_iid_bits_cnn_step100.png}
    %\caption{DCGAN}
    }
    \end{subfigure}
    %\quad
    % Figure 2
    \begin{subfigure}[]{
    \includegraphics[width=0.31\textwidth]{submissions/YeojoonYoun/figure/accuracy_iid_time_cnn_step100.png}
    %\caption{OKGAN}
    }
    \end{subfigure}
    \caption{Comparing FedAQ with FedAvg, FedPAQ, FedCOMGATE, and FedAC on CIFAR-10. We observe how the global training loss and test accuracy change across communication rounds (first column), communicated bits (second column), and human time (third column). We use a CNN model for CIFAR-10. Similar to the MNIST experiment, FedAQ (4 bits) outperforms all other algorithms in every case.}
    \label{cifar10_graph}
\end{figure*}


%!TEX root = ../main.tex
\section{Related Works}
\label{sec:related}

\stitle{Retrieval Augmented Generation Question Answering.}
%
Large languaga models sometimes generate factually incorrect or misleading information, often due to a lack of real-time knowledge or limited access to external facts beyond their training data.
RAG-based Question Answering addresses this by integrating external knowledge retrieval into the generation process. By retrieving relevant document chunks through semantic search, RAG ensures that the model’s responses are grounded in accurate, real-world information, effectively reducing the likelihood of hallucinations.
Early approaches focused on jointly training the retriever and generator, ensuring that the retrieved content aligned with the generation model’s intent to provide more accurate answers~\cite{izacard2023atlas}. With the success of in-context learning, more recent work has treated the retriever as a separate module, directly providing retrieved information to the model via prompts~\cite{wang2023knowledgptenhancinglargelanguage}.
As retrieval technologies have advanced, RAG-based systems now support multimodal retrieval, enabling answers that draw from diverse data sources~\cite{chen2021open, chen-etal-2022-murag, luo-etal-2023-unifying}. 
%For example, OTT-QA~\cite{chen2021open}  retrieves both tables and text, MuRAG~\cite{chen-etal-2022-murag} integrates text and images, and MMQA~\cite{luo-etal-2023-unifying} combines text, tables, and images to handle complex queries.
% \yang{It's hard for us to compare Symphony with existing systems.}


\stitle{Trustworthiness of Large Language Models.}
%
The trustworthiness of LLMs is essential for their effective deployment in real-world applications. To assess LLM trustworthiness, researchers have proposed various approaches. For example, TrustLLM~\cite{huang2024trustllmtrustworthinesslargelanguage} provides a comprehensive framework for evaluating LLMs across different trust dimensions.
However, evaluating LLM trustworthiness remains challenging, with gaps in holistic assessment approaches. Some studies suggest that self-evaluation, where LLMs assess their confidence in the generated outputs, can help improve selective generation and mitigate inaccuracies~\cite{Ren2023SelfEvaluationIS}. Additionally, understanding the internal mechanisms of LLMs, such as the use of local intrinsic dimension (LID) for predicting truthfulness, has been proposed as a way to measure model reliability~\cite{Yin2024CharacterizingTI}.
In our work, we aim to improve the trustworthiness of LLMs through post-verification, ensuring that generated outputs are validated against reliable sources after generation to minimize inaccuracies and enhance their overall reliability.
%\section{Future Work}
%
%As a pioneering attempt to solve the truth discovery problem in crowdsensing, our research still has certain limitations and points out some future opportunities.
%
%\textbf{Multi-class and Continuous-value Event Sensing}. In this work, we focus on binary event sensing. Naturally, our method can be extended to multi-class event sensing. Besides, sometimes we need to sense continuous value for a certain event (e.g., temperature). We will also study this in the future.
%
%%\textbf{Homomorphic Encryption-enabled Truth Discovery}. In FL systems, two widely adopted security mechanisms are homomorphic encryption (HE) and secure multi-party computation (SMC) \cite{yang2019federated}. Shamir's secret sharing is one type of SMC, which we rely on to build FedTruthFinder. HE may be another technical route to reach the goal. Currently, we do not choose HE because its computation efficiency is 10-100 times slower than non-HE computation. Some pioneering efforts have been devoted to developing privacy-preserving truth discovery systems with HE. Due to the large computation burden, these studies only let two non-colluding powerful servers to do HE computation instead of letting each client do computation in a federated (distributed) manner \cite{Tang2018NonInteractivePT,Zheng2018LearningTT}. However, finding two non-colluding servers are not so easy \cite{Bonawitz2017PracticalSA}.
%
%
%% they assume there need to be two non-colluding parties who take all the computation tasks.
%
%%Currently, we do not choose HE because its computation efficiency is 10-100 times slower than non-HE computation. Recently, HE tools have developed rapidly \cite{sealcrypto}, and we believe HE-enabled truth discovery is a promising direction. 
%
%\textbf{Other Truth Discovery Algorithms}. Besides the iterative truth discovery algorithm discussed in this paper, there are also other state-of-the-art truth discovery algorithms, such as optimization-based methods \cite{Li2014ACA,Li2015OnTD} and graphical model-based methods \cite{Pasternack2013LatentCA,Zhao2012ABA}. A comprehensive survey on truth discovery algorithms can be referred to \cite{li2016survey}. In the future, we will study whether our methods can be adapted to more truth discovery algorithms.
%
%\textbf{Against More Serious Attack Scenarios}. In this work, we assume that all the users are semi-honest, i.e., they will follow the protocol to compute and upload the corresponding data. In a more competitive setting, users can be malicious and upload some wrong data for attacking other users' privacy. We will discuss the solutions under such conditions in the future.
%
%\textbf{Missing Negative Sensed Data}. For some crowdsensing applications, users would only upload the positive sensed data ($e_j=1$). The truth discovery algorithms need to be refined as negative sensed data ($e_j=0$) are implicit. We will try to develop the federated version of such truth discovery algorithms in our future work.
%
%\textbf{Deployment on Smartphones}. In this work, we focus on the algorithmic design. In the future, we will implement the algorithm and deploy it to smartphones to further test its communication costs, energy consumption, etc. We will try to find practical guidelines to deploy such a federated crowdsensing truth discovery mechanism.

\section{Conclusion}

In this paper, we propose \textit{FedTruthFinder}, a crowdsensing federated truth discovery mechanism that can not only find aggregate truth from multiple participants' sensed data, but also rank participants' trustworthiness in a privacy-preserving manner. The primary characteristic of FedTruthFinder is its capability to tolerate network connection loss of participants in both event confidence calculation and participant trustworthiness ranking. As a byproduct, our proposed federated ranking algorithm can also serve other applications when the privacy-preserving data ranking is needed and the network connections are unstable.
Following most related papers, this work assumes participants to be semi-honest; in the future, we would explore the more challenging scenario that participants may behave maliciously.%To the best of our knowledge, FedTruthFinder is the best privacy-preserving truth discovery mechanism for crowdsensing regarding the robustness against participants' unpredictable connection loss. 






%%
%% The next two lines define the bibliography style to be used, and
%% the bibliography file.
\bibliographystyle{apalike}
\bibliography{fedtruthfinder}



%\onecolumn
\appendix
\section{Appendix}

\subsection{Theoretical Proof}

%\subsubsection{Correctness} We first prove that the correctness of our secure leader-board algorithm.
%
%\vspace{+.5em}
%\textbf{Lemma 5.1}. $\sum_{k=1}^{2t+1} r_k(x)\tau_{i_k}(x)$ can be represented as:
%$$h_{i}+a_{i1}x+a_{i2}x^2+...+a_{i2t}x^{2t}$$
%where $h_i=\sum_{k=1}^{2t+1} r_k\tau_i^k$. \cite{tang2011secure}

\textbf{Proof of Lemma 5.1}. It is clear that,
\begin{equation}
	\sum_{k=1}^{2t+1} r_k(0)\tau_{i_k}(0) = \sum_{k=1}^{2t+1} r_k\tau_i^k
\end{equation}
Besides, both $r_k(x)$ and $\tau_{i_k}(x)$ are $t$-degree polynomials, and thus the degree of $\sum_k r_k(x)\tau_{i_k}(x)$ is $2t$.$\qed$

\vspace{+.5em}
%\textbf{Theorem 5.1}. With $t+1$ participants' $h'_i(k)$, we can recover $h_i$.

\noindent \textbf{Proof of Theorem 5.1}. With Lemma 5.1, for $N$ ($=2t+1$) groups, $\gamma(gid(u_j))
=\sum_{k=1}^{2t+1} r_k(gid(u_j)) \tau_{i_k}(gid(u_j))$ (Step 4) is:

\[
\small
\left(\begin{array}{ccccc} 
	1 &    1 & 1^2 & ...  & 1^{2t} \\ 
	1 &    2 & 2^2 & ... & 2^{2t}\\
	... & ... & ... & ...& ...\\
	1 & N & N^2 & ... & N^{2t}\\
\end{array}\right) 
\left(\begin{array}{c} 
	h_i    \\ 
	a_{i1}    \\
	... \\
	a_{i2t} \\
\end{array}\right) 
=
\left(\begin{array}{c} 
	\gamma(1)    \\ 
	\gamma(2)    \\
	... \\
	\gamma(N) \\
\end{array}\right) 
\]
then,
\[
\small
\left(\begin{array}{c} 
	h_i    \\ 
	a_{i1}    \\
	... \\
	a_{i2t} \\
\end{array}\right) 
=
\left(\begin{array}{ccccc} 
	1 &    1 & 1^2 & ...  & 1^{2t} \\ 
	1 &    2 & 2^2 & ... & 2^{2t}\\
	... & ... & ... & ...& ...\\
	1 & N & N^2 & ... & N^{2t}\\
\end{array}\right)^{-1} 
\left(\begin{array}{c} 
	\gamma(1)    \\ 
	\gamma(2)    \\
	... \\
	\gamma(N) \\
\end{array}\right) 
\]
so,
$$h_i=\sum_{g=1}^{N} \lambda(g)\gamma(g)$$

In Step 5, $h_i(g)=\lambda(g)\gamma(g)$ is shared with $(t+1,n)$-SSS to all the participants from every group $g \in [1, 2t+1]$. Hence, according to the additive homomorphism property of SSS \cite{shamir1979share}, we can easily recover $h_i$ by receiving $t+1$ participants'  $h'_i(k) = \sum_{g=1}^{2t+1} h_i(g, k)$.$\qed$

\vspace{+.5em}
%\textbf{Theorem 5.2}. Ranking $h_i$ is equivalent to ranking $\tau_i$.

\noindent \textbf{Proof of Theorem 5.2}. As $\tau_i>0$ and $r_k>0$, $h_i=\sum_k r_k\tau_i^k$ will keep the same ranking as $\tau_i$.$\qed$

%\subsubsection{Robustness to Connection Loss} While mobile users may lose network connections during a crowdsensing campaign, we analyze how our secure ranking algorithm can tolerate connection losses. Without the loss of generalizability, we assume that before Step 2, there is no user connection loss.\footnote{If $u_i$ loses the connection in Step 2 and cannot share $\tau_i^k$ with SSS, then there is no way to rank $u_i$'s position because the server has no $u_i$'s information. So we only consider the users who establish the connections to share $\tau_i^k$ in Step 2 for ranking.}

\vspace{+.5em}
%\textbf{Theorem 5.3}. To finish Step 3-5, there needs at least one user online for each group. Suppose that every user has $p_l$ probability to lose connection and there are totally $n$ users, the success probability $\ge (1-p_l^{\lfloor n/(2t+1) \rfloor})^{2t+1}$.

\noindent \textbf{Proof of Theorem 5.3}. For Step 3 to 5, if there is at least one user in every group, then the process can continue. So the probability of failure incurred by one specific group $g$ is all the users in $g$ losing the connections, i.e., $p^{n_g} \le p_l^{\lfloor n/(2t+1) \rfloor}$ ($n_g$ is the user number in $g$). So for $g$, the probability of at least one user online $\ge 1-p_l^{\lfloor n/(2t+1) \rfloor}$. With $2t+1$ groups, the success probability $\ge (1-p_l^{\lfloor n/(2t+1) \rfloor})^{2t+1}$.$\qed$
%For Step 6 to 8, if at least $t'$ users have connections, then the server can recover $h_i$. Hence, the probability of failure is more than $N-t'$ users losing connections:
%$$p^{N-t'+1}$$

\vspace{+.5em}
%\textbf{Theorem 5.4}. To finish Step 6-8, $\ge t+1$ users need to be online.

\noindent \textbf{Proof of Theorem 5.4}. This is based on the property of $(t+1, n)$-SSS in Step 5.$\qed$


\vspace{+.5em}
%\textbf{Theorem 5.5} If there are no more than $t$ collusive participants, then these participants cannot recover all the other users' $\tau_i$.

\noindent \textbf{Proof of Theorem 5.5}. In Step 2, $\tau_i^k (k=1...2t+1)$ is shared with $(t+1, 2t+1)$-SSS. So, if $t$ participants collude, they can get at most $t\cdot(2t+1)$ equations when $t$ participants are from $t$ groups. However, the number of unknown parameters (including $\tau_i$ and $t$ random coefficients for sharing each $\tau_i^k$) is $t\cdot(2t+1)+1$. Hence, these $t$ collusive participants cannot recover other participants' $\tau_i$.$\qed$

\subsection{Mechanism Extension to Multi-class and Continuous-value Events}



\textbf{Multi-class Events}. For a multi-class event ($m$ classes), we can see it as $m$ binary events, so that our method can be directly applied.

\noindent \textbf{Continuous-value Events}. For continuous-value events, following the literature, we may adopt other proper event confidence and participant trustworthiness updating functions such as CRH \cite{Xu2019EfficientAP,Zheng2020PrivacyAwareAE}. Specifically, suppose that the discovered truth sensed value of a continuous event $e_j$ is $\rho_j$, and $u_i$'s sensed data of $e_j$ is $\hat \rho_{ij}$, then the event truth (confidence) and participant trustworthiness updating functions  can be:
\begin{equation}
	\rho_j = \frac{\sum_{u_i \in \mathcal U_{e_j}}\tau_i \cdot \hat \rho_{ij}}{\sum_{u_i \in \mathcal U_{e_j}}\tau_i}
	\label{eq:rho_function_cont}
\end{equation}
\begin{equation}
	\tau_i = \log(\sum_{u_i \in \mathcal U} \sum_{e_j \in \mathcal E_{u_i}} \frac{(\rho_j- \hat \rho_{ij})^2}{|\mathcal E_{u_i}|}) - \log(\sum_{e_j \in \mathcal E_{u_i}} \frac{(\rho_j- \hat \rho_{ij})^2}{|\mathcal E_{u_i}|})
	\label{eq:tau_function_cont}
\end{equation}
where $\mathcal U_{e_j}$ is the set of users who sense $e_j$, and $\mathcal E_{u_i}$ is the set of events that $u_i$ has sensed. For $\rho$-computation, following Sec.~\ref{sub:basic_rho_computation}, we can just adapt $d_{ij}$ and $s_{ij}$ according to Eq.~\ref{eq:rho_function_cont} (the participant $u_i \not \in \mathcal U_{e_j}$ can still send $d_{ij}=s_{ij}=0$ to protect her task completion information). For $\tau$-computation, Eq.~\ref{eq:tau_function_cont} requires $\sum_{u_i \in \mathcal U} \sum_{e_j \in \mathcal E_{u_i}} \frac{(\rho_j- \hat \rho_{ij})^2}{|\mathcal E_{u_i}|}$, which can be done with the same SSS-based method as $\rho$-computation. In particular, each participant $u_i$ can send $\sum_{e_j \in \mathcal E_{u_i}} \frac{(\rho_j- \hat \rho_{ij})^2}{|\mathcal E_{u_i}|}$ by secret shares, and then the server can compute the sum in a privacy-preserving manner. In a word, for continuous-value events, our mechanism can still work without revealing each participant's raw sensed data and completed tasks.






\end{document}
\endinput
%%
%% End of file `sample-sigconf.tex'.
