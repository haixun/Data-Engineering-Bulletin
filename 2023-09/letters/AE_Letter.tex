\documentclass[11pt]{article}

\usepackage{deauthor,times,graphicx}
%\usepackage{url}

\begin{document}
Similarity search in high-dimensional data spaces was a relevant and challenging data management problem in the early 1970s, when the first solutions to this problem were proposed. 
Today, fifty years later, we can safely say that the exact same problem is more relevant and challenging than ever. 
This is true, not because the research community has been idle; on the contrary, the literature on this topic is very large and diverse, demonstrating both the interest in this problem, as well as the wide range of ideas that have been applied to it and led to impressive advances. 
This is true, rather because very large amounts of high-dimensional data are now omnipresent (ranging from traditional multidimensional data to time series and deep embeddings), and the performance requirements (in terms of both response-time and accuracy) of a variety of applications that need to process and analyze these data have become very stringent and demanding.

In these past fifty years, high-dimensional similarity search has been studied in its many flavors. Similarity search algorithms for exact and approximate, one-off and progressive query answering. 
Approximate algorithms with and without (deterministic or probabilistic) quality guarantees.
Solutions for on-disk and in-memory data, static and streaming data.
Approaches based on multidimensional space-partitioning and metric trees, random projections and locality-sensitive hashing (LSH), product quantization (PQ) and inverted files, k-nearest neighbor graphs and optimized linear scans.

Another interesting aspect of the work in high-dimensional similarity search is that research on this problem has been conducted by different (sub-)communities in a somewhat independent fashion, that is, with not much interaction among them.
A notable example is the work on data-series (and time-series) similarity search, which was recently shown to achieve the state-of-the-art performance for several variations of the problem, on both time-series and general high-dimensional vector data.
It is only very recently that a conscious effort is being made in order to gather the state-of-the-art methods from these different communities, and thus, enable the comparison of the various approaches, the extraction of useful insights, and the development of improved solutions.
This special issue contributes to this effort by including a selection of papers that represent the research activity in several of these communities, highlighting similarities and differences, and revealing open research directions.

In the first paper, Wang et al. summarize and discuss state-of-the-art solutions for approximate similarity search based on k-nearest neighbor graphs and data-series tree indexes, and point to promising research directions.
In the second paper, Zhang et al. list the similarity search requirements of modern applications, and present novel algorithms based on k-nearest neighbor graphs that exploit multi-core architectures and NVMe memory.
In the third paper, Tian et al. summarize and discuss state-of-the-art solutions, as well as future research directions, for approximate similarity search based on locality-sensitive hashing, product quantization and k-nearest neighbor graphs.
In the fourth paper, Dong et al. propose the use of neural networks in order to learn effective space partitions, and develop a novel solution based on locality-sensitive hashing for approximate similarity search.
In the fifth paper, Paparrizos et al. compare many distance measures proposed for time-series similarity search, comment on the lower bounds that speedup some of these measures, and discuss the open research problems that their findings point to. 
Finally, in the sixth paper, Aum\"{u}ller and Ceccarello study the very important problem of creating appropriate benchmarks for approximate similarity search; they review recent benchmarks, and offer guidelines for future efforts in this area.

Overall, we believe that the above papers represent an interesting sample of the ongoing work on high-dimensional similarity search. 
We hope that this special issue will further help and inspire the research community in its quest to solve this challenging problem.

We would like to thank all the authors for their valuable contributions, as well as Haixun Wang for giving us the opportunity to put together this special issue, and Nurendra Choudhary for his help in its publication.

Themis Palpanas

Universit{\'e} Paris Cit{\'e}
\end{document}
