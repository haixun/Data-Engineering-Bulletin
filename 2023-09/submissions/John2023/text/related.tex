%\clearpage
\section{Related Work}\label{sec:related}

\noindent{\bf Parallel graph algorithms.}
There are numerous efforts that aim to parallelize generic search schemes on graphs (e.g., BFS~\cite{shun2013ligra}, DFS~\cite{naumov2017parallel}, random walk~\cite{talati2021deep}, beam search~\cite{meister2020best}, and bucketing~\cite{sridhar2019delta,zhang2020optimizing}). However, many of these algorithms were designed without considering having a vector associated with each vertex and a target of achieving high recall under a stringent latency constraint. In contrast, we analyze the search efficiency challenges of ANN and propose optimizations to handle them to allow vector-based similarity search to scale on modern multi-core architectures. Among different algorithms, the most related work to ours is perhaps $\Delta$-stepping~\cite{zhang2020optimizing}, which stages the expansion of nodes in order to avoid redundant expansions. We have applied $\Delta$-stepping to vector search and will provide a more detailed discussion in Section~\ref{subsec:delta-step} and comparison in Section~\ref{sec:eval}.

\noindent{\bf Parallel graph frameworks.}
There has been many graph engines and frameworks developed in the past decade.
Some of them are shared-memory, focusing on processing in-memory datasets within a computation node~\cite{uta2018exploring}, e.g., Galois~\cite{nguyen2013lightweight}, Ligra~\cite{shun2013ligra}, Polymer~\cite{zhang2015numa}, GraphGrind~\cite{sun2017graphgrind}, GraphIt~\cite{zhang2020optimizing}, Graptor~\cite{vandierendonck2020graptor}, and GraphBLAS~\cite{aznaveh2020parallel}.
Some are distributed systems~\cite{rivas2018mpi}, e.g., Pregel \cite{malewicz2010pregel}, GraphLab~\cite{low2014graphlab},  PowerGraph~\cite{joseph2012powergraph}, and Gluon~\cite{dathathri2019gluon}.
Some efforts focus on out-of-core designs (e.g., GraphChi~\cite{aapo2012graphchi} and X-Stream~\cite{roy2013x}) and process large graphs with disk support.
Many graph frameworks are also on GPUs~\cite{DBLP:conf/ppopp/MengLTS19}, such as CuSha~\cite{khorasani2014cusha}, Gunrock~\cite{wang2016gunrock}, GraphReduce~\cite{sengupta2015graphreduce}, Graphie~\cite{han2017graphie}, Multigraph~\cite{hong2017multigraph},  GraphBLAST~\cite{yang2019graphblast}, and Ascetic~\cite{tang2021ascetic}.
These graph systems are either based on a vertex-centric model~\cite{malewicz2010pregel,zhang2018simple} or its variants (e.g., edge-centric~\cite{roy2013x}).
Many of these parallel graph frameworks are designed primarily for generic parallel graph analytics instead of vector-based similarity search. Enabling ANN search in these frameworks, which have matured compilers and optimization technologies, is possible but requires addressing non-trivial portability challenges. For example, the input graphs of existing ANN search can have varying structures, such as hierarchical~\cite{hnsw}, heterogeneous~\cite{diskann,hm-ann}, etc. The search engine also performs additional optimizations such as transaction support~\cite{milvus}, vector reordering, prefetching, and specialized optimizations against different data types, e.g., FP32/INT8. Therefore, porting those changes, to existing frameworks, beyond the search algorithm itself, requires changes across the entire system stack.

\noindent{\bf Heterogeneous memory.}
Heterogeneous memory (HM) is emerging. It combines multiple memory components to construct main memory. HM is typically composed of a high-capacity memory technology such as non-volatile memory (but high memory access latency) and a high-performance memory technology (with limited memory capacity) such as DRAM. 
To make HM performance close to that of DRAM-only, previous work focuses on hardware-~\cite{asplos15:agarwal,hetero_mem_arch,qureshi_micro09, ibm_isca09,gpu_pcm_pact13} and software-based~\cite{eurosys16:dulloor,asplos16:lin,wen:ICS18,sc18:wu,unimem:sc17,luo:NGS,cluster20:ren} solutions to manage data placement on HM. Optane PMM and DRAM are commonly used to build HM. With PMM, the memory capacity on a single machine can achieve 6TB~\cite{optane:ucsd}. However, the latency and bandwidth of PMM is only 1/3 and 1/6 of DRAM. There are two operating modes for PMM, \textit{Memory Mode} and \textit{App-direct Mode}. 
In Memory Mode, DRAM works as a hardware-managed cache to PMM. Running the application in this mode does not require application modifications. App-direct Mode allows the programmer to explicitly control memory accesses to PMM and DRAM. \name works in App-direct Mode and outperforms Memory Mode in billion-scale dataset search (Section~\ref{sec:eval}). 