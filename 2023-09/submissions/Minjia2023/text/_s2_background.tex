\vspace{-1em}
\section{Background}\label{minjia_sec:background}

The literature on nearest neighbor search is vast, and hence, we focus our attention on the most relevant works here.  
There has been a lot of work on building effective ANN indices to accelerate the search process. Earlier works focus on space partitioning-based methods. For example, Tree-based methods (e.g., KD-tree~\cite{silpa2008optimised} and R* tree~\cite{r-star-tree}) hierarchically split the data space into lots of regions that correspond to the leaves of a tree structure and only search a limited number of promising regions. However, the complexity of these methods becomes no more efficient than brute-force search as the dimension becomes large (e.g., $>$16)~\cite{worst-case-kdtree}. 
Prior works also have spent extensive efforts on locality-sensitive hashing-based methods~\cite{indyk1998approximate,datar2004locality,andoni2006near,andoni2015practical}, which map data points into multiple buckets with a certain hash function such that the collision probability of nearby points is higher than the probability of others. These methods have solid theoretical foundations. LSH and its variations are often designed for large sparse vectors with hundreds of thousands of dimensions. In practice, LSH-based methods have been outperformed by other methods, such as graph-based approaches, by a large margin on large-scale datasets~\cite{ann-benchmark,hnsw,nsg}. 
More recently, Malkov and Yashunin found graphs that satisfy the small-world property exhibit excellent navigability in finding nearest neighbors. They introduce the Hierarchical Navigable Small World (HNSW)~\cite{hnsw}, which builds a hierarchical k-NN graph with additional long-range links that help create the small-world property. For each query, it then performs a walk, which eventually converges to the nearest neighbor in logarithmic complexity. Subsequently, Fu et al. proposed NSG, which approximates Monotonic Relative Neighbor Graph (MRNG)~\cite{nsg} that also involves long-ranged links for enhancing connectivity. 

