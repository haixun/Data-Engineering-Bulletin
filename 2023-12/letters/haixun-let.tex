\documentclass[11pt]{article} 

\usepackage{deauthor,times,graphicx}
%\usepackage{url}
\usepackage{hyperref}

\begin{document}

In this December edition of the Data Engineering Bulletin, we revisit the topic of personalization after more than a decade. Our renewed focus comes at a time when swift technological advances are poised to elevate personalization to a new level. This issue of the Bulletin, guided by the expert curation of Luna Dong, Alon Halevy, and Shane Moon, not only charts the progress in technologies that promise enhanced personal experiences but also confronts the new privacy challenges they bring. The associate editors provide a comprehensive view of the new personalization landscape, underlining the necessity for robust privacy frameworks to manage personal information effectively. Building on this foundation, I would like to expand the conversation with my reflections on industry and business implications, especially in the context of e-commerce. 

%From a business viewpoint, the practice of personalization is still in its early stages, with companies' insights into their customers remaining relatively shallow. Nevertheless, technological advancements, notably generative AI and multimodal models, along with the increasing zeal for continuous personal life documentation, are paving the way for businesses to not just predict, but actively shape customer behaviors, with the potential to boost their profits.

It is crucial for e-commerce businesses to understand their customers' needs. For many years, e-commerce platforms have maintained detailed user profiles, tracked customers' browsing patterns and purchasing histories, and used the data to make personalized recommendations. How successful has it been?

%Businesses are interested in obtaining deep insights into human needs. For example, this is  crucial for e-commerce platforms, which track users’ purchasing and browsing habits, and use this data to make personalized recommendations. How successful has it been?

Elle Hunt's recent Guardian article~\cite{guardian2023} offers a critical view of the current personalization practice. Hunt notes, "{\it Every tech company from Monzo to my bank is crunching my data. All the results tell us is how dull it is to reduce human experience to numbers.}" This statement highlights a major limitation of today’s personalization technologies.


%Elle Hunt's recent Guardian article~\cite{guardian2023} offers a critical view. Hunt notes, "Every tech company from Monzo to my bank is crunching my data. All the results tell us is how dull it is to reduce human experience to numbers." This statement highlights a significant gap in current personalization technologies. Companies like Spotify, while adept at aggregating quantitative data, fail to capture the qualitative aspects of our experiences such as emotion. For instance, knowing the number of hours spent listening to Taylor Swift or the amount spent on groceries doesn't reveal the emotional motivations behind these actions. 

While companies like Spotify are adept at aggregating quantitative data, they fail to capture the qualitative, emotional aspects of customer experiences. Indeed, reducing complex human activities to stark statistics strips away the nuances that give human experiences meaning. More specifically, simply tracking the number of hours spent listening to Taylor Swift, or the frequency a particular dish is ordered through DoorDash, doesn't reveal the motivations or feelings behind these actions.

However, we might be surprised just how soon multimodal and generative AI technologies could revolutionize the field of personalization. I believe a digital personal assistant, fueled by these advanced technologies, has the potential to elevate personalization to new, unprecedented heights.
Instead of tracking basic statistics, the personal assistant, equipped with multimodal generative AI technologies, will have the capability to process a variety of sensory inputs, including visual and auditory cues. It’s easy to envision an assistant capable of assessing a customer's reactions through facial recognition and tone analysis during virtual try-ons and interactive product consultations, thereby obtaining a deep, empathetic understanding of the individual’s preferences and needs.

In contrast to existing personalization methods that are confined to singular platforms and only analyze user behavior in isolated contexts, this personal assistant has the potential to track an individual's activities across a multitude of platforms. By being a constant in the individual’s life, it will gain a nuanced understanding of the individual’s actions and motivations, achieving a level of insight that traditional data analysis methods cannot match. It might even come to know the individual more intimately than they know themselves.
With its profound understanding, the personal assistant will help the individual in engaging with business platforms, in a more clear and effective way than the individual could alone. This holistic approach to personalization is poised to create a more intuitive and emotionally engaging shopping experience.

However, while the advancements in personalization technology offer many benefits, they also tread a delicate line between enhancing user experience and encroaching on personal privacy. As we revel in the marvels of such technology, caution is warranted. While contemplating a personal assistant capable of understanding a customer’s sentiment and emotion, I can’t help thinking about lifelogging~\cite{lifelogging}. Lifelogging refers to a comprehensive recording of a person's daily life, often aided by wearable technology or mobile devices. This practice has evolved significantly, with projects like DARPA LifeLog contributing to its development, and contemporary apps like Foursquare and Swarm allowing users to document their lives on their platforms. In 2022, Jennifer Egan, an award winning author, explored a related concept in her novel “The Candy House”~\cite{candyhouse}. She imagined a world where technology has advanced to the point that individuals can upload their memories to a shared database. The novel suggests a kind of collective consciousness, where the boundaries between personal and shared experiences become blurred.

If we feed lifelogging data to multimodal and generative AI, which are capable of analyzing data in various forms (text, images, videos, sensor data), we can gain a deeper, more holistic understanding of an individual's habits, preferences, and needs. Technologies will not only be able to predict an individual’s future behaviors, but also act as a proxy of the individual, interacting with the world on her behalf.

Such advancements in personalization and AI's understanding of individuals raise significant ethical issues. This situation echoes the ethical concerns similar to those confronting social networks, which have been criticized for manipulating users’ emotions and fostering addictive behaviors to enhance user engagement and profitability. As businesses gain their power to understand their customers, they may use the power not just to {\it predict} but to {\it change} customers’ behavior in order to boost sales and ad revenue. As playwright Ayad Akhta pointed out in his essay titled “The Singularity is Here” ~\cite{atlantic2021} on The Atlantic, “{\it Our affinities are increasingly no longer our own, but rather are selected for us for the purpose of automated economic gain.}”

Therefore, as e-commerce platforms harness the power of generative AI to better understand their customers, there’s an imperative to balance commercial goals with ethical considerations. The potential to use these advancements for positive outcomes, such as promoting health, personal growth, and societal benefits, should be weighed against the commercial opportunities they present, ensuring a responsible and human-centric approach to technology application.



%Nevertheless, the future of personalization could be revolutionized by the development of a multimodal, generative AI-powered personal assistant. This assistant will monitor an individual's behavior across various platforms, leveraging its ability to process diverse sensory information, including visual, auditory, and emotional cues. Furthermore, the increasing practice of lifelogging~\cite{atlantic2021}, capturing  every detail of a person's life, will feed the AI invaluable data for a more profound understanding. Thus, by being a constant in a user's life, the assistant’s deep and empathetic understanding of the individual will far surpass the insights gleaned from conventional data analysis. This holistic approach to personalization will make e-commerce interactions more fulfilling and human-like.

%However, this advancement in personalization and AI's deepened understanding of individual behaviors also brings to the forefront significant ethical considerations. This situation echoes the ethical concerns similar to those confronting social networks, which have been accused of manipulating emotions and fostering addictive behaviors to enhance user engagement and profitability. As businesses gain the power to understand their customers, they may use it to not just predict but change customers’ behavior to boost sales and ad revenue. As Ayad Akhtar~\cite{lifelog} pointed out, “Our affinities are increasingly no longer our own, but rather are selected for us for the purpose of automated economic gain.”

%Therefore, we must carefully weigh the application of these advancements, ensuring they serve to advance health, personal growth, and societal welfare, and not just commercial objectives.

\begin{thebibliography}{3}
\bibitem{guardian2023}
Elle Hunt. (2023, December 28). So, Spotify knows how many hours I spent listening to Taylor Swift. But only I know why. \textit{The Guardian}. Retrieved from \texttt{https://www.theguardian.com/commentisfree/2023/dec/28/\\spotify-wrapped-monzo-analyses-meaningless}

\bibitem{atlantic2021}
Ayad Akhtar. (2021, November 5). The Singularity Is Here. \textit{The Atlantic}. Retrieved from \texttt{https://www.theatlantic.com/magazine/archive/2021/12/\\ai-ad-technology-singularity/620521/}

\bibitem{lifelogging}
Wikipedia contributors. Lifelog. Retrieved from \texttt{https://en.wikipedia.org/wiki/Lifelog}

\bibitem{candyhouse}
Jennifer Egan. (2022, April). The Candy House: A Novel.
\textit{Scribner}
\end{thebibliography}

\vspace{2cm}
\end{document}


