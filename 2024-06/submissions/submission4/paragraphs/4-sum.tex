\section{Differential Privacy for Sum/Mean Queries}\label{sec4}
This section provides a comprehensive review of the literature concerning sum and mean queries within the differential privacy framework. It explores the concept of sum/mean queries, the basic queries, and the downstream applications.
\subsection{Essential Information of Sum/Mean Queries}
While preserving the occurrence of an element in a time series is important, maintaining the accuracy of the element's value is equally crucial. To ensure precise results for queries such as sum or mean, small deviations in the perturbation process are necessary. Since the mean is directly correlated to the sum, we discuss sum and mean queries together.

The sum query can be denoted as 
\begin{equation}\nonumber
	\mathrm{F}_{sum}(T, S)=\sum_{i=1}^{T}S_i,
\end{equation}
where $T$ represents the timestamp for sum release and $S$ is the corresponding time Series. The range query on sum is 
\begin{equation}\nonumber
	\mathrm{F}_{rsum}((T_1, T_2), S)=\sum_{i=T_1}^{T_2}S_i,
\end{equation}
where $(T_1, T_2)$ represents the query range, and $S$ is the corresponding time Series.

As for the mean query, it can be denoted as
\begin{equation}\nonumber
	\mathrm{F}_{mean}(T, S)=\frac{1}{T}\sum_{i=1}^{T}S_i,
\end{equation}
where $T$ represents the timestamp for mean release, and $S$ is the corresponding time Series.
Another common mean query is to release the mean at a timestamp from users' time series, 
\begin{equation}\nonumber
	\mathrm{F}_{rtm}(t, S)=\frac{1}{n}\sum_{i=1}^{n}S_t^{u_i},
\end{equation}
where $S_t^{u_i}$ represents the value at timestamp $t$ from the user $u_i$, and $n$ is the number of users.


\subsection{Basic Sum/Mean Queries}
\begin{table}[h]
	\renewcommand\arraystretch{1.5}
	\scriptsize
	\begin{tabular}{cccccc}% 其中,tabular是表格内容的环境;c表示centering,即文本格式居中;c的个数代表列的个数
		\toprule %[2pt]设置线宽     
		\textbf{Query}&\textbf{Ref.} & \textbf{Type}&\textbf{Privacy Level}  & \textbf{Method Primitive}& \textbf{Error Bound} \\ %换行
		\midrule %[2pt]  
		\multirow{1}{*}{\parbox{3.2cm}{Sum}}&\cite{dong2023continual} & infinite  &user-level CDP & binary tree mechanism & $O(\frac{\varphi(D_t)}{\epsilon\theta}\cdot \log^{1.5}(tR)\cdot \log^{1+\theta}(\varphi(D_t))\cdot\log(1/\beta))$\\
		\hline
		\multirow{2}{*}{\parbox{3.2cm}{Window Sum}}&\cite{bolot2013private} & finite  &event-level CDP & binary tree mechanism & $O(\frac{1}{\epsilon}\log W\frac{1}{\beta})$\\
		&\cite{henzinger2024unifying} & finite  &event-level ($\epsilon, \delta$)-CDP & matrix mechanism & $O(2\sigma^2_{\epsilon, \delta}\Delta^2(1+\frac{\log W}{\pi}+\frac{2}{W})^2)$\\
		\hline
		\multirow{2}{*}{\parbox{3.2cm}{Exponential Decay Sum}}&\cite{bolot2013private} & finite  &event-level CDP & binary tree mechanism & $O(\frac{1}{\epsilon}\log \frac{\alpha}{1-\alpha}\frac{1}{\beta})$\\
		&\cite{henzinger2024unifying} & finite  &event-level ($\epsilon, \delta$)-CDP & matrix mechanism & $O(\sigma^2_{\epsilon, \delta}\Delta^2(1+\frac{1}{\pi}S_{T, 2\alpha})^2)$\\
		\hline
		\multirow{2}{*}{\parbox{3.2cm}{Polynomial Decay Sum}}&\cite{bolot2013private} & finite  &event-level CDP & binary tree mechanism & $\Omega(1-\frac{\epsilon^{c-1}}{\log^{c-1}(1/\beta)})$\\
		&\cite{henzinger2024unifying} & finite  &event-level ($\epsilon, \delta$)-CDP & matrix mechanism & $O(\sigma^2_{\epsilon, \delta}\Delta^2(1+\frac{H_{T, 2c}-1}{4})^2)$\\
		\bottomrule %[2pt]     
	\end{tabular}
	
	\caption{The table provides a brief summary of sum-based queries,  where $\varphi(D_t)$ denotes the maximum contribution from any user at time $t$, $\theta$ is a small constant,  $W$ is the window length, $\beta$ is the confidence parameter, $\alpha$ indicates the exponential decay parameter, $c$ represents the polynomial decay parameter, $H_{T, 2c}$ is the generalized Harmonic sum, $S_{T, 2\alpha}$ is a defined series sum with $\alpha>1$.\tablefootnote{The error bound of~\cite{henzinger2024unifying} in the table is the $L_2$ norm error.}} 
	\label{sum_query}
\end{table}


There have proposed a line of work to release sum/mean queries under differential privacy, with a brief summary provided in Table~\ref{sum_query}.
The pioneering work by Bolot et al.~\cite{bolot2013private} was the first to study the continual decaying sums problem. They explored three variants: the window sum (range sum query), which releases the sum of $W$ consecutive elements; the exponential decay sum, which releases the sum of elements weighted by an exponential function; and the polynomial sum, which releases the sum of elements weighted by a polynomial function. 
Henzinger et al.~\cite{henzinger2024unifying} also investigated the continual decaying sum problem. Their work introduced the use of the Gaussian mechanism for adding noise and derived tighter error bounds. 
In contrast, Dong et al.~\cite{dong2023continual} addressed sum queries by reducing them to count queries, as their approach could only handle the latter. For each timestamp with value $x_i$, they expand it into $R$ steps, where the first $x_i (w.l.o.g., x_i<R)$ steps are filled with $1$, and the remaining steps are filled with a special symbol $\perp$. However, this method requires prior knowledge of the maximum value $R$ that can occur in the time series.
~\cite{wang2021continuous} proposed a method for answering sum queries with a threshold under a multi-branch tree structure. The threshold is optimized based on the expected squared error between the true result and the estimated one.   Their proposed mechanism cannot handle infinite time series, so they claim that most queries focus on a limited range, allowing for truncation of the infinite time series. Additionally, their work introduced the first mechanism to release time series under LDP with a threshold for value truncation. 
Instead of directly perturbing the values, the mechanisms proposed by Ye et al.~\cite{ye2021beyond, ye2023stateful} perturb the temporal order. This approach makes them naturally adaptable for sum/mean queries while preserving the original values. These methods demonstrate superior performance for calculating moving averages.

%\subsection{Privacy Budget Allocation Strategies for Sum/Mean Queries}
%In contrast to count queries, sum/mean queries are more sensitive to the actual values, making them more critical from a privacy perspective. Consequently, various privacy budget allocation strategies have been proposed following the introduction of $w$-event level privacy~\cite{Kellaris14}. Although the work~\cite{Kellaris14} is for time series release, the introduced strategies are inspirational for sum/mean queries. The budget distribution strategy, which allocates the privacy budget in an exponentially decreasing manner, and the budget absorption strategy, which allocates the budget uniformly and absorbs invalid allocations. While the budget absorption strategy salvages the privacy budget for invalid allocations, it can lead to compulsory empty outputs when the sum of the privacy budget exceeds the predefined value. 
%Building upon the ideas of the budget distribution and absorption strategies, Ren et al.~\cite{ren2022ldp} proposed corresponding strategies under the local differential privacy (LDP) framework. To mitigate the utility degradation caused by dividing the privacy budget, the authors instead divide the users, with each user reporting only one timestamp within a $w$-length window. 
%To enable real-time computation of the mean at any timestamp from users' time series under $w$-event LDP, Wang et al.~\cite{wang2020towards} proposed sampling strategy to select important elements and privacy budget allocation strategy according to the importance of the elements. However, the sampling process may inadvertently reveal some private information due to the intentional selection.

\subsection{Downstream Applications Based on Sum/Mean Queries}
Due to utility considerations, existing mechanisms for downstream applications based on sum and mean queries primarily operate under event-level privacy or $w$-event level privacy.
Perrier et al.~\cite{perrier2018private} introduced a differentially private mechanism for publishing statistics of real-valued time series under event-level privacy, such as moving averages derived from energy data collected through smart meters. Their approach addresses scenarios where the bound on observations is either overly conservative or unknown, which is crucial for real-time monitoring applications. The proposed mechanism optimizes utility by scaling the added noise to the threshold value instead of a potentially larger bound, thereby improving accuracy.
To enable real-time computation of the mean at any timestamp from users' time series under $w$-event LDP, Wang et al.~\cite{wang2020towards} proposed sampling strategy to select important elements and a privacy budget allocation strategy according to the importance of the elements. However, the sampling process may inadvertently reveal some private information due to the intentional selection.
Kurt et al.~\cite{kurt2022online} proposed an online anomaly detection method for networks based on the cumulative sum algorithm, satisfying event-level $(\epsilon, \delta)$-differential privacy. Their approach adds noise to the statistic at each timestamp from each network node, operating under event-level privacy, and then derives the mean from the data of all nodes. This allows for detecting anomalies by monitoring changes in the released means.










