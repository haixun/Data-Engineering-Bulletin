\section{Challenges and Opportunities}
\label{sec:future}
The work discussed covers limited aspects of building an effective System 2 RAG solution. Numerous intriguing research questions remain unanswered in the fields of NLP, AI, databases, and HCI, presenting ample opportunities for interdisciplinary collaboration. Here, we will discuss some of these questions.
\subsection{Planning and reasoning}
Despite emerging attempts to explore LLMs' reasoning capabilities and use them as planners~\cite{zhao2024large, huang2022language} or `routers' of existing tools and APIs~\cite{liang2024taskmatrix, qin2023toolllm,kim2023llmcompiler}, LLMs alone still cannot solve the planning problem~\cite{kambhampati2024llms,valmeekam2023planning}. Key questions remain, including: How to exploit LLMs for planning, yet add verification and constraints? How to perform planning over multi-modal (relational, graph, documents, parametric) data sources?
How to interact with the user in regards to planning, present and refine plans collaboratively? How to learn feedback and attribute back to agents and operators?


In addition, Optimization is critical for planning for production both as a driver of QoS and business-wise, as cost and performance affect the bottom line. Optimization is a well-studied subject, but new questions emerge: How to perform cost estimation for (new) agents, given the dependence on data (size and beyond)? How to handle uncertainty in sources such as LLMs? How to estimate the overall plan cost? How to incorporate an accrued budget into planners?

Additional future research opportunities in reasoning and decision-making systems exist in addressing the overhead introduced by deliberate, system 2 thinking processes, such as planning, which can be slow and cause notable differences in enterprise setups. A promising direction involves alleviating this burden by moving parts of the system 2 thinking process offline, continuously distilling and materializing knowledge into diverse representations. Metaphorically, this is akin to learning to drive: while initial skill acquisition requires deliberate system 2 reasoning, skilled drivers rely on system 1 instincts, reacting fluidly without explicitly thinking about each action, as their expertise becomes ingrained like muscle memory. Similarly, RAG systems can benefit significantly from insights derived from both online and offline learning processes. These insights can extend beyond traditional model weights to include artifacts like graphs, tables, and natural language documents. This calls for research in areas such as knowledge distillation, insight extraction, and planning, with a focus on understanding when and where to trust instinctual, system 1 insights versus when to engage in more rigorous, system 2 reasoning.
\subsection{Multi-modal data}
Enterprise environments often contain highly varied data sources, including databases, document collections, graphs, and structured tables with heterogeneous schemas. RAG systems designed for these environments must be capable of handling data from diverse formats while preserving contextual coherence across data types. They face challenges ranging from architectural and representational choices to managing ambiguity and uncertainty across modalities. How can RAG models balance capturing detailed, structured knowledge from tables or graphs with synthesizing general information from unstructured text? How should confidence levels and uncertainty be managed when retrieving from different data types? And how should the relevance of retrieved information be measured when dealing with multiple data types, given that existing metrics are often optimized for text-based retrieval?

\subsection{From Data to Insights}
A core opportunity in RAG systems lies in their ability to transform raw data into actionable insights. This insight-driven retrieval allows systems to dynamically generate responses tailored to user-specific needs or industry contexts. For example, RAG systems in human resources might leverage real-time job market statistics to enhance job recommendations, improving matches based on current industry trends. By deriving insights, systems can synthesize contextual knowledge from data, supporting more accurate and adaptive output generation.

However, research challenges remain. For instance, how can RAG systems accurately capture and prioritize real-time, evolving information from different data sources to ensure that insights remain relevant and current? In dynamic fields such as job searching, the timeliness and accuracy of insights can be critical. Moreover, what methods can be developed to quantify and communicate the reliability or confidence level of synthesized insights to end users? Trustworthiness becomes especially important when RAG systems support high-stakes decision-making.
%\subsection{Agentic Workflows}
