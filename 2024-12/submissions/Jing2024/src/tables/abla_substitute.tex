\begin{table}[t]
    \centering
    \setlength{\tabcolsep}{2mm}{
    \begin{tabular}{lcccccccccc}
    \toprule
    \multirow{2}{*}{\textbf{Substitute Model}}  & \multicolumn{3}{c}{\textbf{LLaMA 2 7B}} & \multicolumn{3}{c}{\textbf{LLaMA 2 13B}} & \multicolumn{3}{c}{\textbf{LLaMA 2 70B}} \\
    \cmidrule(lr){2-4} \cmidrule(lr){5-7} \cmidrule(lr){8-10} 
    
    & P & R & F & P & R & F & P & R & F  \\
    \midrule
    \textbf{LLaMA 2 7B} & .858 & .845 & .850 & .928 & .934 & .930 & .837 & .861 & .848 \\
    \textbf{LLaMA 2 13B} & .850 & .868 & .855 & .930 & .940 & .932 & .837 & .851 & .844 \\
    \textbf{LLaMA 2 70B} & .868 & .883 & .871 & .924 & .942 & .931 & .858 & .876 & .866 \\
 
    \bottomrule
    \end{tabular}}
    \caption{Ablation on the LLaMA models as substitute models. The $i$-th row and $j$-th column denote the result of using $i$-th LLM as the substitute hidden state input for training on $j$-th model's labels.  P, R, and F are Precision, Recall, and F1 Score.  
    }
    \label{tab:abla_substitute}
    \end{table}


    

