\documentclass[11pt]{article} % DO NOT CHANGE THIS
\usepackage{deauthor}
\usepackage{natbib}
\usepackage{hyperref}  
\usepackage{graphicx}  % hyperlinks
\usepackage{url} 
\usepackage{xcolor}
\usepackage{wrapfig}% simple URL typesetting
\usepackage{booktabs}       % professional-quality tables
\usepackage{amsfonts}       % blackboard math symbols
\usepackage{nicefrac}       % compact symbols for 1/2, etc.
\usepackage{microtype}      % microtypography

\usepackage{microtype}
\usepackage{amsmath}
\usepackage{algpseudocode}
\usepackage{algorithm}
\usepackage{multirow}
\usepackage{subcaption}
\usepackage{bm}
\usepackage{stfloats}
\usepackage{arydshln}
%\usepackage{inconsolata}
\usepackage{graphicx}  
\usepackage{comment}
\usepackage{enumitem}

\newcommand{\ie}{\textit{i.e.}}
\newcommand{\eg}{\textit{e.g.}}
\setcounter{secnumdepth}{2} %May be changed to 1 or 2 if section numbers are desired.

\begin{document}

\title{\textsc{ReFIT}: Reranker Relevance Feedback during Inference}
\pagestyle{plain}
\author{\textbf{Revanth Gangi Reddy}$^1$ \hspace{0.2em} \textbf{Pradeep Dasigi}$^2$ \hspace{0.2em} \textbf{Md Arafat Sultan}$^3$ \hspace{0.2em} \textbf{Arman Cohan}$^{2,4}$ \\ \textbf{Avirup Sil}$^3$ \hspace{0.2em} \textbf{Heng Ji}$^1$ \hspace{0.2em} \textbf{Hannaneh Hajishirzi}$^{2,5}$ \\
$^1$University of Illinois at Urbana-Champaign \hspace{0.5em}  $^2$Allen Institute for AI \\ $^3$IBM Research AI \hspace{0.5em} $^4$Yale University \hspace{0.2em} $^5$University of Washington\\
  \texttt{\{revanth3,hengji\}@illinois.edu} \hspace{0.4em} \texttt{\{pradeepd,armanc,hannah\}@allenai.org} \\
  \texttt{arafat.sultan@ibm.com}\hspace{0.4em} \texttt{avi@us.ibm.com}
  }
\maketitle

\begin{abstract}

Retrieve-and-rerank is a prevalent framework in neural information retrieval, wherein a bi-encoder network initially retrieves a pre-defined number of candidates (\eg{}, $K$=100), which are then reranked by a more powerful cross-encoder model. While the reranker often yields improved candidate scores compared to the retriever, its scope is confined to only the top $K$ retrieved candidates. As a result, the reranker cannot improve retrieval performance in terms of Recall@K. In this work, we propose to leverage the reranker to improve recall by making it provide relevance feedback to the retriever at \textit{inference-time}. Specifically, given a test instance during inference, we distill the reranker's predictions for that instance into the retriever's query representation using a lightweight update mechanism. The aim of the distillation loss is to align the retriever's candidate scores more closely with those produced by the reranker. The algorithm then proceeds by executing a second retrieval step using the updated query vector. We empirically demonstrate that this method, applicable to various retrieve-and-rerank frameworks, substantially enhances the retrieval recall across multiple domains, languages, and modalities.

\end{abstract}

\section{Introduction}

Information Retrieval (IR) involves retrieving a set of candidates from a large document collection
given a user query. The retrieved candidates may be further reranked to bring the most relevant ones to the top, constituting a typical retrieve-and-rerank (R\&R) framework \cite{wang2018evidence, hu2019retrieve}.
Reranking generally improves the ranks of relevant candidates among those retrieved, thus improving on metrics such as Mean Reciprocal Rank (MRR) \cite{Craswell2009} and Normalized Discounted Cumulative Gain (nDCG) \cite{jarvelin2002cumulated}, which assign better scores when relevant results are ranked higher. 
However, retrieval metrics like Recall@K, which mainly evaluate the presence of relevant candidates in the top $K$ retrieved results, remain unaffected.
Increasing Recall@K can be key, especially when the retrieved results are used in downstream knowledge-intensive tasks \cite{petroni2021kilt} such as open-domain question answering \cite{chen2017reading, chen2020open, gangi2021synthetic}, fact-checking \cite{thorne2018fever}, entity linking \cite{hoffart2011robust,sil2013re,sil2018neural} and dialog generation \cite{dinan2018wizard, komeili2022internet}.

Most existing neural IR methods use a dual-encoder retriever \cite{karpukhin2020dense, khattab2020colbert} and a subsequent cross-encoder reranker \cite{nogueira2019passage}. 
Dual-encoder\footnote{We use the terms bi-encoder and dual-encoder interchangeably in this paper.} models leverage separate query and passage encoders and perform a late interaction between the query and passage output representations. This enables them to perform inference at scale as passage representations can be pre-computed. Cross-encoder models, on the other hand, accept the query and the passage together as input, leaving out scope for pre-computation. The cross-encoder typically provides better ranking than the dual-encoder---thanks to its more elaborate computation of query-passage similarity informed by cross-attention---but is limited to seeing only the retrieved candidates in an R\&R
framework.


\begin{wrapfigure}{r}{0.42\linewidth}
    \centering
    \includegraphics[width=1.0\linewidth]{submissions/Revanth2024/figures/cross_encoder_feedback_2.png}
    \caption{\textsc{ReFIT}: The proposed method for reranker relevance feedback. We introduce an inference-time distillation process (step 3) into the traditional retrieve-and-rerank framework (steps 1 and 2) to compute a new query vector, which improves recall when used for a second retrieval step (step 4).}
    \label{fig:overall_framework}
    \vspace{-1em}
\end{wrapfigure}

Since the more sophisticated reranker often generalizes better at passage scoring than the simpler, but more efficient retriever, here we propose to use relevance feedback from the former to improve the quality of query representations for the latter directly \textit{at inference}.
Concretely, after the R\&R pipeline is invoked for a test instance, we update the retriever's corresponding query vector by minimizing a distillation loss that brings its score distribution over the retrieved passages closer to that of the reranker.
The new query vector is then used to retrieve documents for the second time. 
This process effectively teaches the retriever how to rank passages like the reranker---a stronger model---for the given test instance.
Our approach, \textsc{ReFIT}\footnote{\textsc{ReFIT} stands for \textbf{Re}ranker \textbf{F}eedback at \textbf{I}nference \textbf{T}ime}, is lightweight as only the output query vectors (and no model parameters) are updated, ensuring comparable inference-time latency when incorporated into the R\&R framework. 
Figure \ref{fig:overall_framework} shows a schematic diagram of our approach, which introduces a distillation and a second retrieval step into the R\&R framework.
By operating exclusively in the representation space---as we only update the query vectors---our framework yields a parameter-free and architecture-agnostic solution, thereby providing flexibility along important application dimensions, e.g., the language, domain, and modality of retrieval. 
We empirically demonstrate this effect by showing improvements in retrieval on multiple English domains, across 26 languages in multilingual and cross-lingual settings, and in different modalities such as text and video retrieval.
 

Our main contributions are as follows:
\begin{itemize}
    \item We propose \textsc{ReFIT}, an inference-time mechanism to improve the recall of retrieval in IR using relevance feedback from a reranker.
    \item Empirically, \textsc{ReFIT} improves retrieval performance in multi-domain, multilingual, cross-lingual and multi-modal evaluation.
    \item The proposed distillation step is fast, considerably increasing recall without any loss in ranking performance over a standard R\&R pipeline with comparable latency.
\end{itemize}




\section{Related Work}
\label{sec:related-work}
\noindent \textbf{Pseudo-relevance feedback:} Our method has similarities with %the existing approach of 
Pseudo-Relevance Feedback (PRF) \cite{rocchio1971relevance, lv2009adaptive, li2022does} in IR: \cite{bendersky2011parameterized, xu2017quary} use the retrieved documents to improve sparse approaches via query expansion or query term reweighting, \cite{li2018nprf, zheng2020bert} score similarity between a target document and a top-ranked feedback document, while \cite{yu2021improving} train a separate query encoder that computes a new query embedding using the retrieved documents as additional input. In contrast, our approach does not require customized training feedback models or availability of explicit feedback data, as we improve the query vector by directly distilling from the reranker's output within an R\&R framework. %\pradeep{Why is our approach better?} 

Further, previous approaches to PRF have been dependent on the choice of retriever architecture and language; \cite{yu2021improving}'s PRF model is tied to the retriever used, \cite{chandradevan2022learning} explore cross-lingual relevance feedback, but require feedback documents in target language and thereby could only apply to three languages, while \cite{li2022interpolate} explore interpolating relevance feedback between dense and sparse approaches.
On the other hand, our approach is independent of the choice of the retriever and reranker architecture, and can be used for neural retrieval in any domain, language or modality. \\

\noindent \textbf{Distillation in Neural IR:} Existing approaches primarily leverage reranker feedback \textit{during training} of the dual-encoder retriever, to sample better negatives \cite{qu2021rocketqa}, for standard knowledge distillation of the cross-attention scores \cite{izacard2020distilling}, to train smaller and more efficient rankers by distilling larger models \cite{hofstatter2020improving}, or to align the geometry of dual-encoder embeddings with that from cross-encoders \cite{wang2021enhancing}. Instead, we leverage distillation at inference time, updating only the query representation to replicate the cross-encoder’s scores for the corresponding test instance.
A key implication of this design choice is that unlike existing methods, we keep the retriever parameters unchanged, meaning \textsc{ReFIT} can be incorporated out-of-the-box into any neural R\&R framework. In contrast, extending training-time distillation to new languages or modalities would require re-training the bi-encoder.

More recently, \textsc{TouR}~\cite{sung2023optimizing} has proposed test-time optimization of query representations with two variants: \textsc{TouR}$_{\text{hard}}$ and  \textsc{TouR}$_{\text{soft}}$. 
\textsc{TouR}$_{\text{hard}}$ optimizes the marginal likelihood of a small set of (pseudo) positive contexts.
\textsc{ReFIT} shares similarities with \textsc{TouR}$_{\text{soft}}$, which uses the normalized scores of a cross-encoder over the retrieved results as soft labels.
Crucially, \textsc{TouR} relies on multiple iterations of relevance feedback via distillation, where each iteration runs until the top-1 retrieval result has the highest reranker score (in \textsc{TouR}$_{\text{soft}}$) or is a pseudo-positive (in \textsc{TouR}$_{\text{hard}}$).
This makes inference highly computationally expensive, as each additional iteration involves labeling top-$K$ retrieval results with a reranker and then retrieving again.
\textsc{ReFIT} improves efficiency over \textsc{TouR} by requiring only a single iteration of feedback that simply updates the query vector for longer, foregoing additional retrieval and reranking steps. More specifics on the inference process of the two methods can be found in \S{\ref{sec:tour_comparison}}.
\textsc{TouR} was evaluated only on English phrase and passage retrieval tasks, while we demonstrate \textsc{ReFIT}'s effectiveness in multidomain, multilingual and multimodal settings, with an empirical comparison with \textsc{TouR} in \S{\ref{sec:tour_comparison}}.

\section{Method}

Here we discuss the standard retrieve-and-rerank (R\&R) framework for IR (\S{\ref{sec:retrieve_and_rerank}}) and how our proposal fits into it (\S{\ref{sec:cross_encoder_feedback}}). While our approach can be applied to any R\&R framework, we consider a text-based retriever and reranker for simplicity while elaborating our method. A multi-modal R\&R framework is described in \S\ref{sec:multimodal_results}.


\subsection{Retrieve-and-Rerank}
\label{sec:retrieve_and_rerank}
R\&R for IR consists of a first-stage retriever and a second-stage reranker. Modern neural approaches typically use a dual-encoder model as the retriever and a cross-encoder for reranking.  

\paragraph{\textbf{The Retriever}:} The dual-encoder retriever model is based on a Siamese neural network \cite{chicco2021siamese}, containing separate Bert-based \cite{devlin2019bert} encoders $E_Q(.)$ and $E_P(.)$ for the query and the passage, respectively.
Given a query $q$ and a passage $p$, a separate representation is obtained for each, such as the \textsc{cls} output or a pooled representation of the individual token outputs from $E_Q(q)$ and $E_P(p)$. The question-passage similarity $sim(q,p)$ is computed as the dot product of their corresponding representations: query/passage.}
\begin{equation}
    Q_q = Pool(E_Q(q))
\end{equation}
\begin{equation}
    P_p = Pool(E_P(p))
\end{equation}
\begin{equation}\label{eq:sim}
   sim(q,p) = S(Q_q,P_p) = Q_q^TP_p
\end{equation}

Since Eq.~\ref{eq:sim} is decomposable, the representations of all passages in the retrieval corpus can be pre-computed and stored in a dense index \cite{johnson2019billion}. During inference, given a new query, the top $K$ most relevant passages are retrieved from the index via approximate nearest-neighbor search.

\paragraph{\textbf{The Reranker}:} The cross-encoder reranker model uses a Bert-based encoder $E_R(.)$, which takes the query $q$ and a corresponding retrieved passage $p$ together as input and outputs a similarity score. 
A feed-forward layer $F$ is used on top of the \textsc{cls} output from $E_R(.)$ to compute a single logit, which is used as the final reranker score $R(q,p)$. The top $K$ retrieved passages are then ranked based on their corresponding reranker scores.

\begin{equation}
   R(q,p) = F(CLS(E_R(q,p))
\end{equation}


\begin{algorithm}[t]
\caption{\textsc{\textbf{ReFIT}}}
\label{alg4}
\begin{flushleft}
\textbf{Input}: Query $q$ and its representation $Q_q$, retrieved passages $P$ and their representations $\hat{P}$.\newline
\textbf{Output}: Updated query representation $Q_{q,n}$
\end{flushleft}
\begin{algorithmic}[1]
    \State Initialize query vector $Q_{q,0}$ = $Q_q$
    \State Compute reranker distribution $D_{CE}(q,P)$ (Eq.~\ref{eq:d-ce})
    \For{\textit{i in 0 to n}}
        \State Compute retriever distribution $D_{Q_{q,i}}(\hat{P})$ (Eq.~\ref{eq:d-q})
        \State Compute loss $\mathcal{L}$ (Eq.~\ref{eq:loss})
        \State Update $Q_{q,i+1} = Q_{q,i} - \alpha \frac{\partial}{\partial Q_{q,i}}\mathcal{L}$
    \EndFor
    \State return $Q_{q,n}$
\end{algorithmic}
%\vspace{-0.4em}
\end{algorithm}

\subsection{Reranker Relevance Feedback}
\label{sec:cross_encoder_feedback}
The main idea underlying our proposal is to compute an improved query representation for the retriever using feedback from the more powerful reranker.
More specifically, we perform a lightweight inference-time distillation of the reranker's knowledge into a new query vector.

Given an input query $q$ during inference, we use the following output provided by the R\&R pipeline:
\begin{itemize}
   \item Query representation $Q_q$ from the retriever.
    \item Retrieved passages $P = \{p_1, p_2,  ..., p_K\}$ and their representations $\hat{P} = [P_{p_1}, P_{p_1},  ..., P_{p_K}]$ from the retriever. 
    \item The reranking scores $R(q,P) = [R(q,p_1),..., R(q,p_K)]$.
\end{itemize}
Note that $\hat{P}$ above is directly obtained from the passage index and is not computed during inference.

The proposed reranker feedback mechanism begins with using the reranking scores $R(q,P)$ to compute a cross-encoder ranking distribution $D_{CE}(q,P)$ over passages $P$ as follows:

\begin{equation}
D_{CE}(q,P)=\mathrm{softmax}([R(q,p_1), ..., R(q,p_K)])
\label{eq:d-ce}
\end{equation} 

The query and passage representations from the retriever are used to compute a similar distribution $D_{Q_q}(\hat{P})$ over $P$:

\begin{equation}
    D_{Q_q}(\hat{P}) = \mathrm{softmax}([Q_q^TP_{p_1}, ..., Q_q^TP_{p_K}])
    \label{eq:d-q}
\end{equation}

Next, we compute the loss as the KL-divergence between the retriever and reranker distributions:

\begin{equation}
    \mathcal{L} = D_{KL}(D_{CE}(q,P) || D_{Q_q}(\hat{P}))
    \label{eq:loss}
\end{equation}

which is then used to update the query vector via gradient descent. The query vector update process is repeated for $n$ times, where $n$ is a hyper-parameter. 
A schematic description of the process can be found in Algorithm \ref{alg4}. 

Finally, the updated query vector $Q_{q,n}$ is used for a second-stage retrieval from the passage index.  
From dual-encoder retrieval with the updated $Q_{q,n}$, we aim to achieve better recall than with the initial $Q_q$, while obtaining a ranking performance that is comparable with that of the reranker.








%!TEX root = ../main.tex
\section{Experiments}
\label{sec:exp}

We conduct preliminary experiments to demonstrate the effectiveness of \sys. 
We evaluate its performance in two key processes: Reasoning and Verification. %The multimodal data lakes and data discovery techniques used in these experiments are discussed respectively for each process. 

%\subsection{Question Answering using Retrieved Multimodal Data}

%To validate the effectiveness of our method in text reasoning and mutimodal (text+image) reasoning, we design the experiment on natural language question answering and visual-based entity question answering.

\subsection{Question Answering}

\stitle{Experiment Setting.} In this experiment, we focus on evaluating question answering performance using a multimodal data lake consisting of 400K web tables and 6M English passages extracted from Wikipedia. The data lake includes both tables and texts, and each query is designed to retrieve relevant data items to answer a given question. We use 18 manually crafted user queries, each with corresponding ground truth annotations specifying the required data items, sub-queries for decomposition, and final answers.

\stitle{Data Discovery Evaluation.}
The effectiveness of data discovery is measured using the recall at $K$ (R@$K$) metric, which calculates the proportion of relevant data items retrieved in the top-$K$ recommendations. The experimental results show that when $K$ is 5, 10, 15, and 20, the R@$K$ values are 40.8\%, 46.3\%, 59.3\%, and 77.8\%, respectively. For 12 out of the 18 queries, \sys successfully discovers all the relevant items needed to answer the query. The remaining 6 queries show partial success. In total, 30 out of 38 related items are correctly discovered, demonstrating the potential of the proposed data discovery methodology, even though it is still in a preliminary stage.


\stitle{Query Decomposition Evaluation.}
To decompose queries into manageable sub-queries, \sys serializes the discovered data items and uses GPT-3 to generate sub-queries. The output includes the sub-queries and corresponding data item ids. Evaluation of the decomposition quality is based on two criteria: (1) whether each sub-query is useful for solving the original query, and (2) whether the sub-query can be answered correctly using the selected data item. The human evaluation results show that 77.8\% of the queries scored 2 (both criteria met), 16.7\% scored 1 (only the first criterion met), and 5.5\% scored 0. 

Table~\ref{tab:results_of_decomposition} shows the results of 8 instances. \sys is able to handle different aggregation operations, such as sum (Instance 2) and comparison (Instance 3). Further, it correctly understands long sentences (Instance 1). However, \sys has difficulty in dealing with sentences with complex syntactic structures (\eg in Instance 4, GPT-3 mistakenly generates $q_3$ because it takes the subject of released'' as Academy Award for Best Foreign Language Film''.). 


\begin{table}[!t]
\centering
\resizebox{\textwidth}{!}{
{\small
\begin{tabular}{p{15cm}}
\hline
\underline{Instance 1}: $Q$: Did the 2014 United States house of representatives elections take place on the same day in Rhode Island, South Carolina and Louisiana? $E = \{P_1, P_2, P_3\}$\\
$\rightarrow$ $q_1$: Did the 2014 United States House of Representatives elections take place on November 4, 2014 in Rhode Island? $e_1 = P_1$; $q_2$: Did the 2014 United States House of Representatives elections take place on November 4, 2014 in South Carolina? $e_2 = P_2$; $q_3$: Did the 2014 United States House of Representatives elections take place on November 4, 2014 in Louisiana? $e_3 = P_3$ \\
$\bigstar$ Score: 2 \\
\hline
\underline{Instance 2}: $Q$: How many representatives were elected in the 2014 United States house of representatives elections in Rhode Island and South Carolina?   %\srm{Question is vague, are we asking for a sum?}
$E = \{P_1, P_2\}$ \\
$\rightarrow$ $q_1$: How many representatives were elected in the 2014 United States house of representatives elections in Rhode Island? $e_1 = P_1$; $q_2$: How many representatives were elected in the 2014 United States house of representatives elections in South Carolina? $e_2 = P_2$ \\
$\bigstar$ Score: 2 \\
\hline
\underline{Instance 3}: $Q$: Which number is larger, the number of living former prime ministers of Tunisia or the number of former presidents of Ghana? $E = \{T_1, P_1\}$ \\
$\rightarrow$ $q_1$: What is the number of living former prime ministers of Tunisia? $e_1 = P_1$; $q_2$: What is the number of former presidents of Ghana? $e_2 = T_1$ \\
$\bigstar$ Score: 2 \\
\hline
\underline{Instance 4}: $Q$: What year was the first German film that won the Academy Award for Best Foreign Language Film released? $E = \{T_1, P_1\}$ \\
$\rightarrow$ $q_1$: What year was the film The Tin Drum released? $e_1 = T_1$; $q_2$: What was the first German film that won the Academy Award for Best Foreign Language Film? $e_2 = T_1$; $q_3$: What year was the first Academy Award for Best Foreign Language Film given out? $e_3 = T_1$ \\
$\bigstar$ Score: 0 \\
\hline
\underline{Instance 5}: $Q$: Did The Tin Drum and Kolya both win the Academy Awards for Best Foreign Language Film? $E = \{P_1, P_2\}$ \\
$\rightarrow$ $q_1$: whether The Tin Drum has ever won the Academy Award for Best Foreign Language Film? $e_1 = P_1$; $q_2$: whether Kolya has ever won the Academy Award for Best Foreign Language Film? $e_2 = P_2$ \\
$\bigstar$ Score: 2 \\
\hline
\underline{Instance 6}: $Q$: How many years elapsed between the premiere of season 10 and the premiere of season 5 of Law \& Order : Special Victims Unit? $E = \{T_1, T_2\}$ \\
$\rightarrow$ $q_1$: What is the premiere date of season 10 of Law \& Order : Special Victims Unit? $e_1 = T_2$; $q_2$: What is the premiere date of season 5 of Law \& Order : Special Victims Unit? $e_1 = T_1$ \\
$\bigstar$ Score: 2 \\
\hline
\underline{Instance 7}: $Q$: Did You Let Me Walk Alone participate in the Eurovision Song Contest in 1996? $E = \{T_1, P_1\}$ \\
$\rightarrow$ $q_1$: What year was the Eurovision Song Contest held in Lisbon, Portugal? $e_1 = T_1$; $q_2$: What year was You Let Me Walk Alone released? $e_2 = P_1$ \\
$\bigstar$ Score: 1 \\
\hline
\underline{Instance 8}: $Q$: Are the tallest building in the united kingdom and the tallest building in poland above 200 meters? $E = \{T_1, T_2\}$\\
$\rightarrow$ $q_1$: What is the height of the tallest building in the United Kingdom? $e_1 = T_1$; $q_2$: What is the height of the tallest building in Poland? $e_2 = T_2$ \\
$\bigstar$ Score: 2 \\
\hline
\end{tabular}
}
}
\caption{Example sub-queries generated by \sys. $q_i$ and $e_i$ represent the $i_{th}$ sub-query and its corresponding data item. $T_i$ represents a table and $P_i$ represents a text. 
}
\label{tab:results_of_decomposition}
\end{table}


% \subsubsection{Visual-based Entity Question Answering}

% \stitle{Experiment Setting.}
% The experiments were conducted in a zero-shot setting using RTX 4090 GPUs. For GPT-4V, we used the interface of the GPT-4-vision-preview model. It's worth noting that GPT-4V often refrains from answering person identify questions without additional clues due to policy reasons. However, with the incorporation of matching graph techniques, it can leverage weak signals and combine them with its own knowledge base. For the dataset, we chose NewsPersonQA~\cite{zhang2024mar}. For the task evaluation, we use accuracy (\textbf{Acc}) as an evaluation metric. Furthermore, we assess the accuracy only for instances where relevant clues are successfully retrieved, which is denoted as \textbf{Acc}$^{{hit}}$.

% \stitle{Baseline.} For answering queries, we selected two well-known and highly capable MLLMs, as well as human evaluation,to serve as baselines. \textbf{LLaVA:} 
% This model utilizes CLIP-ViT-L-336px with an MLP projection. We refer to the 1.5 version with 7 billion parameters as LLaVA-7b and the version with 13 billion parameters as LLaVA-13b.
% \textbf{GPT-4V:} 
% Recognized as OpenAI's most powerful general-purpose MLLM to date, GPT-4V boasts 1.37 trillion parameters. 

% \stitle{Main Results.} The main results of visual-based entity question answering are summarized in Table~\ref{tbl:single}, which leads to the following insights: LLaVA-13b demonstrates higher accuracy (27.93\%) compared to LLaVA-7b (22.26\%), suggesting that a model's recognition ability is positively correlated with its parameter size, which to some extent reflects its knowledge base. Incorporating a matching graph leads to an 8.9\% improvement in accuracy for LLaVA-7b and a 3.2\% improvement for LLaVA-13b. GPT-4V, with matching, achieves a character recognition accuracy of 34.83\%.
%  The enhancement from matching is more pronounced for LLaVA-7b than for LLaVA-13b, indicating that while matching can compensate for differences in parameters, a model's inherent capabilities still set an upper limit on its performance.

% \begin{table}[t!]
% \caption{Result for Visual-based Entity Question Answering. (Note: GPT-4V could not answer these queries directly due to policy constraints. Values within parentheses are those GPT-4V still refuses to answer.)}
% \small
% \centering
% \label{tbl:single}
% \resizebox{0.5\columnwidth}{!}{
% \begin{tabular}{l|l|l}
% \toprule
% {\textbf{Models}}     & \textbf{Acc} (\%) & \textbf{Acc}$^{{hit}}$ (\%) \\ 
% \midrule
% \textbf{LLaVA-7b}                        & 22.26        & 27.53         \\
% {\textbf{LLaVA-7b + Symphony}} & 31.19        & 62.81         \\ 
% \hline
% \textbf{LLaVA-13b}                       & 27.93        & 32.86         \\
% {\textbf{LLaVA-13b + Symphony}} & 31.13        & 62.34         \\ 
% \hline
% \textbf{GPT-4V}                          & -            & -             \\
% {\textbf{GPT-4V + Symphony}} & 34.84 (4.2)  & 68.31 (2.6)  \\ 
% \midrule
% \textbf{Symphony(Graph Reasoning)}                            & {\bf 39.09}        & {\bf 79.65}         \\ 
% \bottomrule
% \end{tabular}
% }
% \end{table}


\subsection{Answer Verification}

We showcase preliminary experimental results that highlight the initial achievements of \sys in facilitating the verification of generative AI. 
% \yang{Do we need to include tuple-tuple and text generation?}

\stitle{Experiment Setting.}
We perform a controlled study to assess textual claims, employing 1,300 textual claims from the TabFact~\cite{chen2019tabfact} benchmark, which is currently the most advanced benchmark for verifying the credibility of textual hypotheses by utilizing a given table. The data lake consists of 16,573 tables from the TabFact and 2,925 tables sourced from WikiTable-TURL~\cite{deng2022turl}.


\begin{figure}[t!]
\vspace{1em}
\begin{center}
  \includegraphics[width=0.65\textwidth]{submissions/Nan2024/figs/tabfact.pdf}
  \caption{Verifying a textual claim using retrieved tables.}
  \label{fig:claim_case} 
\end{center}
\end{figure}


\stitle{Evaluation for Retrieval.} 
We use Elasticsearch~\cite{elasticsearch} to retrieve the top-5 tables for each textual claim. Given the limited amount of relevant data, we focus on the recall metric for evaluation. Each textual claim is associated with a corresponding table in the original dataset, which we consider relevant evidence, while other retrieved tables are deemed irrelevant. The retrieval performance, measured by R@5, is 0.88.

\stitle{Evaluation for Verification.} 
We evaluate the verification process using two different verifiers: GPT-3.5, the default verifier for both data types, and PASTA~\cite{pasta}, a specialized model for text verification.
%
The performance of the verifiers is measured by accuracy. When the retrieved data cannot support or refute a claim, the verifier outputs ``not related''. However, in this case, since PASTA that only offers two different answers: ``true'' or ``false'', we consider it's also correct when PASTA outputs ``false''.

We conduct experiments in two settings. When a relevant table is retrieved and provided as evidence to the verifier, PASTA achieves higher accuracy than GPT-3.5 (0.89 vs. 0.75) in verifying the textual claim based on the table. However, in cases where many of the retrieved tables are irrelevant to the claim, the verifier must accurately determine which tables are not related. In this setting, PASTA's accuracy drops to 0.72 because it has not encountered this scenario during training, while GPT-3.5 improves to 0.91. 
% Thus, GPT-3.5-turbo demonstrates superior generalization capabilities and performs better than PASTA when dealing with irrelevant tables.
Thus, when the retrieved data is highly related to the generative data, local models like PASTA have higher accuracy while protecting privacy. In contrast, GPT-3.5 is better at generalizing and providing explanations for further judgments. Users can select the appropriate model based on their requirements.


In Figure~\ref{fig:claim_case}, we present a case of verifying a textual claim based on retrieved tables using GPT-3.5. \sys retrieves two tables $E_1$ and $E_2$, where $E_1$ can be used with an aggregation query to refute the claim while $E_2$ is not related because it is for the year 1959. The red boxes in Figure~\ref{fig:claim_case} show that GPT-3.5 can provide not only a verification result but also some explanation.






\section{Conclusion and Future Work}
We demonstrate that query representations can be improved using feedback from a cross-encoder reranker \textit{at inference time} for better performance of dual-encoder retrieval. This work proposes for distillation using relevance feedback from the reranker as a better and faster alternative to the traditional strategy of reranking a larger pool of candidates for improving recall.
\textsc{ReFIT} is lightweight and improves retrieval accuracy across different domains, languages and modalities over a state-of-the-art retrieve-and-rerank pipeline with comparable latency. Future work will focus on the potential integration of textual relevance feedback from large language models (LLMs). Additionally, a promising area of exploration lies in enhancing the interpretability by examining how relevance feedback influences the significance of individual query terms within the query representation.


\section{Limitations}

\textsc{ReFIT} introduces an additional latency into a traditional retrieve-and-rerank framework.
The distillation time is only dependent on the number of updates, and is unaffected by the model architecture  and number of retrieved passages; the overall additional latency (as per Table \ref{tab:inf_times}) amounts to an extra 17.5\% on GPU (or 4.4\% on CPU) when the number of retrieved passages $K$=100. However, it is noteworthy that \textsc{ReFIT} remains faster and exhibits superior performance compared to the standard approach of reranking a larger pool of candidates for improving recall.
Moreover, the efficacy of our approach is contingent upon the reranker providing a better ranking than the retriever. We anticipate that our method might provide minimal gains in situations where the retriever performs similar to the reranker.

\bibliographystyle{ieeetr}

\bibliography{submissions/Revanth2024/sample-base}

\end{document}
